\chapter{Wnioski}

Przedstawiona implementacja systemu syntezującego i przetwarzającego sygnały dźwiękowe udowadnia, iż jest to możliwe przy użyciu karty graficznej. Problematyka przedstawionego zagadnienia jest wysoce kompatybilna z metodami równoległego przeważania danych. Wiele algorytmów przetwarzania sygnałów dźwiękowych, które są obecnie wykonywane na CPU, może być bezproblemowo przeniesione na GPU przy użyciu takich platform jak CUDA. Platforma ta zapewnia łatwy dostęp do zasobów karty graficznej, co pozwala na wykorzystanie jej mocy obliczeniowej i przeniesienie kodu na GPU w sposób stosunkowo nieskomplikowany, nieingerujący nadto w strukturę systemu. Jedynym mankamentem okazał się brak pełnego wsparcia dla polimorfizmu w języku CUDA, co wymusiło zmianę sposobu zastosowania wzorca strategii. Na rynku dostępne jest obecnie wiele innych technologii, pozwalających na wykorzystanie mocy obliczeniowej GPU, takich jak OpenCL, OpenACC, czy SYCL. Każda z nich mogła by być równie dobrze wykorzystane do tego rodzaju zadań w zależności od indywidualnych preferencji programisty oraz wymagań stawianych przed projektem. Powstały system bazujący na GPU bezproblemowo dorownuja, a niekiedy nawet przewyższa wydajnością swój odpowiednik powstały na CPU, pomimo iż kwestie optymalizacji nie zostały poruszone w tej pracy. Implementacja bardziej złożonych algorytmów, a w szczególności algorytmów sekwencyjnych w sposób optymalny może okazać się jednak trudniejsza i wymagać głębszej analizy problemu. 


% wzorzec startegii