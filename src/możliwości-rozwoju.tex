\chapter{Możliwości rozwoju}

\section{Optymalizacja obecnego rozwiązania}
Ideą stojącą za wykorzystanie karty graficznej było zagospodarowanie dodatkowej mocy obliczeniowej w celu umożliwienia przetwarzania większej ilości danych w czasie rzeczywistym.

\subsection{Przystosowanie algorytmów do wykorzystania GPU}
Przedstawiona implementacja kodu w języku CUDA\cite{bib:CUDA} (przystosowana do wykorzystania GPU) przedstawia algorytmy, których zasada działania jest bliźniaczo podobna do implementacji w języku C++ (przystosowanej do wykorzystania CPU). W związku z tym, wiele z nich obarczonych jest koniecznością wykonywania sekwencyjnego, co nie pozwala na pełne wykorzystanie mocy obliczeniowej karty graficznej. Rozwiązaniem może być całkowita rezygnacja z sekwencyjności, co prawdopodobnie wiązałoby się z brakiem możliwości uzyskania identycznych wyników do klasycznych algorytmów. Nie koniecznie jest to wada, w wielu przypadkach ludzka percepcja nie byłaby wystarczająca aby dostrzec wynikającą ze zmiany różnicę. 
//
Warto również zwrócić uwagę na wymóg transferu danych pomiędzy CPU a GPU, który w przypadku przenoszenia dokładnych informacji o stanie klawiatury w czasie, skutkował przesyłaniem dużej ilości danych względem pojedynczego strumienia audio. Podejściem rozwiązującym problem mogłoby być zastosowanie algorytmów kompresji tego rodzaju danych. Praktycznym i uniwersalnym rozwiązaniem mogło by się okazać opracowanie odmiany formatu MIDI, przystosowanego do interpretacji przy wykorzystaniu GPU.

\subsection{Grupowanie wywołań kerneli}
Rezygnacja z sekwencyjności w wielu przypadkach nie jest jednak konieczna. Można by zastosować system przypominający relację klient-serwer: klient (instancja wykorzystująca konkretny algorytm) wysyła swoje dane do serwera (jednostki opartej o wzorzec singleton\cite{bib:DesignPatterns}u), gdzie są one przetwarzane jednocześnie dla wszystkich możliwych klientów jednocześnie. Algorytm decydujący o możliwym połączeniu wywołań algorytmów mógłby bazować na obecnej implementacji zawartej w klasie ExecutionQueue - algorytmu przechodzenia po drzewie. Różnicą była by konieczność powiązania ze sobą gałęzi drzewa, które mogą być przetwarzane jednocześnie, a następnie pogrupowanie ich w kolejności pozwalającej na uzyskanie najmniejszej liczby wywołań kerneli. Równocześnie rozwiązanie to pozwalało by w większości przypadków na obliczenia asynchroniczne.
//
Zastosowanie tego rozwiązanie przy zachowaniu sekwencyjności było by najskuteczniejsze w przypadków w których dany algorytm jest wykorzystany wielokrotnie w tym samym czasie. Za przykład może posłużyć synteza dźwięku dla wielu syntezatorów, gdzie każdy z nich syntezuje wiele głosów (wciśniętych klawiszy) jednocześnie. Ten samo rozwiązanie mogło by również przynieść korzyści dla algorytmów nie sekwencyjnych, poprzez zwiększenie rozmiaru bloku bądź \textit{gridu}.

\subsection{Implementacja systemu hybrydowego}
Niezaprzeczalnie zarówno CPU jak i GPU posiadają swoje mocne oraz słabe strony, wynikające z ich architektury. Tę wiedzę można wykorzystać w celu doboru odpowiedniego urządzenia do konkretnego zadania. Skutecznym rozwiązaniem może okazać się system hybrydowy, który posiadał by implementację zarówno na GPU, jak i CPU. W zależności od potrzeb, system mógłby decydować o wyborze jednej z opcji biorąc pod uwagę zarówno zysk w wykorzystaniu zasobów biorący się z konkretnego rozwiązania oraz czas potrzebny na ewentualne przesłanie danych pomiędzy urządzeniami. W tym przypadku skutecznym może okazać się asynchroniczny przesył danych pomiędzy urządzeniami. Pozwoliło by to na równomierne wykorzystanie mocy obliczeniowej obu urządzeń w najbardziej opłacalny w danym kontekście sposób przy jednoczesnym zachowaniu możliwości zachowania algorytmów sekwencyjnych, które preferowane są dla implementacji CPU.

\section{Wykorzystanie innych technologii}
Możliwości karty graficznej nie kończą się na przyspieszaniu obliczeń wykonywanych uprzednio na CPU. Nowoczesne karty graficzne posiadają wieli technologii, które mogą być wykorzystane do utworzenia zupełnie nowych algorytmów, które nie mogły by zostać skutecznie zastosowane w przypadku CPU.

\subsection{Wykorzystanie rdzeni ray-tracingu}
Ray-tracing\cite{bib:ray-tracing-cores} to technika generowania obrazów poprzez śledzenie promieni światła. Jest to technika stosowana w grafice komputerowej, która pozwala na uzyskanie bardzo realistycznych obrazów. W przeciwieństwie do tradycyjnego renderowania, gdzie obliczenia są wykonywane dla każdego piksela, ray-tracing pozwala na uzyskanie obrazu poprzez śledzenie promieni światła od obserwatora do obiektów na scenie. Można by wykorzystać tę technologię to śledzenia fal dźwiękowych od źródła dźwięku przez obiekty na scenie, aż do obserwatora. Take wykorzystanie tej technologii może pozwolić na uzyskanie bardziej realistycznych efektów, takich jak echa, czy innych zjawisk związanych z propagacją fal dźwiękowych w przestrzeni. Z kolei takie zastosowanie pozwoliło by na dokładne symulowanie akustyki konkretnych pomieszczeń (sal koncertowych, filharmonii, katedr, itp.) oraz ostatecznie takie narzędzie mogło by okazać się przydatne dla architektów, dając im w ten sposób możliwość symulowania akustyki budynków/pomieszczeń jeszcze w fazie projektu.

\subsection{Wykorzystanie rdzeni tensorowych}
Rdzenie tensorowe\cite{bib:tensor-cores} to jednostki obliczeniowe, które są specjalnie zaprojektowane do wykonywania operacji na tensorach - wielowymiarowych tablic, które mogą przechowywać dane w dowolnym wymiarze. Rdzenie tensorowe są powszechnie wykorzystywane w technologiach związanych z uczeniem maszynowym, takich jak sieci neuronowe. System przetwarzania dźwięku, który pozwala na wykorzystanie karty graficznej, umożliwia na prostą integrację wykorzystania rdzeni tensorowych w algorytmach przetwarzających sygnał dźwiękowy, a co za tym idzie, na wykorzystanie zaawansowanych technik uczenia maszynowego w celu uzyskania lepszych rezultatów niż było to możliwe w przypadku samego CPU.

\subsection{Przyszłe technologie}
Karty graficzne są jednym z najszybciej rozwijających się obszarów technologii komputerowych. Są one obecnie odbiciem rozwoju szeroko pojętej informatyki, zaczynając od przetwarzania dużych zbiorów danych, poprzez branżę rozrywkową, aż po techniki uczenia komputerowego. W związku z tym jest to obszar, w którym można się spodziewać ciągłych innowacji, nowych rozwiązań i wzrostu mocy obliczeniowej. Wykorzystanie karty graficznej do przetwarzania dźwięku to dziedzina, która ma ogromny potencjał, zarówno w celu przyspieszenia obliczeń dla algorytmów obecnie implementowanych na CPU, jak i potencjalnie w celu uzyskania zupełnie nowych sposobów na przetwarzanie sygnałów dźwiękowych.
