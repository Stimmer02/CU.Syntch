\chapter{Projekt systemu}
\label{chap:Projekt systemu}

\section{Założenia wstępne}
Aby zrealizować projekt w sposób składny i efektywny, należy zdefiniować cele i wymagania, które system ma spełniać. Będą one stanowiły podstawę do dalszych prac projektowych.
\\\\
\textbf{System musi spełniać następujące wymagania:}
\begin{enumerate}
    \item Możliwość przetwarzania dźwięku w czasie rzeczywistym (online).
    \item Możliwość przetwarzania dźwięku w trybie offline (możliwe najszybciej bez jednoczesnego odtwarzania).
    \item Pierwsza wersja systemu ma przetwarzać dźwięk jedynie przy użyciu CPU.
    \item Łatwy w re-implementacji w celu wykorzystania GPU, przy zachowaniu minimalnych zmian w kodzie.
    \item Uniwersalny podsystem komponentów odpowiedzialnych za przetwarzanie dźwięku - komponent implementuje pojedynczy algorytm przetwarzania dźwięku.
    \item Uniwersalny podsystem strumieni dźwięku - strumień może zostać przetworzony przez dowolną liczbę komponentów oraz dowolnie łączony z innymi strumieniami.
    \item Możliwość odtwarzania strumienia wyjściowego.
    \item Możliwość zapisu strumienia wyjściowego w przypadku przetwarzania offline.
    \item Możliwość pełnej kontroli systemu, bez konieczności ingerencji w kod źródłowy.
    \item Możliwość syntezy dźwięku - początek strumienia dźwięku.
    \item Obsługa urządzeń wejścia, przy pomocy których są generowane dane potrzebne do syntezy dźwięku.
    \item Obsługa plików wejściowych, stosowanych zamiennie z urządzeniami wejścia.
    \item Podsystem zarządzania plikami wejściowymi - możliwość dodawania, usuwania, rozpoczęcia/wstrzymania odtwarzania, przewijania pliku.
    \item Możliwość zmiany formatu strumieni dźwiękowych - zmiana częstotliwości próbkowania, rozmiaru bufora.
    \item Podsystem gromadzenia danych statystycznych dotyczących wydajności systemu.
    \item Funkcja rzetelnego testowania wydajności systemu - brak czynnika ludzkiego w procesie pomiaru.
\end{enumerate}

\section{Założenia implementacyjne}
Przed rozpoczęciem procesu implementacji należy najpierw określić wykorzystane technologie oraz standardy.
\\\\
\textbf{Zbiór założeń implementacyjnych:}
\begin{enumerate}
    \item System zostanie zaimplementowany w języku C++.
    \item Elementy systemu odpowiedzialne za przetwarzanie dźwięku zostaną ponownie zaimplementowane na GPU przy użyciu technologii CUDA.
    \item Jako formatu przetwarzanego sygnału zostanie użyty format LPCM.
    \item Próbki LPCM będą reprezentowane jako liczby zmiennoprzecinkowe 32-bitowe (typ float w języku C++).
    \item Bufory audio będą zawsze grupowane po dwa - sygnał stereo.
    \item Interfejs użytkownika zostanie zaimplementowany jako aplikacja konsolowa.
    \item Jako urządzenia wejścia posłużą urządzenia MIDI oraz klawiatura komputerowa.
    \item Jako format wejściowy zostaną użyte pliki .mid - pliki zapisane w formacie MIDI.
    \item Jako format wyjściowy danych dźwiękowych zostaną użyte pliki .wav.
    \item W celu odtwarzania dźwięku zostanie użyty system PulseAudio.
    \item System nie będzie korzystał z wielowątkowości (w przypadku przetwarzania na CPU) lub asynchronicznych wywołań kerneli (w przypadku przetwarzania na GPU) w celu przetwarzania dźwięku.
\end{enumerate}

\section{Architektura systemu}
System będzie się składał z wielu podsystemów, posiadających odmienne zbiory obowiązków, połączonych ze sobą klasą implementującą wzorzec mediatora oraz fasady.
\\\\
\textbf{Dziedziny jakimi będą zajmować się podsystemy:}
\begin{enumerate}
    \item Wejście:
        \begin{itemize}
            \item zarządza urządzeniami wejścia,
            \item zarządza plikami wejściowymi,
            \item interpretuje strumienie wejściowe,
            \item zarządza zbiorem syntezatorów,
            \item generuje strumienie dźwiękowe przy wykorzystaniu syntezatorów oraz strumieni wejściowych.
        \end{itemize}
    \item Wyjście:
        \begin{itemize}
            \item zarządza komunikacją programu z serwerem PulseAudio,
            \item konwertuje otrzymany strumień dźwiękowy na wybrany przez użytkownika format,
            \item odtwarza otrzymany strumień dźwiękowy,
            \item pozwala na zapis otrzymanego strumienia dźwiękowego do pliku.
        \end{itemize}
    \item Przetwarzanie sygnału:
        \begin{itemize}
            \item zarządza komponentami implementującymi algorytmy przetwarzania sygnału dźwiękowego,
            \item pozwala na łączenie strumieni audio,
            \item pozwala na dodawanie komponentów do strumieni audio,
            \item generuje listę instrukcji na bazie obecnej konfiguracji w kolejności, w której należy je wykonać w celu uzyskania zdefiniowanego przez użytkownika efektu.
        \end{itemize}
    \item Dane statystyczne:
        \begin{itemize}
            \item wykonuje i gromadzi pomiary czasu poszczególnych części pętli programowej,
            \item pozwala na zapis zgromadzonych i wstępnie przetworzonych danych do pliku.
        \end{itemize}
    \item Interface użytkownika:
        \begin{itemize}
            \item zarządza operacjami wejścia-wyjścia w kontekście terminalu,
            \item wykorzystuje fasadę systemu w celu wykonywania na nim operacji,
            \item rozpoznaje słowa-klucze (komendy) podawane przez użytkownika i wywołuje odpowiednie metody systemu,
            \item wykonuje pliki tekstowe zawierające zestawy komend, w roli automatyzacji procesu konfiguracji systemu.
        \end{itemize}
\end{enumerate}

\section{Komunikacja podsystemów}
W związku z koniecznością komunikacji pomiędzy podsystemami oraz wewnątrz klas wchodzących w ich skład, należy zdefiniować ustandaryzowany format dla każdego typu przekazywanych i przechowywanych informacji, z uwzględnieniem koniecznej w przyszłości reimplementacji znaczącej części systemu przy użyciu technologii CUDA.

\begin{enumerate}
    \item \textbf{audioFormatInfo} - struktura przechowująca informacje dotyczące formatu przetwarzanego i odtwarzanego/zapisanego dźwięku takie jak:
    \begin{itemize}
        \item częstotliwość próbkowania,
        \item rozmiar bufora,
        \item liczba kanałów wyjściowych,
        \item głębia bitowa próbki.
    \end{itemize}
    \item \textbf{IDManager} - klasa-kontener przechowująca obiekty wskazanego typu i nadająca im unikalne identyfikatory, pomaga w zarządzaniu obiektami w systemie, nadając im możliwie najkrótsze identyfikatory, co ułatwia posługiwanie się nimi użytkownikowi.
    \item Pozwala na operacje takie jak:
    \begin{itemize}
        \item dodawanie nowego obiektu i jednoczesne przydzielanie mu ID,
        \item usuwanie obiektu o wskazanym ID,
        \item pobieranie obiektu o wskazanym ID,
        \item pobieranie tablicy wszystkich obiektów,
        \item sprawdzanie czy obiekt o wskazanym ID istnieje.
    \end{itemize}
    \item \textbf{audioBuffer} - struktura przechowująca próbki dźwiękowe w postaci tablicy bajtów, wykorzystywana w komunikacji pomiędzy systemem a serwerem PulseAudio. Przechowuje:
    \begin{itemize}
        \item tablicę bajtów zawierającą próbki dźwiękowe,
        \item ilość próbek w tablicy,
        \item wielkość pojedynczej próbki w bajtach.
    \end{itemize}
    \item \textbf{pipelineAudioBuffer} - struktura przechowująca dwie tablice typu float próbek dźwiękowych, wykorzystywana w wewnętrznej komunikacji pomiędzy komponentami przetwarzania dźwięku, systemem wyjścia oraz systemem wejścia. Przechowuje:
    \begin{itemize}
        \item dwie tablice typu float zawierające próbki dźwiękowe,
        \item ilość próbek w tablicach.
    \end{itemize}
    \item \textbf{audioBufferQueue} - struktura przechowująca pipelineAudioBuffer wraz z informacją o kolejności wykonywania na nim operacji oraz informacja o jego pochodzeniu. Przechowuje:
    \begin{itemize}
        \item obiekt pipelineAudioBuffer,
        \item typ podsystemu, który utworzył buffer,
        \item ID elementu podsystemu który utworzył buffer,
        \item listę ID komponentów przetwarzania dźwięku, które mają zostać na nim wykonane.
    \end{itemize}
    \item \textbf{keyboardTransferBuffer} - struktura przechowująca informacje o stanie poszczególnych klawiszy, na osi czasu, urządzenia wejściowego. Przechowuje:
    \begin{itemize}
        \item dwuwymiarową tablicę bajtów (gdzie pierwszy wymiar stanowi indeks klawisza, a drugi oś czasu) zawierających informację o stanie klawiszy, wyróżniając 128 stanów zgodnie ze standardem MIDI,
        \item jednowymiarową tablicę (wielkości równej ilości klawiszy) zawierającą informację o ostatnim stanie z poprzedniej zawartości bufora,
        \item ilość obsługiwanych klawiszy,
        \item wielkość bufora w dziedzinie czasu.
    \end{itemize}
\end{enumerate}


% PulseAudio
% PulseAudio simple
% mediator, fasada
