\chapter{Projekt i wyniki testów}

W celu wyeliminowania czynników zewnętrznych, wszystkie z przeprowadzonych testów zostały oparte o ten sam zestaw plików MIDI. Testy zostały przeprowadzone na komputerze:

\begin{table}[H]
    \begin{center}
        \caption{Specyfikacja komputera}
        \label{tab:Specyfikacja komputera}
        \begin{tabular}{r l c}
            \hline
            \textbf{komponent} & \textbf{nazwa} & \textbf{przepustowość danych} \\
            \hline
            Procesor & AMD Ryzen 5 3600 & - \\
            \hline
            Pamięć RAM & 16 GB DDR4 3200 MHz & 25 GB/s \\
            \hline
            Karta graficzna & NVIDIA RTX 3070 & 448 GB/s \\
            \hline
            Szyna PCI-E & 3.0 x16 & 16 GB/s \\
            \hline
        \end{tabular}
    \end{center}
\end{table}

Przedstawiona przepustowość jest czysto teoretyczna i tyczy się ona możliwości pamięci w przypadku GPU oraz pamięci RAM, jak również możliwościom szyny danych PCI-E.

\section{Projekt testów}

Pierwsze testy i porównanie obu implementacji nastąpiło już w czasie implementacji. Zebrane w tamtym czasie wyniki pozwoliły na zaprojektowanie testów, które zostaną przedstawione w dalszej części tego rozdziału. Każdy z przeprowadzonych testów jest wywoływany za pomocą skryptu, który automatycznie konfiguruje system i wywołuje odpowiednie komendy w celu rozpoczęcia i zapisu pomiaru.

\lstset{
    language=bash,
    caption={Przykłady skryptów testujących wydajność},
    label=lst:Przykład skryptu testującego,
    literate={ę}{{\k{e}}}1 {ą}{{\k{a}}}1 {ł}{{\l{}}}1 {ó}{{\'o}}1 {ś}{{\'s}}1 {ć}{{\'c}}1 {ż}{{\.z}}1 {ź}{{\'z}}1 {ń}{{\'n}}1,
    frame=single,
    float,
}
\begin{lstlisting}
# skrypt testujący wydajność implementacji GPU w trybie online:

# wykonanie skryptu konfigurującego system
execute ./config/scripts/Vth_symphony_setup.txt
# uruchomienie pipeline systemu
pStart
# rozpoczęcie pomiaru statystyk 
# (pomiar co 0.05 sekundy)
statRecord 0.05 Online_GPU48k16b1024.csv
# rozpoczęcie nagrywania otwartych w plików midi do pliku test.wav
midiRecord test.wav
# zatrzymanie pomiaru statystyk po zakończeniu nagrywania 
statStop


# skrypt testujący wydajność implementacji GPU w trybie offline:

# wykonanie skryptu konfigurującego system
execute ./config/scripts/Vth_symphony_setup.txt
# wielokrotne wykonanie nagrania otwartych w plikach midi
# przy jednoczesnym zapisie czasu trwania operacji
midiRecord test.wav offline time Offline_GPU48k16b128.csv
midiRecord test.wav offline time Offline_GPU48k16b128.csv
midiRecord test.wav offline time Offline_GPU48k16b128.csv
# [...] 
\end{lstlisting}

Jako dane testowe została wybrana pierwsza część utworu \textit{Symphony No. 5 in C minor, Op. 67} Ludwiga van Beethovena. Utwór ten został wybrany z powodu swojej złożoności, która pozwala na przetestowanie wydajności systemu w różnych warunkach. Drugą zaletą tego utworu jest jego długość (pierwsza część utworu trwa 7:37 minut), jak i wykorzystanie wielu instrumentów, dzięki czemu tworzy warunki testowania obu implementacji pod dużym i długotrwałym obciążeniem, generując przy tym reprezentatywną ilość danych.
\\
Przedstawiony w \autoref{lst:Vth_symphony_setup} plik konfiguracyjny przygotowuje dziewięć syntezatorów o różnych konfiguracjach, tworzących strumienie, które następnie są przetwarzane przez łącznie dziewiętnaście komponentów. W przypadku przetwarzania online test został przeprowadzony jednorazowo dla każdej wybranej konfiguracji systemu. Wynikiem testu wydajność systemu w trybie offline, jest pomiar czasu potrzebnego na wygenerowanie pliku .wav na bazie konfiguracji systemu. Test został więc przeprowadzony dziesięciokrotnie, w celu uzyskania bardziej reprezentatywnych wyników.
\\
Zarówno dla przetwarzania online jak i offline testy zostały przeprowadzone przy zapisie próbki w formacie 16-bitowym, częstotliwości próbkowania 48 kHz oraz stereofonicznej konfiguracji strumieni dźwiękowych. Wyróżniono cztery rozmiary bufora, reprezentujące różne przypadki użycia:
\begin{itemize}
    \item \textbf{128 $\sim$ 2.67ms} - mały rozmiar bufora, reprezentujący przypadek, w którym kluczowe jest niskie opóźnienie (przetwarzanie audio w przypadku występu na żywo, próba muzyczna, itp.),
    \item \textbf{1024 $\sim$ 21.33ms} - średni rozmiar bufora, reprezentujący przypadek, w którym równowaga między opóźnieniem a wydajnością jest kluczowa (praca inżyniera dźwięku w studiu nagrań, odtwarzanie muzyki),
    \item \textbf{8192 $\sim$ 170.67ms (online)} - duży rozmiar bufora, reprezentujący przypadek, w którym wysokie opóźnienie nie stanowi dużego problemu (renderowanie projektu online, hobbystyczna praca przy wykorzystaniu niewydajnego urządzenia).
    \item \textbf{65536 $\sim$ 1365.33ms (offline)} - nad przeciętnie duży rozmiar bufora, reprezentujący przypadek pracy offline (renderowanie projektu offline, przetwarzanie dużej ilości plików). 
\end{itemize}

\lstset{
    language=bash,
    caption={Wybrane fragmenty pliku Vth\_symphony\_setup.txt, przygotowującego system do odtworzenia "Piątej Symfonii" Beethovena},
    label=lst:Vth_symphony_setup,
    literate={ę}{{\k{e}}}1 {ą}{{\k{a}}}1 {ł}{{\l{}}}1 {ó}{{\'o}}1 {ś}{{\'s}}1 {ć}{{\'c}}1 {ż}{{\.z}}1 {ź}{{\'z}}1 {ń}{{\'n}}1,
    frame=single,
    float,
}
\begin{lstlisting}
# [...]
# konfiguracja pojedynczego syntezatora dla trzeciego instrumentu
#
# utworzenie nowego strumienia wejściowego
midiAdd #MIDI 2
# ustawienie ścieżki do pliku MIDI
midiSet 2 ./V_Bethoven/viole.mid
# utworzenie nowego syntezatora
synthAdd #SYNTH 4
# wczytanie pliku konfiguracyjnego dla syntezatora
synthSave 4 load ./config/synthSave/Vth_symphony/lead.bin
# połączenie pliku midi z syntezatorem
synthConnect 4 2
# utworzenie komponentu 'pan'
compAdd PAN #COMP 6
# ustawienie położenia w przestrzeni
compSet 6 pan 0.8
# połączenie komponentu 'pan' z syntezatorem
compConnect 6 SYNTH 4
#
# [...]
#
# utworzenie komponentu 'sum7' łączącego wszystkie strumienie
compAdd SUM7 #COMP 11
# ustawienie poprzednio utworzonego komponentu
# jako strumienia wyjściowego systemu
setOut COMP 11
# połączenie wszystkich strumieni z komponentem sumującym
aCompConnect 11 0 COMP 2
aCompConnect 11 1 COMP 5
aCompConnect 11 2 SYNTH 4
# [...]
# ustawienie głośności strumieni wchodzących
compSet 11 vol2 0.17 vol3 0.18 vol4 0.18 vol5 0.064 vol6 0.064
# dodatnie komponentów przetwarzających sygnał wyjściowy
compAdd VOLUME #COMP 12
compConnect 12 COMP 11
compSet 12 vol 2
compAdd COMPRESSOR #COMP 13
compConnect 13 COMP 11
compSet 13 ratio 10
# wyświetlenie konfiguracji plików MIDI
midiList
\end{lstlisting}

\section{Wyniki testów wydajności przetwarzania offline}
Wyniki testów wydajności przetwarzania offline zostały podzielone na dwie części. Pierwsza część przedstawia uśrednione czasy trwania samej syntezy oraz przetwarzania strumieni audio (\ref{fig:Średni czas trwania przetwarzania dźwięku w trybie offline}). Druga część przedstawia uśrednione czasy trwania wszelkich obliczeń, włączając w to odczyt i zapis do plików, przetworzenie danych MIDI i przygotowanie ich do formatu przyjmowanego przez syntezator, a w przypadku GPU transfer danych pomiędzy urządzeniami (\ref{fig:Średni czas trwania wszelkich operacji w trybie offline}).

\begin{figure}[H]
    \centering
    \scalebox{1.0}{%% Creator: Matplotlib, PGF backend
%%
%% To include the figure in your LaTeX document, write
%%   \input{<filename>.pgf}
%%
%% Make sure the required packages are loaded in your preamble
%%   \usepackage{pgf}
%%
%% Also ensure that all the required font packages are loaded; for instance,
%% the lmodern package is sometimes necessary when using math font.
%%   \usepackage{lmodern}
%%
%% Figures using additional raster images can only be included by \input if
%% they are in the same directory as the main LaTeX file. For loading figures
%% from other directories you can use the `import` package
%%   \usepackage{import}
%%
%% and then include the figures with
%%   \import{<path to file>}{<filename>.pgf}
%%
%% Matplotlib used the following preamble
%%   \def\mathdefault#1{#1}
%%   \everymath=\expandafter{\the\everymath\displaystyle}
%%   
%%   \usepackage{fontspec}
%%   \setmainfont{DejaVuSerif.ttf}[Path=\detokenize{/usr/lib/python3.12/site-packages/matplotlib/mpl-data/fonts/ttf/}]
%%   \setsansfont{DejaVuSans.ttf}[Path=\detokenize{/usr/lib/python3.12/site-packages/matplotlib/mpl-data/fonts/ttf/}]
%%   \setmonofont{DejaVuSansMono.ttf}[Path=\detokenize{/usr/lib/python3.12/site-packages/matplotlib/mpl-data/fonts/ttf/}]
%%   \makeatletter\@ifpackageloaded{underscore}{}{\usepackage[strings]{underscore}}\makeatother
%%
\begingroup%
\makeatletter%
\begin{pgfpicture}%
\pgfpathrectangle{\pgfpointorigin}{\pgfqpoint{5.990000in}{3.000000in}}%
\pgfusepath{use as bounding box, clip}%
\begin{pgfscope}%
\pgfsetbuttcap%
\pgfsetmiterjoin%
\definecolor{currentfill}{rgb}{1.000000,1.000000,1.000000}%
\pgfsetfillcolor{currentfill}%
\pgfsetlinewidth{0.000000pt}%
\definecolor{currentstroke}{rgb}{1.000000,1.000000,1.000000}%
\pgfsetstrokecolor{currentstroke}%
\pgfsetdash{}{0pt}%
\pgfpathmoveto{\pgfqpoint{0.000000in}{0.000000in}}%
\pgfpathlineto{\pgfqpoint{5.990000in}{0.000000in}}%
\pgfpathlineto{\pgfqpoint{5.990000in}{3.000000in}}%
\pgfpathlineto{\pgfqpoint{0.000000in}{3.000000in}}%
\pgfpathlineto{\pgfqpoint{0.000000in}{0.000000in}}%
\pgfpathclose%
\pgfusepath{fill}%
\end{pgfscope}%
\begin{pgfscope}%
\pgfsetbuttcap%
\pgfsetmiterjoin%
\definecolor{currentfill}{rgb}{1.000000,1.000000,1.000000}%
\pgfsetfillcolor{currentfill}%
\pgfsetlinewidth{0.000000pt}%
\definecolor{currentstroke}{rgb}{0.000000,0.000000,0.000000}%
\pgfsetstrokecolor{currentstroke}%
\pgfsetstrokeopacity{0.000000}%
\pgfsetdash{}{0pt}%
\pgfpathmoveto{\pgfqpoint{0.717829in}{0.387222in}}%
\pgfpathlineto{\pgfqpoint{5.840250in}{0.387222in}}%
\pgfpathlineto{\pgfqpoint{5.840250in}{2.850000in}}%
\pgfpathlineto{\pgfqpoint{0.717829in}{2.850000in}}%
\pgfpathlineto{\pgfqpoint{0.717829in}{0.387222in}}%
\pgfpathclose%
\pgfusepath{fill}%
\end{pgfscope}%
\begin{pgfscope}%
\pgfpathrectangle{\pgfqpoint{0.717829in}{0.387222in}}{\pgfqpoint{5.122421in}{2.462778in}}%
\pgfusepath{clip}%
\pgfsetbuttcap%
\pgfsetmiterjoin%
\definecolor{currentfill}{rgb}{1.000000,0.000000,0.000000}%
\pgfsetfillcolor{currentfill}%
\pgfsetlinewidth{0.000000pt}%
\definecolor{currentstroke}{rgb}{0.000000,0.000000,0.000000}%
\pgfsetstrokecolor{currentstroke}%
\pgfsetstrokeopacity{0.000000}%
\pgfsetdash{}{0pt}%
\pgfpathmoveto{\pgfqpoint{0.950667in}{0.387222in}}%
\pgfpathlineto{\pgfqpoint{1.592976in}{0.387222in}}%
\pgfpathlineto{\pgfqpoint{1.592976in}{2.608362in}}%
\pgfpathlineto{\pgfqpoint{0.950667in}{2.608362in}}%
\pgfpathlineto{\pgfqpoint{0.950667in}{0.387222in}}%
\pgfpathclose%
\pgfusepath{fill}%
\end{pgfscope}%
\begin{pgfscope}%
\pgfpathrectangle{\pgfqpoint{0.717829in}{0.387222in}}{\pgfqpoint{5.122421in}{2.462778in}}%
\pgfusepath{clip}%
\pgfsetbuttcap%
\pgfsetmiterjoin%
\definecolor{currentfill}{rgb}{0.000000,0.000000,1.000000}%
\pgfsetfillcolor{currentfill}%
\pgfsetlinewidth{0.000000pt}%
\definecolor{currentstroke}{rgb}{0.000000,0.000000,0.000000}%
\pgfsetstrokecolor{currentstroke}%
\pgfsetstrokeopacity{0.000000}%
\pgfsetdash{}{0pt}%
\pgfpathmoveto{\pgfqpoint{1.753554in}{0.387222in}}%
\pgfpathlineto{\pgfqpoint{2.395864in}{0.387222in}}%
\pgfpathlineto{\pgfqpoint{2.395864in}{2.626111in}}%
\pgfpathlineto{\pgfqpoint{1.753554in}{2.626111in}}%
\pgfpathlineto{\pgfqpoint{1.753554in}{0.387222in}}%
\pgfpathclose%
\pgfusepath{fill}%
\end{pgfscope}%
\begin{pgfscope}%
\pgfpathrectangle{\pgfqpoint{0.717829in}{0.387222in}}{\pgfqpoint{5.122421in}{2.462778in}}%
\pgfusepath{clip}%
\pgfsetbuttcap%
\pgfsetmiterjoin%
\definecolor{currentfill}{rgb}{1.000000,0.000000,0.000000}%
\pgfsetfillcolor{currentfill}%
\pgfsetlinewidth{0.000000pt}%
\definecolor{currentstroke}{rgb}{0.000000,0.000000,0.000000}%
\pgfsetstrokecolor{currentstroke}%
\pgfsetstrokeopacity{0.000000}%
\pgfsetdash{}{0pt}%
\pgfpathmoveto{\pgfqpoint{2.556441in}{0.387222in}}%
\pgfpathlineto{\pgfqpoint{3.198751in}{0.387222in}}%
\pgfpathlineto{\pgfqpoint{3.198751in}{2.552804in}}%
\pgfpathlineto{\pgfqpoint{2.556441in}{2.552804in}}%
\pgfpathlineto{\pgfqpoint{2.556441in}{0.387222in}}%
\pgfpathclose%
\pgfusepath{fill}%
\end{pgfscope}%
\begin{pgfscope}%
\pgfpathrectangle{\pgfqpoint{0.717829in}{0.387222in}}{\pgfqpoint{5.122421in}{2.462778in}}%
\pgfusepath{clip}%
\pgfsetbuttcap%
\pgfsetmiterjoin%
\definecolor{currentfill}{rgb}{0.000000,0.000000,1.000000}%
\pgfsetfillcolor{currentfill}%
\pgfsetlinewidth{0.000000pt}%
\definecolor{currentstroke}{rgb}{0.000000,0.000000,0.000000}%
\pgfsetstrokecolor{currentstroke}%
\pgfsetstrokeopacity{0.000000}%
\pgfsetdash{}{0pt}%
\pgfpathmoveto{\pgfqpoint{3.359328in}{0.387222in}}%
\pgfpathlineto{\pgfqpoint{4.001638in}{0.387222in}}%
\pgfpathlineto{\pgfqpoint{4.001638in}{2.033502in}}%
\pgfpathlineto{\pgfqpoint{3.359328in}{2.033502in}}%
\pgfpathlineto{\pgfqpoint{3.359328in}{0.387222in}}%
\pgfpathclose%
\pgfusepath{fill}%
\end{pgfscope}%
\begin{pgfscope}%
\pgfpathrectangle{\pgfqpoint{0.717829in}{0.387222in}}{\pgfqpoint{5.122421in}{2.462778in}}%
\pgfusepath{clip}%
\pgfsetbuttcap%
\pgfsetmiterjoin%
\definecolor{currentfill}{rgb}{1.000000,0.000000,0.000000}%
\pgfsetfillcolor{currentfill}%
\pgfsetlinewidth{0.000000pt}%
\definecolor{currentstroke}{rgb}{0.000000,0.000000,0.000000}%
\pgfsetstrokecolor{currentstroke}%
\pgfsetstrokeopacity{0.000000}%
\pgfsetdash{}{0pt}%
\pgfpathmoveto{\pgfqpoint{4.162216in}{0.387222in}}%
\pgfpathlineto{\pgfqpoint{4.804525in}{0.387222in}}%
\pgfpathlineto{\pgfqpoint{4.804525in}{2.607936in}}%
\pgfpathlineto{\pgfqpoint{4.162216in}{2.607936in}}%
\pgfpathlineto{\pgfqpoint{4.162216in}{0.387222in}}%
\pgfpathclose%
\pgfusepath{fill}%
\end{pgfscope}%
\begin{pgfscope}%
\pgfpathrectangle{\pgfqpoint{0.717829in}{0.387222in}}{\pgfqpoint{5.122421in}{2.462778in}}%
\pgfusepath{clip}%
\pgfsetbuttcap%
\pgfsetmiterjoin%
\definecolor{currentfill}{rgb}{0.000000,0.000000,1.000000}%
\pgfsetfillcolor{currentfill}%
\pgfsetlinewidth{0.000000pt}%
\definecolor{currentstroke}{rgb}{0.000000,0.000000,0.000000}%
\pgfsetstrokecolor{currentstroke}%
\pgfsetstrokeopacity{0.000000}%
\pgfsetdash{}{0pt}%
\pgfpathmoveto{\pgfqpoint{4.965103in}{0.387222in}}%
\pgfpathlineto{\pgfqpoint{5.607413in}{0.387222in}}%
\pgfpathlineto{\pgfqpoint{5.607413in}{1.920089in}}%
\pgfpathlineto{\pgfqpoint{4.965103in}{1.920089in}}%
\pgfpathlineto{\pgfqpoint{4.965103in}{0.387222in}}%
\pgfpathclose%
\pgfusepath{fill}%
\end{pgfscope}%
\begin{pgfscope}%
\pgfsetbuttcap%
\pgfsetroundjoin%
\definecolor{currentfill}{rgb}{0.000000,0.000000,0.000000}%
\pgfsetfillcolor{currentfill}%
\pgfsetlinewidth{0.803000pt}%
\definecolor{currentstroke}{rgb}{0.000000,0.000000,0.000000}%
\pgfsetstrokecolor{currentstroke}%
\pgfsetdash{}{0pt}%
\pgfsys@defobject{currentmarker}{\pgfqpoint{0.000000in}{-0.048611in}}{\pgfqpoint{0.000000in}{0.000000in}}{%
\pgfpathmoveto{\pgfqpoint{0.000000in}{0.000000in}}%
\pgfpathlineto{\pgfqpoint{0.000000in}{-0.048611in}}%
\pgfusepath{stroke,fill}%
}%
\begin{pgfscope}%
\pgfsys@transformshift{1.271822in}{0.387222in}%
\pgfsys@useobject{currentmarker}{}%
\end{pgfscope}%
\end{pgfscope}%
\begin{pgfscope}%
\definecolor{textcolor}{rgb}{0.000000,0.000000,0.000000}%
\pgfsetstrokecolor{textcolor}%
\pgfsetfillcolor{textcolor}%
\pgftext[x=1.271822in,y=0.290000in,,top]{\color{textcolor}{\sffamily\fontsize{10.000000}{12.000000}\selectfont\catcode`\^=\active\def^{\ifmmode\sp\else\^{}\fi}\catcode`\%=\active\def%{\%}CPU 128}}%
\end{pgfscope}%
\begin{pgfscope}%
\pgfsetbuttcap%
\pgfsetroundjoin%
\definecolor{currentfill}{rgb}{0.000000,0.000000,0.000000}%
\pgfsetfillcolor{currentfill}%
\pgfsetlinewidth{0.803000pt}%
\definecolor{currentstroke}{rgb}{0.000000,0.000000,0.000000}%
\pgfsetstrokecolor{currentstroke}%
\pgfsetdash{}{0pt}%
\pgfsys@defobject{currentmarker}{\pgfqpoint{0.000000in}{-0.048611in}}{\pgfqpoint{0.000000in}{0.000000in}}{%
\pgfpathmoveto{\pgfqpoint{0.000000in}{0.000000in}}%
\pgfpathlineto{\pgfqpoint{0.000000in}{-0.048611in}}%
\pgfusepath{stroke,fill}%
}%
\begin{pgfscope}%
\pgfsys@transformshift{2.074709in}{0.387222in}%
\pgfsys@useobject{currentmarker}{}%
\end{pgfscope}%
\end{pgfscope}%
\begin{pgfscope}%
\definecolor{textcolor}{rgb}{0.000000,0.000000,0.000000}%
\pgfsetstrokecolor{textcolor}%
\pgfsetfillcolor{textcolor}%
\pgftext[x=2.074709in,y=0.290000in,,top]{\color{textcolor}{\sffamily\fontsize{10.000000}{12.000000}\selectfont\catcode`\^=\active\def^{\ifmmode\sp\else\^{}\fi}\catcode`\%=\active\def%{\%}GPU 128}}%
\end{pgfscope}%
\begin{pgfscope}%
\pgfsetbuttcap%
\pgfsetroundjoin%
\definecolor{currentfill}{rgb}{0.000000,0.000000,0.000000}%
\pgfsetfillcolor{currentfill}%
\pgfsetlinewidth{0.803000pt}%
\definecolor{currentstroke}{rgb}{0.000000,0.000000,0.000000}%
\pgfsetstrokecolor{currentstroke}%
\pgfsetdash{}{0pt}%
\pgfsys@defobject{currentmarker}{\pgfqpoint{0.000000in}{-0.048611in}}{\pgfqpoint{0.000000in}{0.000000in}}{%
\pgfpathmoveto{\pgfqpoint{0.000000in}{0.000000in}}%
\pgfpathlineto{\pgfqpoint{0.000000in}{-0.048611in}}%
\pgfusepath{stroke,fill}%
}%
\begin{pgfscope}%
\pgfsys@transformshift{2.877596in}{0.387222in}%
\pgfsys@useobject{currentmarker}{}%
\end{pgfscope}%
\end{pgfscope}%
\begin{pgfscope}%
\definecolor{textcolor}{rgb}{0.000000,0.000000,0.000000}%
\pgfsetstrokecolor{textcolor}%
\pgfsetfillcolor{textcolor}%
\pgftext[x=2.877596in,y=0.290000in,,top]{\color{textcolor}{\sffamily\fontsize{10.000000}{12.000000}\selectfont\catcode`\^=\active\def^{\ifmmode\sp\else\^{}\fi}\catcode`\%=\active\def%{\%}CPU 1024}}%
\end{pgfscope}%
\begin{pgfscope}%
\pgfsetbuttcap%
\pgfsetroundjoin%
\definecolor{currentfill}{rgb}{0.000000,0.000000,0.000000}%
\pgfsetfillcolor{currentfill}%
\pgfsetlinewidth{0.803000pt}%
\definecolor{currentstroke}{rgb}{0.000000,0.000000,0.000000}%
\pgfsetstrokecolor{currentstroke}%
\pgfsetdash{}{0pt}%
\pgfsys@defobject{currentmarker}{\pgfqpoint{0.000000in}{-0.048611in}}{\pgfqpoint{0.000000in}{0.000000in}}{%
\pgfpathmoveto{\pgfqpoint{0.000000in}{0.000000in}}%
\pgfpathlineto{\pgfqpoint{0.000000in}{-0.048611in}}%
\pgfusepath{stroke,fill}%
}%
\begin{pgfscope}%
\pgfsys@transformshift{3.680483in}{0.387222in}%
\pgfsys@useobject{currentmarker}{}%
\end{pgfscope}%
\end{pgfscope}%
\begin{pgfscope}%
\definecolor{textcolor}{rgb}{0.000000,0.000000,0.000000}%
\pgfsetstrokecolor{textcolor}%
\pgfsetfillcolor{textcolor}%
\pgftext[x=3.680483in,y=0.290000in,,top]{\color{textcolor}{\sffamily\fontsize{10.000000}{12.000000}\selectfont\catcode`\^=\active\def^{\ifmmode\sp\else\^{}\fi}\catcode`\%=\active\def%{\%}GPU 1024}}%
\end{pgfscope}%
\begin{pgfscope}%
\pgfsetbuttcap%
\pgfsetroundjoin%
\definecolor{currentfill}{rgb}{0.000000,0.000000,0.000000}%
\pgfsetfillcolor{currentfill}%
\pgfsetlinewidth{0.803000pt}%
\definecolor{currentstroke}{rgb}{0.000000,0.000000,0.000000}%
\pgfsetstrokecolor{currentstroke}%
\pgfsetdash{}{0pt}%
\pgfsys@defobject{currentmarker}{\pgfqpoint{0.000000in}{-0.048611in}}{\pgfqpoint{0.000000in}{0.000000in}}{%
\pgfpathmoveto{\pgfqpoint{0.000000in}{0.000000in}}%
\pgfpathlineto{\pgfqpoint{0.000000in}{-0.048611in}}%
\pgfusepath{stroke,fill}%
}%
\begin{pgfscope}%
\pgfsys@transformshift{4.483371in}{0.387222in}%
\pgfsys@useobject{currentmarker}{}%
\end{pgfscope}%
\end{pgfscope}%
\begin{pgfscope}%
\definecolor{textcolor}{rgb}{0.000000,0.000000,0.000000}%
\pgfsetstrokecolor{textcolor}%
\pgfsetfillcolor{textcolor}%
\pgftext[x=4.483371in,y=0.290000in,,top]{\color{textcolor}{\sffamily\fontsize{10.000000}{12.000000}\selectfont\catcode`\^=\active\def^{\ifmmode\sp\else\^{}\fi}\catcode`\%=\active\def%{\%}CPU 65536}}%
\end{pgfscope}%
\begin{pgfscope}%
\pgfsetbuttcap%
\pgfsetroundjoin%
\definecolor{currentfill}{rgb}{0.000000,0.000000,0.000000}%
\pgfsetfillcolor{currentfill}%
\pgfsetlinewidth{0.803000pt}%
\definecolor{currentstroke}{rgb}{0.000000,0.000000,0.000000}%
\pgfsetstrokecolor{currentstroke}%
\pgfsetdash{}{0pt}%
\pgfsys@defobject{currentmarker}{\pgfqpoint{0.000000in}{-0.048611in}}{\pgfqpoint{0.000000in}{0.000000in}}{%
\pgfpathmoveto{\pgfqpoint{0.000000in}{0.000000in}}%
\pgfpathlineto{\pgfqpoint{0.000000in}{-0.048611in}}%
\pgfusepath{stroke,fill}%
}%
\begin{pgfscope}%
\pgfsys@transformshift{5.286258in}{0.387222in}%
\pgfsys@useobject{currentmarker}{}%
\end{pgfscope}%
\end{pgfscope}%
\begin{pgfscope}%
\definecolor{textcolor}{rgb}{0.000000,0.000000,0.000000}%
\pgfsetstrokecolor{textcolor}%
\pgfsetfillcolor{textcolor}%
\pgftext[x=5.286258in,y=0.290000in,,top]{\color{textcolor}{\sffamily\fontsize{10.000000}{12.000000}\selectfont\catcode`\^=\active\def^{\ifmmode\sp\else\^{}\fi}\catcode`\%=\active\def%{\%}GPU 65536}}%
\end{pgfscope}%
\begin{pgfscope}%
\pgfsetbuttcap%
\pgfsetroundjoin%
\definecolor{currentfill}{rgb}{0.000000,0.000000,0.000000}%
\pgfsetfillcolor{currentfill}%
\pgfsetlinewidth{0.803000pt}%
\definecolor{currentstroke}{rgb}{0.000000,0.000000,0.000000}%
\pgfsetstrokecolor{currentstroke}%
\pgfsetdash{}{0pt}%
\pgfsys@defobject{currentmarker}{\pgfqpoint{-0.048611in}{0.000000in}}{\pgfqpoint{-0.000000in}{0.000000in}}{%
\pgfpathmoveto{\pgfqpoint{-0.000000in}{0.000000in}}%
\pgfpathlineto{\pgfqpoint{-0.048611in}{0.000000in}}%
\pgfusepath{stroke,fill}%
}%
\begin{pgfscope}%
\pgfsys@transformshift{0.717829in}{0.387222in}%
\pgfsys@useobject{currentmarker}{}%
\end{pgfscope}%
\end{pgfscope}%
\begin{pgfscope}%
\definecolor{textcolor}{rgb}{0.000000,0.000000,0.000000}%
\pgfsetstrokecolor{textcolor}%
\pgfsetfillcolor{textcolor}%
\pgftext[x=0.532242in, y=0.334461in, left, base]{\color{textcolor}{\sffamily\fontsize{10.000000}{12.000000}\selectfont\catcode`\^=\active\def^{\ifmmode\sp\else\^{}\fi}\catcode`\%=\active\def%{\%}0}}%
\end{pgfscope}%
\begin{pgfscope}%
\pgfsetbuttcap%
\pgfsetroundjoin%
\definecolor{currentfill}{rgb}{0.000000,0.000000,0.000000}%
\pgfsetfillcolor{currentfill}%
\pgfsetlinewidth{0.803000pt}%
\definecolor{currentstroke}{rgb}{0.000000,0.000000,0.000000}%
\pgfsetstrokecolor{currentstroke}%
\pgfsetdash{}{0pt}%
\pgfsys@defobject{currentmarker}{\pgfqpoint{-0.048611in}{0.000000in}}{\pgfqpoint{-0.000000in}{0.000000in}}{%
\pgfpathmoveto{\pgfqpoint{-0.000000in}{0.000000in}}%
\pgfpathlineto{\pgfqpoint{-0.048611in}{0.000000in}}%
\pgfusepath{stroke,fill}%
}%
\begin{pgfscope}%
\pgfsys@transformshift{0.717829in}{0.859025in}%
\pgfsys@useobject{currentmarker}{}%
\end{pgfscope}%
\end{pgfscope}%
\begin{pgfscope}%
\definecolor{textcolor}{rgb}{0.000000,0.000000,0.000000}%
\pgfsetstrokecolor{textcolor}%
\pgfsetfillcolor{textcolor}%
\pgftext[x=0.443876in, y=0.806264in, left, base]{\color{textcolor}{\sffamily\fontsize{10.000000}{12.000000}\selectfont\catcode`\^=\active\def^{\ifmmode\sp\else\^{}\fi}\catcode`\%=\active\def%{\%}50}}%
\end{pgfscope}%
\begin{pgfscope}%
\pgfsetbuttcap%
\pgfsetroundjoin%
\definecolor{currentfill}{rgb}{0.000000,0.000000,0.000000}%
\pgfsetfillcolor{currentfill}%
\pgfsetlinewidth{0.803000pt}%
\definecolor{currentstroke}{rgb}{0.000000,0.000000,0.000000}%
\pgfsetstrokecolor{currentstroke}%
\pgfsetdash{}{0pt}%
\pgfsys@defobject{currentmarker}{\pgfqpoint{-0.048611in}{0.000000in}}{\pgfqpoint{-0.000000in}{0.000000in}}{%
\pgfpathmoveto{\pgfqpoint{-0.000000in}{0.000000in}}%
\pgfpathlineto{\pgfqpoint{-0.048611in}{0.000000in}}%
\pgfusepath{stroke,fill}%
}%
\begin{pgfscope}%
\pgfsys@transformshift{0.717829in}{1.330829in}%
\pgfsys@useobject{currentmarker}{}%
\end{pgfscope}%
\end{pgfscope}%
\begin{pgfscope}%
\definecolor{textcolor}{rgb}{0.000000,0.000000,0.000000}%
\pgfsetstrokecolor{textcolor}%
\pgfsetfillcolor{textcolor}%
\pgftext[x=0.355511in, y=1.278067in, left, base]{\color{textcolor}{\sffamily\fontsize{10.000000}{12.000000}\selectfont\catcode`\^=\active\def^{\ifmmode\sp\else\^{}\fi}\catcode`\%=\active\def%{\%}100}}%
\end{pgfscope}%
\begin{pgfscope}%
\pgfsetbuttcap%
\pgfsetroundjoin%
\definecolor{currentfill}{rgb}{0.000000,0.000000,0.000000}%
\pgfsetfillcolor{currentfill}%
\pgfsetlinewidth{0.803000pt}%
\definecolor{currentstroke}{rgb}{0.000000,0.000000,0.000000}%
\pgfsetstrokecolor{currentstroke}%
\pgfsetdash{}{0pt}%
\pgfsys@defobject{currentmarker}{\pgfqpoint{-0.048611in}{0.000000in}}{\pgfqpoint{-0.000000in}{0.000000in}}{%
\pgfpathmoveto{\pgfqpoint{-0.000000in}{0.000000in}}%
\pgfpathlineto{\pgfqpoint{-0.048611in}{0.000000in}}%
\pgfusepath{stroke,fill}%
}%
\begin{pgfscope}%
\pgfsys@transformshift{0.717829in}{1.802632in}%
\pgfsys@useobject{currentmarker}{}%
\end{pgfscope}%
\end{pgfscope}%
\begin{pgfscope}%
\definecolor{textcolor}{rgb}{0.000000,0.000000,0.000000}%
\pgfsetstrokecolor{textcolor}%
\pgfsetfillcolor{textcolor}%
\pgftext[x=0.355511in, y=1.749870in, left, base]{\color{textcolor}{\sffamily\fontsize{10.000000}{12.000000}\selectfont\catcode`\^=\active\def^{\ifmmode\sp\else\^{}\fi}\catcode`\%=\active\def%{\%}150}}%
\end{pgfscope}%
\begin{pgfscope}%
\pgfsetbuttcap%
\pgfsetroundjoin%
\definecolor{currentfill}{rgb}{0.000000,0.000000,0.000000}%
\pgfsetfillcolor{currentfill}%
\pgfsetlinewidth{0.803000pt}%
\definecolor{currentstroke}{rgb}{0.000000,0.000000,0.000000}%
\pgfsetstrokecolor{currentstroke}%
\pgfsetdash{}{0pt}%
\pgfsys@defobject{currentmarker}{\pgfqpoint{-0.048611in}{0.000000in}}{\pgfqpoint{-0.000000in}{0.000000in}}{%
\pgfpathmoveto{\pgfqpoint{-0.000000in}{0.000000in}}%
\pgfpathlineto{\pgfqpoint{-0.048611in}{0.000000in}}%
\pgfusepath{stroke,fill}%
}%
\begin{pgfscope}%
\pgfsys@transformshift{0.717829in}{2.274435in}%
\pgfsys@useobject{currentmarker}{}%
\end{pgfscope}%
\end{pgfscope}%
\begin{pgfscope}%
\definecolor{textcolor}{rgb}{0.000000,0.000000,0.000000}%
\pgfsetstrokecolor{textcolor}%
\pgfsetfillcolor{textcolor}%
\pgftext[x=0.355511in, y=2.221674in, left, base]{\color{textcolor}{\sffamily\fontsize{10.000000}{12.000000}\selectfont\catcode`\^=\active\def^{\ifmmode\sp\else\^{}\fi}\catcode`\%=\active\def%{\%}200}}%
\end{pgfscope}%
\begin{pgfscope}%
\pgfsetbuttcap%
\pgfsetroundjoin%
\definecolor{currentfill}{rgb}{0.000000,0.000000,0.000000}%
\pgfsetfillcolor{currentfill}%
\pgfsetlinewidth{0.803000pt}%
\definecolor{currentstroke}{rgb}{0.000000,0.000000,0.000000}%
\pgfsetstrokecolor{currentstroke}%
\pgfsetdash{}{0pt}%
\pgfsys@defobject{currentmarker}{\pgfqpoint{-0.048611in}{0.000000in}}{\pgfqpoint{-0.000000in}{0.000000in}}{%
\pgfpathmoveto{\pgfqpoint{-0.000000in}{0.000000in}}%
\pgfpathlineto{\pgfqpoint{-0.048611in}{0.000000in}}%
\pgfusepath{stroke,fill}%
}%
\begin{pgfscope}%
\pgfsys@transformshift{0.717829in}{2.746239in}%
\pgfsys@useobject{currentmarker}{}%
\end{pgfscope}%
\end{pgfscope}%
\begin{pgfscope}%
\definecolor{textcolor}{rgb}{0.000000,0.000000,0.000000}%
\pgfsetstrokecolor{textcolor}%
\pgfsetfillcolor{textcolor}%
\pgftext[x=0.355511in, y=2.693477in, left, base]{\color{textcolor}{\sffamily\fontsize{10.000000}{12.000000}\selectfont\catcode`\^=\active\def^{\ifmmode\sp\else\^{}\fi}\catcode`\%=\active\def%{\%}250}}%
\end{pgfscope}%
\begin{pgfscope}%
\definecolor{textcolor}{rgb}{0.000000,0.000000,0.000000}%
\pgfsetstrokecolor{textcolor}%
\pgfsetfillcolor{textcolor}%
\pgftext[x=0.299956in,y=1.618611in,,bottom,rotate=90.000000]{\color{textcolor}{\sffamily\fontsize{10.000000}{12.000000}\selectfont\catcode`\^=\active\def^{\ifmmode\sp\else\^{}\fi}\catcode`\%=\active\def%{\%}Średni czas trwania (s)}}%
\end{pgfscope}%
\begin{pgfscope}%
\pgfsetrectcap%
\pgfsetmiterjoin%
\pgfsetlinewidth{0.803000pt}%
\definecolor{currentstroke}{rgb}{0.000000,0.000000,0.000000}%
\pgfsetstrokecolor{currentstroke}%
\pgfsetdash{}{0pt}%
\pgfpathmoveto{\pgfqpoint{0.717829in}{0.387222in}}%
\pgfpathlineto{\pgfqpoint{0.717829in}{2.850000in}}%
\pgfusepath{stroke}%
\end{pgfscope}%
\begin{pgfscope}%
\pgfsetrectcap%
\pgfsetmiterjoin%
\pgfsetlinewidth{0.803000pt}%
\definecolor{currentstroke}{rgb}{0.000000,0.000000,0.000000}%
\pgfsetstrokecolor{currentstroke}%
\pgfsetdash{}{0pt}%
\pgfpathmoveto{\pgfqpoint{5.840250in}{0.387222in}}%
\pgfpathlineto{\pgfqpoint{5.840250in}{2.850000in}}%
\pgfusepath{stroke}%
\end{pgfscope}%
\begin{pgfscope}%
\pgfsetrectcap%
\pgfsetmiterjoin%
\pgfsetlinewidth{0.803000pt}%
\definecolor{currentstroke}{rgb}{0.000000,0.000000,0.000000}%
\pgfsetstrokecolor{currentstroke}%
\pgfsetdash{}{0pt}%
\pgfpathmoveto{\pgfqpoint{0.717829in}{0.387222in}}%
\pgfpathlineto{\pgfqpoint{5.840250in}{0.387222in}}%
\pgfusepath{stroke}%
\end{pgfscope}%
\begin{pgfscope}%
\pgfsetrectcap%
\pgfsetmiterjoin%
\pgfsetlinewidth{0.803000pt}%
\definecolor{currentstroke}{rgb}{0.000000,0.000000,0.000000}%
\pgfsetstrokecolor{currentstroke}%
\pgfsetdash{}{0pt}%
\pgfpathmoveto{\pgfqpoint{0.717829in}{2.850000in}}%
\pgfpathlineto{\pgfqpoint{5.840250in}{2.850000in}}%
\pgfusepath{stroke}%
\end{pgfscope}%
\begin{pgfscope}%
\definecolor{textcolor}{rgb}{0.000000,0.000000,0.000000}%
\pgfsetstrokecolor{textcolor}%
\pgfsetfillcolor{textcolor}%
\pgftext[x=1.271822in,y=2.608362in,,bottom]{\color{textcolor}{\sffamily\fontsize{10.000000}{12.000000}\selectfont\catcode`\^=\active\def^{\ifmmode\sp\else\^{}\fi}\catcode`\%=\active\def%{\%}235.388}}%
\end{pgfscope}%
\begin{pgfscope}%
\definecolor{textcolor}{rgb}{0.000000,0.000000,0.000000}%
\pgfsetstrokecolor{textcolor}%
\pgfsetfillcolor{textcolor}%
\pgftext[x=2.074709in,y=2.626111in,,bottom]{\color{textcolor}{\sffamily\fontsize{10.000000}{12.000000}\selectfont\catcode`\^=\active\def^{\ifmmode\sp\else\^{}\fi}\catcode`\%=\active\def%{\%}237.269}}%
\end{pgfscope}%
\begin{pgfscope}%
\definecolor{textcolor}{rgb}{0.000000,0.000000,0.000000}%
\pgfsetstrokecolor{textcolor}%
\pgfsetfillcolor{textcolor}%
\pgftext[x=2.877596in,y=2.552804in,,bottom]{\color{textcolor}{\sffamily\fontsize{10.000000}{12.000000}\selectfont\catcode`\^=\active\def^{\ifmmode\sp\else\^{}\fi}\catcode`\%=\active\def%{\%}229.501}}%
\end{pgfscope}%
\begin{pgfscope}%
\definecolor{textcolor}{rgb}{0.000000,0.000000,0.000000}%
\pgfsetstrokecolor{textcolor}%
\pgfsetfillcolor{textcolor}%
\pgftext[x=3.680483in,y=2.033502in,,bottom]{\color{textcolor}{\sffamily\fontsize{10.000000}{12.000000}\selectfont\catcode`\^=\active\def^{\ifmmode\sp\else\^{}\fi}\catcode`\%=\active\def%{\%}174.467}}%
\end{pgfscope}%
\begin{pgfscope}%
\definecolor{textcolor}{rgb}{0.000000,0.000000,0.000000}%
\pgfsetstrokecolor{textcolor}%
\pgfsetfillcolor{textcolor}%
\pgftext[x=4.483371in,y=2.607936in,,bottom]{\color{textcolor}{\sffamily\fontsize{10.000000}{12.000000}\selectfont\catcode`\^=\active\def^{\ifmmode\sp\else\^{}\fi}\catcode`\%=\active\def%{\%}235.343}}%
\end{pgfscope}%
\begin{pgfscope}%
\definecolor{textcolor}{rgb}{0.000000,0.000000,0.000000}%
\pgfsetstrokecolor{textcolor}%
\pgfsetfillcolor{textcolor}%
\pgftext[x=5.286258in,y=1.920089in,,bottom]{\color{textcolor}{\sffamily\fontsize{10.000000}{12.000000}\selectfont\catcode`\^=\active\def^{\ifmmode\sp\else\^{}\fi}\catcode`\%=\active\def%{\%}162.448}}%
\end{pgfscope}%
\end{pgfpicture}%
\makeatother%
\endgroup%
}
    \caption{Średni czas trwania przetwarzania dźwięku w trybie offline}
    \label{fig:Średni czas trwania przetwarzania dźwięku w trybie offline}
\end{figure}

Można zauważyć, iż w przypadku przetwarzania offline, wielkość bufora nie miała znaczenia dla implementacji CPU. W przypadku implementacji GPU, zwiększenie rozmiaru bufora znacząco wpłynęło na czas przetwarzania, zmniejszając wykorzystany czas wraz ze wzrostem bufora. W przypadku implementacji CPU, czas przetwarzania był zbliżony dla wszystkich rozmiarów bufora, co może wskazywać na to, że w przypadku CPU, czas przetwarzania jest zdominowany przez inne czynniki niż rozmiar bufora. W przypadku implementacji GPU, zwiększenie rozmiaru bufora pozwala na wykonanie kerneli o wiekszych rozmiarach, co skutkuje zmniejszeniem czasu przetwarzania.

\begin{figure}[H]
    \centering
    \scalebox{1.0}{%% Creator: Matplotlib, PGF backend
%%
%% To include the figure in your LaTeX document, write
%%   \input{<filename>.pgf}
%%
%% Make sure the required packages are loaded in your preamble
%%   \usepackage{pgf}
%%
%% Also ensure that all the required font packages are loaded; for instance,
%% the lmodern package is sometimes necessary when using math font.
%%   \usepackage{lmodern}
%%
%% Figures using additional raster images can only be included by \input if
%% they are in the same directory as the main LaTeX file. For loading figures
%% from other directories you can use the `import` package
%%   \usepackage{import}
%%
%% and then include the figures with
%%   \import{<path to file>}{<filename>.pgf}
%%
%% Matplotlib used the following preamble
%%   \def\mathdefault#1{#1}
%%   \everymath=\expandafter{\the\everymath\displaystyle}
%%   
%%   \usepackage{fontspec}
%%   \setmainfont{DejaVuSerif.ttf}[Path=\detokenize{/usr/lib/python3.12/site-packages/matplotlib/mpl-data/fonts/ttf/}]
%%   \setsansfont{DejaVuSans.ttf}[Path=\detokenize{/usr/lib/python3.12/site-packages/matplotlib/mpl-data/fonts/ttf/}]
%%   \setmonofont{DejaVuSansMono.ttf}[Path=\detokenize{/usr/lib/python3.12/site-packages/matplotlib/mpl-data/fonts/ttf/}]
%%   \makeatletter\@ifpackageloaded{underscore}{}{\usepackage[strings]{underscore}}\makeatother
%%
\begingroup%
\makeatletter%
\begin{pgfpicture}%
\pgfpathrectangle{\pgfpointorigin}{\pgfqpoint{5.990000in}{3.000000in}}%
\pgfusepath{use as bounding box, clip}%
\begin{pgfscope}%
\pgfsetbuttcap%
\pgfsetmiterjoin%
\definecolor{currentfill}{rgb}{1.000000,1.000000,1.000000}%
\pgfsetfillcolor{currentfill}%
\pgfsetlinewidth{0.000000pt}%
\definecolor{currentstroke}{rgb}{1.000000,1.000000,1.000000}%
\pgfsetstrokecolor{currentstroke}%
\pgfsetdash{}{0pt}%
\pgfpathmoveto{\pgfqpoint{0.000000in}{0.000000in}}%
\pgfpathlineto{\pgfqpoint{5.990000in}{0.000000in}}%
\pgfpathlineto{\pgfqpoint{5.990000in}{3.000000in}}%
\pgfpathlineto{\pgfqpoint{0.000000in}{3.000000in}}%
\pgfpathlineto{\pgfqpoint{0.000000in}{0.000000in}}%
\pgfpathclose%
\pgfusepath{fill}%
\end{pgfscope}%
\begin{pgfscope}%
\pgfsetbuttcap%
\pgfsetmiterjoin%
\definecolor{currentfill}{rgb}{1.000000,1.000000,1.000000}%
\pgfsetfillcolor{currentfill}%
\pgfsetlinewidth{0.000000pt}%
\definecolor{currentstroke}{rgb}{0.000000,0.000000,0.000000}%
\pgfsetstrokecolor{currentstroke}%
\pgfsetstrokeopacity{0.000000}%
\pgfsetdash{}{0pt}%
\pgfpathmoveto{\pgfqpoint{0.821406in}{0.387222in}}%
\pgfpathlineto{\pgfqpoint{5.840250in}{0.387222in}}%
\pgfpathlineto{\pgfqpoint{5.840250in}{2.850000in}}%
\pgfpathlineto{\pgfqpoint{0.821406in}{2.850000in}}%
\pgfpathlineto{\pgfqpoint{0.821406in}{0.387222in}}%
\pgfpathclose%
\pgfusepath{fill}%
\end{pgfscope}%
\begin{pgfscope}%
\pgfpathrectangle{\pgfqpoint{0.821406in}{0.387222in}}{\pgfqpoint{5.018844in}{2.462778in}}%
\pgfusepath{clip}%
\pgfsetbuttcap%
\pgfsetmiterjoin%
\definecolor{currentfill}{rgb}{1.000000,0.000000,0.000000}%
\pgfsetfillcolor{currentfill}%
\pgfsetlinewidth{0.000000pt}%
\definecolor{currentstroke}{rgb}{0.000000,0.000000,0.000000}%
\pgfsetstrokecolor{currentstroke}%
\pgfsetstrokeopacity{0.000000}%
\pgfsetdash{}{0pt}%
\pgfpathmoveto{\pgfqpoint{1.049536in}{0.387222in}}%
\pgfpathlineto{\pgfqpoint{1.678858in}{0.387222in}}%
\pgfpathlineto{\pgfqpoint{1.678858in}{1.059370in}}%
\pgfpathlineto{\pgfqpoint{1.049536in}{1.059370in}}%
\pgfpathlineto{\pgfqpoint{1.049536in}{0.387222in}}%
\pgfpathclose%
\pgfusepath{fill}%
\end{pgfscope}%
\begin{pgfscope}%
\pgfpathrectangle{\pgfqpoint{0.821406in}{0.387222in}}{\pgfqpoint{5.018844in}{2.462778in}}%
\pgfusepath{clip}%
\pgfsetbuttcap%
\pgfsetmiterjoin%
\definecolor{currentfill}{rgb}{0.000000,0.000000,1.000000}%
\pgfsetfillcolor{currentfill}%
\pgfsetlinewidth{0.000000pt}%
\definecolor{currentstroke}{rgb}{0.000000,0.000000,0.000000}%
\pgfsetstrokecolor{currentstroke}%
\pgfsetstrokeopacity{0.000000}%
\pgfsetdash{}{0pt}%
\pgfpathmoveto{\pgfqpoint{1.836188in}{0.387222in}}%
\pgfpathlineto{\pgfqpoint{2.465510in}{0.387222in}}%
\pgfpathlineto{\pgfqpoint{2.465510in}{2.626111in}}%
\pgfpathlineto{\pgfqpoint{1.836188in}{2.626111in}}%
\pgfpathlineto{\pgfqpoint{1.836188in}{0.387222in}}%
\pgfpathclose%
\pgfusepath{fill}%
\end{pgfscope}%
\begin{pgfscope}%
\pgfpathrectangle{\pgfqpoint{0.821406in}{0.387222in}}{\pgfqpoint{5.018844in}{2.462778in}}%
\pgfusepath{clip}%
\pgfsetbuttcap%
\pgfsetmiterjoin%
\definecolor{currentfill}{rgb}{1.000000,0.000000,0.000000}%
\pgfsetfillcolor{currentfill}%
\pgfsetlinewidth{0.000000pt}%
\definecolor{currentstroke}{rgb}{0.000000,0.000000,0.000000}%
\pgfsetstrokecolor{currentstroke}%
\pgfsetstrokeopacity{0.000000}%
\pgfsetdash{}{0pt}%
\pgfpathmoveto{\pgfqpoint{2.622841in}{0.387222in}}%
\pgfpathlineto{\pgfqpoint{3.252163in}{0.387222in}}%
\pgfpathlineto{\pgfqpoint{3.252163in}{1.042967in}}%
\pgfpathlineto{\pgfqpoint{2.622841in}{1.042967in}}%
\pgfpathlineto{\pgfqpoint{2.622841in}{0.387222in}}%
\pgfpathclose%
\pgfusepath{fill}%
\end{pgfscope}%
\begin{pgfscope}%
\pgfpathrectangle{\pgfqpoint{0.821406in}{0.387222in}}{\pgfqpoint{5.018844in}{2.462778in}}%
\pgfusepath{clip}%
\pgfsetbuttcap%
\pgfsetmiterjoin%
\definecolor{currentfill}{rgb}{0.000000,0.000000,1.000000}%
\pgfsetfillcolor{currentfill}%
\pgfsetlinewidth{0.000000pt}%
\definecolor{currentstroke}{rgb}{0.000000,0.000000,0.000000}%
\pgfsetstrokecolor{currentstroke}%
\pgfsetstrokeopacity{0.000000}%
\pgfsetdash{}{0pt}%
\pgfpathmoveto{\pgfqpoint{3.409493in}{0.387222in}}%
\pgfpathlineto{\pgfqpoint{4.038816in}{0.387222in}}%
\pgfpathlineto{\pgfqpoint{4.038816in}{1.037193in}}%
\pgfpathlineto{\pgfqpoint{3.409493in}{1.037193in}}%
\pgfpathlineto{\pgfqpoint{3.409493in}{0.387222in}}%
\pgfpathclose%
\pgfusepath{fill}%
\end{pgfscope}%
\begin{pgfscope}%
\pgfpathrectangle{\pgfqpoint{0.821406in}{0.387222in}}{\pgfqpoint{5.018844in}{2.462778in}}%
\pgfusepath{clip}%
\pgfsetbuttcap%
\pgfsetmiterjoin%
\definecolor{currentfill}{rgb}{1.000000,0.000000,0.000000}%
\pgfsetfillcolor{currentfill}%
\pgfsetlinewidth{0.000000pt}%
\definecolor{currentstroke}{rgb}{0.000000,0.000000,0.000000}%
\pgfsetstrokecolor{currentstroke}%
\pgfsetstrokeopacity{0.000000}%
\pgfsetdash{}{0pt}%
\pgfpathmoveto{\pgfqpoint{4.196146in}{0.387222in}}%
\pgfpathlineto{\pgfqpoint{4.825468in}{0.387222in}}%
\pgfpathlineto{\pgfqpoint{4.825468in}{1.060866in}}%
\pgfpathlineto{\pgfqpoint{4.196146in}{1.060866in}}%
\pgfpathlineto{\pgfqpoint{4.196146in}{0.387222in}}%
\pgfpathclose%
\pgfusepath{fill}%
\end{pgfscope}%
\begin{pgfscope}%
\pgfpathrectangle{\pgfqpoint{0.821406in}{0.387222in}}{\pgfqpoint{5.018844in}{2.462778in}}%
\pgfusepath{clip}%
\pgfsetbuttcap%
\pgfsetmiterjoin%
\definecolor{currentfill}{rgb}{0.000000,0.000000,1.000000}%
\pgfsetfillcolor{currentfill}%
\pgfsetlinewidth{0.000000pt}%
\definecolor{currentstroke}{rgb}{0.000000,0.000000,0.000000}%
\pgfsetstrokecolor{currentstroke}%
\pgfsetstrokeopacity{0.000000}%
\pgfsetdash{}{0pt}%
\pgfpathmoveto{\pgfqpoint{4.982799in}{0.387222in}}%
\pgfpathlineto{\pgfqpoint{5.612121in}{0.387222in}}%
\pgfpathlineto{\pgfqpoint{5.612121in}{0.818390in}}%
\pgfpathlineto{\pgfqpoint{4.982799in}{0.818390in}}%
\pgfpathlineto{\pgfqpoint{4.982799in}{0.387222in}}%
\pgfpathclose%
\pgfusepath{fill}%
\end{pgfscope}%
\begin{pgfscope}%
\pgfsetbuttcap%
\pgfsetroundjoin%
\definecolor{currentfill}{rgb}{0.000000,0.000000,0.000000}%
\pgfsetfillcolor{currentfill}%
\pgfsetlinewidth{0.803000pt}%
\definecolor{currentstroke}{rgb}{0.000000,0.000000,0.000000}%
\pgfsetstrokecolor{currentstroke}%
\pgfsetdash{}{0pt}%
\pgfsys@defobject{currentmarker}{\pgfqpoint{0.000000in}{-0.048611in}}{\pgfqpoint{0.000000in}{0.000000in}}{%
\pgfpathmoveto{\pgfqpoint{0.000000in}{0.000000in}}%
\pgfpathlineto{\pgfqpoint{0.000000in}{-0.048611in}}%
\pgfusepath{stroke,fill}%
}%
\begin{pgfscope}%
\pgfsys@transformshift{1.364197in}{0.387222in}%
\pgfsys@useobject{currentmarker}{}%
\end{pgfscope}%
\end{pgfscope}%
\begin{pgfscope}%
\definecolor{textcolor}{rgb}{0.000000,0.000000,0.000000}%
\pgfsetstrokecolor{textcolor}%
\pgfsetfillcolor{textcolor}%
\pgftext[x=1.364197in,y=0.290000in,,top]{\color{textcolor}{\sffamily\fontsize{10.000000}{12.000000}\selectfont\catcode`\^=\active\def^{\ifmmode\sp\else\^{}\fi}\catcode`\%=\active\def%{\%}CPU 128}}%
\end{pgfscope}%
\begin{pgfscope}%
\pgfsetbuttcap%
\pgfsetroundjoin%
\definecolor{currentfill}{rgb}{0.000000,0.000000,0.000000}%
\pgfsetfillcolor{currentfill}%
\pgfsetlinewidth{0.803000pt}%
\definecolor{currentstroke}{rgb}{0.000000,0.000000,0.000000}%
\pgfsetstrokecolor{currentstroke}%
\pgfsetdash{}{0pt}%
\pgfsys@defobject{currentmarker}{\pgfqpoint{0.000000in}{-0.048611in}}{\pgfqpoint{0.000000in}{0.000000in}}{%
\pgfpathmoveto{\pgfqpoint{0.000000in}{0.000000in}}%
\pgfpathlineto{\pgfqpoint{0.000000in}{-0.048611in}}%
\pgfusepath{stroke,fill}%
}%
\begin{pgfscope}%
\pgfsys@transformshift{2.150849in}{0.387222in}%
\pgfsys@useobject{currentmarker}{}%
\end{pgfscope}%
\end{pgfscope}%
\begin{pgfscope}%
\definecolor{textcolor}{rgb}{0.000000,0.000000,0.000000}%
\pgfsetstrokecolor{textcolor}%
\pgfsetfillcolor{textcolor}%
\pgftext[x=2.150849in,y=0.290000in,,top]{\color{textcolor}{\sffamily\fontsize{10.000000}{12.000000}\selectfont\catcode`\^=\active\def^{\ifmmode\sp\else\^{}\fi}\catcode`\%=\active\def%{\%}GPU 128}}%
\end{pgfscope}%
\begin{pgfscope}%
\pgfsetbuttcap%
\pgfsetroundjoin%
\definecolor{currentfill}{rgb}{0.000000,0.000000,0.000000}%
\pgfsetfillcolor{currentfill}%
\pgfsetlinewidth{0.803000pt}%
\definecolor{currentstroke}{rgb}{0.000000,0.000000,0.000000}%
\pgfsetstrokecolor{currentstroke}%
\pgfsetdash{}{0pt}%
\pgfsys@defobject{currentmarker}{\pgfqpoint{0.000000in}{-0.048611in}}{\pgfqpoint{0.000000in}{0.000000in}}{%
\pgfpathmoveto{\pgfqpoint{0.000000in}{0.000000in}}%
\pgfpathlineto{\pgfqpoint{0.000000in}{-0.048611in}}%
\pgfusepath{stroke,fill}%
}%
\begin{pgfscope}%
\pgfsys@transformshift{2.937502in}{0.387222in}%
\pgfsys@useobject{currentmarker}{}%
\end{pgfscope}%
\end{pgfscope}%
\begin{pgfscope}%
\definecolor{textcolor}{rgb}{0.000000,0.000000,0.000000}%
\pgfsetstrokecolor{textcolor}%
\pgfsetfillcolor{textcolor}%
\pgftext[x=2.937502in,y=0.290000in,,top]{\color{textcolor}{\sffamily\fontsize{10.000000}{12.000000}\selectfont\catcode`\^=\active\def^{\ifmmode\sp\else\^{}\fi}\catcode`\%=\active\def%{\%}CPU 1024}}%
\end{pgfscope}%
\begin{pgfscope}%
\pgfsetbuttcap%
\pgfsetroundjoin%
\definecolor{currentfill}{rgb}{0.000000,0.000000,0.000000}%
\pgfsetfillcolor{currentfill}%
\pgfsetlinewidth{0.803000pt}%
\definecolor{currentstroke}{rgb}{0.000000,0.000000,0.000000}%
\pgfsetstrokecolor{currentstroke}%
\pgfsetdash{}{0pt}%
\pgfsys@defobject{currentmarker}{\pgfqpoint{0.000000in}{-0.048611in}}{\pgfqpoint{0.000000in}{0.000000in}}{%
\pgfpathmoveto{\pgfqpoint{0.000000in}{0.000000in}}%
\pgfpathlineto{\pgfqpoint{0.000000in}{-0.048611in}}%
\pgfusepath{stroke,fill}%
}%
\begin{pgfscope}%
\pgfsys@transformshift{3.724155in}{0.387222in}%
\pgfsys@useobject{currentmarker}{}%
\end{pgfscope}%
\end{pgfscope}%
\begin{pgfscope}%
\definecolor{textcolor}{rgb}{0.000000,0.000000,0.000000}%
\pgfsetstrokecolor{textcolor}%
\pgfsetfillcolor{textcolor}%
\pgftext[x=3.724155in,y=0.290000in,,top]{\color{textcolor}{\sffamily\fontsize{10.000000}{12.000000}\selectfont\catcode`\^=\active\def^{\ifmmode\sp\else\^{}\fi}\catcode`\%=\active\def%{\%}GPU 1024}}%
\end{pgfscope}%
\begin{pgfscope}%
\pgfsetbuttcap%
\pgfsetroundjoin%
\definecolor{currentfill}{rgb}{0.000000,0.000000,0.000000}%
\pgfsetfillcolor{currentfill}%
\pgfsetlinewidth{0.803000pt}%
\definecolor{currentstroke}{rgb}{0.000000,0.000000,0.000000}%
\pgfsetstrokecolor{currentstroke}%
\pgfsetdash{}{0pt}%
\pgfsys@defobject{currentmarker}{\pgfqpoint{0.000000in}{-0.048611in}}{\pgfqpoint{0.000000in}{0.000000in}}{%
\pgfpathmoveto{\pgfqpoint{0.000000in}{0.000000in}}%
\pgfpathlineto{\pgfqpoint{0.000000in}{-0.048611in}}%
\pgfusepath{stroke,fill}%
}%
\begin{pgfscope}%
\pgfsys@transformshift{4.510807in}{0.387222in}%
\pgfsys@useobject{currentmarker}{}%
\end{pgfscope}%
\end{pgfscope}%
\begin{pgfscope}%
\definecolor{textcolor}{rgb}{0.000000,0.000000,0.000000}%
\pgfsetstrokecolor{textcolor}%
\pgfsetfillcolor{textcolor}%
\pgftext[x=4.510807in,y=0.290000in,,top]{\color{textcolor}{\sffamily\fontsize{10.000000}{12.000000}\selectfont\catcode`\^=\active\def^{\ifmmode\sp\else\^{}\fi}\catcode`\%=\active\def%{\%}CPU 65536}}%
\end{pgfscope}%
\begin{pgfscope}%
\pgfsetbuttcap%
\pgfsetroundjoin%
\definecolor{currentfill}{rgb}{0.000000,0.000000,0.000000}%
\pgfsetfillcolor{currentfill}%
\pgfsetlinewidth{0.803000pt}%
\definecolor{currentstroke}{rgb}{0.000000,0.000000,0.000000}%
\pgfsetstrokecolor{currentstroke}%
\pgfsetdash{}{0pt}%
\pgfsys@defobject{currentmarker}{\pgfqpoint{0.000000in}{-0.048611in}}{\pgfqpoint{0.000000in}{0.000000in}}{%
\pgfpathmoveto{\pgfqpoint{0.000000in}{0.000000in}}%
\pgfpathlineto{\pgfqpoint{0.000000in}{-0.048611in}}%
\pgfusepath{stroke,fill}%
}%
\begin{pgfscope}%
\pgfsys@transformshift{5.297460in}{0.387222in}%
\pgfsys@useobject{currentmarker}{}%
\end{pgfscope}%
\end{pgfscope}%
\begin{pgfscope}%
\definecolor{textcolor}{rgb}{0.000000,0.000000,0.000000}%
\pgfsetstrokecolor{textcolor}%
\pgfsetfillcolor{textcolor}%
\pgftext[x=5.297460in,y=0.290000in,,top]{\color{textcolor}{\sffamily\fontsize{10.000000}{12.000000}\selectfont\catcode`\^=\active\def^{\ifmmode\sp\else\^{}\fi}\catcode`\%=\active\def%{\%}GPU 65536}}%
\end{pgfscope}%
\begin{pgfscope}%
\pgfsetbuttcap%
\pgfsetroundjoin%
\definecolor{currentfill}{rgb}{0.000000,0.000000,0.000000}%
\pgfsetfillcolor{currentfill}%
\pgfsetlinewidth{0.803000pt}%
\definecolor{currentstroke}{rgb}{0.000000,0.000000,0.000000}%
\pgfsetstrokecolor{currentstroke}%
\pgfsetdash{}{0pt}%
\pgfsys@defobject{currentmarker}{\pgfqpoint{-0.048611in}{0.000000in}}{\pgfqpoint{-0.000000in}{0.000000in}}{%
\pgfpathmoveto{\pgfqpoint{-0.000000in}{0.000000in}}%
\pgfpathlineto{\pgfqpoint{-0.048611in}{0.000000in}}%
\pgfusepath{stroke,fill}%
}%
\begin{pgfscope}%
\pgfsys@transformshift{0.821406in}{0.387222in}%
\pgfsys@useobject{currentmarker}{}%
\end{pgfscope}%
\end{pgfscope}%
\begin{pgfscope}%
\definecolor{textcolor}{rgb}{0.000000,0.000000,0.000000}%
\pgfsetstrokecolor{textcolor}%
\pgfsetfillcolor{textcolor}%
\pgftext[x=0.635819in, y=0.334461in, left, base]{\color{textcolor}{\sffamily\fontsize{10.000000}{12.000000}\selectfont\catcode`\^=\active\def^{\ifmmode\sp\else\^{}\fi}\catcode`\%=\active\def%{\%}0}}%
\end{pgfscope}%
\begin{pgfscope}%
\pgfsetbuttcap%
\pgfsetroundjoin%
\definecolor{currentfill}{rgb}{0.000000,0.000000,0.000000}%
\pgfsetfillcolor{currentfill}%
\pgfsetlinewidth{0.803000pt}%
\definecolor{currentstroke}{rgb}{0.000000,0.000000,0.000000}%
\pgfsetstrokecolor{currentstroke}%
\pgfsetdash{}{0pt}%
\pgfsys@defobject{currentmarker}{\pgfqpoint{-0.048611in}{0.000000in}}{\pgfqpoint{-0.000000in}{0.000000in}}{%
\pgfpathmoveto{\pgfqpoint{-0.000000in}{0.000000in}}%
\pgfpathlineto{\pgfqpoint{-0.048611in}{0.000000in}}%
\pgfusepath{stroke,fill}%
}%
\begin{pgfscope}%
\pgfsys@transformshift{0.821406in}{0.828232in}%
\pgfsys@useobject{currentmarker}{}%
\end{pgfscope}%
\end{pgfscope}%
\begin{pgfscope}%
\definecolor{textcolor}{rgb}{0.000000,0.000000,0.000000}%
\pgfsetstrokecolor{textcolor}%
\pgfsetfillcolor{textcolor}%
\pgftext[x=0.459088in, y=0.775471in, left, base]{\color{textcolor}{\sffamily\fontsize{10.000000}{12.000000}\selectfont\catcode`\^=\active\def^{\ifmmode\sp\else\^{}\fi}\catcode`\%=\active\def%{\%}200}}%
\end{pgfscope}%
\begin{pgfscope}%
\pgfsetbuttcap%
\pgfsetroundjoin%
\definecolor{currentfill}{rgb}{0.000000,0.000000,0.000000}%
\pgfsetfillcolor{currentfill}%
\pgfsetlinewidth{0.803000pt}%
\definecolor{currentstroke}{rgb}{0.000000,0.000000,0.000000}%
\pgfsetstrokecolor{currentstroke}%
\pgfsetdash{}{0pt}%
\pgfsys@defobject{currentmarker}{\pgfqpoint{-0.048611in}{0.000000in}}{\pgfqpoint{-0.000000in}{0.000000in}}{%
\pgfpathmoveto{\pgfqpoint{-0.000000in}{0.000000in}}%
\pgfpathlineto{\pgfqpoint{-0.048611in}{0.000000in}}%
\pgfusepath{stroke,fill}%
}%
\begin{pgfscope}%
\pgfsys@transformshift{0.821406in}{1.269243in}%
\pgfsys@useobject{currentmarker}{}%
\end{pgfscope}%
\end{pgfscope}%
\begin{pgfscope}%
\definecolor{textcolor}{rgb}{0.000000,0.000000,0.000000}%
\pgfsetstrokecolor{textcolor}%
\pgfsetfillcolor{textcolor}%
\pgftext[x=0.459088in, y=1.216481in, left, base]{\color{textcolor}{\sffamily\fontsize{10.000000}{12.000000}\selectfont\catcode`\^=\active\def^{\ifmmode\sp\else\^{}\fi}\catcode`\%=\active\def%{\%}400}}%
\end{pgfscope}%
\begin{pgfscope}%
\pgfsetbuttcap%
\pgfsetroundjoin%
\definecolor{currentfill}{rgb}{0.000000,0.000000,0.000000}%
\pgfsetfillcolor{currentfill}%
\pgfsetlinewidth{0.803000pt}%
\definecolor{currentstroke}{rgb}{0.000000,0.000000,0.000000}%
\pgfsetstrokecolor{currentstroke}%
\pgfsetdash{}{0pt}%
\pgfsys@defobject{currentmarker}{\pgfqpoint{-0.048611in}{0.000000in}}{\pgfqpoint{-0.000000in}{0.000000in}}{%
\pgfpathmoveto{\pgfqpoint{-0.000000in}{0.000000in}}%
\pgfpathlineto{\pgfqpoint{-0.048611in}{0.000000in}}%
\pgfusepath{stroke,fill}%
}%
\begin{pgfscope}%
\pgfsys@transformshift{0.821406in}{1.710253in}%
\pgfsys@useobject{currentmarker}{}%
\end{pgfscope}%
\end{pgfscope}%
\begin{pgfscope}%
\definecolor{textcolor}{rgb}{0.000000,0.000000,0.000000}%
\pgfsetstrokecolor{textcolor}%
\pgfsetfillcolor{textcolor}%
\pgftext[x=0.459088in, y=1.657491in, left, base]{\color{textcolor}{\sffamily\fontsize{10.000000}{12.000000}\selectfont\catcode`\^=\active\def^{\ifmmode\sp\else\^{}\fi}\catcode`\%=\active\def%{\%}600}}%
\end{pgfscope}%
\begin{pgfscope}%
\pgfsetbuttcap%
\pgfsetroundjoin%
\definecolor{currentfill}{rgb}{0.000000,0.000000,0.000000}%
\pgfsetfillcolor{currentfill}%
\pgfsetlinewidth{0.803000pt}%
\definecolor{currentstroke}{rgb}{0.000000,0.000000,0.000000}%
\pgfsetstrokecolor{currentstroke}%
\pgfsetdash{}{0pt}%
\pgfsys@defobject{currentmarker}{\pgfqpoint{-0.048611in}{0.000000in}}{\pgfqpoint{-0.000000in}{0.000000in}}{%
\pgfpathmoveto{\pgfqpoint{-0.000000in}{0.000000in}}%
\pgfpathlineto{\pgfqpoint{-0.048611in}{0.000000in}}%
\pgfusepath{stroke,fill}%
}%
\begin{pgfscope}%
\pgfsys@transformshift{0.821406in}{2.151263in}%
\pgfsys@useobject{currentmarker}{}%
\end{pgfscope}%
\end{pgfscope}%
\begin{pgfscope}%
\definecolor{textcolor}{rgb}{0.000000,0.000000,0.000000}%
\pgfsetstrokecolor{textcolor}%
\pgfsetfillcolor{textcolor}%
\pgftext[x=0.459088in, y=2.098502in, left, base]{\color{textcolor}{\sffamily\fontsize{10.000000}{12.000000}\selectfont\catcode`\^=\active\def^{\ifmmode\sp\else\^{}\fi}\catcode`\%=\active\def%{\%}800}}%
\end{pgfscope}%
\begin{pgfscope}%
\pgfsetbuttcap%
\pgfsetroundjoin%
\definecolor{currentfill}{rgb}{0.000000,0.000000,0.000000}%
\pgfsetfillcolor{currentfill}%
\pgfsetlinewidth{0.803000pt}%
\definecolor{currentstroke}{rgb}{0.000000,0.000000,0.000000}%
\pgfsetstrokecolor{currentstroke}%
\pgfsetdash{}{0pt}%
\pgfsys@defobject{currentmarker}{\pgfqpoint{-0.048611in}{0.000000in}}{\pgfqpoint{-0.000000in}{0.000000in}}{%
\pgfpathmoveto{\pgfqpoint{-0.000000in}{0.000000in}}%
\pgfpathlineto{\pgfqpoint{-0.048611in}{0.000000in}}%
\pgfusepath{stroke,fill}%
}%
\begin{pgfscope}%
\pgfsys@transformshift{0.821406in}{2.592273in}%
\pgfsys@useobject{currentmarker}{}%
\end{pgfscope}%
\end{pgfscope}%
\begin{pgfscope}%
\definecolor{textcolor}{rgb}{0.000000,0.000000,0.000000}%
\pgfsetstrokecolor{textcolor}%
\pgfsetfillcolor{textcolor}%
\pgftext[x=0.370723in, y=2.539512in, left, base]{\color{textcolor}{\sffamily\fontsize{10.000000}{12.000000}\selectfont\catcode`\^=\active\def^{\ifmmode\sp\else\^{}\fi}\catcode`\%=\active\def%{\%}1000}}%
\end{pgfscope}%
\begin{pgfscope}%
\definecolor{textcolor}{rgb}{0.000000,0.000000,0.000000}%
\pgfsetstrokecolor{textcolor}%
\pgfsetfillcolor{textcolor}%
\pgftext[x=0.315167in,y=1.618611in,,bottom,rotate=90.000000]{\color{textcolor}{\sffamily\fontsize{10.000000}{12.000000}\selectfont\catcode`\^=\active\def^{\ifmmode\sp\else\^{}\fi}\catcode`\%=\active\def%{\%}Średni czas wykonania (s)}}%
\end{pgfscope}%
\begin{pgfscope}%
\pgfsetrectcap%
\pgfsetmiterjoin%
\pgfsetlinewidth{0.803000pt}%
\definecolor{currentstroke}{rgb}{0.000000,0.000000,0.000000}%
\pgfsetstrokecolor{currentstroke}%
\pgfsetdash{}{0pt}%
\pgfpathmoveto{\pgfqpoint{0.821406in}{0.387222in}}%
\pgfpathlineto{\pgfqpoint{0.821406in}{2.850000in}}%
\pgfusepath{stroke}%
\end{pgfscope}%
\begin{pgfscope}%
\pgfsetrectcap%
\pgfsetmiterjoin%
\pgfsetlinewidth{0.803000pt}%
\definecolor{currentstroke}{rgb}{0.000000,0.000000,0.000000}%
\pgfsetstrokecolor{currentstroke}%
\pgfsetdash{}{0pt}%
\pgfpathmoveto{\pgfqpoint{5.840250in}{0.387222in}}%
\pgfpathlineto{\pgfqpoint{5.840250in}{2.850000in}}%
\pgfusepath{stroke}%
\end{pgfscope}%
\begin{pgfscope}%
\pgfsetrectcap%
\pgfsetmiterjoin%
\pgfsetlinewidth{0.803000pt}%
\definecolor{currentstroke}{rgb}{0.000000,0.000000,0.000000}%
\pgfsetstrokecolor{currentstroke}%
\pgfsetdash{}{0pt}%
\pgfpathmoveto{\pgfqpoint{0.821406in}{0.387222in}}%
\pgfpathlineto{\pgfqpoint{5.840250in}{0.387222in}}%
\pgfusepath{stroke}%
\end{pgfscope}%
\begin{pgfscope}%
\pgfsetrectcap%
\pgfsetmiterjoin%
\pgfsetlinewidth{0.803000pt}%
\definecolor{currentstroke}{rgb}{0.000000,0.000000,0.000000}%
\pgfsetstrokecolor{currentstroke}%
\pgfsetdash{}{0pt}%
\pgfpathmoveto{\pgfqpoint{0.821406in}{2.850000in}}%
\pgfpathlineto{\pgfqpoint{5.840250in}{2.850000in}}%
\pgfusepath{stroke}%
\end{pgfscope}%
\begin{pgfscope}%
\definecolor{textcolor}{rgb}{0.000000,0.000000,0.000000}%
\pgfsetstrokecolor{textcolor}%
\pgfsetfillcolor{textcolor}%
\pgftext[x=1.364197in,y=1.059370in,,bottom]{\color{textcolor}{\sffamily\fontsize{10.000000}{12.000000}\selectfont\catcode`\^=\active\def^{\ifmmode\sp\else\^{}\fi}\catcode`\%=\active\def%{\%}304.822}}%
\end{pgfscope}%
\begin{pgfscope}%
\definecolor{textcolor}{rgb}{0.000000,0.000000,0.000000}%
\pgfsetstrokecolor{textcolor}%
\pgfsetfillcolor{textcolor}%
\pgftext[x=2.150849in,y=2.626111in,,bottom]{\color{textcolor}{\sffamily\fontsize{10.000000}{12.000000}\selectfont\catcode`\^=\active\def^{\ifmmode\sp\else\^{}\fi}\catcode`\%=\active\def%{\%}1015.346}}%
\end{pgfscope}%
\begin{pgfscope}%
\definecolor{textcolor}{rgb}{0.000000,0.000000,0.000000}%
\pgfsetstrokecolor{textcolor}%
\pgfsetfillcolor{textcolor}%
\pgftext[x=2.937502in,y=1.042967in,,bottom]{\color{textcolor}{\sffamily\fontsize{10.000000}{12.000000}\selectfont\catcode`\^=\active\def^{\ifmmode\sp\else\^{}\fi}\catcode`\%=\active\def%{\%}297.383}}%
\end{pgfscope}%
\begin{pgfscope}%
\definecolor{textcolor}{rgb}{0.000000,0.000000,0.000000}%
\pgfsetstrokecolor{textcolor}%
\pgfsetfillcolor{textcolor}%
\pgftext[x=3.724155in,y=1.037193in,,bottom]{\color{textcolor}{\sffamily\fontsize{10.000000}{12.000000}\selectfont\catcode`\^=\active\def^{\ifmmode\sp\else\^{}\fi}\catcode`\%=\active\def%{\%}294.765}}%
\end{pgfscope}%
\begin{pgfscope}%
\definecolor{textcolor}{rgb}{0.000000,0.000000,0.000000}%
\pgfsetstrokecolor{textcolor}%
\pgfsetfillcolor{textcolor}%
\pgftext[x=4.510807in,y=1.060866in,,bottom]{\color{textcolor}{\sffamily\fontsize{10.000000}{12.000000}\selectfont\catcode`\^=\active\def^{\ifmmode\sp\else\^{}\fi}\catcode`\%=\active\def%{\%}305.5}}%
\end{pgfscope}%
\begin{pgfscope}%
\definecolor{textcolor}{rgb}{0.000000,0.000000,0.000000}%
\pgfsetstrokecolor{textcolor}%
\pgfsetfillcolor{textcolor}%
\pgftext[x=5.297460in,y=0.818390in,,bottom]{\color{textcolor}{\sffamily\fontsize{10.000000}{12.000000}\selectfont\catcode`\^=\active\def^{\ifmmode\sp\else\^{}\fi}\catcode`\%=\active\def%{\%}195.537}}%
\end{pgfscope}%
\end{pgfpicture}%
\makeatother%
\endgroup%
}
    \caption{Średni czas trwania wszelkich operacji w trybie offline}
    \label{fig:Średni czas trwania wszelkich operacji w trybie offline}
\end{figure}

Rozmiar bufora miał bardzo duży wpływ na czas trwania wszelkich operacji w przypadku implementacji GPU. Jest to potwierdzeniem zaleceń przedstawionych w dokumentacji CUDA, głoszących, iż należy unikać wielokrotnych transferów danych pomiędzy GPU a pamięcią RAM, a zamiast tego przesyłać większe porcje danych w jednym transferze. Dla bufora o rozmiarze 128 próbek czas ponad dwukrotnie przekroczył czas trwania utworu, co wskazuje na to, że dla małych buforów, przetwarzanie w czasie rzeczywistym jest niemożliwe z wykorzystaniem przedstawionej w tej pracy implementacji. W związku z tym testy wydajności przetwarzania w czasie rzeczywistym nie zostanie przeprowadzony dla wielkości bufora 128 próbek.

\section{Wyniki testów wydajności przetwarzania online}
Wyniki testów wydajności przetwarzania online zostały podzielone na sekcje: uśrednione statystyki, obciążenie systemu, przebieg trwania obliczeń dla bufora o rozmiarze 1024.

\begin{figure}[H]
    \centering
    \scalebox{1.0}{%% Creator: Matplotlib, PGF backend
%%
%% To include the figure in your LaTeX document, write
%%   \input{<filename>.pgf}
%%
%% Make sure the required packages are loaded in your preamble
%%   \usepackage{pgf}
%%
%% Also ensure that all the required font packages are loaded; for instance,
%% the lmodern package is sometimes necessary when using math font.
%%   \usepackage{lmodern}
%%
%% Figures using additional raster images can only be included by \input if
%% they are in the same directory as the main LaTeX file. For loading figures
%% from other directories you can use the `import` package
%%   \usepackage{import}
%%
%% and then include the figures with
%%   \import{<path to file>}{<filename>.pgf}
%%
%% Matplotlib used the following preamble
%%   \def\mathdefault#1{#1}
%%   \everymath=\expandafter{\the\everymath\displaystyle}
%%   
%%   \usepackage{fontspec}
%%   \setmainfont{DejaVuSerif.ttf}[Path=\detokenize{/usr/lib/python3.12/site-packages/matplotlib/mpl-data/fonts/ttf/}]
%%   \setsansfont{DejaVuSans.ttf}[Path=\detokenize{/usr/lib/python3.12/site-packages/matplotlib/mpl-data/fonts/ttf/}]
%%   \setmonofont{DejaVuSansMono.ttf}[Path=\detokenize{/usr/lib/python3.12/site-packages/matplotlib/mpl-data/fonts/ttf/}]
%%   \makeatletter\@ifpackageloaded{underscore}{}{\usepackage[strings]{underscore}}\makeatother
%%
\begingroup%
\makeatletter%
\begin{pgfpicture}%
\pgfpathrectangle{\pgfpointorigin}{\pgfqpoint{5.990000in}{6.000000in}}%
\pgfusepath{use as bounding box, clip}%
\begin{pgfscope}%
\pgfsetbuttcap%
\pgfsetmiterjoin%
\definecolor{currentfill}{rgb}{1.000000,1.000000,1.000000}%
\pgfsetfillcolor{currentfill}%
\pgfsetlinewidth{0.000000pt}%
\definecolor{currentstroke}{rgb}{1.000000,1.000000,1.000000}%
\pgfsetstrokecolor{currentstroke}%
\pgfsetdash{}{0pt}%
\pgfpathmoveto{\pgfqpoint{0.000000in}{0.000000in}}%
\pgfpathlineto{\pgfqpoint{5.990000in}{0.000000in}}%
\pgfpathlineto{\pgfqpoint{5.990000in}{6.000000in}}%
\pgfpathlineto{\pgfqpoint{0.000000in}{6.000000in}}%
\pgfpathlineto{\pgfqpoint{0.000000in}{0.000000in}}%
\pgfpathclose%
\pgfusepath{fill}%
\end{pgfscope}%
\begin{pgfscope}%
\pgfsetbuttcap%
\pgfsetmiterjoin%
\definecolor{currentfill}{rgb}{1.000000,1.000000,1.000000}%
\pgfsetfillcolor{currentfill}%
\pgfsetlinewidth{0.000000pt}%
\definecolor{currentstroke}{rgb}{0.000000,0.000000,0.000000}%
\pgfsetstrokecolor{currentstroke}%
\pgfsetstrokeopacity{0.000000}%
\pgfsetdash{}{0pt}%
\pgfpathmoveto{\pgfqpoint{0.956181in}{4.169889in}}%
\pgfpathlineto{\pgfqpoint{3.009004in}{4.169889in}}%
\pgfpathlineto{\pgfqpoint{3.009004in}{5.460667in}}%
\pgfpathlineto{\pgfqpoint{0.956181in}{5.460667in}}%
\pgfpathlineto{\pgfqpoint{0.956181in}{4.169889in}}%
\pgfpathclose%
\pgfusepath{fill}%
\end{pgfscope}%
\begin{pgfscope}%
\pgfpathrectangle{\pgfqpoint{0.956181in}{4.169889in}}{\pgfqpoint{2.052823in}{1.290778in}}%
\pgfusepath{clip}%
\pgfsetbuttcap%
\pgfsetmiterjoin%
\definecolor{currentfill}{rgb}{1.000000,0.000000,0.000000}%
\pgfsetfillcolor{currentfill}%
\pgfsetlinewidth{0.000000pt}%
\definecolor{currentstroke}{rgb}{0.000000,0.000000,0.000000}%
\pgfsetstrokecolor{currentstroke}%
\pgfsetstrokeopacity{0.000000}%
\pgfsetdash{}{0pt}%
\pgfpathmoveto{\pgfqpoint{1.049492in}{4.169889in}}%
\pgfpathlineto{\pgfqpoint{1.878915in}{4.169889in}}%
\pgfpathlineto{\pgfqpoint{1.878915in}{5.245537in}}%
\pgfpathlineto{\pgfqpoint{1.049492in}{5.245537in}}%
\pgfpathlineto{\pgfqpoint{1.049492in}{4.169889in}}%
\pgfpathclose%
\pgfusepath{fill}%
\end{pgfscope}%
\begin{pgfscope}%
\pgfpathrectangle{\pgfqpoint{0.956181in}{4.169889in}}{\pgfqpoint{2.052823in}{1.290778in}}%
\pgfusepath{clip}%
\pgfsetbuttcap%
\pgfsetmiterjoin%
\definecolor{currentfill}{rgb}{0.000000,0.000000,1.000000}%
\pgfsetfillcolor{currentfill}%
\pgfsetlinewidth{0.000000pt}%
\definecolor{currentstroke}{rgb}{0.000000,0.000000,0.000000}%
\pgfsetstrokecolor{currentstroke}%
\pgfsetstrokeopacity{0.000000}%
\pgfsetdash{}{0pt}%
\pgfpathmoveto{\pgfqpoint{2.086271in}{4.169889in}}%
\pgfpathlineto{\pgfqpoint{2.915694in}{4.169889in}}%
\pgfpathlineto{\pgfqpoint{2.915694in}{5.245537in}}%
\pgfpathlineto{\pgfqpoint{2.086271in}{5.245537in}}%
\pgfpathlineto{\pgfqpoint{2.086271in}{4.169889in}}%
\pgfpathclose%
\pgfusepath{fill}%
\end{pgfscope}%
\begin{pgfscope}%
\pgfsetbuttcap%
\pgfsetroundjoin%
\definecolor{currentfill}{rgb}{0.000000,0.000000,0.000000}%
\pgfsetfillcolor{currentfill}%
\pgfsetlinewidth{0.803000pt}%
\definecolor{currentstroke}{rgb}{0.000000,0.000000,0.000000}%
\pgfsetstrokecolor{currentstroke}%
\pgfsetdash{}{0pt}%
\pgfsys@defobject{currentmarker}{\pgfqpoint{0.000000in}{-0.048611in}}{\pgfqpoint{0.000000in}{0.000000in}}{%
\pgfpathmoveto{\pgfqpoint{0.000000in}{0.000000in}}%
\pgfpathlineto{\pgfqpoint{0.000000in}{-0.048611in}}%
\pgfusepath{stroke,fill}%
}%
\begin{pgfscope}%
\pgfsys@transformshift{1.464203in}{4.169889in}%
\pgfsys@useobject{currentmarker}{}%
\end{pgfscope}%
\end{pgfscope}%
\begin{pgfscope}%
\definecolor{textcolor}{rgb}{0.000000,0.000000,0.000000}%
\pgfsetstrokecolor{textcolor}%
\pgfsetfillcolor{textcolor}%
\pgftext[x=1.464203in,y=4.072667in,,top]{\color{textcolor}{\sffamily\fontsize{10.000000}{12.000000}\selectfont\catcode`\^=\active\def^{\ifmmode\sp\else\^{}\fi}\catcode`\%=\active\def%{\%}CPU 1024}}%
\end{pgfscope}%
\begin{pgfscope}%
\pgfsetbuttcap%
\pgfsetroundjoin%
\definecolor{currentfill}{rgb}{0.000000,0.000000,0.000000}%
\pgfsetfillcolor{currentfill}%
\pgfsetlinewidth{0.803000pt}%
\definecolor{currentstroke}{rgb}{0.000000,0.000000,0.000000}%
\pgfsetstrokecolor{currentstroke}%
\pgfsetdash{}{0pt}%
\pgfsys@defobject{currentmarker}{\pgfqpoint{0.000000in}{-0.048611in}}{\pgfqpoint{0.000000in}{0.000000in}}{%
\pgfpathmoveto{\pgfqpoint{0.000000in}{0.000000in}}%
\pgfpathlineto{\pgfqpoint{0.000000in}{-0.048611in}}%
\pgfusepath{stroke,fill}%
}%
\begin{pgfscope}%
\pgfsys@transformshift{2.500983in}{4.169889in}%
\pgfsys@useobject{currentmarker}{}%
\end{pgfscope}%
\end{pgfscope}%
\begin{pgfscope}%
\definecolor{textcolor}{rgb}{0.000000,0.000000,0.000000}%
\pgfsetstrokecolor{textcolor}%
\pgfsetfillcolor{textcolor}%
\pgftext[x=2.500983in,y=4.072667in,,top]{\color{textcolor}{\sffamily\fontsize{10.000000}{12.000000}\selectfont\catcode`\^=\active\def^{\ifmmode\sp\else\^{}\fi}\catcode`\%=\active\def%{\%}GPU 1024}}%
\end{pgfscope}%
\begin{pgfscope}%
\pgfsetbuttcap%
\pgfsetroundjoin%
\definecolor{currentfill}{rgb}{0.000000,0.000000,0.000000}%
\pgfsetfillcolor{currentfill}%
\pgfsetlinewidth{0.803000pt}%
\definecolor{currentstroke}{rgb}{0.000000,0.000000,0.000000}%
\pgfsetstrokecolor{currentstroke}%
\pgfsetdash{}{0pt}%
\pgfsys@defobject{currentmarker}{\pgfqpoint{-0.048611in}{0.000000in}}{\pgfqpoint{-0.000000in}{0.000000in}}{%
\pgfpathmoveto{\pgfqpoint{-0.000000in}{0.000000in}}%
\pgfpathlineto{\pgfqpoint{-0.048611in}{0.000000in}}%
\pgfusepath{stroke,fill}%
}%
\begin{pgfscope}%
\pgfsys@transformshift{0.956181in}{4.169889in}%
\pgfsys@useobject{currentmarker}{}%
\end{pgfscope}%
\end{pgfscope}%
\begin{pgfscope}%
\definecolor{textcolor}{rgb}{0.000000,0.000000,0.000000}%
\pgfsetstrokecolor{textcolor}%
\pgfsetfillcolor{textcolor}%
\pgftext[x=0.770594in, y=4.117127in, left, base]{\color{textcolor}{\sffamily\fontsize{10.000000}{12.000000}\selectfont\catcode`\^=\active\def^{\ifmmode\sp\else\^{}\fi}\catcode`\%=\active\def%{\%}0}}%
\end{pgfscope}%
\begin{pgfscope}%
\pgfsetbuttcap%
\pgfsetroundjoin%
\definecolor{currentfill}{rgb}{0.000000,0.000000,0.000000}%
\pgfsetfillcolor{currentfill}%
\pgfsetlinewidth{0.803000pt}%
\definecolor{currentstroke}{rgb}{0.000000,0.000000,0.000000}%
\pgfsetstrokecolor{currentstroke}%
\pgfsetdash{}{0pt}%
\pgfsys@defobject{currentmarker}{\pgfqpoint{-0.048611in}{0.000000in}}{\pgfqpoint{-0.000000in}{0.000000in}}{%
\pgfpathmoveto{\pgfqpoint{-0.000000in}{0.000000in}}%
\pgfpathlineto{\pgfqpoint{-0.048611in}{0.000000in}}%
\pgfusepath{stroke,fill}%
}%
\begin{pgfscope}%
\pgfsys@transformshift{0.956181in}{4.674107in}%
\pgfsys@useobject{currentmarker}{}%
\end{pgfscope}%
\end{pgfscope}%
\begin{pgfscope}%
\definecolor{textcolor}{rgb}{0.000000,0.000000,0.000000}%
\pgfsetstrokecolor{textcolor}%
\pgfsetfillcolor{textcolor}%
\pgftext[x=0.417133in, y=4.621345in, left, base]{\color{textcolor}{\sffamily\fontsize{10.000000}{12.000000}\selectfont\catcode`\^=\active\def^{\ifmmode\sp\else\^{}\fi}\catcode`\%=\active\def%{\%}10000}}%
\end{pgfscope}%
\begin{pgfscope}%
\pgfsetbuttcap%
\pgfsetroundjoin%
\definecolor{currentfill}{rgb}{0.000000,0.000000,0.000000}%
\pgfsetfillcolor{currentfill}%
\pgfsetlinewidth{0.803000pt}%
\definecolor{currentstroke}{rgb}{0.000000,0.000000,0.000000}%
\pgfsetstrokecolor{currentstroke}%
\pgfsetdash{}{0pt}%
\pgfsys@defobject{currentmarker}{\pgfqpoint{-0.048611in}{0.000000in}}{\pgfqpoint{-0.000000in}{0.000000in}}{%
\pgfpathmoveto{\pgfqpoint{-0.000000in}{0.000000in}}%
\pgfpathlineto{\pgfqpoint{-0.048611in}{0.000000in}}%
\pgfusepath{stroke,fill}%
}%
\begin{pgfscope}%
\pgfsys@transformshift{0.956181in}{5.178325in}%
\pgfsys@useobject{currentmarker}{}%
\end{pgfscope}%
\end{pgfscope}%
\begin{pgfscope}%
\definecolor{textcolor}{rgb}{0.000000,0.000000,0.000000}%
\pgfsetstrokecolor{textcolor}%
\pgfsetfillcolor{textcolor}%
\pgftext[x=0.417133in, y=5.125563in, left, base]{\color{textcolor}{\sffamily\fontsize{10.000000}{12.000000}\selectfont\catcode`\^=\active\def^{\ifmmode\sp\else\^{}\fi}\catcode`\%=\active\def%{\%}20000}}%
\end{pgfscope}%
\begin{pgfscope}%
\definecolor{textcolor}{rgb}{0.000000,0.000000,0.000000}%
\pgfsetstrokecolor{textcolor}%
\pgfsetfillcolor{textcolor}%
\pgftext[x=0.361577in,y=4.815278in,,bottom,rotate=90.000000]{\color{textcolor}{\sffamily\fontsize{10.000000}{12.000000}\selectfont\catcode`\^=\active\def^{\ifmmode\sp\else\^{}\fi}\catcode`\%=\active\def%{\%}czas (ns)}}%
\end{pgfscope}%
\begin{pgfscope}%
\pgfsetrectcap%
\pgfsetmiterjoin%
\pgfsetlinewidth{0.803000pt}%
\definecolor{currentstroke}{rgb}{0.000000,0.000000,0.000000}%
\pgfsetstrokecolor{currentstroke}%
\pgfsetdash{}{0pt}%
\pgfpathmoveto{\pgfqpoint{0.956181in}{4.169889in}}%
\pgfpathlineto{\pgfqpoint{0.956181in}{5.460667in}}%
\pgfusepath{stroke}%
\end{pgfscope}%
\begin{pgfscope}%
\pgfsetrectcap%
\pgfsetmiterjoin%
\pgfsetlinewidth{0.803000pt}%
\definecolor{currentstroke}{rgb}{0.000000,0.000000,0.000000}%
\pgfsetstrokecolor{currentstroke}%
\pgfsetdash{}{0pt}%
\pgfpathmoveto{\pgfqpoint{3.009004in}{4.169889in}}%
\pgfpathlineto{\pgfqpoint{3.009004in}{5.460667in}}%
\pgfusepath{stroke}%
\end{pgfscope}%
\begin{pgfscope}%
\pgfsetrectcap%
\pgfsetmiterjoin%
\pgfsetlinewidth{0.803000pt}%
\definecolor{currentstroke}{rgb}{0.000000,0.000000,0.000000}%
\pgfsetstrokecolor{currentstroke}%
\pgfsetdash{}{0pt}%
\pgfpathmoveto{\pgfqpoint{0.956181in}{4.169889in}}%
\pgfpathlineto{\pgfqpoint{3.009004in}{4.169889in}}%
\pgfusepath{stroke}%
\end{pgfscope}%
\begin{pgfscope}%
\pgfsetrectcap%
\pgfsetmiterjoin%
\pgfsetlinewidth{0.803000pt}%
\definecolor{currentstroke}{rgb}{0.000000,0.000000,0.000000}%
\pgfsetstrokecolor{currentstroke}%
\pgfsetdash{}{0pt}%
\pgfpathmoveto{\pgfqpoint{0.956181in}{5.460667in}}%
\pgfpathlineto{\pgfqpoint{3.009004in}{5.460667in}}%
\pgfusepath{stroke}%
\end{pgfscope}%
\begin{pgfscope}%
\definecolor{textcolor}{rgb}{0.000000,0.000000,0.000000}%
\pgfsetstrokecolor{textcolor}%
\pgfsetfillcolor{textcolor}%
\pgftext[x=1.464203in,y=5.245537in,,bottom]{\color{textcolor}{\sffamily\fontsize{10.000000}{12.000000}\selectfont\catcode`\^=\active\def^{\ifmmode\sp\else\^{}\fi}\catcode`\%=\active\def%{\%}21333.0}}%
\end{pgfscope}%
\begin{pgfscope}%
\definecolor{textcolor}{rgb}{0.000000,0.000000,0.000000}%
\pgfsetstrokecolor{textcolor}%
\pgfsetfillcolor{textcolor}%
\pgftext[x=2.500983in,y=5.245537in,,bottom]{\color{textcolor}{\sffamily\fontsize{10.000000}{12.000000}\selectfont\catcode`\^=\active\def^{\ifmmode\sp\else\^{}\fi}\catcode`\%=\active\def%{\%}21333.0}}%
\end{pgfscope}%
\begin{pgfscope}%
\definecolor{textcolor}{rgb}{0.000000,0.000000,0.000000}%
\pgfsetstrokecolor{textcolor}%
\pgfsetfillcolor{textcolor}%
\pgftext[x=1.679004in, y=5.739312in, left, base]{\color{textcolor}{\sffamily\fontsize{12.000000}{14.400000}\selectfont\catcode`\^=\active\def^{\ifmmode\sp\else\^{}\fi}\catcode`\%=\active\def%{\%}(1024) }}%
\end{pgfscope}%
\begin{pgfscope}%
\definecolor{textcolor}{rgb}{0.000000,0.000000,0.000000}%
\pgfsetstrokecolor{textcolor}%
\pgfsetfillcolor{textcolor}%
\pgftext[x=0.966764in, y=5.544000in, left, base]{\color{textcolor}{\sffamily\fontsize{12.000000}{14.400000}\selectfont\catcode`\^=\active\def^{\ifmmode\sp\else\^{}\fi}\catcode`\%=\active\def%{\%}średni czas trwania pętli}}%
\end{pgfscope}%
\begin{pgfscope}%
\pgfsetbuttcap%
\pgfsetmiterjoin%
\definecolor{currentfill}{rgb}{1.000000,1.000000,1.000000}%
\pgfsetfillcolor{currentfill}%
\pgfsetlinewidth{0.000000pt}%
\definecolor{currentstroke}{rgb}{0.000000,0.000000,0.000000}%
\pgfsetstrokecolor{currentstroke}%
\pgfsetstrokeopacity{0.000000}%
\pgfsetdash{}{0pt}%
\pgfpathmoveto{\pgfqpoint{3.787427in}{4.169889in}}%
\pgfpathlineto{\pgfqpoint{5.840250in}{4.169889in}}%
\pgfpathlineto{\pgfqpoint{5.840250in}{5.460667in}}%
\pgfpathlineto{\pgfqpoint{3.787427in}{5.460667in}}%
\pgfpathlineto{\pgfqpoint{3.787427in}{4.169889in}}%
\pgfpathclose%
\pgfusepath{fill}%
\end{pgfscope}%
\begin{pgfscope}%
\pgfpathrectangle{\pgfqpoint{3.787427in}{4.169889in}}{\pgfqpoint{2.052823in}{1.290778in}}%
\pgfusepath{clip}%
\pgfsetbuttcap%
\pgfsetmiterjoin%
\definecolor{currentfill}{rgb}{1.000000,0.000000,0.000000}%
\pgfsetfillcolor{currentfill}%
\pgfsetlinewidth{0.000000pt}%
\definecolor{currentstroke}{rgb}{0.000000,0.000000,0.000000}%
\pgfsetstrokecolor{currentstroke}%
\pgfsetstrokeopacity{0.000000}%
\pgfsetdash{}{0pt}%
\pgfpathmoveto{\pgfqpoint{3.880737in}{4.169889in}}%
\pgfpathlineto{\pgfqpoint{4.710161in}{4.169889in}}%
\pgfpathlineto{\pgfqpoint{4.710161in}{5.245537in}}%
\pgfpathlineto{\pgfqpoint{3.880737in}{5.245537in}}%
\pgfpathlineto{\pgfqpoint{3.880737in}{4.169889in}}%
\pgfpathclose%
\pgfusepath{fill}%
\end{pgfscope}%
\begin{pgfscope}%
\pgfpathrectangle{\pgfqpoint{3.787427in}{4.169889in}}{\pgfqpoint{2.052823in}{1.290778in}}%
\pgfusepath{clip}%
\pgfsetbuttcap%
\pgfsetmiterjoin%
\definecolor{currentfill}{rgb}{0.000000,0.000000,1.000000}%
\pgfsetfillcolor{currentfill}%
\pgfsetlinewidth{0.000000pt}%
\definecolor{currentstroke}{rgb}{0.000000,0.000000,0.000000}%
\pgfsetstrokecolor{currentstroke}%
\pgfsetstrokeopacity{0.000000}%
\pgfsetdash{}{0pt}%
\pgfpathmoveto{\pgfqpoint{4.917516in}{4.169889in}}%
\pgfpathlineto{\pgfqpoint{5.746940in}{4.169889in}}%
\pgfpathlineto{\pgfqpoint{5.746940in}{5.245335in}}%
\pgfpathlineto{\pgfqpoint{4.917516in}{5.245335in}}%
\pgfpathlineto{\pgfqpoint{4.917516in}{4.169889in}}%
\pgfpathclose%
\pgfusepath{fill}%
\end{pgfscope}%
\begin{pgfscope}%
\pgfsetbuttcap%
\pgfsetroundjoin%
\definecolor{currentfill}{rgb}{0.000000,0.000000,0.000000}%
\pgfsetfillcolor{currentfill}%
\pgfsetlinewidth{0.803000pt}%
\definecolor{currentstroke}{rgb}{0.000000,0.000000,0.000000}%
\pgfsetstrokecolor{currentstroke}%
\pgfsetdash{}{0pt}%
\pgfsys@defobject{currentmarker}{\pgfqpoint{0.000000in}{-0.048611in}}{\pgfqpoint{0.000000in}{0.000000in}}{%
\pgfpathmoveto{\pgfqpoint{0.000000in}{0.000000in}}%
\pgfpathlineto{\pgfqpoint{0.000000in}{-0.048611in}}%
\pgfusepath{stroke,fill}%
}%
\begin{pgfscope}%
\pgfsys@transformshift{4.295449in}{4.169889in}%
\pgfsys@useobject{currentmarker}{}%
\end{pgfscope}%
\end{pgfscope}%
\begin{pgfscope}%
\definecolor{textcolor}{rgb}{0.000000,0.000000,0.000000}%
\pgfsetstrokecolor{textcolor}%
\pgfsetfillcolor{textcolor}%
\pgftext[x=4.295449in,y=4.072667in,,top]{\color{textcolor}{\sffamily\fontsize{10.000000}{12.000000}\selectfont\catcode`\^=\active\def^{\ifmmode\sp\else\^{}\fi}\catcode`\%=\active\def%{\%}CPU 8192}}%
\end{pgfscope}%
\begin{pgfscope}%
\pgfsetbuttcap%
\pgfsetroundjoin%
\definecolor{currentfill}{rgb}{0.000000,0.000000,0.000000}%
\pgfsetfillcolor{currentfill}%
\pgfsetlinewidth{0.803000pt}%
\definecolor{currentstroke}{rgb}{0.000000,0.000000,0.000000}%
\pgfsetstrokecolor{currentstroke}%
\pgfsetdash{}{0pt}%
\pgfsys@defobject{currentmarker}{\pgfqpoint{0.000000in}{-0.048611in}}{\pgfqpoint{0.000000in}{0.000000in}}{%
\pgfpathmoveto{\pgfqpoint{0.000000in}{0.000000in}}%
\pgfpathlineto{\pgfqpoint{0.000000in}{-0.048611in}}%
\pgfusepath{stroke,fill}%
}%
\begin{pgfscope}%
\pgfsys@transformshift{5.332228in}{4.169889in}%
\pgfsys@useobject{currentmarker}{}%
\end{pgfscope}%
\end{pgfscope}%
\begin{pgfscope}%
\definecolor{textcolor}{rgb}{0.000000,0.000000,0.000000}%
\pgfsetstrokecolor{textcolor}%
\pgfsetfillcolor{textcolor}%
\pgftext[x=5.332228in,y=4.072667in,,top]{\color{textcolor}{\sffamily\fontsize{10.000000}{12.000000}\selectfont\catcode`\^=\active\def^{\ifmmode\sp\else\^{}\fi}\catcode`\%=\active\def%{\%}GPU 8192}}%
\end{pgfscope}%
\begin{pgfscope}%
\pgfsetbuttcap%
\pgfsetroundjoin%
\definecolor{currentfill}{rgb}{0.000000,0.000000,0.000000}%
\pgfsetfillcolor{currentfill}%
\pgfsetlinewidth{0.803000pt}%
\definecolor{currentstroke}{rgb}{0.000000,0.000000,0.000000}%
\pgfsetstrokecolor{currentstroke}%
\pgfsetdash{}{0pt}%
\pgfsys@defobject{currentmarker}{\pgfqpoint{-0.048611in}{0.000000in}}{\pgfqpoint{-0.000000in}{0.000000in}}{%
\pgfpathmoveto{\pgfqpoint{-0.000000in}{0.000000in}}%
\pgfpathlineto{\pgfqpoint{-0.048611in}{0.000000in}}%
\pgfusepath{stroke,fill}%
}%
\begin{pgfscope}%
\pgfsys@transformshift{3.787427in}{4.169889in}%
\pgfsys@useobject{currentmarker}{}%
\end{pgfscope}%
\end{pgfscope}%
\begin{pgfscope}%
\definecolor{textcolor}{rgb}{0.000000,0.000000,0.000000}%
\pgfsetstrokecolor{textcolor}%
\pgfsetfillcolor{textcolor}%
\pgftext[x=3.601840in, y=4.117127in, left, base]{\color{textcolor}{\sffamily\fontsize{10.000000}{12.000000}\selectfont\catcode`\^=\active\def^{\ifmmode\sp\else\^{}\fi}\catcode`\%=\active\def%{\%}0}}%
\end{pgfscope}%
\begin{pgfscope}%
\pgfsetbuttcap%
\pgfsetroundjoin%
\definecolor{currentfill}{rgb}{0.000000,0.000000,0.000000}%
\pgfsetfillcolor{currentfill}%
\pgfsetlinewidth{0.803000pt}%
\definecolor{currentstroke}{rgb}{0.000000,0.000000,0.000000}%
\pgfsetstrokecolor{currentstroke}%
\pgfsetdash{}{0pt}%
\pgfsys@defobject{currentmarker}{\pgfqpoint{-0.048611in}{0.000000in}}{\pgfqpoint{-0.000000in}{0.000000in}}{%
\pgfpathmoveto{\pgfqpoint{-0.000000in}{0.000000in}}%
\pgfpathlineto{\pgfqpoint{-0.048611in}{0.000000in}}%
\pgfusepath{stroke,fill}%
}%
\begin{pgfscope}%
\pgfsys@transformshift{3.787427in}{4.800053in}%
\pgfsys@useobject{currentmarker}{}%
\end{pgfscope}%
\end{pgfscope}%
\begin{pgfscope}%
\definecolor{textcolor}{rgb}{0.000000,0.000000,0.000000}%
\pgfsetstrokecolor{textcolor}%
\pgfsetfillcolor{textcolor}%
\pgftext[x=3.160013in, y=4.747292in, left, base]{\color{textcolor}{\sffamily\fontsize{10.000000}{12.000000}\selectfont\catcode`\^=\active\def^{\ifmmode\sp\else\^{}\fi}\catcode`\%=\active\def%{\%}100000}}%
\end{pgfscope}%
\begin{pgfscope}%
\pgfsetbuttcap%
\pgfsetroundjoin%
\definecolor{currentfill}{rgb}{0.000000,0.000000,0.000000}%
\pgfsetfillcolor{currentfill}%
\pgfsetlinewidth{0.803000pt}%
\definecolor{currentstroke}{rgb}{0.000000,0.000000,0.000000}%
\pgfsetstrokecolor{currentstroke}%
\pgfsetdash{}{0pt}%
\pgfsys@defobject{currentmarker}{\pgfqpoint{-0.048611in}{0.000000in}}{\pgfqpoint{-0.000000in}{0.000000in}}{%
\pgfpathmoveto{\pgfqpoint{-0.000000in}{0.000000in}}%
\pgfpathlineto{\pgfqpoint{-0.048611in}{0.000000in}}%
\pgfusepath{stroke,fill}%
}%
\begin{pgfscope}%
\pgfsys@transformshift{3.787427in}{5.430218in}%
\pgfsys@useobject{currentmarker}{}%
\end{pgfscope}%
\end{pgfscope}%
\begin{pgfscope}%
\definecolor{textcolor}{rgb}{0.000000,0.000000,0.000000}%
\pgfsetstrokecolor{textcolor}%
\pgfsetfillcolor{textcolor}%
\pgftext[x=3.160013in, y=5.377456in, left, base]{\color{textcolor}{\sffamily\fontsize{10.000000}{12.000000}\selectfont\catcode`\^=\active\def^{\ifmmode\sp\else\^{}\fi}\catcode`\%=\active\def%{\%}200000}}%
\end{pgfscope}%
\begin{pgfscope}%
\pgfsetrectcap%
\pgfsetmiterjoin%
\pgfsetlinewidth{0.803000pt}%
\definecolor{currentstroke}{rgb}{0.000000,0.000000,0.000000}%
\pgfsetstrokecolor{currentstroke}%
\pgfsetdash{}{0pt}%
\pgfpathmoveto{\pgfqpoint{3.787427in}{4.169889in}}%
\pgfpathlineto{\pgfqpoint{3.787427in}{5.460667in}}%
\pgfusepath{stroke}%
\end{pgfscope}%
\begin{pgfscope}%
\pgfsetrectcap%
\pgfsetmiterjoin%
\pgfsetlinewidth{0.803000pt}%
\definecolor{currentstroke}{rgb}{0.000000,0.000000,0.000000}%
\pgfsetstrokecolor{currentstroke}%
\pgfsetdash{}{0pt}%
\pgfpathmoveto{\pgfqpoint{5.840250in}{4.169889in}}%
\pgfpathlineto{\pgfqpoint{5.840250in}{5.460667in}}%
\pgfusepath{stroke}%
\end{pgfscope}%
\begin{pgfscope}%
\pgfsetrectcap%
\pgfsetmiterjoin%
\pgfsetlinewidth{0.803000pt}%
\definecolor{currentstroke}{rgb}{0.000000,0.000000,0.000000}%
\pgfsetstrokecolor{currentstroke}%
\pgfsetdash{}{0pt}%
\pgfpathmoveto{\pgfqpoint{3.787427in}{4.169889in}}%
\pgfpathlineto{\pgfqpoint{5.840250in}{4.169889in}}%
\pgfusepath{stroke}%
\end{pgfscope}%
\begin{pgfscope}%
\pgfsetrectcap%
\pgfsetmiterjoin%
\pgfsetlinewidth{0.803000pt}%
\definecolor{currentstroke}{rgb}{0.000000,0.000000,0.000000}%
\pgfsetstrokecolor{currentstroke}%
\pgfsetdash{}{0pt}%
\pgfpathmoveto{\pgfqpoint{3.787427in}{5.460667in}}%
\pgfpathlineto{\pgfqpoint{5.840250in}{5.460667in}}%
\pgfusepath{stroke}%
\end{pgfscope}%
\begin{pgfscope}%
\definecolor{textcolor}{rgb}{0.000000,0.000000,0.000000}%
\pgfsetstrokecolor{textcolor}%
\pgfsetfillcolor{textcolor}%
\pgftext[x=4.295449in,y=5.245537in,,bottom]{\color{textcolor}{\sffamily\fontsize{10.000000}{12.000000}\selectfont\catcode`\^=\active\def^{\ifmmode\sp\else\^{}\fi}\catcode`\%=\active\def%{\%}170693.238}}%
\end{pgfscope}%
\begin{pgfscope}%
\definecolor{textcolor}{rgb}{0.000000,0.000000,0.000000}%
\pgfsetstrokecolor{textcolor}%
\pgfsetfillcolor{textcolor}%
\pgftext[x=5.332228in,y=5.245335in,,bottom]{\color{textcolor}{\sffamily\fontsize{10.000000}{12.000000}\selectfont\catcode`\^=\active\def^{\ifmmode\sp\else\^{}\fi}\catcode`\%=\active\def%{\%}170661.125}}%
\end{pgfscope}%
\begin{pgfscope}%
\definecolor{textcolor}{rgb}{0.000000,0.000000,0.000000}%
\pgfsetstrokecolor{textcolor}%
\pgfsetfillcolor{textcolor}%
\pgftext[x=4.510250in, y=5.739312in, left, base]{\color{textcolor}{\sffamily\fontsize{12.000000}{14.400000}\selectfont\catcode`\^=\active\def^{\ifmmode\sp\else\^{}\fi}\catcode`\%=\active\def%{\%}(8192) }}%
\end{pgfscope}%
\begin{pgfscope}%
\definecolor{textcolor}{rgb}{0.000000,0.000000,0.000000}%
\pgfsetstrokecolor{textcolor}%
\pgfsetfillcolor{textcolor}%
\pgftext[x=3.798010in, y=5.544000in, left, base]{\color{textcolor}{\sffamily\fontsize{12.000000}{14.400000}\selectfont\catcode`\^=\active\def^{\ifmmode\sp\else\^{}\fi}\catcode`\%=\active\def%{\%}średni czas trwania pętli}}%
\end{pgfscope}%
\begin{pgfscope}%
\pgfsetbuttcap%
\pgfsetmiterjoin%
\definecolor{currentfill}{rgb}{1.000000,1.000000,1.000000}%
\pgfsetfillcolor{currentfill}%
\pgfsetlinewidth{0.000000pt}%
\definecolor{currentstroke}{rgb}{0.000000,0.000000,0.000000}%
\pgfsetstrokecolor{currentstroke}%
\pgfsetstrokeopacity{0.000000}%
\pgfsetdash{}{0pt}%
\pgfpathmoveto{\pgfqpoint{0.956181in}{2.278556in}}%
\pgfpathlineto{\pgfqpoint{3.009004in}{2.278556in}}%
\pgfpathlineto{\pgfqpoint{3.009004in}{3.569333in}}%
\pgfpathlineto{\pgfqpoint{0.956181in}{3.569333in}}%
\pgfpathlineto{\pgfqpoint{0.956181in}{2.278556in}}%
\pgfpathclose%
\pgfusepath{fill}%
\end{pgfscope}%
\begin{pgfscope}%
\pgfpathrectangle{\pgfqpoint{0.956181in}{2.278556in}}{\pgfqpoint{2.052823in}{1.290778in}}%
\pgfusepath{clip}%
\pgfsetbuttcap%
\pgfsetmiterjoin%
\definecolor{currentfill}{rgb}{1.000000,0.000000,0.000000}%
\pgfsetfillcolor{currentfill}%
\pgfsetlinewidth{0.000000pt}%
\definecolor{currentstroke}{rgb}{0.000000,0.000000,0.000000}%
\pgfsetstrokecolor{currentstroke}%
\pgfsetstrokeopacity{0.000000}%
\pgfsetdash{}{0pt}%
\pgfpathmoveto{\pgfqpoint{1.049492in}{2.278556in}}%
\pgfpathlineto{\pgfqpoint{1.878915in}{2.278556in}}%
\pgfpathlineto{\pgfqpoint{1.878915in}{3.354204in}}%
\pgfpathlineto{\pgfqpoint{1.049492in}{3.354204in}}%
\pgfpathlineto{\pgfqpoint{1.049492in}{2.278556in}}%
\pgfpathclose%
\pgfusepath{fill}%
\end{pgfscope}%
\begin{pgfscope}%
\pgfpathrectangle{\pgfqpoint{0.956181in}{2.278556in}}{\pgfqpoint{2.052823in}{1.290778in}}%
\pgfusepath{clip}%
\pgfsetbuttcap%
\pgfsetmiterjoin%
\definecolor{currentfill}{rgb}{0.000000,0.000000,1.000000}%
\pgfsetfillcolor{currentfill}%
\pgfsetlinewidth{0.000000pt}%
\definecolor{currentstroke}{rgb}{0.000000,0.000000,0.000000}%
\pgfsetstrokecolor{currentstroke}%
\pgfsetstrokeopacity{0.000000}%
\pgfsetdash{}{0pt}%
\pgfpathmoveto{\pgfqpoint{2.086271in}{2.278556in}}%
\pgfpathlineto{\pgfqpoint{2.915694in}{2.278556in}}%
\pgfpathlineto{\pgfqpoint{2.915694in}{3.332707in}}%
\pgfpathlineto{\pgfqpoint{2.086271in}{3.332707in}}%
\pgfpathlineto{\pgfqpoint{2.086271in}{2.278556in}}%
\pgfpathclose%
\pgfusepath{fill}%
\end{pgfscope}%
\begin{pgfscope}%
\pgfsetbuttcap%
\pgfsetroundjoin%
\definecolor{currentfill}{rgb}{0.000000,0.000000,0.000000}%
\pgfsetfillcolor{currentfill}%
\pgfsetlinewidth{0.803000pt}%
\definecolor{currentstroke}{rgb}{0.000000,0.000000,0.000000}%
\pgfsetstrokecolor{currentstroke}%
\pgfsetdash{}{0pt}%
\pgfsys@defobject{currentmarker}{\pgfqpoint{0.000000in}{-0.048611in}}{\pgfqpoint{0.000000in}{0.000000in}}{%
\pgfpathmoveto{\pgfqpoint{0.000000in}{0.000000in}}%
\pgfpathlineto{\pgfqpoint{0.000000in}{-0.048611in}}%
\pgfusepath{stroke,fill}%
}%
\begin{pgfscope}%
\pgfsys@transformshift{1.464203in}{2.278556in}%
\pgfsys@useobject{currentmarker}{}%
\end{pgfscope}%
\end{pgfscope}%
\begin{pgfscope}%
\definecolor{textcolor}{rgb}{0.000000,0.000000,0.000000}%
\pgfsetstrokecolor{textcolor}%
\pgfsetfillcolor{textcolor}%
\pgftext[x=1.464203in,y=2.181333in,,top]{\color{textcolor}{\sffamily\fontsize{10.000000}{12.000000}\selectfont\catcode`\^=\active\def^{\ifmmode\sp\else\^{}\fi}\catcode`\%=\active\def%{\%}CPU 1024}}%
\end{pgfscope}%
\begin{pgfscope}%
\pgfsetbuttcap%
\pgfsetroundjoin%
\definecolor{currentfill}{rgb}{0.000000,0.000000,0.000000}%
\pgfsetfillcolor{currentfill}%
\pgfsetlinewidth{0.803000pt}%
\definecolor{currentstroke}{rgb}{0.000000,0.000000,0.000000}%
\pgfsetstrokecolor{currentstroke}%
\pgfsetdash{}{0pt}%
\pgfsys@defobject{currentmarker}{\pgfqpoint{0.000000in}{-0.048611in}}{\pgfqpoint{0.000000in}{0.000000in}}{%
\pgfpathmoveto{\pgfqpoint{0.000000in}{0.000000in}}%
\pgfpathlineto{\pgfqpoint{0.000000in}{-0.048611in}}%
\pgfusepath{stroke,fill}%
}%
\begin{pgfscope}%
\pgfsys@transformshift{2.500983in}{2.278556in}%
\pgfsys@useobject{currentmarker}{}%
\end{pgfscope}%
\end{pgfscope}%
\begin{pgfscope}%
\definecolor{textcolor}{rgb}{0.000000,0.000000,0.000000}%
\pgfsetstrokecolor{textcolor}%
\pgfsetfillcolor{textcolor}%
\pgftext[x=2.500983in,y=2.181333in,,top]{\color{textcolor}{\sffamily\fontsize{10.000000}{12.000000}\selectfont\catcode`\^=\active\def^{\ifmmode\sp\else\^{}\fi}\catcode`\%=\active\def%{\%}GPU 1024}}%
\end{pgfscope}%
\begin{pgfscope}%
\pgfsetbuttcap%
\pgfsetroundjoin%
\definecolor{currentfill}{rgb}{0.000000,0.000000,0.000000}%
\pgfsetfillcolor{currentfill}%
\pgfsetlinewidth{0.803000pt}%
\definecolor{currentstroke}{rgb}{0.000000,0.000000,0.000000}%
\pgfsetstrokecolor{currentstroke}%
\pgfsetdash{}{0pt}%
\pgfsys@defobject{currentmarker}{\pgfqpoint{-0.048611in}{0.000000in}}{\pgfqpoint{-0.000000in}{0.000000in}}{%
\pgfpathmoveto{\pgfqpoint{-0.000000in}{0.000000in}}%
\pgfpathlineto{\pgfqpoint{-0.048611in}{0.000000in}}%
\pgfusepath{stroke,fill}%
}%
\begin{pgfscope}%
\pgfsys@transformshift{0.956181in}{2.278556in}%
\pgfsys@useobject{currentmarker}{}%
\end{pgfscope}%
\end{pgfscope}%
\begin{pgfscope}%
\definecolor{textcolor}{rgb}{0.000000,0.000000,0.000000}%
\pgfsetstrokecolor{textcolor}%
\pgfsetfillcolor{textcolor}%
\pgftext[x=0.770594in, y=2.225794in, left, base]{\color{textcolor}{\sffamily\fontsize{10.000000}{12.000000}\selectfont\catcode`\^=\active\def^{\ifmmode\sp\else\^{}\fi}\catcode`\%=\active\def%{\%}0}}%
\end{pgfscope}%
\begin{pgfscope}%
\pgfsetbuttcap%
\pgfsetroundjoin%
\definecolor{currentfill}{rgb}{0.000000,0.000000,0.000000}%
\pgfsetfillcolor{currentfill}%
\pgfsetlinewidth{0.803000pt}%
\definecolor{currentstroke}{rgb}{0.000000,0.000000,0.000000}%
\pgfsetstrokecolor{currentstroke}%
\pgfsetdash{}{0pt}%
\pgfsys@defobject{currentmarker}{\pgfqpoint{-0.048611in}{0.000000in}}{\pgfqpoint{-0.000000in}{0.000000in}}{%
\pgfpathmoveto{\pgfqpoint{-0.000000in}{0.000000in}}%
\pgfpathlineto{\pgfqpoint{-0.048611in}{0.000000in}}%
\pgfusepath{stroke,fill}%
}%
\begin{pgfscope}%
\pgfsys@transformshift{0.956181in}{2.667736in}%
\pgfsys@useobject{currentmarker}{}%
\end{pgfscope}%
\end{pgfscope}%
\begin{pgfscope}%
\definecolor{textcolor}{rgb}{0.000000,0.000000,0.000000}%
\pgfsetstrokecolor{textcolor}%
\pgfsetfillcolor{textcolor}%
\pgftext[x=0.505498in, y=2.614974in, left, base]{\color{textcolor}{\sffamily\fontsize{10.000000}{12.000000}\selectfont\catcode`\^=\active\def^{\ifmmode\sp\else\^{}\fi}\catcode`\%=\active\def%{\%}5000}}%
\end{pgfscope}%
\begin{pgfscope}%
\pgfsetbuttcap%
\pgfsetroundjoin%
\definecolor{currentfill}{rgb}{0.000000,0.000000,0.000000}%
\pgfsetfillcolor{currentfill}%
\pgfsetlinewidth{0.803000pt}%
\definecolor{currentstroke}{rgb}{0.000000,0.000000,0.000000}%
\pgfsetstrokecolor{currentstroke}%
\pgfsetdash{}{0pt}%
\pgfsys@defobject{currentmarker}{\pgfqpoint{-0.048611in}{0.000000in}}{\pgfqpoint{-0.000000in}{0.000000in}}{%
\pgfpathmoveto{\pgfqpoint{-0.000000in}{0.000000in}}%
\pgfpathlineto{\pgfqpoint{-0.048611in}{0.000000in}}%
\pgfusepath{stroke,fill}%
}%
\begin{pgfscope}%
\pgfsys@transformshift{0.956181in}{3.056916in}%
\pgfsys@useobject{currentmarker}{}%
\end{pgfscope}%
\end{pgfscope}%
\begin{pgfscope}%
\definecolor{textcolor}{rgb}{0.000000,0.000000,0.000000}%
\pgfsetstrokecolor{textcolor}%
\pgfsetfillcolor{textcolor}%
\pgftext[x=0.417133in, y=3.004155in, left, base]{\color{textcolor}{\sffamily\fontsize{10.000000}{12.000000}\selectfont\catcode`\^=\active\def^{\ifmmode\sp\else\^{}\fi}\catcode`\%=\active\def%{\%}10000}}%
\end{pgfscope}%
\begin{pgfscope}%
\pgfsetbuttcap%
\pgfsetroundjoin%
\definecolor{currentfill}{rgb}{0.000000,0.000000,0.000000}%
\pgfsetfillcolor{currentfill}%
\pgfsetlinewidth{0.803000pt}%
\definecolor{currentstroke}{rgb}{0.000000,0.000000,0.000000}%
\pgfsetstrokecolor{currentstroke}%
\pgfsetdash{}{0pt}%
\pgfsys@defobject{currentmarker}{\pgfqpoint{-0.048611in}{0.000000in}}{\pgfqpoint{-0.000000in}{0.000000in}}{%
\pgfpathmoveto{\pgfqpoint{-0.000000in}{0.000000in}}%
\pgfpathlineto{\pgfqpoint{-0.048611in}{0.000000in}}%
\pgfusepath{stroke,fill}%
}%
\begin{pgfscope}%
\pgfsys@transformshift{0.956181in}{3.446097in}%
\pgfsys@useobject{currentmarker}{}%
\end{pgfscope}%
\end{pgfscope}%
\begin{pgfscope}%
\definecolor{textcolor}{rgb}{0.000000,0.000000,0.000000}%
\pgfsetstrokecolor{textcolor}%
\pgfsetfillcolor{textcolor}%
\pgftext[x=0.417133in, y=3.393335in, left, base]{\color{textcolor}{\sffamily\fontsize{10.000000}{12.000000}\selectfont\catcode`\^=\active\def^{\ifmmode\sp\else\^{}\fi}\catcode`\%=\active\def%{\%}15000}}%
\end{pgfscope}%
\begin{pgfscope}%
\definecolor{textcolor}{rgb}{0.000000,0.000000,0.000000}%
\pgfsetstrokecolor{textcolor}%
\pgfsetfillcolor{textcolor}%
\pgftext[x=0.361577in,y=2.923944in,,bottom,rotate=90.000000]{\color{textcolor}{\sffamily\fontsize{10.000000}{12.000000}\selectfont\catcode`\^=\active\def^{\ifmmode\sp\else\^{}\fi}\catcode`\%=\active\def%{\%}czas (ns)}}%
\end{pgfscope}%
\begin{pgfscope}%
\pgfsetrectcap%
\pgfsetmiterjoin%
\pgfsetlinewidth{0.803000pt}%
\definecolor{currentstroke}{rgb}{0.000000,0.000000,0.000000}%
\pgfsetstrokecolor{currentstroke}%
\pgfsetdash{}{0pt}%
\pgfpathmoveto{\pgfqpoint{0.956181in}{2.278556in}}%
\pgfpathlineto{\pgfqpoint{0.956181in}{3.569333in}}%
\pgfusepath{stroke}%
\end{pgfscope}%
\begin{pgfscope}%
\pgfsetrectcap%
\pgfsetmiterjoin%
\pgfsetlinewidth{0.803000pt}%
\definecolor{currentstroke}{rgb}{0.000000,0.000000,0.000000}%
\pgfsetstrokecolor{currentstroke}%
\pgfsetdash{}{0pt}%
\pgfpathmoveto{\pgfqpoint{3.009004in}{2.278556in}}%
\pgfpathlineto{\pgfqpoint{3.009004in}{3.569333in}}%
\pgfusepath{stroke}%
\end{pgfscope}%
\begin{pgfscope}%
\pgfsetrectcap%
\pgfsetmiterjoin%
\pgfsetlinewidth{0.803000pt}%
\definecolor{currentstroke}{rgb}{0.000000,0.000000,0.000000}%
\pgfsetstrokecolor{currentstroke}%
\pgfsetdash{}{0pt}%
\pgfpathmoveto{\pgfqpoint{0.956181in}{2.278556in}}%
\pgfpathlineto{\pgfqpoint{3.009004in}{2.278556in}}%
\pgfusepath{stroke}%
\end{pgfscope}%
\begin{pgfscope}%
\pgfsetrectcap%
\pgfsetmiterjoin%
\pgfsetlinewidth{0.803000pt}%
\definecolor{currentstroke}{rgb}{0.000000,0.000000,0.000000}%
\pgfsetstrokecolor{currentstroke}%
\pgfsetdash{}{0pt}%
\pgfpathmoveto{\pgfqpoint{0.956181in}{3.569333in}}%
\pgfpathlineto{\pgfqpoint{3.009004in}{3.569333in}}%
\pgfusepath{stroke}%
\end{pgfscope}%
\begin{pgfscope}%
\definecolor{textcolor}{rgb}{0.000000,0.000000,0.000000}%
\pgfsetstrokecolor{textcolor}%
\pgfsetfillcolor{textcolor}%
\pgftext[x=1.464203in,y=3.354204in,,bottom]{\color{textcolor}{\sffamily\fontsize{10.000000}{12.000000}\selectfont\catcode`\^=\active\def^{\ifmmode\sp\else\^{}\fi}\catcode`\%=\active\def%{\%}13819.401}}%
\end{pgfscope}%
\begin{pgfscope}%
\definecolor{textcolor}{rgb}{0.000000,0.000000,0.000000}%
\pgfsetstrokecolor{textcolor}%
\pgfsetfillcolor{textcolor}%
\pgftext[x=2.500983in,y=3.332707in,,bottom]{\color{textcolor}{\sffamily\fontsize{10.000000}{12.000000}\selectfont\catcode`\^=\active\def^{\ifmmode\sp\else\^{}\fi}\catcode`\%=\active\def%{\%}13543.225}}%
\end{pgfscope}%
\begin{pgfscope}%
\definecolor{textcolor}{rgb}{0.000000,0.000000,0.000000}%
\pgfsetstrokecolor{textcolor}%
\pgfsetfillcolor{textcolor}%
\pgftext[x=1.982593in,y=3.652667in,,base]{\color{textcolor}{\sffamily\fontsize{12.000000}{14.400000}\selectfont\catcode`\^=\active\def^{\ifmmode\sp\else\^{}\fi}\catcode`\%=\active\def%{\%}średni czas obliczeń}}%
\end{pgfscope}%
\begin{pgfscope}%
\pgfsetbuttcap%
\pgfsetmiterjoin%
\definecolor{currentfill}{rgb}{1.000000,1.000000,1.000000}%
\pgfsetfillcolor{currentfill}%
\pgfsetlinewidth{0.000000pt}%
\definecolor{currentstroke}{rgb}{0.000000,0.000000,0.000000}%
\pgfsetstrokecolor{currentstroke}%
\pgfsetstrokeopacity{0.000000}%
\pgfsetdash{}{0pt}%
\pgfpathmoveto{\pgfqpoint{3.787427in}{2.278556in}}%
\pgfpathlineto{\pgfqpoint{5.840250in}{2.278556in}}%
\pgfpathlineto{\pgfqpoint{5.840250in}{3.569333in}}%
\pgfpathlineto{\pgfqpoint{3.787427in}{3.569333in}}%
\pgfpathlineto{\pgfqpoint{3.787427in}{2.278556in}}%
\pgfpathclose%
\pgfusepath{fill}%
\end{pgfscope}%
\begin{pgfscope}%
\pgfpathrectangle{\pgfqpoint{3.787427in}{2.278556in}}{\pgfqpoint{2.052823in}{1.290778in}}%
\pgfusepath{clip}%
\pgfsetbuttcap%
\pgfsetmiterjoin%
\definecolor{currentfill}{rgb}{1.000000,0.000000,0.000000}%
\pgfsetfillcolor{currentfill}%
\pgfsetlinewidth{0.000000pt}%
\definecolor{currentstroke}{rgb}{0.000000,0.000000,0.000000}%
\pgfsetstrokecolor{currentstroke}%
\pgfsetstrokeopacity{0.000000}%
\pgfsetdash{}{0pt}%
\pgfpathmoveto{\pgfqpoint{3.880737in}{2.278556in}}%
\pgfpathlineto{\pgfqpoint{4.710161in}{2.278556in}}%
\pgfpathlineto{\pgfqpoint{4.710161in}{3.354204in}}%
\pgfpathlineto{\pgfqpoint{3.880737in}{3.354204in}}%
\pgfpathlineto{\pgfqpoint{3.880737in}{2.278556in}}%
\pgfpathclose%
\pgfusepath{fill}%
\end{pgfscope}%
\begin{pgfscope}%
\pgfpathrectangle{\pgfqpoint{3.787427in}{2.278556in}}{\pgfqpoint{2.052823in}{1.290778in}}%
\pgfusepath{clip}%
\pgfsetbuttcap%
\pgfsetmiterjoin%
\definecolor{currentfill}{rgb}{0.000000,0.000000,1.000000}%
\pgfsetfillcolor{currentfill}%
\pgfsetlinewidth{0.000000pt}%
\definecolor{currentstroke}{rgb}{0.000000,0.000000,0.000000}%
\pgfsetstrokecolor{currentstroke}%
\pgfsetstrokeopacity{0.000000}%
\pgfsetdash{}{0pt}%
\pgfpathmoveto{\pgfqpoint{4.917516in}{2.278556in}}%
\pgfpathlineto{\pgfqpoint{5.746940in}{2.278556in}}%
\pgfpathlineto{\pgfqpoint{5.746940in}{3.347777in}}%
\pgfpathlineto{\pgfqpoint{4.917516in}{3.347777in}}%
\pgfpathlineto{\pgfqpoint{4.917516in}{2.278556in}}%
\pgfpathclose%
\pgfusepath{fill}%
\end{pgfscope}%
\begin{pgfscope}%
\pgfsetbuttcap%
\pgfsetroundjoin%
\definecolor{currentfill}{rgb}{0.000000,0.000000,0.000000}%
\pgfsetfillcolor{currentfill}%
\pgfsetlinewidth{0.803000pt}%
\definecolor{currentstroke}{rgb}{0.000000,0.000000,0.000000}%
\pgfsetstrokecolor{currentstroke}%
\pgfsetdash{}{0pt}%
\pgfsys@defobject{currentmarker}{\pgfqpoint{0.000000in}{-0.048611in}}{\pgfqpoint{0.000000in}{0.000000in}}{%
\pgfpathmoveto{\pgfqpoint{0.000000in}{0.000000in}}%
\pgfpathlineto{\pgfqpoint{0.000000in}{-0.048611in}}%
\pgfusepath{stroke,fill}%
}%
\begin{pgfscope}%
\pgfsys@transformshift{4.295449in}{2.278556in}%
\pgfsys@useobject{currentmarker}{}%
\end{pgfscope}%
\end{pgfscope}%
\begin{pgfscope}%
\definecolor{textcolor}{rgb}{0.000000,0.000000,0.000000}%
\pgfsetstrokecolor{textcolor}%
\pgfsetfillcolor{textcolor}%
\pgftext[x=4.295449in,y=2.181333in,,top]{\color{textcolor}{\sffamily\fontsize{10.000000}{12.000000}\selectfont\catcode`\^=\active\def^{\ifmmode\sp\else\^{}\fi}\catcode`\%=\active\def%{\%}CPU 8192}}%
\end{pgfscope}%
\begin{pgfscope}%
\pgfsetbuttcap%
\pgfsetroundjoin%
\definecolor{currentfill}{rgb}{0.000000,0.000000,0.000000}%
\pgfsetfillcolor{currentfill}%
\pgfsetlinewidth{0.803000pt}%
\definecolor{currentstroke}{rgb}{0.000000,0.000000,0.000000}%
\pgfsetstrokecolor{currentstroke}%
\pgfsetdash{}{0pt}%
\pgfsys@defobject{currentmarker}{\pgfqpoint{0.000000in}{-0.048611in}}{\pgfqpoint{0.000000in}{0.000000in}}{%
\pgfpathmoveto{\pgfqpoint{0.000000in}{0.000000in}}%
\pgfpathlineto{\pgfqpoint{0.000000in}{-0.048611in}}%
\pgfusepath{stroke,fill}%
}%
\begin{pgfscope}%
\pgfsys@transformshift{5.332228in}{2.278556in}%
\pgfsys@useobject{currentmarker}{}%
\end{pgfscope}%
\end{pgfscope}%
\begin{pgfscope}%
\definecolor{textcolor}{rgb}{0.000000,0.000000,0.000000}%
\pgfsetstrokecolor{textcolor}%
\pgfsetfillcolor{textcolor}%
\pgftext[x=5.332228in,y=2.181333in,,top]{\color{textcolor}{\sffamily\fontsize{10.000000}{12.000000}\selectfont\catcode`\^=\active\def^{\ifmmode\sp\else\^{}\fi}\catcode`\%=\active\def%{\%}GPU 8192}}%
\end{pgfscope}%
\begin{pgfscope}%
\pgfsetbuttcap%
\pgfsetroundjoin%
\definecolor{currentfill}{rgb}{0.000000,0.000000,0.000000}%
\pgfsetfillcolor{currentfill}%
\pgfsetlinewidth{0.803000pt}%
\definecolor{currentstroke}{rgb}{0.000000,0.000000,0.000000}%
\pgfsetstrokecolor{currentstroke}%
\pgfsetdash{}{0pt}%
\pgfsys@defobject{currentmarker}{\pgfqpoint{-0.048611in}{0.000000in}}{\pgfqpoint{-0.000000in}{0.000000in}}{%
\pgfpathmoveto{\pgfqpoint{-0.000000in}{0.000000in}}%
\pgfpathlineto{\pgfqpoint{-0.048611in}{0.000000in}}%
\pgfusepath{stroke,fill}%
}%
\begin{pgfscope}%
\pgfsys@transformshift{3.787427in}{2.278556in}%
\pgfsys@useobject{currentmarker}{}%
\end{pgfscope}%
\end{pgfscope}%
\begin{pgfscope}%
\definecolor{textcolor}{rgb}{0.000000,0.000000,0.000000}%
\pgfsetstrokecolor{textcolor}%
\pgfsetfillcolor{textcolor}%
\pgftext[x=3.601840in, y=2.225794in, left, base]{\color{textcolor}{\sffamily\fontsize{10.000000}{12.000000}\selectfont\catcode`\^=\active\def^{\ifmmode\sp\else\^{}\fi}\catcode`\%=\active\def%{\%}0}}%
\end{pgfscope}%
\begin{pgfscope}%
\pgfsetbuttcap%
\pgfsetroundjoin%
\definecolor{currentfill}{rgb}{0.000000,0.000000,0.000000}%
\pgfsetfillcolor{currentfill}%
\pgfsetlinewidth{0.803000pt}%
\definecolor{currentstroke}{rgb}{0.000000,0.000000,0.000000}%
\pgfsetstrokecolor{currentstroke}%
\pgfsetdash{}{0pt}%
\pgfsys@defobject{currentmarker}{\pgfqpoint{-0.048611in}{0.000000in}}{\pgfqpoint{-0.000000in}{0.000000in}}{%
\pgfpathmoveto{\pgfqpoint{-0.000000in}{0.000000in}}%
\pgfpathlineto{\pgfqpoint{-0.048611in}{0.000000in}}%
\pgfusepath{stroke,fill}%
}%
\begin{pgfscope}%
\pgfsys@transformshift{3.787427in}{2.747951in}%
\pgfsys@useobject{currentmarker}{}%
\end{pgfscope}%
\end{pgfscope}%
\begin{pgfscope}%
\definecolor{textcolor}{rgb}{0.000000,0.000000,0.000000}%
\pgfsetstrokecolor{textcolor}%
\pgfsetfillcolor{textcolor}%
\pgftext[x=3.248378in, y=2.695189in, left, base]{\color{textcolor}{\sffamily\fontsize{10.000000}{12.000000}\selectfont\catcode`\^=\active\def^{\ifmmode\sp\else\^{}\fi}\catcode`\%=\active\def%{\%}50000}}%
\end{pgfscope}%
\begin{pgfscope}%
\pgfsetbuttcap%
\pgfsetroundjoin%
\definecolor{currentfill}{rgb}{0.000000,0.000000,0.000000}%
\pgfsetfillcolor{currentfill}%
\pgfsetlinewidth{0.803000pt}%
\definecolor{currentstroke}{rgb}{0.000000,0.000000,0.000000}%
\pgfsetstrokecolor{currentstroke}%
\pgfsetdash{}{0pt}%
\pgfsys@defobject{currentmarker}{\pgfqpoint{-0.048611in}{0.000000in}}{\pgfqpoint{-0.000000in}{0.000000in}}{%
\pgfpathmoveto{\pgfqpoint{-0.000000in}{0.000000in}}%
\pgfpathlineto{\pgfqpoint{-0.048611in}{0.000000in}}%
\pgfusepath{stroke,fill}%
}%
\begin{pgfscope}%
\pgfsys@transformshift{3.787427in}{3.217346in}%
\pgfsys@useobject{currentmarker}{}%
\end{pgfscope}%
\end{pgfscope}%
\begin{pgfscope}%
\definecolor{textcolor}{rgb}{0.000000,0.000000,0.000000}%
\pgfsetstrokecolor{textcolor}%
\pgfsetfillcolor{textcolor}%
\pgftext[x=3.160013in, y=3.164584in, left, base]{\color{textcolor}{\sffamily\fontsize{10.000000}{12.000000}\selectfont\catcode`\^=\active\def^{\ifmmode\sp\else\^{}\fi}\catcode`\%=\active\def%{\%}100000}}%
\end{pgfscope}%
\begin{pgfscope}%
\pgfsetrectcap%
\pgfsetmiterjoin%
\pgfsetlinewidth{0.803000pt}%
\definecolor{currentstroke}{rgb}{0.000000,0.000000,0.000000}%
\pgfsetstrokecolor{currentstroke}%
\pgfsetdash{}{0pt}%
\pgfpathmoveto{\pgfqpoint{3.787427in}{2.278556in}}%
\pgfpathlineto{\pgfqpoint{3.787427in}{3.569333in}}%
\pgfusepath{stroke}%
\end{pgfscope}%
\begin{pgfscope}%
\pgfsetrectcap%
\pgfsetmiterjoin%
\pgfsetlinewidth{0.803000pt}%
\definecolor{currentstroke}{rgb}{0.000000,0.000000,0.000000}%
\pgfsetstrokecolor{currentstroke}%
\pgfsetdash{}{0pt}%
\pgfpathmoveto{\pgfqpoint{5.840250in}{2.278556in}}%
\pgfpathlineto{\pgfqpoint{5.840250in}{3.569333in}}%
\pgfusepath{stroke}%
\end{pgfscope}%
\begin{pgfscope}%
\pgfsetrectcap%
\pgfsetmiterjoin%
\pgfsetlinewidth{0.803000pt}%
\definecolor{currentstroke}{rgb}{0.000000,0.000000,0.000000}%
\pgfsetstrokecolor{currentstroke}%
\pgfsetdash{}{0pt}%
\pgfpathmoveto{\pgfqpoint{3.787427in}{2.278556in}}%
\pgfpathlineto{\pgfqpoint{5.840250in}{2.278556in}}%
\pgfusepath{stroke}%
\end{pgfscope}%
\begin{pgfscope}%
\pgfsetrectcap%
\pgfsetmiterjoin%
\pgfsetlinewidth{0.803000pt}%
\definecolor{currentstroke}{rgb}{0.000000,0.000000,0.000000}%
\pgfsetstrokecolor{currentstroke}%
\pgfsetdash{}{0pt}%
\pgfpathmoveto{\pgfqpoint{3.787427in}{3.569333in}}%
\pgfpathlineto{\pgfqpoint{5.840250in}{3.569333in}}%
\pgfusepath{stroke}%
\end{pgfscope}%
\begin{pgfscope}%
\definecolor{textcolor}{rgb}{0.000000,0.000000,0.000000}%
\pgfsetstrokecolor{textcolor}%
\pgfsetfillcolor{textcolor}%
\pgftext[x=4.295449in,y=3.354204in,,bottom]{\color{textcolor}{\sffamily\fontsize{10.000000}{12.000000}\selectfont\catcode`\^=\active\def^{\ifmmode\sp\else\^{}\fi}\catcode`\%=\active\def%{\%}114578.1}}%
\end{pgfscope}%
\begin{pgfscope}%
\definecolor{textcolor}{rgb}{0.000000,0.000000,0.000000}%
\pgfsetstrokecolor{textcolor}%
\pgfsetfillcolor{textcolor}%
\pgftext[x=5.332228in,y=3.347777in,,bottom]{\color{textcolor}{\sffamily\fontsize{10.000000}{12.000000}\selectfont\catcode`\^=\active\def^{\ifmmode\sp\else\^{}\fi}\catcode`\%=\active\def%{\%}113893.504}}%
\end{pgfscope}%
\begin{pgfscope}%
\definecolor{textcolor}{rgb}{0.000000,0.000000,0.000000}%
\pgfsetstrokecolor{textcolor}%
\pgfsetfillcolor{textcolor}%
\pgftext[x=4.813839in,y=3.652667in,,base]{\color{textcolor}{\sffamily\fontsize{12.000000}{14.400000}\selectfont\catcode`\^=\active\def^{\ifmmode\sp\else\^{}\fi}\catcode`\%=\active\def%{\%}średni czas obliczeń}}%
\end{pgfscope}%
\begin{pgfscope}%
\pgfsetbuttcap%
\pgfsetmiterjoin%
\definecolor{currentfill}{rgb}{1.000000,1.000000,1.000000}%
\pgfsetfillcolor{currentfill}%
\pgfsetlinewidth{0.000000pt}%
\definecolor{currentstroke}{rgb}{0.000000,0.000000,0.000000}%
\pgfsetstrokecolor{currentstroke}%
\pgfsetstrokeopacity{0.000000}%
\pgfsetdash{}{0pt}%
\pgfpathmoveto{\pgfqpoint{0.956181in}{0.387222in}}%
\pgfpathlineto{\pgfqpoint{3.009004in}{0.387222in}}%
\pgfpathlineto{\pgfqpoint{3.009004in}{1.678000in}}%
\pgfpathlineto{\pgfqpoint{0.956181in}{1.678000in}}%
\pgfpathlineto{\pgfqpoint{0.956181in}{0.387222in}}%
\pgfpathclose%
\pgfusepath{fill}%
\end{pgfscope}%
\begin{pgfscope}%
\pgfpathrectangle{\pgfqpoint{0.956181in}{0.387222in}}{\pgfqpoint{2.052823in}{1.290778in}}%
\pgfusepath{clip}%
\pgfsetbuttcap%
\pgfsetmiterjoin%
\definecolor{currentfill}{rgb}{1.000000,0.000000,0.000000}%
\pgfsetfillcolor{currentfill}%
\pgfsetlinewidth{0.000000pt}%
\definecolor{currentstroke}{rgb}{0.000000,0.000000,0.000000}%
\pgfsetstrokecolor{currentstroke}%
\pgfsetstrokeopacity{0.000000}%
\pgfsetdash{}{0pt}%
\pgfpathmoveto{\pgfqpoint{1.049492in}{1.032611in}}%
\pgfpathlineto{\pgfqpoint{1.878915in}{1.032611in}}%
\pgfpathlineto{\pgfqpoint{1.878915in}{1.032611in}}%
\pgfpathlineto{\pgfqpoint{1.049492in}{1.032611in}}%
\pgfpathlineto{\pgfqpoint{1.049492in}{1.032611in}}%
\pgfpathclose%
\pgfusepath{fill}%
\end{pgfscope}%
\begin{pgfscope}%
\pgfpathrectangle{\pgfqpoint{0.956181in}{0.387222in}}{\pgfqpoint{2.052823in}{1.290778in}}%
\pgfusepath{clip}%
\pgfsetbuttcap%
\pgfsetmiterjoin%
\definecolor{currentfill}{rgb}{0.000000,0.000000,1.000000}%
\pgfsetfillcolor{currentfill}%
\pgfsetlinewidth{0.000000pt}%
\definecolor{currentstroke}{rgb}{0.000000,0.000000,0.000000}%
\pgfsetstrokecolor{currentstroke}%
\pgfsetstrokeopacity{0.000000}%
\pgfsetdash{}{0pt}%
\pgfpathmoveto{\pgfqpoint{2.086271in}{1.032611in}}%
\pgfpathlineto{\pgfqpoint{2.915694in}{1.032611in}}%
\pgfpathlineto{\pgfqpoint{2.915694in}{1.032168in}}%
\pgfpathlineto{\pgfqpoint{2.086271in}{1.032168in}}%
\pgfpathlineto{\pgfqpoint{2.086271in}{1.032611in}}%
\pgfpathclose%
\pgfusepath{fill}%
\end{pgfscope}%
\begin{pgfscope}%
\pgfsetbuttcap%
\pgfsetroundjoin%
\definecolor{currentfill}{rgb}{0.000000,0.000000,0.000000}%
\pgfsetfillcolor{currentfill}%
\pgfsetlinewidth{0.803000pt}%
\definecolor{currentstroke}{rgb}{0.000000,0.000000,0.000000}%
\pgfsetstrokecolor{currentstroke}%
\pgfsetdash{}{0pt}%
\pgfsys@defobject{currentmarker}{\pgfqpoint{0.000000in}{-0.048611in}}{\pgfqpoint{0.000000in}{0.000000in}}{%
\pgfpathmoveto{\pgfqpoint{0.000000in}{0.000000in}}%
\pgfpathlineto{\pgfqpoint{0.000000in}{-0.048611in}}%
\pgfusepath{stroke,fill}%
}%
\begin{pgfscope}%
\pgfsys@transformshift{1.464203in}{0.387222in}%
\pgfsys@useobject{currentmarker}{}%
\end{pgfscope}%
\end{pgfscope}%
\begin{pgfscope}%
\definecolor{textcolor}{rgb}{0.000000,0.000000,0.000000}%
\pgfsetstrokecolor{textcolor}%
\pgfsetfillcolor{textcolor}%
\pgftext[x=1.464203in,y=0.290000in,,top]{\color{textcolor}{\sffamily\fontsize{10.000000}{12.000000}\selectfont\catcode`\^=\active\def^{\ifmmode\sp\else\^{}\fi}\catcode`\%=\active\def%{\%}CPU 1024}}%
\end{pgfscope}%
\begin{pgfscope}%
\pgfsetbuttcap%
\pgfsetroundjoin%
\definecolor{currentfill}{rgb}{0.000000,0.000000,0.000000}%
\pgfsetfillcolor{currentfill}%
\pgfsetlinewidth{0.803000pt}%
\definecolor{currentstroke}{rgb}{0.000000,0.000000,0.000000}%
\pgfsetstrokecolor{currentstroke}%
\pgfsetdash{}{0pt}%
\pgfsys@defobject{currentmarker}{\pgfqpoint{0.000000in}{-0.048611in}}{\pgfqpoint{0.000000in}{0.000000in}}{%
\pgfpathmoveto{\pgfqpoint{0.000000in}{0.000000in}}%
\pgfpathlineto{\pgfqpoint{0.000000in}{-0.048611in}}%
\pgfusepath{stroke,fill}%
}%
\begin{pgfscope}%
\pgfsys@transformshift{2.500983in}{0.387222in}%
\pgfsys@useobject{currentmarker}{}%
\end{pgfscope}%
\end{pgfscope}%
\begin{pgfscope}%
\definecolor{textcolor}{rgb}{0.000000,0.000000,0.000000}%
\pgfsetstrokecolor{textcolor}%
\pgfsetfillcolor{textcolor}%
\pgftext[x=2.500983in,y=0.290000in,,top]{\color{textcolor}{\sffamily\fontsize{10.000000}{12.000000}\selectfont\catcode`\^=\active\def^{\ifmmode\sp\else\^{}\fi}\catcode`\%=\active\def%{\%}GPU 1024}}%
\end{pgfscope}%
\begin{pgfscope}%
\pgfsetbuttcap%
\pgfsetroundjoin%
\definecolor{currentfill}{rgb}{0.000000,0.000000,0.000000}%
\pgfsetfillcolor{currentfill}%
\pgfsetlinewidth{0.803000pt}%
\definecolor{currentstroke}{rgb}{0.000000,0.000000,0.000000}%
\pgfsetstrokecolor{currentstroke}%
\pgfsetdash{}{0pt}%
\pgfsys@defobject{currentmarker}{\pgfqpoint{-0.048611in}{0.000000in}}{\pgfqpoint{-0.000000in}{0.000000in}}{%
\pgfpathmoveto{\pgfqpoint{-0.000000in}{0.000000in}}%
\pgfpathlineto{\pgfqpoint{-0.048611in}{0.000000in}}%
\pgfusepath{stroke,fill}%
}%
\begin{pgfscope}%
\pgfsys@transformshift{0.956181in}{0.387222in}%
\pgfsys@useobject{currentmarker}{}%
\end{pgfscope}%
\end{pgfscope}%
\begin{pgfscope}%
\definecolor{textcolor}{rgb}{0.000000,0.000000,0.000000}%
\pgfsetstrokecolor{textcolor}%
\pgfsetfillcolor{textcolor}%
\pgftext[x=0.353324in, y=0.334461in, left, base]{\color{textcolor}{\sffamily\fontsize{10.000000}{12.000000}\selectfont\catcode`\^=\active\def^{\ifmmode\sp\else\^{}\fi}\catcode`\%=\active\def%{\%}\ensuremath{-}0.050}}%
\end{pgfscope}%
\begin{pgfscope}%
\pgfsetbuttcap%
\pgfsetroundjoin%
\definecolor{currentfill}{rgb}{0.000000,0.000000,0.000000}%
\pgfsetfillcolor{currentfill}%
\pgfsetlinewidth{0.803000pt}%
\definecolor{currentstroke}{rgb}{0.000000,0.000000,0.000000}%
\pgfsetstrokecolor{currentstroke}%
\pgfsetdash{}{0pt}%
\pgfsys@defobject{currentmarker}{\pgfqpoint{-0.048611in}{0.000000in}}{\pgfqpoint{-0.000000in}{0.000000in}}{%
\pgfpathmoveto{\pgfqpoint{-0.000000in}{0.000000in}}%
\pgfpathlineto{\pgfqpoint{-0.048611in}{0.000000in}}%
\pgfusepath{stroke,fill}%
}%
\begin{pgfscope}%
\pgfsys@transformshift{0.956181in}{0.709917in}%
\pgfsys@useobject{currentmarker}{}%
\end{pgfscope}%
\end{pgfscope}%
\begin{pgfscope}%
\definecolor{textcolor}{rgb}{0.000000,0.000000,0.000000}%
\pgfsetstrokecolor{textcolor}%
\pgfsetfillcolor{textcolor}%
\pgftext[x=0.353324in, y=0.657155in, left, base]{\color{textcolor}{\sffamily\fontsize{10.000000}{12.000000}\selectfont\catcode`\^=\active\def^{\ifmmode\sp\else\^{}\fi}\catcode`\%=\active\def%{\%}\ensuremath{-}0.025}}%
\end{pgfscope}%
\begin{pgfscope}%
\pgfsetbuttcap%
\pgfsetroundjoin%
\definecolor{currentfill}{rgb}{0.000000,0.000000,0.000000}%
\pgfsetfillcolor{currentfill}%
\pgfsetlinewidth{0.803000pt}%
\definecolor{currentstroke}{rgb}{0.000000,0.000000,0.000000}%
\pgfsetstrokecolor{currentstroke}%
\pgfsetdash{}{0pt}%
\pgfsys@defobject{currentmarker}{\pgfqpoint{-0.048611in}{0.000000in}}{\pgfqpoint{-0.000000in}{0.000000in}}{%
\pgfpathmoveto{\pgfqpoint{-0.000000in}{0.000000in}}%
\pgfpathlineto{\pgfqpoint{-0.048611in}{0.000000in}}%
\pgfusepath{stroke,fill}%
}%
\begin{pgfscope}%
\pgfsys@transformshift{0.956181in}{1.032611in}%
\pgfsys@useobject{currentmarker}{}%
\end{pgfscope}%
\end{pgfscope}%
\begin{pgfscope}%
\definecolor{textcolor}{rgb}{0.000000,0.000000,0.000000}%
\pgfsetstrokecolor{textcolor}%
\pgfsetfillcolor{textcolor}%
\pgftext[x=0.461349in, y=0.979850in, left, base]{\color{textcolor}{\sffamily\fontsize{10.000000}{12.000000}\selectfont\catcode`\^=\active\def^{\ifmmode\sp\else\^{}\fi}\catcode`\%=\active\def%{\%}0.000}}%
\end{pgfscope}%
\begin{pgfscope}%
\pgfsetbuttcap%
\pgfsetroundjoin%
\definecolor{currentfill}{rgb}{0.000000,0.000000,0.000000}%
\pgfsetfillcolor{currentfill}%
\pgfsetlinewidth{0.803000pt}%
\definecolor{currentstroke}{rgb}{0.000000,0.000000,0.000000}%
\pgfsetstrokecolor{currentstroke}%
\pgfsetdash{}{0pt}%
\pgfsys@defobject{currentmarker}{\pgfqpoint{-0.048611in}{0.000000in}}{\pgfqpoint{-0.000000in}{0.000000in}}{%
\pgfpathmoveto{\pgfqpoint{-0.000000in}{0.000000in}}%
\pgfpathlineto{\pgfqpoint{-0.048611in}{0.000000in}}%
\pgfusepath{stroke,fill}%
}%
\begin{pgfscope}%
\pgfsys@transformshift{0.956181in}{1.355306in}%
\pgfsys@useobject{currentmarker}{}%
\end{pgfscope}%
\end{pgfscope}%
\begin{pgfscope}%
\definecolor{textcolor}{rgb}{0.000000,0.000000,0.000000}%
\pgfsetstrokecolor{textcolor}%
\pgfsetfillcolor{textcolor}%
\pgftext[x=0.461349in, y=1.302544in, left, base]{\color{textcolor}{\sffamily\fontsize{10.000000}{12.000000}\selectfont\catcode`\^=\active\def^{\ifmmode\sp\else\^{}\fi}\catcode`\%=\active\def%{\%}0.025}}%
\end{pgfscope}%
\begin{pgfscope}%
\pgfsetbuttcap%
\pgfsetroundjoin%
\definecolor{currentfill}{rgb}{0.000000,0.000000,0.000000}%
\pgfsetfillcolor{currentfill}%
\pgfsetlinewidth{0.803000pt}%
\definecolor{currentstroke}{rgb}{0.000000,0.000000,0.000000}%
\pgfsetstrokecolor{currentstroke}%
\pgfsetdash{}{0pt}%
\pgfsys@defobject{currentmarker}{\pgfqpoint{-0.048611in}{0.000000in}}{\pgfqpoint{-0.000000in}{0.000000in}}{%
\pgfpathmoveto{\pgfqpoint{-0.000000in}{0.000000in}}%
\pgfpathlineto{\pgfqpoint{-0.048611in}{0.000000in}}%
\pgfusepath{stroke,fill}%
}%
\begin{pgfscope}%
\pgfsys@transformshift{0.956181in}{1.678000in}%
\pgfsys@useobject{currentmarker}{}%
\end{pgfscope}%
\end{pgfscope}%
\begin{pgfscope}%
\definecolor{textcolor}{rgb}{0.000000,0.000000,0.000000}%
\pgfsetstrokecolor{textcolor}%
\pgfsetfillcolor{textcolor}%
\pgftext[x=0.461349in, y=1.625238in, left, base]{\color{textcolor}{\sffamily\fontsize{10.000000}{12.000000}\selectfont\catcode`\^=\active\def^{\ifmmode\sp\else\^{}\fi}\catcode`\%=\active\def%{\%}0.050}}%
\end{pgfscope}%
\begin{pgfscope}%
\definecolor{textcolor}{rgb}{0.000000,0.000000,0.000000}%
\pgfsetstrokecolor{textcolor}%
\pgfsetfillcolor{textcolor}%
\pgftext[x=0.297769in,y=1.032611in,,bottom,rotate=90.000000]{\color{textcolor}{\sffamily\fontsize{10.000000}{12.000000}\selectfont\catcode`\^=\active\def^{\ifmmode\sp\else\^{}\fi}\catcode`\%=\active\def%{\%}czas (ns)}}%
\end{pgfscope}%
\begin{pgfscope}%
\pgfsetrectcap%
\pgfsetmiterjoin%
\pgfsetlinewidth{0.803000pt}%
\definecolor{currentstroke}{rgb}{0.000000,0.000000,0.000000}%
\pgfsetstrokecolor{currentstroke}%
\pgfsetdash{}{0pt}%
\pgfpathmoveto{\pgfqpoint{0.956181in}{0.387222in}}%
\pgfpathlineto{\pgfqpoint{0.956181in}{1.678000in}}%
\pgfusepath{stroke}%
\end{pgfscope}%
\begin{pgfscope}%
\pgfsetrectcap%
\pgfsetmiterjoin%
\pgfsetlinewidth{0.803000pt}%
\definecolor{currentstroke}{rgb}{0.000000,0.000000,0.000000}%
\pgfsetstrokecolor{currentstroke}%
\pgfsetdash{}{0pt}%
\pgfpathmoveto{\pgfqpoint{3.009004in}{0.387222in}}%
\pgfpathlineto{\pgfqpoint{3.009004in}{1.678000in}}%
\pgfusepath{stroke}%
\end{pgfscope}%
\begin{pgfscope}%
\pgfsetrectcap%
\pgfsetmiterjoin%
\pgfsetlinewidth{0.803000pt}%
\definecolor{currentstroke}{rgb}{0.000000,0.000000,0.000000}%
\pgfsetstrokecolor{currentstroke}%
\pgfsetdash{}{0pt}%
\pgfpathmoveto{\pgfqpoint{0.956181in}{0.387222in}}%
\pgfpathlineto{\pgfqpoint{3.009004in}{0.387222in}}%
\pgfusepath{stroke}%
\end{pgfscope}%
\begin{pgfscope}%
\pgfsetrectcap%
\pgfsetmiterjoin%
\pgfsetlinewidth{0.803000pt}%
\definecolor{currentstroke}{rgb}{0.000000,0.000000,0.000000}%
\pgfsetstrokecolor{currentstroke}%
\pgfsetdash{}{0pt}%
\pgfpathmoveto{\pgfqpoint{0.956181in}{1.678000in}}%
\pgfpathlineto{\pgfqpoint{3.009004in}{1.678000in}}%
\pgfusepath{stroke}%
\end{pgfscope}%
\begin{pgfscope}%
\definecolor{textcolor}{rgb}{0.000000,0.000000,0.000000}%
\pgfsetstrokecolor{textcolor}%
\pgfsetfillcolor{textcolor}%
\pgftext[x=1.464203in,y=1.032611in,,bottom]{\color{textcolor}{\sffamily\fontsize{10.000000}{12.000000}\selectfont\catcode`\^=\active\def^{\ifmmode\sp\else\^{}\fi}\catcode`\%=\active\def%{\%}0.0}}%
\end{pgfscope}%
\begin{pgfscope}%
\definecolor{textcolor}{rgb}{0.000000,0.000000,0.000000}%
\pgfsetstrokecolor{textcolor}%
\pgfsetfillcolor{textcolor}%
\pgftext[x=2.500983in,y=1.032168in,,bottom]{\color{textcolor}{\sffamily\fontsize{10.000000}{12.000000}\selectfont\catcode`\^=\active\def^{\ifmmode\sp\else\^{}\fi}\catcode`\%=\active\def%{\%}-0.0}}%
\end{pgfscope}%
\begin{pgfscope}%
\definecolor{textcolor}{rgb}{0.000000,0.000000,0.000000}%
\pgfsetstrokecolor{textcolor}%
\pgfsetfillcolor{textcolor}%
\pgftext[x=1.982593in,y=1.761333in,,base]{\color{textcolor}{\sffamily\fontsize{12.000000}{14.400000}\selectfont\catcode`\^=\active\def^{\ifmmode\sp\else\^{}\fi}\catcode`\%=\active\def%{\%}średnie opóźnienie}}%
\end{pgfscope}%
\begin{pgfscope}%
\pgfsetbuttcap%
\pgfsetmiterjoin%
\definecolor{currentfill}{rgb}{1.000000,1.000000,1.000000}%
\pgfsetfillcolor{currentfill}%
\pgfsetlinewidth{0.000000pt}%
\definecolor{currentstroke}{rgb}{0.000000,0.000000,0.000000}%
\pgfsetstrokecolor{currentstroke}%
\pgfsetstrokeopacity{0.000000}%
\pgfsetdash{}{0pt}%
\pgfpathmoveto{\pgfqpoint{3.787427in}{0.387222in}}%
\pgfpathlineto{\pgfqpoint{5.840250in}{0.387222in}}%
\pgfpathlineto{\pgfqpoint{5.840250in}{1.678000in}}%
\pgfpathlineto{\pgfqpoint{3.787427in}{1.678000in}}%
\pgfpathlineto{\pgfqpoint{3.787427in}{0.387222in}}%
\pgfpathclose%
\pgfusepath{fill}%
\end{pgfscope}%
\begin{pgfscope}%
\pgfpathrectangle{\pgfqpoint{3.787427in}{0.387222in}}{\pgfqpoint{2.052823in}{1.290778in}}%
\pgfusepath{clip}%
\pgfsetbuttcap%
\pgfsetmiterjoin%
\definecolor{currentfill}{rgb}{1.000000,0.000000,0.000000}%
\pgfsetfillcolor{currentfill}%
\pgfsetlinewidth{0.000000pt}%
\definecolor{currentstroke}{rgb}{0.000000,0.000000,0.000000}%
\pgfsetstrokecolor{currentstroke}%
\pgfsetstrokeopacity{0.000000}%
\pgfsetdash{}{0pt}%
\pgfpathmoveto{\pgfqpoint{3.880737in}{0.683831in}}%
\pgfpathlineto{\pgfqpoint{4.710161in}{0.683831in}}%
\pgfpathlineto{\pgfqpoint{4.710161in}{1.512305in}}%
\pgfpathlineto{\pgfqpoint{3.880737in}{1.512305in}}%
\pgfpathlineto{\pgfqpoint{3.880737in}{0.683831in}}%
\pgfpathclose%
\pgfusepath{fill}%
\end{pgfscope}%
\begin{pgfscope}%
\pgfpathrectangle{\pgfqpoint{3.787427in}{0.387222in}}{\pgfqpoint{2.052823in}{1.290778in}}%
\pgfusepath{clip}%
\pgfsetbuttcap%
\pgfsetmiterjoin%
\definecolor{currentfill}{rgb}{0.000000,0.000000,1.000000}%
\pgfsetfillcolor{currentfill}%
\pgfsetlinewidth{0.000000pt}%
\definecolor{currentstroke}{rgb}{0.000000,0.000000,0.000000}%
\pgfsetstrokecolor{currentstroke}%
\pgfsetstrokeopacity{0.000000}%
\pgfsetdash{}{0pt}%
\pgfpathmoveto{\pgfqpoint{4.917516in}{0.683831in}}%
\pgfpathlineto{\pgfqpoint{5.746940in}{0.683831in}}%
\pgfpathlineto{\pgfqpoint{5.746940in}{0.535527in}}%
\pgfpathlineto{\pgfqpoint{4.917516in}{0.535527in}}%
\pgfpathlineto{\pgfqpoint{4.917516in}{0.683831in}}%
\pgfpathclose%
\pgfusepath{fill}%
\end{pgfscope}%
\begin{pgfscope}%
\pgfsetbuttcap%
\pgfsetroundjoin%
\definecolor{currentfill}{rgb}{0.000000,0.000000,0.000000}%
\pgfsetfillcolor{currentfill}%
\pgfsetlinewidth{0.803000pt}%
\definecolor{currentstroke}{rgb}{0.000000,0.000000,0.000000}%
\pgfsetstrokecolor{currentstroke}%
\pgfsetdash{}{0pt}%
\pgfsys@defobject{currentmarker}{\pgfqpoint{0.000000in}{-0.048611in}}{\pgfqpoint{0.000000in}{0.000000in}}{%
\pgfpathmoveto{\pgfqpoint{0.000000in}{0.000000in}}%
\pgfpathlineto{\pgfqpoint{0.000000in}{-0.048611in}}%
\pgfusepath{stroke,fill}%
}%
\begin{pgfscope}%
\pgfsys@transformshift{4.295449in}{0.387222in}%
\pgfsys@useobject{currentmarker}{}%
\end{pgfscope}%
\end{pgfscope}%
\begin{pgfscope}%
\definecolor{textcolor}{rgb}{0.000000,0.000000,0.000000}%
\pgfsetstrokecolor{textcolor}%
\pgfsetfillcolor{textcolor}%
\pgftext[x=4.295449in,y=0.290000in,,top]{\color{textcolor}{\sffamily\fontsize{10.000000}{12.000000}\selectfont\catcode`\^=\active\def^{\ifmmode\sp\else\^{}\fi}\catcode`\%=\active\def%{\%}CPU 8192}}%
\end{pgfscope}%
\begin{pgfscope}%
\pgfsetbuttcap%
\pgfsetroundjoin%
\definecolor{currentfill}{rgb}{0.000000,0.000000,0.000000}%
\pgfsetfillcolor{currentfill}%
\pgfsetlinewidth{0.803000pt}%
\definecolor{currentstroke}{rgb}{0.000000,0.000000,0.000000}%
\pgfsetstrokecolor{currentstroke}%
\pgfsetdash{}{0pt}%
\pgfsys@defobject{currentmarker}{\pgfqpoint{0.000000in}{-0.048611in}}{\pgfqpoint{0.000000in}{0.000000in}}{%
\pgfpathmoveto{\pgfqpoint{0.000000in}{0.000000in}}%
\pgfpathlineto{\pgfqpoint{0.000000in}{-0.048611in}}%
\pgfusepath{stroke,fill}%
}%
\begin{pgfscope}%
\pgfsys@transformshift{5.332228in}{0.387222in}%
\pgfsys@useobject{currentmarker}{}%
\end{pgfscope}%
\end{pgfscope}%
\begin{pgfscope}%
\definecolor{textcolor}{rgb}{0.000000,0.000000,0.000000}%
\pgfsetstrokecolor{textcolor}%
\pgfsetfillcolor{textcolor}%
\pgftext[x=5.332228in,y=0.290000in,,top]{\color{textcolor}{\sffamily\fontsize{10.000000}{12.000000}\selectfont\catcode`\^=\active\def^{\ifmmode\sp\else\^{}\fi}\catcode`\%=\active\def%{\%}GPU 8192}}%
\end{pgfscope}%
\begin{pgfscope}%
\pgfsetbuttcap%
\pgfsetroundjoin%
\definecolor{currentfill}{rgb}{0.000000,0.000000,0.000000}%
\pgfsetfillcolor{currentfill}%
\pgfsetlinewidth{0.803000pt}%
\definecolor{currentstroke}{rgb}{0.000000,0.000000,0.000000}%
\pgfsetstrokecolor{currentstroke}%
\pgfsetdash{}{0pt}%
\pgfsys@defobject{currentmarker}{\pgfqpoint{-0.048611in}{0.000000in}}{\pgfqpoint{-0.000000in}{0.000000in}}{%
\pgfpathmoveto{\pgfqpoint{-0.000000in}{0.000000in}}%
\pgfpathlineto{\pgfqpoint{-0.048611in}{0.000000in}}%
\pgfusepath{stroke,fill}%
}%
\begin{pgfscope}%
\pgfsys@transformshift{3.787427in}{0.683831in}%
\pgfsys@useobject{currentmarker}{}%
\end{pgfscope}%
\end{pgfscope}%
\begin{pgfscope}%
\definecolor{textcolor}{rgb}{0.000000,0.000000,0.000000}%
\pgfsetstrokecolor{textcolor}%
\pgfsetfillcolor{textcolor}%
\pgftext[x=3.601840in, y=0.631070in, left, base]{\color{textcolor}{\sffamily\fontsize{10.000000}{12.000000}\selectfont\catcode`\^=\active\def^{\ifmmode\sp\else\^{}\fi}\catcode`\%=\active\def%{\%}0}}%
\end{pgfscope}%
\begin{pgfscope}%
\pgfsetbuttcap%
\pgfsetroundjoin%
\definecolor{currentfill}{rgb}{0.000000,0.000000,0.000000}%
\pgfsetfillcolor{currentfill}%
\pgfsetlinewidth{0.803000pt}%
\definecolor{currentstroke}{rgb}{0.000000,0.000000,0.000000}%
\pgfsetstrokecolor{currentstroke}%
\pgfsetdash{}{0pt}%
\pgfsys@defobject{currentmarker}{\pgfqpoint{-0.048611in}{0.000000in}}{\pgfqpoint{-0.000000in}{0.000000in}}{%
\pgfpathmoveto{\pgfqpoint{-0.000000in}{0.000000in}}%
\pgfpathlineto{\pgfqpoint{-0.048611in}{0.000000in}}%
\pgfusepath{stroke,fill}%
}%
\begin{pgfscope}%
\pgfsys@transformshift{3.787427in}{1.292240in}%
\pgfsys@useobject{currentmarker}{}%
\end{pgfscope}%
\end{pgfscope}%
\begin{pgfscope}%
\definecolor{textcolor}{rgb}{0.000000,0.000000,0.000000}%
\pgfsetstrokecolor{textcolor}%
\pgfsetfillcolor{textcolor}%
\pgftext[x=3.513474in, y=1.239478in, left, base]{\color{textcolor}{\sffamily\fontsize{10.000000}{12.000000}\selectfont\catcode`\^=\active\def^{\ifmmode\sp\else\^{}\fi}\catcode`\%=\active\def%{\%}20}}%
\end{pgfscope}%
\begin{pgfscope}%
\pgfsetrectcap%
\pgfsetmiterjoin%
\pgfsetlinewidth{0.803000pt}%
\definecolor{currentstroke}{rgb}{0.000000,0.000000,0.000000}%
\pgfsetstrokecolor{currentstroke}%
\pgfsetdash{}{0pt}%
\pgfpathmoveto{\pgfqpoint{3.787427in}{0.387222in}}%
\pgfpathlineto{\pgfqpoint{3.787427in}{1.678000in}}%
\pgfusepath{stroke}%
\end{pgfscope}%
\begin{pgfscope}%
\pgfsetrectcap%
\pgfsetmiterjoin%
\pgfsetlinewidth{0.803000pt}%
\definecolor{currentstroke}{rgb}{0.000000,0.000000,0.000000}%
\pgfsetstrokecolor{currentstroke}%
\pgfsetdash{}{0pt}%
\pgfpathmoveto{\pgfqpoint{5.840250in}{0.387222in}}%
\pgfpathlineto{\pgfqpoint{5.840250in}{1.678000in}}%
\pgfusepath{stroke}%
\end{pgfscope}%
\begin{pgfscope}%
\pgfsetrectcap%
\pgfsetmiterjoin%
\pgfsetlinewidth{0.803000pt}%
\definecolor{currentstroke}{rgb}{0.000000,0.000000,0.000000}%
\pgfsetstrokecolor{currentstroke}%
\pgfsetdash{}{0pt}%
\pgfpathmoveto{\pgfqpoint{3.787427in}{0.387222in}}%
\pgfpathlineto{\pgfqpoint{5.840250in}{0.387222in}}%
\pgfusepath{stroke}%
\end{pgfscope}%
\begin{pgfscope}%
\pgfsetrectcap%
\pgfsetmiterjoin%
\pgfsetlinewidth{0.803000pt}%
\definecolor{currentstroke}{rgb}{0.000000,0.000000,0.000000}%
\pgfsetstrokecolor{currentstroke}%
\pgfsetdash{}{0pt}%
\pgfpathmoveto{\pgfqpoint{3.787427in}{1.678000in}}%
\pgfpathlineto{\pgfqpoint{5.840250in}{1.678000in}}%
\pgfusepath{stroke}%
\end{pgfscope}%
\begin{pgfscope}%
\definecolor{textcolor}{rgb}{0.000000,0.000000,0.000000}%
\pgfsetstrokecolor{textcolor}%
\pgfsetfillcolor{textcolor}%
\pgftext[x=4.295449in,y=1.512305in,,bottom]{\color{textcolor}{\sffamily\fontsize{10.000000}{12.000000}\selectfont\catcode`\^=\active\def^{\ifmmode\sp\else\^{}\fi}\catcode`\%=\active\def%{\%}27.234}}%
\end{pgfscope}%
\begin{pgfscope}%
\definecolor{textcolor}{rgb}{0.000000,0.000000,0.000000}%
\pgfsetstrokecolor{textcolor}%
\pgfsetfillcolor{textcolor}%
\pgftext[x=5.332228in,y=0.535527in,,top]{\color{textcolor}{\sffamily\fontsize{10.000000}{12.000000}\selectfont\catcode`\^=\active\def^{\ifmmode\sp\else\^{}\fi}\catcode`\%=\active\def%{\%}-4.875}}%
\end{pgfscope}%
\begin{pgfscope}%
\definecolor{textcolor}{rgb}{0.000000,0.000000,0.000000}%
\pgfsetstrokecolor{textcolor}%
\pgfsetfillcolor{textcolor}%
\pgftext[x=4.813839in,y=1.761333in,,base]{\color{textcolor}{\sffamily\fontsize{12.000000}{14.400000}\selectfont\catcode`\^=\active\def^{\ifmmode\sp\else\^{}\fi}\catcode`\%=\active\def%{\%}średni opóźnienie}}%
\end{pgfscope}%
\end{pgfpicture}%
\makeatother%
\endgroup%
}
    \caption{Uśrednione statystyki przetwarzania dźwięku w trybie online}
    \label{fig:Uśrednione statystyki przetwarzania dźwięku w trybie online}
\end{figure}

"Średni czas trwania pętli" przedstawia czas trwania całej pętli przetwarzania. Jako, że częstotliwość próbkowania jest ustawiona na 48kHz, to idealna wartość tej statystyki wynosi 21333ns dla bufora o rozmiarze 1024 próbek oraz 170667ns dla bufora o rozmiarze 8192 próbek. "Średni czas obliczeń" uwzględnia czas poświęcony na wszelkie operacje obliczeniowe, a wyłącza czas spędzony na oczekiwanie na odpowiedni moment na odtworzenie dźwięku (co wynika ze specyfikacji generowania online). Można tu zauważyć jedynie niewielkie różnice między implementacjami. "Średnie opóźnienie" przedstawia czas związany z niedotrzymaniem idealnego czasu trwania pętli. Wartość ta powinna być jak najbliższa 0.

\begin{figure}[H]
    \centering
    \scalebox{1.0}{%% Creator: Matplotlib, PGF backend
%%
%% To include the figure in your LaTeX document, write
%%   \input{<filename>.pgf}
%%
%% Make sure the required packages are loaded in your preamble
%%   \usepackage{pgf}
%%
%% Also ensure that all the required font packages are loaded; for instance,
%% the lmodern package is sometimes necessary when using math font.
%%   \usepackage{lmodern}
%%
%% Figures using additional raster images can only be included by \input if
%% they are in the same directory as the main LaTeX file. For loading figures
%% from other directories you can use the `import` package
%%   \usepackage{import}
%%
%% and then include the figures with
%%   \import{<path to file>}{<filename>.pgf}
%%
%% Matplotlib used the following preamble
%%   \def\mathdefault#1{#1}
%%   \everymath=\expandafter{\the\everymath\displaystyle}
%%   
%%   \usepackage{fontspec}
%%   \setmainfont{DejaVuSerif.ttf}[Path=\detokenize{/usr/lib/python3.12/site-packages/matplotlib/mpl-data/fonts/ttf/}]
%%   \setsansfont{DejaVuSans.ttf}[Path=\detokenize{/usr/lib/python3.12/site-packages/matplotlib/mpl-data/fonts/ttf/}]
%%   \setmonofont{DejaVuSansMono.ttf}[Path=\detokenize{/usr/lib/python3.12/site-packages/matplotlib/mpl-data/fonts/ttf/}]
%%   \makeatletter\@ifpackageloaded{underscore}{}{\usepackage[strings]{underscore}}\makeatother
%%
\begingroup%
\makeatletter%
\begin{pgfpicture}%
\pgfpathrectangle{\pgfpointorigin}{\pgfqpoint{5.990000in}{3.000000in}}%
\pgfusepath{use as bounding box, clip}%
\begin{pgfscope}%
\pgfsetbuttcap%
\pgfsetmiterjoin%
\definecolor{currentfill}{rgb}{1.000000,1.000000,1.000000}%
\pgfsetfillcolor{currentfill}%
\pgfsetlinewidth{0.000000pt}%
\definecolor{currentstroke}{rgb}{1.000000,1.000000,1.000000}%
\pgfsetstrokecolor{currentstroke}%
\pgfsetdash{}{0pt}%
\pgfpathmoveto{\pgfqpoint{0.000000in}{0.000000in}}%
\pgfpathlineto{\pgfqpoint{5.990000in}{0.000000in}}%
\pgfpathlineto{\pgfqpoint{5.990000in}{3.000000in}}%
\pgfpathlineto{\pgfqpoint{0.000000in}{3.000000in}}%
\pgfpathlineto{\pgfqpoint{0.000000in}{0.000000in}}%
\pgfpathclose%
\pgfusepath{fill}%
\end{pgfscope}%
\begin{pgfscope}%
\pgfsetbuttcap%
\pgfsetmiterjoin%
\definecolor{currentfill}{rgb}{1.000000,1.000000,1.000000}%
\pgfsetfillcolor{currentfill}%
\pgfsetlinewidth{0.000000pt}%
\definecolor{currentstroke}{rgb}{0.000000,0.000000,0.000000}%
\pgfsetstrokecolor{currentstroke}%
\pgfsetstrokeopacity{0.000000}%
\pgfsetdash{}{0pt}%
\pgfpathmoveto{\pgfqpoint{0.707846in}{0.549222in}}%
\pgfpathlineto{\pgfqpoint{3.017740in}{0.549222in}}%
\pgfpathlineto{\pgfqpoint{3.017740in}{2.636667in}}%
\pgfpathlineto{\pgfqpoint{0.707846in}{2.636667in}}%
\pgfpathlineto{\pgfqpoint{0.707846in}{0.549222in}}%
\pgfpathclose%
\pgfusepath{fill}%
\end{pgfscope}%
\begin{pgfscope}%
\pgfpathrectangle{\pgfqpoint{0.707846in}{0.549222in}}{\pgfqpoint{2.309894in}{2.087444in}}%
\pgfusepath{clip}%
\pgfsetbuttcap%
\pgfsetmiterjoin%
\definecolor{currentfill}{rgb}{1.000000,0.000000,0.000000}%
\pgfsetfillcolor{currentfill}%
\pgfsetlinewidth{0.000000pt}%
\definecolor{currentstroke}{rgb}{0.000000,0.000000,0.000000}%
\pgfsetstrokecolor{currentstroke}%
\pgfsetstrokeopacity{0.000000}%
\pgfsetdash{}{0pt}%
\pgfpathmoveto{\pgfqpoint{0.812841in}{0.549222in}}%
\pgfpathlineto{\pgfqpoint{1.254926in}{0.549222in}}%
\pgfpathlineto{\pgfqpoint{1.254926in}{1.851555in}}%
\pgfpathlineto{\pgfqpoint{0.812841in}{1.851555in}}%
\pgfpathlineto{\pgfqpoint{0.812841in}{0.549222in}}%
\pgfpathclose%
\pgfusepath{fill}%
\end{pgfscope}%
\begin{pgfscope}%
\pgfpathrectangle{\pgfqpoint{0.707846in}{0.549222in}}{\pgfqpoint{2.309894in}{2.087444in}}%
\pgfusepath{clip}%
\pgfsetbuttcap%
\pgfsetmiterjoin%
\definecolor{currentfill}{rgb}{0.000000,0.000000,1.000000}%
\pgfsetfillcolor{currentfill}%
\pgfsetlinewidth{0.000000pt}%
\definecolor{currentstroke}{rgb}{0.000000,0.000000,0.000000}%
\pgfsetstrokecolor{currentstroke}%
\pgfsetstrokeopacity{0.000000}%
\pgfsetdash{}{0pt}%
\pgfpathmoveto{\pgfqpoint{1.365447in}{0.549222in}}%
\pgfpathlineto{\pgfqpoint{1.807532in}{0.549222in}}%
\pgfpathlineto{\pgfqpoint{1.807532in}{1.919268in}}%
\pgfpathlineto{\pgfqpoint{1.365447in}{1.919268in}}%
\pgfpathlineto{\pgfqpoint{1.365447in}{0.549222in}}%
\pgfpathclose%
\pgfusepath{fill}%
\end{pgfscope}%
\begin{pgfscope}%
\pgfpathrectangle{\pgfqpoint{0.707846in}{0.549222in}}{\pgfqpoint{2.309894in}{2.087444in}}%
\pgfusepath{clip}%
\pgfsetbuttcap%
\pgfsetmiterjoin%
\definecolor{currentfill}{rgb}{1.000000,0.000000,0.000000}%
\pgfsetfillcolor{currentfill}%
\pgfsetlinewidth{0.000000pt}%
\definecolor{currentstroke}{rgb}{0.000000,0.000000,0.000000}%
\pgfsetstrokecolor{currentstroke}%
\pgfsetstrokeopacity{0.000000}%
\pgfsetdash{}{0pt}%
\pgfpathmoveto{\pgfqpoint{1.918054in}{0.549222in}}%
\pgfpathlineto{\pgfqpoint{2.360138in}{0.549222in}}%
\pgfpathlineto{\pgfqpoint{2.360138in}{1.970746in}}%
\pgfpathlineto{\pgfqpoint{1.918054in}{1.970746in}}%
\pgfpathlineto{\pgfqpoint{1.918054in}{0.549222in}}%
\pgfpathclose%
\pgfusepath{fill}%
\end{pgfscope}%
\begin{pgfscope}%
\pgfpathrectangle{\pgfqpoint{0.707846in}{0.549222in}}{\pgfqpoint{2.309894in}{2.087444in}}%
\pgfusepath{clip}%
\pgfsetbuttcap%
\pgfsetmiterjoin%
\definecolor{currentfill}{rgb}{0.000000,0.000000,1.000000}%
\pgfsetfillcolor{currentfill}%
\pgfsetlinewidth{0.000000pt}%
\definecolor{currentstroke}{rgb}{0.000000,0.000000,0.000000}%
\pgfsetstrokecolor{currentstroke}%
\pgfsetstrokeopacity{0.000000}%
\pgfsetdash{}{0pt}%
\pgfpathmoveto{\pgfqpoint{2.470660in}{0.549222in}}%
\pgfpathlineto{\pgfqpoint{2.912745in}{0.549222in}}%
\pgfpathlineto{\pgfqpoint{2.912745in}{1.900228in}}%
\pgfpathlineto{\pgfqpoint{2.470660in}{1.900228in}}%
\pgfpathlineto{\pgfqpoint{2.470660in}{0.549222in}}%
\pgfpathclose%
\pgfusepath{fill}%
\end{pgfscope}%
\begin{pgfscope}%
\pgfsetbuttcap%
\pgfsetroundjoin%
\definecolor{currentfill}{rgb}{0.000000,0.000000,0.000000}%
\pgfsetfillcolor{currentfill}%
\pgfsetlinewidth{0.803000pt}%
\definecolor{currentstroke}{rgb}{0.000000,0.000000,0.000000}%
\pgfsetstrokecolor{currentstroke}%
\pgfsetdash{}{0pt}%
\pgfsys@defobject{currentmarker}{\pgfqpoint{0.000000in}{-0.048611in}}{\pgfqpoint{0.000000in}{0.000000in}}{%
\pgfpathmoveto{\pgfqpoint{0.000000in}{0.000000in}}%
\pgfpathlineto{\pgfqpoint{0.000000in}{-0.048611in}}%
\pgfusepath{stroke,fill}%
}%
\begin{pgfscope}%
\pgfsys@transformshift{1.033884in}{0.549222in}%
\pgfsys@useobject{currentmarker}{}%
\end{pgfscope}%
\end{pgfscope}%
\begin{pgfscope}%
\definecolor{textcolor}{rgb}{0.000000,0.000000,0.000000}%
\pgfsetstrokecolor{textcolor}%
\pgfsetfillcolor{textcolor}%
\pgftext[x=0.892689in, y=0.346477in, left, base]{\color{textcolor}{\sffamily\fontsize{10.000000}{12.000000}\selectfont\catcode`\^=\active\def^{\ifmmode\sp\else\^{}\fi}\catcode`\%=\active\def%{\%}CPU}}%
\end{pgfscope}%
\begin{pgfscope}%
\definecolor{textcolor}{rgb}{0.000000,0.000000,0.000000}%
\pgfsetstrokecolor{textcolor}%
\pgfsetfillcolor{textcolor}%
\pgftext[x=0.857153in, y=0.190960in, left, base]{\color{textcolor}{\sffamily\fontsize{10.000000}{12.000000}\selectfont\catcode`\^=\active\def^{\ifmmode\sp\else\^{}\fi}\catcode`\%=\active\def%{\%}1024}}%
\end{pgfscope}%
\begin{pgfscope}%
\pgfsetbuttcap%
\pgfsetroundjoin%
\definecolor{currentfill}{rgb}{0.000000,0.000000,0.000000}%
\pgfsetfillcolor{currentfill}%
\pgfsetlinewidth{0.803000pt}%
\definecolor{currentstroke}{rgb}{0.000000,0.000000,0.000000}%
\pgfsetstrokecolor{currentstroke}%
\pgfsetdash{}{0pt}%
\pgfsys@defobject{currentmarker}{\pgfqpoint{0.000000in}{-0.048611in}}{\pgfqpoint{0.000000in}{0.000000in}}{%
\pgfpathmoveto{\pgfqpoint{0.000000in}{0.000000in}}%
\pgfpathlineto{\pgfqpoint{0.000000in}{-0.048611in}}%
\pgfusepath{stroke,fill}%
}%
\begin{pgfscope}%
\pgfsys@transformshift{1.586490in}{0.549222in}%
\pgfsys@useobject{currentmarker}{}%
\end{pgfscope}%
\end{pgfscope}%
\begin{pgfscope}%
\definecolor{textcolor}{rgb}{0.000000,0.000000,0.000000}%
\pgfsetstrokecolor{textcolor}%
\pgfsetfillcolor{textcolor}%
\pgftext[x=1.439972in, y=0.346477in, left, base]{\color{textcolor}{\sffamily\fontsize{10.000000}{12.000000}\selectfont\catcode`\^=\active\def^{\ifmmode\sp\else\^{}\fi}\catcode`\%=\active\def%{\%}GPU}}%
\end{pgfscope}%
\begin{pgfscope}%
\definecolor{textcolor}{rgb}{0.000000,0.000000,0.000000}%
\pgfsetstrokecolor{textcolor}%
\pgfsetfillcolor{textcolor}%
\pgftext[x=1.409759in, y=0.190960in, left, base]{\color{textcolor}{\sffamily\fontsize{10.000000}{12.000000}\selectfont\catcode`\^=\active\def^{\ifmmode\sp\else\^{}\fi}\catcode`\%=\active\def%{\%}1024}}%
\end{pgfscope}%
\begin{pgfscope}%
\pgfsetbuttcap%
\pgfsetroundjoin%
\definecolor{currentfill}{rgb}{0.000000,0.000000,0.000000}%
\pgfsetfillcolor{currentfill}%
\pgfsetlinewidth{0.803000pt}%
\definecolor{currentstroke}{rgb}{0.000000,0.000000,0.000000}%
\pgfsetstrokecolor{currentstroke}%
\pgfsetdash{}{0pt}%
\pgfsys@defobject{currentmarker}{\pgfqpoint{0.000000in}{-0.048611in}}{\pgfqpoint{0.000000in}{0.000000in}}{%
\pgfpathmoveto{\pgfqpoint{0.000000in}{0.000000in}}%
\pgfpathlineto{\pgfqpoint{0.000000in}{-0.048611in}}%
\pgfusepath{stroke,fill}%
}%
\begin{pgfscope}%
\pgfsys@transformshift{2.139096in}{0.549222in}%
\pgfsys@useobject{currentmarker}{}%
\end{pgfscope}%
\end{pgfscope}%
\begin{pgfscope}%
\definecolor{textcolor}{rgb}{0.000000,0.000000,0.000000}%
\pgfsetstrokecolor{textcolor}%
\pgfsetfillcolor{textcolor}%
\pgftext[x=1.997901in, y=0.346477in, left, base]{\color{textcolor}{\sffamily\fontsize{10.000000}{12.000000}\selectfont\catcode`\^=\active\def^{\ifmmode\sp\else\^{}\fi}\catcode`\%=\active\def%{\%}CPU}}%
\end{pgfscope}%
\begin{pgfscope}%
\definecolor{textcolor}{rgb}{0.000000,0.000000,0.000000}%
\pgfsetstrokecolor{textcolor}%
\pgfsetfillcolor{textcolor}%
\pgftext[x=1.962365in, y=0.190960in, left, base]{\color{textcolor}{\sffamily\fontsize{10.000000}{12.000000}\selectfont\catcode`\^=\active\def^{\ifmmode\sp\else\^{}\fi}\catcode`\%=\active\def%{\%}8192}}%
\end{pgfscope}%
\begin{pgfscope}%
\pgfsetbuttcap%
\pgfsetroundjoin%
\definecolor{currentfill}{rgb}{0.000000,0.000000,0.000000}%
\pgfsetfillcolor{currentfill}%
\pgfsetlinewidth{0.803000pt}%
\definecolor{currentstroke}{rgb}{0.000000,0.000000,0.000000}%
\pgfsetstrokecolor{currentstroke}%
\pgfsetdash{}{0pt}%
\pgfsys@defobject{currentmarker}{\pgfqpoint{0.000000in}{-0.048611in}}{\pgfqpoint{0.000000in}{0.000000in}}{%
\pgfpathmoveto{\pgfqpoint{0.000000in}{0.000000in}}%
\pgfpathlineto{\pgfqpoint{0.000000in}{-0.048611in}}%
\pgfusepath{stroke,fill}%
}%
\begin{pgfscope}%
\pgfsys@transformshift{2.691702in}{0.549222in}%
\pgfsys@useobject{currentmarker}{}%
\end{pgfscope}%
\end{pgfscope}%
\begin{pgfscope}%
\definecolor{textcolor}{rgb}{0.000000,0.000000,0.000000}%
\pgfsetstrokecolor{textcolor}%
\pgfsetfillcolor{textcolor}%
\pgftext[x=2.545184in, y=0.346477in, left, base]{\color{textcolor}{\sffamily\fontsize{10.000000}{12.000000}\selectfont\catcode`\^=\active\def^{\ifmmode\sp\else\^{}\fi}\catcode`\%=\active\def%{\%}GPU}}%
\end{pgfscope}%
\begin{pgfscope}%
\definecolor{textcolor}{rgb}{0.000000,0.000000,0.000000}%
\pgfsetstrokecolor{textcolor}%
\pgfsetfillcolor{textcolor}%
\pgftext[x=2.514971in, y=0.190960in, left, base]{\color{textcolor}{\sffamily\fontsize{10.000000}{12.000000}\selectfont\catcode`\^=\active\def^{\ifmmode\sp\else\^{}\fi}\catcode`\%=\active\def%{\%}8192}}%
\end{pgfscope}%
\begin{pgfscope}%
\pgfsetbuttcap%
\pgfsetroundjoin%
\definecolor{currentfill}{rgb}{0.000000,0.000000,0.000000}%
\pgfsetfillcolor{currentfill}%
\pgfsetlinewidth{0.803000pt}%
\definecolor{currentstroke}{rgb}{0.000000,0.000000,0.000000}%
\pgfsetstrokecolor{currentstroke}%
\pgfsetdash{}{0pt}%
\pgfsys@defobject{currentmarker}{\pgfqpoint{-0.048611in}{0.000000in}}{\pgfqpoint{-0.000000in}{0.000000in}}{%
\pgfpathmoveto{\pgfqpoint{-0.000000in}{0.000000in}}%
\pgfpathlineto{\pgfqpoint{-0.048611in}{0.000000in}}%
\pgfusepath{stroke,fill}%
}%
\begin{pgfscope}%
\pgfsys@transformshift{0.707846in}{0.549222in}%
\pgfsys@useobject{currentmarker}{}%
\end{pgfscope}%
\end{pgfscope}%
\begin{pgfscope}%
\definecolor{textcolor}{rgb}{0.000000,0.000000,0.000000}%
\pgfsetstrokecolor{textcolor}%
\pgfsetfillcolor{textcolor}%
\pgftext[x=0.522259in, y=0.496461in, left, base]{\color{textcolor}{\sffamily\fontsize{10.000000}{12.000000}\selectfont\catcode`\^=\active\def^{\ifmmode\sp\else\^{}\fi}\catcode`\%=\active\def%{\%}0}}%
\end{pgfscope}%
\begin{pgfscope}%
\pgfsetbuttcap%
\pgfsetroundjoin%
\definecolor{currentfill}{rgb}{0.000000,0.000000,0.000000}%
\pgfsetfillcolor{currentfill}%
\pgfsetlinewidth{0.803000pt}%
\definecolor{currentstroke}{rgb}{0.000000,0.000000,0.000000}%
\pgfsetstrokecolor{currentstroke}%
\pgfsetdash{}{0pt}%
\pgfsys@defobject{currentmarker}{\pgfqpoint{-0.048611in}{0.000000in}}{\pgfqpoint{-0.000000in}{0.000000in}}{%
\pgfpathmoveto{\pgfqpoint{-0.000000in}{0.000000in}}%
\pgfpathlineto{\pgfqpoint{-0.048611in}{0.000000in}}%
\pgfusepath{stroke,fill}%
}%
\begin{pgfscope}%
\pgfsys@transformshift{0.707846in}{0.946831in}%
\pgfsys@useobject{currentmarker}{}%
\end{pgfscope}%
\end{pgfscope}%
\begin{pgfscope}%
\definecolor{textcolor}{rgb}{0.000000,0.000000,0.000000}%
\pgfsetstrokecolor{textcolor}%
\pgfsetfillcolor{textcolor}%
\pgftext[x=0.433893in, y=0.894069in, left, base]{\color{textcolor}{\sffamily\fontsize{10.000000}{12.000000}\selectfont\catcode`\^=\active\def^{\ifmmode\sp\else\^{}\fi}\catcode`\%=\active\def%{\%}20}}%
\end{pgfscope}%
\begin{pgfscope}%
\pgfsetbuttcap%
\pgfsetroundjoin%
\definecolor{currentfill}{rgb}{0.000000,0.000000,0.000000}%
\pgfsetfillcolor{currentfill}%
\pgfsetlinewidth{0.803000pt}%
\definecolor{currentstroke}{rgb}{0.000000,0.000000,0.000000}%
\pgfsetstrokecolor{currentstroke}%
\pgfsetdash{}{0pt}%
\pgfsys@defobject{currentmarker}{\pgfqpoint{-0.048611in}{0.000000in}}{\pgfqpoint{-0.000000in}{0.000000in}}{%
\pgfpathmoveto{\pgfqpoint{-0.000000in}{0.000000in}}%
\pgfpathlineto{\pgfqpoint{-0.048611in}{0.000000in}}%
\pgfusepath{stroke,fill}%
}%
\begin{pgfscope}%
\pgfsys@transformshift{0.707846in}{1.344439in}%
\pgfsys@useobject{currentmarker}{}%
\end{pgfscope}%
\end{pgfscope}%
\begin{pgfscope}%
\definecolor{textcolor}{rgb}{0.000000,0.000000,0.000000}%
\pgfsetstrokecolor{textcolor}%
\pgfsetfillcolor{textcolor}%
\pgftext[x=0.433893in, y=1.291678in, left, base]{\color{textcolor}{\sffamily\fontsize{10.000000}{12.000000}\selectfont\catcode`\^=\active\def^{\ifmmode\sp\else\^{}\fi}\catcode`\%=\active\def%{\%}40}}%
\end{pgfscope}%
\begin{pgfscope}%
\pgfsetbuttcap%
\pgfsetroundjoin%
\definecolor{currentfill}{rgb}{0.000000,0.000000,0.000000}%
\pgfsetfillcolor{currentfill}%
\pgfsetlinewidth{0.803000pt}%
\definecolor{currentstroke}{rgb}{0.000000,0.000000,0.000000}%
\pgfsetstrokecolor{currentstroke}%
\pgfsetdash{}{0pt}%
\pgfsys@defobject{currentmarker}{\pgfqpoint{-0.048611in}{0.000000in}}{\pgfqpoint{-0.000000in}{0.000000in}}{%
\pgfpathmoveto{\pgfqpoint{-0.000000in}{0.000000in}}%
\pgfpathlineto{\pgfqpoint{-0.048611in}{0.000000in}}%
\pgfusepath{stroke,fill}%
}%
\begin{pgfscope}%
\pgfsys@transformshift{0.707846in}{1.742048in}%
\pgfsys@useobject{currentmarker}{}%
\end{pgfscope}%
\end{pgfscope}%
\begin{pgfscope}%
\definecolor{textcolor}{rgb}{0.000000,0.000000,0.000000}%
\pgfsetstrokecolor{textcolor}%
\pgfsetfillcolor{textcolor}%
\pgftext[x=0.433893in, y=1.689286in, left, base]{\color{textcolor}{\sffamily\fontsize{10.000000}{12.000000}\selectfont\catcode`\^=\active\def^{\ifmmode\sp\else\^{}\fi}\catcode`\%=\active\def%{\%}60}}%
\end{pgfscope}%
\begin{pgfscope}%
\pgfsetbuttcap%
\pgfsetroundjoin%
\definecolor{currentfill}{rgb}{0.000000,0.000000,0.000000}%
\pgfsetfillcolor{currentfill}%
\pgfsetlinewidth{0.803000pt}%
\definecolor{currentstroke}{rgb}{0.000000,0.000000,0.000000}%
\pgfsetstrokecolor{currentstroke}%
\pgfsetdash{}{0pt}%
\pgfsys@defobject{currentmarker}{\pgfqpoint{-0.048611in}{0.000000in}}{\pgfqpoint{-0.000000in}{0.000000in}}{%
\pgfpathmoveto{\pgfqpoint{-0.000000in}{0.000000in}}%
\pgfpathlineto{\pgfqpoint{-0.048611in}{0.000000in}}%
\pgfusepath{stroke,fill}%
}%
\begin{pgfscope}%
\pgfsys@transformshift{0.707846in}{2.139656in}%
\pgfsys@useobject{currentmarker}{}%
\end{pgfscope}%
\end{pgfscope}%
\begin{pgfscope}%
\definecolor{textcolor}{rgb}{0.000000,0.000000,0.000000}%
\pgfsetstrokecolor{textcolor}%
\pgfsetfillcolor{textcolor}%
\pgftext[x=0.433893in, y=2.086895in, left, base]{\color{textcolor}{\sffamily\fontsize{10.000000}{12.000000}\selectfont\catcode`\^=\active\def^{\ifmmode\sp\else\^{}\fi}\catcode`\%=\active\def%{\%}80}}%
\end{pgfscope}%
\begin{pgfscope}%
\pgfsetbuttcap%
\pgfsetroundjoin%
\definecolor{currentfill}{rgb}{0.000000,0.000000,0.000000}%
\pgfsetfillcolor{currentfill}%
\pgfsetlinewidth{0.803000pt}%
\definecolor{currentstroke}{rgb}{0.000000,0.000000,0.000000}%
\pgfsetstrokecolor{currentstroke}%
\pgfsetdash{}{0pt}%
\pgfsys@defobject{currentmarker}{\pgfqpoint{-0.048611in}{0.000000in}}{\pgfqpoint{-0.000000in}{0.000000in}}{%
\pgfpathmoveto{\pgfqpoint{-0.000000in}{0.000000in}}%
\pgfpathlineto{\pgfqpoint{-0.048611in}{0.000000in}}%
\pgfusepath{stroke,fill}%
}%
\begin{pgfscope}%
\pgfsys@transformshift{0.707846in}{2.537265in}%
\pgfsys@useobject{currentmarker}{}%
\end{pgfscope}%
\end{pgfscope}%
\begin{pgfscope}%
\definecolor{textcolor}{rgb}{0.000000,0.000000,0.000000}%
\pgfsetstrokecolor{textcolor}%
\pgfsetfillcolor{textcolor}%
\pgftext[x=0.345528in, y=2.484503in, left, base]{\color{textcolor}{\sffamily\fontsize{10.000000}{12.000000}\selectfont\catcode`\^=\active\def^{\ifmmode\sp\else\^{}\fi}\catcode`\%=\active\def%{\%}100}}%
\end{pgfscope}%
\begin{pgfscope}%
\definecolor{textcolor}{rgb}{0.000000,0.000000,0.000000}%
\pgfsetstrokecolor{textcolor}%
\pgfsetfillcolor{textcolor}%
\pgftext[x=0.289972in,y=1.592944in,,bottom,rotate=90.000000]{\color{textcolor}{\sffamily\fontsize{10.000000}{12.000000}\selectfont\catcode`\^=\active\def^{\ifmmode\sp\else\^{}\fi}\catcode`\%=\active\def%{\%}obciążenie (%)}}%
\end{pgfscope}%
\begin{pgfscope}%
\pgfsetrectcap%
\pgfsetmiterjoin%
\pgfsetlinewidth{0.803000pt}%
\definecolor{currentstroke}{rgb}{0.000000,0.000000,0.000000}%
\pgfsetstrokecolor{currentstroke}%
\pgfsetdash{}{0pt}%
\pgfpathmoveto{\pgfqpoint{0.707846in}{0.549222in}}%
\pgfpathlineto{\pgfqpoint{0.707846in}{2.636667in}}%
\pgfusepath{stroke}%
\end{pgfscope}%
\begin{pgfscope}%
\pgfsetrectcap%
\pgfsetmiterjoin%
\pgfsetlinewidth{0.803000pt}%
\definecolor{currentstroke}{rgb}{0.000000,0.000000,0.000000}%
\pgfsetstrokecolor{currentstroke}%
\pgfsetdash{}{0pt}%
\pgfpathmoveto{\pgfqpoint{3.017740in}{0.549222in}}%
\pgfpathlineto{\pgfqpoint{3.017740in}{2.636667in}}%
\pgfusepath{stroke}%
\end{pgfscope}%
\begin{pgfscope}%
\pgfsetrectcap%
\pgfsetmiterjoin%
\pgfsetlinewidth{0.803000pt}%
\definecolor{currentstroke}{rgb}{0.000000,0.000000,0.000000}%
\pgfsetstrokecolor{currentstroke}%
\pgfsetdash{}{0pt}%
\pgfpathmoveto{\pgfqpoint{0.707846in}{0.549222in}}%
\pgfpathlineto{\pgfqpoint{3.017740in}{0.549222in}}%
\pgfusepath{stroke}%
\end{pgfscope}%
\begin{pgfscope}%
\pgfsetrectcap%
\pgfsetmiterjoin%
\pgfsetlinewidth{0.803000pt}%
\definecolor{currentstroke}{rgb}{0.000000,0.000000,0.000000}%
\pgfsetstrokecolor{currentstroke}%
\pgfsetdash{}{0pt}%
\pgfpathmoveto{\pgfqpoint{0.707846in}{2.636667in}}%
\pgfpathlineto{\pgfqpoint{3.017740in}{2.636667in}}%
\pgfusepath{stroke}%
\end{pgfscope}%
\begin{pgfscope}%
\definecolor{textcolor}{rgb}{0.000000,0.000000,0.000000}%
\pgfsetstrokecolor{textcolor}%
\pgfsetfillcolor{textcolor}%
\pgftext[x=1.033884in,y=1.851555in,,bottom]{\color{textcolor}{\sffamily\fontsize{10.000000}{12.000000}\selectfont\catcode`\^=\active\def^{\ifmmode\sp\else\^{}\fi}\catcode`\%=\active\def%{\%}65.508}}%
\end{pgfscope}%
\begin{pgfscope}%
\definecolor{textcolor}{rgb}{0.000000,0.000000,0.000000}%
\pgfsetstrokecolor{textcolor}%
\pgfsetfillcolor{textcolor}%
\pgftext[x=1.586490in,y=1.919268in,,bottom]{\color{textcolor}{\sffamily\fontsize{10.000000}{12.000000}\selectfont\catcode`\^=\active\def^{\ifmmode\sp\else\^{}\fi}\catcode`\%=\active\def%{\%}68.914}}%
\end{pgfscope}%
\begin{pgfscope}%
\definecolor{textcolor}{rgb}{0.000000,0.000000,0.000000}%
\pgfsetstrokecolor{textcolor}%
\pgfsetfillcolor{textcolor}%
\pgftext[x=2.139096in,y=1.970746in,,bottom]{\color{textcolor}{\sffamily\fontsize{10.000000}{12.000000}\selectfont\catcode`\^=\active\def^{\ifmmode\sp\else\^{}\fi}\catcode`\%=\active\def%{\%}71.504}}%
\end{pgfscope}%
\begin{pgfscope}%
\definecolor{textcolor}{rgb}{0.000000,0.000000,0.000000}%
\pgfsetstrokecolor{textcolor}%
\pgfsetfillcolor{textcolor}%
\pgftext[x=2.691702in,y=1.900228in,,bottom]{\color{textcolor}{\sffamily\fontsize{10.000000}{12.000000}\selectfont\catcode`\^=\active\def^{\ifmmode\sp\else\^{}\fi}\catcode`\%=\active\def%{\%}67.957}}%
\end{pgfscope}%
\begin{pgfscope}%
\definecolor{textcolor}{rgb}{0.000000,0.000000,0.000000}%
\pgfsetstrokecolor{textcolor}%
\pgfsetfillcolor{textcolor}%
\pgftext[x=1.862793in,y=2.720000in,,base]{\color{textcolor}{\sffamily\fontsize{12.000000}{14.400000}\selectfont\catcode`\^=\active\def^{\ifmmode\sp\else\^{}\fi}\catcode`\%=\active\def%{\%}maksymalne obciążenie}}%
\end{pgfscope}%
\begin{pgfscope}%
\pgfsetbuttcap%
\pgfsetmiterjoin%
\definecolor{currentfill}{rgb}{1.000000,1.000000,1.000000}%
\pgfsetfillcolor{currentfill}%
\pgfsetlinewidth{0.000000pt}%
\definecolor{currentstroke}{rgb}{0.000000,0.000000,0.000000}%
\pgfsetstrokecolor{currentstroke}%
\pgfsetstrokeopacity{0.000000}%
\pgfsetdash{}{0pt}%
\pgfpathmoveto{\pgfqpoint{3.530356in}{0.549222in}}%
\pgfpathlineto{\pgfqpoint{5.840250in}{0.549222in}}%
\pgfpathlineto{\pgfqpoint{5.840250in}{2.636667in}}%
\pgfpathlineto{\pgfqpoint{3.530356in}{2.636667in}}%
\pgfpathlineto{\pgfqpoint{3.530356in}{0.549222in}}%
\pgfpathclose%
\pgfusepath{fill}%
\end{pgfscope}%
\begin{pgfscope}%
\pgfpathrectangle{\pgfqpoint{3.530356in}{0.549222in}}{\pgfqpoint{2.309894in}{2.087444in}}%
\pgfusepath{clip}%
\pgfsetbuttcap%
\pgfsetmiterjoin%
\definecolor{currentfill}{rgb}{1.000000,0.000000,0.000000}%
\pgfsetfillcolor{currentfill}%
\pgfsetlinewidth{0.000000pt}%
\definecolor{currentstroke}{rgb}{0.000000,0.000000,0.000000}%
\pgfsetstrokecolor{currentstroke}%
\pgfsetstrokeopacity{0.000000}%
\pgfsetdash{}{0pt}%
\pgfpathmoveto{\pgfqpoint{3.635351in}{0.549222in}}%
\pgfpathlineto{\pgfqpoint{4.077436in}{0.549222in}}%
\pgfpathlineto{\pgfqpoint{4.077436in}{1.837065in}}%
\pgfpathlineto{\pgfqpoint{3.635351in}{1.837065in}}%
\pgfpathlineto{\pgfqpoint{3.635351in}{0.549222in}}%
\pgfpathclose%
\pgfusepath{fill}%
\end{pgfscope}%
\begin{pgfscope}%
\pgfpathrectangle{\pgfqpoint{3.530356in}{0.549222in}}{\pgfqpoint{2.309894in}{2.087444in}}%
\pgfusepath{clip}%
\pgfsetbuttcap%
\pgfsetmiterjoin%
\definecolor{currentfill}{rgb}{0.000000,0.000000,1.000000}%
\pgfsetfillcolor{currentfill}%
\pgfsetlinewidth{0.000000pt}%
\definecolor{currentstroke}{rgb}{0.000000,0.000000,0.000000}%
\pgfsetstrokecolor{currentstroke}%
\pgfsetstrokeopacity{0.000000}%
\pgfsetdash{}{0pt}%
\pgfpathmoveto{\pgfqpoint{4.187958in}{0.549222in}}%
\pgfpathlineto{\pgfqpoint{4.630043in}{0.549222in}}%
\pgfpathlineto{\pgfqpoint{4.630043in}{1.811328in}}%
\pgfpathlineto{\pgfqpoint{4.187958in}{1.811328in}}%
\pgfpathlineto{\pgfqpoint{4.187958in}{0.549222in}}%
\pgfpathclose%
\pgfusepath{fill}%
\end{pgfscope}%
\begin{pgfscope}%
\pgfpathrectangle{\pgfqpoint{3.530356in}{0.549222in}}{\pgfqpoint{2.309894in}{2.087444in}}%
\pgfusepath{clip}%
\pgfsetbuttcap%
\pgfsetmiterjoin%
\definecolor{currentfill}{rgb}{1.000000,0.000000,0.000000}%
\pgfsetfillcolor{currentfill}%
\pgfsetlinewidth{0.000000pt}%
\definecolor{currentstroke}{rgb}{0.000000,0.000000,0.000000}%
\pgfsetstrokecolor{currentstroke}%
\pgfsetstrokeopacity{0.000000}%
\pgfsetdash{}{0pt}%
\pgfpathmoveto{\pgfqpoint{4.740564in}{0.549222in}}%
\pgfpathlineto{\pgfqpoint{5.182649in}{0.549222in}}%
\pgfpathlineto{\pgfqpoint{5.182649in}{1.883912in}}%
\pgfpathlineto{\pgfqpoint{4.740564in}{1.883912in}}%
\pgfpathlineto{\pgfqpoint{4.740564in}{0.549222in}}%
\pgfpathclose%
\pgfusepath{fill}%
\end{pgfscope}%
\begin{pgfscope}%
\pgfpathrectangle{\pgfqpoint{3.530356in}{0.549222in}}{\pgfqpoint{2.309894in}{2.087444in}}%
\pgfusepath{clip}%
\pgfsetbuttcap%
\pgfsetmiterjoin%
\definecolor{currentfill}{rgb}{0.000000,0.000000,1.000000}%
\pgfsetfillcolor{currentfill}%
\pgfsetlinewidth{0.000000pt}%
\definecolor{currentstroke}{rgb}{0.000000,0.000000,0.000000}%
\pgfsetstrokecolor{currentstroke}%
\pgfsetstrokeopacity{0.000000}%
\pgfsetdash{}{0pt}%
\pgfpathmoveto{\pgfqpoint{5.293170in}{0.549222in}}%
\pgfpathlineto{\pgfqpoint{5.735255in}{0.549222in}}%
\pgfpathlineto{\pgfqpoint{5.735255in}{1.875937in}}%
\pgfpathlineto{\pgfqpoint{5.293170in}{1.875937in}}%
\pgfpathlineto{\pgfqpoint{5.293170in}{0.549222in}}%
\pgfpathclose%
\pgfusepath{fill}%
\end{pgfscope}%
\begin{pgfscope}%
\pgfsetbuttcap%
\pgfsetroundjoin%
\definecolor{currentfill}{rgb}{0.000000,0.000000,0.000000}%
\pgfsetfillcolor{currentfill}%
\pgfsetlinewidth{0.803000pt}%
\definecolor{currentstroke}{rgb}{0.000000,0.000000,0.000000}%
\pgfsetstrokecolor{currentstroke}%
\pgfsetdash{}{0pt}%
\pgfsys@defobject{currentmarker}{\pgfqpoint{0.000000in}{-0.048611in}}{\pgfqpoint{0.000000in}{0.000000in}}{%
\pgfpathmoveto{\pgfqpoint{0.000000in}{0.000000in}}%
\pgfpathlineto{\pgfqpoint{0.000000in}{-0.048611in}}%
\pgfusepath{stroke,fill}%
}%
\begin{pgfscope}%
\pgfsys@transformshift{3.856394in}{0.549222in}%
\pgfsys@useobject{currentmarker}{}%
\end{pgfscope}%
\end{pgfscope}%
\begin{pgfscope}%
\definecolor{textcolor}{rgb}{0.000000,0.000000,0.000000}%
\pgfsetstrokecolor{textcolor}%
\pgfsetfillcolor{textcolor}%
\pgftext[x=3.715199in, y=0.346477in, left, base]{\color{textcolor}{\sffamily\fontsize{10.000000}{12.000000}\selectfont\catcode`\^=\active\def^{\ifmmode\sp\else\^{}\fi}\catcode`\%=\active\def%{\%}CPU}}%
\end{pgfscope}%
\begin{pgfscope}%
\definecolor{textcolor}{rgb}{0.000000,0.000000,0.000000}%
\pgfsetstrokecolor{textcolor}%
\pgfsetfillcolor{textcolor}%
\pgftext[x=3.679663in, y=0.190960in, left, base]{\color{textcolor}{\sffamily\fontsize{10.000000}{12.000000}\selectfont\catcode`\^=\active\def^{\ifmmode\sp\else\^{}\fi}\catcode`\%=\active\def%{\%}1024}}%
\end{pgfscope}%
\begin{pgfscope}%
\pgfsetbuttcap%
\pgfsetroundjoin%
\definecolor{currentfill}{rgb}{0.000000,0.000000,0.000000}%
\pgfsetfillcolor{currentfill}%
\pgfsetlinewidth{0.803000pt}%
\definecolor{currentstroke}{rgb}{0.000000,0.000000,0.000000}%
\pgfsetstrokecolor{currentstroke}%
\pgfsetdash{}{0pt}%
\pgfsys@defobject{currentmarker}{\pgfqpoint{0.000000in}{-0.048611in}}{\pgfqpoint{0.000000in}{0.000000in}}{%
\pgfpathmoveto{\pgfqpoint{0.000000in}{0.000000in}}%
\pgfpathlineto{\pgfqpoint{0.000000in}{-0.048611in}}%
\pgfusepath{stroke,fill}%
}%
\begin{pgfscope}%
\pgfsys@transformshift{4.409000in}{0.549222in}%
\pgfsys@useobject{currentmarker}{}%
\end{pgfscope}%
\end{pgfscope}%
\begin{pgfscope}%
\definecolor{textcolor}{rgb}{0.000000,0.000000,0.000000}%
\pgfsetstrokecolor{textcolor}%
\pgfsetfillcolor{textcolor}%
\pgftext[x=4.262482in, y=0.346477in, left, base]{\color{textcolor}{\sffamily\fontsize{10.000000}{12.000000}\selectfont\catcode`\^=\active\def^{\ifmmode\sp\else\^{}\fi}\catcode`\%=\active\def%{\%}GPU}}%
\end{pgfscope}%
\begin{pgfscope}%
\definecolor{textcolor}{rgb}{0.000000,0.000000,0.000000}%
\pgfsetstrokecolor{textcolor}%
\pgfsetfillcolor{textcolor}%
\pgftext[x=4.232269in, y=0.190960in, left, base]{\color{textcolor}{\sffamily\fontsize{10.000000}{12.000000}\selectfont\catcode`\^=\active\def^{\ifmmode\sp\else\^{}\fi}\catcode`\%=\active\def%{\%}1024}}%
\end{pgfscope}%
\begin{pgfscope}%
\pgfsetbuttcap%
\pgfsetroundjoin%
\definecolor{currentfill}{rgb}{0.000000,0.000000,0.000000}%
\pgfsetfillcolor{currentfill}%
\pgfsetlinewidth{0.803000pt}%
\definecolor{currentstroke}{rgb}{0.000000,0.000000,0.000000}%
\pgfsetstrokecolor{currentstroke}%
\pgfsetdash{}{0pt}%
\pgfsys@defobject{currentmarker}{\pgfqpoint{0.000000in}{-0.048611in}}{\pgfqpoint{0.000000in}{0.000000in}}{%
\pgfpathmoveto{\pgfqpoint{0.000000in}{0.000000in}}%
\pgfpathlineto{\pgfqpoint{0.000000in}{-0.048611in}}%
\pgfusepath{stroke,fill}%
}%
\begin{pgfscope}%
\pgfsys@transformshift{4.961606in}{0.549222in}%
\pgfsys@useobject{currentmarker}{}%
\end{pgfscope}%
\end{pgfscope}%
\begin{pgfscope}%
\definecolor{textcolor}{rgb}{0.000000,0.000000,0.000000}%
\pgfsetstrokecolor{textcolor}%
\pgfsetfillcolor{textcolor}%
\pgftext[x=4.820412in, y=0.346477in, left, base]{\color{textcolor}{\sffamily\fontsize{10.000000}{12.000000}\selectfont\catcode`\^=\active\def^{\ifmmode\sp\else\^{}\fi}\catcode`\%=\active\def%{\%}CPU}}%
\end{pgfscope}%
\begin{pgfscope}%
\definecolor{textcolor}{rgb}{0.000000,0.000000,0.000000}%
\pgfsetstrokecolor{textcolor}%
\pgfsetfillcolor{textcolor}%
\pgftext[x=4.784876in, y=0.190960in, left, base]{\color{textcolor}{\sffamily\fontsize{10.000000}{12.000000}\selectfont\catcode`\^=\active\def^{\ifmmode\sp\else\^{}\fi}\catcode`\%=\active\def%{\%}8192}}%
\end{pgfscope}%
\begin{pgfscope}%
\pgfsetbuttcap%
\pgfsetroundjoin%
\definecolor{currentfill}{rgb}{0.000000,0.000000,0.000000}%
\pgfsetfillcolor{currentfill}%
\pgfsetlinewidth{0.803000pt}%
\definecolor{currentstroke}{rgb}{0.000000,0.000000,0.000000}%
\pgfsetstrokecolor{currentstroke}%
\pgfsetdash{}{0pt}%
\pgfsys@defobject{currentmarker}{\pgfqpoint{0.000000in}{-0.048611in}}{\pgfqpoint{0.000000in}{0.000000in}}{%
\pgfpathmoveto{\pgfqpoint{0.000000in}{0.000000in}}%
\pgfpathlineto{\pgfqpoint{0.000000in}{-0.048611in}}%
\pgfusepath{stroke,fill}%
}%
\begin{pgfscope}%
\pgfsys@transformshift{5.514212in}{0.549222in}%
\pgfsys@useobject{currentmarker}{}%
\end{pgfscope}%
\end{pgfscope}%
\begin{pgfscope}%
\definecolor{textcolor}{rgb}{0.000000,0.000000,0.000000}%
\pgfsetstrokecolor{textcolor}%
\pgfsetfillcolor{textcolor}%
\pgftext[x=5.367694in, y=0.346477in, left, base]{\color{textcolor}{\sffamily\fontsize{10.000000}{12.000000}\selectfont\catcode`\^=\active\def^{\ifmmode\sp\else\^{}\fi}\catcode`\%=\active\def%{\%}GPU}}%
\end{pgfscope}%
\begin{pgfscope}%
\definecolor{textcolor}{rgb}{0.000000,0.000000,0.000000}%
\pgfsetstrokecolor{textcolor}%
\pgfsetfillcolor{textcolor}%
\pgftext[x=5.337482in, y=0.190960in, left, base]{\color{textcolor}{\sffamily\fontsize{10.000000}{12.000000}\selectfont\catcode`\^=\active\def^{\ifmmode\sp\else\^{}\fi}\catcode`\%=\active\def%{\%}8192}}%
\end{pgfscope}%
\begin{pgfscope}%
\pgfsetbuttcap%
\pgfsetroundjoin%
\definecolor{currentfill}{rgb}{0.000000,0.000000,0.000000}%
\pgfsetfillcolor{currentfill}%
\pgfsetlinewidth{0.803000pt}%
\definecolor{currentstroke}{rgb}{0.000000,0.000000,0.000000}%
\pgfsetstrokecolor{currentstroke}%
\pgfsetdash{}{0pt}%
\pgfsys@defobject{currentmarker}{\pgfqpoint{-0.048611in}{0.000000in}}{\pgfqpoint{-0.000000in}{0.000000in}}{%
\pgfpathmoveto{\pgfqpoint{-0.000000in}{0.000000in}}%
\pgfpathlineto{\pgfqpoint{-0.048611in}{0.000000in}}%
\pgfusepath{stroke,fill}%
}%
\begin{pgfscope}%
\pgfsys@transformshift{3.530356in}{0.549222in}%
\pgfsys@useobject{currentmarker}{}%
\end{pgfscope}%
\end{pgfscope}%
\begin{pgfscope}%
\definecolor{textcolor}{rgb}{0.000000,0.000000,0.000000}%
\pgfsetstrokecolor{textcolor}%
\pgfsetfillcolor{textcolor}%
\pgftext[x=3.344769in, y=0.496461in, left, base]{\color{textcolor}{\sffamily\fontsize{10.000000}{12.000000}\selectfont\catcode`\^=\active\def^{\ifmmode\sp\else\^{}\fi}\catcode`\%=\active\def%{\%}0}}%
\end{pgfscope}%
\begin{pgfscope}%
\pgfsetbuttcap%
\pgfsetroundjoin%
\definecolor{currentfill}{rgb}{0.000000,0.000000,0.000000}%
\pgfsetfillcolor{currentfill}%
\pgfsetlinewidth{0.803000pt}%
\definecolor{currentstroke}{rgb}{0.000000,0.000000,0.000000}%
\pgfsetstrokecolor{currentstroke}%
\pgfsetdash{}{0pt}%
\pgfsys@defobject{currentmarker}{\pgfqpoint{-0.048611in}{0.000000in}}{\pgfqpoint{-0.000000in}{0.000000in}}{%
\pgfpathmoveto{\pgfqpoint{-0.000000in}{0.000000in}}%
\pgfpathlineto{\pgfqpoint{-0.048611in}{0.000000in}}%
\pgfusepath{stroke,fill}%
}%
\begin{pgfscope}%
\pgfsys@transformshift{3.530356in}{0.946831in}%
\pgfsys@useobject{currentmarker}{}%
\end{pgfscope}%
\end{pgfscope}%
\begin{pgfscope}%
\definecolor{textcolor}{rgb}{0.000000,0.000000,0.000000}%
\pgfsetstrokecolor{textcolor}%
\pgfsetfillcolor{textcolor}%
\pgftext[x=3.256403in, y=0.894069in, left, base]{\color{textcolor}{\sffamily\fontsize{10.000000}{12.000000}\selectfont\catcode`\^=\active\def^{\ifmmode\sp\else\^{}\fi}\catcode`\%=\active\def%{\%}20}}%
\end{pgfscope}%
\begin{pgfscope}%
\pgfsetbuttcap%
\pgfsetroundjoin%
\definecolor{currentfill}{rgb}{0.000000,0.000000,0.000000}%
\pgfsetfillcolor{currentfill}%
\pgfsetlinewidth{0.803000pt}%
\definecolor{currentstroke}{rgb}{0.000000,0.000000,0.000000}%
\pgfsetstrokecolor{currentstroke}%
\pgfsetdash{}{0pt}%
\pgfsys@defobject{currentmarker}{\pgfqpoint{-0.048611in}{0.000000in}}{\pgfqpoint{-0.000000in}{0.000000in}}{%
\pgfpathmoveto{\pgfqpoint{-0.000000in}{0.000000in}}%
\pgfpathlineto{\pgfqpoint{-0.048611in}{0.000000in}}%
\pgfusepath{stroke,fill}%
}%
\begin{pgfscope}%
\pgfsys@transformshift{3.530356in}{1.344439in}%
\pgfsys@useobject{currentmarker}{}%
\end{pgfscope}%
\end{pgfscope}%
\begin{pgfscope}%
\definecolor{textcolor}{rgb}{0.000000,0.000000,0.000000}%
\pgfsetstrokecolor{textcolor}%
\pgfsetfillcolor{textcolor}%
\pgftext[x=3.256403in, y=1.291678in, left, base]{\color{textcolor}{\sffamily\fontsize{10.000000}{12.000000}\selectfont\catcode`\^=\active\def^{\ifmmode\sp\else\^{}\fi}\catcode`\%=\active\def%{\%}40}}%
\end{pgfscope}%
\begin{pgfscope}%
\pgfsetbuttcap%
\pgfsetroundjoin%
\definecolor{currentfill}{rgb}{0.000000,0.000000,0.000000}%
\pgfsetfillcolor{currentfill}%
\pgfsetlinewidth{0.803000pt}%
\definecolor{currentstroke}{rgb}{0.000000,0.000000,0.000000}%
\pgfsetstrokecolor{currentstroke}%
\pgfsetdash{}{0pt}%
\pgfsys@defobject{currentmarker}{\pgfqpoint{-0.048611in}{0.000000in}}{\pgfqpoint{-0.000000in}{0.000000in}}{%
\pgfpathmoveto{\pgfqpoint{-0.000000in}{0.000000in}}%
\pgfpathlineto{\pgfqpoint{-0.048611in}{0.000000in}}%
\pgfusepath{stroke,fill}%
}%
\begin{pgfscope}%
\pgfsys@transformshift{3.530356in}{1.742048in}%
\pgfsys@useobject{currentmarker}{}%
\end{pgfscope}%
\end{pgfscope}%
\begin{pgfscope}%
\definecolor{textcolor}{rgb}{0.000000,0.000000,0.000000}%
\pgfsetstrokecolor{textcolor}%
\pgfsetfillcolor{textcolor}%
\pgftext[x=3.256403in, y=1.689286in, left, base]{\color{textcolor}{\sffamily\fontsize{10.000000}{12.000000}\selectfont\catcode`\^=\active\def^{\ifmmode\sp\else\^{}\fi}\catcode`\%=\active\def%{\%}60}}%
\end{pgfscope}%
\begin{pgfscope}%
\pgfsetbuttcap%
\pgfsetroundjoin%
\definecolor{currentfill}{rgb}{0.000000,0.000000,0.000000}%
\pgfsetfillcolor{currentfill}%
\pgfsetlinewidth{0.803000pt}%
\definecolor{currentstroke}{rgb}{0.000000,0.000000,0.000000}%
\pgfsetstrokecolor{currentstroke}%
\pgfsetdash{}{0pt}%
\pgfsys@defobject{currentmarker}{\pgfqpoint{-0.048611in}{0.000000in}}{\pgfqpoint{-0.000000in}{0.000000in}}{%
\pgfpathmoveto{\pgfqpoint{-0.000000in}{0.000000in}}%
\pgfpathlineto{\pgfqpoint{-0.048611in}{0.000000in}}%
\pgfusepath{stroke,fill}%
}%
\begin{pgfscope}%
\pgfsys@transformshift{3.530356in}{2.139656in}%
\pgfsys@useobject{currentmarker}{}%
\end{pgfscope}%
\end{pgfscope}%
\begin{pgfscope}%
\definecolor{textcolor}{rgb}{0.000000,0.000000,0.000000}%
\pgfsetstrokecolor{textcolor}%
\pgfsetfillcolor{textcolor}%
\pgftext[x=3.256403in, y=2.086895in, left, base]{\color{textcolor}{\sffamily\fontsize{10.000000}{12.000000}\selectfont\catcode`\^=\active\def^{\ifmmode\sp\else\^{}\fi}\catcode`\%=\active\def%{\%}80}}%
\end{pgfscope}%
\begin{pgfscope}%
\pgfsetbuttcap%
\pgfsetroundjoin%
\definecolor{currentfill}{rgb}{0.000000,0.000000,0.000000}%
\pgfsetfillcolor{currentfill}%
\pgfsetlinewidth{0.803000pt}%
\definecolor{currentstroke}{rgb}{0.000000,0.000000,0.000000}%
\pgfsetstrokecolor{currentstroke}%
\pgfsetdash{}{0pt}%
\pgfsys@defobject{currentmarker}{\pgfqpoint{-0.048611in}{0.000000in}}{\pgfqpoint{-0.000000in}{0.000000in}}{%
\pgfpathmoveto{\pgfqpoint{-0.000000in}{0.000000in}}%
\pgfpathlineto{\pgfqpoint{-0.048611in}{0.000000in}}%
\pgfusepath{stroke,fill}%
}%
\begin{pgfscope}%
\pgfsys@transformshift{3.530356in}{2.537265in}%
\pgfsys@useobject{currentmarker}{}%
\end{pgfscope}%
\end{pgfscope}%
\begin{pgfscope}%
\definecolor{textcolor}{rgb}{0.000000,0.000000,0.000000}%
\pgfsetstrokecolor{textcolor}%
\pgfsetfillcolor{textcolor}%
\pgftext[x=3.168038in, y=2.484503in, left, base]{\color{textcolor}{\sffamily\fontsize{10.000000}{12.000000}\selectfont\catcode`\^=\active\def^{\ifmmode\sp\else\^{}\fi}\catcode`\%=\active\def%{\%}100}}%
\end{pgfscope}%
\begin{pgfscope}%
\pgfsetrectcap%
\pgfsetmiterjoin%
\pgfsetlinewidth{0.803000pt}%
\definecolor{currentstroke}{rgb}{0.000000,0.000000,0.000000}%
\pgfsetstrokecolor{currentstroke}%
\pgfsetdash{}{0pt}%
\pgfpathmoveto{\pgfqpoint{3.530356in}{0.549222in}}%
\pgfpathlineto{\pgfqpoint{3.530356in}{2.636667in}}%
\pgfusepath{stroke}%
\end{pgfscope}%
\begin{pgfscope}%
\pgfsetrectcap%
\pgfsetmiterjoin%
\pgfsetlinewidth{0.803000pt}%
\definecolor{currentstroke}{rgb}{0.000000,0.000000,0.000000}%
\pgfsetstrokecolor{currentstroke}%
\pgfsetdash{}{0pt}%
\pgfpathmoveto{\pgfqpoint{5.840250in}{0.549222in}}%
\pgfpathlineto{\pgfqpoint{5.840250in}{2.636667in}}%
\pgfusepath{stroke}%
\end{pgfscope}%
\begin{pgfscope}%
\pgfsetrectcap%
\pgfsetmiterjoin%
\pgfsetlinewidth{0.803000pt}%
\definecolor{currentstroke}{rgb}{0.000000,0.000000,0.000000}%
\pgfsetstrokecolor{currentstroke}%
\pgfsetdash{}{0pt}%
\pgfpathmoveto{\pgfqpoint{3.530356in}{0.549222in}}%
\pgfpathlineto{\pgfqpoint{5.840250in}{0.549222in}}%
\pgfusepath{stroke}%
\end{pgfscope}%
\begin{pgfscope}%
\pgfsetrectcap%
\pgfsetmiterjoin%
\pgfsetlinewidth{0.803000pt}%
\definecolor{currentstroke}{rgb}{0.000000,0.000000,0.000000}%
\pgfsetstrokecolor{currentstroke}%
\pgfsetdash{}{0pt}%
\pgfpathmoveto{\pgfqpoint{3.530356in}{2.636667in}}%
\pgfpathlineto{\pgfqpoint{5.840250in}{2.636667in}}%
\pgfusepath{stroke}%
\end{pgfscope}%
\begin{pgfscope}%
\definecolor{textcolor}{rgb}{0.000000,0.000000,0.000000}%
\pgfsetstrokecolor{textcolor}%
\pgfsetfillcolor{textcolor}%
\pgftext[x=3.856394in,y=1.837065in,,bottom]{\color{textcolor}{\sffamily\fontsize{10.000000}{12.000000}\selectfont\catcode`\^=\active\def^{\ifmmode\sp\else\^{}\fi}\catcode`\%=\active\def%{\%}64.779}}%
\end{pgfscope}%
\begin{pgfscope}%
\definecolor{textcolor}{rgb}{0.000000,0.000000,0.000000}%
\pgfsetstrokecolor{textcolor}%
\pgfsetfillcolor{textcolor}%
\pgftext[x=4.409000in,y=1.811328in,,bottom]{\color{textcolor}{\sffamily\fontsize{10.000000}{12.000000}\selectfont\catcode`\^=\active\def^{\ifmmode\sp\else\^{}\fi}\catcode`\%=\active\def%{\%}63.485}}%
\end{pgfscope}%
\begin{pgfscope}%
\definecolor{textcolor}{rgb}{0.000000,0.000000,0.000000}%
\pgfsetstrokecolor{textcolor}%
\pgfsetfillcolor{textcolor}%
\pgftext[x=4.961606in,y=1.883912in,,bottom]{\color{textcolor}{\sffamily\fontsize{10.000000}{12.000000}\selectfont\catcode`\^=\active\def^{\ifmmode\sp\else\^{}\fi}\catcode`\%=\active\def%{\%}67.136}}%
\end{pgfscope}%
\begin{pgfscope}%
\definecolor{textcolor}{rgb}{0.000000,0.000000,0.000000}%
\pgfsetstrokecolor{textcolor}%
\pgfsetfillcolor{textcolor}%
\pgftext[x=5.514212in,y=1.875937in,,bottom]{\color{textcolor}{\sffamily\fontsize{10.000000}{12.000000}\selectfont\catcode`\^=\active\def^{\ifmmode\sp\else\^{}\fi}\catcode`\%=\active\def%{\%}66.735}}%
\end{pgfscope}%
\begin{pgfscope}%
\definecolor{textcolor}{rgb}{0.000000,0.000000,0.000000}%
\pgfsetstrokecolor{textcolor}%
\pgfsetfillcolor{textcolor}%
\pgftext[x=4.685303in,y=2.720000in,,base]{\color{textcolor}{\sffamily\fontsize{12.000000}{14.400000}\selectfont\catcode`\^=\active\def^{\ifmmode\sp\else\^{}\fi}\catcode`\%=\active\def%{\%}średnie obciążenie}}%
\end{pgfscope}%
\end{pgfpicture}%
\makeatother%
\endgroup%
}
    \caption{Obciążenie systemu w trakcie przetwarzania dźwięku w trybie online}
    \label{fig:Obciążenie systemu w trakcie przetwarzania dźwięku w trybie online}
\end{figure}

Obciążenie systemu w trakcie przetwarzania dźwięku w trybie online jest porównywalne w przypadku obu implementacji. Zarówno dla bufora 1024 jak i 8192 próbek, różnice są znikome. Warto zauważyć, że potwierdza się tendencja GPU do zwiększania przewagi nad CPU, gdy zwiększa się ilość przetwarzanych danych.

\begin{figure}[H]
    \centering
    \scalebox{1.0}{%% Creator: Matplotlib, PGF backend
%%
%% To include the figure in your LaTeX document, write
%%   \input{<filename>.pgf}
%%
%% Make sure the required packages are loaded in your preamble
%%   \usepackage{pgf}
%%
%% Also ensure that all the required font packages are loaded; for instance,
%% the lmodern package is sometimes necessary when using math font.
%%   \usepackage{lmodern}
%%
%% Figures using additional raster images can only be included by \input if
%% they are in the same directory as the main LaTeX file. For loading figures
%% from other directories you can use the `import` package
%%   \usepackage{import}
%%
%% and then include the figures with
%%   \import{<path to file>}{<filename>.pgf}
%%
%% Matplotlib used the following preamble
%%   \def\mathdefault#1{#1}
%%   \everymath=\expandafter{\the\everymath\displaystyle}
%%   
%%   \usepackage{fontspec}
%%   \setmainfont{DejaVuSerif.ttf}[Path=\detokenize{/usr/lib/python3.12/site-packages/matplotlib/mpl-data/fonts/ttf/}]
%%   \setsansfont{DejaVuSans.ttf}[Path=\detokenize{/usr/lib/python3.12/site-packages/matplotlib/mpl-data/fonts/ttf/}]
%%   \setmonofont{DejaVuSansMono.ttf}[Path=\detokenize{/usr/lib/python3.12/site-packages/matplotlib/mpl-data/fonts/ttf/}]
%%   \makeatletter\@ifpackageloaded{underscore}{}{\usepackage[strings]{underscore}}\makeatother
%%
\begingroup%
\makeatletter%
\begin{pgfpicture}%
\pgfpathrectangle{\pgfpointorigin}{\pgfqpoint{5.990000in}{6.000000in}}%
\pgfusepath{use as bounding box, clip}%
\begin{pgfscope}%
\pgfsetbuttcap%
\pgfsetmiterjoin%
\definecolor{currentfill}{rgb}{1.000000,1.000000,1.000000}%
\pgfsetfillcolor{currentfill}%
\pgfsetlinewidth{0.000000pt}%
\definecolor{currentstroke}{rgb}{1.000000,1.000000,1.000000}%
\pgfsetstrokecolor{currentstroke}%
\pgfsetdash{}{0pt}%
\pgfpathmoveto{\pgfqpoint{0.000000in}{0.000000in}}%
\pgfpathlineto{\pgfqpoint{5.990000in}{0.000000in}}%
\pgfpathlineto{\pgfqpoint{5.990000in}{6.000000in}}%
\pgfpathlineto{\pgfqpoint{0.000000in}{6.000000in}}%
\pgfpathlineto{\pgfqpoint{0.000000in}{0.000000in}}%
\pgfpathclose%
\pgfusepath{fill}%
\end{pgfscope}%
\begin{pgfscope}%
\pgfsetbuttcap%
\pgfsetmiterjoin%
\definecolor{currentfill}{rgb}{1.000000,1.000000,1.000000}%
\pgfsetfillcolor{currentfill}%
\pgfsetlinewidth{0.000000pt}%
\definecolor{currentstroke}{rgb}{0.000000,0.000000,0.000000}%
\pgfsetstrokecolor{currentstroke}%
\pgfsetstrokeopacity{0.000000}%
\pgfsetdash{}{0pt}%
\pgfpathmoveto{\pgfqpoint{0.885050in}{4.360741in}}%
\pgfpathlineto{\pgfqpoint{5.840250in}{4.360741in}}%
\pgfpathlineto{\pgfqpoint{5.840250in}{5.646667in}}%
\pgfpathlineto{\pgfqpoint{0.885050in}{5.646667in}}%
\pgfpathlineto{\pgfqpoint{0.885050in}{4.360741in}}%
\pgfpathclose%
\pgfusepath{fill}%
\end{pgfscope}%
\begin{pgfscope}%
\pgfsetbuttcap%
\pgfsetroundjoin%
\definecolor{currentfill}{rgb}{0.000000,0.000000,0.000000}%
\pgfsetfillcolor{currentfill}%
\pgfsetlinewidth{0.803000pt}%
\definecolor{currentstroke}{rgb}{0.000000,0.000000,0.000000}%
\pgfsetstrokecolor{currentstroke}%
\pgfsetdash{}{0pt}%
\pgfsys@defobject{currentmarker}{\pgfqpoint{0.000000in}{-0.048611in}}{\pgfqpoint{0.000000in}{0.000000in}}{%
\pgfpathmoveto{\pgfqpoint{0.000000in}{0.000000in}}%
\pgfpathlineto{\pgfqpoint{0.000000in}{-0.048611in}}%
\pgfusepath{stroke,fill}%
}%
\begin{pgfscope}%
\pgfsys@transformshift{1.096806in}{4.360741in}%
\pgfsys@useobject{currentmarker}{}%
\end{pgfscope}%
\end{pgfscope}%
\begin{pgfscope}%
\definecolor{textcolor}{rgb}{0.000000,0.000000,0.000000}%
\pgfsetstrokecolor{textcolor}%
\pgfsetfillcolor{textcolor}%
\pgftext[x=1.096806in,y=4.263519in,,top]{\color{textcolor}{\sffamily\fontsize{10.000000}{12.000000}\selectfont\catcode`\^=\active\def^{\ifmmode\sp\else\^{}\fi}\catcode`\%=\active\def%{\%}0}}%
\end{pgfscope}%
\begin{pgfscope}%
\pgfsetbuttcap%
\pgfsetroundjoin%
\definecolor{currentfill}{rgb}{0.000000,0.000000,0.000000}%
\pgfsetfillcolor{currentfill}%
\pgfsetlinewidth{0.803000pt}%
\definecolor{currentstroke}{rgb}{0.000000,0.000000,0.000000}%
\pgfsetstrokecolor{currentstroke}%
\pgfsetdash{}{0pt}%
\pgfsys@defobject{currentmarker}{\pgfqpoint{0.000000in}{-0.048611in}}{\pgfqpoint{0.000000in}{0.000000in}}{%
\pgfpathmoveto{\pgfqpoint{0.000000in}{0.000000in}}%
\pgfpathlineto{\pgfqpoint{0.000000in}{-0.048611in}}%
\pgfusepath{stroke,fill}%
}%
\begin{pgfscope}%
\pgfsys@transformshift{1.939362in}{4.360741in}%
\pgfsys@useobject{currentmarker}{}%
\end{pgfscope}%
\end{pgfscope}%
\begin{pgfscope}%
\definecolor{textcolor}{rgb}{0.000000,0.000000,0.000000}%
\pgfsetstrokecolor{textcolor}%
\pgfsetfillcolor{textcolor}%
\pgftext[x=1.939362in,y=4.263519in,,top]{\color{textcolor}{\sffamily\fontsize{10.000000}{12.000000}\selectfont\catcode`\^=\active\def^{\ifmmode\sp\else\^{}\fi}\catcode`\%=\active\def%{\%}2000}}%
\end{pgfscope}%
\begin{pgfscope}%
\pgfsetbuttcap%
\pgfsetroundjoin%
\definecolor{currentfill}{rgb}{0.000000,0.000000,0.000000}%
\pgfsetfillcolor{currentfill}%
\pgfsetlinewidth{0.803000pt}%
\definecolor{currentstroke}{rgb}{0.000000,0.000000,0.000000}%
\pgfsetstrokecolor{currentstroke}%
\pgfsetdash{}{0pt}%
\pgfsys@defobject{currentmarker}{\pgfqpoint{0.000000in}{-0.048611in}}{\pgfqpoint{0.000000in}{0.000000in}}{%
\pgfpathmoveto{\pgfqpoint{0.000000in}{0.000000in}}%
\pgfpathlineto{\pgfqpoint{0.000000in}{-0.048611in}}%
\pgfusepath{stroke,fill}%
}%
\begin{pgfscope}%
\pgfsys@transformshift{2.781918in}{4.360741in}%
\pgfsys@useobject{currentmarker}{}%
\end{pgfscope}%
\end{pgfscope}%
\begin{pgfscope}%
\definecolor{textcolor}{rgb}{0.000000,0.000000,0.000000}%
\pgfsetstrokecolor{textcolor}%
\pgfsetfillcolor{textcolor}%
\pgftext[x=2.781918in,y=4.263519in,,top]{\color{textcolor}{\sffamily\fontsize{10.000000}{12.000000}\selectfont\catcode`\^=\active\def^{\ifmmode\sp\else\^{}\fi}\catcode`\%=\active\def%{\%}4000}}%
\end{pgfscope}%
\begin{pgfscope}%
\pgfsetbuttcap%
\pgfsetroundjoin%
\definecolor{currentfill}{rgb}{0.000000,0.000000,0.000000}%
\pgfsetfillcolor{currentfill}%
\pgfsetlinewidth{0.803000pt}%
\definecolor{currentstroke}{rgb}{0.000000,0.000000,0.000000}%
\pgfsetstrokecolor{currentstroke}%
\pgfsetdash{}{0pt}%
\pgfsys@defobject{currentmarker}{\pgfqpoint{0.000000in}{-0.048611in}}{\pgfqpoint{0.000000in}{0.000000in}}{%
\pgfpathmoveto{\pgfqpoint{0.000000in}{0.000000in}}%
\pgfpathlineto{\pgfqpoint{0.000000in}{-0.048611in}}%
\pgfusepath{stroke,fill}%
}%
\begin{pgfscope}%
\pgfsys@transformshift{3.624474in}{4.360741in}%
\pgfsys@useobject{currentmarker}{}%
\end{pgfscope}%
\end{pgfscope}%
\begin{pgfscope}%
\definecolor{textcolor}{rgb}{0.000000,0.000000,0.000000}%
\pgfsetstrokecolor{textcolor}%
\pgfsetfillcolor{textcolor}%
\pgftext[x=3.624474in,y=4.263519in,,top]{\color{textcolor}{\sffamily\fontsize{10.000000}{12.000000}\selectfont\catcode`\^=\active\def^{\ifmmode\sp\else\^{}\fi}\catcode`\%=\active\def%{\%}6000}}%
\end{pgfscope}%
\begin{pgfscope}%
\pgfsetbuttcap%
\pgfsetroundjoin%
\definecolor{currentfill}{rgb}{0.000000,0.000000,0.000000}%
\pgfsetfillcolor{currentfill}%
\pgfsetlinewidth{0.803000pt}%
\definecolor{currentstroke}{rgb}{0.000000,0.000000,0.000000}%
\pgfsetstrokecolor{currentstroke}%
\pgfsetdash{}{0pt}%
\pgfsys@defobject{currentmarker}{\pgfqpoint{0.000000in}{-0.048611in}}{\pgfqpoint{0.000000in}{0.000000in}}{%
\pgfpathmoveto{\pgfqpoint{0.000000in}{0.000000in}}%
\pgfpathlineto{\pgfqpoint{0.000000in}{-0.048611in}}%
\pgfusepath{stroke,fill}%
}%
\begin{pgfscope}%
\pgfsys@transformshift{4.467031in}{4.360741in}%
\pgfsys@useobject{currentmarker}{}%
\end{pgfscope}%
\end{pgfscope}%
\begin{pgfscope}%
\definecolor{textcolor}{rgb}{0.000000,0.000000,0.000000}%
\pgfsetstrokecolor{textcolor}%
\pgfsetfillcolor{textcolor}%
\pgftext[x=4.467031in,y=4.263519in,,top]{\color{textcolor}{\sffamily\fontsize{10.000000}{12.000000}\selectfont\catcode`\^=\active\def^{\ifmmode\sp\else\^{}\fi}\catcode`\%=\active\def%{\%}8000}}%
\end{pgfscope}%
\begin{pgfscope}%
\pgfsetbuttcap%
\pgfsetroundjoin%
\definecolor{currentfill}{rgb}{0.000000,0.000000,0.000000}%
\pgfsetfillcolor{currentfill}%
\pgfsetlinewidth{0.803000pt}%
\definecolor{currentstroke}{rgb}{0.000000,0.000000,0.000000}%
\pgfsetstrokecolor{currentstroke}%
\pgfsetdash{}{0pt}%
\pgfsys@defobject{currentmarker}{\pgfqpoint{0.000000in}{-0.048611in}}{\pgfqpoint{0.000000in}{0.000000in}}{%
\pgfpathmoveto{\pgfqpoint{0.000000in}{0.000000in}}%
\pgfpathlineto{\pgfqpoint{0.000000in}{-0.048611in}}%
\pgfusepath{stroke,fill}%
}%
\begin{pgfscope}%
\pgfsys@transformshift{5.309587in}{4.360741in}%
\pgfsys@useobject{currentmarker}{}%
\end{pgfscope}%
\end{pgfscope}%
\begin{pgfscope}%
\definecolor{textcolor}{rgb}{0.000000,0.000000,0.000000}%
\pgfsetstrokecolor{textcolor}%
\pgfsetfillcolor{textcolor}%
\pgftext[x=5.309587in,y=4.263519in,,top]{\color{textcolor}{\sffamily\fontsize{10.000000}{12.000000}\selectfont\catcode`\^=\active\def^{\ifmmode\sp\else\^{}\fi}\catcode`\%=\active\def%{\%}10000}}%
\end{pgfscope}%
\begin{pgfscope}%
\pgfsetbuttcap%
\pgfsetroundjoin%
\definecolor{currentfill}{rgb}{0.000000,0.000000,0.000000}%
\pgfsetfillcolor{currentfill}%
\pgfsetlinewidth{0.803000pt}%
\definecolor{currentstroke}{rgb}{0.000000,0.000000,0.000000}%
\pgfsetstrokecolor{currentstroke}%
\pgfsetdash{}{0pt}%
\pgfsys@defobject{currentmarker}{\pgfqpoint{-0.048611in}{0.000000in}}{\pgfqpoint{-0.000000in}{0.000000in}}{%
\pgfpathmoveto{\pgfqpoint{-0.000000in}{0.000000in}}%
\pgfpathlineto{\pgfqpoint{-0.048611in}{0.000000in}}%
\pgfusepath{stroke,fill}%
}%
\begin{pgfscope}%
\pgfsys@transformshift{0.885050in}{4.458014in}%
\pgfsys@useobject{currentmarker}{}%
\end{pgfscope}%
\end{pgfscope}%
\begin{pgfscope}%
\definecolor{textcolor}{rgb}{0.000000,0.000000,0.000000}%
\pgfsetstrokecolor{textcolor}%
\pgfsetfillcolor{textcolor}%
\pgftext[x=0.346001in, y=4.405253in, left, base]{\color{textcolor}{\sffamily\fontsize{10.000000}{12.000000}\selectfont\catcode`\^=\active\def^{\ifmmode\sp\else\^{}\fi}\catcode`\%=\active\def%{\%}12000}}%
\end{pgfscope}%
\begin{pgfscope}%
\pgfsetbuttcap%
\pgfsetroundjoin%
\definecolor{currentfill}{rgb}{0.000000,0.000000,0.000000}%
\pgfsetfillcolor{currentfill}%
\pgfsetlinewidth{0.803000pt}%
\definecolor{currentstroke}{rgb}{0.000000,0.000000,0.000000}%
\pgfsetstrokecolor{currentstroke}%
\pgfsetdash{}{0pt}%
\pgfsys@defobject{currentmarker}{\pgfqpoint{-0.048611in}{0.000000in}}{\pgfqpoint{-0.000000in}{0.000000in}}{%
\pgfpathmoveto{\pgfqpoint{-0.000000in}{0.000000in}}%
\pgfpathlineto{\pgfqpoint{-0.048611in}{0.000000in}}%
\pgfusepath{stroke,fill}%
}%
\begin{pgfscope}%
\pgfsys@transformshift{0.885050in}{4.877034in}%
\pgfsys@useobject{currentmarker}{}%
\end{pgfscope}%
\end{pgfscope}%
\begin{pgfscope}%
\definecolor{textcolor}{rgb}{0.000000,0.000000,0.000000}%
\pgfsetstrokecolor{textcolor}%
\pgfsetfillcolor{textcolor}%
\pgftext[x=0.346001in, y=4.824272in, left, base]{\color{textcolor}{\sffamily\fontsize{10.000000}{12.000000}\selectfont\catcode`\^=\active\def^{\ifmmode\sp\else\^{}\fi}\catcode`\%=\active\def%{\%}13000}}%
\end{pgfscope}%
\begin{pgfscope}%
\pgfsetbuttcap%
\pgfsetroundjoin%
\definecolor{currentfill}{rgb}{0.000000,0.000000,0.000000}%
\pgfsetfillcolor{currentfill}%
\pgfsetlinewidth{0.803000pt}%
\definecolor{currentstroke}{rgb}{0.000000,0.000000,0.000000}%
\pgfsetstrokecolor{currentstroke}%
\pgfsetdash{}{0pt}%
\pgfsys@defobject{currentmarker}{\pgfqpoint{-0.048611in}{0.000000in}}{\pgfqpoint{-0.000000in}{0.000000in}}{%
\pgfpathmoveto{\pgfqpoint{-0.000000in}{0.000000in}}%
\pgfpathlineto{\pgfqpoint{-0.048611in}{0.000000in}}%
\pgfusepath{stroke,fill}%
}%
\begin{pgfscope}%
\pgfsys@transformshift{0.885050in}{5.296054in}%
\pgfsys@useobject{currentmarker}{}%
\end{pgfscope}%
\end{pgfscope}%
\begin{pgfscope}%
\definecolor{textcolor}{rgb}{0.000000,0.000000,0.000000}%
\pgfsetstrokecolor{textcolor}%
\pgfsetfillcolor{textcolor}%
\pgftext[x=0.346001in, y=5.243292in, left, base]{\color{textcolor}{\sffamily\fontsize{10.000000}{12.000000}\selectfont\catcode`\^=\active\def^{\ifmmode\sp\else\^{}\fi}\catcode`\%=\active\def%{\%}14000}}%
\end{pgfscope}%
\begin{pgfscope}%
\definecolor{textcolor}{rgb}{0.000000,0.000000,0.000000}%
\pgfsetstrokecolor{textcolor}%
\pgfsetfillcolor{textcolor}%
\pgftext[x=0.290446in,y=5.003704in,,bottom,rotate=90.000000]{\color{textcolor}{\sffamily\fontsize{10.000000}{12.000000}\selectfont\catcode`\^=\active\def^{\ifmmode\sp\else\^{}\fi}\catcode`\%=\active\def%{\%}czas (ns)}}%
\end{pgfscope}%
\begin{pgfscope}%
\pgfpathrectangle{\pgfqpoint{0.885050in}{4.360741in}}{\pgfqpoint{4.955200in}{1.285926in}}%
\pgfusepath{clip}%
\pgfsetrectcap%
\pgfsetroundjoin%
\pgfsetlinewidth{1.505625pt}%
\definecolor{currentstroke}{rgb}{0.145098,0.145098,1.000000}%
\pgfsetstrokecolor{currentstroke}%
\pgfsetdash{}{0pt}%
\pgfpathmoveto{\pgfqpoint{1.123767in}{5.231022in}}%
\pgfpathlineto{\pgfqpoint{1.124610in}{5.229723in}}%
\pgfpathlineto{\pgfqpoint{1.125031in}{5.229136in}}%
\pgfpathlineto{\pgfqpoint{1.125874in}{5.229807in}}%
\pgfpathlineto{\pgfqpoint{1.126716in}{5.229891in}}%
\pgfpathlineto{\pgfqpoint{1.127138in}{5.232447in}}%
\pgfpathlineto{\pgfqpoint{1.127559in}{5.230854in}}%
\pgfpathlineto{\pgfqpoint{1.130508in}{5.223186in}}%
\pgfpathlineto{\pgfqpoint{1.130929in}{5.223815in}}%
\pgfpathlineto{\pgfqpoint{1.131772in}{5.223186in}}%
\pgfpathlineto{\pgfqpoint{1.135563in}{5.224401in}}%
\pgfpathlineto{\pgfqpoint{1.137670in}{5.223144in}}%
\pgfpathlineto{\pgfqpoint{1.138512in}{5.224192in}}%
\pgfpathlineto{\pgfqpoint{1.138933in}{5.223228in}}%
\pgfpathlineto{\pgfqpoint{1.139776in}{5.223480in}}%
\pgfpathlineto{\pgfqpoint{1.140619in}{5.221133in}}%
\pgfpathlineto{\pgfqpoint{1.141461in}{5.222725in}}%
\pgfpathlineto{\pgfqpoint{1.144410in}{5.229514in}}%
\pgfpathlineto{\pgfqpoint{1.146517in}{5.228382in}}%
\pgfpathlineto{\pgfqpoint{1.147359in}{5.231231in}}%
\pgfpathlineto{\pgfqpoint{1.147780in}{5.230854in}}%
\pgfpathlineto{\pgfqpoint{1.149044in}{5.230854in}}%
\pgfpathlineto{\pgfqpoint{1.149887in}{5.231231in}}%
\pgfpathlineto{\pgfqpoint{1.150308in}{5.230058in}}%
\pgfpathlineto{\pgfqpoint{1.151151in}{5.230980in}}%
\pgfpathlineto{\pgfqpoint{1.151572in}{5.230058in}}%
\pgfpathlineto{\pgfqpoint{1.152414in}{5.230561in}}%
\pgfpathlineto{\pgfqpoint{1.152836in}{5.230100in}}%
\pgfpathlineto{\pgfqpoint{1.153257in}{5.230896in}}%
\pgfpathlineto{\pgfqpoint{1.157470in}{5.233159in}}%
\pgfpathlineto{\pgfqpoint{1.157891in}{5.232321in}}%
\pgfpathlineto{\pgfqpoint{1.158312in}{5.233201in}}%
\pgfpathlineto{\pgfqpoint{1.159997in}{5.235086in}}%
\pgfpathlineto{\pgfqpoint{1.160419in}{5.232153in}}%
\pgfpathlineto{\pgfqpoint{1.161261in}{5.233955in}}%
\pgfpathlineto{\pgfqpoint{1.162946in}{5.233285in}}%
\pgfpathlineto{\pgfqpoint{1.163789in}{5.233327in}}%
\pgfpathlineto{\pgfqpoint{1.166738in}{5.225742in}}%
\pgfpathlineto{\pgfqpoint{1.167159in}{5.227502in}}%
\pgfpathlineto{\pgfqpoint{1.167580in}{5.226832in}}%
\pgfpathlineto{\pgfqpoint{1.170951in}{5.215937in}}%
\pgfpathlineto{\pgfqpoint{1.171372in}{5.215686in}}%
\pgfpathlineto{\pgfqpoint{1.173478in}{5.210783in}}%
\pgfpathlineto{\pgfqpoint{1.173900in}{5.210448in}}%
\pgfpathlineto{\pgfqpoint{1.176427in}{5.202696in}}%
\pgfpathlineto{\pgfqpoint{1.176849in}{5.202068in}}%
\pgfpathlineto{\pgfqpoint{1.177270in}{5.202696in}}%
\pgfpathlineto{\pgfqpoint{1.180219in}{5.209023in}}%
\pgfpathlineto{\pgfqpoint{1.180640in}{5.206467in}}%
\pgfpathlineto{\pgfqpoint{1.181483in}{5.207683in}}%
\pgfpathlineto{\pgfqpoint{1.181904in}{5.207976in}}%
\pgfpathlineto{\pgfqpoint{1.184010in}{5.213800in}}%
\pgfpathlineto{\pgfqpoint{1.185695in}{5.216733in}}%
\pgfpathlineto{\pgfqpoint{1.187802in}{5.220840in}}%
\pgfpathlineto{\pgfqpoint{1.188223in}{5.220672in}}%
\pgfpathlineto{\pgfqpoint{1.189908in}{5.224150in}}%
\pgfpathlineto{\pgfqpoint{1.190329in}{5.223857in}}%
\pgfpathlineto{\pgfqpoint{1.194542in}{5.221887in}}%
\pgfpathlineto{\pgfqpoint{1.195806in}{5.222097in}}%
\pgfpathlineto{\pgfqpoint{1.196227in}{5.221259in}}%
\pgfpathlineto{\pgfqpoint{1.196649in}{5.221971in}}%
\pgfpathlineto{\pgfqpoint{1.198755in}{5.221678in}}%
\pgfpathlineto{\pgfqpoint{1.202125in}{5.219080in}}%
\pgfpathlineto{\pgfqpoint{1.202546in}{5.219080in}}%
\pgfpathlineto{\pgfqpoint{1.203389in}{5.217697in}}%
\pgfpathlineto{\pgfqpoint{1.203810in}{5.218074in}}%
\pgfpathlineto{\pgfqpoint{1.205495in}{5.217697in}}%
\pgfpathlineto{\pgfqpoint{1.206759in}{5.217739in}}%
\pgfpathlineto{\pgfqpoint{1.207602in}{5.220044in}}%
\pgfpathlineto{\pgfqpoint{1.208023in}{5.218996in}}%
\pgfpathlineto{\pgfqpoint{1.208866in}{5.219625in}}%
\pgfpathlineto{\pgfqpoint{1.210972in}{5.217111in}}%
\pgfpathlineto{\pgfqpoint{1.212657in}{5.216482in}}%
\pgfpathlineto{\pgfqpoint{1.213500in}{5.215686in}}%
\pgfpathlineto{\pgfqpoint{1.215185in}{5.217236in}}%
\pgfpathlineto{\pgfqpoint{1.216027in}{5.217488in}}%
\pgfpathlineto{\pgfqpoint{1.218134in}{5.219248in}}%
\pgfpathlineto{\pgfqpoint{1.220240in}{5.221552in}}%
\pgfpathlineto{\pgfqpoint{1.220661in}{5.219038in}}%
\pgfpathlineto{\pgfqpoint{1.221504in}{5.220588in}}%
\pgfpathlineto{\pgfqpoint{1.221925in}{5.219918in}}%
\pgfpathlineto{\pgfqpoint{1.222347in}{5.220882in}}%
\pgfpathlineto{\pgfqpoint{1.229087in}{5.228759in}}%
\pgfpathlineto{\pgfqpoint{1.229508in}{5.228424in}}%
\pgfpathlineto{\pgfqpoint{1.237513in}{5.231734in}}%
\pgfpathlineto{\pgfqpoint{1.238355in}{5.229933in}}%
\pgfpathlineto{\pgfqpoint{1.238776in}{5.230435in}}%
\pgfpathlineto{\pgfqpoint{1.240040in}{5.231357in}}%
\pgfpathlineto{\pgfqpoint{1.240883in}{5.227879in}}%
\pgfpathlineto{\pgfqpoint{1.241304in}{5.227963in}}%
\pgfpathlineto{\pgfqpoint{1.246781in}{5.231399in}}%
\pgfpathlineto{\pgfqpoint{1.248466in}{5.235422in}}%
\pgfpathlineto{\pgfqpoint{1.249730in}{5.235212in}}%
\pgfpathlineto{\pgfqpoint{1.251836in}{5.237559in}}%
\pgfpathlineto{\pgfqpoint{1.254785in}{5.237894in}}%
\pgfpathlineto{\pgfqpoint{1.256470in}{5.239067in}}%
\pgfpathlineto{\pgfqpoint{1.260262in}{5.242377in}}%
\pgfpathlineto{\pgfqpoint{1.261104in}{5.239528in}}%
\pgfpathlineto{\pgfqpoint{1.261525in}{5.241288in}}%
\pgfpathlineto{\pgfqpoint{1.262789in}{5.241204in}}%
\pgfpathlineto{\pgfqpoint{1.263211in}{5.240408in}}%
\pgfpathlineto{\pgfqpoint{1.263632in}{5.266136in}}%
\pgfpathlineto{\pgfqpoint{1.264474in}{5.264837in}}%
\pgfpathlineto{\pgfqpoint{1.268266in}{5.274391in}}%
\pgfpathlineto{\pgfqpoint{1.270794in}{5.263370in}}%
\pgfpathlineto{\pgfqpoint{1.271636in}{5.265340in}}%
\pgfpathlineto{\pgfqpoint{1.273321in}{5.266974in}}%
\pgfpathlineto{\pgfqpoint{1.273742in}{5.266471in}}%
\pgfpathlineto{\pgfqpoint{1.274585in}{5.267980in}}%
\pgfpathlineto{\pgfqpoint{1.275006in}{5.266974in}}%
\pgfpathlineto{\pgfqpoint{1.276691in}{5.263370in}}%
\pgfpathlineto{\pgfqpoint{1.277955in}{5.234667in}}%
\pgfpathlineto{\pgfqpoint{1.281747in}{5.211998in}}%
\pgfpathlineto{\pgfqpoint{1.282168in}{5.211454in}}%
\pgfpathlineto{\pgfqpoint{1.282589in}{5.211789in}}%
\pgfpathlineto{\pgfqpoint{1.284274in}{5.213214in}}%
\pgfpathlineto{\pgfqpoint{1.288487in}{5.191341in}}%
\pgfpathlineto{\pgfqpoint{1.290172in}{5.189958in}}%
\pgfpathlineto{\pgfqpoint{1.293543in}{5.190419in}}%
\pgfpathlineto{\pgfqpoint{1.296070in}{5.197207in}}%
\pgfpathlineto{\pgfqpoint{1.300283in}{5.208563in}}%
\pgfpathlineto{\pgfqpoint{1.308287in}{5.225700in}}%
\pgfpathlineto{\pgfqpoint{1.309130in}{5.225072in}}%
\pgfpathlineto{\pgfqpoint{1.310394in}{5.225575in}}%
\pgfpathlineto{\pgfqpoint{1.311236in}{5.224946in}}%
\pgfpathlineto{\pgfqpoint{1.311658in}{5.225784in}}%
\pgfpathlineto{\pgfqpoint{1.312079in}{5.226748in}}%
\pgfpathlineto{\pgfqpoint{1.312921in}{5.226161in}}%
\pgfpathlineto{\pgfqpoint{1.314606in}{5.227167in}}%
\pgfpathlineto{\pgfqpoint{1.315028in}{5.226916in}}%
\pgfpathlineto{\pgfqpoint{1.315449in}{5.227796in}}%
\pgfpathlineto{\pgfqpoint{1.317134in}{5.229053in}}%
\pgfpathlineto{\pgfqpoint{1.317555in}{5.228466in}}%
\pgfpathlineto{\pgfqpoint{1.324296in}{5.231441in}}%
\pgfpathlineto{\pgfqpoint{1.324717in}{5.231986in}}%
\pgfpathlineto{\pgfqpoint{1.325138in}{5.231148in}}%
\pgfpathlineto{\pgfqpoint{1.326402in}{5.230938in}}%
\pgfpathlineto{\pgfqpoint{1.328087in}{5.232489in}}%
\pgfpathlineto{\pgfqpoint{1.328930in}{5.231651in}}%
\pgfpathlineto{\pgfqpoint{1.332721in}{5.223731in}}%
\pgfpathlineto{\pgfqpoint{1.340726in}{5.208730in}}%
\pgfpathlineto{\pgfqpoint{1.341990in}{5.208060in}}%
\pgfpathlineto{\pgfqpoint{1.342411in}{5.206886in}}%
\pgfpathlineto{\pgfqpoint{1.343253in}{5.207599in}}%
\pgfpathlineto{\pgfqpoint{1.344517in}{5.208227in}}%
\pgfpathlineto{\pgfqpoint{1.344939in}{5.207808in}}%
\pgfpathlineto{\pgfqpoint{1.347045in}{5.208018in}}%
\pgfpathlineto{\pgfqpoint{1.349994in}{5.207976in}}%
\pgfpathlineto{\pgfqpoint{1.352100in}{5.208395in}}%
\pgfpathlineto{\pgfqpoint{1.353785in}{5.208604in}}%
\pgfpathlineto{\pgfqpoint{1.355470in}{5.208185in}}%
\pgfpathlineto{\pgfqpoint{1.359683in}{5.214010in}}%
\pgfpathlineto{\pgfqpoint{1.361790in}{5.212040in}}%
\pgfpathlineto{\pgfqpoint{1.364317in}{5.210658in}}%
\pgfpathlineto{\pgfqpoint{1.365160in}{5.209736in}}%
\pgfpathlineto{\pgfqpoint{1.373164in}{5.200936in}}%
\pgfpathlineto{\pgfqpoint{1.376113in}{5.204372in}}%
\pgfpathlineto{\pgfqpoint{1.376956in}{5.205210in}}%
\pgfpathlineto{\pgfqpoint{1.380747in}{5.209736in}}%
\pgfpathlineto{\pgfqpoint{1.382854in}{5.212082in}}%
\pgfpathlineto{\pgfqpoint{1.384960in}{5.217362in}}%
\pgfpathlineto{\pgfqpoint{1.385802in}{5.217949in}}%
\pgfpathlineto{\pgfqpoint{1.388751in}{5.222642in}}%
\pgfpathlineto{\pgfqpoint{1.389173in}{5.222264in}}%
\pgfpathlineto{\pgfqpoint{1.390015in}{5.224318in}}%
\pgfpathlineto{\pgfqpoint{1.390858in}{5.224024in}}%
\pgfpathlineto{\pgfqpoint{1.393807in}{5.228131in}}%
\pgfpathlineto{\pgfqpoint{1.394228in}{5.227921in}}%
\pgfpathlineto{\pgfqpoint{1.395913in}{5.230729in}}%
\pgfpathlineto{\pgfqpoint{1.397177in}{5.230226in}}%
\pgfpathlineto{\pgfqpoint{1.398020in}{5.229430in}}%
\pgfpathlineto{\pgfqpoint{1.399705in}{5.231315in}}%
\pgfpathlineto{\pgfqpoint{1.400968in}{5.230771in}}%
\pgfpathlineto{\pgfqpoint{1.402654in}{5.231818in}}%
\pgfpathlineto{\pgfqpoint{1.403496in}{5.230896in}}%
\pgfpathlineto{\pgfqpoint{1.403917in}{5.231734in}}%
\pgfpathlineto{\pgfqpoint{1.404339in}{5.231441in}}%
\pgfpathlineto{\pgfqpoint{1.404760in}{5.232489in}}%
\pgfpathlineto{\pgfqpoint{1.406445in}{5.233746in}}%
\pgfpathlineto{\pgfqpoint{1.406866in}{5.232908in}}%
\pgfpathlineto{\pgfqpoint{1.407288in}{5.233829in}}%
\pgfpathlineto{\pgfqpoint{1.409394in}{5.235715in}}%
\pgfpathlineto{\pgfqpoint{1.410658in}{5.235631in}}%
\pgfpathlineto{\pgfqpoint{1.411500in}{5.241917in}}%
\pgfpathlineto{\pgfqpoint{1.411922in}{5.240408in}}%
\pgfpathlineto{\pgfqpoint{1.414028in}{5.241497in}}%
\pgfpathlineto{\pgfqpoint{1.415292in}{5.239863in}}%
\pgfpathlineto{\pgfqpoint{1.415713in}{5.238983in}}%
\pgfpathlineto{\pgfqpoint{1.416977in}{5.257295in}}%
\pgfpathlineto{\pgfqpoint{1.418241in}{5.255870in}}%
\pgfpathlineto{\pgfqpoint{1.426245in}{5.236511in}}%
\pgfpathlineto{\pgfqpoint{1.429194in}{5.233159in}}%
\pgfpathlineto{\pgfqpoint{1.430458in}{5.214177in}}%
\pgfpathlineto{\pgfqpoint{1.435092in}{5.217278in}}%
\pgfpathlineto{\pgfqpoint{1.435513in}{5.216692in}}%
\pgfpathlineto{\pgfqpoint{1.435935in}{5.217027in}}%
\pgfpathlineto{\pgfqpoint{1.437620in}{5.218870in}}%
\pgfpathlineto{\pgfqpoint{1.438041in}{5.218074in}}%
\pgfpathlineto{\pgfqpoint{1.438462in}{5.218661in}}%
\pgfpathlineto{\pgfqpoint{1.441411in}{5.220337in}}%
\pgfpathlineto{\pgfqpoint{1.441832in}{5.219667in}}%
\pgfpathlineto{\pgfqpoint{1.442254in}{5.220253in}}%
\pgfpathlineto{\pgfqpoint{1.443939in}{5.221259in}}%
\pgfpathlineto{\pgfqpoint{1.444781in}{5.221259in}}%
\pgfpathlineto{\pgfqpoint{1.445203in}{5.221678in}}%
\pgfpathlineto{\pgfqpoint{1.445624in}{5.220588in}}%
\pgfpathlineto{\pgfqpoint{1.458684in}{5.224108in}}%
\pgfpathlineto{\pgfqpoint{1.462475in}{5.231567in}}%
\pgfpathlineto{\pgfqpoint{1.464160in}{5.235086in}}%
\pgfpathlineto{\pgfqpoint{1.465003in}{5.236008in}}%
\pgfpathlineto{\pgfqpoint{1.468373in}{5.241037in}}%
\pgfpathlineto{\pgfqpoint{1.471322in}{5.247531in}}%
\pgfpathlineto{\pgfqpoint{1.471743in}{5.248621in}}%
\pgfpathlineto{\pgfqpoint{1.472586in}{5.248202in}}%
\pgfpathlineto{\pgfqpoint{1.480590in}{5.251889in}}%
\pgfpathlineto{\pgfqpoint{1.481011in}{5.250967in}}%
\pgfpathlineto{\pgfqpoint{1.481433in}{5.251763in}}%
\pgfpathlineto{\pgfqpoint{1.483118in}{5.253859in}}%
\pgfpathlineto{\pgfqpoint{1.483539in}{5.253565in}}%
\pgfpathlineto{\pgfqpoint{1.484382in}{5.253859in}}%
\pgfpathlineto{\pgfqpoint{1.486067in}{5.251428in}}%
\pgfpathlineto{\pgfqpoint{1.486488in}{5.251973in}}%
\pgfpathlineto{\pgfqpoint{1.486909in}{5.250716in}}%
\pgfpathlineto{\pgfqpoint{1.491122in}{5.241204in}}%
\pgfpathlineto{\pgfqpoint{1.496599in}{5.228173in}}%
\pgfpathlineto{\pgfqpoint{1.498284in}{5.225365in}}%
\pgfpathlineto{\pgfqpoint{1.498705in}{5.225533in}}%
\pgfpathlineto{\pgfqpoint{1.500811in}{5.223019in}}%
\pgfpathlineto{\pgfqpoint{1.501654in}{5.223270in}}%
\pgfpathlineto{\pgfqpoint{1.502075in}{5.222306in}}%
\pgfpathlineto{\pgfqpoint{1.502497in}{5.223103in}}%
\pgfpathlineto{\pgfqpoint{1.504182in}{5.223857in}}%
\pgfpathlineto{\pgfqpoint{1.505024in}{5.223815in}}%
\pgfpathlineto{\pgfqpoint{1.507973in}{5.228843in}}%
\pgfpathlineto{\pgfqpoint{1.508394in}{5.228424in}}%
\pgfpathlineto{\pgfqpoint{1.508816in}{5.229220in}}%
\pgfpathlineto{\pgfqpoint{1.510501in}{5.231483in}}%
\pgfpathlineto{\pgfqpoint{1.511343in}{5.232028in}}%
\pgfpathlineto{\pgfqpoint{1.516820in}{5.237307in}}%
\pgfpathlineto{\pgfqpoint{1.518084in}{5.237894in}}%
\pgfpathlineto{\pgfqpoint{1.519348in}{5.237852in}}%
\pgfpathlineto{\pgfqpoint{1.519769in}{5.236637in}}%
\pgfpathlineto{\pgfqpoint{1.520611in}{5.237433in}}%
\pgfpathlineto{\pgfqpoint{1.521454in}{5.237014in}}%
\pgfpathlineto{\pgfqpoint{1.521875in}{5.237684in}}%
\pgfpathlineto{\pgfqpoint{1.522718in}{5.237098in}}%
\pgfpathlineto{\pgfqpoint{1.524403in}{5.237433in}}%
\pgfpathlineto{\pgfqpoint{1.524824in}{5.236972in}}%
\pgfpathlineto{\pgfqpoint{1.525246in}{5.237601in}}%
\pgfpathlineto{\pgfqpoint{1.526931in}{5.238020in}}%
\pgfpathlineto{\pgfqpoint{1.527352in}{5.237182in}}%
\pgfpathlineto{\pgfqpoint{1.527773in}{5.237601in}}%
\pgfpathlineto{\pgfqpoint{1.537463in}{5.248495in}}%
\pgfpathlineto{\pgfqpoint{1.538305in}{5.250045in}}%
\pgfpathlineto{\pgfqpoint{1.547573in}{5.211202in}}%
\pgfpathlineto{\pgfqpoint{1.548837in}{5.212711in}}%
\pgfpathlineto{\pgfqpoint{1.550944in}{5.215015in}}%
\pgfpathlineto{\pgfqpoint{1.553050in}{5.213297in}}%
\pgfpathlineto{\pgfqpoint{1.554735in}{5.214010in}}%
\pgfpathlineto{\pgfqpoint{1.555156in}{5.213339in}}%
\pgfpathlineto{\pgfqpoint{1.555578in}{5.213758in}}%
\pgfpathlineto{\pgfqpoint{1.556841in}{5.214974in}}%
\pgfpathlineto{\pgfqpoint{1.557263in}{5.214555in}}%
\pgfpathlineto{\pgfqpoint{1.557684in}{5.214052in}}%
\pgfpathlineto{\pgfqpoint{1.558105in}{5.214764in}}%
\pgfpathlineto{\pgfqpoint{1.558527in}{5.215183in}}%
\pgfpathlineto{\pgfqpoint{1.558948in}{5.214722in}}%
\pgfpathlineto{\pgfqpoint{1.561054in}{5.211957in}}%
\pgfpathlineto{\pgfqpoint{1.566110in}{5.182919in}}%
\pgfpathlineto{\pgfqpoint{1.566952in}{5.183044in}}%
\pgfpathlineto{\pgfqpoint{1.572429in}{5.177513in}}%
\pgfpathlineto{\pgfqpoint{1.574535in}{5.178225in}}%
\pgfpathlineto{\pgfqpoint{1.574956in}{5.177597in}}%
\pgfpathlineto{\pgfqpoint{1.575378in}{5.178435in}}%
\pgfpathlineto{\pgfqpoint{1.577063in}{5.179860in}}%
\pgfpathlineto{\pgfqpoint{1.577905in}{5.179860in}}%
\pgfpathlineto{\pgfqpoint{1.582539in}{5.185516in}}%
\pgfpathlineto{\pgfqpoint{1.587173in}{5.195112in}}%
\pgfpathlineto{\pgfqpoint{1.588016in}{5.196746in}}%
\pgfpathlineto{\pgfqpoint{1.589701in}{5.200140in}}%
\pgfpathlineto{\pgfqpoint{1.590122in}{5.199805in}}%
\pgfpathlineto{\pgfqpoint{1.594756in}{5.209778in}}%
\pgfpathlineto{\pgfqpoint{1.595178in}{5.209401in}}%
\pgfpathlineto{\pgfqpoint{1.595599in}{5.210322in}}%
\pgfpathlineto{\pgfqpoint{1.601076in}{5.217865in}}%
\pgfpathlineto{\pgfqpoint{1.602761in}{5.216608in}}%
\pgfpathlineto{\pgfqpoint{1.603603in}{5.217739in}}%
\pgfpathlineto{\pgfqpoint{1.604025in}{5.216817in}}%
\pgfpathlineto{\pgfqpoint{1.609922in}{5.213591in}}%
\pgfpathlineto{\pgfqpoint{1.611608in}{5.209233in}}%
\pgfpathlineto{\pgfqpoint{1.612029in}{5.209401in}}%
\pgfpathlineto{\pgfqpoint{1.612871in}{5.208395in}}%
\pgfpathlineto{\pgfqpoint{1.615399in}{5.207850in}}%
\pgfpathlineto{\pgfqpoint{1.620454in}{5.212837in}}%
\pgfpathlineto{\pgfqpoint{1.622140in}{5.217111in}}%
\pgfpathlineto{\pgfqpoint{1.623403in}{5.217152in}}%
\pgfpathlineto{\pgfqpoint{1.625088in}{5.219583in}}%
\pgfpathlineto{\pgfqpoint{1.625510in}{5.219248in}}%
\pgfpathlineto{\pgfqpoint{1.633093in}{5.222893in}}%
\pgfpathlineto{\pgfqpoint{1.634357in}{5.220211in}}%
\pgfpathlineto{\pgfqpoint{1.636463in}{5.216398in}}%
\pgfpathlineto{\pgfqpoint{1.638991in}{5.213884in}}%
\pgfpathlineto{\pgfqpoint{1.642782in}{5.213046in}}%
\pgfpathlineto{\pgfqpoint{1.644046in}{5.211454in}}%
\pgfpathlineto{\pgfqpoint{1.644467in}{5.212711in}}%
\pgfpathlineto{\pgfqpoint{1.646152in}{5.215393in}}%
\pgfpathlineto{\pgfqpoint{1.646574in}{5.214429in}}%
\pgfpathlineto{\pgfqpoint{1.646995in}{5.215225in}}%
\pgfpathlineto{\pgfqpoint{1.648680in}{5.217739in}}%
\pgfpathlineto{\pgfqpoint{1.649101in}{5.216859in}}%
\pgfpathlineto{\pgfqpoint{1.650786in}{5.217697in}}%
\pgfpathlineto{\pgfqpoint{1.651208in}{5.218745in}}%
\pgfpathlineto{\pgfqpoint{1.651629in}{5.217697in}}%
\pgfpathlineto{\pgfqpoint{1.655420in}{5.215393in}}%
\pgfpathlineto{\pgfqpoint{1.656684in}{5.214261in}}%
\pgfpathlineto{\pgfqpoint{1.657527in}{5.214806in}}%
\pgfpathlineto{\pgfqpoint{1.659633in}{5.209191in}}%
\pgfpathlineto{\pgfqpoint{1.660476in}{5.208395in}}%
\pgfpathlineto{\pgfqpoint{1.661318in}{5.207808in}}%
\pgfpathlineto{\pgfqpoint{1.661740in}{5.206300in}}%
\pgfpathlineto{\pgfqpoint{1.662161in}{5.206677in}}%
\pgfpathlineto{\pgfqpoint{1.667638in}{5.216189in}}%
\pgfpathlineto{\pgfqpoint{1.668059in}{5.216231in}}%
\pgfpathlineto{\pgfqpoint{1.670586in}{5.223815in}}%
\pgfpathlineto{\pgfqpoint{1.672693in}{5.226664in}}%
\pgfpathlineto{\pgfqpoint{1.673957in}{5.225198in}}%
\pgfpathlineto{\pgfqpoint{1.674799in}{5.226706in}}%
\pgfpathlineto{\pgfqpoint{1.675221in}{5.227628in}}%
\pgfpathlineto{\pgfqpoint{1.676063in}{5.227209in}}%
\pgfpathlineto{\pgfqpoint{1.677748in}{5.226748in}}%
\pgfpathlineto{\pgfqpoint{1.679855in}{5.223061in}}%
\pgfpathlineto{\pgfqpoint{1.681118in}{5.223270in}}%
\pgfpathlineto{\pgfqpoint{1.681540in}{5.222851in}}%
\pgfpathlineto{\pgfqpoint{1.683225in}{5.218535in}}%
\pgfpathlineto{\pgfqpoint{1.684067in}{5.218326in}}%
\pgfpathlineto{\pgfqpoint{1.685753in}{5.216650in}}%
\pgfpathlineto{\pgfqpoint{1.687859in}{5.218368in}}%
\pgfpathlineto{\pgfqpoint{1.688280in}{5.217571in}}%
\pgfpathlineto{\pgfqpoint{1.688701in}{5.218200in}}%
\pgfpathlineto{\pgfqpoint{1.691650in}{5.223061in}}%
\pgfpathlineto{\pgfqpoint{1.692072in}{5.223563in}}%
\pgfpathlineto{\pgfqpoint{1.693757in}{5.228675in}}%
\pgfpathlineto{\pgfqpoint{1.694599in}{5.227796in}}%
\pgfpathlineto{\pgfqpoint{1.696706in}{5.231399in}}%
\pgfpathlineto{\pgfqpoint{1.697127in}{5.230519in}}%
\pgfpathlineto{\pgfqpoint{1.697548in}{5.231148in}}%
\pgfpathlineto{\pgfqpoint{1.699233in}{5.233955in}}%
\pgfpathlineto{\pgfqpoint{1.700497in}{5.233620in}}%
\pgfpathlineto{\pgfqpoint{1.703025in}{5.234542in}}%
\pgfpathlineto{\pgfqpoint{1.704289in}{5.233117in}}%
\pgfpathlineto{\pgfqpoint{1.704710in}{5.233075in}}%
\pgfpathlineto{\pgfqpoint{1.705131in}{5.234290in}}%
\pgfpathlineto{\pgfqpoint{1.705553in}{5.233578in}}%
\pgfpathlineto{\pgfqpoint{1.707238in}{5.230268in}}%
\pgfpathlineto{\pgfqpoint{1.708923in}{5.230435in}}%
\pgfpathlineto{\pgfqpoint{1.710187in}{5.229220in}}%
\pgfpathlineto{\pgfqpoint{1.711450in}{5.229472in}}%
\pgfpathlineto{\pgfqpoint{1.713557in}{5.229220in}}%
\pgfpathlineto{\pgfqpoint{1.715663in}{5.229094in}}%
\pgfpathlineto{\pgfqpoint{1.720719in}{5.226790in}}%
\pgfpathlineto{\pgfqpoint{1.721561in}{5.228801in}}%
\pgfpathlineto{\pgfqpoint{1.721982in}{5.227754in}}%
\pgfpathlineto{\pgfqpoint{1.724510in}{5.228592in}}%
\pgfpathlineto{\pgfqpoint{1.726616in}{5.230184in}}%
\pgfpathlineto{\pgfqpoint{1.727038in}{5.229891in}}%
\pgfpathlineto{\pgfqpoint{1.727880in}{5.231399in}}%
\pgfpathlineto{\pgfqpoint{1.728302in}{5.230519in}}%
\pgfpathlineto{\pgfqpoint{1.741783in}{5.231064in}}%
\pgfpathlineto{\pgfqpoint{1.742204in}{5.230310in}}%
\pgfpathlineto{\pgfqpoint{1.742625in}{5.231022in}}%
\pgfpathlineto{\pgfqpoint{1.744310in}{5.234039in}}%
\pgfpathlineto{\pgfqpoint{1.744731in}{5.233243in}}%
\pgfpathlineto{\pgfqpoint{1.745995in}{5.232949in}}%
\pgfpathlineto{\pgfqpoint{1.748102in}{5.235506in}}%
\pgfpathlineto{\pgfqpoint{1.748523in}{5.234248in}}%
\pgfpathlineto{\pgfqpoint{1.749366in}{5.234961in}}%
\pgfpathlineto{\pgfqpoint{1.749787in}{5.234416in}}%
\pgfpathlineto{\pgfqpoint{1.750208in}{5.234877in}}%
\pgfpathlineto{\pgfqpoint{1.751472in}{5.234793in}}%
\pgfpathlineto{\pgfqpoint{1.751893in}{5.235212in}}%
\pgfpathlineto{\pgfqpoint{1.752314in}{5.234584in}}%
\pgfpathlineto{\pgfqpoint{1.753578in}{5.234919in}}%
\pgfpathlineto{\pgfqpoint{1.754421in}{5.235966in}}%
\pgfpathlineto{\pgfqpoint{1.754842in}{5.234961in}}%
\pgfpathlineto{\pgfqpoint{1.756949in}{5.235925in}}%
\pgfpathlineto{\pgfqpoint{1.758634in}{5.234667in}}%
\pgfpathlineto{\pgfqpoint{1.761583in}{5.234542in}}%
\pgfpathlineto{\pgfqpoint{1.764110in}{5.236846in}}%
\pgfpathlineto{\pgfqpoint{1.765795in}{5.236637in}}%
\pgfpathlineto{\pgfqpoint{1.766638in}{5.237182in}}%
\pgfpathlineto{\pgfqpoint{1.767059in}{5.238355in}}%
\pgfpathlineto{\pgfqpoint{1.767902in}{5.237601in}}%
\pgfpathlineto{\pgfqpoint{1.771272in}{5.244514in}}%
\pgfpathlineto{\pgfqpoint{1.773378in}{5.250004in}}%
\pgfpathlineto{\pgfqpoint{1.773800in}{5.249585in}}%
\pgfpathlineto{\pgfqpoint{1.775063in}{5.250548in}}%
\pgfpathlineto{\pgfqpoint{1.777170in}{5.254403in}}%
\pgfpathlineto{\pgfqpoint{1.777591in}{5.254278in}}%
\pgfpathlineto{\pgfqpoint{1.779698in}{5.258845in}}%
\pgfpathlineto{\pgfqpoint{1.780119in}{5.258174in}}%
\pgfpathlineto{\pgfqpoint{1.784753in}{5.250716in}}%
\pgfpathlineto{\pgfqpoint{1.788966in}{5.239067in}}%
\pgfpathlineto{\pgfqpoint{1.792757in}{5.225533in}}%
\pgfpathlineto{\pgfqpoint{1.794442in}{5.224234in}}%
\pgfpathlineto{\pgfqpoint{1.796549in}{5.220505in}}%
\pgfpathlineto{\pgfqpoint{1.796970in}{5.219960in}}%
\pgfpathlineto{\pgfqpoint{1.798655in}{5.215602in}}%
\pgfpathlineto{\pgfqpoint{1.801604in}{5.215937in}}%
\pgfpathlineto{\pgfqpoint{1.802868in}{5.216398in}}%
\pgfpathlineto{\pgfqpoint{1.810030in}{5.218912in}}%
\pgfpathlineto{\pgfqpoint{1.811715in}{5.220295in}}%
\pgfpathlineto{\pgfqpoint{1.812979in}{5.219708in}}%
\pgfpathlineto{\pgfqpoint{1.813821in}{5.218619in}}%
\pgfpathlineto{\pgfqpoint{1.814242in}{5.219331in}}%
\pgfpathlineto{\pgfqpoint{1.814664in}{5.219876in}}%
\pgfpathlineto{\pgfqpoint{1.815085in}{5.219164in}}%
\pgfpathlineto{\pgfqpoint{1.816349in}{5.219625in}}%
\pgfpathlineto{\pgfqpoint{1.818034in}{5.220421in}}%
\pgfpathlineto{\pgfqpoint{1.819719in}{5.220295in}}%
\pgfpathlineto{\pgfqpoint{1.820140in}{5.219373in}}%
\pgfpathlineto{\pgfqpoint{1.820562in}{5.220002in}}%
\pgfpathlineto{\pgfqpoint{1.822247in}{5.221385in}}%
\pgfpathlineto{\pgfqpoint{1.827723in}{5.225868in}}%
\pgfpathlineto{\pgfqpoint{1.832779in}{5.238481in}}%
\pgfpathlineto{\pgfqpoint{1.838255in}{5.244724in}}%
\pgfpathlineto{\pgfqpoint{1.839940in}{5.246316in}}%
\pgfpathlineto{\pgfqpoint{1.841625in}{5.244850in}}%
\pgfpathlineto{\pgfqpoint{1.842468in}{5.245436in}}%
\pgfpathlineto{\pgfqpoint{1.844574in}{5.243215in}}%
\pgfpathlineto{\pgfqpoint{1.846259in}{5.243215in}}%
\pgfpathlineto{\pgfqpoint{1.846681in}{5.242419in}}%
\pgfpathlineto{\pgfqpoint{1.847102in}{5.242713in}}%
\pgfpathlineto{\pgfqpoint{1.847523in}{5.243425in}}%
\pgfpathlineto{\pgfqpoint{1.847945in}{5.241833in}}%
\pgfpathlineto{\pgfqpoint{1.848787in}{5.242336in}}%
\pgfpathlineto{\pgfqpoint{1.850472in}{5.242000in}}%
\pgfpathlineto{\pgfqpoint{1.850894in}{5.242671in}}%
\pgfpathlineto{\pgfqpoint{1.851315in}{5.242252in}}%
\pgfpathlineto{\pgfqpoint{1.852157in}{5.241330in}}%
\pgfpathlineto{\pgfqpoint{1.852579in}{5.242084in}}%
\pgfpathlineto{\pgfqpoint{1.861847in}{5.231692in}}%
\pgfpathlineto{\pgfqpoint{1.865638in}{5.228717in}}%
\pgfpathlineto{\pgfqpoint{1.866902in}{5.227418in}}%
\pgfpathlineto{\pgfqpoint{1.868587in}{5.228927in}}%
\pgfpathlineto{\pgfqpoint{1.869008in}{5.228215in}}%
\pgfpathlineto{\pgfqpoint{1.869430in}{5.228885in}}%
\pgfpathlineto{\pgfqpoint{1.871957in}{5.229262in}}%
\pgfpathlineto{\pgfqpoint{1.873221in}{5.227670in}}%
\pgfpathlineto{\pgfqpoint{1.874064in}{5.228550in}}%
\pgfpathlineto{\pgfqpoint{1.874906in}{5.230100in}}%
\pgfpathlineto{\pgfqpoint{1.875749in}{5.229472in}}%
\pgfpathlineto{\pgfqpoint{1.877434in}{5.230771in}}%
\pgfpathlineto{\pgfqpoint{1.879119in}{5.231273in}}%
\pgfpathlineto{\pgfqpoint{1.881226in}{5.233327in}}%
\pgfpathlineto{\pgfqpoint{1.881647in}{5.232237in}}%
\pgfpathlineto{\pgfqpoint{1.885438in}{5.228424in}}%
\pgfpathlineto{\pgfqpoint{1.885860in}{5.229304in}}%
\pgfpathlineto{\pgfqpoint{1.887545in}{5.231064in}}%
\pgfpathlineto{\pgfqpoint{1.889230in}{5.228298in}}%
\pgfpathlineto{\pgfqpoint{1.890072in}{5.228256in}}%
\pgfpathlineto{\pgfqpoint{1.892179in}{5.224485in}}%
\pgfpathlineto{\pgfqpoint{1.893443in}{5.224401in}}%
\pgfpathlineto{\pgfqpoint{1.897655in}{5.224862in}}%
\pgfpathlineto{\pgfqpoint{1.902711in}{5.227712in}}%
\pgfpathlineto{\pgfqpoint{1.903132in}{5.226874in}}%
\pgfpathlineto{\pgfqpoint{1.903553in}{5.227921in}}%
\pgfpathlineto{\pgfqpoint{1.903975in}{5.229304in}}%
\pgfpathlineto{\pgfqpoint{1.904817in}{5.228298in}}%
\pgfpathlineto{\pgfqpoint{1.906502in}{5.230100in}}%
\pgfpathlineto{\pgfqpoint{1.906924in}{5.229597in}}%
\pgfpathlineto{\pgfqpoint{1.909451in}{5.227754in}}%
\pgfpathlineto{\pgfqpoint{1.911558in}{5.228759in}}%
\pgfpathlineto{\pgfqpoint{1.913664in}{5.226664in}}%
\pgfpathlineto{\pgfqpoint{1.914928in}{5.226538in}}%
\pgfpathlineto{\pgfqpoint{1.916192in}{5.226329in}}%
\pgfpathlineto{\pgfqpoint{1.917877in}{5.226916in}}%
\pgfpathlineto{\pgfqpoint{1.918719in}{5.227167in}}%
\pgfpathlineto{\pgfqpoint{1.920404in}{5.227628in}}%
\pgfpathlineto{\pgfqpoint{1.921247in}{5.227083in}}%
\pgfpathlineto{\pgfqpoint{1.927145in}{5.235631in}}%
\pgfpathlineto{\pgfqpoint{1.928409in}{5.236134in}}%
\pgfpathlineto{\pgfqpoint{1.930936in}{5.236846in}}%
\pgfpathlineto{\pgfqpoint{1.932621in}{5.236553in}}%
\pgfpathlineto{\pgfqpoint{1.933885in}{5.236511in}}%
\pgfpathlineto{\pgfqpoint{1.934307in}{5.236846in}}%
\pgfpathlineto{\pgfqpoint{1.936834in}{5.227293in}}%
\pgfpathlineto{\pgfqpoint{1.939783in}{5.215518in}}%
\pgfpathlineto{\pgfqpoint{1.941468in}{5.216356in}}%
\pgfpathlineto{\pgfqpoint{1.941890in}{5.215895in}}%
\pgfpathlineto{\pgfqpoint{1.942311in}{5.216817in}}%
\pgfpathlineto{\pgfqpoint{1.945681in}{5.220672in}}%
\pgfpathlineto{\pgfqpoint{1.947366in}{5.220840in}}%
\pgfpathlineto{\pgfqpoint{1.949051in}{5.221091in}}%
\pgfpathlineto{\pgfqpoint{1.949894in}{5.220421in}}%
\pgfpathlineto{\pgfqpoint{1.951579in}{5.222139in}}%
\pgfpathlineto{\pgfqpoint{1.952000in}{5.221804in}}%
\pgfpathlineto{\pgfqpoint{1.952422in}{5.222558in}}%
\pgfpathlineto{\pgfqpoint{1.952843in}{5.222809in}}%
\pgfpathlineto{\pgfqpoint{1.953685in}{5.221426in}}%
\pgfpathlineto{\pgfqpoint{1.954107in}{5.222139in}}%
\pgfpathlineto{\pgfqpoint{1.954528in}{5.221720in}}%
\pgfpathlineto{\pgfqpoint{1.954949in}{5.222516in}}%
\pgfpathlineto{\pgfqpoint{1.956634in}{5.222474in}}%
\pgfpathlineto{\pgfqpoint{1.957477in}{5.222055in}}%
\pgfpathlineto{\pgfqpoint{1.957898in}{5.222432in}}%
\pgfpathlineto{\pgfqpoint{1.959583in}{5.222893in}}%
\pgfpathlineto{\pgfqpoint{1.961690in}{5.224862in}}%
\pgfpathlineto{\pgfqpoint{1.962532in}{5.224737in}}%
\pgfpathlineto{\pgfqpoint{1.964217in}{5.225491in}}%
\pgfpathlineto{\pgfqpoint{1.964639in}{5.224988in}}%
\pgfpathlineto{\pgfqpoint{1.965060in}{5.225617in}}%
\pgfpathlineto{\pgfqpoint{1.965481in}{5.226161in}}%
\pgfpathlineto{\pgfqpoint{1.970115in}{5.205210in}}%
\pgfpathlineto{\pgfqpoint{1.973064in}{5.190545in}}%
\pgfpathlineto{\pgfqpoint{1.975171in}{5.183170in}}%
\pgfpathlineto{\pgfqpoint{1.976013in}{5.182877in}}%
\pgfpathlineto{\pgfqpoint{1.977698in}{5.182709in}}%
\pgfpathlineto{\pgfqpoint{1.980226in}{5.183338in}}%
\pgfpathlineto{\pgfqpoint{1.985281in}{5.184804in}}%
\pgfpathlineto{\pgfqpoint{1.985703in}{5.185600in}}%
\pgfpathlineto{\pgfqpoint{1.986124in}{5.185014in}}%
\pgfpathlineto{\pgfqpoint{1.986545in}{5.184385in}}%
\pgfpathlineto{\pgfqpoint{1.986966in}{5.185139in}}%
\pgfpathlineto{\pgfqpoint{1.987809in}{5.187151in}}%
\pgfpathlineto{\pgfqpoint{2.000447in}{5.222767in}}%
\pgfpathlineto{\pgfqpoint{2.009715in}{5.213842in}}%
\pgfpathlineto{\pgfqpoint{2.013928in}{5.202403in}}%
\pgfpathlineto{\pgfqpoint{2.014349in}{5.202696in}}%
\pgfpathlineto{\pgfqpoint{2.014771in}{5.202403in}}%
\pgfpathlineto{\pgfqpoint{2.016456in}{5.204498in}}%
\pgfpathlineto{\pgfqpoint{2.018141in}{5.204414in}}%
\pgfpathlineto{\pgfqpoint{2.018562in}{5.203241in}}%
\pgfpathlineto{\pgfqpoint{2.018984in}{5.204205in}}%
\pgfpathlineto{\pgfqpoint{2.020669in}{5.205378in}}%
\pgfpathlineto{\pgfqpoint{2.021932in}{5.217488in}}%
\pgfpathlineto{\pgfqpoint{2.022354in}{5.217278in}}%
\pgfpathlineto{\pgfqpoint{2.023618in}{5.218745in}}%
\pgfpathlineto{\pgfqpoint{2.026988in}{5.226203in}}%
\pgfpathlineto{\pgfqpoint{2.027409in}{5.225742in}}%
\pgfpathlineto{\pgfqpoint{2.027830in}{5.226329in}}%
\pgfpathlineto{\pgfqpoint{2.029094in}{5.226371in}}%
\pgfpathlineto{\pgfqpoint{2.029937in}{5.224779in}}%
\pgfpathlineto{\pgfqpoint{2.030358in}{5.225491in}}%
\pgfpathlineto{\pgfqpoint{2.032043in}{5.226119in}}%
\pgfpathlineto{\pgfqpoint{2.033728in}{5.223731in}}%
\pgfpathlineto{\pgfqpoint{2.034150in}{5.224192in}}%
\pgfpathlineto{\pgfqpoint{2.035413in}{5.212418in}}%
\pgfpathlineto{\pgfqpoint{2.037098in}{5.212585in}}%
\pgfpathlineto{\pgfqpoint{2.037520in}{5.211747in}}%
\pgfpathlineto{\pgfqpoint{2.037941in}{5.212501in}}%
\pgfpathlineto{\pgfqpoint{2.040890in}{5.214848in}}%
\pgfpathlineto{\pgfqpoint{2.042996in}{5.213926in}}%
\pgfpathlineto{\pgfqpoint{2.044681in}{5.215686in}}%
\pgfpathlineto{\pgfqpoint{2.046367in}{5.214429in}}%
\pgfpathlineto{\pgfqpoint{2.046788in}{5.214429in}}%
\pgfpathlineto{\pgfqpoint{2.055213in}{5.193729in}}%
\pgfpathlineto{\pgfqpoint{2.055635in}{5.194274in}}%
\pgfpathlineto{\pgfqpoint{2.056056in}{5.193562in}}%
\pgfpathlineto{\pgfqpoint{2.059005in}{5.186061in}}%
\pgfpathlineto{\pgfqpoint{2.062796in}{5.189790in}}%
\pgfpathlineto{\pgfqpoint{2.069958in}{5.207054in}}%
\pgfpathlineto{\pgfqpoint{2.070379in}{5.207012in}}%
\pgfpathlineto{\pgfqpoint{2.073750in}{5.214555in}}%
\pgfpathlineto{\pgfqpoint{2.075435in}{5.214135in}}%
\pgfpathlineto{\pgfqpoint{2.075856in}{5.215057in}}%
\pgfpathlineto{\pgfqpoint{2.076699in}{5.214387in}}%
\pgfpathlineto{\pgfqpoint{2.080490in}{5.211705in}}%
\pgfpathlineto{\pgfqpoint{2.084282in}{5.207641in}}%
\pgfpathlineto{\pgfqpoint{2.086388in}{5.204875in}}%
\pgfpathlineto{\pgfqpoint{2.093128in}{5.199512in}}%
\pgfpathlineto{\pgfqpoint{2.094814in}{5.198757in}}%
\pgfpathlineto{\pgfqpoint{2.095656in}{5.197500in}}%
\pgfpathlineto{\pgfqpoint{2.096077in}{5.198422in}}%
\pgfpathlineto{\pgfqpoint{2.096499in}{5.197710in}}%
\pgfpathlineto{\pgfqpoint{2.096920in}{5.198171in}}%
\pgfpathlineto{\pgfqpoint{2.098605in}{5.199595in}}%
\pgfpathlineto{\pgfqpoint{2.099448in}{5.200224in}}%
\pgfpathlineto{\pgfqpoint{2.101554in}{5.201732in}}%
\pgfpathlineto{\pgfqpoint{2.105346in}{5.203744in}}%
\pgfpathlineto{\pgfqpoint{2.107452in}{5.206090in}}%
\pgfpathlineto{\pgfqpoint{2.107873in}{5.205797in}}%
\pgfpathlineto{\pgfqpoint{2.108294in}{5.206677in}}%
\pgfpathlineto{\pgfqpoint{2.109980in}{5.208688in}}%
\pgfpathlineto{\pgfqpoint{2.110822in}{5.208563in}}%
\pgfpathlineto{\pgfqpoint{2.112507in}{5.209987in}}%
\pgfpathlineto{\pgfqpoint{2.114192in}{5.209610in}}%
\pgfpathlineto{\pgfqpoint{2.116299in}{5.213884in}}%
\pgfpathlineto{\pgfqpoint{2.116720in}{5.213130in}}%
\pgfpathlineto{\pgfqpoint{2.118826in}{5.214890in}}%
\pgfpathlineto{\pgfqpoint{2.119248in}{5.214219in}}%
\pgfpathlineto{\pgfqpoint{2.119669in}{5.214596in}}%
\pgfpathlineto{\pgfqpoint{2.122618in}{5.216231in}}%
\pgfpathlineto{\pgfqpoint{2.124303in}{5.214848in}}%
\pgfpathlineto{\pgfqpoint{2.127673in}{5.216566in}}%
\pgfpathlineto{\pgfqpoint{2.128095in}{5.216063in}}%
\pgfpathlineto{\pgfqpoint{2.128516in}{5.216608in}}%
\pgfpathlineto{\pgfqpoint{2.131465in}{5.218703in}}%
\pgfpathlineto{\pgfqpoint{2.136520in}{5.221301in}}%
\pgfpathlineto{\pgfqpoint{2.137363in}{5.222600in}}%
\pgfpathlineto{\pgfqpoint{2.139048in}{5.224611in}}%
\pgfpathlineto{\pgfqpoint{2.139890in}{5.225533in}}%
\pgfpathlineto{\pgfqpoint{2.144103in}{5.235673in}}%
\pgfpathlineto{\pgfqpoint{2.144524in}{5.235338in}}%
\pgfpathlineto{\pgfqpoint{2.145367in}{5.234626in}}%
\pgfpathlineto{\pgfqpoint{2.147052in}{5.231525in}}%
\pgfpathlineto{\pgfqpoint{2.149580in}{5.237475in}}%
\pgfpathlineto{\pgfqpoint{2.150422in}{5.239067in}}%
\pgfpathlineto{\pgfqpoint{2.150844in}{5.237852in}}%
\pgfpathlineto{\pgfqpoint{2.153371in}{5.228634in}}%
\pgfpathlineto{\pgfqpoint{2.153793in}{5.229220in}}%
\pgfpathlineto{\pgfqpoint{2.154214in}{5.229891in}}%
\pgfpathlineto{\pgfqpoint{2.154635in}{5.229472in}}%
\pgfpathlineto{\pgfqpoint{2.155899in}{5.228885in}}%
\pgfpathlineto{\pgfqpoint{2.157163in}{5.228843in}}%
\pgfpathlineto{\pgfqpoint{2.158848in}{5.224066in}}%
\pgfpathlineto{\pgfqpoint{2.160533in}{5.222055in}}%
\pgfpathlineto{\pgfqpoint{2.164324in}{5.204289in}}%
\pgfpathlineto{\pgfqpoint{2.164746in}{5.204372in}}%
\pgfpathlineto{\pgfqpoint{2.166431in}{5.207724in}}%
\pgfpathlineto{\pgfqpoint{2.171907in}{5.184301in}}%
\pgfpathlineto{\pgfqpoint{2.173171in}{5.183966in}}%
\pgfpathlineto{\pgfqpoint{2.174856in}{5.182080in}}%
\pgfpathlineto{\pgfqpoint{2.176963in}{5.185139in}}%
\pgfpathlineto{\pgfqpoint{2.190444in}{5.218577in}}%
\pgfpathlineto{\pgfqpoint{2.195920in}{5.219289in}}%
\pgfpathlineto{\pgfqpoint{2.198027in}{5.222306in}}%
\pgfpathlineto{\pgfqpoint{2.199712in}{5.221343in}}%
\pgfpathlineto{\pgfqpoint{2.200133in}{5.221887in}}%
\pgfpathlineto{\pgfqpoint{2.206874in}{5.225868in}}%
\pgfpathlineto{\pgfqpoint{2.210244in}{5.222474in}}%
\pgfpathlineto{\pgfqpoint{2.210665in}{5.222516in}}%
\pgfpathlineto{\pgfqpoint{2.217406in}{5.201732in}}%
\pgfpathlineto{\pgfqpoint{2.220776in}{5.191467in}}%
\pgfpathlineto{\pgfqpoint{2.222461in}{5.188827in}}%
\pgfpathlineto{\pgfqpoint{2.224146in}{5.184804in}}%
\pgfpathlineto{\pgfqpoint{2.224989in}{5.184176in}}%
\pgfpathlineto{\pgfqpoint{2.227095in}{5.187193in}}%
\pgfpathlineto{\pgfqpoint{2.235099in}{5.195489in}}%
\pgfpathlineto{\pgfqpoint{2.240576in}{5.206174in}}%
\pgfpathlineto{\pgfqpoint{2.240997in}{5.206509in}}%
\pgfpathlineto{\pgfqpoint{2.241840in}{5.208856in}}%
\pgfpathlineto{\pgfqpoint{2.242261in}{5.208563in}}%
\pgfpathlineto{\pgfqpoint{2.243525in}{5.208646in}}%
\pgfpathlineto{\pgfqpoint{2.251950in}{5.199973in}}%
\pgfpathlineto{\pgfqpoint{2.253635in}{5.197752in}}%
\pgfpathlineto{\pgfqpoint{2.254478in}{5.196788in}}%
\pgfpathlineto{\pgfqpoint{2.256163in}{5.194274in}}%
\pgfpathlineto{\pgfqpoint{2.258691in}{5.196704in}}%
\pgfpathlineto{\pgfqpoint{2.260797in}{5.200056in}}%
\pgfpathlineto{\pgfqpoint{2.261640in}{5.200936in}}%
\pgfpathlineto{\pgfqpoint{2.269644in}{5.218661in}}%
\pgfpathlineto{\pgfqpoint{2.271329in}{5.216608in}}%
\pgfpathlineto{\pgfqpoint{2.273436in}{5.218326in}}%
\pgfpathlineto{\pgfqpoint{2.275121in}{5.216985in}}%
\pgfpathlineto{\pgfqpoint{2.276806in}{5.217655in}}%
\pgfpathlineto{\pgfqpoint{2.277648in}{5.216231in}}%
\pgfpathlineto{\pgfqpoint{2.278070in}{5.217362in}}%
\pgfpathlineto{\pgfqpoint{2.278491in}{5.217488in}}%
\pgfpathlineto{\pgfqpoint{2.280176in}{5.214680in}}%
\pgfpathlineto{\pgfqpoint{2.281440in}{5.215183in}}%
\pgfpathlineto{\pgfqpoint{2.284389in}{5.217990in}}%
\pgfpathlineto{\pgfqpoint{2.286916in}{5.219625in}}%
\pgfpathlineto{\pgfqpoint{2.289444in}{5.223103in}}%
\pgfpathlineto{\pgfqpoint{2.290287in}{5.222306in}}%
\pgfpathlineto{\pgfqpoint{2.291129in}{5.224318in}}%
\pgfpathlineto{\pgfqpoint{2.291550in}{5.222767in}}%
\pgfpathlineto{\pgfqpoint{2.294078in}{5.225994in}}%
\pgfpathlineto{\pgfqpoint{2.295342in}{5.225952in}}%
\pgfpathlineto{\pgfqpoint{2.299555in}{5.222642in}}%
\pgfpathlineto{\pgfqpoint{2.300397in}{5.223019in}}%
\pgfpathlineto{\pgfqpoint{2.302504in}{5.220924in}}%
\pgfpathlineto{\pgfqpoint{2.307980in}{5.214513in}}%
\pgfpathlineto{\pgfqpoint{2.309665in}{5.215393in}}%
\pgfpathlineto{\pgfqpoint{2.311351in}{5.214596in}}%
\pgfpathlineto{\pgfqpoint{2.313457in}{5.216189in}}%
\pgfpathlineto{\pgfqpoint{2.313878in}{5.215434in}}%
\pgfpathlineto{\pgfqpoint{2.314299in}{5.216692in}}%
\pgfpathlineto{\pgfqpoint{2.315985in}{5.217655in}}%
\pgfpathlineto{\pgfqpoint{2.320619in}{5.216356in}}%
\pgfpathlineto{\pgfqpoint{2.322304in}{5.217362in}}%
\pgfpathlineto{\pgfqpoint{2.323989in}{5.217488in}}%
\pgfpathlineto{\pgfqpoint{2.326095in}{5.218954in}}%
\pgfpathlineto{\pgfqpoint{2.327780in}{5.219164in}}%
\pgfpathlineto{\pgfqpoint{2.331993in}{5.220211in}}%
\pgfpathlineto{\pgfqpoint{2.334942in}{5.221468in}}%
\pgfpathlineto{\pgfqpoint{2.335785in}{5.221133in}}%
\pgfpathlineto{\pgfqpoint{2.337049in}{5.221804in}}%
\pgfpathlineto{\pgfqpoint{2.339155in}{5.221385in}}%
\pgfpathlineto{\pgfqpoint{2.341261in}{5.223019in}}%
\pgfpathlineto{\pgfqpoint{2.342104in}{5.223563in}}%
\pgfpathlineto{\pgfqpoint{2.347580in}{5.236930in}}%
\pgfpathlineto{\pgfqpoint{2.348423in}{5.237182in}}%
\pgfpathlineto{\pgfqpoint{2.351793in}{5.243593in}}%
\pgfpathlineto{\pgfqpoint{2.353900in}{5.248621in}}%
\pgfpathlineto{\pgfqpoint{2.356427in}{5.249124in}}%
\pgfpathlineto{\pgfqpoint{2.356849in}{5.247992in}}%
\pgfpathlineto{\pgfqpoint{2.357270in}{5.248411in}}%
\pgfpathlineto{\pgfqpoint{2.358955in}{5.249375in}}%
\pgfpathlineto{\pgfqpoint{2.360640in}{5.247992in}}%
\pgfpathlineto{\pgfqpoint{2.365274in}{5.250800in}}%
\pgfpathlineto{\pgfqpoint{2.366538in}{5.252182in}}%
\pgfpathlineto{\pgfqpoint{2.368223in}{5.250045in}}%
\pgfpathlineto{\pgfqpoint{2.368644in}{5.250548in}}%
\pgfpathlineto{\pgfqpoint{2.369066in}{5.249962in}}%
\pgfpathlineto{\pgfqpoint{2.377912in}{5.230729in}}%
\pgfpathlineto{\pgfqpoint{2.378755in}{5.228592in}}%
\pgfpathlineto{\pgfqpoint{2.381283in}{5.222306in}}%
\pgfpathlineto{\pgfqpoint{2.384653in}{5.221636in}}%
\pgfpathlineto{\pgfqpoint{2.386338in}{5.222977in}}%
\pgfpathlineto{\pgfqpoint{2.387181in}{5.222767in}}%
\pgfpathlineto{\pgfqpoint{2.395185in}{5.232321in}}%
\pgfpathlineto{\pgfqpoint{2.396027in}{5.232740in}}%
\pgfpathlineto{\pgfqpoint{2.400240in}{5.236469in}}%
\pgfpathlineto{\pgfqpoint{2.403189in}{5.235086in}}%
\pgfpathlineto{\pgfqpoint{2.404032in}{5.235506in}}%
\pgfpathlineto{\pgfqpoint{2.404453in}{5.235003in}}%
\pgfpathlineto{\pgfqpoint{2.412036in}{5.236637in}}%
\pgfpathlineto{\pgfqpoint{2.414564in}{5.240827in}}%
\pgfpathlineto{\pgfqpoint{2.417091in}{5.245101in}}%
\pgfpathlineto{\pgfqpoint{2.419198in}{5.248998in}}%
\pgfpathlineto{\pgfqpoint{2.420040in}{5.249752in}}%
\pgfpathlineto{\pgfqpoint{2.420462in}{5.250339in}}%
\pgfpathlineto{\pgfqpoint{2.427202in}{5.219206in}}%
\pgfpathlineto{\pgfqpoint{2.430994in}{5.203492in}}%
\pgfpathlineto{\pgfqpoint{2.433521in}{5.203618in}}%
\pgfpathlineto{\pgfqpoint{2.434785in}{5.204079in}}%
\pgfpathlineto{\pgfqpoint{2.436470in}{5.201774in}}%
\pgfpathlineto{\pgfqpoint{2.436891in}{5.202445in}}%
\pgfpathlineto{\pgfqpoint{2.437734in}{5.203073in}}%
\pgfpathlineto{\pgfqpoint{2.438155in}{5.202529in}}%
\pgfpathlineto{\pgfqpoint{2.439419in}{5.200978in}}%
\pgfpathlineto{\pgfqpoint{2.439840in}{5.201272in}}%
\pgfpathlineto{\pgfqpoint{2.440262in}{5.202822in}}%
\pgfpathlineto{\pgfqpoint{2.440683in}{5.201858in}}%
\pgfpathlineto{\pgfqpoint{2.443632in}{5.198757in}}%
\pgfpathlineto{\pgfqpoint{2.446581in}{5.181620in}}%
\pgfpathlineto{\pgfqpoint{2.448266in}{5.173742in}}%
\pgfpathlineto{\pgfqpoint{2.449951in}{5.175334in}}%
\pgfpathlineto{\pgfqpoint{2.451215in}{5.174496in}}%
\pgfpathlineto{\pgfqpoint{2.452900in}{5.175334in}}%
\pgfpathlineto{\pgfqpoint{2.454585in}{5.174748in}}%
\pgfpathlineto{\pgfqpoint{2.456270in}{5.175586in}}%
\pgfpathlineto{\pgfqpoint{2.457113in}{5.174203in}}%
\pgfpathlineto{\pgfqpoint{2.457534in}{5.174412in}}%
\pgfpathlineto{\pgfqpoint{2.458798in}{5.175418in}}%
\pgfpathlineto{\pgfqpoint{2.461326in}{5.176298in}}%
\pgfpathlineto{\pgfqpoint{2.465117in}{5.184343in}}%
\pgfpathlineto{\pgfqpoint{2.476913in}{5.209401in}}%
\pgfpathlineto{\pgfqpoint{2.477334in}{5.208982in}}%
\pgfpathlineto{\pgfqpoint{2.477755in}{5.209442in}}%
\pgfpathlineto{\pgfqpoint{2.480704in}{5.215602in}}%
\pgfpathlineto{\pgfqpoint{2.481126in}{5.214848in}}%
\pgfpathlineto{\pgfqpoint{2.483232in}{5.217865in}}%
\pgfpathlineto{\pgfqpoint{2.483653in}{5.216482in}}%
\pgfpathlineto{\pgfqpoint{2.487866in}{5.217362in}}%
\pgfpathlineto{\pgfqpoint{2.488287in}{5.217446in}}%
\pgfpathlineto{\pgfqpoint{2.489972in}{5.214555in}}%
\pgfpathlineto{\pgfqpoint{2.492500in}{5.213088in}}%
\pgfpathlineto{\pgfqpoint{2.494607in}{5.210741in}}%
\pgfpathlineto{\pgfqpoint{2.496292in}{5.208227in}}%
\pgfpathlineto{\pgfqpoint{2.496713in}{5.208563in}}%
\pgfpathlineto{\pgfqpoint{2.498398in}{5.210113in}}%
\pgfpathlineto{\pgfqpoint{2.498819in}{5.209401in}}%
\pgfpathlineto{\pgfqpoint{2.499241in}{5.209820in}}%
\pgfpathlineto{\pgfqpoint{2.501347in}{5.212459in}}%
\pgfpathlineto{\pgfqpoint{2.502190in}{5.212082in}}%
\pgfpathlineto{\pgfqpoint{2.502611in}{5.212459in}}%
\pgfpathlineto{\pgfqpoint{2.504717in}{5.217949in}}%
\pgfpathlineto{\pgfqpoint{2.506402in}{5.216608in}}%
\pgfpathlineto{\pgfqpoint{2.506824in}{5.217446in}}%
\pgfpathlineto{\pgfqpoint{2.507245in}{5.216943in}}%
\pgfpathlineto{\pgfqpoint{2.508930in}{5.215770in}}%
\pgfpathlineto{\pgfqpoint{2.510615in}{5.218242in}}%
\pgfpathlineto{\pgfqpoint{2.512300in}{5.217949in}}%
\pgfpathlineto{\pgfqpoint{2.513985in}{5.215979in}}%
\pgfpathlineto{\pgfqpoint{2.514407in}{5.216608in}}%
\pgfpathlineto{\pgfqpoint{2.515670in}{5.220672in}}%
\pgfpathlineto{\pgfqpoint{2.516092in}{5.219457in}}%
\pgfpathlineto{\pgfqpoint{2.518619in}{5.212795in}}%
\pgfpathlineto{\pgfqpoint{2.519041in}{5.213088in}}%
\pgfpathlineto{\pgfqpoint{2.520304in}{5.211621in}}%
\pgfpathlineto{\pgfqpoint{2.520726in}{5.212501in}}%
\pgfpathlineto{\pgfqpoint{2.522411in}{5.213716in}}%
\pgfpathlineto{\pgfqpoint{2.522832in}{5.213214in}}%
\pgfpathlineto{\pgfqpoint{2.525360in}{5.212753in}}%
\pgfpathlineto{\pgfqpoint{2.527887in}{5.216356in}}%
\pgfpathlineto{\pgfqpoint{2.528309in}{5.215560in}}%
\pgfpathlineto{\pgfqpoint{2.529151in}{5.213549in}}%
\pgfpathlineto{\pgfqpoint{2.529573in}{5.215015in}}%
\pgfpathlineto{\pgfqpoint{2.530415in}{5.215644in}}%
\pgfpathlineto{\pgfqpoint{2.532100in}{5.217655in}}%
\pgfpathlineto{\pgfqpoint{2.533364in}{5.217823in}}%
\pgfpathlineto{\pgfqpoint{2.539262in}{5.215560in}}%
\pgfpathlineto{\pgfqpoint{2.539683in}{5.215979in}}%
\pgfpathlineto{\pgfqpoint{2.540105in}{5.214890in}}%
\pgfpathlineto{\pgfqpoint{2.543054in}{5.210825in}}%
\pgfpathlineto{\pgfqpoint{2.545581in}{5.209401in}}%
\pgfpathlineto{\pgfqpoint{2.550637in}{5.216985in}}%
\pgfpathlineto{\pgfqpoint{2.552322in}{5.221049in}}%
\pgfpathlineto{\pgfqpoint{2.552743in}{5.220463in}}%
\pgfpathlineto{\pgfqpoint{2.553164in}{5.220840in}}%
\pgfpathlineto{\pgfqpoint{2.554849in}{5.223773in}}%
\pgfpathlineto{\pgfqpoint{2.555271in}{5.223144in}}%
\pgfpathlineto{\pgfqpoint{2.556534in}{5.253607in}}%
\pgfpathlineto{\pgfqpoint{2.558641in}{5.259473in}}%
\pgfpathlineto{\pgfqpoint{2.559483in}{5.261066in}}%
\pgfpathlineto{\pgfqpoint{2.561168in}{5.264418in}}%
\pgfpathlineto{\pgfqpoint{2.561590in}{5.263999in}}%
\pgfpathlineto{\pgfqpoint{2.562011in}{5.264921in}}%
\pgfpathlineto{\pgfqpoint{2.562854in}{5.266890in}}%
\pgfpathlineto{\pgfqpoint{2.564539in}{5.270075in}}%
\pgfpathlineto{\pgfqpoint{2.565803in}{5.271541in}}%
\pgfpathlineto{\pgfqpoint{2.567488in}{5.275815in}}%
\pgfpathlineto{\pgfqpoint{2.568751in}{5.279461in}}%
\pgfpathlineto{\pgfqpoint{2.569173in}{5.251428in}}%
\pgfpathlineto{\pgfqpoint{2.570015in}{5.254403in}}%
\pgfpathlineto{\pgfqpoint{2.570858in}{5.253775in}}%
\pgfpathlineto{\pgfqpoint{2.571279in}{5.254655in}}%
\pgfpathlineto{\pgfqpoint{2.573807in}{5.259767in}}%
\pgfpathlineto{\pgfqpoint{2.574228in}{5.259892in}}%
\pgfpathlineto{\pgfqpoint{2.576334in}{5.268608in}}%
\pgfpathlineto{\pgfqpoint{2.576756in}{5.267980in}}%
\pgfpathlineto{\pgfqpoint{2.578862in}{5.270829in}}%
\pgfpathlineto{\pgfqpoint{2.579283in}{5.270117in}}%
\pgfpathlineto{\pgfqpoint{2.580547in}{5.269698in}}%
\pgfpathlineto{\pgfqpoint{2.582654in}{5.273050in}}%
\pgfpathlineto{\pgfqpoint{2.583496in}{5.271751in}}%
\pgfpathlineto{\pgfqpoint{2.583917in}{5.272170in}}%
\pgfpathlineto{\pgfqpoint{2.585603in}{5.272882in}}%
\pgfpathlineto{\pgfqpoint{2.586866in}{5.270200in}}%
\pgfpathlineto{\pgfqpoint{2.587288in}{5.270996in}}%
\pgfpathlineto{\pgfqpoint{2.587709in}{5.271416in}}%
\pgfpathlineto{\pgfqpoint{2.590237in}{5.265843in}}%
\pgfpathlineto{\pgfqpoint{2.590658in}{5.266262in}}%
\pgfpathlineto{\pgfqpoint{2.591922in}{5.266848in}}%
\pgfpathlineto{\pgfqpoint{2.592764in}{5.265968in}}%
\pgfpathlineto{\pgfqpoint{2.593186in}{5.266262in}}%
\pgfpathlineto{\pgfqpoint{2.595292in}{5.267519in}}%
\pgfpathlineto{\pgfqpoint{2.596556in}{5.267519in}}%
\pgfpathlineto{\pgfqpoint{2.596977in}{5.268315in}}%
\pgfpathlineto{\pgfqpoint{2.597398in}{5.267225in}}%
\pgfpathlineto{\pgfqpoint{2.598662in}{5.266890in}}%
\pgfpathlineto{\pgfqpoint{2.599505in}{5.266220in}}%
\pgfpathlineto{\pgfqpoint{2.599926in}{5.266722in}}%
\pgfpathlineto{\pgfqpoint{2.600347in}{5.267477in}}%
\pgfpathlineto{\pgfqpoint{2.600769in}{5.266681in}}%
\pgfpathlineto{\pgfqpoint{2.602875in}{5.266094in}}%
\pgfpathlineto{\pgfqpoint{2.603718in}{5.267519in}}%
\pgfpathlineto{\pgfqpoint{2.604139in}{5.267142in}}%
\pgfpathlineto{\pgfqpoint{2.605403in}{5.265382in}}%
\pgfpathlineto{\pgfqpoint{2.605824in}{5.266220in}}%
\pgfpathlineto{\pgfqpoint{2.608352in}{5.267435in}}%
\pgfpathlineto{\pgfqpoint{2.618041in}{5.273385in}}%
\pgfpathlineto{\pgfqpoint{2.619305in}{5.272924in}}%
\pgfpathlineto{\pgfqpoint{2.620147in}{5.275187in}}%
\pgfpathlineto{\pgfqpoint{2.620569in}{5.274684in}}%
\pgfpathlineto{\pgfqpoint{2.625624in}{5.272714in}}%
\pgfpathlineto{\pgfqpoint{2.626888in}{5.274810in}}%
\pgfpathlineto{\pgfqpoint{2.627730in}{5.273888in}}%
\pgfpathlineto{\pgfqpoint{2.628994in}{5.273846in}}%
\pgfpathlineto{\pgfqpoint{2.630258in}{5.274810in}}%
\pgfpathlineto{\pgfqpoint{2.630679in}{5.273175in}}%
\pgfpathlineto{\pgfqpoint{2.631101in}{5.273720in}}%
\pgfpathlineto{\pgfqpoint{2.634050in}{5.276402in}}%
\pgfpathlineto{\pgfqpoint{2.635313in}{5.275480in}}%
\pgfpathlineto{\pgfqpoint{2.635735in}{5.275941in}}%
\pgfpathlineto{\pgfqpoint{2.637420in}{5.282771in}}%
\pgfpathlineto{\pgfqpoint{2.639105in}{5.284615in}}%
\pgfpathlineto{\pgfqpoint{2.639526in}{5.285411in}}%
\pgfpathlineto{\pgfqpoint{2.639947in}{5.284740in}}%
\pgfpathlineto{\pgfqpoint{2.642054in}{5.282520in}}%
\pgfpathlineto{\pgfqpoint{2.642475in}{5.282939in}}%
\pgfpathlineto{\pgfqpoint{2.642896in}{5.282226in}}%
\pgfpathlineto{\pgfqpoint{2.645003in}{5.280550in}}%
\pgfpathlineto{\pgfqpoint{2.646688in}{5.280131in}}%
\pgfpathlineto{\pgfqpoint{2.648373in}{5.276653in}}%
\pgfpathlineto{\pgfqpoint{2.648794in}{5.277952in}}%
\pgfpathlineto{\pgfqpoint{2.649216in}{5.277407in}}%
\pgfpathlineto{\pgfqpoint{2.650901in}{5.270033in}}%
\pgfpathlineto{\pgfqpoint{2.652165in}{5.268273in}}%
\pgfpathlineto{\pgfqpoint{2.656377in}{5.273469in}}%
\pgfpathlineto{\pgfqpoint{2.656799in}{5.273133in}}%
\pgfpathlineto{\pgfqpoint{2.661011in}{5.273930in}}%
\pgfpathlineto{\pgfqpoint{2.662275in}{5.276444in}}%
\pgfpathlineto{\pgfqpoint{2.662696in}{5.276067in}}%
\pgfpathlineto{\pgfqpoint{2.664803in}{5.275480in}}%
\pgfpathlineto{\pgfqpoint{2.666488in}{5.275354in}}%
\pgfpathlineto{\pgfqpoint{2.668173in}{5.274265in}}%
\pgfpathlineto{\pgfqpoint{2.669016in}{5.274642in}}%
\pgfpathlineto{\pgfqpoint{2.674914in}{5.260647in}}%
\pgfpathlineto{\pgfqpoint{2.678284in}{5.256834in}}%
\pgfpathlineto{\pgfqpoint{2.682497in}{5.247741in}}%
\pgfpathlineto{\pgfqpoint{2.682918in}{5.248579in}}%
\pgfpathlineto{\pgfqpoint{2.683760in}{5.248411in}}%
\pgfpathlineto{\pgfqpoint{2.686709in}{5.251805in}}%
\pgfpathlineto{\pgfqpoint{2.690080in}{5.252099in}}%
\pgfpathlineto{\pgfqpoint{2.691765in}{5.253104in}}%
\pgfpathlineto{\pgfqpoint{2.695556in}{5.254152in}}%
\pgfpathlineto{\pgfqpoint{2.695977in}{5.253565in}}%
\pgfpathlineto{\pgfqpoint{2.696399in}{5.254319in}}%
\pgfpathlineto{\pgfqpoint{2.698926in}{5.255283in}}%
\pgfpathlineto{\pgfqpoint{2.700612in}{5.257253in}}%
\pgfpathlineto{\pgfqpoint{2.701033in}{5.256666in}}%
\pgfpathlineto{\pgfqpoint{2.701454in}{5.257169in}}%
\pgfpathlineto{\pgfqpoint{2.703139in}{5.257797in}}%
\pgfpathlineto{\pgfqpoint{2.703982in}{5.258049in}}%
\pgfpathlineto{\pgfqpoint{2.709458in}{5.266220in}}%
\pgfpathlineto{\pgfqpoint{2.709880in}{5.265298in}}%
\pgfpathlineto{\pgfqpoint{2.712829in}{5.273930in}}%
\pgfpathlineto{\pgfqpoint{2.723361in}{5.285536in}}%
\pgfpathlineto{\pgfqpoint{2.723782in}{5.284447in}}%
\pgfpathlineto{\pgfqpoint{2.727573in}{5.281765in}}%
\pgfpathlineto{\pgfqpoint{2.727995in}{5.282687in}}%
\pgfpathlineto{\pgfqpoint{2.729680in}{5.282059in}}%
\pgfpathlineto{\pgfqpoint{2.732207in}{5.280173in}}%
\pgfpathlineto{\pgfqpoint{2.734735in}{5.276444in}}%
\pgfpathlineto{\pgfqpoint{2.739790in}{5.268859in}}%
\pgfpathlineto{\pgfqpoint{2.741054in}{5.267812in}}%
\pgfpathlineto{\pgfqpoint{2.742739in}{5.264753in}}%
\pgfpathlineto{\pgfqpoint{2.743161in}{5.265130in}}%
\pgfpathlineto{\pgfqpoint{2.747373in}{5.262029in}}%
\pgfpathlineto{\pgfqpoint{2.749480in}{5.263538in}}%
\pgfpathlineto{\pgfqpoint{2.749901in}{5.262574in}}%
\pgfpathlineto{\pgfqpoint{2.750322in}{5.262826in}}%
\pgfpathlineto{\pgfqpoint{2.750744in}{5.263789in}}%
\pgfpathlineto{\pgfqpoint{2.751586in}{5.262951in}}%
\pgfpathlineto{\pgfqpoint{2.754535in}{5.266262in}}%
\pgfpathlineto{\pgfqpoint{2.756220in}{5.264502in}}%
\pgfpathlineto{\pgfqpoint{2.760854in}{5.271583in}}%
\pgfpathlineto{\pgfqpoint{2.761697in}{5.271164in}}%
\pgfpathlineto{\pgfqpoint{2.763382in}{5.272882in}}%
\pgfpathlineto{\pgfqpoint{2.767595in}{5.264753in}}%
\pgfpathlineto{\pgfqpoint{2.769280in}{5.266010in}}%
\pgfpathlineto{\pgfqpoint{2.769701in}{5.267309in}}%
\pgfpathlineto{\pgfqpoint{2.770122in}{5.266471in}}%
\pgfpathlineto{\pgfqpoint{2.775178in}{5.259432in}}%
\pgfpathlineto{\pgfqpoint{2.775599in}{5.259809in}}%
\pgfpathlineto{\pgfqpoint{2.776020in}{5.260228in}}%
\pgfpathlineto{\pgfqpoint{2.776442in}{5.259348in}}%
\pgfpathlineto{\pgfqpoint{2.777705in}{5.259390in}}%
\pgfpathlineto{\pgfqpoint{2.779812in}{5.262155in}}%
\pgfpathlineto{\pgfqpoint{2.780654in}{5.261778in}}%
\pgfpathlineto{\pgfqpoint{2.781918in}{5.264292in}}%
\pgfpathlineto{\pgfqpoint{2.782339in}{5.266513in}}%
\pgfpathlineto{\pgfqpoint{2.783182in}{5.266052in}}%
\pgfpathlineto{\pgfqpoint{2.784867in}{5.267351in}}%
\pgfpathlineto{\pgfqpoint{2.786552in}{5.267309in}}%
\pgfpathlineto{\pgfqpoint{2.788659in}{5.271248in}}%
\pgfpathlineto{\pgfqpoint{2.789080in}{5.270661in}}%
\pgfpathlineto{\pgfqpoint{2.790344in}{5.269991in}}%
\pgfpathlineto{\pgfqpoint{2.791608in}{5.269027in}}%
\pgfpathlineto{\pgfqpoint{2.792029in}{5.269488in}}%
\pgfpathlineto{\pgfqpoint{2.795399in}{5.267686in}}%
\pgfpathlineto{\pgfqpoint{2.797084in}{5.265591in}}%
\pgfpathlineto{\pgfqpoint{2.797506in}{5.265717in}}%
\pgfpathlineto{\pgfqpoint{2.797927in}{5.264418in}}%
\pgfpathlineto{\pgfqpoint{2.798769in}{5.265005in}}%
\pgfpathlineto{\pgfqpoint{2.799191in}{5.264334in}}%
\pgfpathlineto{\pgfqpoint{2.799612in}{5.264921in}}%
\pgfpathlineto{\pgfqpoint{2.800033in}{5.265884in}}%
\pgfpathlineto{\pgfqpoint{2.800876in}{5.265382in}}%
\pgfpathlineto{\pgfqpoint{2.802561in}{5.263957in}}%
\pgfpathlineto{\pgfqpoint{2.802982in}{5.263580in}}%
\pgfpathlineto{\pgfqpoint{2.803825in}{5.265130in}}%
\pgfpathlineto{\pgfqpoint{2.804246in}{5.264879in}}%
\pgfpathlineto{\pgfqpoint{2.805510in}{5.264795in}}%
\pgfpathlineto{\pgfqpoint{2.806774in}{5.264669in}}%
\pgfpathlineto{\pgfqpoint{2.808037in}{5.263454in}}%
\pgfpathlineto{\pgfqpoint{2.809723in}{5.266178in}}%
\pgfpathlineto{\pgfqpoint{2.810565in}{5.266262in}}%
\pgfpathlineto{\pgfqpoint{2.812672in}{5.267980in}}%
\pgfpathlineto{\pgfqpoint{2.813093in}{5.267602in}}%
\pgfpathlineto{\pgfqpoint{2.818148in}{5.265717in}}%
\pgfpathlineto{\pgfqpoint{2.821518in}{5.254906in}}%
\pgfpathlineto{\pgfqpoint{2.822782in}{5.253523in}}%
\pgfpathlineto{\pgfqpoint{2.823203in}{5.254068in}}%
\pgfpathlineto{\pgfqpoint{2.826152in}{5.257253in}}%
\pgfpathlineto{\pgfqpoint{2.826574in}{5.257504in}}%
\pgfpathlineto{\pgfqpoint{2.827416in}{5.260144in}}%
\pgfpathlineto{\pgfqpoint{2.827838in}{5.260018in}}%
\pgfpathlineto{\pgfqpoint{2.828259in}{5.259473in}}%
\pgfpathlineto{\pgfqpoint{2.828680in}{5.260270in}}%
\pgfpathlineto{\pgfqpoint{2.829523in}{5.261024in}}%
\pgfpathlineto{\pgfqpoint{2.829944in}{5.262029in}}%
\pgfpathlineto{\pgfqpoint{2.830786in}{5.261443in}}%
\pgfpathlineto{\pgfqpoint{2.832050in}{5.261610in}}%
\pgfpathlineto{\pgfqpoint{2.833314in}{5.261610in}}%
\pgfpathlineto{\pgfqpoint{2.835421in}{5.263706in}}%
\pgfpathlineto{\pgfqpoint{2.835842in}{5.262574in}}%
\pgfpathlineto{\pgfqpoint{2.838791in}{5.252685in}}%
\pgfpathlineto{\pgfqpoint{2.841318in}{5.244933in}}%
\pgfpathlineto{\pgfqpoint{2.843846in}{5.237894in}}%
\pgfpathlineto{\pgfqpoint{2.845952in}{5.231525in}}%
\pgfpathlineto{\pgfqpoint{2.847216in}{5.228675in}}%
\pgfpathlineto{\pgfqpoint{2.848480in}{5.223480in}}%
\pgfpathlineto{\pgfqpoint{2.851008in}{5.211747in}}%
\pgfpathlineto{\pgfqpoint{2.858170in}{5.183631in}}%
\pgfpathlineto{\pgfqpoint{2.865753in}{5.183170in}}%
\pgfpathlineto{\pgfqpoint{2.866595in}{5.184469in}}%
\pgfpathlineto{\pgfqpoint{2.867016in}{5.183882in}}%
\pgfpathlineto{\pgfqpoint{2.870387in}{5.185935in}}%
\pgfpathlineto{\pgfqpoint{2.872493in}{5.190964in}}%
\pgfpathlineto{\pgfqpoint{2.877127in}{5.204163in}}%
\pgfpathlineto{\pgfqpoint{2.881761in}{5.211370in}}%
\pgfpathlineto{\pgfqpoint{2.882604in}{5.212124in}}%
\pgfpathlineto{\pgfqpoint{2.883446in}{5.213423in}}%
\pgfpathlineto{\pgfqpoint{2.883868in}{5.212627in}}%
\pgfpathlineto{\pgfqpoint{2.885553in}{5.211370in}}%
\pgfpathlineto{\pgfqpoint{2.885974in}{5.212040in}}%
\pgfpathlineto{\pgfqpoint{2.886395in}{5.211663in}}%
\pgfpathlineto{\pgfqpoint{2.888080in}{5.210239in}}%
\pgfpathlineto{\pgfqpoint{2.888923in}{5.210448in}}%
\pgfpathlineto{\pgfqpoint{2.889344in}{5.209023in}}%
\pgfpathlineto{\pgfqpoint{2.890187in}{5.209652in}}%
\pgfpathlineto{\pgfqpoint{2.891029in}{5.209526in}}%
\pgfpathlineto{\pgfqpoint{2.892714in}{5.210867in}}%
\pgfpathlineto{\pgfqpoint{2.893136in}{5.210155in}}%
\pgfpathlineto{\pgfqpoint{2.893557in}{5.210825in}}%
\pgfpathlineto{\pgfqpoint{2.895242in}{5.212208in}}%
\pgfpathlineto{\pgfqpoint{2.896927in}{5.211705in}}%
\pgfpathlineto{\pgfqpoint{2.899034in}{5.213297in}}%
\pgfpathlineto{\pgfqpoint{2.899876in}{5.212711in}}%
\pgfpathlineto{\pgfqpoint{2.900297in}{5.213214in}}%
\pgfpathlineto{\pgfqpoint{2.900719in}{5.212711in}}%
\pgfpathlineto{\pgfqpoint{2.901561in}{5.213423in}}%
\pgfpathlineto{\pgfqpoint{2.903668in}{5.211244in}}%
\pgfpathlineto{\pgfqpoint{2.906617in}{5.203367in}}%
\pgfpathlineto{\pgfqpoint{2.909987in}{5.193855in}}%
\pgfpathlineto{\pgfqpoint{2.911672in}{5.190461in}}%
\pgfpathlineto{\pgfqpoint{2.913778in}{5.186187in}}%
\pgfpathlineto{\pgfqpoint{2.915885in}{5.183212in}}%
\pgfpathlineto{\pgfqpoint{2.916306in}{5.183840in}}%
\pgfpathlineto{\pgfqpoint{2.918412in}{5.187821in}}%
\pgfpathlineto{\pgfqpoint{2.921783in}{5.197123in}}%
\pgfpathlineto{\pgfqpoint{2.922625in}{5.198422in}}%
\pgfpathlineto{\pgfqpoint{2.928102in}{5.211957in}}%
\pgfpathlineto{\pgfqpoint{2.928944in}{5.212711in}}%
\pgfpathlineto{\pgfqpoint{2.931893in}{5.217069in}}%
\pgfpathlineto{\pgfqpoint{2.932736in}{5.217781in}}%
\pgfpathlineto{\pgfqpoint{2.936949in}{5.223438in}}%
\pgfpathlineto{\pgfqpoint{2.937370in}{5.222432in}}%
\pgfpathlineto{\pgfqpoint{2.937791in}{5.223144in}}%
\pgfpathlineto{\pgfqpoint{2.939476in}{5.223815in}}%
\pgfpathlineto{\pgfqpoint{2.939898in}{5.223270in}}%
\pgfpathlineto{\pgfqpoint{2.940319in}{5.223689in}}%
\pgfpathlineto{\pgfqpoint{2.940740in}{5.224485in}}%
\pgfpathlineto{\pgfqpoint{2.941161in}{5.223522in}}%
\pgfpathlineto{\pgfqpoint{2.942004in}{5.222725in}}%
\pgfpathlineto{\pgfqpoint{2.943689in}{5.220966in}}%
\pgfpathlineto{\pgfqpoint{2.944532in}{5.221385in}}%
\pgfpathlineto{\pgfqpoint{2.946638in}{5.217949in}}%
\pgfpathlineto{\pgfqpoint{2.947059in}{5.218242in}}%
\pgfpathlineto{\pgfqpoint{2.950429in}{5.210616in}}%
\pgfpathlineto{\pgfqpoint{2.951693in}{5.208353in}}%
\pgfpathlineto{\pgfqpoint{2.953800in}{5.203995in}}%
\pgfpathlineto{\pgfqpoint{2.955064in}{5.203744in}}%
\pgfpathlineto{\pgfqpoint{2.955485in}{5.204582in}}%
\pgfpathlineto{\pgfqpoint{2.956327in}{5.203995in}}%
\pgfpathlineto{\pgfqpoint{2.957591in}{5.202529in}}%
\pgfpathlineto{\pgfqpoint{2.960540in}{5.201774in}}%
\pgfpathlineto{\pgfqpoint{2.974021in}{5.212585in}}%
\pgfpathlineto{\pgfqpoint{2.975706in}{5.213591in}}%
\pgfpathlineto{\pgfqpoint{2.979076in}{5.215183in}}%
\pgfpathlineto{\pgfqpoint{2.980761in}{5.216692in}}%
\pgfpathlineto{\pgfqpoint{2.982447in}{5.215476in}}%
\pgfpathlineto{\pgfqpoint{2.983289in}{5.216314in}}%
\pgfpathlineto{\pgfqpoint{2.983710in}{5.214848in}}%
\pgfpathlineto{\pgfqpoint{2.984553in}{5.215351in}}%
\pgfpathlineto{\pgfqpoint{2.986238in}{5.214261in}}%
\pgfpathlineto{\pgfqpoint{2.988344in}{5.216440in}}%
\pgfpathlineto{\pgfqpoint{2.990030in}{5.215267in}}%
\pgfpathlineto{\pgfqpoint{2.990451in}{5.215895in}}%
\pgfpathlineto{\pgfqpoint{2.992136in}{5.216859in}}%
\pgfpathlineto{\pgfqpoint{2.993821in}{5.216021in}}%
\pgfpathlineto{\pgfqpoint{2.997191in}{5.218745in}}%
\pgfpathlineto{\pgfqpoint{2.997613in}{5.218116in}}%
\pgfpathlineto{\pgfqpoint{2.998034in}{5.218745in}}%
\pgfpathlineto{\pgfqpoint{2.999719in}{5.219373in}}%
\pgfpathlineto{\pgfqpoint{3.000140in}{5.218493in}}%
\pgfpathlineto{\pgfqpoint{3.000562in}{5.219080in}}%
\pgfpathlineto{\pgfqpoint{3.000983in}{5.219499in}}%
\pgfpathlineto{\pgfqpoint{3.001404in}{5.218745in}}%
\pgfpathlineto{\pgfqpoint{3.003932in}{5.217655in}}%
\pgfpathlineto{\pgfqpoint{3.006038in}{5.215728in}}%
\pgfpathlineto{\pgfqpoint{3.007723in}{5.214135in}}%
\pgfpathlineto{\pgfqpoint{3.008987in}{5.213591in}}%
\pgfpathlineto{\pgfqpoint{3.009408in}{5.212795in}}%
\pgfpathlineto{\pgfqpoint{3.010251in}{5.213297in}}%
\pgfpathlineto{\pgfqpoint{3.012357in}{5.213716in}}%
\pgfpathlineto{\pgfqpoint{3.014042in}{5.212166in}}%
\pgfpathlineto{\pgfqpoint{3.018255in}{5.212292in}}%
\pgfpathlineto{\pgfqpoint{3.019098in}{5.212459in}}%
\pgfpathlineto{\pgfqpoint{3.029208in}{5.203576in}}%
\pgfpathlineto{\pgfqpoint{3.030894in}{5.203367in}}%
\pgfpathlineto{\pgfqpoint{3.032579in}{5.201355in}}%
\pgfpathlineto{\pgfqpoint{3.036370in}{5.209275in}}%
\pgfpathlineto{\pgfqpoint{3.039319in}{5.213297in}}%
\pgfpathlineto{\pgfqpoint{3.043532in}{5.218996in}}%
\pgfpathlineto{\pgfqpoint{3.044374in}{5.218829in}}%
\pgfpathlineto{\pgfqpoint{3.046060in}{5.221468in}}%
\pgfpathlineto{\pgfqpoint{3.047745in}{5.219918in}}%
\pgfpathlineto{\pgfqpoint{3.049851in}{5.222264in}}%
\pgfpathlineto{\pgfqpoint{3.051115in}{5.222055in}}%
\pgfpathlineto{\pgfqpoint{3.057434in}{5.222809in}}%
\pgfpathlineto{\pgfqpoint{3.059119in}{5.222264in}}%
\pgfpathlineto{\pgfqpoint{3.059540in}{5.221887in}}%
\pgfpathlineto{\pgfqpoint{3.059962in}{5.223354in}}%
\pgfpathlineto{\pgfqpoint{3.060804in}{5.223103in}}%
\pgfpathlineto{\pgfqpoint{3.061226in}{5.223103in}}%
\pgfpathlineto{\pgfqpoint{3.063332in}{5.220253in}}%
\pgfpathlineto{\pgfqpoint{3.065017in}{5.220672in}}%
\pgfpathlineto{\pgfqpoint{3.065438in}{5.219834in}}%
\pgfpathlineto{\pgfqpoint{3.066281in}{5.220630in}}%
\pgfpathlineto{\pgfqpoint{3.066702in}{5.219918in}}%
\pgfpathlineto{\pgfqpoint{3.067124in}{5.220756in}}%
\pgfpathlineto{\pgfqpoint{3.068809in}{5.222097in}}%
\pgfpathlineto{\pgfqpoint{3.069230in}{5.220463in}}%
\pgfpathlineto{\pgfqpoint{3.070072in}{5.221133in}}%
\pgfpathlineto{\pgfqpoint{3.070494in}{5.220798in}}%
\pgfpathlineto{\pgfqpoint{3.070915in}{5.221385in}}%
\pgfpathlineto{\pgfqpoint{3.073864in}{5.224485in}}%
\pgfpathlineto{\pgfqpoint{3.074285in}{5.223941in}}%
\pgfpathlineto{\pgfqpoint{3.075128in}{5.224611in}}%
\pgfpathlineto{\pgfqpoint{3.076813in}{5.226287in}}%
\pgfpathlineto{\pgfqpoint{3.079762in}{5.232363in}}%
\pgfpathlineto{\pgfqpoint{3.081447in}{5.233955in}}%
\pgfpathlineto{\pgfqpoint{3.082290in}{5.232363in}}%
\pgfpathlineto{\pgfqpoint{3.083975in}{5.234248in}}%
\pgfpathlineto{\pgfqpoint{3.084396in}{5.233829in}}%
\pgfpathlineto{\pgfqpoint{3.084817in}{5.234374in}}%
\pgfpathlineto{\pgfqpoint{3.087766in}{5.236846in}}%
\pgfpathlineto{\pgfqpoint{3.088187in}{5.236553in}}%
\pgfpathlineto{\pgfqpoint{3.088609in}{5.237391in}}%
\pgfpathlineto{\pgfqpoint{3.091136in}{5.239989in}}%
\pgfpathlineto{\pgfqpoint{3.093243in}{5.239319in}}%
\pgfpathlineto{\pgfqpoint{3.095770in}{5.240240in}}%
\pgfpathlineto{\pgfqpoint{3.097034in}{5.240240in}}%
\pgfpathlineto{\pgfqpoint{3.098719in}{5.240115in}}%
\pgfpathlineto{\pgfqpoint{3.100826in}{5.236888in}}%
\pgfpathlineto{\pgfqpoint{3.107987in}{5.231525in}}%
\pgfpathlineto{\pgfqpoint{3.110936in}{5.233620in}}%
\pgfpathlineto{\pgfqpoint{3.111779in}{5.232321in}}%
\pgfpathlineto{\pgfqpoint{3.113043in}{5.232530in}}%
\pgfpathlineto{\pgfqpoint{3.115149in}{5.231441in}}%
\pgfpathlineto{\pgfqpoint{3.117256in}{5.227921in}}%
\pgfpathlineto{\pgfqpoint{3.117677in}{5.227963in}}%
\pgfpathlineto{\pgfqpoint{3.120626in}{5.222642in}}%
\pgfpathlineto{\pgfqpoint{3.123996in}{5.218954in}}%
\pgfpathlineto{\pgfqpoint{3.124417in}{5.217446in}}%
\pgfpathlineto{\pgfqpoint{3.124839in}{5.218200in}}%
\pgfpathlineto{\pgfqpoint{3.126524in}{5.219457in}}%
\pgfpathlineto{\pgfqpoint{3.128209in}{5.218787in}}%
\pgfpathlineto{\pgfqpoint{3.129051in}{5.219792in}}%
\pgfpathlineto{\pgfqpoint{3.129473in}{5.219080in}}%
\pgfpathlineto{\pgfqpoint{3.133685in}{5.220463in}}%
\pgfpathlineto{\pgfqpoint{3.135371in}{5.221552in}}%
\pgfpathlineto{\pgfqpoint{3.137056in}{5.219541in}}%
\pgfpathlineto{\pgfqpoint{3.139162in}{5.220211in}}%
\pgfpathlineto{\pgfqpoint{3.140005in}{5.219792in}}%
\pgfpathlineto{\pgfqpoint{3.140426in}{5.220169in}}%
\pgfpathlineto{\pgfqpoint{3.140847in}{5.219625in}}%
\pgfpathlineto{\pgfqpoint{3.141268in}{5.220630in}}%
\pgfpathlineto{\pgfqpoint{3.141690in}{5.221217in}}%
\pgfpathlineto{\pgfqpoint{3.142532in}{5.220546in}}%
\pgfpathlineto{\pgfqpoint{3.144217in}{5.220714in}}%
\pgfpathlineto{\pgfqpoint{3.144639in}{5.220169in}}%
\pgfpathlineto{\pgfqpoint{3.145060in}{5.220756in}}%
\pgfpathlineto{\pgfqpoint{3.146745in}{5.221762in}}%
\pgfpathlineto{\pgfqpoint{3.147166in}{5.221007in}}%
\pgfpathlineto{\pgfqpoint{3.147588in}{5.221301in}}%
\pgfpathlineto{\pgfqpoint{3.149273in}{5.223186in}}%
\pgfpathlineto{\pgfqpoint{3.150115in}{5.223480in}}%
\pgfpathlineto{\pgfqpoint{3.151800in}{5.224234in}}%
\pgfpathlineto{\pgfqpoint{3.152222in}{5.223731in}}%
\pgfpathlineto{\pgfqpoint{3.152643in}{5.224234in}}%
\pgfpathlineto{\pgfqpoint{3.154328in}{5.225533in}}%
\pgfpathlineto{\pgfqpoint{3.155171in}{5.225826in}}%
\pgfpathlineto{\pgfqpoint{3.156856in}{5.227712in}}%
\pgfpathlineto{\pgfqpoint{3.157277in}{5.227251in}}%
\pgfpathlineto{\pgfqpoint{3.157698in}{5.228047in}}%
\pgfpathlineto{\pgfqpoint{3.158962in}{5.228927in}}%
\pgfpathlineto{\pgfqpoint{3.161911in}{5.230058in}}%
\pgfpathlineto{\pgfqpoint{3.164860in}{5.229388in}}%
\pgfpathlineto{\pgfqpoint{3.165703in}{5.230603in}}%
\pgfpathlineto{\pgfqpoint{3.166124in}{5.230184in}}%
\pgfpathlineto{\pgfqpoint{3.167809in}{5.229053in}}%
\pgfpathlineto{\pgfqpoint{3.168230in}{5.229514in}}%
\pgfpathlineto{\pgfqpoint{3.168652in}{5.228927in}}%
\pgfpathlineto{\pgfqpoint{3.172443in}{5.224653in}}%
\pgfpathlineto{\pgfqpoint{3.176656in}{5.227921in}}%
\pgfpathlineto{\pgfqpoint{3.183818in}{5.240576in}}%
\pgfpathlineto{\pgfqpoint{3.185924in}{5.244850in}}%
\pgfpathlineto{\pgfqpoint{3.190979in}{5.252141in}}%
\pgfpathlineto{\pgfqpoint{3.193086in}{5.252392in}}%
\pgfpathlineto{\pgfqpoint{3.198562in}{5.228634in}}%
\pgfpathlineto{\pgfqpoint{3.198984in}{5.228801in}}%
\pgfpathlineto{\pgfqpoint{3.201933in}{5.226538in}}%
\pgfpathlineto{\pgfqpoint{3.202354in}{5.225323in}}%
\pgfpathlineto{\pgfqpoint{3.203196in}{5.225868in}}%
\pgfpathlineto{\pgfqpoint{3.206145in}{5.222139in}}%
\pgfpathlineto{\pgfqpoint{3.207830in}{5.227544in}}%
\pgfpathlineto{\pgfqpoint{3.211201in}{5.243425in}}%
\pgfpathlineto{\pgfqpoint{3.211622in}{5.243341in}}%
\pgfpathlineto{\pgfqpoint{3.212043in}{5.243718in}}%
\pgfpathlineto{\pgfqpoint{3.212464in}{5.242796in}}%
\pgfpathlineto{\pgfqpoint{3.217099in}{5.242922in}}%
\pgfpathlineto{\pgfqpoint{3.217520in}{5.242252in}}%
\pgfpathlineto{\pgfqpoint{3.217941in}{5.243006in}}%
\pgfpathlineto{\pgfqpoint{3.219626in}{5.244012in}}%
\pgfpathlineto{\pgfqpoint{3.225103in}{5.224820in}}%
\pgfpathlineto{\pgfqpoint{3.225524in}{5.225240in}}%
\pgfpathlineto{\pgfqpoint{3.229737in}{5.229053in}}%
\pgfpathlineto{\pgfqpoint{3.230158in}{5.228801in}}%
\pgfpathlineto{\pgfqpoint{3.230579in}{5.229555in}}%
\pgfpathlineto{\pgfqpoint{3.232265in}{5.230435in}}%
\pgfpathlineto{\pgfqpoint{3.232686in}{5.229849in}}%
\pgfpathlineto{\pgfqpoint{3.233107in}{5.230519in}}%
\pgfpathlineto{\pgfqpoint{3.234792in}{5.231399in}}%
\pgfpathlineto{\pgfqpoint{3.236477in}{5.230477in}}%
\pgfpathlineto{\pgfqpoint{3.237320in}{5.231860in}}%
\pgfpathlineto{\pgfqpoint{3.237741in}{5.231315in}}%
\pgfpathlineto{\pgfqpoint{3.246167in}{5.211370in}}%
\pgfpathlineto{\pgfqpoint{3.246588in}{5.211831in}}%
\pgfpathlineto{\pgfqpoint{3.249958in}{5.220127in}}%
\pgfpathlineto{\pgfqpoint{3.254171in}{5.228005in}}%
\pgfpathlineto{\pgfqpoint{3.254592in}{5.227460in}}%
\pgfpathlineto{\pgfqpoint{3.257963in}{5.218074in}}%
\pgfpathlineto{\pgfqpoint{3.260069in}{5.212124in}}%
\pgfpathlineto{\pgfqpoint{3.265124in}{5.198255in}}%
\pgfpathlineto{\pgfqpoint{3.266809in}{5.194274in}}%
\pgfpathlineto{\pgfqpoint{3.268073in}{5.190628in}}%
\pgfpathlineto{\pgfqpoint{3.268916in}{5.190964in}}%
\pgfpathlineto{\pgfqpoint{3.269758in}{5.191173in}}%
\pgfpathlineto{\pgfqpoint{3.271022in}{5.193394in}}%
\pgfpathlineto{\pgfqpoint{3.271443in}{5.192975in}}%
\pgfpathlineto{\pgfqpoint{3.273129in}{5.192011in}}%
\pgfpathlineto{\pgfqpoint{3.275656in}{5.195573in}}%
\pgfpathlineto{\pgfqpoint{3.280712in}{5.205420in}}%
\pgfpathlineto{\pgfqpoint{3.282397in}{5.208563in}}%
\pgfpathlineto{\pgfqpoint{3.289980in}{5.217530in}}%
\pgfpathlineto{\pgfqpoint{3.290822in}{5.216901in}}%
\pgfpathlineto{\pgfqpoint{3.292086in}{5.217697in}}%
\pgfpathlineto{\pgfqpoint{3.293771in}{5.217278in}}%
\pgfpathlineto{\pgfqpoint{3.294614in}{5.215937in}}%
\pgfpathlineto{\pgfqpoint{3.295035in}{5.216440in}}%
\pgfpathlineto{\pgfqpoint{3.297563in}{5.218368in}}%
\pgfpathlineto{\pgfqpoint{3.299248in}{5.217278in}}%
\pgfpathlineto{\pgfqpoint{3.300933in}{5.219248in}}%
\pgfpathlineto{\pgfqpoint{3.302618in}{5.220421in}}%
\pgfpathlineto{\pgfqpoint{3.303039in}{5.220002in}}%
\pgfpathlineto{\pgfqpoint{3.303461in}{5.220630in}}%
\pgfpathlineto{\pgfqpoint{3.305146in}{5.223647in}}%
\pgfpathlineto{\pgfqpoint{3.305567in}{5.223647in}}%
\pgfpathlineto{\pgfqpoint{3.307673in}{5.227586in}}%
\pgfpathlineto{\pgfqpoint{3.308095in}{5.227293in}}%
\pgfpathlineto{\pgfqpoint{3.313571in}{5.221762in}}%
\pgfpathlineto{\pgfqpoint{3.316099in}{5.213088in}}%
\pgfpathlineto{\pgfqpoint{3.316941in}{5.212166in}}%
\pgfpathlineto{\pgfqpoint{3.321575in}{5.195824in}}%
\pgfpathlineto{\pgfqpoint{3.327052in}{5.190628in}}%
\pgfpathlineto{\pgfqpoint{3.328737in}{5.193310in}}%
\pgfpathlineto{\pgfqpoint{3.330422in}{5.194735in}}%
\pgfpathlineto{\pgfqpoint{3.337584in}{5.208814in}}%
\pgfpathlineto{\pgfqpoint{3.338427in}{5.209903in}}%
\pgfpathlineto{\pgfqpoint{3.340112in}{5.213256in}}%
\pgfpathlineto{\pgfqpoint{3.341376in}{5.212962in}}%
\pgfpathlineto{\pgfqpoint{3.343903in}{5.212250in}}%
\pgfpathlineto{\pgfqpoint{3.344325in}{5.211160in}}%
\pgfpathlineto{\pgfqpoint{3.345167in}{5.211789in}}%
\pgfpathlineto{\pgfqpoint{3.349801in}{5.208982in}}%
\pgfpathlineto{\pgfqpoint{3.351486in}{5.209945in}}%
\pgfpathlineto{\pgfqpoint{3.352329in}{5.209275in}}%
\pgfpathlineto{\pgfqpoint{3.354014in}{5.211244in}}%
\pgfpathlineto{\pgfqpoint{3.354435in}{5.210197in}}%
\pgfpathlineto{\pgfqpoint{3.354856in}{5.211915in}}%
\pgfpathlineto{\pgfqpoint{3.356542in}{5.212543in}}%
\pgfpathlineto{\pgfqpoint{3.356963in}{5.212082in}}%
\pgfpathlineto{\pgfqpoint{3.357384in}{5.212920in}}%
\pgfpathlineto{\pgfqpoint{3.359069in}{5.212878in}}%
\pgfpathlineto{\pgfqpoint{3.359491in}{5.212627in}}%
\pgfpathlineto{\pgfqpoint{3.359912in}{5.213381in}}%
\pgfpathlineto{\pgfqpoint{3.360754in}{5.213591in}}%
\pgfpathlineto{\pgfqpoint{3.361597in}{5.214764in}}%
\pgfpathlineto{\pgfqpoint{3.362018in}{5.214261in}}%
\pgfpathlineto{\pgfqpoint{3.363703in}{5.211621in}}%
\pgfpathlineto{\pgfqpoint{3.366652in}{5.206886in}}%
\pgfpathlineto{\pgfqpoint{3.369601in}{5.199973in}}%
\pgfpathlineto{\pgfqpoint{3.372550in}{5.197417in}}%
\pgfpathlineto{\pgfqpoint{3.375078in}{5.192765in}}%
\pgfpathlineto{\pgfqpoint{3.375920in}{5.191215in}}%
\pgfpathlineto{\pgfqpoint{3.376342in}{5.191634in}}%
\pgfpathlineto{\pgfqpoint{3.378869in}{5.199470in}}%
\pgfpathlineto{\pgfqpoint{3.383082in}{5.213423in}}%
\pgfpathlineto{\pgfqpoint{3.389401in}{5.234709in}}%
\pgfpathlineto{\pgfqpoint{3.390665in}{5.233913in}}%
\pgfpathlineto{\pgfqpoint{3.400355in}{5.193310in}}%
\pgfpathlineto{\pgfqpoint{3.402040in}{5.189497in}}%
\pgfpathlineto{\pgfqpoint{3.403725in}{5.190628in}}%
\pgfpathlineto{\pgfqpoint{3.405831in}{5.194316in}}%
\pgfpathlineto{\pgfqpoint{3.411729in}{5.211202in}}%
\pgfpathlineto{\pgfqpoint{3.412993in}{5.214932in}}%
\pgfpathlineto{\pgfqpoint{3.414257in}{5.217488in}}%
\pgfpathlineto{\pgfqpoint{3.414678in}{5.217069in}}%
\pgfpathlineto{\pgfqpoint{3.420155in}{5.205504in}}%
\pgfpathlineto{\pgfqpoint{3.420576in}{5.205462in}}%
\pgfpathlineto{\pgfqpoint{3.423946in}{5.196201in}}%
\pgfpathlineto{\pgfqpoint{3.427316in}{5.192011in}}%
\pgfpathlineto{\pgfqpoint{3.427738in}{5.193017in}}%
\pgfpathlineto{\pgfqpoint{3.430687in}{5.198506in}}%
\pgfpathlineto{\pgfqpoint{3.431529in}{5.199847in}}%
\pgfpathlineto{\pgfqpoint{3.433635in}{5.204875in}}%
\pgfpathlineto{\pgfqpoint{3.436584in}{5.211831in}}%
\pgfpathlineto{\pgfqpoint{3.439112in}{5.217152in}}%
\pgfpathlineto{\pgfqpoint{3.439955in}{5.215518in}}%
\pgfpathlineto{\pgfqpoint{3.443746in}{5.209275in}}%
\pgfpathlineto{\pgfqpoint{3.445010in}{5.208102in}}%
\pgfpathlineto{\pgfqpoint{3.451750in}{5.196201in}}%
\pgfpathlineto{\pgfqpoint{3.457648in}{5.207724in}}%
\pgfpathlineto{\pgfqpoint{3.463546in}{5.203911in}}%
\pgfpathlineto{\pgfqpoint{3.463968in}{5.203870in}}%
\pgfpathlineto{\pgfqpoint{3.466495in}{5.199428in}}%
\pgfpathlineto{\pgfqpoint{3.469444in}{5.195028in}}%
\pgfpathlineto{\pgfqpoint{3.474499in}{5.189288in}}%
\pgfpathlineto{\pgfqpoint{3.475763in}{5.188827in}}%
\pgfpathlineto{\pgfqpoint{3.478291in}{5.191089in}}%
\pgfpathlineto{\pgfqpoint{3.479976in}{5.190880in}}%
\pgfpathlineto{\pgfqpoint{3.482925in}{5.191760in}}%
\pgfpathlineto{\pgfqpoint{3.484610in}{5.191131in}}%
\pgfpathlineto{\pgfqpoint{3.485031in}{5.190922in}}%
\pgfpathlineto{\pgfqpoint{3.485453in}{5.191886in}}%
\pgfpathlineto{\pgfqpoint{3.486717in}{5.192472in}}%
\pgfpathlineto{\pgfqpoint{3.490929in}{5.195112in}}%
\pgfpathlineto{\pgfqpoint{3.492614in}{5.194819in}}%
\pgfpathlineto{\pgfqpoint{3.494721in}{5.195992in}}%
\pgfpathlineto{\pgfqpoint{3.496406in}{5.194651in}}%
\pgfpathlineto{\pgfqpoint{3.497248in}{5.194609in}}%
\pgfpathlineto{\pgfqpoint{3.498934in}{5.193604in}}%
\pgfpathlineto{\pgfqpoint{3.500619in}{5.194567in}}%
\pgfpathlineto{\pgfqpoint{3.501461in}{5.194442in}}%
\pgfpathlineto{\pgfqpoint{3.502304in}{5.197165in}}%
\pgfpathlineto{\pgfqpoint{3.502725in}{5.196411in}}%
\pgfpathlineto{\pgfqpoint{3.503146in}{5.196788in}}%
\pgfpathlineto{\pgfqpoint{3.503568in}{5.196118in}}%
\pgfpathlineto{\pgfqpoint{3.503989in}{5.195657in}}%
\pgfpathlineto{\pgfqpoint{3.504410in}{5.196537in}}%
\pgfpathlineto{\pgfqpoint{3.504831in}{5.196537in}}%
\pgfpathlineto{\pgfqpoint{3.505253in}{5.195280in}}%
\pgfpathlineto{\pgfqpoint{3.505674in}{5.196243in}}%
\pgfpathlineto{\pgfqpoint{3.509466in}{5.195699in}}%
\pgfpathlineto{\pgfqpoint{3.510308in}{5.194777in}}%
\pgfpathlineto{\pgfqpoint{3.510729in}{5.195657in}}%
\pgfpathlineto{\pgfqpoint{3.513678in}{5.196285in}}%
\pgfpathlineto{\pgfqpoint{3.514100in}{5.195741in}}%
\pgfpathlineto{\pgfqpoint{3.514942in}{5.198045in}}%
\pgfpathlineto{\pgfqpoint{3.515363in}{5.197710in}}%
\pgfpathlineto{\pgfqpoint{3.519155in}{5.199176in}}%
\pgfpathlineto{\pgfqpoint{3.520840in}{5.200601in}}%
\pgfpathlineto{\pgfqpoint{3.522946in}{5.199260in}}%
\pgfpathlineto{\pgfqpoint{3.523789in}{5.199847in}}%
\pgfpathlineto{\pgfqpoint{3.524210in}{5.199512in}}%
\pgfpathlineto{\pgfqpoint{3.525474in}{5.198548in}}%
\pgfpathlineto{\pgfqpoint{3.525895in}{5.199679in}}%
\pgfpathlineto{\pgfqpoint{3.527159in}{5.201481in}}%
\pgfpathlineto{\pgfqpoint{3.528423in}{5.202571in}}%
\pgfpathlineto{\pgfqpoint{3.529266in}{5.200978in}}%
\pgfpathlineto{\pgfqpoint{3.529687in}{5.201942in}}%
\pgfpathlineto{\pgfqpoint{3.531793in}{5.201691in}}%
\pgfpathlineto{\pgfqpoint{3.533478in}{5.202403in}}%
\pgfpathlineto{\pgfqpoint{3.540219in}{5.196243in}}%
\pgfpathlineto{\pgfqpoint{3.541061in}{5.196118in}}%
\pgfpathlineto{\pgfqpoint{3.542747in}{5.193771in}}%
\pgfpathlineto{\pgfqpoint{3.544853in}{5.191886in}}%
\pgfpathlineto{\pgfqpoint{3.547802in}{5.185558in}}%
\pgfpathlineto{\pgfqpoint{3.549487in}{5.184595in}}%
\pgfpathlineto{\pgfqpoint{3.549908in}{5.185056in}}%
\pgfpathlineto{\pgfqpoint{3.551593in}{5.183044in}}%
\pgfpathlineto{\pgfqpoint{3.553700in}{5.185935in}}%
\pgfpathlineto{\pgfqpoint{3.555385in}{5.183673in}}%
\pgfpathlineto{\pgfqpoint{3.557491in}{5.184595in}}%
\pgfpathlineto{\pgfqpoint{3.557913in}{5.183463in}}%
\pgfpathlineto{\pgfqpoint{3.558334in}{5.184301in}}%
\pgfpathlineto{\pgfqpoint{3.558755in}{5.185056in}}%
\pgfpathlineto{\pgfqpoint{3.559176in}{5.183798in}}%
\pgfpathlineto{\pgfqpoint{3.559598in}{5.184427in}}%
\pgfpathlineto{\pgfqpoint{3.560019in}{5.184720in}}%
\pgfpathlineto{\pgfqpoint{3.561704in}{5.182039in}}%
\pgfpathlineto{\pgfqpoint{3.562547in}{5.183421in}}%
\pgfpathlineto{\pgfqpoint{3.562968in}{5.182374in}}%
\pgfpathlineto{\pgfqpoint{3.565074in}{5.185433in}}%
\pgfpathlineto{\pgfqpoint{3.565496in}{5.185097in}}%
\pgfpathlineto{\pgfqpoint{3.567602in}{5.191006in}}%
\pgfpathlineto{\pgfqpoint{3.568444in}{5.192053in}}%
\pgfpathlineto{\pgfqpoint{3.570130in}{5.195573in}}%
\pgfpathlineto{\pgfqpoint{3.570551in}{5.194944in}}%
\pgfpathlineto{\pgfqpoint{3.570972in}{5.195992in}}%
\pgfpathlineto{\pgfqpoint{3.578134in}{5.216943in}}%
\pgfpathlineto{\pgfqpoint{3.578976in}{5.221678in}}%
\pgfpathlineto{\pgfqpoint{3.579398in}{5.220379in}}%
\pgfpathlineto{\pgfqpoint{3.580240in}{5.221887in}}%
\pgfpathlineto{\pgfqpoint{3.581925in}{5.218242in}}%
\pgfpathlineto{\pgfqpoint{3.582768in}{5.218535in}}%
\pgfpathlineto{\pgfqpoint{3.583189in}{5.217194in}}%
\pgfpathlineto{\pgfqpoint{3.583610in}{5.217865in}}%
\pgfpathlineto{\pgfqpoint{3.584453in}{5.217362in}}%
\pgfpathlineto{\pgfqpoint{3.585296in}{5.220295in}}%
\pgfpathlineto{\pgfqpoint{3.586981in}{5.217571in}}%
\pgfpathlineto{\pgfqpoint{3.587823in}{5.218661in}}%
\pgfpathlineto{\pgfqpoint{3.588245in}{5.217613in}}%
\pgfpathlineto{\pgfqpoint{3.590351in}{5.219667in}}%
\pgfpathlineto{\pgfqpoint{3.592036in}{5.216985in}}%
\pgfpathlineto{\pgfqpoint{3.594142in}{5.217655in}}%
\pgfpathlineto{\pgfqpoint{3.594564in}{5.216566in}}%
\pgfpathlineto{\pgfqpoint{3.594985in}{5.217111in}}%
\pgfpathlineto{\pgfqpoint{3.595828in}{5.216440in}}%
\pgfpathlineto{\pgfqpoint{3.596670in}{5.218284in}}%
\pgfpathlineto{\pgfqpoint{3.597091in}{5.217027in}}%
\pgfpathlineto{\pgfqpoint{3.597513in}{5.218200in}}%
\pgfpathlineto{\pgfqpoint{3.597934in}{5.218703in}}%
\pgfpathlineto{\pgfqpoint{3.600040in}{5.210155in}}%
\pgfpathlineto{\pgfqpoint{3.601725in}{5.205294in}}%
\pgfpathlineto{\pgfqpoint{3.603832in}{5.198841in}}%
\pgfpathlineto{\pgfqpoint{3.605096in}{5.197542in}}%
\pgfpathlineto{\pgfqpoint{3.608466in}{5.192095in}}%
\pgfpathlineto{\pgfqpoint{3.610572in}{5.189246in}}%
\pgfpathlineto{\pgfqpoint{3.612257in}{5.183589in}}%
\pgfpathlineto{\pgfqpoint{3.613100in}{5.183254in}}%
\pgfpathlineto{\pgfqpoint{3.614785in}{5.180991in}}%
\pgfpathlineto{\pgfqpoint{3.615628in}{5.180404in}}%
\pgfpathlineto{\pgfqpoint{3.616470in}{5.178770in}}%
\pgfpathlineto{\pgfqpoint{3.617734in}{5.180614in}}%
\pgfpathlineto{\pgfqpoint{3.618577in}{5.178854in}}%
\pgfpathlineto{\pgfqpoint{3.619419in}{5.179566in}}%
\pgfpathlineto{\pgfqpoint{3.621104in}{5.177806in}}%
\pgfpathlineto{\pgfqpoint{3.621526in}{5.178770in}}%
\pgfpathlineto{\pgfqpoint{3.622368in}{5.178267in}}%
\pgfpathlineto{\pgfqpoint{3.627423in}{5.178351in}}%
\pgfpathlineto{\pgfqpoint{3.630372in}{5.183212in}}%
\pgfpathlineto{\pgfqpoint{3.630794in}{5.182835in}}%
\pgfpathlineto{\pgfqpoint{3.631215in}{5.183463in}}%
\pgfpathlineto{\pgfqpoint{3.632900in}{5.184217in}}%
\pgfpathlineto{\pgfqpoint{3.633321in}{5.183840in}}%
\pgfpathlineto{\pgfqpoint{3.633743in}{5.184553in}}%
\pgfpathlineto{\pgfqpoint{3.635428in}{5.185307in}}%
\pgfpathlineto{\pgfqpoint{3.636692in}{5.185139in}}%
\pgfpathlineto{\pgfqpoint{3.637113in}{5.183757in}}%
\pgfpathlineto{\pgfqpoint{3.637534in}{5.184176in}}%
\pgfpathlineto{\pgfqpoint{3.639219in}{5.185600in}}%
\pgfpathlineto{\pgfqpoint{3.640904in}{5.183170in}}%
\pgfpathlineto{\pgfqpoint{3.643011in}{5.185097in}}%
\pgfpathlineto{\pgfqpoint{3.643432in}{5.183505in}}%
\pgfpathlineto{\pgfqpoint{3.643853in}{5.184343in}}%
\pgfpathlineto{\pgfqpoint{3.645538in}{5.187947in}}%
\pgfpathlineto{\pgfqpoint{3.645960in}{5.187528in}}%
\pgfpathlineto{\pgfqpoint{3.646802in}{5.187821in}}%
\pgfpathlineto{\pgfqpoint{3.647223in}{5.186187in}}%
\pgfpathlineto{\pgfqpoint{3.648066in}{5.187276in}}%
\pgfpathlineto{\pgfqpoint{3.648487in}{5.186187in}}%
\pgfpathlineto{\pgfqpoint{3.649330in}{5.187025in}}%
\pgfpathlineto{\pgfqpoint{3.649751in}{5.186480in}}%
\pgfpathlineto{\pgfqpoint{3.654385in}{5.200978in}}%
\pgfpathlineto{\pgfqpoint{3.654807in}{5.201230in}}%
\pgfpathlineto{\pgfqpoint{3.656913in}{5.207976in}}%
\pgfpathlineto{\pgfqpoint{3.657334in}{5.207222in}}%
\pgfpathlineto{\pgfqpoint{3.657755in}{5.208353in}}%
\pgfpathlineto{\pgfqpoint{3.658177in}{5.208353in}}%
\pgfpathlineto{\pgfqpoint{3.658598in}{5.206719in}}%
\pgfpathlineto{\pgfqpoint{3.659019in}{5.207180in}}%
\pgfpathlineto{\pgfqpoint{3.664496in}{5.217697in}}%
\pgfpathlineto{\pgfqpoint{3.664917in}{5.216733in}}%
\pgfpathlineto{\pgfqpoint{3.665338in}{5.217781in}}%
\pgfpathlineto{\pgfqpoint{3.667024in}{5.219792in}}%
\pgfpathlineto{\pgfqpoint{3.667445in}{5.218996in}}%
\pgfpathlineto{\pgfqpoint{3.667866in}{5.219625in}}%
\pgfpathlineto{\pgfqpoint{3.668287in}{5.220127in}}%
\pgfpathlineto{\pgfqpoint{3.668709in}{5.218870in}}%
\pgfpathlineto{\pgfqpoint{3.669130in}{5.219667in}}%
\pgfpathlineto{\pgfqpoint{3.669551in}{5.219876in}}%
\pgfpathlineto{\pgfqpoint{3.669973in}{5.218409in}}%
\pgfpathlineto{\pgfqpoint{3.670815in}{5.219499in}}%
\pgfpathlineto{\pgfqpoint{3.671236in}{5.218074in}}%
\pgfpathlineto{\pgfqpoint{3.671658in}{5.219248in}}%
\pgfpathlineto{\pgfqpoint{3.672079in}{5.219708in}}%
\pgfpathlineto{\pgfqpoint{3.672500in}{5.218242in}}%
\pgfpathlineto{\pgfqpoint{3.672921in}{5.219625in}}%
\pgfpathlineto{\pgfqpoint{3.673343in}{5.220169in}}%
\pgfpathlineto{\pgfqpoint{3.673764in}{5.219038in}}%
\pgfpathlineto{\pgfqpoint{3.674607in}{5.219080in}}%
\pgfpathlineto{\pgfqpoint{3.675028in}{5.218074in}}%
\pgfpathlineto{\pgfqpoint{3.675449in}{5.218996in}}%
\pgfpathlineto{\pgfqpoint{3.675870in}{5.219373in}}%
\pgfpathlineto{\pgfqpoint{3.676292in}{5.217613in}}%
\pgfpathlineto{\pgfqpoint{3.676713in}{5.218451in}}%
\pgfpathlineto{\pgfqpoint{3.678398in}{5.220840in}}%
\pgfpathlineto{\pgfqpoint{3.679241in}{5.219708in}}%
\pgfpathlineto{\pgfqpoint{3.679662in}{5.220714in}}%
\pgfpathlineto{\pgfqpoint{3.681347in}{5.217069in}}%
\pgfpathlineto{\pgfqpoint{3.682190in}{5.218954in}}%
\pgfpathlineto{\pgfqpoint{3.682611in}{5.217278in}}%
\pgfpathlineto{\pgfqpoint{3.684717in}{5.220086in}}%
\pgfpathlineto{\pgfqpoint{3.685560in}{5.218829in}}%
\pgfpathlineto{\pgfqpoint{3.685981in}{5.218912in}}%
\pgfpathlineto{\pgfqpoint{3.687666in}{5.216440in}}%
\pgfpathlineto{\pgfqpoint{3.688930in}{5.215225in}}%
\pgfpathlineto{\pgfqpoint{3.694407in}{5.212459in}}%
\pgfpathlineto{\pgfqpoint{3.696092in}{5.212208in}}%
\pgfpathlineto{\pgfqpoint{3.697777in}{5.211202in}}%
\pgfpathlineto{\pgfqpoint{3.700305in}{5.210783in}}%
\pgfpathlineto{\pgfqpoint{3.703253in}{5.211789in}}%
\pgfpathlineto{\pgfqpoint{3.705360in}{5.207222in}}%
\pgfpathlineto{\pgfqpoint{3.707045in}{5.208856in}}%
\pgfpathlineto{\pgfqpoint{3.709151in}{5.207473in}}%
\pgfpathlineto{\pgfqpoint{3.709573in}{5.208898in}}%
\pgfpathlineto{\pgfqpoint{3.709994in}{5.208604in}}%
\pgfpathlineto{\pgfqpoint{3.711679in}{5.204624in}}%
\pgfpathlineto{\pgfqpoint{3.714628in}{5.207976in}}%
\pgfpathlineto{\pgfqpoint{3.715471in}{5.206761in}}%
\pgfpathlineto{\pgfqpoint{3.717156in}{5.208521in}}%
\pgfpathlineto{\pgfqpoint{3.720105in}{5.205252in}}%
\pgfpathlineto{\pgfqpoint{3.723054in}{5.199302in}}%
\pgfpathlineto{\pgfqpoint{3.723475in}{5.199637in}}%
\pgfpathlineto{\pgfqpoint{3.723896in}{5.198967in}}%
\pgfpathlineto{\pgfqpoint{3.725581in}{5.198087in}}%
\pgfpathlineto{\pgfqpoint{3.726003in}{5.199009in}}%
\pgfpathlineto{\pgfqpoint{3.726424in}{5.198003in}}%
\pgfpathlineto{\pgfqpoint{3.727688in}{5.196076in}}%
\pgfpathlineto{\pgfqpoint{3.728109in}{5.196788in}}%
\pgfpathlineto{\pgfqpoint{3.728530in}{5.197165in}}%
\pgfpathlineto{\pgfqpoint{3.728951in}{5.196495in}}%
\pgfpathlineto{\pgfqpoint{3.730637in}{5.195154in}}%
\pgfpathlineto{\pgfqpoint{3.731479in}{5.194567in}}%
\pgfpathlineto{\pgfqpoint{3.731900in}{5.194148in}}%
\pgfpathlineto{\pgfqpoint{3.735692in}{5.201942in}}%
\pgfpathlineto{\pgfqpoint{3.737377in}{5.203241in}}%
\pgfpathlineto{\pgfqpoint{3.738220in}{5.202529in}}%
\pgfpathlineto{\pgfqpoint{3.738641in}{5.203450in}}%
\pgfpathlineto{\pgfqpoint{3.739062in}{5.202990in}}%
\pgfpathlineto{\pgfqpoint{3.740326in}{5.202612in}}%
\pgfpathlineto{\pgfqpoint{3.740747in}{5.202612in}}%
\pgfpathlineto{\pgfqpoint{3.741169in}{5.204163in}}%
\pgfpathlineto{\pgfqpoint{3.741590in}{5.202822in}}%
\pgfpathlineto{\pgfqpoint{3.745381in}{5.203702in}}%
\pgfpathlineto{\pgfqpoint{3.746645in}{5.203450in}}%
\pgfpathlineto{\pgfqpoint{3.750015in}{5.205294in}}%
\pgfpathlineto{\pgfqpoint{3.750858in}{5.204498in}}%
\pgfpathlineto{\pgfqpoint{3.751279in}{5.205462in}}%
\pgfpathlineto{\pgfqpoint{3.752122in}{5.204708in}}%
\pgfpathlineto{\pgfqpoint{3.752543in}{5.205043in}}%
\pgfpathlineto{\pgfqpoint{3.752964in}{5.204372in}}%
\pgfpathlineto{\pgfqpoint{3.754228in}{5.204330in}}%
\pgfpathlineto{\pgfqpoint{3.758862in}{5.204540in}}%
\pgfpathlineto{\pgfqpoint{3.773186in}{5.156604in}}%
\pgfpathlineto{\pgfqpoint{3.776135in}{5.166367in}}%
\pgfpathlineto{\pgfqpoint{3.788773in}{5.193059in}}%
\pgfpathlineto{\pgfqpoint{3.789194in}{5.192179in}}%
\pgfpathlineto{\pgfqpoint{3.791301in}{5.189162in}}%
\pgfpathlineto{\pgfqpoint{3.793407in}{5.188240in}}%
\pgfpathlineto{\pgfqpoint{3.793828in}{5.189413in}}%
\pgfpathlineto{\pgfqpoint{3.794250in}{5.188408in}}%
\pgfpathlineto{\pgfqpoint{3.795935in}{5.187653in}}%
\pgfpathlineto{\pgfqpoint{3.796356in}{5.188408in}}%
\pgfpathlineto{\pgfqpoint{3.796777in}{5.187905in}}%
\pgfpathlineto{\pgfqpoint{3.797199in}{5.187151in}}%
\pgfpathlineto{\pgfqpoint{3.797620in}{5.188659in}}%
\pgfpathlineto{\pgfqpoint{3.800569in}{5.189204in}}%
\pgfpathlineto{\pgfqpoint{3.802254in}{5.188240in}}%
\pgfpathlineto{\pgfqpoint{3.804360in}{5.191215in}}%
\pgfpathlineto{\pgfqpoint{3.806045in}{5.193436in}}%
\pgfpathlineto{\pgfqpoint{3.807730in}{5.195782in}}%
\pgfpathlineto{\pgfqpoint{3.808573in}{5.194609in}}%
\pgfpathlineto{\pgfqpoint{3.808994in}{5.195782in}}%
\pgfpathlineto{\pgfqpoint{3.809837in}{5.196495in}}%
\pgfpathlineto{\pgfqpoint{3.811522in}{5.199763in}}%
\pgfpathlineto{\pgfqpoint{3.812365in}{5.199386in}}%
\pgfpathlineto{\pgfqpoint{3.812786in}{5.200643in}}%
\pgfpathlineto{\pgfqpoint{3.813207in}{5.199847in}}%
\pgfpathlineto{\pgfqpoint{3.816156in}{5.196034in}}%
\pgfpathlineto{\pgfqpoint{3.817841in}{5.197081in}}%
\pgfpathlineto{\pgfqpoint{3.819948in}{5.195238in}}%
\pgfpathlineto{\pgfqpoint{3.821633in}{5.197375in}}%
\pgfpathlineto{\pgfqpoint{3.822475in}{5.197668in}}%
\pgfpathlineto{\pgfqpoint{3.822896in}{5.199135in}}%
\pgfpathlineto{\pgfqpoint{3.823318in}{5.198297in}}%
\pgfpathlineto{\pgfqpoint{3.825003in}{5.197752in}}%
\pgfpathlineto{\pgfqpoint{3.825424in}{5.198087in}}%
\pgfpathlineto{\pgfqpoint{3.826267in}{5.195699in}}%
\pgfpathlineto{\pgfqpoint{3.827109in}{5.196285in}}%
\pgfpathlineto{\pgfqpoint{3.828794in}{5.201774in}}%
\pgfpathlineto{\pgfqpoint{3.830479in}{5.206509in}}%
\pgfpathlineto{\pgfqpoint{3.831322in}{5.206467in}}%
\pgfpathlineto{\pgfqpoint{3.831743in}{5.208102in}}%
\pgfpathlineto{\pgfqpoint{3.832165in}{5.206970in}}%
\pgfpathlineto{\pgfqpoint{3.832586in}{5.206384in}}%
\pgfpathlineto{\pgfqpoint{3.833007in}{5.207180in}}%
\pgfpathlineto{\pgfqpoint{3.835535in}{5.209442in}}%
\pgfpathlineto{\pgfqpoint{3.835956in}{5.208898in}}%
\pgfpathlineto{\pgfqpoint{3.838905in}{5.207641in}}%
\pgfpathlineto{\pgfqpoint{3.840590in}{5.210951in}}%
\pgfpathlineto{\pgfqpoint{3.845224in}{5.206761in}}%
\pgfpathlineto{\pgfqpoint{3.846909in}{5.209107in}}%
\pgfpathlineto{\pgfqpoint{3.849016in}{5.207599in}}%
\pgfpathlineto{\pgfqpoint{3.850280in}{5.204708in}}%
\pgfpathlineto{\pgfqpoint{3.851122in}{5.203450in}}%
\pgfpathlineto{\pgfqpoint{3.851543in}{5.204372in}}%
\pgfpathlineto{\pgfqpoint{3.853229in}{5.206048in}}%
\pgfpathlineto{\pgfqpoint{3.854914in}{5.204540in}}%
\pgfpathlineto{\pgfqpoint{3.855756in}{5.205923in}}%
\pgfpathlineto{\pgfqpoint{3.856177in}{5.205127in}}%
\pgfpathlineto{\pgfqpoint{3.858284in}{5.202152in}}%
\pgfpathlineto{\pgfqpoint{3.860390in}{5.196788in}}%
\pgfpathlineto{\pgfqpoint{3.862497in}{5.199344in}}%
\pgfpathlineto{\pgfqpoint{3.864603in}{5.207389in}}%
\pgfpathlineto{\pgfqpoint{3.865024in}{5.207096in}}%
\pgfpathlineto{\pgfqpoint{3.867973in}{5.210951in}}%
\pgfpathlineto{\pgfqpoint{3.869658in}{5.207389in}}%
\pgfpathlineto{\pgfqpoint{3.871343in}{5.205210in}}%
\pgfpathlineto{\pgfqpoint{3.872186in}{5.205462in}}%
\pgfpathlineto{\pgfqpoint{3.876399in}{5.194651in}}%
\pgfpathlineto{\pgfqpoint{3.882718in}{5.170976in}}%
\pgfpathlineto{\pgfqpoint{3.883561in}{5.172024in}}%
\pgfpathlineto{\pgfqpoint{3.883982in}{5.171605in}}%
\pgfpathlineto{\pgfqpoint{3.884403in}{5.172275in}}%
\pgfpathlineto{\pgfqpoint{3.886088in}{5.172778in}}%
\pgfpathlineto{\pgfqpoint{3.887773in}{5.171773in}}%
\pgfpathlineto{\pgfqpoint{3.891144in}{5.173323in}}%
\pgfpathlineto{\pgfqpoint{3.891565in}{5.172569in}}%
\pgfpathlineto{\pgfqpoint{3.892407in}{5.175167in}}%
\pgfpathlineto{\pgfqpoint{3.892829in}{5.174371in}}%
\pgfpathlineto{\pgfqpoint{3.893671in}{5.174706in}}%
\pgfpathlineto{\pgfqpoint{3.894092in}{5.174035in}}%
\pgfpathlineto{\pgfqpoint{3.894514in}{5.174580in}}%
\pgfpathlineto{\pgfqpoint{3.897463in}{5.175879in}}%
\pgfpathlineto{\pgfqpoint{3.897884in}{5.175502in}}%
\pgfpathlineto{\pgfqpoint{3.898305in}{5.176172in}}%
\pgfpathlineto{\pgfqpoint{3.901254in}{5.178728in}}%
\pgfpathlineto{\pgfqpoint{3.902939in}{5.179231in}}%
\pgfpathlineto{\pgfqpoint{3.906731in}{5.180698in}}%
\pgfpathlineto{\pgfqpoint{3.907995in}{5.179818in}}%
\pgfpathlineto{\pgfqpoint{3.908837in}{5.221049in}}%
\pgfpathlineto{\pgfqpoint{3.909258in}{5.220505in}}%
\pgfpathlineto{\pgfqpoint{3.913050in}{5.217739in}}%
\pgfpathlineto{\pgfqpoint{3.917263in}{5.213675in}}%
\pgfpathlineto{\pgfqpoint{3.917684in}{5.214429in}}%
\pgfpathlineto{\pgfqpoint{3.918105in}{5.213549in}}%
\pgfpathlineto{\pgfqpoint{3.919790in}{5.212208in}}%
\pgfpathlineto{\pgfqpoint{3.921476in}{5.213675in}}%
\pgfpathlineto{\pgfqpoint{3.922318in}{5.172359in}}%
\pgfpathlineto{\pgfqpoint{3.922739in}{5.172653in}}%
\pgfpathlineto{\pgfqpoint{3.925267in}{5.173826in}}%
\pgfpathlineto{\pgfqpoint{3.927373in}{5.175544in}}%
\pgfpathlineto{\pgfqpoint{3.930744in}{5.176549in}}%
\pgfpathlineto{\pgfqpoint{3.931165in}{5.175921in}}%
\pgfpathlineto{\pgfqpoint{3.931586in}{5.176214in}}%
\pgfpathlineto{\pgfqpoint{3.933271in}{5.176675in}}%
\pgfpathlineto{\pgfqpoint{3.933693in}{5.175669in}}%
\pgfpathlineto{\pgfqpoint{3.934114in}{5.176005in}}%
\pgfpathlineto{\pgfqpoint{3.935799in}{5.176885in}}%
\pgfpathlineto{\pgfqpoint{3.936220in}{5.176466in}}%
\pgfpathlineto{\pgfqpoint{3.936642in}{5.177220in}}%
\pgfpathlineto{\pgfqpoint{3.938327in}{5.178100in}}%
\pgfpathlineto{\pgfqpoint{3.938748in}{5.176591in}}%
\pgfpathlineto{\pgfqpoint{3.939591in}{5.177094in}}%
\pgfpathlineto{\pgfqpoint{3.940012in}{5.176591in}}%
\pgfpathlineto{\pgfqpoint{3.940433in}{5.176968in}}%
\pgfpathlineto{\pgfqpoint{3.942118in}{5.178938in}}%
\pgfpathlineto{\pgfqpoint{3.942539in}{5.178393in}}%
\pgfpathlineto{\pgfqpoint{3.942961in}{5.178728in}}%
\pgfpathlineto{\pgfqpoint{3.944646in}{5.179357in}}%
\pgfpathlineto{\pgfqpoint{3.945067in}{5.178645in}}%
\pgfpathlineto{\pgfqpoint{3.945488in}{5.179357in}}%
\pgfpathlineto{\pgfqpoint{3.947174in}{5.180907in}}%
\pgfpathlineto{\pgfqpoint{3.948859in}{5.180153in}}%
\pgfpathlineto{\pgfqpoint{3.950965in}{5.180404in}}%
\pgfpathlineto{\pgfqpoint{3.951808in}{5.180321in}}%
\pgfpathlineto{\pgfqpoint{3.953914in}{5.183505in}}%
\pgfpathlineto{\pgfqpoint{3.954335in}{5.183631in}}%
\pgfpathlineto{\pgfqpoint{3.957284in}{5.190419in}}%
\pgfpathlineto{\pgfqpoint{3.957705in}{5.190796in}}%
\pgfpathlineto{\pgfqpoint{3.960654in}{5.199637in}}%
\pgfpathlineto{\pgfqpoint{3.966974in}{5.216859in}}%
\pgfpathlineto{\pgfqpoint{3.967816in}{5.217613in}}%
\pgfpathlineto{\pgfqpoint{3.969923in}{5.218535in}}%
\pgfpathlineto{\pgfqpoint{3.970344in}{5.218200in}}%
\pgfpathlineto{\pgfqpoint{3.970765in}{5.218787in}}%
\pgfpathlineto{\pgfqpoint{3.972450in}{5.220044in}}%
\pgfpathlineto{\pgfqpoint{3.972871in}{5.219122in}}%
\pgfpathlineto{\pgfqpoint{3.973293in}{5.220127in}}%
\pgfpathlineto{\pgfqpoint{3.974557in}{5.220546in}}%
\pgfpathlineto{\pgfqpoint{3.975399in}{5.220756in}}%
\pgfpathlineto{\pgfqpoint{3.977506in}{5.222181in}}%
\pgfpathlineto{\pgfqpoint{3.982561in}{5.215812in}}%
\pgfpathlineto{\pgfqpoint{3.983825in}{5.216356in}}%
\pgfpathlineto{\pgfqpoint{3.989301in}{5.212543in}}%
\pgfpathlineto{\pgfqpoint{3.990144in}{5.212334in}}%
\pgfpathlineto{\pgfqpoint{3.990986in}{5.211621in}}%
\pgfpathlineto{\pgfqpoint{3.991408in}{5.211915in}}%
\pgfpathlineto{\pgfqpoint{3.995199in}{5.219541in}}%
\pgfpathlineto{\pgfqpoint{3.995621in}{5.219918in}}%
\pgfpathlineto{\pgfqpoint{3.997306in}{5.222725in}}%
\pgfpathlineto{\pgfqpoint{3.998991in}{5.225617in}}%
\pgfpathlineto{\pgfqpoint{4.001097in}{5.229346in}}%
\pgfpathlineto{\pgfqpoint{4.001518in}{5.228885in}}%
\pgfpathlineto{\pgfqpoint{4.004889in}{5.236008in}}%
\pgfpathlineto{\pgfqpoint{4.005731in}{5.236427in}}%
\pgfpathlineto{\pgfqpoint{4.009523in}{5.240743in}}%
\pgfpathlineto{\pgfqpoint{4.016263in}{5.257672in}}%
\pgfpathlineto{\pgfqpoint{4.017106in}{5.258174in}}%
\pgfpathlineto{\pgfqpoint{4.018791in}{5.259976in}}%
\pgfpathlineto{\pgfqpoint{4.019212in}{5.260018in}}%
\pgfpathlineto{\pgfqpoint{4.021318in}{5.265424in}}%
\pgfpathlineto{\pgfqpoint{4.021740in}{5.265172in}}%
\pgfpathlineto{\pgfqpoint{4.025953in}{5.260856in}}%
\pgfpathlineto{\pgfqpoint{4.027216in}{5.261694in}}%
\pgfpathlineto{\pgfqpoint{4.027638in}{5.261359in}}%
\pgfpathlineto{\pgfqpoint{4.029323in}{5.258677in}}%
\pgfpathlineto{\pgfqpoint{4.030165in}{5.259180in}}%
\pgfpathlineto{\pgfqpoint{4.030587in}{5.258677in}}%
\pgfpathlineto{\pgfqpoint{4.036906in}{5.263287in}}%
\pgfpathlineto{\pgfqpoint{4.038170in}{5.265633in}}%
\pgfpathlineto{\pgfqpoint{4.038591in}{5.265382in}}%
\pgfpathlineto{\pgfqpoint{4.039012in}{5.264753in}}%
\pgfpathlineto{\pgfqpoint{4.039433in}{5.265256in}}%
\pgfpathlineto{\pgfqpoint{4.040276in}{5.267561in}}%
\pgfpathlineto{\pgfqpoint{4.040697in}{5.265968in}}%
\pgfpathlineto{\pgfqpoint{4.042382in}{5.265340in}}%
\pgfpathlineto{\pgfqpoint{4.044067in}{5.268147in}}%
\pgfpathlineto{\pgfqpoint{4.044910in}{5.267183in}}%
\pgfpathlineto{\pgfqpoint{4.045331in}{5.267980in}}%
\pgfpathlineto{\pgfqpoint{4.048702in}{5.264208in}}%
\pgfpathlineto{\pgfqpoint{4.050387in}{5.265424in}}%
\pgfpathlineto{\pgfqpoint{4.050808in}{5.264292in}}%
\pgfpathlineto{\pgfqpoint{4.051651in}{5.264795in}}%
\pgfpathlineto{\pgfqpoint{4.052493in}{5.266262in}}%
\pgfpathlineto{\pgfqpoint{4.052914in}{5.265088in}}%
\pgfpathlineto{\pgfqpoint{4.054599in}{5.262532in}}%
\pgfpathlineto{\pgfqpoint{4.055863in}{5.262784in}}%
\pgfpathlineto{\pgfqpoint{4.060076in}{5.258761in}}%
\pgfpathlineto{\pgfqpoint{4.065553in}{5.253733in}}%
\pgfpathlineto{\pgfqpoint{4.065974in}{5.252560in}}%
\pgfpathlineto{\pgfqpoint{4.066817in}{5.252895in}}%
\pgfpathlineto{\pgfqpoint{4.070187in}{5.253900in}}%
\pgfpathlineto{\pgfqpoint{4.070608in}{5.253188in}}%
\pgfpathlineto{\pgfqpoint{4.071029in}{5.253691in}}%
\pgfpathlineto{\pgfqpoint{4.072714in}{5.254948in}}%
\pgfpathlineto{\pgfqpoint{4.074821in}{5.255074in}}%
\pgfpathlineto{\pgfqpoint{4.076085in}{5.256582in}}%
\pgfpathlineto{\pgfqpoint{4.079455in}{5.253733in}}%
\pgfpathlineto{\pgfqpoint{4.080719in}{5.253565in}}%
\pgfpathlineto{\pgfqpoint{4.084931in}{5.254948in}}%
\pgfpathlineto{\pgfqpoint{4.087459in}{5.254403in}}%
\pgfpathlineto{\pgfqpoint{4.089144in}{5.255535in}}%
\pgfpathlineto{\pgfqpoint{4.089566in}{5.255241in}}%
\pgfpathlineto{\pgfqpoint{4.089987in}{5.255870in}}%
\pgfpathlineto{\pgfqpoint{4.090829in}{5.256079in}}%
\pgfpathlineto{\pgfqpoint{4.092093in}{5.255199in}}%
\pgfpathlineto{\pgfqpoint{4.093357in}{5.256163in}}%
\pgfpathlineto{\pgfqpoint{4.095463in}{5.257169in}}%
\pgfpathlineto{\pgfqpoint{4.095885in}{5.256289in}}%
\pgfpathlineto{\pgfqpoint{4.096306in}{5.257253in}}%
\pgfpathlineto{\pgfqpoint{4.097149in}{5.258426in}}%
\pgfpathlineto{\pgfqpoint{4.097570in}{5.257336in}}%
\pgfpathlineto{\pgfqpoint{4.099676in}{5.258133in}}%
\pgfpathlineto{\pgfqpoint{4.100940in}{5.258719in}}%
\pgfpathlineto{\pgfqpoint{4.101783in}{5.259557in}}%
\pgfpathlineto{\pgfqpoint{4.102204in}{5.258845in}}%
\pgfpathlineto{\pgfqpoint{4.103889in}{5.259390in}}%
\pgfpathlineto{\pgfqpoint{4.110208in}{5.260479in}}%
\pgfpathlineto{\pgfqpoint{4.112315in}{5.258971in}}%
\pgfpathlineto{\pgfqpoint{4.114421in}{5.261736in}}%
\pgfpathlineto{\pgfqpoint{4.114842in}{5.261275in}}%
\pgfpathlineto{\pgfqpoint{4.115263in}{5.262197in}}%
\pgfpathlineto{\pgfqpoint{4.127059in}{5.281640in}}%
\pgfpathlineto{\pgfqpoint{4.130008in}{5.278371in}}%
\pgfpathlineto{\pgfqpoint{4.132536in}{5.275480in}}%
\pgfpathlineto{\pgfqpoint{4.133378in}{5.274642in}}%
\pgfpathlineto{\pgfqpoint{4.135906in}{5.271709in}}%
\pgfpathlineto{\pgfqpoint{4.136327in}{5.273511in}}%
\pgfpathlineto{\pgfqpoint{4.136749in}{5.271835in}}%
\pgfpathlineto{\pgfqpoint{4.145174in}{5.247741in}}%
\pgfpathlineto{\pgfqpoint{4.146017in}{5.245855in}}%
\pgfpathlineto{\pgfqpoint{4.150230in}{5.232279in}}%
\pgfpathlineto{\pgfqpoint{4.151493in}{5.231902in}}%
\pgfpathlineto{\pgfqpoint{4.156127in}{5.229681in}}%
\pgfpathlineto{\pgfqpoint{4.156970in}{5.230687in}}%
\pgfpathlineto{\pgfqpoint{4.157391in}{5.230268in}}%
\pgfpathlineto{\pgfqpoint{4.161604in}{5.227167in}}%
\pgfpathlineto{\pgfqpoint{4.162025in}{5.228089in}}%
\pgfpathlineto{\pgfqpoint{4.165396in}{5.229723in}}%
\pgfpathlineto{\pgfqpoint{4.167081in}{5.232153in}}%
\pgfpathlineto{\pgfqpoint{4.171715in}{5.230477in}}%
\pgfpathlineto{\pgfqpoint{4.173400in}{5.231148in}}%
\pgfpathlineto{\pgfqpoint{4.173821in}{5.230561in}}%
\pgfpathlineto{\pgfqpoint{4.174242in}{5.231567in}}%
\pgfpathlineto{\pgfqpoint{4.174664in}{5.231860in}}%
\pgfpathlineto{\pgfqpoint{4.176770in}{5.228131in}}%
\pgfpathlineto{\pgfqpoint{4.177191in}{5.228969in}}%
\pgfpathlineto{\pgfqpoint{4.177613in}{5.227921in}}%
\pgfpathlineto{\pgfqpoint{4.178455in}{5.228717in}}%
\pgfpathlineto{\pgfqpoint{4.180562in}{5.223982in}}%
\pgfpathlineto{\pgfqpoint{4.184353in}{5.224234in}}%
\pgfpathlineto{\pgfqpoint{4.185617in}{5.224569in}}%
\pgfpathlineto{\pgfqpoint{4.186460in}{5.222851in}}%
\pgfpathlineto{\pgfqpoint{4.186881in}{5.223815in}}%
\pgfpathlineto{\pgfqpoint{4.187302in}{5.224234in}}%
\pgfpathlineto{\pgfqpoint{4.187723in}{5.222767in}}%
\pgfpathlineto{\pgfqpoint{4.188566in}{5.223605in}}%
\pgfpathlineto{\pgfqpoint{4.188987in}{5.223186in}}%
\pgfpathlineto{\pgfqpoint{4.189408in}{5.224192in}}%
\pgfpathlineto{\pgfqpoint{4.190251in}{5.223773in}}%
\pgfpathlineto{\pgfqpoint{4.191936in}{5.223061in}}%
\pgfpathlineto{\pgfqpoint{4.193621in}{5.225491in}}%
\pgfpathlineto{\pgfqpoint{4.196149in}{5.227418in}}%
\pgfpathlineto{\pgfqpoint{4.204153in}{5.198757in}}%
\pgfpathlineto{\pgfqpoint{4.205838in}{5.195992in}}%
\pgfpathlineto{\pgfqpoint{4.206681in}{5.195280in}}%
\pgfpathlineto{\pgfqpoint{4.211736in}{5.190126in}}%
\pgfpathlineto{\pgfqpoint{4.215528in}{5.191886in}}%
\pgfpathlineto{\pgfqpoint{4.218055in}{5.193981in}}%
\pgfpathlineto{\pgfqpoint{4.223111in}{5.180195in}}%
\pgfpathlineto{\pgfqpoint{4.224375in}{5.181242in}}%
\pgfpathlineto{\pgfqpoint{4.227323in}{5.183002in}}%
\pgfpathlineto{\pgfqpoint{4.229009in}{5.182583in}}%
\pgfpathlineto{\pgfqpoint{4.231115in}{5.184259in}}%
\pgfpathlineto{\pgfqpoint{4.231536in}{5.183631in}}%
\pgfpathlineto{\pgfqpoint{4.231958in}{5.184553in}}%
\pgfpathlineto{\pgfqpoint{4.233643in}{5.185391in}}%
\pgfpathlineto{\pgfqpoint{4.234485in}{5.185475in}}%
\pgfpathlineto{\pgfqpoint{4.237434in}{5.186773in}}%
\pgfpathlineto{\pgfqpoint{4.239119in}{5.185642in}}%
\pgfpathlineto{\pgfqpoint{4.240804in}{5.186229in}}%
\pgfpathlineto{\pgfqpoint{4.243753in}{5.191047in}}%
\pgfpathlineto{\pgfqpoint{4.244175in}{5.190754in}}%
\pgfpathlineto{\pgfqpoint{4.244596in}{5.191676in}}%
\pgfpathlineto{\pgfqpoint{4.246281in}{5.195238in}}%
\pgfpathlineto{\pgfqpoint{4.247124in}{5.195866in}}%
\pgfpathlineto{\pgfqpoint{4.256392in}{5.214555in}}%
\pgfpathlineto{\pgfqpoint{4.257234in}{5.215560in}}%
\pgfpathlineto{\pgfqpoint{4.258919in}{5.218409in}}%
\pgfpathlineto{\pgfqpoint{4.263132in}{5.221510in}}%
\pgfpathlineto{\pgfqpoint{4.266502in}{5.226790in}}%
\pgfpathlineto{\pgfqpoint{4.269030in}{5.225323in}}%
\pgfpathlineto{\pgfqpoint{4.277456in}{5.203325in}}%
\pgfpathlineto{\pgfqpoint{4.278719in}{5.198799in}}%
\pgfpathlineto{\pgfqpoint{4.281668in}{5.192640in}}%
\pgfpathlineto{\pgfqpoint{4.284196in}{5.189120in}}%
\pgfpathlineto{\pgfqpoint{4.284617in}{5.189581in}}%
\pgfpathlineto{\pgfqpoint{4.285881in}{5.189371in}}%
\pgfpathlineto{\pgfqpoint{4.286724in}{5.189288in}}%
\pgfpathlineto{\pgfqpoint{4.287988in}{5.188324in}}%
\pgfpathlineto{\pgfqpoint{4.290094in}{5.189623in}}%
\pgfpathlineto{\pgfqpoint{4.290936in}{5.189246in}}%
\pgfpathlineto{\pgfqpoint{4.291358in}{5.189665in}}%
\pgfpathlineto{\pgfqpoint{4.293043in}{5.188659in}}%
\pgfpathlineto{\pgfqpoint{4.294728in}{5.189330in}}%
\pgfpathlineto{\pgfqpoint{4.296834in}{5.188869in}}%
\pgfpathlineto{\pgfqpoint{4.300205in}{5.197081in}}%
\pgfpathlineto{\pgfqpoint{4.301047in}{5.198255in}}%
\pgfpathlineto{\pgfqpoint{4.305260in}{5.206342in}}%
\pgfpathlineto{\pgfqpoint{4.306102in}{5.207808in}}%
\pgfpathlineto{\pgfqpoint{4.311579in}{5.216943in}}%
\pgfpathlineto{\pgfqpoint{4.313264in}{5.216147in}}%
\pgfpathlineto{\pgfqpoint{4.314949in}{5.218619in}}%
\pgfpathlineto{\pgfqpoint{4.319583in}{5.226455in}}%
\pgfpathlineto{\pgfqpoint{4.322532in}{5.229053in}}%
\pgfpathlineto{\pgfqpoint{4.323375in}{5.228634in}}%
\pgfpathlineto{\pgfqpoint{4.330958in}{5.214387in}}%
\pgfpathlineto{\pgfqpoint{4.333064in}{5.208688in}}%
\pgfpathlineto{\pgfqpoint{4.339805in}{5.198422in}}%
\pgfpathlineto{\pgfqpoint{4.341911in}{5.199637in}}%
\pgfpathlineto{\pgfqpoint{4.345281in}{5.197165in}}%
\pgfpathlineto{\pgfqpoint{4.347809in}{5.198548in}}%
\pgfpathlineto{\pgfqpoint{4.349915in}{5.196830in}}%
\pgfpathlineto{\pgfqpoint{4.352022in}{5.195447in}}%
\pgfpathlineto{\pgfqpoint{4.352443in}{5.195950in}}%
\pgfpathlineto{\pgfqpoint{4.356235in}{5.197500in}}%
\pgfpathlineto{\pgfqpoint{4.358341in}{5.201313in}}%
\pgfpathlineto{\pgfqpoint{4.361290in}{5.203450in}}%
\pgfpathlineto{\pgfqpoint{4.362554in}{5.205336in}}%
\pgfpathlineto{\pgfqpoint{4.362975in}{5.205168in}}%
\pgfpathlineto{\pgfqpoint{4.368030in}{5.209191in}}%
\pgfpathlineto{\pgfqpoint{4.374771in}{5.211998in}}%
\pgfpathlineto{\pgfqpoint{4.376877in}{5.214429in}}%
\pgfpathlineto{\pgfqpoint{4.377298in}{5.213842in}}%
\pgfpathlineto{\pgfqpoint{4.377720in}{5.214177in}}%
\pgfpathlineto{\pgfqpoint{4.378141in}{5.214638in}}%
\pgfpathlineto{\pgfqpoint{4.378562in}{5.213507in}}%
\pgfpathlineto{\pgfqpoint{4.379405in}{5.213926in}}%
\pgfpathlineto{\pgfqpoint{4.379826in}{5.212292in}}%
\pgfpathlineto{\pgfqpoint{4.380247in}{5.212543in}}%
\pgfpathlineto{\pgfqpoint{4.381933in}{5.214471in}}%
\pgfpathlineto{\pgfqpoint{4.382775in}{5.213800in}}%
\pgfpathlineto{\pgfqpoint{4.383196in}{5.214387in}}%
\pgfpathlineto{\pgfqpoint{4.383618in}{5.213842in}}%
\pgfpathlineto{\pgfqpoint{4.384039in}{5.214261in}}%
\pgfpathlineto{\pgfqpoint{4.384460in}{5.214638in}}%
\pgfpathlineto{\pgfqpoint{4.384882in}{5.213800in}}%
\pgfpathlineto{\pgfqpoint{4.386988in}{5.212878in}}%
\pgfpathlineto{\pgfqpoint{4.387409in}{5.211663in}}%
\pgfpathlineto{\pgfqpoint{4.387830in}{5.212250in}}%
\pgfpathlineto{\pgfqpoint{4.395835in}{5.218116in}}%
\pgfpathlineto{\pgfqpoint{4.396677in}{5.218745in}}%
\pgfpathlineto{\pgfqpoint{4.399626in}{5.223186in}}%
\pgfpathlineto{\pgfqpoint{4.400048in}{5.222809in}}%
\pgfpathlineto{\pgfqpoint{4.400469in}{5.223354in}}%
\pgfpathlineto{\pgfqpoint{4.404682in}{5.225365in}}%
\pgfpathlineto{\pgfqpoint{4.405103in}{5.224276in}}%
\pgfpathlineto{\pgfqpoint{4.405524in}{5.224737in}}%
\pgfpathlineto{\pgfqpoint{4.405945in}{5.225281in}}%
\pgfpathlineto{\pgfqpoint{4.406367in}{5.224360in}}%
\pgfpathlineto{\pgfqpoint{4.408052in}{5.223061in}}%
\pgfpathlineto{\pgfqpoint{4.408473in}{5.223731in}}%
\pgfpathlineto{\pgfqpoint{4.409737in}{5.225240in}}%
\pgfpathlineto{\pgfqpoint{4.410158in}{5.224443in}}%
\pgfpathlineto{\pgfqpoint{4.411422in}{5.223396in}}%
\pgfpathlineto{\pgfqpoint{4.411843in}{5.224569in}}%
\pgfpathlineto{\pgfqpoint{4.413528in}{5.226538in}}%
\pgfpathlineto{\pgfqpoint{4.416477in}{5.224150in}}%
\pgfpathlineto{\pgfqpoint{4.420690in}{5.231064in}}%
\pgfpathlineto{\pgfqpoint{4.422797in}{5.231357in}}%
\pgfpathlineto{\pgfqpoint{4.427852in}{5.227083in}}%
\pgfpathlineto{\pgfqpoint{4.432486in}{5.220714in}}%
\pgfpathlineto{\pgfqpoint{4.434592in}{5.223103in}}%
\pgfpathlineto{\pgfqpoint{4.436277in}{5.221636in}}%
\pgfpathlineto{\pgfqpoint{4.437963in}{5.223563in}}%
\pgfpathlineto{\pgfqpoint{4.439648in}{5.224318in}}%
\pgfpathlineto{\pgfqpoint{4.440490in}{5.224904in}}%
\pgfpathlineto{\pgfqpoint{4.442175in}{5.226580in}}%
\pgfpathlineto{\pgfqpoint{4.443018in}{5.225281in}}%
\pgfpathlineto{\pgfqpoint{4.443439in}{5.225994in}}%
\pgfpathlineto{\pgfqpoint{4.446809in}{5.224108in}}%
\pgfpathlineto{\pgfqpoint{4.448495in}{5.226203in}}%
\pgfpathlineto{\pgfqpoint{4.450180in}{5.225072in}}%
\pgfpathlineto{\pgfqpoint{4.451865in}{5.226664in}}%
\pgfpathlineto{\pgfqpoint{4.452707in}{5.226497in}}%
\pgfpathlineto{\pgfqpoint{4.453129in}{5.225868in}}%
\pgfpathlineto{\pgfqpoint{4.453971in}{5.226538in}}%
\pgfpathlineto{\pgfqpoint{4.456078in}{5.229053in}}%
\pgfpathlineto{\pgfqpoint{4.456499in}{5.228508in}}%
\pgfpathlineto{\pgfqpoint{4.456920in}{5.228885in}}%
\pgfpathlineto{\pgfqpoint{4.459448in}{5.231315in}}%
\pgfpathlineto{\pgfqpoint{4.459869in}{5.231525in}}%
\pgfpathlineto{\pgfqpoint{4.460290in}{5.230268in}}%
\pgfpathlineto{\pgfqpoint{4.460712in}{5.230603in}}%
\pgfpathlineto{\pgfqpoint{4.462397in}{5.232530in}}%
\pgfpathlineto{\pgfqpoint{4.464082in}{5.231190in}}%
\pgfpathlineto{\pgfqpoint{4.464924in}{5.233201in}}%
\pgfpathlineto{\pgfqpoint{4.465346in}{5.232153in}}%
\pgfpathlineto{\pgfqpoint{4.467873in}{5.232028in}}%
\pgfpathlineto{\pgfqpoint{4.468716in}{5.234500in}}%
\pgfpathlineto{\pgfqpoint{4.469558in}{5.234081in}}%
\pgfpathlineto{\pgfqpoint{4.469980in}{5.234290in}}%
\pgfpathlineto{\pgfqpoint{4.470822in}{5.233201in}}%
\pgfpathlineto{\pgfqpoint{4.472507in}{5.234835in}}%
\pgfpathlineto{\pgfqpoint{4.474614in}{5.234500in}}%
\pgfpathlineto{\pgfqpoint{4.476299in}{5.236385in}}%
\pgfpathlineto{\pgfqpoint{4.479248in}{5.238020in}}%
\pgfpathlineto{\pgfqpoint{4.481354in}{5.238983in}}%
\pgfpathlineto{\pgfqpoint{4.481775in}{5.237852in}}%
\pgfpathlineto{\pgfqpoint{4.482197in}{5.239151in}}%
\pgfpathlineto{\pgfqpoint{4.483882in}{5.240450in}}%
\pgfpathlineto{\pgfqpoint{4.484724in}{5.239780in}}%
\pgfpathlineto{\pgfqpoint{4.485146in}{5.240282in}}%
\pgfpathlineto{\pgfqpoint{4.485567in}{5.239654in}}%
\pgfpathlineto{\pgfqpoint{4.486410in}{5.240324in}}%
\pgfpathlineto{\pgfqpoint{4.488516in}{5.238690in}}%
\pgfpathlineto{\pgfqpoint{4.489780in}{5.239989in}}%
\pgfpathlineto{\pgfqpoint{4.491886in}{5.237265in}}%
\pgfpathlineto{\pgfqpoint{4.492307in}{5.238648in}}%
\pgfpathlineto{\pgfqpoint{4.493150in}{5.237810in}}%
\pgfpathlineto{\pgfqpoint{4.497363in}{5.232908in}}%
\pgfpathlineto{\pgfqpoint{4.498205in}{5.235631in}}%
\pgfpathlineto{\pgfqpoint{4.498627in}{5.235422in}}%
\pgfpathlineto{\pgfqpoint{4.500733in}{5.233369in}}%
\pgfpathlineto{\pgfqpoint{4.501154in}{5.234332in}}%
\pgfpathlineto{\pgfqpoint{4.501576in}{5.234793in}}%
\pgfpathlineto{\pgfqpoint{4.501997in}{5.233662in}}%
\pgfpathlineto{\pgfqpoint{4.503261in}{5.231692in}}%
\pgfpathlineto{\pgfqpoint{4.504946in}{5.233578in}}%
\pgfpathlineto{\pgfqpoint{4.506631in}{5.228382in}}%
\pgfpathlineto{\pgfqpoint{4.507895in}{5.230729in}}%
\pgfpathlineto{\pgfqpoint{4.508316in}{5.230142in}}%
\pgfpathlineto{\pgfqpoint{4.509159in}{5.228592in}}%
\pgfpathlineto{\pgfqpoint{4.509580in}{5.229220in}}%
\pgfpathlineto{\pgfqpoint{4.510844in}{5.231441in}}%
\pgfpathlineto{\pgfqpoint{4.512107in}{5.228089in}}%
\pgfpathlineto{\pgfqpoint{4.512529in}{5.228340in}}%
\pgfpathlineto{\pgfqpoint{4.513371in}{5.229011in}}%
\pgfpathlineto{\pgfqpoint{4.513793in}{5.230016in}}%
\pgfpathlineto{\pgfqpoint{4.514214in}{5.229053in}}%
\pgfpathlineto{\pgfqpoint{4.515478in}{5.227963in}}%
\pgfpathlineto{\pgfqpoint{4.515899in}{5.228550in}}%
\pgfpathlineto{\pgfqpoint{4.517163in}{5.228466in}}%
\pgfpathlineto{\pgfqpoint{4.518005in}{5.226916in}}%
\pgfpathlineto{\pgfqpoint{4.518848in}{5.227837in}}%
\pgfpathlineto{\pgfqpoint{4.519691in}{5.229304in}}%
\pgfpathlineto{\pgfqpoint{4.521376in}{5.231357in}}%
\pgfpathlineto{\pgfqpoint{4.523903in}{5.230142in}}%
\pgfpathlineto{\pgfqpoint{4.526010in}{5.232028in}}%
\pgfpathlineto{\pgfqpoint{4.526431in}{5.232572in}}%
\pgfpathlineto{\pgfqpoint{4.526852in}{5.231902in}}%
\pgfpathlineto{\pgfqpoint{4.528959in}{5.230603in}}%
\pgfpathlineto{\pgfqpoint{4.530644in}{5.225952in}}%
\pgfpathlineto{\pgfqpoint{4.531486in}{5.226622in}}%
\pgfpathlineto{\pgfqpoint{4.531908in}{5.226036in}}%
\pgfpathlineto{\pgfqpoint{4.533593in}{5.226329in}}%
\pgfpathlineto{\pgfqpoint{4.536120in}{5.222139in}}%
\pgfpathlineto{\pgfqpoint{4.536963in}{5.220756in}}%
\pgfpathlineto{\pgfqpoint{4.539491in}{5.217152in}}%
\pgfpathlineto{\pgfqpoint{4.541597in}{5.218996in}}%
\pgfpathlineto{\pgfqpoint{4.542018in}{5.218493in}}%
\pgfpathlineto{\pgfqpoint{4.543703in}{5.220882in}}%
\pgfpathlineto{\pgfqpoint{4.544125in}{5.221049in}}%
\pgfpathlineto{\pgfqpoint{4.544546in}{5.219708in}}%
\pgfpathlineto{\pgfqpoint{4.545388in}{5.220505in}}%
\pgfpathlineto{\pgfqpoint{4.547074in}{5.218912in}}%
\pgfpathlineto{\pgfqpoint{4.549180in}{5.221720in}}%
\pgfpathlineto{\pgfqpoint{4.549601in}{5.221133in}}%
\pgfpathlineto{\pgfqpoint{4.551708in}{5.222097in}}%
\pgfpathlineto{\pgfqpoint{4.552129in}{5.221552in}}%
\pgfpathlineto{\pgfqpoint{4.552550in}{5.222348in}}%
\pgfpathlineto{\pgfqpoint{4.555920in}{5.223689in}}%
\pgfpathlineto{\pgfqpoint{4.557606in}{5.225072in}}%
\pgfpathlineto{\pgfqpoint{4.562661in}{5.227418in}}%
\pgfpathlineto{\pgfqpoint{4.564346in}{5.230771in}}%
\pgfpathlineto{\pgfqpoint{4.565610in}{5.230268in}}%
\pgfpathlineto{\pgfqpoint{4.566874in}{5.229681in}}%
\pgfpathlineto{\pgfqpoint{4.567716in}{5.229262in}}%
\pgfpathlineto{\pgfqpoint{4.568137in}{5.230016in}}%
\pgfpathlineto{\pgfqpoint{4.568980in}{5.230393in}}%
\pgfpathlineto{\pgfqpoint{4.573193in}{5.235925in}}%
\pgfpathlineto{\pgfqpoint{4.573614in}{5.234626in}}%
\pgfpathlineto{\pgfqpoint{4.574457in}{5.233704in}}%
\pgfpathlineto{\pgfqpoint{4.577827in}{5.228843in}}%
\pgfpathlineto{\pgfqpoint{4.579091in}{5.228592in}}%
\pgfpathlineto{\pgfqpoint{4.579512in}{5.228340in}}%
\pgfpathlineto{\pgfqpoint{4.579933in}{5.229094in}}%
\pgfpathlineto{\pgfqpoint{4.580355in}{5.229136in}}%
\pgfpathlineto{\pgfqpoint{4.582882in}{5.224820in}}%
\pgfpathlineto{\pgfqpoint{4.586252in}{5.221217in}}%
\pgfpathlineto{\pgfqpoint{4.592993in}{5.225323in}}%
\pgfpathlineto{\pgfqpoint{4.593414in}{5.224779in}}%
\pgfpathlineto{\pgfqpoint{4.593835in}{5.225156in}}%
\pgfpathlineto{\pgfqpoint{4.594257in}{5.225659in}}%
\pgfpathlineto{\pgfqpoint{4.594678in}{5.224401in}}%
\pgfpathlineto{\pgfqpoint{4.595521in}{5.225281in}}%
\pgfpathlineto{\pgfqpoint{4.601418in}{5.225910in}}%
\pgfpathlineto{\pgfqpoint{4.605210in}{5.226999in}}%
\pgfpathlineto{\pgfqpoint{4.605631in}{5.227670in}}%
\pgfpathlineto{\pgfqpoint{4.606474in}{5.226999in}}%
\pgfpathlineto{\pgfqpoint{4.608580in}{5.226413in}}%
\pgfpathlineto{\pgfqpoint{4.609423in}{5.227167in}}%
\pgfpathlineto{\pgfqpoint{4.610265in}{5.226497in}}%
\pgfpathlineto{\pgfqpoint{4.610687in}{5.227083in}}%
\pgfpathlineto{\pgfqpoint{4.611529in}{5.226874in}}%
\pgfpathlineto{\pgfqpoint{4.612793in}{5.228634in}}%
\pgfpathlineto{\pgfqpoint{4.614899in}{5.227879in}}%
\pgfpathlineto{\pgfqpoint{4.615742in}{5.229346in}}%
\pgfpathlineto{\pgfqpoint{4.616163in}{5.228927in}}%
\pgfpathlineto{\pgfqpoint{4.617848in}{5.230268in}}%
\pgfpathlineto{\pgfqpoint{4.622061in}{5.236888in}}%
\pgfpathlineto{\pgfqpoint{4.623746in}{5.236260in}}%
\pgfpathlineto{\pgfqpoint{4.625010in}{5.237894in}}%
\pgfpathlineto{\pgfqpoint{4.626274in}{5.234877in}}%
\pgfpathlineto{\pgfqpoint{4.626695in}{5.236050in}}%
\pgfpathlineto{\pgfqpoint{4.628380in}{5.239528in}}%
\pgfpathlineto{\pgfqpoint{4.630065in}{5.237517in}}%
\pgfpathlineto{\pgfqpoint{4.631750in}{5.238900in}}%
\pgfpathlineto{\pgfqpoint{4.635542in}{5.236553in}}%
\pgfpathlineto{\pgfqpoint{4.640176in}{5.240073in}}%
\pgfpathlineto{\pgfqpoint{4.641440in}{5.240073in}}%
\pgfpathlineto{\pgfqpoint{4.643546in}{5.238900in}}%
\pgfpathlineto{\pgfqpoint{4.654078in}{5.223563in}}%
\pgfpathlineto{\pgfqpoint{4.655763in}{5.222558in}}%
\pgfpathlineto{\pgfqpoint{4.656185in}{5.222097in}}%
\pgfpathlineto{\pgfqpoint{4.656606in}{5.222809in}}%
\pgfpathlineto{\pgfqpoint{4.658291in}{5.223563in}}%
\pgfpathlineto{\pgfqpoint{4.659976in}{5.222432in}}%
\pgfpathlineto{\pgfqpoint{4.662925in}{5.224904in}}%
\pgfpathlineto{\pgfqpoint{4.664610in}{5.225449in}}%
\pgfpathlineto{\pgfqpoint{4.665031in}{5.224946in}}%
\pgfpathlineto{\pgfqpoint{4.665453in}{5.225742in}}%
\pgfpathlineto{\pgfqpoint{4.667980in}{5.229723in}}%
\pgfpathlineto{\pgfqpoint{4.670087in}{5.232530in}}%
\pgfpathlineto{\pgfqpoint{4.670508in}{5.232447in}}%
\pgfpathlineto{\pgfqpoint{4.673457in}{5.237265in}}%
\pgfpathlineto{\pgfqpoint{4.673878in}{5.236721in}}%
\pgfpathlineto{\pgfqpoint{4.674300in}{5.237140in}}%
\pgfpathlineto{\pgfqpoint{4.675985in}{5.239067in}}%
\pgfpathlineto{\pgfqpoint{4.676406in}{5.237894in}}%
\pgfpathlineto{\pgfqpoint{4.677249in}{5.237056in}}%
\pgfpathlineto{\pgfqpoint{4.681040in}{5.223144in}}%
\pgfpathlineto{\pgfqpoint{4.681883in}{5.221259in}}%
\pgfpathlineto{\pgfqpoint{4.682304in}{5.221971in}}%
\pgfpathlineto{\pgfqpoint{4.683146in}{5.221385in}}%
\pgfpathlineto{\pgfqpoint{4.683568in}{5.221678in}}%
\pgfpathlineto{\pgfqpoint{4.688623in}{5.223605in}}%
\pgfpathlineto{\pgfqpoint{4.689044in}{5.223061in}}%
\pgfpathlineto{\pgfqpoint{4.689466in}{5.223605in}}%
\pgfpathlineto{\pgfqpoint{4.691151in}{5.225910in}}%
\pgfpathlineto{\pgfqpoint{4.691572in}{5.225198in}}%
\pgfpathlineto{\pgfqpoint{4.691993in}{5.226078in}}%
\pgfpathlineto{\pgfqpoint{4.693678in}{5.227712in}}%
\pgfpathlineto{\pgfqpoint{4.695363in}{5.226832in}}%
\pgfpathlineto{\pgfqpoint{4.698734in}{5.230226in}}%
\pgfpathlineto{\pgfqpoint{4.699998in}{5.230687in}}%
\pgfpathlineto{\pgfqpoint{4.702946in}{5.230058in}}%
\pgfpathlineto{\pgfqpoint{4.705895in}{5.237810in}}%
\pgfpathlineto{\pgfqpoint{4.708844in}{5.246484in}}%
\pgfpathlineto{\pgfqpoint{4.710529in}{5.245269in}}%
\pgfpathlineto{\pgfqpoint{4.711372in}{5.243634in}}%
\pgfpathlineto{\pgfqpoint{4.714321in}{5.238648in}}%
\pgfpathlineto{\pgfqpoint{4.716427in}{5.239528in}}%
\pgfpathlineto{\pgfqpoint{4.718534in}{5.238313in}}%
\pgfpathlineto{\pgfqpoint{4.720219in}{5.237852in}}%
\pgfpathlineto{\pgfqpoint{4.724432in}{5.233452in}}%
\pgfpathlineto{\pgfqpoint{4.724853in}{5.234374in}}%
\pgfpathlineto{\pgfqpoint{4.725274in}{5.234081in}}%
\pgfpathlineto{\pgfqpoint{4.726538in}{5.232740in}}%
\pgfpathlineto{\pgfqpoint{4.727381in}{5.234248in}}%
\pgfpathlineto{\pgfqpoint{4.727802in}{5.233704in}}%
\pgfpathlineto{\pgfqpoint{4.730751in}{5.232279in}}%
\pgfpathlineto{\pgfqpoint{4.738755in}{5.234961in}}%
\pgfpathlineto{\pgfqpoint{4.741283in}{5.235925in}}%
\pgfpathlineto{\pgfqpoint{4.742125in}{5.236553in}}%
\pgfpathlineto{\pgfqpoint{4.743810in}{5.237936in}}%
\pgfpathlineto{\pgfqpoint{4.744653in}{5.238103in}}%
\pgfpathlineto{\pgfqpoint{4.746338in}{5.239360in}}%
\pgfpathlineto{\pgfqpoint{4.748023in}{5.239067in}}%
\pgfpathlineto{\pgfqpoint{4.748866in}{5.239780in}}%
\pgfpathlineto{\pgfqpoint{4.749287in}{5.239151in}}%
\pgfpathlineto{\pgfqpoint{4.760240in}{5.243048in}}%
\pgfpathlineto{\pgfqpoint{4.761925in}{5.241665in}}%
\pgfpathlineto{\pgfqpoint{4.762768in}{5.242503in}}%
\pgfpathlineto{\pgfqpoint{4.763189in}{5.241791in}}%
\pgfpathlineto{\pgfqpoint{4.764032in}{5.240450in}}%
\pgfpathlineto{\pgfqpoint{4.765717in}{5.238774in}}%
\pgfpathlineto{\pgfqpoint{4.767823in}{5.237768in}}%
\pgfpathlineto{\pgfqpoint{4.770772in}{5.234165in}}%
\pgfpathlineto{\pgfqpoint{4.774985in}{5.234416in}}%
\pgfpathlineto{\pgfqpoint{4.776670in}{5.235128in}}%
\pgfpathlineto{\pgfqpoint{4.777091in}{5.235003in}}%
\pgfpathlineto{\pgfqpoint{4.777934in}{5.236637in}}%
\pgfpathlineto{\pgfqpoint{4.778355in}{5.235799in}}%
\pgfpathlineto{\pgfqpoint{4.785938in}{5.236302in}}%
\pgfpathlineto{\pgfqpoint{4.789309in}{5.223061in}}%
\pgfpathlineto{\pgfqpoint{4.798998in}{5.177723in}}%
\pgfpathlineto{\pgfqpoint{4.801526in}{5.173784in}}%
\pgfpathlineto{\pgfqpoint{4.803211in}{5.173113in}}%
\pgfpathlineto{\pgfqpoint{4.805317in}{5.173910in}}%
\pgfpathlineto{\pgfqpoint{4.807002in}{5.173407in}}%
\pgfpathlineto{\pgfqpoint{4.810372in}{5.178016in}}%
\pgfpathlineto{\pgfqpoint{4.825538in}{5.240827in}}%
\pgfpathlineto{\pgfqpoint{4.826381in}{5.240324in}}%
\pgfpathlineto{\pgfqpoint{4.826802in}{5.241120in}}%
\pgfpathlineto{\pgfqpoint{4.827224in}{5.240534in}}%
\pgfpathlineto{\pgfqpoint{4.828909in}{5.240450in}}%
\pgfpathlineto{\pgfqpoint{4.830172in}{5.240240in}}%
\pgfpathlineto{\pgfqpoint{4.832700in}{5.240785in}}%
\pgfpathlineto{\pgfqpoint{4.835649in}{5.242000in}}%
\pgfpathlineto{\pgfqpoint{4.836070in}{5.241246in}}%
\pgfpathlineto{\pgfqpoint{4.836492in}{5.242000in}}%
\pgfpathlineto{\pgfqpoint{4.836913in}{5.242755in}}%
\pgfpathlineto{\pgfqpoint{4.837334in}{5.241497in}}%
\pgfpathlineto{\pgfqpoint{4.844496in}{5.244263in}}%
\pgfpathlineto{\pgfqpoint{4.857977in}{5.193143in}}%
\pgfpathlineto{\pgfqpoint{4.861768in}{5.189874in}}%
\pgfpathlineto{\pgfqpoint{4.862611in}{5.190587in}}%
\pgfpathlineto{\pgfqpoint{4.864296in}{5.189246in}}%
\pgfpathlineto{\pgfqpoint{4.868088in}{5.188156in}}%
\pgfpathlineto{\pgfqpoint{4.870194in}{5.188491in}}%
\pgfpathlineto{\pgfqpoint{4.873143in}{5.190126in}}%
\pgfpathlineto{\pgfqpoint{4.878619in}{5.197123in}}%
\pgfpathlineto{\pgfqpoint{4.880305in}{5.199135in}}%
\pgfpathlineto{\pgfqpoint{4.883254in}{5.203953in}}%
\pgfpathlineto{\pgfqpoint{4.883675in}{5.203534in}}%
\pgfpathlineto{\pgfqpoint{4.884096in}{5.204205in}}%
\pgfpathlineto{\pgfqpoint{4.887888in}{5.209023in}}%
\pgfpathlineto{\pgfqpoint{4.888730in}{5.209191in}}%
\pgfpathlineto{\pgfqpoint{4.890837in}{5.213633in}}%
\pgfpathlineto{\pgfqpoint{4.891258in}{5.213130in}}%
\pgfpathlineto{\pgfqpoint{4.894628in}{5.215518in}}%
\pgfpathlineto{\pgfqpoint{4.897998in}{5.212292in}}%
\pgfpathlineto{\pgfqpoint{4.898420in}{5.212166in}}%
\pgfpathlineto{\pgfqpoint{4.900105in}{5.209401in}}%
\pgfpathlineto{\pgfqpoint{4.902211in}{5.208814in}}%
\pgfpathlineto{\pgfqpoint{4.904317in}{5.206048in}}%
\pgfpathlineto{\pgfqpoint{4.906003in}{5.206467in}}%
\pgfpathlineto{\pgfqpoint{4.906424in}{5.206048in}}%
\pgfpathlineto{\pgfqpoint{4.906845in}{5.206677in}}%
\pgfpathlineto{\pgfqpoint{4.911900in}{5.214764in}}%
\pgfpathlineto{\pgfqpoint{4.914428in}{5.218661in}}%
\pgfpathlineto{\pgfqpoint{4.917377in}{5.223522in}}%
\pgfpathlineto{\pgfqpoint{4.921169in}{5.225198in}}%
\pgfpathlineto{\pgfqpoint{4.928330in}{5.219625in}}%
\pgfpathlineto{\pgfqpoint{4.928752in}{5.220086in}}%
\pgfpathlineto{\pgfqpoint{4.929173in}{5.219331in}}%
\pgfpathlineto{\pgfqpoint{4.931701in}{5.217571in}}%
\pgfpathlineto{\pgfqpoint{4.932543in}{5.218870in}}%
\pgfpathlineto{\pgfqpoint{4.932964in}{5.218158in}}%
\pgfpathlineto{\pgfqpoint{4.934649in}{5.216189in}}%
\pgfpathlineto{\pgfqpoint{4.937598in}{5.213423in}}%
\pgfpathlineto{\pgfqpoint{4.938862in}{5.211957in}}%
\pgfpathlineto{\pgfqpoint{4.939284in}{5.212543in}}%
\pgfpathlineto{\pgfqpoint{4.939705in}{5.213297in}}%
\pgfpathlineto{\pgfqpoint{4.940126in}{5.212837in}}%
\pgfpathlineto{\pgfqpoint{4.944760in}{5.206258in}}%
\pgfpathlineto{\pgfqpoint{4.948130in}{5.202612in}}%
\pgfpathlineto{\pgfqpoint{4.951501in}{5.206132in}}%
\pgfpathlineto{\pgfqpoint{4.953607in}{5.209317in}}%
\pgfpathlineto{\pgfqpoint{4.954450in}{5.210155in}}%
\pgfpathlineto{\pgfqpoint{4.958662in}{5.216231in}}%
\pgfpathlineto{\pgfqpoint{4.959084in}{5.216356in}}%
\pgfpathlineto{\pgfqpoint{4.961190in}{5.220086in}}%
\pgfpathlineto{\pgfqpoint{4.961611in}{5.219625in}}%
\pgfpathlineto{\pgfqpoint{4.962033in}{5.220253in}}%
\pgfpathlineto{\pgfqpoint{4.963718in}{5.220714in}}%
\pgfpathlineto{\pgfqpoint{4.970458in}{5.221887in}}%
\pgfpathlineto{\pgfqpoint{4.972143in}{5.223103in}}%
\pgfpathlineto{\pgfqpoint{4.974250in}{5.222516in}}%
\pgfpathlineto{\pgfqpoint{4.975092in}{5.223731in}}%
\pgfpathlineto{\pgfqpoint{4.975513in}{5.222809in}}%
\pgfpathlineto{\pgfqpoint{4.977199in}{5.223731in}}%
\pgfpathlineto{\pgfqpoint{4.978884in}{5.224695in}}%
\pgfpathlineto{\pgfqpoint{4.980569in}{5.223061in}}%
\pgfpathlineto{\pgfqpoint{4.981411in}{5.224401in}}%
\pgfpathlineto{\pgfqpoint{4.981833in}{5.223480in}}%
\pgfpathlineto{\pgfqpoint{4.984782in}{5.222851in}}%
\pgfpathlineto{\pgfqpoint{4.985203in}{5.223605in}}%
\pgfpathlineto{\pgfqpoint{4.985624in}{5.222055in}}%
\pgfpathlineto{\pgfqpoint{4.986467in}{5.222474in}}%
\pgfpathlineto{\pgfqpoint{4.986888in}{5.222139in}}%
\pgfpathlineto{\pgfqpoint{4.987309in}{5.222935in}}%
\pgfpathlineto{\pgfqpoint{4.987731in}{5.223605in}}%
\pgfpathlineto{\pgfqpoint{4.988573in}{5.223019in}}%
\pgfpathlineto{\pgfqpoint{4.991101in}{5.223563in}}%
\pgfpathlineto{\pgfqpoint{4.991522in}{5.223857in}}%
\pgfpathlineto{\pgfqpoint{4.991943in}{5.222809in}}%
\pgfpathlineto{\pgfqpoint{4.993207in}{5.222558in}}%
\pgfpathlineto{\pgfqpoint{4.994050in}{5.225407in}}%
\pgfpathlineto{\pgfqpoint{4.994892in}{5.225030in}}%
\pgfpathlineto{\pgfqpoint{4.996999in}{5.225198in}}%
\pgfpathlineto{\pgfqpoint{4.997841in}{5.226748in}}%
\pgfpathlineto{\pgfqpoint{4.998262in}{5.226287in}}%
\pgfpathlineto{\pgfqpoint{5.000790in}{5.226245in}}%
\pgfpathlineto{\pgfqpoint{5.001633in}{5.227251in}}%
\pgfpathlineto{\pgfqpoint{5.002054in}{5.225784in}}%
\pgfpathlineto{\pgfqpoint{5.002475in}{5.226329in}}%
\pgfpathlineto{\pgfqpoint{5.004160in}{5.228717in}}%
\pgfpathlineto{\pgfqpoint{5.004582in}{5.227754in}}%
\pgfpathlineto{\pgfqpoint{5.005424in}{5.228298in}}%
\pgfpathlineto{\pgfqpoint{5.005845in}{5.227670in}}%
\pgfpathlineto{\pgfqpoint{5.006267in}{5.228340in}}%
\pgfpathlineto{\pgfqpoint{5.007531in}{5.228466in}}%
\pgfpathlineto{\pgfqpoint{5.012586in}{5.223480in}}%
\pgfpathlineto{\pgfqpoint{5.013428in}{5.224318in}}%
\pgfpathlineto{\pgfqpoint{5.014271in}{5.225198in}}%
\pgfpathlineto{\pgfqpoint{5.016799in}{5.222139in}}%
\pgfpathlineto{\pgfqpoint{5.018063in}{5.220882in}}%
\pgfpathlineto{\pgfqpoint{5.019748in}{5.219415in}}%
\pgfpathlineto{\pgfqpoint{5.020169in}{5.220086in}}%
\pgfpathlineto{\pgfqpoint{5.022275in}{5.219834in}}%
\pgfpathlineto{\pgfqpoint{5.023960in}{5.222013in}}%
\pgfpathlineto{\pgfqpoint{5.024382in}{5.221133in}}%
\pgfpathlineto{\pgfqpoint{5.024803in}{5.221468in}}%
\pgfpathlineto{\pgfqpoint{5.026909in}{5.222683in}}%
\pgfpathlineto{\pgfqpoint{5.028173in}{5.222223in}}%
\pgfpathlineto{\pgfqpoint{5.029858in}{5.223186in}}%
\pgfpathlineto{\pgfqpoint{5.030280in}{5.223480in}}%
\pgfpathlineto{\pgfqpoint{5.031965in}{5.221678in}}%
\pgfpathlineto{\pgfqpoint{5.033650in}{5.221468in}}%
\pgfpathlineto{\pgfqpoint{5.038284in}{5.215560in}}%
\pgfpathlineto{\pgfqpoint{5.038705in}{5.216063in}}%
\pgfpathlineto{\pgfqpoint{5.040390in}{5.217613in}}%
\pgfpathlineto{\pgfqpoint{5.041233in}{5.216398in}}%
\pgfpathlineto{\pgfqpoint{5.041654in}{5.216859in}}%
\pgfpathlineto{\pgfqpoint{5.047552in}{5.216272in}}%
\pgfpathlineto{\pgfqpoint{5.048816in}{5.218284in}}%
\pgfpathlineto{\pgfqpoint{5.049237in}{5.217990in}}%
\pgfpathlineto{\pgfqpoint{5.049658in}{5.216817in}}%
\pgfpathlineto{\pgfqpoint{5.050080in}{5.217404in}}%
\pgfpathlineto{\pgfqpoint{5.051765in}{5.221385in}}%
\pgfpathlineto{\pgfqpoint{5.052607in}{5.220714in}}%
\pgfpathlineto{\pgfqpoint{5.055135in}{5.220588in}}%
\pgfpathlineto{\pgfqpoint{5.055978in}{5.219331in}}%
\pgfpathlineto{\pgfqpoint{5.056399in}{5.220044in}}%
\pgfpathlineto{\pgfqpoint{5.058084in}{5.220924in}}%
\pgfpathlineto{\pgfqpoint{5.058927in}{5.220253in}}%
\pgfpathlineto{\pgfqpoint{5.059348in}{5.220882in}}%
\pgfpathlineto{\pgfqpoint{5.060190in}{5.221217in}}%
\pgfpathlineto{\pgfqpoint{5.060612in}{5.221510in}}%
\pgfpathlineto{\pgfqpoint{5.061033in}{5.220588in}}%
\pgfpathlineto{\pgfqpoint{5.062297in}{5.220714in}}%
\pgfpathlineto{\pgfqpoint{5.063139in}{5.223522in}}%
\pgfpathlineto{\pgfqpoint{5.063982in}{5.222851in}}%
\pgfpathlineto{\pgfqpoint{5.066088in}{5.221887in}}%
\pgfpathlineto{\pgfqpoint{5.066510in}{5.222600in}}%
\pgfpathlineto{\pgfqpoint{5.068616in}{5.220379in}}%
\pgfpathlineto{\pgfqpoint{5.069037in}{5.222348in}}%
\pgfpathlineto{\pgfqpoint{5.070722in}{5.225072in}}%
\pgfpathlineto{\pgfqpoint{5.072407in}{5.223982in}}%
\pgfpathlineto{\pgfqpoint{5.074093in}{5.225072in}}%
\pgfpathlineto{\pgfqpoint{5.076199in}{5.224066in}}%
\pgfpathlineto{\pgfqpoint{5.076620in}{5.224820in}}%
\pgfpathlineto{\pgfqpoint{5.077041in}{5.224024in}}%
\pgfpathlineto{\pgfqpoint{5.077884in}{5.224024in}}%
\pgfpathlineto{\pgfqpoint{5.078305in}{5.224485in}}%
\pgfpathlineto{\pgfqpoint{5.078727in}{5.224066in}}%
\pgfpathlineto{\pgfqpoint{5.079990in}{5.223270in}}%
\pgfpathlineto{\pgfqpoint{5.080412in}{5.223773in}}%
\pgfpathlineto{\pgfqpoint{5.080833in}{5.224192in}}%
\pgfpathlineto{\pgfqpoint{5.081254in}{5.223689in}}%
\pgfpathlineto{\pgfqpoint{5.086731in}{5.217362in}}%
\pgfpathlineto{\pgfqpoint{5.088416in}{5.216733in}}%
\pgfpathlineto{\pgfqpoint{5.089259in}{5.215225in}}%
\pgfpathlineto{\pgfqpoint{5.089680in}{5.214848in}}%
\pgfpathlineto{\pgfqpoint{5.090522in}{5.219206in}}%
\pgfpathlineto{\pgfqpoint{5.090944in}{5.218577in}}%
\pgfpathlineto{\pgfqpoint{5.092629in}{5.216608in}}%
\pgfpathlineto{\pgfqpoint{5.096842in}{5.207724in}}%
\pgfpathlineto{\pgfqpoint{5.097684in}{5.207850in}}%
\pgfpathlineto{\pgfqpoint{5.098948in}{5.208269in}}%
\pgfpathlineto{\pgfqpoint{5.105688in}{5.194483in}}%
\pgfpathlineto{\pgfqpoint{5.108637in}{5.192179in}}%
\pgfpathlineto{\pgfqpoint{5.109059in}{5.192891in}}%
\pgfpathlineto{\pgfqpoint{5.110322in}{5.194819in}}%
\pgfpathlineto{\pgfqpoint{5.112008in}{5.194735in}}%
\pgfpathlineto{\pgfqpoint{5.112429in}{5.193520in}}%
\pgfpathlineto{\pgfqpoint{5.112850in}{5.194735in}}%
\pgfpathlineto{\pgfqpoint{5.114535in}{5.195028in}}%
\pgfpathlineto{\pgfqpoint{5.114957in}{5.194609in}}%
\pgfpathlineto{\pgfqpoint{5.115378in}{5.195238in}}%
\pgfpathlineto{\pgfqpoint{5.116642in}{5.197333in}}%
\pgfpathlineto{\pgfqpoint{5.117063in}{5.197249in}}%
\pgfpathlineto{\pgfqpoint{5.118748in}{5.195238in}}%
\pgfpathlineto{\pgfqpoint{5.120854in}{5.196830in}}%
\pgfpathlineto{\pgfqpoint{5.121276in}{5.196243in}}%
\pgfpathlineto{\pgfqpoint{5.121697in}{5.196830in}}%
\pgfpathlineto{\pgfqpoint{5.123382in}{5.198548in}}%
\pgfpathlineto{\pgfqpoint{5.125067in}{5.196998in}}%
\pgfpathlineto{\pgfqpoint{5.125488in}{5.197500in}}%
\pgfpathlineto{\pgfqpoint{5.127174in}{5.199051in}}%
\pgfpathlineto{\pgfqpoint{5.127595in}{5.198297in}}%
\pgfpathlineto{\pgfqpoint{5.128437in}{5.199051in}}%
\pgfpathlineto{\pgfqpoint{5.135599in}{5.207557in}}%
\pgfpathlineto{\pgfqpoint{5.136020in}{5.207515in}}%
\pgfpathlineto{\pgfqpoint{5.138127in}{5.204582in}}%
\pgfpathlineto{\pgfqpoint{5.139812in}{5.204582in}}%
\pgfpathlineto{\pgfqpoint{5.140233in}{5.204205in}}%
\pgfpathlineto{\pgfqpoint{5.140654in}{5.204791in}}%
\pgfpathlineto{\pgfqpoint{5.142340in}{5.204959in}}%
\pgfpathlineto{\pgfqpoint{5.145710in}{5.198380in}}%
\pgfpathlineto{\pgfqpoint{5.149080in}{5.198171in}}%
\pgfpathlineto{\pgfqpoint{5.151186in}{5.201691in}}%
\pgfpathlineto{\pgfqpoint{5.152029in}{5.202068in}}%
\pgfpathlineto{\pgfqpoint{5.154135in}{5.203450in}}%
\pgfpathlineto{\pgfqpoint{5.156663in}{5.203828in}}%
\pgfpathlineto{\pgfqpoint{5.159191in}{5.204456in}}%
\pgfpathlineto{\pgfqpoint{5.160876in}{5.206342in}}%
\pgfpathlineto{\pgfqpoint{5.161297in}{5.205671in}}%
\pgfpathlineto{\pgfqpoint{5.162982in}{5.202822in}}%
\pgfpathlineto{\pgfqpoint{5.165510in}{5.197584in}}%
\pgfpathlineto{\pgfqpoint{5.175621in}{5.184301in}}%
\pgfpathlineto{\pgfqpoint{5.178570in}{5.188533in}}%
\pgfpathlineto{\pgfqpoint{5.178991in}{5.187989in}}%
\pgfpathlineto{\pgfqpoint{5.181097in}{5.193352in}}%
\pgfpathlineto{\pgfqpoint{5.181518in}{5.193059in}}%
\pgfpathlineto{\pgfqpoint{5.184046in}{5.195699in}}%
\pgfpathlineto{\pgfqpoint{5.188259in}{5.207096in}}%
\pgfpathlineto{\pgfqpoint{5.189101in}{5.206593in}}%
\pgfpathlineto{\pgfqpoint{5.191629in}{5.204749in}}%
\pgfpathlineto{\pgfqpoint{5.192050in}{5.205420in}}%
\pgfpathlineto{\pgfqpoint{5.192472in}{5.206467in}}%
\pgfpathlineto{\pgfqpoint{5.192893in}{5.204917in}}%
\pgfpathlineto{\pgfqpoint{5.195421in}{5.202529in}}%
\pgfpathlineto{\pgfqpoint{5.196263in}{5.202822in}}%
\pgfpathlineto{\pgfqpoint{5.201740in}{5.191173in}}%
\pgfpathlineto{\pgfqpoint{5.207638in}{5.198087in}}%
\pgfpathlineto{\pgfqpoint{5.208059in}{5.197458in}}%
\pgfpathlineto{\pgfqpoint{5.209323in}{5.197458in}}%
\pgfpathlineto{\pgfqpoint{5.214799in}{5.211412in}}%
\pgfpathlineto{\pgfqpoint{5.215221in}{5.211119in}}%
\pgfpathlineto{\pgfqpoint{5.218170in}{5.208060in}}%
\pgfpathlineto{\pgfqpoint{5.219433in}{5.207683in}}%
\pgfpathlineto{\pgfqpoint{5.221119in}{5.204624in}}%
\pgfpathlineto{\pgfqpoint{5.222804in}{5.206677in}}%
\pgfpathlineto{\pgfqpoint{5.223225in}{5.205629in}}%
\pgfpathlineto{\pgfqpoint{5.226174in}{5.201062in}}%
\pgfpathlineto{\pgfqpoint{5.227859in}{5.196579in}}%
\pgfpathlineto{\pgfqpoint{5.228280in}{5.196662in}}%
\pgfpathlineto{\pgfqpoint{5.228702in}{5.198003in}}%
\pgfpathlineto{\pgfqpoint{5.229544in}{5.197039in}}%
\pgfpathlineto{\pgfqpoint{5.231229in}{5.196746in}}%
\pgfpathlineto{\pgfqpoint{5.234599in}{5.198003in}}%
\pgfpathlineto{\pgfqpoint{5.236706in}{5.196285in}}%
\pgfpathlineto{\pgfqpoint{5.238391in}{5.198171in}}%
\pgfpathlineto{\pgfqpoint{5.243868in}{5.205504in}}%
\pgfpathlineto{\pgfqpoint{5.245553in}{5.203786in}}%
\pgfpathlineto{\pgfqpoint{5.247659in}{5.205043in}}%
\pgfpathlineto{\pgfqpoint{5.248502in}{5.206593in}}%
\pgfpathlineto{\pgfqpoint{5.252714in}{5.217823in}}%
\pgfpathlineto{\pgfqpoint{5.253978in}{5.216314in}}%
\pgfpathlineto{\pgfqpoint{5.254821in}{5.213675in}}%
\pgfpathlineto{\pgfqpoint{5.255242in}{5.213842in}}%
\pgfpathlineto{\pgfqpoint{5.256927in}{5.218661in}}%
\pgfpathlineto{\pgfqpoint{5.259034in}{5.225072in}}%
\pgfpathlineto{\pgfqpoint{5.259455in}{5.224485in}}%
\pgfpathlineto{\pgfqpoint{5.261983in}{5.226957in}}%
\pgfpathlineto{\pgfqpoint{5.264089in}{5.229891in}}%
\pgfpathlineto{\pgfqpoint{5.264932in}{5.229136in}}%
\pgfpathlineto{\pgfqpoint{5.267038in}{5.223480in}}%
\pgfpathlineto{\pgfqpoint{5.269144in}{5.225575in}}%
\pgfpathlineto{\pgfqpoint{5.270829in}{5.222558in}}%
\pgfpathlineto{\pgfqpoint{5.271251in}{5.222558in}}%
\pgfpathlineto{\pgfqpoint{5.275042in}{5.229849in}}%
\pgfpathlineto{\pgfqpoint{5.275463in}{5.230603in}}%
\pgfpathlineto{\pgfqpoint{5.275885in}{5.229388in}}%
\pgfpathlineto{\pgfqpoint{5.276727in}{5.229849in}}%
\pgfpathlineto{\pgfqpoint{5.277570in}{5.228508in}}%
\pgfpathlineto{\pgfqpoint{5.277991in}{5.228885in}}%
\pgfpathlineto{\pgfqpoint{5.278412in}{5.228801in}}%
\pgfpathlineto{\pgfqpoint{5.281783in}{5.236804in}}%
\pgfpathlineto{\pgfqpoint{5.282625in}{5.237265in}}%
\pgfpathlineto{\pgfqpoint{5.283889in}{5.237684in}}%
\pgfpathlineto{\pgfqpoint{5.285995in}{5.236176in}}%
\pgfpathlineto{\pgfqpoint{5.288102in}{5.235086in}}%
\pgfpathlineto{\pgfqpoint{5.293157in}{5.217404in}}%
\pgfpathlineto{\pgfqpoint{5.297370in}{5.208688in}}%
\pgfpathlineto{\pgfqpoint{5.299055in}{5.210825in}}%
\pgfpathlineto{\pgfqpoint{5.300319in}{5.209317in}}%
\pgfpathlineto{\pgfqpoint{5.300740in}{5.207557in}}%
\pgfpathlineto{\pgfqpoint{5.301583in}{5.208604in}}%
\pgfpathlineto{\pgfqpoint{5.302425in}{5.209987in}}%
\pgfpathlineto{\pgfqpoint{5.307481in}{5.226916in}}%
\pgfpathlineto{\pgfqpoint{5.311272in}{5.232153in}}%
\pgfpathlineto{\pgfqpoint{5.311693in}{5.232321in}}%
\pgfpathlineto{\pgfqpoint{5.313379in}{5.227335in}}%
\pgfpathlineto{\pgfqpoint{5.314221in}{5.229220in}}%
\pgfpathlineto{\pgfqpoint{5.315064in}{5.228173in}}%
\pgfpathlineto{\pgfqpoint{5.318013in}{5.230603in}}%
\pgfpathlineto{\pgfqpoint{5.318434in}{5.229639in}}%
\pgfpathlineto{\pgfqpoint{5.321383in}{5.226413in}}%
\pgfpathlineto{\pgfqpoint{5.322225in}{5.226287in}}%
\pgfpathlineto{\pgfqpoint{5.324332in}{5.230896in}}%
\pgfpathlineto{\pgfqpoint{5.324753in}{5.230226in}}%
\pgfpathlineto{\pgfqpoint{5.325174in}{5.228592in}}%
\pgfpathlineto{\pgfqpoint{5.325596in}{5.229681in}}%
\pgfpathlineto{\pgfqpoint{5.328123in}{5.233410in}}%
\pgfpathlineto{\pgfqpoint{5.330230in}{5.228675in}}%
\pgfpathlineto{\pgfqpoint{5.331915in}{5.224737in}}%
\pgfpathlineto{\pgfqpoint{5.333600in}{5.223438in}}%
\pgfpathlineto{\pgfqpoint{5.335706in}{5.222348in}}%
\pgfpathlineto{\pgfqpoint{5.339919in}{5.209778in}}%
\pgfpathlineto{\pgfqpoint{5.343711in}{5.209610in}}%
\pgfpathlineto{\pgfqpoint{5.344553in}{5.207850in}}%
\pgfpathlineto{\pgfqpoint{5.345396in}{5.208940in}}%
\pgfpathlineto{\pgfqpoint{5.346238in}{5.210155in}}%
\pgfpathlineto{\pgfqpoint{5.347923in}{5.211747in}}%
\pgfpathlineto{\pgfqpoint{5.348766in}{5.211957in}}%
\pgfpathlineto{\pgfqpoint{5.350451in}{5.213800in}}%
\pgfpathlineto{\pgfqpoint{5.350872in}{5.213297in}}%
\pgfpathlineto{\pgfqpoint{5.351294in}{5.213842in}}%
\pgfpathlineto{\pgfqpoint{5.354242in}{5.216482in}}%
\pgfpathlineto{\pgfqpoint{5.355085in}{5.217236in}}%
\pgfpathlineto{\pgfqpoint{5.359298in}{5.225868in}}%
\pgfpathlineto{\pgfqpoint{5.359719in}{5.225449in}}%
\pgfpathlineto{\pgfqpoint{5.361404in}{5.228592in}}%
\pgfpathlineto{\pgfqpoint{5.363511in}{5.231609in}}%
\pgfpathlineto{\pgfqpoint{5.365196in}{5.233788in}}%
\pgfpathlineto{\pgfqpoint{5.367302in}{5.232530in}}%
\pgfpathlineto{\pgfqpoint{5.368145in}{5.233327in}}%
\pgfpathlineto{\pgfqpoint{5.368566in}{5.232782in}}%
\pgfpathlineto{\pgfqpoint{5.370672in}{5.232908in}}%
\pgfpathlineto{\pgfqpoint{5.371936in}{5.232530in}}%
\pgfpathlineto{\pgfqpoint{5.372357in}{5.231399in}}%
\pgfpathlineto{\pgfqpoint{5.372779in}{5.231734in}}%
\pgfpathlineto{\pgfqpoint{5.374464in}{5.232489in}}%
\pgfpathlineto{\pgfqpoint{5.376149in}{5.229807in}}%
\pgfpathlineto{\pgfqpoint{5.378677in}{5.230226in}}%
\pgfpathlineto{\pgfqpoint{5.379098in}{5.231944in}}%
\pgfpathlineto{\pgfqpoint{5.379940in}{5.231651in}}%
\pgfpathlineto{\pgfqpoint{5.382047in}{5.232866in}}%
\pgfpathlineto{\pgfqpoint{5.384996in}{5.229891in}}%
\pgfpathlineto{\pgfqpoint{5.387102in}{5.227251in}}%
\pgfpathlineto{\pgfqpoint{5.388366in}{5.228047in}}%
\pgfpathlineto{\pgfqpoint{5.392579in}{5.224779in}}%
\pgfpathlineto{\pgfqpoint{5.393000in}{5.225617in}}%
\pgfpathlineto{\pgfqpoint{5.398898in}{5.227251in}}%
\pgfpathlineto{\pgfqpoint{5.400583in}{5.228969in}}%
\pgfpathlineto{\pgfqpoint{5.401426in}{5.227251in}}%
\pgfpathlineto{\pgfqpoint{5.401847in}{5.227418in}}%
\pgfpathlineto{\pgfqpoint{5.402268in}{5.227041in}}%
\pgfpathlineto{\pgfqpoint{5.402689in}{5.227544in}}%
\pgfpathlineto{\pgfqpoint{5.404375in}{5.228717in}}%
\pgfpathlineto{\pgfqpoint{5.406902in}{5.221217in}}%
\pgfpathlineto{\pgfqpoint{5.409009in}{5.213800in}}%
\pgfpathlineto{\pgfqpoint{5.410694in}{5.215434in}}%
\pgfpathlineto{\pgfqpoint{5.411958in}{5.214806in}}%
\pgfpathlineto{\pgfqpoint{5.412379in}{5.213758in}}%
\pgfpathlineto{\pgfqpoint{5.413221in}{5.214638in}}%
\pgfpathlineto{\pgfqpoint{5.415328in}{5.213297in}}%
\pgfpathlineto{\pgfqpoint{5.415749in}{5.214010in}}%
\pgfpathlineto{\pgfqpoint{5.416592in}{5.213507in}}%
\pgfpathlineto{\pgfqpoint{5.417013in}{5.213716in}}%
\pgfpathlineto{\pgfqpoint{5.417434in}{5.212753in}}%
\pgfpathlineto{\pgfqpoint{5.418698in}{5.212669in}}%
\pgfpathlineto{\pgfqpoint{5.423332in}{5.228131in}}%
\pgfpathlineto{\pgfqpoint{5.423753in}{5.227377in}}%
\pgfpathlineto{\pgfqpoint{5.424175in}{5.227712in}}%
\pgfpathlineto{\pgfqpoint{5.425860in}{5.230938in}}%
\pgfpathlineto{\pgfqpoint{5.426281in}{5.229597in}}%
\pgfpathlineto{\pgfqpoint{5.427966in}{5.229681in}}%
\pgfpathlineto{\pgfqpoint{5.428809in}{5.231064in}}%
\pgfpathlineto{\pgfqpoint{5.429230in}{5.230477in}}%
\pgfpathlineto{\pgfqpoint{5.430073in}{5.229807in}}%
\pgfpathlineto{\pgfqpoint{5.430915in}{5.228508in}}%
\pgfpathlineto{\pgfqpoint{5.431336in}{5.228801in}}%
\pgfpathlineto{\pgfqpoint{5.432600in}{5.228508in}}%
\pgfpathlineto{\pgfqpoint{5.438077in}{5.222013in}}%
\pgfpathlineto{\pgfqpoint{5.438919in}{5.220840in}}%
\pgfpathlineto{\pgfqpoint{5.442290in}{5.217655in}}%
\pgfpathlineto{\pgfqpoint{5.445660in}{5.219918in}}%
\pgfpathlineto{\pgfqpoint{5.447345in}{5.218912in}}%
\pgfpathlineto{\pgfqpoint{5.451979in}{5.220798in}}%
\pgfpathlineto{\pgfqpoint{5.452822in}{5.220798in}}%
\pgfpathlineto{\pgfqpoint{5.454507in}{5.222600in}}%
\pgfpathlineto{\pgfqpoint{5.454928in}{5.222264in}}%
\pgfpathlineto{\pgfqpoint{5.455349in}{5.223228in}}%
\pgfpathlineto{\pgfqpoint{5.455771in}{5.223689in}}%
\pgfpathlineto{\pgfqpoint{5.470094in}{5.185391in}}%
\pgfpathlineto{\pgfqpoint{5.481890in}{5.206551in}}%
\pgfpathlineto{\pgfqpoint{5.485681in}{5.210322in}}%
\pgfpathlineto{\pgfqpoint{5.487366in}{5.208856in}}%
\pgfpathlineto{\pgfqpoint{5.489473in}{5.210825in}}%
\pgfpathlineto{\pgfqpoint{5.491158in}{5.210197in}}%
\pgfpathlineto{\pgfqpoint{5.493264in}{5.212166in}}%
\pgfpathlineto{\pgfqpoint{5.493686in}{5.211873in}}%
\pgfpathlineto{\pgfqpoint{5.494107in}{5.212501in}}%
\pgfpathlineto{\pgfqpoint{5.494528in}{5.213381in}}%
\pgfpathlineto{\pgfqpoint{5.495371in}{5.212920in}}%
\pgfpathlineto{\pgfqpoint{5.499162in}{5.213465in}}%
\pgfpathlineto{\pgfqpoint{5.500847in}{5.215057in}}%
\pgfpathlineto{\pgfqpoint{5.502532in}{5.213633in}}%
\pgfpathlineto{\pgfqpoint{5.505481in}{5.214890in}}%
\pgfpathlineto{\pgfqpoint{5.508852in}{5.215560in}}%
\pgfpathlineto{\pgfqpoint{5.510958in}{5.218326in}}%
\pgfpathlineto{\pgfqpoint{5.511379in}{5.217446in}}%
\pgfpathlineto{\pgfqpoint{5.511801in}{5.218619in}}%
\pgfpathlineto{\pgfqpoint{5.513486in}{5.219708in}}%
\pgfpathlineto{\pgfqpoint{5.513907in}{5.219289in}}%
\pgfpathlineto{\pgfqpoint{5.514328in}{5.220127in}}%
\pgfpathlineto{\pgfqpoint{5.516013in}{5.221636in}}%
\pgfpathlineto{\pgfqpoint{5.516435in}{5.221091in}}%
\pgfpathlineto{\pgfqpoint{5.516856in}{5.221594in}}%
\pgfpathlineto{\pgfqpoint{5.523596in}{5.237433in}}%
\pgfpathlineto{\pgfqpoint{5.526545in}{5.230184in}}%
\pgfpathlineto{\pgfqpoint{5.527388in}{5.228131in}}%
\pgfpathlineto{\pgfqpoint{5.530337in}{5.218074in}}%
\pgfpathlineto{\pgfqpoint{5.530758in}{5.217823in}}%
\pgfpathlineto{\pgfqpoint{5.532443in}{5.212040in}}%
\pgfpathlineto{\pgfqpoint{5.532864in}{5.212501in}}%
\pgfpathlineto{\pgfqpoint{5.534971in}{5.210322in}}%
\pgfpathlineto{\pgfqpoint{5.535813in}{5.211538in}}%
\pgfpathlineto{\pgfqpoint{5.538762in}{5.205965in}}%
\pgfpathlineto{\pgfqpoint{5.539605in}{5.206593in}}%
\pgfpathlineto{\pgfqpoint{5.540026in}{5.206048in}}%
\pgfpathlineto{\pgfqpoint{5.541290in}{5.205085in}}%
\pgfpathlineto{\pgfqpoint{5.544660in}{5.207557in}}%
\pgfpathlineto{\pgfqpoint{5.546345in}{5.207222in}}%
\pgfpathlineto{\pgfqpoint{5.548452in}{5.208940in}}%
\pgfpathlineto{\pgfqpoint{5.548873in}{5.207222in}}%
\pgfpathlineto{\pgfqpoint{5.549294in}{5.207934in}}%
\pgfpathlineto{\pgfqpoint{5.550979in}{5.210658in}}%
\pgfpathlineto{\pgfqpoint{5.551822in}{5.211244in}}%
\pgfpathlineto{\pgfqpoint{5.553086in}{5.216733in}}%
\pgfpathlineto{\pgfqpoint{5.555192in}{5.226538in}}%
\pgfpathlineto{\pgfqpoint{5.555613in}{5.226916in}}%
\pgfpathlineto{\pgfqpoint{5.557299in}{5.235925in}}%
\pgfpathlineto{\pgfqpoint{5.558141in}{5.235003in}}%
\pgfpathlineto{\pgfqpoint{5.558562in}{5.235338in}}%
\pgfpathlineto{\pgfqpoint{5.559405in}{5.233704in}}%
\pgfpathlineto{\pgfqpoint{5.561090in}{5.237894in}}%
\pgfpathlineto{\pgfqpoint{5.562775in}{5.230226in}}%
\pgfpathlineto{\pgfqpoint{5.563618in}{5.231064in}}%
\pgfpathlineto{\pgfqpoint{5.564039in}{5.230854in}}%
\pgfpathlineto{\pgfqpoint{5.565724in}{5.234248in}}%
\pgfpathlineto{\pgfqpoint{5.566567in}{5.231022in}}%
\pgfpathlineto{\pgfqpoint{5.566988in}{5.231315in}}%
\pgfpathlineto{\pgfqpoint{5.567830in}{5.231944in}}%
\pgfpathlineto{\pgfqpoint{5.569094in}{5.235045in}}%
\pgfpathlineto{\pgfqpoint{5.570358in}{5.232656in}}%
\pgfpathlineto{\pgfqpoint{5.570779in}{5.232866in}}%
\pgfpathlineto{\pgfqpoint{5.572465in}{5.237684in}}%
\pgfpathlineto{\pgfqpoint{5.572886in}{5.236344in}}%
\pgfpathlineto{\pgfqpoint{5.573307in}{5.233829in}}%
\pgfpathlineto{\pgfqpoint{5.574150in}{5.234290in}}%
\pgfpathlineto{\pgfqpoint{5.575414in}{5.237182in}}%
\pgfpathlineto{\pgfqpoint{5.576256in}{5.239528in}}%
\pgfpathlineto{\pgfqpoint{5.576677in}{5.237014in}}%
\pgfpathlineto{\pgfqpoint{5.577520in}{5.237601in}}%
\pgfpathlineto{\pgfqpoint{5.579626in}{5.240366in}}%
\pgfpathlineto{\pgfqpoint{5.583839in}{5.222725in}}%
\pgfpathlineto{\pgfqpoint{5.585524in}{5.216482in}}%
\pgfpathlineto{\pgfqpoint{5.588052in}{5.204079in}}%
\pgfpathlineto{\pgfqpoint{5.588894in}{5.199805in}}%
\pgfpathlineto{\pgfqpoint{5.589316in}{5.200643in}}%
\pgfpathlineto{\pgfqpoint{5.590158in}{5.204289in}}%
\pgfpathlineto{\pgfqpoint{5.591001in}{5.203492in}}%
\pgfpathlineto{\pgfqpoint{5.592265in}{5.201984in}}%
\pgfpathlineto{\pgfqpoint{5.593107in}{5.204037in}}%
\pgfpathlineto{\pgfqpoint{5.596899in}{5.217111in}}%
\pgfpathlineto{\pgfqpoint{5.597320in}{5.216482in}}%
\pgfpathlineto{\pgfqpoint{5.597741in}{5.215728in}}%
\pgfpathlineto{\pgfqpoint{5.598163in}{5.216733in}}%
\pgfpathlineto{\pgfqpoint{5.601533in}{5.229849in}}%
\pgfpathlineto{\pgfqpoint{5.602375in}{5.233704in}}%
\pgfpathlineto{\pgfqpoint{5.603218in}{5.233327in}}%
\pgfpathlineto{\pgfqpoint{5.607852in}{5.232824in}}%
\pgfpathlineto{\pgfqpoint{5.609537in}{5.233410in}}%
\pgfpathlineto{\pgfqpoint{5.610380in}{5.232111in}}%
\pgfpathlineto{\pgfqpoint{5.611222in}{5.233746in}}%
\pgfpathlineto{\pgfqpoint{5.611643in}{5.233201in}}%
\pgfpathlineto{\pgfqpoint{5.613750in}{5.234039in}}%
\pgfpathlineto{\pgfqpoint{5.613750in}{5.234039in}}%
\pgfusepath{stroke}%
\end{pgfscope}%
\begin{pgfscope}%
\pgfpathrectangle{\pgfqpoint{0.885050in}{4.360741in}}{\pgfqpoint{4.955200in}{1.285926in}}%
\pgfusepath{clip}%
\pgfsetrectcap%
\pgfsetroundjoin%
\pgfsetlinewidth{1.505625pt}%
\definecolor{currentstroke}{rgb}{1.000000,0.145098,0.145098}%
\pgfsetstrokecolor{currentstroke}%
\pgfsetdash{}{0pt}%
\pgfpathmoveto{\pgfqpoint{1.110287in}{4.885561in}}%
\pgfpathlineto{\pgfqpoint{1.111550in}{4.831361in}}%
\pgfpathlineto{\pgfqpoint{1.111972in}{4.836787in}}%
\pgfpathlineto{\pgfqpoint{1.112814in}{4.859603in}}%
\pgfpathlineto{\pgfqpoint{1.119133in}{5.169028in}}%
\pgfpathlineto{\pgfqpoint{1.122504in}{5.217257in}}%
\pgfpathlineto{\pgfqpoint{1.123346in}{5.209023in}}%
\pgfpathlineto{\pgfqpoint{1.123767in}{5.207494in}}%
\pgfpathlineto{\pgfqpoint{1.124610in}{5.207997in}}%
\pgfpathlineto{\pgfqpoint{1.125874in}{5.209861in}}%
\pgfpathlineto{\pgfqpoint{1.126295in}{5.207955in}}%
\pgfpathlineto{\pgfqpoint{1.127980in}{5.155221in}}%
\pgfpathlineto{\pgfqpoint{1.132614in}{4.970915in}}%
\pgfpathlineto{\pgfqpoint{1.133457in}{4.963059in}}%
\pgfpathlineto{\pgfqpoint{1.136406in}{4.903286in}}%
\pgfpathlineto{\pgfqpoint{1.137248in}{4.907036in}}%
\pgfpathlineto{\pgfqpoint{1.138091in}{4.908104in}}%
\pgfpathlineto{\pgfqpoint{1.138512in}{4.907560in}}%
\pgfpathlineto{\pgfqpoint{1.139355in}{4.907057in}}%
\pgfpathlineto{\pgfqpoint{1.139776in}{4.908984in}}%
\pgfpathlineto{\pgfqpoint{1.141461in}{4.960922in}}%
\pgfpathlineto{\pgfqpoint{1.146095in}{5.149523in}}%
\pgfpathlineto{\pgfqpoint{1.146938in}{5.157903in}}%
\pgfpathlineto{\pgfqpoint{1.149465in}{5.210029in}}%
\pgfpathlineto{\pgfqpoint{1.151993in}{5.321488in}}%
\pgfpathlineto{\pgfqpoint{1.152836in}{5.325448in}}%
\pgfpathlineto{\pgfqpoint{1.155363in}{5.404852in}}%
\pgfpathlineto{\pgfqpoint{1.156206in}{5.403260in}}%
\pgfpathlineto{\pgfqpoint{1.158734in}{5.537263in}}%
\pgfpathlineto{\pgfqpoint{1.159155in}{5.536927in}}%
\pgfpathlineto{\pgfqpoint{1.160419in}{5.535859in}}%
\pgfpathlineto{\pgfqpoint{1.160840in}{5.538101in}}%
\pgfpathlineto{\pgfqpoint{1.162104in}{5.567998in}}%
\pgfpathlineto{\pgfqpoint{1.162525in}{5.561692in}}%
\pgfpathlineto{\pgfqpoint{1.163368in}{5.516898in}}%
\pgfpathlineto{\pgfqpoint{1.165474in}{5.319812in}}%
\pgfpathlineto{\pgfqpoint{1.165895in}{5.355597in}}%
\pgfpathlineto{\pgfqpoint{1.167159in}{5.271227in}}%
\pgfpathlineto{\pgfqpoint{1.171793in}{4.908104in}}%
\pgfpathlineto{\pgfqpoint{1.176427in}{4.715397in}}%
\pgfpathlineto{\pgfqpoint{1.177270in}{4.741376in}}%
\pgfpathlineto{\pgfqpoint{1.178955in}{4.812254in}}%
\pgfpathlineto{\pgfqpoint{1.179376in}{4.766706in}}%
\pgfpathlineto{\pgfqpoint{1.180219in}{4.790234in}}%
\pgfpathlineto{\pgfqpoint{1.183168in}{4.892978in}}%
\pgfpathlineto{\pgfqpoint{1.185274in}{4.926541in}}%
\pgfpathlineto{\pgfqpoint{1.190329in}{5.096265in}}%
\pgfpathlineto{\pgfqpoint{1.192857in}{5.173910in}}%
\pgfpathlineto{\pgfqpoint{1.194121in}{5.228047in}}%
\pgfpathlineto{\pgfqpoint{1.194542in}{5.227020in}}%
\pgfpathlineto{\pgfqpoint{1.195385in}{5.238481in}}%
\pgfpathlineto{\pgfqpoint{1.200019in}{5.311935in}}%
\pgfpathlineto{\pgfqpoint{1.203389in}{5.253880in}}%
\pgfpathlineto{\pgfqpoint{1.204232in}{5.370032in}}%
\pgfpathlineto{\pgfqpoint{1.204653in}{5.357398in}}%
\pgfpathlineto{\pgfqpoint{1.205495in}{5.379711in}}%
\pgfpathlineto{\pgfqpoint{1.205917in}{5.378119in}}%
\pgfpathlineto{\pgfqpoint{1.211393in}{5.235799in}}%
\pgfpathlineto{\pgfqpoint{1.213078in}{5.210616in}}%
\pgfpathlineto{\pgfqpoint{1.213500in}{5.212732in}}%
\pgfpathlineto{\pgfqpoint{1.214764in}{5.220966in}}%
\pgfpathlineto{\pgfqpoint{1.216870in}{5.262972in}}%
\pgfpathlineto{\pgfqpoint{1.218555in}{5.101817in}}%
\pgfpathlineto{\pgfqpoint{1.219819in}{5.083737in}}%
\pgfpathlineto{\pgfqpoint{1.221083in}{5.096391in}}%
\pgfpathlineto{\pgfqpoint{1.221504in}{5.093646in}}%
\pgfpathlineto{\pgfqpoint{1.221925in}{5.088534in}}%
\pgfpathlineto{\pgfqpoint{1.222347in}{5.092683in}}%
\pgfpathlineto{\pgfqpoint{1.227402in}{5.178163in}}%
\pgfpathlineto{\pgfqpoint{1.229930in}{5.179084in}}%
\pgfpathlineto{\pgfqpoint{1.230351in}{5.178770in}}%
\pgfpathlineto{\pgfqpoint{1.230772in}{5.179399in}}%
\pgfpathlineto{\pgfqpoint{1.231615in}{5.180216in}}%
\pgfpathlineto{\pgfqpoint{1.232879in}{5.190775in}}%
\pgfpathlineto{\pgfqpoint{1.233300in}{5.189288in}}%
\pgfpathlineto{\pgfqpoint{1.234564in}{5.181850in}}%
\pgfpathlineto{\pgfqpoint{1.238355in}{5.144285in}}%
\pgfpathlineto{\pgfqpoint{1.238776in}{5.144767in}}%
\pgfpathlineto{\pgfqpoint{1.239619in}{5.150696in}}%
\pgfpathlineto{\pgfqpoint{1.240040in}{5.148622in}}%
\pgfpathlineto{\pgfqpoint{1.241304in}{5.141834in}}%
\pgfpathlineto{\pgfqpoint{1.244674in}{5.145374in}}%
\pgfpathlineto{\pgfqpoint{1.245096in}{5.146652in}}%
\pgfpathlineto{\pgfqpoint{1.245517in}{5.142693in}}%
\pgfpathlineto{\pgfqpoint{1.245938in}{5.143468in}}%
\pgfpathlineto{\pgfqpoint{1.248045in}{5.148056in}}%
\pgfpathlineto{\pgfqpoint{1.248466in}{5.149208in}}%
\pgfpathlineto{\pgfqpoint{1.248887in}{5.147092in}}%
\pgfpathlineto{\pgfqpoint{1.249730in}{5.147867in}}%
\pgfpathlineto{\pgfqpoint{1.250151in}{5.148747in}}%
\pgfpathlineto{\pgfqpoint{1.250572in}{5.146443in}}%
\pgfpathlineto{\pgfqpoint{1.250993in}{5.147218in}}%
\pgfpathlineto{\pgfqpoint{1.251836in}{5.149208in}}%
\pgfpathlineto{\pgfqpoint{1.253942in}{5.126539in}}%
\pgfpathlineto{\pgfqpoint{1.255206in}{5.134752in}}%
\pgfpathlineto{\pgfqpoint{1.255628in}{5.134815in}}%
\pgfpathlineto{\pgfqpoint{1.256470in}{5.142735in}}%
\pgfpathlineto{\pgfqpoint{1.256891in}{5.142399in}}%
\pgfpathlineto{\pgfqpoint{1.257313in}{5.141980in}}%
\pgfpathlineto{\pgfqpoint{1.258576in}{5.157274in}}%
\pgfpathlineto{\pgfqpoint{1.258998in}{5.152414in}}%
\pgfpathlineto{\pgfqpoint{1.259419in}{5.159956in}}%
\pgfpathlineto{\pgfqpoint{1.261947in}{5.208018in}}%
\pgfpathlineto{\pgfqpoint{1.262368in}{5.209212in}}%
\pgfpathlineto{\pgfqpoint{1.264053in}{5.155054in}}%
\pgfpathlineto{\pgfqpoint{1.264896in}{5.118033in}}%
\pgfpathlineto{\pgfqpoint{1.265317in}{5.132427in}}%
\pgfpathlineto{\pgfqpoint{1.267423in}{5.186983in}}%
\pgfpathlineto{\pgfqpoint{1.267845in}{5.178351in}}%
\pgfpathlineto{\pgfqpoint{1.268266in}{5.176633in}}%
\pgfpathlineto{\pgfqpoint{1.269530in}{5.131023in}}%
\pgfpathlineto{\pgfqpoint{1.271215in}{5.047512in}}%
\pgfpathlineto{\pgfqpoint{1.271636in}{5.055411in}}%
\pgfpathlineto{\pgfqpoint{1.272479in}{5.078394in}}%
\pgfpathlineto{\pgfqpoint{1.272900in}{5.073722in}}%
\pgfpathlineto{\pgfqpoint{1.274585in}{5.058491in}}%
\pgfpathlineto{\pgfqpoint{1.276270in}{5.050047in}}%
\pgfpathlineto{\pgfqpoint{1.277113in}{5.047491in}}%
\pgfpathlineto{\pgfqpoint{1.278377in}{5.062806in}}%
\pgfpathlineto{\pgfqpoint{1.282589in}{4.877956in}}%
\pgfpathlineto{\pgfqpoint{1.283011in}{4.878061in}}%
\pgfpathlineto{\pgfqpoint{1.284274in}{4.899954in}}%
\pgfpathlineto{\pgfqpoint{1.284696in}{4.889626in}}%
\pgfpathlineto{\pgfqpoint{1.290594in}{4.598700in}}%
\pgfpathlineto{\pgfqpoint{1.293543in}{4.554053in}}%
\pgfpathlineto{\pgfqpoint{1.297755in}{4.725349in}}%
\pgfpathlineto{\pgfqpoint{1.300283in}{4.847095in}}%
\pgfpathlineto{\pgfqpoint{1.305338in}{5.219625in}}%
\pgfpathlineto{\pgfqpoint{1.308287in}{5.367832in}}%
\pgfpathlineto{\pgfqpoint{1.310815in}{5.427019in}}%
\pgfpathlineto{\pgfqpoint{1.311236in}{5.429889in}}%
\pgfpathlineto{\pgfqpoint{1.312500in}{5.467182in}}%
\pgfpathlineto{\pgfqpoint{1.313343in}{5.461127in}}%
\pgfpathlineto{\pgfqpoint{1.326824in}{5.458948in}}%
\pgfpathlineto{\pgfqpoint{1.328087in}{5.429009in}}%
\pgfpathlineto{\pgfqpoint{1.331458in}{5.206069in}}%
\pgfpathlineto{\pgfqpoint{1.340304in}{4.687847in}}%
\pgfpathlineto{\pgfqpoint{1.341568in}{4.656630in}}%
\pgfpathlineto{\pgfqpoint{1.341990in}{4.656902in}}%
\pgfpathlineto{\pgfqpoint{1.342411in}{4.657803in}}%
\pgfpathlineto{\pgfqpoint{1.359683in}{5.059496in}}%
\pgfpathlineto{\pgfqpoint{1.360947in}{5.042610in}}%
\pgfpathlineto{\pgfqpoint{1.371058in}{4.798992in}}%
\pgfpathlineto{\pgfqpoint{1.373164in}{4.784745in}}%
\pgfpathlineto{\pgfqpoint{1.374849in}{4.828176in}}%
\pgfpathlineto{\pgfqpoint{1.381168in}{5.011120in}}%
\pgfpathlineto{\pgfqpoint{1.385381in}{5.127608in}}%
\pgfpathlineto{\pgfqpoint{1.387488in}{5.144872in}}%
\pgfpathlineto{\pgfqpoint{1.389173in}{5.161004in}}%
\pgfpathlineto{\pgfqpoint{1.388330in}{5.143279in}}%
\pgfpathlineto{\pgfqpoint{1.389594in}{5.156499in}}%
\pgfpathlineto{\pgfqpoint{1.390858in}{5.151681in}}%
\pgfpathlineto{\pgfqpoint{1.391700in}{5.155829in}}%
\pgfpathlineto{\pgfqpoint{1.392543in}{5.150130in}}%
\pgfpathlineto{\pgfqpoint{1.392964in}{5.152393in}}%
\pgfpathlineto{\pgfqpoint{1.393807in}{5.171312in}}%
\pgfpathlineto{\pgfqpoint{1.394228in}{5.170662in}}%
\pgfpathlineto{\pgfqpoint{1.395913in}{5.161234in}}%
\pgfpathlineto{\pgfqpoint{1.396334in}{5.165697in}}%
\pgfpathlineto{\pgfqpoint{1.396756in}{5.164544in}}%
\pgfpathlineto{\pgfqpoint{1.398020in}{5.149355in}}%
\pgfpathlineto{\pgfqpoint{1.398441in}{5.153084in}}%
\pgfpathlineto{\pgfqpoint{1.398862in}{5.147909in}}%
\pgfpathlineto{\pgfqpoint{1.403075in}{5.092724in}}%
\pgfpathlineto{\pgfqpoint{1.403496in}{5.098926in}}%
\pgfpathlineto{\pgfqpoint{1.410658in}{5.227523in}}%
\pgfpathlineto{\pgfqpoint{1.412343in}{5.282101in}}%
\pgfpathlineto{\pgfqpoint{1.415292in}{5.317571in}}%
\pgfpathlineto{\pgfqpoint{1.415713in}{5.317340in}}%
\pgfpathlineto{\pgfqpoint{1.416135in}{5.317214in}}%
\pgfpathlineto{\pgfqpoint{1.416556in}{5.319058in}}%
\pgfpathlineto{\pgfqpoint{1.418662in}{5.344325in}}%
\pgfpathlineto{\pgfqpoint{1.419083in}{5.339150in}}%
\pgfpathlineto{\pgfqpoint{1.422032in}{5.251722in}}%
\pgfpathlineto{\pgfqpoint{1.424981in}{5.148161in}}%
\pgfpathlineto{\pgfqpoint{1.425403in}{5.152602in}}%
\pgfpathlineto{\pgfqpoint{1.425824in}{5.153021in}}%
\pgfpathlineto{\pgfqpoint{1.429194in}{5.143405in}}%
\pgfpathlineto{\pgfqpoint{1.429615in}{5.143698in}}%
\pgfpathlineto{\pgfqpoint{1.430458in}{5.148433in}}%
\pgfpathlineto{\pgfqpoint{1.431301in}{5.148119in}}%
\pgfpathlineto{\pgfqpoint{1.432143in}{5.147490in}}%
\pgfpathlineto{\pgfqpoint{1.432986in}{5.145207in}}%
\pgfpathlineto{\pgfqpoint{1.433407in}{5.145814in}}%
\pgfpathlineto{\pgfqpoint{1.434249in}{5.144641in}}%
\pgfpathlineto{\pgfqpoint{1.436356in}{5.141938in}}%
\pgfpathlineto{\pgfqpoint{1.438041in}{5.145856in}}%
\pgfpathlineto{\pgfqpoint{1.438462in}{5.147616in}}%
\pgfpathlineto{\pgfqpoint{1.438884in}{5.146946in}}%
\pgfpathlineto{\pgfqpoint{1.439726in}{5.146338in}}%
\pgfpathlineto{\pgfqpoint{1.440147in}{5.147281in}}%
\pgfpathlineto{\pgfqpoint{1.440990in}{5.146778in}}%
\pgfpathlineto{\pgfqpoint{1.442254in}{5.145751in}}%
\pgfpathlineto{\pgfqpoint{1.443096in}{5.145249in}}%
\pgfpathlineto{\pgfqpoint{1.443939in}{5.149837in}}%
\pgfpathlineto{\pgfqpoint{1.444360in}{5.149690in}}%
\pgfpathlineto{\pgfqpoint{1.444781in}{5.149627in}}%
\pgfpathlineto{\pgfqpoint{1.445203in}{5.152791in}}%
\pgfpathlineto{\pgfqpoint{1.446045in}{5.151450in}}%
\pgfpathlineto{\pgfqpoint{1.446467in}{5.150403in}}%
\pgfpathlineto{\pgfqpoint{1.446888in}{5.152267in}}%
\pgfpathlineto{\pgfqpoint{1.447730in}{5.152770in}}%
\pgfpathlineto{\pgfqpoint{1.448152in}{5.151094in}}%
\pgfpathlineto{\pgfqpoint{1.448994in}{5.151534in}}%
\pgfpathlineto{\pgfqpoint{1.449837in}{5.150738in}}%
\pgfpathlineto{\pgfqpoint{1.450258in}{5.156625in}}%
\pgfpathlineto{\pgfqpoint{1.451101in}{5.155661in}}%
\pgfpathlineto{\pgfqpoint{1.451522in}{5.155033in}}%
\pgfpathlineto{\pgfqpoint{1.451943in}{5.157714in}}%
\pgfpathlineto{\pgfqpoint{1.452364in}{5.156059in}}%
\pgfpathlineto{\pgfqpoint{1.453207in}{5.151995in}}%
\pgfpathlineto{\pgfqpoint{1.454050in}{5.153587in}}%
\pgfpathlineto{\pgfqpoint{1.454471in}{5.153482in}}%
\pgfpathlineto{\pgfqpoint{1.454892in}{5.150717in}}%
\pgfpathlineto{\pgfqpoint{1.455735in}{5.150968in}}%
\pgfpathlineto{\pgfqpoint{1.456577in}{5.149732in}}%
\pgfpathlineto{\pgfqpoint{1.456998in}{5.156101in}}%
\pgfpathlineto{\pgfqpoint{1.457841in}{5.153147in}}%
\pgfpathlineto{\pgfqpoint{1.459526in}{5.141813in}}%
\pgfpathlineto{\pgfqpoint{1.459947in}{5.145856in}}%
\pgfpathlineto{\pgfqpoint{1.462054in}{5.181536in}}%
\pgfpathlineto{\pgfqpoint{1.463739in}{5.232719in}}%
\pgfpathlineto{\pgfqpoint{1.464581in}{5.226811in}}%
\pgfpathlineto{\pgfqpoint{1.465845in}{5.215204in}}%
\pgfpathlineto{\pgfqpoint{1.466688in}{5.220693in}}%
\pgfpathlineto{\pgfqpoint{1.469637in}{5.286416in}}%
\pgfpathlineto{\pgfqpoint{1.470479in}{5.308373in}}%
\pgfpathlineto{\pgfqpoint{1.470901in}{5.303659in}}%
\pgfpathlineto{\pgfqpoint{1.471322in}{5.302318in}}%
\pgfpathlineto{\pgfqpoint{1.472164in}{5.306927in}}%
\pgfpathlineto{\pgfqpoint{1.473007in}{5.301857in}}%
\pgfpathlineto{\pgfqpoint{1.473850in}{5.306802in}}%
\pgfpathlineto{\pgfqpoint{1.474692in}{5.302297in}}%
\pgfpathlineto{\pgfqpoint{1.475113in}{5.307200in}}%
\pgfpathlineto{\pgfqpoint{1.475535in}{5.305503in}}%
\pgfpathlineto{\pgfqpoint{1.476377in}{5.302591in}}%
\pgfpathlineto{\pgfqpoint{1.476799in}{5.307598in}}%
\pgfpathlineto{\pgfqpoint{1.477220in}{5.304518in}}%
\pgfpathlineto{\pgfqpoint{1.478062in}{5.303261in}}%
\pgfpathlineto{\pgfqpoint{1.478484in}{5.307849in}}%
\pgfpathlineto{\pgfqpoint{1.478905in}{5.302025in}}%
\pgfpathlineto{\pgfqpoint{1.479326in}{5.298757in}}%
\pgfpathlineto{\pgfqpoint{1.479748in}{5.299448in}}%
\pgfpathlineto{\pgfqpoint{1.480169in}{5.304162in}}%
\pgfpathlineto{\pgfqpoint{1.480590in}{5.297353in}}%
\pgfpathlineto{\pgfqpoint{1.481011in}{5.294064in}}%
\pgfpathlineto{\pgfqpoint{1.481433in}{5.296012in}}%
\pgfpathlineto{\pgfqpoint{1.481854in}{5.297416in}}%
\pgfpathlineto{\pgfqpoint{1.482696in}{5.284321in}}%
\pgfpathlineto{\pgfqpoint{1.483118in}{5.292744in}}%
\pgfpathlineto{\pgfqpoint{1.484803in}{5.306194in}}%
\pgfpathlineto{\pgfqpoint{1.486067in}{5.302863in}}%
\pgfpathlineto{\pgfqpoint{1.486488in}{5.307933in}}%
\pgfpathlineto{\pgfqpoint{1.486909in}{5.305209in}}%
\pgfpathlineto{\pgfqpoint{1.487331in}{5.303806in}}%
\pgfpathlineto{\pgfqpoint{1.487752in}{5.304225in}}%
\pgfpathlineto{\pgfqpoint{1.488173in}{5.309714in}}%
\pgfpathlineto{\pgfqpoint{1.488594in}{5.306236in}}%
\pgfpathlineto{\pgfqpoint{1.489016in}{5.305293in}}%
\pgfpathlineto{\pgfqpoint{1.489437in}{5.306529in}}%
\pgfpathlineto{\pgfqpoint{1.489858in}{5.312123in}}%
\pgfpathlineto{\pgfqpoint{1.490279in}{5.308017in}}%
\pgfpathlineto{\pgfqpoint{1.490701in}{5.306927in}}%
\pgfpathlineto{\pgfqpoint{1.491122in}{5.308792in}}%
\pgfpathlineto{\pgfqpoint{1.491543in}{5.312207in}}%
\pgfpathlineto{\pgfqpoint{1.491965in}{5.308226in}}%
\pgfpathlineto{\pgfqpoint{1.492386in}{5.309756in}}%
\pgfpathlineto{\pgfqpoint{1.493228in}{5.317382in}}%
\pgfpathlineto{\pgfqpoint{1.493650in}{5.313778in}}%
\pgfpathlineto{\pgfqpoint{1.495756in}{5.324799in}}%
\pgfpathlineto{\pgfqpoint{1.496177in}{5.333451in}}%
\pgfpathlineto{\pgfqpoint{1.496599in}{5.326098in}}%
\pgfpathlineto{\pgfqpoint{1.497441in}{5.312270in}}%
\pgfpathlineto{\pgfqpoint{1.498284in}{5.313150in}}%
\pgfpathlineto{\pgfqpoint{1.498705in}{5.311600in}}%
\pgfpathlineto{\pgfqpoint{1.499126in}{5.311977in}}%
\pgfpathlineto{\pgfqpoint{1.499548in}{5.317026in}}%
\pgfpathlineto{\pgfqpoint{1.499969in}{5.312773in}}%
\pgfpathlineto{\pgfqpoint{1.500390in}{5.311432in}}%
\pgfpathlineto{\pgfqpoint{1.500811in}{5.312626in}}%
\pgfpathlineto{\pgfqpoint{1.501233in}{5.316167in}}%
\pgfpathlineto{\pgfqpoint{1.501654in}{5.311453in}}%
\pgfpathlineto{\pgfqpoint{1.502075in}{5.310342in}}%
\pgfpathlineto{\pgfqpoint{1.502918in}{5.314491in}}%
\pgfpathlineto{\pgfqpoint{1.503760in}{5.308918in}}%
\pgfpathlineto{\pgfqpoint{1.504182in}{5.313234in}}%
\pgfpathlineto{\pgfqpoint{1.504603in}{5.311222in}}%
\pgfpathlineto{\pgfqpoint{1.505445in}{5.301187in}}%
\pgfpathlineto{\pgfqpoint{1.506709in}{5.313506in}}%
\pgfpathlineto{\pgfqpoint{1.507131in}{5.313150in}}%
\pgfpathlineto{\pgfqpoint{1.507552in}{5.317906in}}%
\pgfpathlineto{\pgfqpoint{1.507973in}{5.312123in}}%
\pgfpathlineto{\pgfqpoint{1.508816in}{5.305817in}}%
\pgfpathlineto{\pgfqpoint{1.510080in}{5.315769in}}%
\pgfpathlineto{\pgfqpoint{1.510922in}{5.321509in}}%
\pgfpathlineto{\pgfqpoint{1.511765in}{5.312312in}}%
\pgfpathlineto{\pgfqpoint{1.512186in}{5.315015in}}%
\pgfpathlineto{\pgfqpoint{1.512607in}{5.315957in}}%
\pgfpathlineto{\pgfqpoint{1.513450in}{5.307472in}}%
\pgfpathlineto{\pgfqpoint{1.513871in}{5.311872in}}%
\pgfpathlineto{\pgfqpoint{1.514292in}{5.310741in}}%
\pgfpathlineto{\pgfqpoint{1.515135in}{5.303261in}}%
\pgfpathlineto{\pgfqpoint{1.515556in}{5.308645in}}%
\pgfpathlineto{\pgfqpoint{1.516820in}{5.299322in}}%
\pgfpathlineto{\pgfqpoint{1.517241in}{5.304539in}}%
\pgfpathlineto{\pgfqpoint{1.517663in}{5.299490in}}%
\pgfpathlineto{\pgfqpoint{1.518084in}{5.297709in}}%
\pgfpathlineto{\pgfqpoint{1.518505in}{5.300663in}}%
\pgfpathlineto{\pgfqpoint{1.518926in}{5.308122in}}%
\pgfpathlineto{\pgfqpoint{1.519348in}{5.296578in}}%
\pgfpathlineto{\pgfqpoint{1.521033in}{5.285327in}}%
\pgfpathlineto{\pgfqpoint{1.521454in}{5.284447in}}%
\pgfpathlineto{\pgfqpoint{1.522297in}{5.292932in}}%
\pgfpathlineto{\pgfqpoint{1.523982in}{5.275438in}}%
\pgfpathlineto{\pgfqpoint{1.524824in}{5.269635in}}%
\pgfpathlineto{\pgfqpoint{1.525246in}{5.276926in}}%
\pgfpathlineto{\pgfqpoint{1.525667in}{5.273951in}}%
\pgfpathlineto{\pgfqpoint{1.526509in}{5.269844in}}%
\pgfpathlineto{\pgfqpoint{1.526931in}{5.276109in}}%
\pgfpathlineto{\pgfqpoint{1.527352in}{5.270892in}}%
\pgfpathlineto{\pgfqpoint{1.527773in}{5.268461in}}%
\pgfpathlineto{\pgfqpoint{1.528194in}{5.269509in}}%
\pgfpathlineto{\pgfqpoint{1.528616in}{5.275375in}}%
\pgfpathlineto{\pgfqpoint{1.529037in}{5.268985in}}%
\pgfpathlineto{\pgfqpoint{1.529458in}{5.266932in}}%
\pgfpathlineto{\pgfqpoint{1.529880in}{5.268818in}}%
\pgfpathlineto{\pgfqpoint{1.530301in}{5.276004in}}%
\pgfpathlineto{\pgfqpoint{1.530722in}{5.270347in}}%
\pgfpathlineto{\pgfqpoint{1.531143in}{5.268336in}}%
\pgfpathlineto{\pgfqpoint{1.531986in}{5.274391in}}%
\pgfpathlineto{\pgfqpoint{1.532829in}{5.266890in}}%
\pgfpathlineto{\pgfqpoint{1.533250in}{5.271730in}}%
\pgfpathlineto{\pgfqpoint{1.533671in}{5.271520in}}%
\pgfpathlineto{\pgfqpoint{1.534514in}{5.261212in}}%
\pgfpathlineto{\pgfqpoint{1.534935in}{5.268231in}}%
\pgfpathlineto{\pgfqpoint{1.536199in}{5.283839in}}%
\pgfpathlineto{\pgfqpoint{1.537041in}{5.295090in}}%
\pgfpathlineto{\pgfqpoint{1.537463in}{5.294734in}}%
\pgfpathlineto{\pgfqpoint{1.538305in}{5.287925in}}%
\pgfpathlineto{\pgfqpoint{1.542518in}{5.046339in}}%
\pgfpathlineto{\pgfqpoint{1.547995in}{4.727486in}}%
\pgfpathlineto{\pgfqpoint{1.548837in}{4.728806in}}%
\pgfpathlineto{\pgfqpoint{1.550101in}{4.728764in}}%
\pgfpathlineto{\pgfqpoint{1.550522in}{4.731236in}}%
\pgfpathlineto{\pgfqpoint{1.551365in}{4.730964in}}%
\pgfpathlineto{\pgfqpoint{1.556420in}{4.729434in}}%
\pgfpathlineto{\pgfqpoint{1.561054in}{4.754282in}}%
\pgfpathlineto{\pgfqpoint{1.562318in}{4.681708in}}%
\pgfpathlineto{\pgfqpoint{1.566110in}{4.487304in}}%
\pgfpathlineto{\pgfqpoint{1.566952in}{4.489084in}}%
\pgfpathlineto{\pgfqpoint{1.571586in}{4.516321in}}%
\pgfpathlineto{\pgfqpoint{1.574114in}{4.517976in}}%
\pgfpathlineto{\pgfqpoint{1.575378in}{4.519338in}}%
\pgfpathlineto{\pgfqpoint{1.578327in}{4.519505in}}%
\pgfpathlineto{\pgfqpoint{1.578748in}{4.518793in}}%
\pgfpathlineto{\pgfqpoint{1.582539in}{4.656022in}}%
\pgfpathlineto{\pgfqpoint{1.592650in}{5.044956in}}%
\pgfpathlineto{\pgfqpoint{1.596863in}{5.105295in}}%
\pgfpathlineto{\pgfqpoint{1.598548in}{5.154949in}}%
\pgfpathlineto{\pgfqpoint{1.598969in}{5.152980in}}%
\pgfpathlineto{\pgfqpoint{1.599390in}{5.152372in}}%
\pgfpathlineto{\pgfqpoint{1.601076in}{5.183107in}}%
\pgfpathlineto{\pgfqpoint{1.601918in}{5.177471in}}%
\pgfpathlineto{\pgfqpoint{1.602761in}{5.173302in}}%
\pgfpathlineto{\pgfqpoint{1.603182in}{5.174412in}}%
\pgfpathlineto{\pgfqpoint{1.603603in}{5.175984in}}%
\pgfpathlineto{\pgfqpoint{1.606131in}{5.138104in}}%
\pgfpathlineto{\pgfqpoint{1.613293in}{4.961906in}}%
\pgfpathlineto{\pgfqpoint{1.615399in}{4.922644in}}%
\pgfpathlineto{\pgfqpoint{1.615820in}{4.921702in}}%
\pgfpathlineto{\pgfqpoint{1.616242in}{4.922581in}}%
\pgfpathlineto{\pgfqpoint{1.620454in}{4.971670in}}%
\pgfpathlineto{\pgfqpoint{1.623403in}{5.042463in}}%
\pgfpathlineto{\pgfqpoint{1.624246in}{5.045208in}}%
\pgfpathlineto{\pgfqpoint{1.625931in}{5.055306in}}%
\pgfpathlineto{\pgfqpoint{1.626352in}{5.055264in}}%
\pgfpathlineto{\pgfqpoint{1.627195in}{5.068526in}}%
\pgfpathlineto{\pgfqpoint{1.630565in}{5.122014in}}%
\pgfpathlineto{\pgfqpoint{1.633093in}{5.145123in}}%
\pgfpathlineto{\pgfqpoint{1.633935in}{5.164838in}}%
\pgfpathlineto{\pgfqpoint{1.634357in}{5.159893in}}%
\pgfpathlineto{\pgfqpoint{1.635620in}{5.133558in}}%
\pgfpathlineto{\pgfqpoint{1.636042in}{5.135506in}}%
\pgfpathlineto{\pgfqpoint{1.639412in}{5.178582in}}%
\pgfpathlineto{\pgfqpoint{1.642782in}{5.236595in}}%
\pgfpathlineto{\pgfqpoint{1.644889in}{5.266178in}}%
\pgfpathlineto{\pgfqpoint{1.646152in}{5.310845in}}%
\pgfpathlineto{\pgfqpoint{1.646574in}{5.308520in}}%
\pgfpathlineto{\pgfqpoint{1.646995in}{5.304057in}}%
\pgfpathlineto{\pgfqpoint{1.647416in}{5.304749in}}%
\pgfpathlineto{\pgfqpoint{1.649101in}{5.329513in}}%
\pgfpathlineto{\pgfqpoint{1.649523in}{5.325092in}}%
\pgfpathlineto{\pgfqpoint{1.650365in}{5.320168in}}%
\pgfpathlineto{\pgfqpoint{1.650786in}{5.323123in}}%
\pgfpathlineto{\pgfqpoint{1.652050in}{5.328926in}}%
\pgfpathlineto{\pgfqpoint{1.654999in}{5.301040in}}%
\pgfpathlineto{\pgfqpoint{1.657106in}{5.269299in}}%
\pgfpathlineto{\pgfqpoint{1.657527in}{5.269865in}}%
\pgfpathlineto{\pgfqpoint{1.657948in}{5.266052in}}%
\pgfpathlineto{\pgfqpoint{1.660897in}{5.192388in}}%
\pgfpathlineto{\pgfqpoint{1.662582in}{5.164901in}}%
\pgfpathlineto{\pgfqpoint{1.667216in}{5.226308in}}%
\pgfpathlineto{\pgfqpoint{1.668059in}{5.228592in}}%
\pgfpathlineto{\pgfqpoint{1.668901in}{5.261506in}}%
\pgfpathlineto{\pgfqpoint{1.670586in}{5.308813in}}%
\pgfpathlineto{\pgfqpoint{1.671429in}{5.313485in}}%
\pgfpathlineto{\pgfqpoint{1.673114in}{5.336950in}}%
\pgfpathlineto{\pgfqpoint{1.673535in}{5.332236in}}%
\pgfpathlineto{\pgfqpoint{1.673957in}{5.328486in}}%
\pgfpathlineto{\pgfqpoint{1.674378in}{5.329534in}}%
\pgfpathlineto{\pgfqpoint{1.675642in}{5.357021in}}%
\pgfpathlineto{\pgfqpoint{1.676063in}{5.356875in}}%
\pgfpathlineto{\pgfqpoint{1.680276in}{5.299951in}}%
\pgfpathlineto{\pgfqpoint{1.680697in}{5.304560in}}%
\pgfpathlineto{\pgfqpoint{1.681540in}{5.316900in}}%
\pgfpathlineto{\pgfqpoint{1.681961in}{5.309965in}}%
\pgfpathlineto{\pgfqpoint{1.684489in}{5.236197in}}%
\pgfpathlineto{\pgfqpoint{1.686174in}{5.204456in}}%
\pgfpathlineto{\pgfqpoint{1.687016in}{5.214324in}}%
\pgfpathlineto{\pgfqpoint{1.687859in}{5.222264in}}%
\pgfpathlineto{\pgfqpoint{1.688280in}{5.220777in}}%
\pgfpathlineto{\pgfqpoint{1.689123in}{5.214135in}}%
\pgfpathlineto{\pgfqpoint{1.692914in}{5.290795in}}%
\pgfpathlineto{\pgfqpoint{1.693757in}{5.320860in}}%
\pgfpathlineto{\pgfqpoint{1.694178in}{5.319247in}}%
\pgfpathlineto{\pgfqpoint{1.695021in}{5.309756in}}%
\pgfpathlineto{\pgfqpoint{1.695442in}{5.314114in}}%
\pgfpathlineto{\pgfqpoint{1.696706in}{5.336070in}}%
\pgfpathlineto{\pgfqpoint{1.697127in}{5.333305in}}%
\pgfpathlineto{\pgfqpoint{1.697970in}{5.325679in}}%
\pgfpathlineto{\pgfqpoint{1.698391in}{5.329869in}}%
\pgfpathlineto{\pgfqpoint{1.699655in}{5.348683in}}%
\pgfpathlineto{\pgfqpoint{1.700076in}{5.346252in}}%
\pgfpathlineto{\pgfqpoint{1.704710in}{5.285097in}}%
\pgfpathlineto{\pgfqpoint{1.705553in}{5.310782in}}%
\pgfpathlineto{\pgfqpoint{1.705974in}{5.308205in}}%
\pgfpathlineto{\pgfqpoint{1.707659in}{5.285788in}}%
\pgfpathlineto{\pgfqpoint{1.708502in}{5.307765in}}%
\pgfpathlineto{\pgfqpoint{1.711029in}{5.351595in}}%
\pgfpathlineto{\pgfqpoint{1.715242in}{5.430685in}}%
\pgfpathlineto{\pgfqpoint{1.717770in}{5.454779in}}%
\pgfpathlineto{\pgfqpoint{1.718612in}{5.474829in}}%
\pgfpathlineto{\pgfqpoint{1.719033in}{5.470911in}}%
\pgfpathlineto{\pgfqpoint{1.721140in}{5.476840in}}%
\pgfpathlineto{\pgfqpoint{1.721561in}{5.486457in}}%
\pgfpathlineto{\pgfqpoint{1.721982in}{5.484361in}}%
\pgfpathlineto{\pgfqpoint{1.723668in}{5.474829in}}%
\pgfpathlineto{\pgfqpoint{1.724089in}{5.472587in}}%
\pgfpathlineto{\pgfqpoint{1.724510in}{5.473928in}}%
\pgfpathlineto{\pgfqpoint{1.724931in}{5.478495in}}%
\pgfpathlineto{\pgfqpoint{1.725353in}{5.477699in}}%
\pgfpathlineto{\pgfqpoint{1.725774in}{5.474766in}}%
\pgfpathlineto{\pgfqpoint{1.726616in}{5.475772in}}%
\pgfpathlineto{\pgfqpoint{1.727459in}{5.474116in}}%
\pgfpathlineto{\pgfqpoint{1.727880in}{5.477510in}}%
\pgfpathlineto{\pgfqpoint{1.728723in}{5.475625in}}%
\pgfpathlineto{\pgfqpoint{1.729565in}{5.472126in}}%
\pgfpathlineto{\pgfqpoint{1.729987in}{5.472859in}}%
\pgfpathlineto{\pgfqpoint{1.730829in}{5.477573in}}%
\pgfpathlineto{\pgfqpoint{1.731672in}{5.491610in}}%
\pgfpathlineto{\pgfqpoint{1.732093in}{5.485493in}}%
\pgfpathlineto{\pgfqpoint{1.732936in}{5.486415in}}%
\pgfpathlineto{\pgfqpoint{1.733778in}{5.479228in}}%
\pgfpathlineto{\pgfqpoint{1.734199in}{5.481826in}}%
\pgfpathlineto{\pgfqpoint{1.734621in}{5.491799in}}%
\pgfpathlineto{\pgfqpoint{1.735042in}{5.489369in}}%
\pgfpathlineto{\pgfqpoint{1.736727in}{5.477196in}}%
\pgfpathlineto{\pgfqpoint{1.737570in}{5.478097in}}%
\pgfpathlineto{\pgfqpoint{1.737991in}{5.482392in}}%
\pgfpathlineto{\pgfqpoint{1.738412in}{5.481449in}}%
\pgfpathlineto{\pgfqpoint{1.739676in}{5.478118in}}%
\pgfpathlineto{\pgfqpoint{1.741783in}{5.474787in}}%
\pgfpathlineto{\pgfqpoint{1.743468in}{5.468921in}}%
\pgfpathlineto{\pgfqpoint{1.743889in}{5.468606in}}%
\pgfpathlineto{\pgfqpoint{1.744731in}{5.477594in}}%
\pgfpathlineto{\pgfqpoint{1.746838in}{5.458550in}}%
\pgfpathlineto{\pgfqpoint{1.747259in}{5.461127in}}%
\pgfpathlineto{\pgfqpoint{1.747680in}{5.471686in}}%
\pgfpathlineto{\pgfqpoint{1.748102in}{5.469863in}}%
\pgfpathlineto{\pgfqpoint{1.749787in}{5.458152in}}%
\pgfpathlineto{\pgfqpoint{1.750208in}{5.458089in}}%
\pgfpathlineto{\pgfqpoint{1.751051in}{5.466134in}}%
\pgfpathlineto{\pgfqpoint{1.751893in}{5.462719in}}%
\pgfpathlineto{\pgfqpoint{1.753578in}{5.460729in}}%
\pgfpathlineto{\pgfqpoint{1.754000in}{5.462258in}}%
\pgfpathlineto{\pgfqpoint{1.754421in}{5.462677in}}%
\pgfpathlineto{\pgfqpoint{1.755685in}{5.455449in}}%
\pgfpathlineto{\pgfqpoint{1.756527in}{5.455784in}}%
\pgfpathlineto{\pgfqpoint{1.756949in}{5.456580in}}%
\pgfpathlineto{\pgfqpoint{1.757370in}{5.464877in}}%
\pgfpathlineto{\pgfqpoint{1.757791in}{5.464521in}}%
\pgfpathlineto{\pgfqpoint{1.759476in}{5.447990in}}%
\pgfpathlineto{\pgfqpoint{1.759897in}{5.447634in}}%
\pgfpathlineto{\pgfqpoint{1.760319in}{5.450148in}}%
\pgfpathlineto{\pgfqpoint{1.761161in}{5.461211in}}%
\pgfpathlineto{\pgfqpoint{1.761583in}{5.455554in}}%
\pgfpathlineto{\pgfqpoint{1.763268in}{5.451091in}}%
\pgfpathlineto{\pgfqpoint{1.770008in}{5.519245in}}%
\pgfpathlineto{\pgfqpoint{1.772115in}{5.583041in}}%
\pgfpathlineto{\pgfqpoint{1.772536in}{5.582517in}}%
\pgfpathlineto{\pgfqpoint{1.773800in}{5.588215in}}%
\pgfpathlineto{\pgfqpoint{1.775906in}{5.562634in}}%
\pgfpathlineto{\pgfqpoint{1.776749in}{5.568082in}}%
\pgfpathlineto{\pgfqpoint{1.777170in}{5.566154in}}%
\pgfpathlineto{\pgfqpoint{1.778434in}{5.540971in}}%
\pgfpathlineto{\pgfqpoint{1.778855in}{5.544826in}}%
\pgfpathlineto{\pgfqpoint{1.779276in}{5.551027in}}%
\pgfpathlineto{\pgfqpoint{1.779698in}{5.546858in}}%
\pgfpathlineto{\pgfqpoint{1.781383in}{5.527353in}}%
\pgfpathlineto{\pgfqpoint{1.781804in}{5.531417in}}%
\pgfpathlineto{\pgfqpoint{1.782225in}{5.526808in}}%
\pgfpathlineto{\pgfqpoint{1.782646in}{5.527688in}}%
\pgfpathlineto{\pgfqpoint{1.783068in}{5.525593in}}%
\pgfpathlineto{\pgfqpoint{1.784753in}{5.478034in}}%
\pgfpathlineto{\pgfqpoint{1.785595in}{5.478390in}}%
\pgfpathlineto{\pgfqpoint{1.786017in}{5.486708in}}%
\pgfpathlineto{\pgfqpoint{1.786438in}{5.485095in}}%
\pgfpathlineto{\pgfqpoint{1.788544in}{5.431272in}}%
\pgfpathlineto{\pgfqpoint{1.791493in}{5.314952in}}%
\pgfpathlineto{\pgfqpoint{1.791915in}{5.314868in}}%
\pgfpathlineto{\pgfqpoint{1.792336in}{5.308771in}}%
\pgfpathlineto{\pgfqpoint{1.792757in}{5.309253in}}%
\pgfpathlineto{\pgfqpoint{1.793600in}{5.323039in}}%
\pgfpathlineto{\pgfqpoint{1.794021in}{5.314135in}}%
\pgfpathlineto{\pgfqpoint{1.799076in}{5.151429in}}%
\pgfpathlineto{\pgfqpoint{1.799498in}{5.147847in}}%
\pgfpathlineto{\pgfqpoint{1.800761in}{5.162324in}}%
\pgfpathlineto{\pgfqpoint{1.801183in}{5.159139in}}%
\pgfpathlineto{\pgfqpoint{1.802447in}{5.170411in}}%
\pgfpathlineto{\pgfqpoint{1.803289in}{5.169552in}}%
\pgfpathlineto{\pgfqpoint{1.803710in}{5.169803in}}%
\pgfpathlineto{\pgfqpoint{1.806238in}{5.160941in}}%
\pgfpathlineto{\pgfqpoint{1.806659in}{5.161988in}}%
\pgfpathlineto{\pgfqpoint{1.807502in}{5.162177in}}%
\pgfpathlineto{\pgfqpoint{1.807923in}{5.163183in}}%
\pgfpathlineto{\pgfqpoint{1.808766in}{5.162387in}}%
\pgfpathlineto{\pgfqpoint{1.809187in}{5.162114in}}%
\pgfpathlineto{\pgfqpoint{1.809608in}{5.164712in}}%
\pgfpathlineto{\pgfqpoint{1.810451in}{5.164125in}}%
\pgfpathlineto{\pgfqpoint{1.810872in}{5.164733in}}%
\pgfpathlineto{\pgfqpoint{1.811293in}{5.169971in}}%
\pgfpathlineto{\pgfqpoint{1.812136in}{5.169447in}}%
\pgfpathlineto{\pgfqpoint{1.812557in}{5.169698in}}%
\pgfpathlineto{\pgfqpoint{1.812979in}{5.172422in}}%
\pgfpathlineto{\pgfqpoint{1.814242in}{5.163958in}}%
\pgfpathlineto{\pgfqpoint{1.814664in}{5.166577in}}%
\pgfpathlineto{\pgfqpoint{1.815085in}{5.163392in}}%
\pgfpathlineto{\pgfqpoint{1.815506in}{5.162931in}}%
\pgfpathlineto{\pgfqpoint{1.815927in}{5.159139in}}%
\pgfpathlineto{\pgfqpoint{1.816770in}{5.159621in}}%
\pgfpathlineto{\pgfqpoint{1.817191in}{5.159537in}}%
\pgfpathlineto{\pgfqpoint{1.817613in}{5.160794in}}%
\pgfpathlineto{\pgfqpoint{1.818876in}{5.169028in}}%
\pgfpathlineto{\pgfqpoint{1.819298in}{5.168127in}}%
\pgfpathlineto{\pgfqpoint{1.820140in}{5.154006in}}%
\pgfpathlineto{\pgfqpoint{1.820562in}{5.156311in}}%
\pgfpathlineto{\pgfqpoint{1.824353in}{5.236574in}}%
\pgfpathlineto{\pgfqpoint{1.826038in}{5.261778in}}%
\pgfpathlineto{\pgfqpoint{1.826459in}{5.263287in}}%
\pgfpathlineto{\pgfqpoint{1.827723in}{5.287715in}}%
\pgfpathlineto{\pgfqpoint{1.833200in}{5.525886in}}%
\pgfpathlineto{\pgfqpoint{1.834042in}{5.516375in}}%
\pgfpathlineto{\pgfqpoint{1.834464in}{5.517359in}}%
\pgfpathlineto{\pgfqpoint{1.837413in}{5.546795in}}%
\pgfpathlineto{\pgfqpoint{1.838255in}{5.559052in}}%
\pgfpathlineto{\pgfqpoint{1.838676in}{5.559010in}}%
\pgfpathlineto{\pgfqpoint{1.839519in}{5.566825in}}%
\pgfpathlineto{\pgfqpoint{1.839940in}{5.561252in}}%
\pgfpathlineto{\pgfqpoint{1.844153in}{5.452830in}}%
\pgfpathlineto{\pgfqpoint{1.844996in}{5.436258in}}%
\pgfpathlineto{\pgfqpoint{1.847523in}{5.386583in}}%
\pgfpathlineto{\pgfqpoint{1.847945in}{5.390354in}}%
\pgfpathlineto{\pgfqpoint{1.849208in}{5.379187in}}%
\pgfpathlineto{\pgfqpoint{1.849630in}{5.381639in}}%
\pgfpathlineto{\pgfqpoint{1.850051in}{5.378831in}}%
\pgfpathlineto{\pgfqpoint{1.850472in}{5.379627in}}%
\pgfpathlineto{\pgfqpoint{1.851315in}{5.381576in}}%
\pgfpathlineto{\pgfqpoint{1.854264in}{5.332509in}}%
\pgfpathlineto{\pgfqpoint{1.855106in}{5.334625in}}%
\pgfpathlineto{\pgfqpoint{1.855528in}{5.334646in}}%
\pgfpathlineto{\pgfqpoint{1.857213in}{5.330120in}}%
\pgfpathlineto{\pgfqpoint{1.859740in}{5.338899in}}%
\pgfpathlineto{\pgfqpoint{1.860162in}{5.335274in}}%
\pgfpathlineto{\pgfqpoint{1.860583in}{5.333912in}}%
\pgfpathlineto{\pgfqpoint{1.861425in}{5.340952in}}%
\pgfpathlineto{\pgfqpoint{1.861847in}{5.339150in}}%
\pgfpathlineto{\pgfqpoint{1.863953in}{5.333137in}}%
\pgfpathlineto{\pgfqpoint{1.864374in}{5.334436in}}%
\pgfpathlineto{\pgfqpoint{1.864796in}{5.336804in}}%
\pgfpathlineto{\pgfqpoint{1.865217in}{5.336720in}}%
\pgfpathlineto{\pgfqpoint{1.865638in}{5.335169in}}%
\pgfpathlineto{\pgfqpoint{1.866060in}{5.336992in}}%
\pgfpathlineto{\pgfqpoint{1.866481in}{5.336950in}}%
\pgfpathlineto{\pgfqpoint{1.867323in}{5.332152in}}%
\pgfpathlineto{\pgfqpoint{1.868166in}{5.339988in}}%
\pgfpathlineto{\pgfqpoint{1.868587in}{5.338773in}}%
\pgfpathlineto{\pgfqpoint{1.870272in}{5.335840in}}%
\pgfpathlineto{\pgfqpoint{1.870694in}{5.334604in}}%
\pgfpathlineto{\pgfqpoint{1.871115in}{5.341161in}}%
\pgfpathlineto{\pgfqpoint{1.871536in}{5.338228in}}%
\pgfpathlineto{\pgfqpoint{1.874064in}{5.296703in}}%
\pgfpathlineto{\pgfqpoint{1.874906in}{5.304204in}}%
\pgfpathlineto{\pgfqpoint{1.875328in}{5.304120in}}%
\pgfpathlineto{\pgfqpoint{1.875749in}{5.301857in}}%
\pgfpathlineto{\pgfqpoint{1.876170in}{5.305649in}}%
\pgfpathlineto{\pgfqpoint{1.876592in}{5.303135in}}%
\pgfpathlineto{\pgfqpoint{1.877434in}{5.301794in}}%
\pgfpathlineto{\pgfqpoint{1.879540in}{5.334268in}}%
\pgfpathlineto{\pgfqpoint{1.879962in}{5.332802in}}%
\pgfpathlineto{\pgfqpoint{1.881226in}{5.364878in}}%
\pgfpathlineto{\pgfqpoint{1.881647in}{5.359619in}}%
\pgfpathlineto{\pgfqpoint{1.882068in}{5.358006in}}%
\pgfpathlineto{\pgfqpoint{1.882489in}{5.358886in}}%
\pgfpathlineto{\pgfqpoint{1.885860in}{5.393329in}}%
\pgfpathlineto{\pgfqpoint{1.887966in}{5.448849in}}%
\pgfpathlineto{\pgfqpoint{1.888387in}{5.445351in}}%
\pgfpathlineto{\pgfqpoint{1.888809in}{5.443779in}}%
\pgfpathlineto{\pgfqpoint{1.889230in}{5.445833in}}%
\pgfpathlineto{\pgfqpoint{1.890072in}{5.448766in}}%
\pgfpathlineto{\pgfqpoint{1.890915in}{5.472524in}}%
\pgfpathlineto{\pgfqpoint{1.891758in}{5.466134in}}%
\pgfpathlineto{\pgfqpoint{1.892600in}{5.452390in}}%
\pgfpathlineto{\pgfqpoint{1.893021in}{5.454255in}}%
\pgfpathlineto{\pgfqpoint{1.893864in}{5.453123in}}%
\pgfpathlineto{\pgfqpoint{1.894285in}{5.455617in}}%
\pgfpathlineto{\pgfqpoint{1.895128in}{5.454318in}}%
\pgfpathlineto{\pgfqpoint{1.897234in}{5.454234in}}%
\pgfpathlineto{\pgfqpoint{1.899762in}{5.475688in}}%
\pgfpathlineto{\pgfqpoint{1.900183in}{5.468711in}}%
\pgfpathlineto{\pgfqpoint{1.904396in}{5.413233in}}%
\pgfpathlineto{\pgfqpoint{1.904817in}{5.413023in}}%
\pgfpathlineto{\pgfqpoint{1.906924in}{5.407073in}}%
\pgfpathlineto{\pgfqpoint{1.907766in}{5.407492in}}%
\pgfpathlineto{\pgfqpoint{1.908187in}{5.405544in}}%
\pgfpathlineto{\pgfqpoint{1.909030in}{5.406047in}}%
\pgfpathlineto{\pgfqpoint{1.909451in}{5.406696in}}%
\pgfpathlineto{\pgfqpoint{1.909872in}{5.405523in}}%
\pgfpathlineto{\pgfqpoint{1.910294in}{5.405481in}}%
\pgfpathlineto{\pgfqpoint{1.910715in}{5.407450in}}%
\pgfpathlineto{\pgfqpoint{1.911136in}{5.407304in}}%
\pgfpathlineto{\pgfqpoint{1.913243in}{5.387170in}}%
\pgfpathlineto{\pgfqpoint{1.913664in}{5.392764in}}%
\pgfpathlineto{\pgfqpoint{1.916192in}{5.423708in}}%
\pgfpathlineto{\pgfqpoint{1.917034in}{5.431355in}}%
\pgfpathlineto{\pgfqpoint{1.917877in}{5.444764in}}%
\pgfpathlineto{\pgfqpoint{1.918298in}{5.444240in}}%
\pgfpathlineto{\pgfqpoint{1.918719in}{5.443193in}}%
\pgfpathlineto{\pgfqpoint{1.919141in}{5.444219in}}%
\pgfpathlineto{\pgfqpoint{1.919562in}{5.446880in}}%
\pgfpathlineto{\pgfqpoint{1.920404in}{5.445979in}}%
\pgfpathlineto{\pgfqpoint{1.921247in}{5.448221in}}%
\pgfpathlineto{\pgfqpoint{1.921668in}{5.447278in}}%
\pgfpathlineto{\pgfqpoint{1.923775in}{5.446922in}}%
\pgfpathlineto{\pgfqpoint{1.926302in}{5.456036in}}%
\pgfpathlineto{\pgfqpoint{1.929251in}{5.424023in}}%
\pgfpathlineto{\pgfqpoint{1.929673in}{5.422933in}}%
\pgfpathlineto{\pgfqpoint{1.930936in}{5.412981in}}%
\pgfpathlineto{\pgfqpoint{1.931358in}{5.413819in}}%
\pgfpathlineto{\pgfqpoint{1.933464in}{5.408980in}}%
\pgfpathlineto{\pgfqpoint{1.934728in}{5.372336in}}%
\pgfpathlineto{\pgfqpoint{1.937256in}{5.208604in}}%
\pgfpathlineto{\pgfqpoint{1.940205in}{5.049440in}}%
\pgfpathlineto{\pgfqpoint{1.942311in}{5.037540in}}%
\pgfpathlineto{\pgfqpoint{1.942732in}{5.041415in}}%
\pgfpathlineto{\pgfqpoint{1.943153in}{5.035738in}}%
\pgfpathlineto{\pgfqpoint{1.943996in}{5.031694in}}%
\pgfpathlineto{\pgfqpoint{1.944839in}{5.043846in}}%
\pgfpathlineto{\pgfqpoint{1.945681in}{5.043301in}}%
\pgfpathlineto{\pgfqpoint{1.946524in}{5.043615in}}%
\pgfpathlineto{\pgfqpoint{1.948209in}{5.036136in}}%
\pgfpathlineto{\pgfqpoint{1.948630in}{5.036387in}}%
\pgfpathlineto{\pgfqpoint{1.949894in}{5.031652in}}%
\pgfpathlineto{\pgfqpoint{1.952000in}{5.035528in}}%
\pgfpathlineto{\pgfqpoint{1.952843in}{5.047743in}}%
\pgfpathlineto{\pgfqpoint{1.953264in}{5.045459in}}%
\pgfpathlineto{\pgfqpoint{1.953685in}{5.045459in}}%
\pgfpathlineto{\pgfqpoint{1.954528in}{5.048036in}}%
\pgfpathlineto{\pgfqpoint{1.955371in}{5.045543in}}%
\pgfpathlineto{\pgfqpoint{1.955792in}{5.047114in}}%
\pgfpathlineto{\pgfqpoint{1.956213in}{5.047575in}}%
\pgfpathlineto{\pgfqpoint{1.957056in}{5.044705in}}%
\pgfpathlineto{\pgfqpoint{1.957898in}{5.052394in}}%
\pgfpathlineto{\pgfqpoint{1.958319in}{5.049084in}}%
\pgfpathlineto{\pgfqpoint{1.962954in}{5.029327in}}%
\pgfpathlineto{\pgfqpoint{1.963796in}{5.029159in}}%
\pgfpathlineto{\pgfqpoint{1.964639in}{5.026456in}}%
\pgfpathlineto{\pgfqpoint{1.965060in}{5.027211in}}%
\pgfpathlineto{\pgfqpoint{1.965481in}{5.027693in}}%
\pgfpathlineto{\pgfqpoint{1.966324in}{4.987697in}}%
\pgfpathlineto{\pgfqpoint{1.972222in}{4.602639in}}%
\pgfpathlineto{\pgfqpoint{1.975592in}{4.425687in}}%
\pgfpathlineto{\pgfqpoint{1.977277in}{4.424953in}}%
\pgfpathlineto{\pgfqpoint{1.977698in}{4.425729in}}%
\pgfpathlineto{\pgfqpoint{1.978120in}{4.426252in}}%
\pgfpathlineto{\pgfqpoint{1.978541in}{4.425435in}}%
\pgfpathlineto{\pgfqpoint{1.980647in}{4.423843in}}%
\pgfpathlineto{\pgfqpoint{1.982332in}{4.424157in}}%
\pgfpathlineto{\pgfqpoint{1.985281in}{4.421936in}}%
\pgfpathlineto{\pgfqpoint{1.986545in}{4.423382in}}%
\pgfpathlineto{\pgfqpoint{1.986966in}{4.424765in}}%
\pgfpathlineto{\pgfqpoint{1.997920in}{4.999493in}}%
\pgfpathlineto{\pgfqpoint{2.000447in}{5.161423in}}%
\pgfpathlineto{\pgfqpoint{2.002975in}{5.198234in}}%
\pgfpathlineto{\pgfqpoint{2.005503in}{5.237265in}}%
\pgfpathlineto{\pgfqpoint{2.006766in}{5.229115in}}%
\pgfpathlineto{\pgfqpoint{2.007188in}{5.229199in}}%
\pgfpathlineto{\pgfqpoint{2.008873in}{5.229178in}}%
\pgfpathlineto{\pgfqpoint{2.010137in}{5.204812in}}%
\pgfpathlineto{\pgfqpoint{2.013086in}{5.028887in}}%
\pgfpathlineto{\pgfqpoint{2.014771in}{4.990295in}}%
\pgfpathlineto{\pgfqpoint{2.017720in}{4.939573in}}%
\pgfpathlineto{\pgfqpoint{2.018984in}{4.918957in}}%
\pgfpathlineto{\pgfqpoint{2.019405in}{4.923545in}}%
\pgfpathlineto{\pgfqpoint{2.020669in}{4.928511in}}%
\pgfpathlineto{\pgfqpoint{2.022354in}{4.928720in}}%
\pgfpathlineto{\pgfqpoint{2.023618in}{4.952856in}}%
\pgfpathlineto{\pgfqpoint{2.029515in}{5.172527in}}%
\pgfpathlineto{\pgfqpoint{2.032464in}{5.223417in}}%
\pgfpathlineto{\pgfqpoint{2.032886in}{5.218829in}}%
\pgfpathlineto{\pgfqpoint{2.034150in}{5.213716in}}%
\pgfpathlineto{\pgfqpoint{2.038784in}{5.212376in}}%
\pgfpathlineto{\pgfqpoint{2.042996in}{5.211496in}}%
\pgfpathlineto{\pgfqpoint{2.044260in}{5.211077in}}%
\pgfpathlineto{\pgfqpoint{2.044681in}{5.208961in}}%
\pgfpathlineto{\pgfqpoint{2.046367in}{5.154718in}}%
\pgfpathlineto{\pgfqpoint{2.059005in}{4.505992in}}%
\pgfpathlineto{\pgfqpoint{2.059426in}{4.510706in}}%
\pgfpathlineto{\pgfqpoint{2.061533in}{4.580158in}}%
\pgfpathlineto{\pgfqpoint{2.075435in}{5.079902in}}%
\pgfpathlineto{\pgfqpoint{2.081333in}{5.188827in}}%
\pgfpathlineto{\pgfqpoint{2.081754in}{5.187863in}}%
\pgfpathlineto{\pgfqpoint{2.082597in}{5.189246in}}%
\pgfpathlineto{\pgfqpoint{2.083018in}{5.187842in}}%
\pgfpathlineto{\pgfqpoint{2.085967in}{5.126246in}}%
\pgfpathlineto{\pgfqpoint{2.090180in}{5.038650in}}%
\pgfpathlineto{\pgfqpoint{2.095656in}{4.942171in}}%
\pgfpathlineto{\pgfqpoint{2.096077in}{4.943700in}}%
\pgfpathlineto{\pgfqpoint{2.097341in}{4.953673in}}%
\pgfpathlineto{\pgfqpoint{2.101975in}{5.052708in}}%
\pgfpathlineto{\pgfqpoint{2.110822in}{5.194882in}}%
\pgfpathlineto{\pgfqpoint{2.114192in}{5.195908in}}%
\pgfpathlineto{\pgfqpoint{2.115456in}{5.204666in}}%
\pgfpathlineto{\pgfqpoint{2.118826in}{5.179881in}}%
\pgfpathlineto{\pgfqpoint{2.120933in}{5.154509in}}%
\pgfpathlineto{\pgfqpoint{2.121354in}{5.159810in}}%
\pgfpathlineto{\pgfqpoint{2.122197in}{5.162491in}}%
\pgfpathlineto{\pgfqpoint{2.123460in}{5.153587in}}%
\pgfpathlineto{\pgfqpoint{2.123882in}{5.153943in}}%
\pgfpathlineto{\pgfqpoint{2.125146in}{5.154006in}}%
\pgfpathlineto{\pgfqpoint{2.126831in}{5.156374in}}%
\pgfpathlineto{\pgfqpoint{2.127252in}{5.157924in}}%
\pgfpathlineto{\pgfqpoint{2.127673in}{5.156164in}}%
\pgfpathlineto{\pgfqpoint{2.128095in}{5.155117in}}%
\pgfpathlineto{\pgfqpoint{2.128516in}{5.156101in}}%
\pgfpathlineto{\pgfqpoint{2.130201in}{5.160752in}}%
\pgfpathlineto{\pgfqpoint{2.132307in}{5.161444in}}%
\pgfpathlineto{\pgfqpoint{2.133992in}{5.163183in}}%
\pgfpathlineto{\pgfqpoint{2.134835in}{5.159474in}}%
\pgfpathlineto{\pgfqpoint{2.136099in}{5.142043in}}%
\pgfpathlineto{\pgfqpoint{2.136520in}{5.146233in}}%
\pgfpathlineto{\pgfqpoint{2.139048in}{5.158343in}}%
\pgfpathlineto{\pgfqpoint{2.139469in}{5.157714in}}%
\pgfpathlineto{\pgfqpoint{2.142418in}{5.214429in}}%
\pgfpathlineto{\pgfqpoint{2.144524in}{5.238397in}}%
\pgfpathlineto{\pgfqpoint{2.146631in}{5.152498in}}%
\pgfpathlineto{\pgfqpoint{2.147052in}{5.131400in}}%
\pgfpathlineto{\pgfqpoint{2.147473in}{5.139319in}}%
\pgfpathlineto{\pgfqpoint{2.149580in}{5.198066in}}%
\pgfpathlineto{\pgfqpoint{2.150001in}{5.194148in}}%
\pgfpathlineto{\pgfqpoint{2.151265in}{5.173218in}}%
\pgfpathlineto{\pgfqpoint{2.153371in}{5.061130in}}%
\pgfpathlineto{\pgfqpoint{2.154214in}{5.070558in}}%
\pgfpathlineto{\pgfqpoint{2.154635in}{5.075440in}}%
\pgfpathlineto{\pgfqpoint{2.156741in}{5.046025in}}%
\pgfpathlineto{\pgfqpoint{2.157163in}{5.047324in}}%
\pgfpathlineto{\pgfqpoint{2.160533in}{5.080971in}}%
\pgfpathlineto{\pgfqpoint{2.165167in}{4.892831in}}%
\pgfpathlineto{\pgfqpoint{2.165588in}{4.898886in}}%
\pgfpathlineto{\pgfqpoint{2.166852in}{4.913468in}}%
\pgfpathlineto{\pgfqpoint{2.174856in}{4.583741in}}%
\pgfpathlineto{\pgfqpoint{2.175699in}{4.570039in}}%
\pgfpathlineto{\pgfqpoint{2.176120in}{4.578566in}}%
\pgfpathlineto{\pgfqpoint{2.180754in}{4.759918in}}%
\pgfpathlineto{\pgfqpoint{2.182018in}{4.823169in}}%
\pgfpathlineto{\pgfqpoint{2.192129in}{5.405544in}}%
\pgfpathlineto{\pgfqpoint{2.195078in}{5.470995in}}%
\pgfpathlineto{\pgfqpoint{2.195499in}{5.464060in}}%
\pgfpathlineto{\pgfqpoint{2.196342in}{5.461881in}}%
\pgfpathlineto{\pgfqpoint{2.196763in}{5.461986in}}%
\pgfpathlineto{\pgfqpoint{2.201397in}{5.462237in}}%
\pgfpathlineto{\pgfqpoint{2.203925in}{5.462321in}}%
\pgfpathlineto{\pgfqpoint{2.207295in}{5.461588in}}%
\pgfpathlineto{\pgfqpoint{2.208980in}{5.461797in}}%
\pgfpathlineto{\pgfqpoint{2.209401in}{5.459304in}}%
\pgfpathlineto{\pgfqpoint{2.211086in}{5.396074in}}%
\pgfpathlineto{\pgfqpoint{2.224146in}{4.655163in}}%
\pgfpathlineto{\pgfqpoint{2.224989in}{4.661532in}}%
\pgfpathlineto{\pgfqpoint{2.231308in}{4.837395in}}%
\pgfpathlineto{\pgfqpoint{2.239733in}{5.039446in}}%
\pgfpathlineto{\pgfqpoint{2.242261in}{5.063414in}}%
\pgfpathlineto{\pgfqpoint{2.243946in}{5.032323in}}%
\pgfpathlineto{\pgfqpoint{2.254478in}{4.784745in}}%
\pgfpathlineto{\pgfqpoint{2.255742in}{4.781016in}}%
\pgfpathlineto{\pgfqpoint{2.257427in}{4.812295in}}%
\pgfpathlineto{\pgfqpoint{2.265431in}{5.038357in}}%
\pgfpathlineto{\pgfqpoint{2.267959in}{5.110344in}}%
\pgfpathlineto{\pgfqpoint{2.270065in}{5.131526in}}%
\pgfpathlineto{\pgfqpoint{2.270487in}{5.130436in}}%
\pgfpathlineto{\pgfqpoint{2.272172in}{5.114912in}}%
\pgfpathlineto{\pgfqpoint{2.272593in}{5.103808in}}%
\pgfpathlineto{\pgfqpoint{2.273436in}{5.104583in}}%
\pgfpathlineto{\pgfqpoint{2.273857in}{5.104206in}}%
\pgfpathlineto{\pgfqpoint{2.275963in}{5.088073in}}%
\pgfpathlineto{\pgfqpoint{2.277227in}{5.082228in}}%
\pgfpathlineto{\pgfqpoint{2.278070in}{5.077200in}}%
\pgfpathlineto{\pgfqpoint{2.278491in}{5.077849in}}%
\pgfpathlineto{\pgfqpoint{2.278912in}{5.077200in}}%
\pgfpathlineto{\pgfqpoint{2.281019in}{5.058323in}}%
\pgfpathlineto{\pgfqpoint{2.281440in}{5.059433in}}%
\pgfpathlineto{\pgfqpoint{2.284389in}{5.045941in}}%
\pgfpathlineto{\pgfqpoint{2.286074in}{5.059182in}}%
\pgfpathlineto{\pgfqpoint{2.297870in}{5.248034in}}%
\pgfpathlineto{\pgfqpoint{2.299133in}{5.245101in}}%
\pgfpathlineto{\pgfqpoint{2.301240in}{5.272505in}}%
\pgfpathlineto{\pgfqpoint{2.301661in}{5.273175in}}%
\pgfpathlineto{\pgfqpoint{2.303346in}{5.243949in}}%
\pgfpathlineto{\pgfqpoint{2.304610in}{5.204812in}}%
\pgfpathlineto{\pgfqpoint{2.307138in}{5.143950in}}%
\pgfpathlineto{\pgfqpoint{2.307559in}{5.142399in}}%
\pgfpathlineto{\pgfqpoint{2.308823in}{5.147490in}}%
\pgfpathlineto{\pgfqpoint{2.309244in}{5.146883in}}%
\pgfpathlineto{\pgfqpoint{2.309665in}{5.148706in}}%
\pgfpathlineto{\pgfqpoint{2.310087in}{5.146757in}}%
\pgfpathlineto{\pgfqpoint{2.310508in}{5.146841in}}%
\pgfpathlineto{\pgfqpoint{2.311772in}{5.144683in}}%
\pgfpathlineto{\pgfqpoint{2.312614in}{5.146212in}}%
\pgfpathlineto{\pgfqpoint{2.313036in}{5.149711in}}%
\pgfpathlineto{\pgfqpoint{2.313878in}{5.149187in}}%
\pgfpathlineto{\pgfqpoint{2.314721in}{5.150633in}}%
\pgfpathlineto{\pgfqpoint{2.315563in}{5.147847in}}%
\pgfpathlineto{\pgfqpoint{2.315985in}{5.148224in}}%
\pgfpathlineto{\pgfqpoint{2.317248in}{5.148014in}}%
\pgfpathlineto{\pgfqpoint{2.318512in}{5.145270in}}%
\pgfpathlineto{\pgfqpoint{2.318934in}{5.145249in}}%
\pgfpathlineto{\pgfqpoint{2.321461in}{5.152204in}}%
\pgfpathlineto{\pgfqpoint{2.322725in}{5.150235in}}%
\pgfpathlineto{\pgfqpoint{2.323989in}{5.151597in}}%
\pgfpathlineto{\pgfqpoint{2.324410in}{5.148852in}}%
\pgfpathlineto{\pgfqpoint{2.325253in}{5.149900in}}%
\pgfpathlineto{\pgfqpoint{2.325674in}{5.150088in}}%
\pgfpathlineto{\pgfqpoint{2.328202in}{5.156855in}}%
\pgfpathlineto{\pgfqpoint{2.329044in}{5.154278in}}%
\pgfpathlineto{\pgfqpoint{2.329465in}{5.154509in}}%
\pgfpathlineto{\pgfqpoint{2.329887in}{5.156499in}}%
\pgfpathlineto{\pgfqpoint{2.330729in}{5.155619in}}%
\pgfpathlineto{\pgfqpoint{2.331993in}{5.153440in}}%
\pgfpathlineto{\pgfqpoint{2.332414in}{5.153336in}}%
\pgfpathlineto{\pgfqpoint{2.333257in}{5.157589in}}%
\pgfpathlineto{\pgfqpoint{2.334521in}{5.156248in}}%
\pgfpathlineto{\pgfqpoint{2.334942in}{5.156290in}}%
\pgfpathlineto{\pgfqpoint{2.335785in}{5.153084in}}%
\pgfpathlineto{\pgfqpoint{2.336627in}{5.153503in}}%
\pgfpathlineto{\pgfqpoint{2.339155in}{5.148391in}}%
\pgfpathlineto{\pgfqpoint{2.339576in}{5.160459in}}%
\pgfpathlineto{\pgfqpoint{2.340419in}{5.155934in}}%
\pgfpathlineto{\pgfqpoint{2.342104in}{5.142588in}}%
\pgfpathlineto{\pgfqpoint{2.342525in}{5.145060in}}%
\pgfpathlineto{\pgfqpoint{2.346317in}{5.238816in}}%
\pgfpathlineto{\pgfqpoint{2.347159in}{5.230792in}}%
\pgfpathlineto{\pgfqpoint{2.348844in}{5.218954in}}%
\pgfpathlineto{\pgfqpoint{2.349266in}{5.223563in}}%
\pgfpathlineto{\pgfqpoint{2.354321in}{5.311327in}}%
\pgfpathlineto{\pgfqpoint{2.355585in}{5.306320in}}%
\pgfpathlineto{\pgfqpoint{2.356006in}{5.311369in}}%
\pgfpathlineto{\pgfqpoint{2.356427in}{5.307786in}}%
\pgfpathlineto{\pgfqpoint{2.356849in}{5.306299in}}%
\pgfpathlineto{\pgfqpoint{2.357270in}{5.306865in}}%
\pgfpathlineto{\pgfqpoint{2.357691in}{5.311725in}}%
\pgfpathlineto{\pgfqpoint{2.358112in}{5.307409in}}%
\pgfpathlineto{\pgfqpoint{2.358534in}{5.306341in}}%
\pgfpathlineto{\pgfqpoint{2.358955in}{5.307493in}}%
\pgfpathlineto{\pgfqpoint{2.359376in}{5.311935in}}%
\pgfpathlineto{\pgfqpoint{2.359798in}{5.307598in}}%
\pgfpathlineto{\pgfqpoint{2.360219in}{5.306341in}}%
\pgfpathlineto{\pgfqpoint{2.361061in}{5.310133in}}%
\pgfpathlineto{\pgfqpoint{2.361904in}{5.301585in}}%
\pgfpathlineto{\pgfqpoint{2.362325in}{5.305796in}}%
\pgfpathlineto{\pgfqpoint{2.363168in}{5.298945in}}%
\pgfpathlineto{\pgfqpoint{2.363589in}{5.297185in}}%
\pgfpathlineto{\pgfqpoint{2.364010in}{5.302193in}}%
\pgfpathlineto{\pgfqpoint{2.364432in}{5.297960in}}%
\pgfpathlineto{\pgfqpoint{2.365274in}{5.289203in}}%
\pgfpathlineto{\pgfqpoint{2.366538in}{5.305838in}}%
\pgfpathlineto{\pgfqpoint{2.367381in}{5.306592in}}%
\pgfpathlineto{\pgfqpoint{2.368223in}{5.302276in}}%
\pgfpathlineto{\pgfqpoint{2.368644in}{5.303135in}}%
\pgfpathlineto{\pgfqpoint{2.369066in}{5.308415in}}%
\pgfpathlineto{\pgfqpoint{2.369487in}{5.304183in}}%
\pgfpathlineto{\pgfqpoint{2.369908in}{5.303491in}}%
\pgfpathlineto{\pgfqpoint{2.370751in}{5.308897in}}%
\pgfpathlineto{\pgfqpoint{2.371172in}{5.304895in}}%
\pgfpathlineto{\pgfqpoint{2.371593in}{5.304853in}}%
\pgfpathlineto{\pgfqpoint{2.372436in}{5.310196in}}%
\pgfpathlineto{\pgfqpoint{2.373278in}{5.306152in}}%
\pgfpathlineto{\pgfqpoint{2.373700in}{5.311537in}}%
\pgfpathlineto{\pgfqpoint{2.374542in}{5.309504in}}%
\pgfpathlineto{\pgfqpoint{2.374964in}{5.311055in}}%
\pgfpathlineto{\pgfqpoint{2.375385in}{5.317948in}}%
\pgfpathlineto{\pgfqpoint{2.376227in}{5.315140in}}%
\pgfpathlineto{\pgfqpoint{2.378755in}{5.333221in}}%
\pgfpathlineto{\pgfqpoint{2.380019in}{5.313234in}}%
\pgfpathlineto{\pgfqpoint{2.380440in}{5.316712in}}%
\pgfpathlineto{\pgfqpoint{2.381283in}{5.311118in}}%
\pgfpathlineto{\pgfqpoint{2.381704in}{5.314721in}}%
\pgfpathlineto{\pgfqpoint{2.382125in}{5.315706in}}%
\pgfpathlineto{\pgfqpoint{2.382968in}{5.310405in}}%
\pgfpathlineto{\pgfqpoint{2.383389in}{5.315350in}}%
\pgfpathlineto{\pgfqpoint{2.384232in}{5.312228in}}%
\pgfpathlineto{\pgfqpoint{2.384653in}{5.311725in}}%
\pgfpathlineto{\pgfqpoint{2.385074in}{5.316334in}}%
\pgfpathlineto{\pgfqpoint{2.385495in}{5.312940in}}%
\pgfpathlineto{\pgfqpoint{2.386338in}{5.310866in}}%
\pgfpathlineto{\pgfqpoint{2.386759in}{5.315224in}}%
\pgfpathlineto{\pgfqpoint{2.387181in}{5.308897in}}%
\pgfpathlineto{\pgfqpoint{2.387602in}{5.302842in}}%
\pgfpathlineto{\pgfqpoint{2.388023in}{5.304162in}}%
\pgfpathlineto{\pgfqpoint{2.388444in}{5.315266in}}%
\pgfpathlineto{\pgfqpoint{2.389287in}{5.315161in}}%
\pgfpathlineto{\pgfqpoint{2.390130in}{5.319624in}}%
\pgfpathlineto{\pgfqpoint{2.390972in}{5.307137in}}%
\pgfpathlineto{\pgfqpoint{2.391393in}{5.313737in}}%
\pgfpathlineto{\pgfqpoint{2.393078in}{5.323144in}}%
\pgfpathlineto{\pgfqpoint{2.396027in}{5.310929in}}%
\pgfpathlineto{\pgfqpoint{2.396449in}{5.315999in}}%
\pgfpathlineto{\pgfqpoint{2.396870in}{5.309923in}}%
\pgfpathlineto{\pgfqpoint{2.397713in}{5.305587in}}%
\pgfpathlineto{\pgfqpoint{2.398134in}{5.310175in}}%
\pgfpathlineto{\pgfqpoint{2.398555in}{5.304120in}}%
\pgfpathlineto{\pgfqpoint{2.398976in}{5.301166in}}%
\pgfpathlineto{\pgfqpoint{2.399398in}{5.302109in}}%
\pgfpathlineto{\pgfqpoint{2.399819in}{5.305649in}}%
\pgfpathlineto{\pgfqpoint{2.400662in}{5.298903in}}%
\pgfpathlineto{\pgfqpoint{2.401083in}{5.306257in}}%
\pgfpathlineto{\pgfqpoint{2.401504in}{5.305587in}}%
\pgfpathlineto{\pgfqpoint{2.403189in}{5.287506in}}%
\pgfpathlineto{\pgfqpoint{2.403610in}{5.283190in}}%
\pgfpathlineto{\pgfqpoint{2.404032in}{5.283860in}}%
\pgfpathlineto{\pgfqpoint{2.404453in}{5.292932in}}%
\pgfpathlineto{\pgfqpoint{2.404874in}{5.286102in}}%
\pgfpathlineto{\pgfqpoint{2.406559in}{5.272065in}}%
\pgfpathlineto{\pgfqpoint{2.406981in}{5.269865in}}%
\pgfpathlineto{\pgfqpoint{2.407402in}{5.271562in}}%
\pgfpathlineto{\pgfqpoint{2.407823in}{5.277449in}}%
\pgfpathlineto{\pgfqpoint{2.408245in}{5.271017in}}%
\pgfpathlineto{\pgfqpoint{2.408666in}{5.269258in}}%
\pgfpathlineto{\pgfqpoint{2.409087in}{5.271332in}}%
\pgfpathlineto{\pgfqpoint{2.409508in}{5.276276in}}%
\pgfpathlineto{\pgfqpoint{2.409930in}{5.269781in}}%
\pgfpathlineto{\pgfqpoint{2.410351in}{5.268503in}}%
\pgfpathlineto{\pgfqpoint{2.411193in}{5.274139in}}%
\pgfpathlineto{\pgfqpoint{2.412036in}{5.267330in}}%
\pgfpathlineto{\pgfqpoint{2.412457in}{5.272966in}}%
\pgfpathlineto{\pgfqpoint{2.412879in}{5.271939in}}%
\pgfpathlineto{\pgfqpoint{2.413721in}{5.265989in}}%
\pgfpathlineto{\pgfqpoint{2.414142in}{5.272673in}}%
\pgfpathlineto{\pgfqpoint{2.414564in}{5.269090in}}%
\pgfpathlineto{\pgfqpoint{2.415406in}{5.265214in}}%
\pgfpathlineto{\pgfqpoint{2.415828in}{5.271227in}}%
\pgfpathlineto{\pgfqpoint{2.416249in}{5.264837in}}%
\pgfpathlineto{\pgfqpoint{2.416670in}{5.259809in}}%
\pgfpathlineto{\pgfqpoint{2.417091in}{5.264334in}}%
\pgfpathlineto{\pgfqpoint{2.419198in}{5.294985in}}%
\pgfpathlineto{\pgfqpoint{2.419619in}{5.294105in}}%
\pgfpathlineto{\pgfqpoint{2.420462in}{5.288700in}}%
\pgfpathlineto{\pgfqpoint{2.422147in}{5.214366in}}%
\pgfpathlineto{\pgfqpoint{2.430994in}{4.719126in}}%
\pgfpathlineto{\pgfqpoint{2.432679in}{4.718875in}}%
\pgfpathlineto{\pgfqpoint{2.435628in}{4.726313in}}%
\pgfpathlineto{\pgfqpoint{2.438577in}{4.725328in}}%
\pgfpathlineto{\pgfqpoint{2.443211in}{4.725035in}}%
\pgfpathlineto{\pgfqpoint{2.443632in}{4.721640in}}%
\pgfpathlineto{\pgfqpoint{2.449108in}{4.419318in}}%
\pgfpathlineto{\pgfqpoint{2.450372in}{4.419380in}}%
\pgfpathlineto{\pgfqpoint{2.452900in}{4.419443in}}%
\pgfpathlineto{\pgfqpoint{2.460904in}{4.421957in}}%
\pgfpathlineto{\pgfqpoint{2.461747in}{4.446386in}}%
\pgfpathlineto{\pgfqpoint{2.471858in}{4.910325in}}%
\pgfpathlineto{\pgfqpoint{2.476913in}{5.078017in}}%
\pgfpathlineto{\pgfqpoint{2.483653in}{5.181138in}}%
\pgfpathlineto{\pgfqpoint{2.484917in}{5.170306in}}%
\pgfpathlineto{\pgfqpoint{2.485338in}{5.171123in}}%
\pgfpathlineto{\pgfqpoint{2.485760in}{5.172359in}}%
\pgfpathlineto{\pgfqpoint{2.486181in}{5.171856in}}%
\pgfpathlineto{\pgfqpoint{2.487866in}{5.148622in}}%
\pgfpathlineto{\pgfqpoint{2.489972in}{5.100434in}}%
\pgfpathlineto{\pgfqpoint{2.492079in}{5.063686in}}%
\pgfpathlineto{\pgfqpoint{2.492921in}{5.039237in}}%
\pgfpathlineto{\pgfqpoint{2.495449in}{4.960880in}}%
\pgfpathlineto{\pgfqpoint{2.497977in}{4.915332in}}%
\pgfpathlineto{\pgfqpoint{2.498398in}{4.915018in}}%
\pgfpathlineto{\pgfqpoint{2.502611in}{4.962954in}}%
\pgfpathlineto{\pgfqpoint{2.503453in}{4.992537in}}%
\pgfpathlineto{\pgfqpoint{2.505138in}{5.034963in}}%
\pgfpathlineto{\pgfqpoint{2.505981in}{5.037749in}}%
\pgfpathlineto{\pgfqpoint{2.509351in}{5.061780in}}%
\pgfpathlineto{\pgfqpoint{2.512721in}{5.114702in}}%
\pgfpathlineto{\pgfqpoint{2.516513in}{5.159055in}}%
\pgfpathlineto{\pgfqpoint{2.516934in}{5.147260in}}%
\pgfpathlineto{\pgfqpoint{2.517777in}{5.128425in}}%
\pgfpathlineto{\pgfqpoint{2.518198in}{5.129305in}}%
\pgfpathlineto{\pgfqpoint{2.521147in}{5.154802in}}%
\pgfpathlineto{\pgfqpoint{2.523675in}{5.204582in}}%
\pgfpathlineto{\pgfqpoint{2.527466in}{5.276653in}}%
\pgfpathlineto{\pgfqpoint{2.528730in}{5.306467in}}%
\pgfpathlineto{\pgfqpoint{2.529573in}{5.300307in}}%
\pgfpathlineto{\pgfqpoint{2.529994in}{5.304728in}}%
\pgfpathlineto{\pgfqpoint{2.531258in}{5.326265in}}%
\pgfpathlineto{\pgfqpoint{2.531679in}{5.323437in}}%
\pgfpathlineto{\pgfqpoint{2.532522in}{5.315287in}}%
\pgfpathlineto{\pgfqpoint{2.532943in}{5.315894in}}%
\pgfpathlineto{\pgfqpoint{2.534207in}{5.322515in}}%
\pgfpathlineto{\pgfqpoint{2.534628in}{5.320189in}}%
\pgfpathlineto{\pgfqpoint{2.539262in}{5.262909in}}%
\pgfpathlineto{\pgfqpoint{2.539683in}{5.264083in}}%
\pgfpathlineto{\pgfqpoint{2.540105in}{5.262302in}}%
\pgfpathlineto{\pgfqpoint{2.544739in}{5.158238in}}%
\pgfpathlineto{\pgfqpoint{2.551058in}{5.250402in}}%
\pgfpathlineto{\pgfqpoint{2.552743in}{5.304707in}}%
\pgfpathlineto{\pgfqpoint{2.553585in}{5.307367in}}%
\pgfpathlineto{\pgfqpoint{2.555271in}{5.334310in}}%
\pgfpathlineto{\pgfqpoint{2.555692in}{5.330057in}}%
\pgfpathlineto{\pgfqpoint{2.556534in}{5.324547in}}%
\pgfpathlineto{\pgfqpoint{2.558220in}{5.354256in}}%
\pgfpathlineto{\pgfqpoint{2.558641in}{5.350296in}}%
\pgfpathlineto{\pgfqpoint{2.562432in}{5.297814in}}%
\pgfpathlineto{\pgfqpoint{2.562854in}{5.298128in}}%
\pgfpathlineto{\pgfqpoint{2.563696in}{5.312961in}}%
\pgfpathlineto{\pgfqpoint{2.564117in}{5.310845in}}%
\pgfpathlineto{\pgfqpoint{2.567066in}{5.231169in}}%
\pgfpathlineto{\pgfqpoint{2.568751in}{5.202571in}}%
\pgfpathlineto{\pgfqpoint{2.569173in}{5.209359in}}%
\pgfpathlineto{\pgfqpoint{2.570437in}{5.219248in}}%
\pgfpathlineto{\pgfqpoint{2.571279in}{5.210155in}}%
\pgfpathlineto{\pgfqpoint{2.571700in}{5.214534in}}%
\pgfpathlineto{\pgfqpoint{2.573807in}{5.244054in}}%
\pgfpathlineto{\pgfqpoint{2.574228in}{5.245248in}}%
\pgfpathlineto{\pgfqpoint{2.575071in}{5.275669in}}%
\pgfpathlineto{\pgfqpoint{2.576334in}{5.314742in}}%
\pgfpathlineto{\pgfqpoint{2.576756in}{5.309043in}}%
\pgfpathlineto{\pgfqpoint{2.577177in}{5.303491in}}%
\pgfpathlineto{\pgfqpoint{2.577598in}{5.305880in}}%
\pgfpathlineto{\pgfqpoint{2.579283in}{5.329135in}}%
\pgfpathlineto{\pgfqpoint{2.579705in}{5.324338in}}%
\pgfpathlineto{\pgfqpoint{2.580126in}{5.320231in}}%
\pgfpathlineto{\pgfqpoint{2.580547in}{5.320399in}}%
\pgfpathlineto{\pgfqpoint{2.582232in}{5.341811in}}%
\pgfpathlineto{\pgfqpoint{2.582654in}{5.337034in}}%
\pgfpathlineto{\pgfqpoint{2.586866in}{5.283714in}}%
\pgfpathlineto{\pgfqpoint{2.587288in}{5.286919in}}%
\pgfpathlineto{\pgfqpoint{2.588130in}{5.309358in}}%
\pgfpathlineto{\pgfqpoint{2.588552in}{5.296829in}}%
\pgfpathlineto{\pgfqpoint{2.589815in}{5.283714in}}%
\pgfpathlineto{\pgfqpoint{2.590237in}{5.283211in}}%
\pgfpathlineto{\pgfqpoint{2.592764in}{5.341476in}}%
\pgfpathlineto{\pgfqpoint{2.598662in}{5.441579in}}%
\pgfpathlineto{\pgfqpoint{2.600347in}{5.462698in}}%
\pgfpathlineto{\pgfqpoint{2.601190in}{5.476630in}}%
\pgfpathlineto{\pgfqpoint{2.601611in}{5.472587in}}%
\pgfpathlineto{\pgfqpoint{2.602032in}{5.478244in}}%
\pgfpathlineto{\pgfqpoint{2.602875in}{5.475562in}}%
\pgfpathlineto{\pgfqpoint{2.603296in}{5.475772in}}%
\pgfpathlineto{\pgfqpoint{2.603718in}{5.485639in}}%
\pgfpathlineto{\pgfqpoint{2.604139in}{5.484927in}}%
\pgfpathlineto{\pgfqpoint{2.605824in}{5.469884in}}%
\pgfpathlineto{\pgfqpoint{2.606667in}{5.469402in}}%
\pgfpathlineto{\pgfqpoint{2.607509in}{5.475981in}}%
\pgfpathlineto{\pgfqpoint{2.607930in}{5.474389in}}%
\pgfpathlineto{\pgfqpoint{2.609615in}{5.472692in}}%
\pgfpathlineto{\pgfqpoint{2.610037in}{5.472671in}}%
\pgfpathlineto{\pgfqpoint{2.610458in}{5.475122in}}%
\pgfpathlineto{\pgfqpoint{2.610879in}{5.474431in}}%
\pgfpathlineto{\pgfqpoint{2.611722in}{5.469842in}}%
\pgfpathlineto{\pgfqpoint{2.612564in}{5.470555in}}%
\pgfpathlineto{\pgfqpoint{2.612986in}{5.470764in}}%
\pgfpathlineto{\pgfqpoint{2.613828in}{5.487253in}}%
\pgfpathlineto{\pgfqpoint{2.614250in}{5.486163in}}%
\pgfpathlineto{\pgfqpoint{2.615935in}{5.474724in}}%
\pgfpathlineto{\pgfqpoint{2.616356in}{5.474263in}}%
\pgfpathlineto{\pgfqpoint{2.617198in}{5.486415in}}%
\pgfpathlineto{\pgfqpoint{2.617620in}{5.482455in}}%
\pgfpathlineto{\pgfqpoint{2.619305in}{5.473027in}}%
\pgfpathlineto{\pgfqpoint{2.619726in}{5.472461in}}%
\pgfpathlineto{\pgfqpoint{2.620569in}{5.477867in}}%
\pgfpathlineto{\pgfqpoint{2.620990in}{5.475813in}}%
\pgfpathlineto{\pgfqpoint{2.622675in}{5.471875in}}%
\pgfpathlineto{\pgfqpoint{2.626045in}{5.463431in}}%
\pgfpathlineto{\pgfqpoint{2.626888in}{5.475541in}}%
\pgfpathlineto{\pgfqpoint{2.627309in}{5.472252in}}%
\pgfpathlineto{\pgfqpoint{2.628994in}{5.455219in}}%
\pgfpathlineto{\pgfqpoint{2.629416in}{5.454967in}}%
\pgfpathlineto{\pgfqpoint{2.630258in}{5.467747in}}%
\pgfpathlineto{\pgfqpoint{2.630679in}{5.464207in}}%
\pgfpathlineto{\pgfqpoint{2.632364in}{5.454443in}}%
\pgfpathlineto{\pgfqpoint{2.632786in}{5.454213in}}%
\pgfpathlineto{\pgfqpoint{2.633628in}{5.460163in}}%
\pgfpathlineto{\pgfqpoint{2.634050in}{5.458277in}}%
\pgfpathlineto{\pgfqpoint{2.635735in}{5.454821in}}%
\pgfpathlineto{\pgfqpoint{2.636156in}{5.453312in}}%
\pgfpathlineto{\pgfqpoint{2.636999in}{5.456308in}}%
\pgfpathlineto{\pgfqpoint{2.638684in}{5.449478in}}%
\pgfpathlineto{\pgfqpoint{2.639105in}{5.449352in}}%
\pgfpathlineto{\pgfqpoint{2.639947in}{5.459807in}}%
\pgfpathlineto{\pgfqpoint{2.640369in}{5.456518in}}%
\pgfpathlineto{\pgfqpoint{2.642054in}{5.440741in}}%
\pgfpathlineto{\pgfqpoint{2.642475in}{5.440679in}}%
\pgfpathlineto{\pgfqpoint{2.643318in}{5.454024in}}%
\pgfpathlineto{\pgfqpoint{2.643739in}{5.450442in}}%
\pgfpathlineto{\pgfqpoint{2.645003in}{5.443088in}}%
\pgfpathlineto{\pgfqpoint{2.645845in}{5.441559in}}%
\pgfpathlineto{\pgfqpoint{2.647952in}{5.470806in}}%
\pgfpathlineto{\pgfqpoint{2.648373in}{5.471791in}}%
\pgfpathlineto{\pgfqpoint{2.649216in}{5.471120in}}%
\pgfpathlineto{\pgfqpoint{2.651322in}{5.496157in}}%
\pgfpathlineto{\pgfqpoint{2.653007in}{5.526159in}}%
\pgfpathlineto{\pgfqpoint{2.655113in}{5.568312in}}%
\pgfpathlineto{\pgfqpoint{2.655956in}{5.574304in}}%
\pgfpathlineto{\pgfqpoint{2.656377in}{5.570931in}}%
\pgfpathlineto{\pgfqpoint{2.658062in}{5.552515in}}%
\pgfpathlineto{\pgfqpoint{2.658905in}{5.556202in}}%
\pgfpathlineto{\pgfqpoint{2.659326in}{5.559156in}}%
\pgfpathlineto{\pgfqpoint{2.661011in}{5.530495in}}%
\pgfpathlineto{\pgfqpoint{2.661433in}{5.530747in}}%
\pgfpathlineto{\pgfqpoint{2.662275in}{5.540196in}}%
\pgfpathlineto{\pgfqpoint{2.663118in}{5.516291in}}%
\pgfpathlineto{\pgfqpoint{2.663960in}{5.519727in}}%
\pgfpathlineto{\pgfqpoint{2.664382in}{5.524504in}}%
\pgfpathlineto{\pgfqpoint{2.664803in}{5.520460in}}%
\pgfpathlineto{\pgfqpoint{2.665224in}{5.521047in}}%
\pgfpathlineto{\pgfqpoint{2.665645in}{5.520376in}}%
\pgfpathlineto{\pgfqpoint{2.666067in}{5.517108in}}%
\pgfpathlineto{\pgfqpoint{2.667331in}{5.476819in}}%
\pgfpathlineto{\pgfqpoint{2.667752in}{5.479040in}}%
\pgfpathlineto{\pgfqpoint{2.668173in}{5.477552in}}%
\pgfpathlineto{\pgfqpoint{2.669016in}{5.486855in}}%
\pgfpathlineto{\pgfqpoint{2.669437in}{5.480884in}}%
\pgfpathlineto{\pgfqpoint{2.672807in}{5.353522in}}%
\pgfpathlineto{\pgfqpoint{2.674492in}{5.313359in}}%
\pgfpathlineto{\pgfqpoint{2.675335in}{5.305691in}}%
\pgfpathlineto{\pgfqpoint{2.676599in}{5.320965in}}%
\pgfpathlineto{\pgfqpoint{2.680811in}{5.161947in}}%
\pgfpathlineto{\pgfqpoint{2.682075in}{5.140891in}}%
\pgfpathlineto{\pgfqpoint{2.682497in}{5.145835in}}%
\pgfpathlineto{\pgfqpoint{2.683339in}{5.158846in}}%
\pgfpathlineto{\pgfqpoint{2.684182in}{5.157526in}}%
\pgfpathlineto{\pgfqpoint{2.686709in}{5.167750in}}%
\pgfpathlineto{\pgfqpoint{2.687131in}{5.165864in}}%
\pgfpathlineto{\pgfqpoint{2.689237in}{5.164293in}}%
\pgfpathlineto{\pgfqpoint{2.690501in}{5.166304in}}%
\pgfpathlineto{\pgfqpoint{2.690922in}{5.165739in}}%
\pgfpathlineto{\pgfqpoint{2.691765in}{5.165466in}}%
\pgfpathlineto{\pgfqpoint{2.692186in}{5.166975in}}%
\pgfpathlineto{\pgfqpoint{2.692607in}{5.166535in}}%
\pgfpathlineto{\pgfqpoint{2.693450in}{5.165299in}}%
\pgfpathlineto{\pgfqpoint{2.694714in}{5.174831in}}%
\pgfpathlineto{\pgfqpoint{2.695135in}{5.174769in}}%
\pgfpathlineto{\pgfqpoint{2.695556in}{5.177786in}}%
\pgfpathlineto{\pgfqpoint{2.695977in}{5.176172in}}%
\pgfpathlineto{\pgfqpoint{2.696820in}{5.166996in}}%
\pgfpathlineto{\pgfqpoint{2.697663in}{5.167624in}}%
\pgfpathlineto{\pgfqpoint{2.700190in}{5.159118in}}%
\pgfpathlineto{\pgfqpoint{2.701454in}{5.166325in}}%
\pgfpathlineto{\pgfqpoint{2.701875in}{5.166493in}}%
\pgfpathlineto{\pgfqpoint{2.703139in}{5.152665in}}%
\pgfpathlineto{\pgfqpoint{2.703560in}{5.155682in}}%
\pgfpathlineto{\pgfqpoint{2.708195in}{5.260060in}}%
\pgfpathlineto{\pgfqpoint{2.708616in}{5.259243in}}%
\pgfpathlineto{\pgfqpoint{2.710301in}{5.279691in}}%
\pgfpathlineto{\pgfqpoint{2.715778in}{5.520649in}}%
\pgfpathlineto{\pgfqpoint{2.716199in}{5.520146in}}%
\pgfpathlineto{\pgfqpoint{2.717041in}{5.512184in}}%
\pgfpathlineto{\pgfqpoint{2.717463in}{5.516395in}}%
\pgfpathlineto{\pgfqpoint{2.717884in}{5.516479in}}%
\pgfpathlineto{\pgfqpoint{2.718726in}{5.524315in}}%
\pgfpathlineto{\pgfqpoint{2.720833in}{5.553877in}}%
\pgfpathlineto{\pgfqpoint{2.721254in}{5.553604in}}%
\pgfpathlineto{\pgfqpoint{2.722518in}{5.559743in}}%
\pgfpathlineto{\pgfqpoint{2.727995in}{5.424085in}}%
\pgfpathlineto{\pgfqpoint{2.731786in}{5.375416in}}%
\pgfpathlineto{\pgfqpoint{2.732207in}{5.376673in}}%
\pgfpathlineto{\pgfqpoint{2.733471in}{5.372651in}}%
\pgfpathlineto{\pgfqpoint{2.733893in}{5.377490in}}%
\pgfpathlineto{\pgfqpoint{2.734314in}{5.370954in}}%
\pgfpathlineto{\pgfqpoint{2.736841in}{5.323416in}}%
\pgfpathlineto{\pgfqpoint{2.738527in}{5.325176in}}%
\pgfpathlineto{\pgfqpoint{2.738948in}{5.328360in}}%
\pgfpathlineto{\pgfqpoint{2.739369in}{5.325741in}}%
\pgfpathlineto{\pgfqpoint{2.740212in}{5.323185in}}%
\pgfpathlineto{\pgfqpoint{2.741476in}{5.331356in}}%
\pgfpathlineto{\pgfqpoint{2.741897in}{5.330204in}}%
\pgfpathlineto{\pgfqpoint{2.742318in}{5.335735in}}%
\pgfpathlineto{\pgfqpoint{2.742739in}{5.329974in}}%
\pgfpathlineto{\pgfqpoint{2.743161in}{5.329282in}}%
\pgfpathlineto{\pgfqpoint{2.743582in}{5.330078in}}%
\pgfpathlineto{\pgfqpoint{2.744003in}{5.337474in}}%
\pgfpathlineto{\pgfqpoint{2.744846in}{5.334436in}}%
\pgfpathlineto{\pgfqpoint{2.746531in}{5.330644in}}%
\pgfpathlineto{\pgfqpoint{2.746952in}{5.331461in}}%
\pgfpathlineto{\pgfqpoint{2.748216in}{5.335966in}}%
\pgfpathlineto{\pgfqpoint{2.748637in}{5.334038in}}%
\pgfpathlineto{\pgfqpoint{2.749059in}{5.337684in}}%
\pgfpathlineto{\pgfqpoint{2.749480in}{5.332990in}}%
\pgfpathlineto{\pgfqpoint{2.749901in}{5.331754in}}%
\pgfpathlineto{\pgfqpoint{2.750744in}{5.339401in}}%
\pgfpathlineto{\pgfqpoint{2.751586in}{5.337244in}}%
\pgfpathlineto{\pgfqpoint{2.753271in}{5.332907in}}%
\pgfpathlineto{\pgfqpoint{2.754114in}{5.337285in}}%
\pgfpathlineto{\pgfqpoint{2.756642in}{5.292744in}}%
\pgfpathlineto{\pgfqpoint{2.757905in}{5.298987in}}%
\pgfpathlineto{\pgfqpoint{2.760012in}{5.295383in}}%
\pgfpathlineto{\pgfqpoint{2.763803in}{5.359452in}}%
\pgfpathlineto{\pgfqpoint{2.764225in}{5.355366in}}%
\pgfpathlineto{\pgfqpoint{2.765067in}{5.351825in}}%
\pgfpathlineto{\pgfqpoint{2.769701in}{5.421886in}}%
\pgfpathlineto{\pgfqpoint{2.770544in}{5.447990in}}%
\pgfpathlineto{\pgfqpoint{2.771386in}{5.440909in}}%
\pgfpathlineto{\pgfqpoint{2.771808in}{5.440260in}}%
\pgfpathlineto{\pgfqpoint{2.773914in}{5.461106in}}%
\pgfpathlineto{\pgfqpoint{2.775178in}{5.439044in}}%
\pgfpathlineto{\pgfqpoint{2.776442in}{5.440092in}}%
\pgfpathlineto{\pgfqpoint{2.777284in}{5.443046in}}%
\pgfpathlineto{\pgfqpoint{2.778127in}{5.441475in}}%
\pgfpathlineto{\pgfqpoint{2.778969in}{5.442355in}}%
\pgfpathlineto{\pgfqpoint{2.779391in}{5.441370in}}%
\pgfpathlineto{\pgfqpoint{2.779812in}{5.441286in}}%
\pgfpathlineto{\pgfqpoint{2.781497in}{5.448179in}}%
\pgfpathlineto{\pgfqpoint{2.782339in}{5.462216in}}%
\pgfpathlineto{\pgfqpoint{2.782761in}{5.457251in}}%
\pgfpathlineto{\pgfqpoint{2.786974in}{5.407471in}}%
\pgfpathlineto{\pgfqpoint{2.787395in}{5.408561in}}%
\pgfpathlineto{\pgfqpoint{2.787816in}{5.407199in}}%
\pgfpathlineto{\pgfqpoint{2.791608in}{5.400117in}}%
\pgfpathlineto{\pgfqpoint{2.792029in}{5.401542in}}%
\pgfpathlineto{\pgfqpoint{2.792871in}{5.400620in}}%
\pgfpathlineto{\pgfqpoint{2.793714in}{5.401814in}}%
\pgfpathlineto{\pgfqpoint{2.795399in}{5.395613in}}%
\pgfpathlineto{\pgfqpoint{2.795820in}{5.384216in}}%
\pgfpathlineto{\pgfqpoint{2.796663in}{5.391318in}}%
\pgfpathlineto{\pgfqpoint{2.798769in}{5.418533in}}%
\pgfpathlineto{\pgfqpoint{2.799191in}{5.419183in}}%
\pgfpathlineto{\pgfqpoint{2.800876in}{5.439924in}}%
\pgfpathlineto{\pgfqpoint{2.801718in}{5.437787in}}%
\pgfpathlineto{\pgfqpoint{2.802140in}{5.441077in}}%
\pgfpathlineto{\pgfqpoint{2.802982in}{5.440322in}}%
\pgfpathlineto{\pgfqpoint{2.803403in}{5.440783in}}%
\pgfpathlineto{\pgfqpoint{2.803825in}{5.443193in}}%
\pgfpathlineto{\pgfqpoint{2.804246in}{5.441454in}}%
\pgfpathlineto{\pgfqpoint{2.806774in}{5.441244in}}%
\pgfpathlineto{\pgfqpoint{2.808880in}{5.452516in}}%
\pgfpathlineto{\pgfqpoint{2.812672in}{5.416669in}}%
\pgfpathlineto{\pgfqpoint{2.813514in}{5.407660in}}%
\pgfpathlineto{\pgfqpoint{2.814357in}{5.409315in}}%
\pgfpathlineto{\pgfqpoint{2.815199in}{5.409881in}}%
\pgfpathlineto{\pgfqpoint{2.816884in}{5.394607in}}%
\pgfpathlineto{\pgfqpoint{2.818991in}{5.279188in}}%
\pgfpathlineto{\pgfqpoint{2.823203in}{5.047973in}}%
\pgfpathlineto{\pgfqpoint{2.823625in}{5.048476in}}%
\pgfpathlineto{\pgfqpoint{2.826574in}{5.033203in}}%
\pgfpathlineto{\pgfqpoint{2.827838in}{5.046905in}}%
\pgfpathlineto{\pgfqpoint{2.828259in}{5.046674in}}%
\pgfpathlineto{\pgfqpoint{2.829101in}{5.047282in}}%
\pgfpathlineto{\pgfqpoint{2.832050in}{5.033266in}}%
\pgfpathlineto{\pgfqpoint{2.832472in}{5.033349in}}%
\pgfpathlineto{\pgfqpoint{2.834578in}{5.032448in}}%
\pgfpathlineto{\pgfqpoint{2.834999in}{5.034565in}}%
\pgfpathlineto{\pgfqpoint{2.835421in}{5.046297in}}%
\pgfpathlineto{\pgfqpoint{2.836263in}{5.040913in}}%
\pgfpathlineto{\pgfqpoint{2.836684in}{5.040557in}}%
\pgfpathlineto{\pgfqpoint{2.837106in}{5.045103in}}%
\pgfpathlineto{\pgfqpoint{2.837527in}{5.040054in}}%
\pgfpathlineto{\pgfqpoint{2.838369in}{5.040137in}}%
\pgfpathlineto{\pgfqpoint{2.838791in}{5.044244in}}%
\pgfpathlineto{\pgfqpoint{2.839212in}{5.039216in}}%
\pgfpathlineto{\pgfqpoint{2.839633in}{5.039090in}}%
\pgfpathlineto{\pgfqpoint{2.840476in}{5.046695in}}%
\pgfpathlineto{\pgfqpoint{2.841318in}{5.041688in}}%
\pgfpathlineto{\pgfqpoint{2.842582in}{5.039258in}}%
\pgfpathlineto{\pgfqpoint{2.845110in}{5.022874in}}%
\pgfpathlineto{\pgfqpoint{2.846795in}{5.023167in}}%
\pgfpathlineto{\pgfqpoint{2.848059in}{5.024236in}}%
\pgfpathlineto{\pgfqpoint{2.848480in}{5.022099in}}%
\pgfpathlineto{\pgfqpoint{2.858591in}{4.420428in}}%
\pgfpathlineto{\pgfqpoint{2.860276in}{4.420323in}}%
\pgfpathlineto{\pgfqpoint{2.862382in}{4.420658in}}%
\pgfpathlineto{\pgfqpoint{2.867438in}{4.421790in}}%
\pgfpathlineto{\pgfqpoint{2.868280in}{4.421538in}}%
\pgfpathlineto{\pgfqpoint{2.869544in}{4.430883in}}%
\pgfpathlineto{\pgfqpoint{2.870387in}{4.444186in}}%
\pgfpathlineto{\pgfqpoint{2.875442in}{4.707038in}}%
\pgfpathlineto{\pgfqpoint{2.881761in}{5.137832in}}%
\pgfpathlineto{\pgfqpoint{2.884710in}{5.265088in}}%
\pgfpathlineto{\pgfqpoint{2.889344in}{5.341937in}}%
\pgfpathlineto{\pgfqpoint{2.891029in}{5.334960in}}%
\pgfpathlineto{\pgfqpoint{2.894399in}{5.335253in}}%
\pgfpathlineto{\pgfqpoint{2.901561in}{5.335672in}}%
\pgfpathlineto{\pgfqpoint{2.902825in}{5.305461in}}%
\pgfpathlineto{\pgfqpoint{2.906617in}{5.063917in}}%
\pgfpathlineto{\pgfqpoint{2.915042in}{4.554787in}}%
\pgfpathlineto{\pgfqpoint{2.915885in}{4.543264in}}%
\pgfpathlineto{\pgfqpoint{2.916306in}{4.546197in}}%
\pgfpathlineto{\pgfqpoint{2.918412in}{4.620154in}}%
\pgfpathlineto{\pgfqpoint{2.930208in}{5.067667in}}%
\pgfpathlineto{\pgfqpoint{2.937791in}{5.182458in}}%
\pgfpathlineto{\pgfqpoint{2.938212in}{5.182835in}}%
\pgfpathlineto{\pgfqpoint{2.939476in}{5.193771in}}%
\pgfpathlineto{\pgfqpoint{2.943268in}{5.136638in}}%
\pgfpathlineto{\pgfqpoint{2.946217in}{5.086376in}}%
\pgfpathlineto{\pgfqpoint{2.952536in}{4.987131in}}%
\pgfpathlineto{\pgfqpoint{2.953378in}{4.993082in}}%
\pgfpathlineto{\pgfqpoint{2.954221in}{4.998298in}}%
\pgfpathlineto{\pgfqpoint{2.959276in}{5.079169in}}%
\pgfpathlineto{\pgfqpoint{2.963910in}{5.147847in}}%
\pgfpathlineto{\pgfqpoint{2.967702in}{5.191865in}}%
\pgfpathlineto{\pgfqpoint{2.968123in}{5.190293in}}%
\pgfpathlineto{\pgfqpoint{2.968544in}{5.188156in}}%
\pgfpathlineto{\pgfqpoint{2.968966in}{5.190189in}}%
\pgfpathlineto{\pgfqpoint{2.969387in}{5.191781in}}%
\pgfpathlineto{\pgfqpoint{2.969808in}{5.191089in}}%
\pgfpathlineto{\pgfqpoint{2.970230in}{5.188659in}}%
\pgfpathlineto{\pgfqpoint{2.970651in}{5.190419in}}%
\pgfpathlineto{\pgfqpoint{2.971072in}{5.190943in}}%
\pgfpathlineto{\pgfqpoint{2.974021in}{5.173721in}}%
\pgfpathlineto{\pgfqpoint{2.974864in}{5.175334in}}%
\pgfpathlineto{\pgfqpoint{2.975706in}{5.170997in}}%
\pgfpathlineto{\pgfqpoint{2.976127in}{5.173260in}}%
\pgfpathlineto{\pgfqpoint{2.976549in}{5.173574in}}%
\pgfpathlineto{\pgfqpoint{2.977391in}{5.169824in}}%
\pgfpathlineto{\pgfqpoint{2.977813in}{5.171961in}}%
\pgfpathlineto{\pgfqpoint{2.978234in}{5.172715in}}%
\pgfpathlineto{\pgfqpoint{2.979076in}{5.165194in}}%
\pgfpathlineto{\pgfqpoint{2.979498in}{5.170138in}}%
\pgfpathlineto{\pgfqpoint{2.979919in}{5.172715in}}%
\pgfpathlineto{\pgfqpoint{2.980340in}{5.170557in}}%
\pgfpathlineto{\pgfqpoint{2.980761in}{5.166221in}}%
\pgfpathlineto{\pgfqpoint{2.981183in}{5.168022in}}%
\pgfpathlineto{\pgfqpoint{2.982025in}{5.171102in}}%
\pgfpathlineto{\pgfqpoint{2.982447in}{5.168672in}}%
\pgfpathlineto{\pgfqpoint{2.982868in}{5.168818in}}%
\pgfpathlineto{\pgfqpoint{2.983710in}{5.172129in}}%
\pgfpathlineto{\pgfqpoint{2.984132in}{5.170474in}}%
\pgfpathlineto{\pgfqpoint{2.984553in}{5.169510in}}%
\pgfpathlineto{\pgfqpoint{2.985396in}{5.178058in}}%
\pgfpathlineto{\pgfqpoint{2.985817in}{5.172422in}}%
\pgfpathlineto{\pgfqpoint{2.986238in}{5.164314in}}%
\pgfpathlineto{\pgfqpoint{2.986659in}{5.164691in}}%
\pgfpathlineto{\pgfqpoint{2.989187in}{5.191194in}}%
\pgfpathlineto{\pgfqpoint{2.989608in}{5.188617in}}%
\pgfpathlineto{\pgfqpoint{2.990030in}{5.189120in}}%
\pgfpathlineto{\pgfqpoint{2.990872in}{5.192724in}}%
\pgfpathlineto{\pgfqpoint{2.991293in}{5.190209in}}%
\pgfpathlineto{\pgfqpoint{2.991715in}{5.190105in}}%
\pgfpathlineto{\pgfqpoint{2.992136in}{5.192221in}}%
\pgfpathlineto{\pgfqpoint{2.992557in}{5.190670in}}%
\pgfpathlineto{\pgfqpoint{2.993400in}{5.177304in}}%
\pgfpathlineto{\pgfqpoint{2.993821in}{5.179315in}}%
\pgfpathlineto{\pgfqpoint{2.995928in}{5.184176in}}%
\pgfpathlineto{\pgfqpoint{2.996349in}{5.183526in}}%
\pgfpathlineto{\pgfqpoint{2.996770in}{5.182583in}}%
\pgfpathlineto{\pgfqpoint{2.997191in}{5.183757in}}%
\pgfpathlineto{\pgfqpoint{2.998455in}{5.183715in}}%
\pgfpathlineto{\pgfqpoint{3.000140in}{5.194064in}}%
\pgfpathlineto{\pgfqpoint{3.000983in}{5.189539in}}%
\pgfpathlineto{\pgfqpoint{3.003511in}{5.171898in}}%
\pgfpathlineto{\pgfqpoint{3.012779in}{4.960356in}}%
\pgfpathlineto{\pgfqpoint{3.016991in}{4.851411in}}%
\pgfpathlineto{\pgfqpoint{3.017413in}{4.851516in}}%
\pgfpathlineto{\pgfqpoint{3.017834in}{4.850971in}}%
\pgfpathlineto{\pgfqpoint{3.018677in}{4.853695in}}%
\pgfpathlineto{\pgfqpoint{3.019098in}{4.852542in}}%
\pgfpathlineto{\pgfqpoint{3.027102in}{4.791512in}}%
\pgfpathlineto{\pgfqpoint{3.027945in}{4.796519in}}%
\pgfpathlineto{\pgfqpoint{3.028787in}{4.811353in}}%
\pgfpathlineto{\pgfqpoint{3.031736in}{4.875588in}}%
\pgfpathlineto{\pgfqpoint{3.033843in}{4.919166in}}%
\pgfpathlineto{\pgfqpoint{3.038898in}{5.113466in}}%
\pgfpathlineto{\pgfqpoint{3.042689in}{5.236532in}}%
\pgfpathlineto{\pgfqpoint{3.043532in}{5.242629in}}%
\pgfpathlineto{\pgfqpoint{3.043953in}{5.239905in}}%
\pgfpathlineto{\pgfqpoint{3.044374in}{5.238711in}}%
\pgfpathlineto{\pgfqpoint{3.044796in}{5.240680in}}%
\pgfpathlineto{\pgfqpoint{3.045217in}{5.242440in}}%
\pgfpathlineto{\pgfqpoint{3.045638in}{5.242252in}}%
\pgfpathlineto{\pgfqpoint{3.046060in}{5.239528in}}%
\pgfpathlineto{\pgfqpoint{3.046481in}{5.244054in}}%
\pgfpathlineto{\pgfqpoint{3.046902in}{5.244954in}}%
\pgfpathlineto{\pgfqpoint{3.047745in}{5.232468in}}%
\pgfpathlineto{\pgfqpoint{3.048587in}{5.235757in}}%
\pgfpathlineto{\pgfqpoint{3.049430in}{5.236176in}}%
\pgfpathlineto{\pgfqpoint{3.050694in}{5.239235in}}%
\pgfpathlineto{\pgfqpoint{3.051115in}{5.237140in}}%
\pgfpathlineto{\pgfqpoint{3.051536in}{5.238397in}}%
\pgfpathlineto{\pgfqpoint{3.052379in}{5.240010in}}%
\pgfpathlineto{\pgfqpoint{3.052800in}{5.235987in}}%
\pgfpathlineto{\pgfqpoint{3.053221in}{5.239067in}}%
\pgfpathlineto{\pgfqpoint{3.053643in}{5.239926in}}%
\pgfpathlineto{\pgfqpoint{3.054064in}{5.239675in}}%
\pgfpathlineto{\pgfqpoint{3.056170in}{5.225135in}}%
\pgfpathlineto{\pgfqpoint{3.059119in}{5.241937in}}%
\pgfpathlineto{\pgfqpoint{3.059540in}{5.241351in}}%
\pgfpathlineto{\pgfqpoint{3.059962in}{5.240052in}}%
\pgfpathlineto{\pgfqpoint{3.060804in}{5.244473in}}%
\pgfpathlineto{\pgfqpoint{3.063753in}{5.215036in}}%
\pgfpathlineto{\pgfqpoint{3.064596in}{5.207159in}}%
\pgfpathlineto{\pgfqpoint{3.065017in}{5.207536in}}%
\pgfpathlineto{\pgfqpoint{3.065860in}{5.220945in}}%
\pgfpathlineto{\pgfqpoint{3.066281in}{5.219876in}}%
\pgfpathlineto{\pgfqpoint{3.066702in}{5.216859in}}%
\pgfpathlineto{\pgfqpoint{3.067124in}{5.222013in}}%
\pgfpathlineto{\pgfqpoint{3.067966in}{5.220086in}}%
\pgfpathlineto{\pgfqpoint{3.068387in}{5.216398in}}%
\pgfpathlineto{\pgfqpoint{3.069230in}{5.236281in}}%
\pgfpathlineto{\pgfqpoint{3.069651in}{5.234248in}}%
\pgfpathlineto{\pgfqpoint{3.070072in}{5.229597in}}%
\pgfpathlineto{\pgfqpoint{3.070915in}{5.231692in}}%
\pgfpathlineto{\pgfqpoint{3.071336in}{5.232132in}}%
\pgfpathlineto{\pgfqpoint{3.071758in}{5.231462in}}%
\pgfpathlineto{\pgfqpoint{3.074285in}{5.272379in}}%
\pgfpathlineto{\pgfqpoint{3.074707in}{5.274391in}}%
\pgfpathlineto{\pgfqpoint{3.075128in}{5.268482in}}%
\pgfpathlineto{\pgfqpoint{3.078498in}{5.331671in}}%
\pgfpathlineto{\pgfqpoint{3.079341in}{5.322997in}}%
\pgfpathlineto{\pgfqpoint{3.080183in}{5.324694in}}%
\pgfpathlineto{\pgfqpoint{3.081447in}{5.317256in}}%
\pgfpathlineto{\pgfqpoint{3.083553in}{5.287778in}}%
\pgfpathlineto{\pgfqpoint{3.083975in}{5.291088in}}%
\pgfpathlineto{\pgfqpoint{3.085238in}{5.296871in}}%
\pgfpathlineto{\pgfqpoint{3.086502in}{5.284489in}}%
\pgfpathlineto{\pgfqpoint{3.086924in}{5.286458in}}%
\pgfpathlineto{\pgfqpoint{3.088609in}{5.291927in}}%
\pgfpathlineto{\pgfqpoint{3.089030in}{5.289182in}}%
\pgfpathlineto{\pgfqpoint{3.089451in}{5.289936in}}%
\pgfpathlineto{\pgfqpoint{3.090294in}{5.294189in}}%
\pgfpathlineto{\pgfqpoint{3.091136in}{5.293540in}}%
\pgfpathlineto{\pgfqpoint{3.092400in}{5.292052in}}%
\pgfpathlineto{\pgfqpoint{3.094507in}{5.283420in}}%
\pgfpathlineto{\pgfqpoint{3.099141in}{5.300223in}}%
\pgfpathlineto{\pgfqpoint{3.101668in}{5.299448in}}%
\pgfpathlineto{\pgfqpoint{3.103353in}{5.305251in}}%
\pgfpathlineto{\pgfqpoint{3.103775in}{5.304476in}}%
\pgfpathlineto{\pgfqpoint{3.104617in}{5.301375in}}%
\pgfpathlineto{\pgfqpoint{3.105039in}{5.301480in}}%
\pgfpathlineto{\pgfqpoint{3.105881in}{5.310154in}}%
\pgfpathlineto{\pgfqpoint{3.110936in}{5.385913in}}%
\pgfpathlineto{\pgfqpoint{3.111358in}{5.384530in}}%
\pgfpathlineto{\pgfqpoint{3.118519in}{5.333891in}}%
\pgfpathlineto{\pgfqpoint{3.119362in}{5.316565in}}%
\pgfpathlineto{\pgfqpoint{3.124417in}{5.131819in}}%
\pgfpathlineto{\pgfqpoint{3.124839in}{5.129703in}}%
\pgfpathlineto{\pgfqpoint{3.125260in}{5.131505in}}%
\pgfpathlineto{\pgfqpoint{3.127788in}{5.141959in}}%
\pgfpathlineto{\pgfqpoint{3.129051in}{5.141079in}}%
\pgfpathlineto{\pgfqpoint{3.131158in}{5.130876in}}%
\pgfpathlineto{\pgfqpoint{3.132843in}{5.126477in}}%
\pgfpathlineto{\pgfqpoint{3.135792in}{5.145542in}}%
\pgfpathlineto{\pgfqpoint{3.136213in}{5.145458in}}%
\pgfpathlineto{\pgfqpoint{3.137056in}{5.148475in}}%
\pgfpathlineto{\pgfqpoint{3.138320in}{5.152162in}}%
\pgfpathlineto{\pgfqpoint{3.140426in}{5.151995in}}%
\pgfpathlineto{\pgfqpoint{3.141690in}{5.152414in}}%
\pgfpathlineto{\pgfqpoint{3.142111in}{5.152162in}}%
\pgfpathlineto{\pgfqpoint{3.142532in}{5.152980in}}%
\pgfpathlineto{\pgfqpoint{3.144217in}{5.153252in}}%
\pgfpathlineto{\pgfqpoint{3.148851in}{5.130918in}}%
\pgfpathlineto{\pgfqpoint{3.150115in}{5.134291in}}%
\pgfpathlineto{\pgfqpoint{3.151800in}{5.129305in}}%
\pgfpathlineto{\pgfqpoint{3.152222in}{5.128886in}}%
\pgfpathlineto{\pgfqpoint{3.153486in}{5.130290in}}%
\pgfpathlineto{\pgfqpoint{3.156434in}{5.127545in}}%
\pgfpathlineto{\pgfqpoint{3.157277in}{5.122182in}}%
\pgfpathlineto{\pgfqpoint{3.157698in}{5.122684in}}%
\pgfpathlineto{\pgfqpoint{3.158120in}{5.124381in}}%
\pgfpathlineto{\pgfqpoint{3.158962in}{5.123648in}}%
\pgfpathlineto{\pgfqpoint{3.159383in}{5.124151in}}%
\pgfpathlineto{\pgfqpoint{3.159805in}{5.123585in}}%
\pgfpathlineto{\pgfqpoint{3.162754in}{5.122705in}}%
\pgfpathlineto{\pgfqpoint{3.163596in}{5.118515in}}%
\pgfpathlineto{\pgfqpoint{3.164017in}{5.118641in}}%
\pgfpathlineto{\pgfqpoint{3.165281in}{5.120066in}}%
\pgfpathlineto{\pgfqpoint{3.167388in}{5.119793in}}%
\pgfpathlineto{\pgfqpoint{3.168230in}{5.120359in}}%
\pgfpathlineto{\pgfqpoint{3.169915in}{5.115477in}}%
\pgfpathlineto{\pgfqpoint{3.171179in}{5.100916in}}%
\pgfpathlineto{\pgfqpoint{3.172022in}{5.101021in}}%
\pgfpathlineto{\pgfqpoint{3.172443in}{5.102215in}}%
\pgfpathlineto{\pgfqpoint{3.174128in}{5.147386in}}%
\pgfpathlineto{\pgfqpoint{3.177920in}{5.257546in}}%
\pgfpathlineto{\pgfqpoint{3.180869in}{5.251596in}}%
\pgfpathlineto{\pgfqpoint{3.182975in}{5.276423in}}%
\pgfpathlineto{\pgfqpoint{3.184660in}{5.305356in}}%
\pgfpathlineto{\pgfqpoint{3.185081in}{5.304330in}}%
\pgfpathlineto{\pgfqpoint{3.185924in}{5.312354in}}%
\pgfpathlineto{\pgfqpoint{3.186345in}{5.310322in}}%
\pgfpathlineto{\pgfqpoint{3.186766in}{5.306278in}}%
\pgfpathlineto{\pgfqpoint{3.187188in}{5.310845in}}%
\pgfpathlineto{\pgfqpoint{3.187609in}{5.315182in}}%
\pgfpathlineto{\pgfqpoint{3.188030in}{5.313003in}}%
\pgfpathlineto{\pgfqpoint{3.188452in}{5.308792in}}%
\pgfpathlineto{\pgfqpoint{3.188873in}{5.311097in}}%
\pgfpathlineto{\pgfqpoint{3.189294in}{5.312982in}}%
\pgfpathlineto{\pgfqpoint{3.190137in}{5.303848in}}%
\pgfpathlineto{\pgfqpoint{3.190558in}{5.309840in}}%
\pgfpathlineto{\pgfqpoint{3.192243in}{5.332362in}}%
\pgfpathlineto{\pgfqpoint{3.192664in}{5.331021in}}%
\pgfpathlineto{\pgfqpoint{3.193507in}{5.309651in}}%
\pgfpathlineto{\pgfqpoint{3.195192in}{5.218870in}}%
\pgfpathlineto{\pgfqpoint{3.198984in}{5.038273in}}%
\pgfpathlineto{\pgfqpoint{3.201933in}{5.019459in}}%
\pgfpathlineto{\pgfqpoint{3.203618in}{5.011393in}}%
\pgfpathlineto{\pgfqpoint{3.205724in}{4.979212in}}%
\pgfpathlineto{\pgfqpoint{3.206145in}{4.979903in}}%
\pgfpathlineto{\pgfqpoint{3.207409in}{5.014933in}}%
\pgfpathlineto{\pgfqpoint{3.209516in}{5.138041in}}%
\pgfpathlineto{\pgfqpoint{3.212043in}{5.249710in}}%
\pgfpathlineto{\pgfqpoint{3.215413in}{5.263245in}}%
\pgfpathlineto{\pgfqpoint{3.216677in}{5.264481in}}%
\pgfpathlineto{\pgfqpoint{3.219205in}{5.301229in}}%
\pgfpathlineto{\pgfqpoint{3.220047in}{5.292827in}}%
\pgfpathlineto{\pgfqpoint{3.222154in}{5.188471in}}%
\pgfpathlineto{\pgfqpoint{3.225524in}{5.031003in}}%
\pgfpathlineto{\pgfqpoint{3.228473in}{5.025241in}}%
\pgfpathlineto{\pgfqpoint{3.228894in}{5.024529in}}%
\pgfpathlineto{\pgfqpoint{3.229316in}{5.025283in}}%
\pgfpathlineto{\pgfqpoint{3.231001in}{5.026980in}}%
\pgfpathlineto{\pgfqpoint{3.234371in}{5.027169in}}%
\pgfpathlineto{\pgfqpoint{3.236056in}{5.026771in}}%
\pgfpathlineto{\pgfqpoint{3.238162in}{5.024990in}}%
\pgfpathlineto{\pgfqpoint{3.240690in}{4.926185in}}%
\pgfpathlineto{\pgfqpoint{3.243218in}{4.924153in}}%
\pgfpathlineto{\pgfqpoint{3.244482in}{4.923922in}}%
\pgfpathlineto{\pgfqpoint{3.245745in}{4.922456in}}%
\pgfpathlineto{\pgfqpoint{3.246167in}{4.920843in}}%
\pgfpathlineto{\pgfqpoint{3.246588in}{4.923922in}}%
\pgfpathlineto{\pgfqpoint{3.253328in}{5.317717in}}%
\pgfpathlineto{\pgfqpoint{3.254171in}{5.310594in}}%
\pgfpathlineto{\pgfqpoint{3.255014in}{5.297562in}}%
\pgfpathlineto{\pgfqpoint{3.265546in}{4.909885in}}%
\pgfpathlineto{\pgfqpoint{3.268073in}{4.854805in}}%
\pgfpathlineto{\pgfqpoint{3.270601in}{4.864694in}}%
\pgfpathlineto{\pgfqpoint{3.273129in}{4.848750in}}%
\pgfpathlineto{\pgfqpoint{3.273550in}{4.850342in}}%
\pgfpathlineto{\pgfqpoint{3.280712in}{5.033224in}}%
\pgfpathlineto{\pgfqpoint{3.283239in}{5.096077in}}%
\pgfpathlineto{\pgfqpoint{3.284924in}{5.150654in}}%
\pgfpathlineto{\pgfqpoint{3.287031in}{5.195699in}}%
\pgfpathlineto{\pgfqpoint{3.287873in}{5.200915in}}%
\pgfpathlineto{\pgfqpoint{3.288295in}{5.200098in}}%
\pgfpathlineto{\pgfqpoint{3.289980in}{5.179818in}}%
\pgfpathlineto{\pgfqpoint{3.291665in}{5.167394in}}%
\pgfpathlineto{\pgfqpoint{3.293350in}{5.145458in}}%
\pgfpathlineto{\pgfqpoint{3.295456in}{5.116734in}}%
\pgfpathlineto{\pgfqpoint{3.295878in}{5.116231in}}%
\pgfpathlineto{\pgfqpoint{3.297141in}{5.105169in}}%
\pgfpathlineto{\pgfqpoint{3.298405in}{5.105777in}}%
\pgfpathlineto{\pgfqpoint{3.299669in}{5.105672in}}%
\pgfpathlineto{\pgfqpoint{3.300090in}{5.102131in}}%
\pgfpathlineto{\pgfqpoint{3.300512in}{5.102844in}}%
\pgfpathlineto{\pgfqpoint{3.300933in}{5.114744in}}%
\pgfpathlineto{\pgfqpoint{3.301775in}{5.112460in}}%
\pgfpathlineto{\pgfqpoint{3.302197in}{5.112083in}}%
\pgfpathlineto{\pgfqpoint{3.307252in}{5.195280in}}%
\pgfpathlineto{\pgfqpoint{3.307673in}{5.190084in}}%
\pgfpathlineto{\pgfqpoint{3.308937in}{5.177681in}}%
\pgfpathlineto{\pgfqpoint{3.317784in}{4.966034in}}%
\pgfpathlineto{\pgfqpoint{3.321154in}{4.861572in}}%
\pgfpathlineto{\pgfqpoint{3.321997in}{4.850363in}}%
\pgfpathlineto{\pgfqpoint{3.322418in}{4.854051in}}%
\pgfpathlineto{\pgfqpoint{3.323261in}{4.856879in}}%
\pgfpathlineto{\pgfqpoint{3.323682in}{4.856607in}}%
\pgfpathlineto{\pgfqpoint{3.324524in}{4.866852in}}%
\pgfpathlineto{\pgfqpoint{3.324946in}{4.862305in}}%
\pgfpathlineto{\pgfqpoint{3.326631in}{4.852479in}}%
\pgfpathlineto{\pgfqpoint{3.327052in}{4.852584in}}%
\pgfpathlineto{\pgfqpoint{3.328316in}{4.886273in}}%
\pgfpathlineto{\pgfqpoint{3.337163in}{5.148517in}}%
\pgfpathlineto{\pgfqpoint{3.337584in}{5.145689in}}%
\pgfpathlineto{\pgfqpoint{3.340533in}{5.207766in}}%
\pgfpathlineto{\pgfqpoint{3.340954in}{5.201418in}}%
\pgfpathlineto{\pgfqpoint{3.346431in}{5.111015in}}%
\pgfpathlineto{\pgfqpoint{3.348116in}{5.075126in}}%
\pgfpathlineto{\pgfqpoint{3.348959in}{5.076341in}}%
\pgfpathlineto{\pgfqpoint{3.349801in}{5.070768in}}%
\pgfpathlineto{\pgfqpoint{3.350222in}{5.073324in}}%
\pgfpathlineto{\pgfqpoint{3.350644in}{5.073031in}}%
\pgfpathlineto{\pgfqpoint{3.351065in}{5.070894in}}%
\pgfpathlineto{\pgfqpoint{3.351486in}{5.060837in}}%
\pgfpathlineto{\pgfqpoint{3.351908in}{5.065153in}}%
\pgfpathlineto{\pgfqpoint{3.354435in}{5.089959in}}%
\pgfpathlineto{\pgfqpoint{3.354856in}{5.082731in}}%
\pgfpathlineto{\pgfqpoint{3.355278in}{5.087214in}}%
\pgfpathlineto{\pgfqpoint{3.355699in}{5.088157in}}%
\pgfpathlineto{\pgfqpoint{3.357805in}{5.074644in}}%
\pgfpathlineto{\pgfqpoint{3.358648in}{5.058700in}}%
\pgfpathlineto{\pgfqpoint{3.359069in}{5.061214in}}%
\pgfpathlineto{\pgfqpoint{3.360333in}{5.080594in}}%
\pgfpathlineto{\pgfqpoint{3.361176in}{5.108689in}}%
\pgfpathlineto{\pgfqpoint{3.362018in}{5.098025in}}%
\pgfpathlineto{\pgfqpoint{3.364125in}{5.048246in}}%
\pgfpathlineto{\pgfqpoint{3.365388in}{5.010555in}}%
\pgfpathlineto{\pgfqpoint{3.370444in}{4.846865in}}%
\pgfpathlineto{\pgfqpoint{3.374657in}{4.710578in}}%
\pgfpathlineto{\pgfqpoint{3.376763in}{4.669996in}}%
\pgfpathlineto{\pgfqpoint{3.377184in}{4.670834in}}%
\pgfpathlineto{\pgfqpoint{3.386031in}{5.025158in}}%
\pgfpathlineto{\pgfqpoint{3.390665in}{5.167017in}}%
\pgfpathlineto{\pgfqpoint{3.401618in}{4.734944in}}%
\pgfpathlineto{\pgfqpoint{3.402040in}{4.737165in}}%
\pgfpathlineto{\pgfqpoint{3.407516in}{4.911750in}}%
\pgfpathlineto{\pgfqpoint{3.413835in}{5.142839in}}%
\pgfpathlineto{\pgfqpoint{3.415099in}{5.163413in}}%
\pgfpathlineto{\pgfqpoint{3.426052in}{4.730838in}}%
\pgfpathlineto{\pgfqpoint{3.426474in}{4.733583in}}%
\pgfpathlineto{\pgfqpoint{3.431950in}{4.913112in}}%
\pgfpathlineto{\pgfqpoint{3.437427in}{5.139361in}}%
\pgfpathlineto{\pgfqpoint{3.439533in}{5.180153in}}%
\pgfpathlineto{\pgfqpoint{3.450487in}{4.728848in}}%
\pgfpathlineto{\pgfqpoint{3.450908in}{4.731466in}}%
\pgfpathlineto{\pgfqpoint{3.457648in}{4.942024in}}%
\pgfpathlineto{\pgfqpoint{3.458491in}{4.933120in}}%
\pgfpathlineto{\pgfqpoint{3.461019in}{4.903013in}}%
\pgfpathlineto{\pgfqpoint{3.463125in}{4.883026in}}%
\pgfpathlineto{\pgfqpoint{3.464389in}{4.915877in}}%
\pgfpathlineto{\pgfqpoint{3.464810in}{4.910870in}}%
\pgfpathlineto{\pgfqpoint{3.466074in}{4.904585in}}%
\pgfpathlineto{\pgfqpoint{3.468602in}{4.902992in}}%
\pgfpathlineto{\pgfqpoint{3.471972in}{4.867899in}}%
\pgfpathlineto{\pgfqpoint{3.473236in}{4.868821in}}%
\pgfpathlineto{\pgfqpoint{3.474921in}{4.869638in}}%
\pgfpathlineto{\pgfqpoint{3.475342in}{4.869785in}}%
\pgfpathlineto{\pgfqpoint{3.475763in}{4.868549in}}%
\pgfpathlineto{\pgfqpoint{3.476185in}{4.874687in}}%
\pgfpathlineto{\pgfqpoint{3.477027in}{4.908125in}}%
\pgfpathlineto{\pgfqpoint{3.477448in}{4.907245in}}%
\pgfpathlineto{\pgfqpoint{3.478712in}{4.904249in}}%
\pgfpathlineto{\pgfqpoint{3.481240in}{4.904689in}}%
\pgfpathlineto{\pgfqpoint{3.481661in}{4.903244in}}%
\pgfpathlineto{\pgfqpoint{3.484189in}{4.872446in}}%
\pgfpathlineto{\pgfqpoint{3.485453in}{4.870539in}}%
\pgfpathlineto{\pgfqpoint{3.488402in}{4.870832in}}%
\pgfpathlineto{\pgfqpoint{3.488823in}{4.872236in}}%
\pgfpathlineto{\pgfqpoint{3.489665in}{4.910996in}}%
\pgfpathlineto{\pgfqpoint{3.490508in}{4.907748in}}%
\pgfpathlineto{\pgfqpoint{3.491772in}{4.905925in}}%
\pgfpathlineto{\pgfqpoint{3.494300in}{4.906868in}}%
\pgfpathlineto{\pgfqpoint{3.496406in}{4.891029in}}%
\pgfpathlineto{\pgfqpoint{3.497248in}{4.890128in}}%
\pgfpathlineto{\pgfqpoint{3.501883in}{4.938337in}}%
\pgfpathlineto{\pgfqpoint{3.502725in}{4.970161in}}%
\pgfpathlineto{\pgfqpoint{3.503146in}{4.968527in}}%
\pgfpathlineto{\pgfqpoint{3.503989in}{4.965824in}}%
\pgfpathlineto{\pgfqpoint{3.504410in}{4.965992in}}%
\pgfpathlineto{\pgfqpoint{3.506938in}{4.965824in}}%
\pgfpathlineto{\pgfqpoint{3.507359in}{4.963771in}}%
\pgfpathlineto{\pgfqpoint{3.509044in}{4.947660in}}%
\pgfpathlineto{\pgfqpoint{3.511151in}{4.939803in}}%
\pgfpathlineto{\pgfqpoint{3.514100in}{4.937939in}}%
\pgfpathlineto{\pgfqpoint{3.514521in}{4.942380in}}%
\pgfpathlineto{\pgfqpoint{3.515363in}{4.978018in}}%
\pgfpathlineto{\pgfqpoint{3.515785in}{4.976069in}}%
\pgfpathlineto{\pgfqpoint{3.517049in}{4.969030in}}%
\pgfpathlineto{\pgfqpoint{3.519576in}{4.968799in}}%
\pgfpathlineto{\pgfqpoint{3.519997in}{4.967647in}}%
\pgfpathlineto{\pgfqpoint{3.522525in}{4.941500in}}%
\pgfpathlineto{\pgfqpoint{3.522946in}{4.939154in}}%
\pgfpathlineto{\pgfqpoint{3.523789in}{4.939426in}}%
\pgfpathlineto{\pgfqpoint{3.526317in}{4.939636in}}%
\pgfpathlineto{\pgfqpoint{3.527159in}{4.936451in}}%
\pgfpathlineto{\pgfqpoint{3.528423in}{4.975692in}}%
\pgfpathlineto{\pgfqpoint{3.528844in}{4.970999in}}%
\pgfpathlineto{\pgfqpoint{3.529266in}{4.967773in}}%
\pgfpathlineto{\pgfqpoint{3.530108in}{4.968255in}}%
\pgfpathlineto{\pgfqpoint{3.532636in}{4.969658in}}%
\pgfpathlineto{\pgfqpoint{3.535585in}{4.941794in}}%
\pgfpathlineto{\pgfqpoint{3.538112in}{4.941039in}}%
\pgfpathlineto{\pgfqpoint{3.538955in}{4.944957in}}%
\pgfpathlineto{\pgfqpoint{3.539376in}{4.943721in}}%
\pgfpathlineto{\pgfqpoint{3.539798in}{4.941647in}}%
\pgfpathlineto{\pgfqpoint{3.541061in}{4.982061in}}%
\pgfpathlineto{\pgfqpoint{3.541904in}{4.973576in}}%
\pgfpathlineto{\pgfqpoint{3.542747in}{4.972612in}}%
\pgfpathlineto{\pgfqpoint{3.543168in}{4.972864in}}%
\pgfpathlineto{\pgfqpoint{3.545274in}{4.972550in}}%
\pgfpathlineto{\pgfqpoint{3.545695in}{4.970329in}}%
\pgfpathlineto{\pgfqpoint{3.547381in}{4.953715in}}%
\pgfpathlineto{\pgfqpoint{3.548644in}{4.945209in}}%
\pgfpathlineto{\pgfqpoint{3.549487in}{4.945586in}}%
\pgfpathlineto{\pgfqpoint{3.551593in}{4.945690in}}%
\pgfpathlineto{\pgfqpoint{3.552436in}{4.938986in}}%
\pgfpathlineto{\pgfqpoint{3.552857in}{4.942296in}}%
\pgfpathlineto{\pgfqpoint{3.553700in}{4.978667in}}%
\pgfpathlineto{\pgfqpoint{3.554121in}{4.976845in}}%
\pgfpathlineto{\pgfqpoint{3.555806in}{4.969176in}}%
\pgfpathlineto{\pgfqpoint{3.558334in}{4.968171in}}%
\pgfpathlineto{\pgfqpoint{3.560861in}{4.944161in}}%
\pgfpathlineto{\pgfqpoint{3.562125in}{4.941437in}}%
\pgfpathlineto{\pgfqpoint{3.564653in}{4.941542in}}%
\pgfpathlineto{\pgfqpoint{3.565496in}{4.937750in}}%
\pgfpathlineto{\pgfqpoint{3.566759in}{4.977955in}}%
\pgfpathlineto{\pgfqpoint{3.567181in}{4.973828in}}%
\pgfpathlineto{\pgfqpoint{3.568023in}{4.982438in}}%
\pgfpathlineto{\pgfqpoint{3.568444in}{4.979547in}}%
\pgfpathlineto{\pgfqpoint{3.568866in}{4.977934in}}%
\pgfpathlineto{\pgfqpoint{3.569287in}{4.984513in}}%
\pgfpathlineto{\pgfqpoint{3.569708in}{4.982857in}}%
\pgfpathlineto{\pgfqpoint{3.570130in}{4.980008in}}%
\pgfpathlineto{\pgfqpoint{3.570551in}{4.980113in}}%
\pgfpathlineto{\pgfqpoint{3.570972in}{4.985330in}}%
\pgfpathlineto{\pgfqpoint{3.571393in}{4.982690in}}%
\pgfpathlineto{\pgfqpoint{3.571815in}{4.979903in}}%
\pgfpathlineto{\pgfqpoint{3.573921in}{5.078645in}}%
\pgfpathlineto{\pgfqpoint{3.580662in}{5.412227in}}%
\pgfpathlineto{\pgfqpoint{3.581083in}{5.409357in}}%
\pgfpathlineto{\pgfqpoint{3.585296in}{5.481994in}}%
\pgfpathlineto{\pgfqpoint{3.585717in}{5.492805in}}%
\pgfpathlineto{\pgfqpoint{3.586138in}{5.489494in}}%
\pgfpathlineto{\pgfqpoint{3.586981in}{5.483502in}}%
\pgfpathlineto{\pgfqpoint{3.587402in}{5.483670in}}%
\pgfpathlineto{\pgfqpoint{3.590772in}{5.483628in}}%
\pgfpathlineto{\pgfqpoint{3.591615in}{5.497267in}}%
\pgfpathlineto{\pgfqpoint{3.592036in}{5.495780in}}%
\pgfpathlineto{\pgfqpoint{3.593300in}{5.470471in}}%
\pgfpathlineto{\pgfqpoint{3.593721in}{5.470701in}}%
\pgfpathlineto{\pgfqpoint{3.596670in}{5.471477in}}%
\pgfpathlineto{\pgfqpoint{3.598777in}{5.418156in}}%
\pgfpathlineto{\pgfqpoint{3.603832in}{5.099932in}}%
\pgfpathlineto{\pgfqpoint{3.606360in}{5.063184in}}%
\pgfpathlineto{\pgfqpoint{3.613521in}{4.940788in}}%
\pgfpathlineto{\pgfqpoint{3.614364in}{4.941018in}}%
\pgfpathlineto{\pgfqpoint{3.616049in}{4.939740in}}%
\pgfpathlineto{\pgfqpoint{3.616470in}{4.937268in}}%
\pgfpathlineto{\pgfqpoint{3.617313in}{4.969323in}}%
\pgfpathlineto{\pgfqpoint{3.617734in}{4.977452in}}%
\pgfpathlineto{\pgfqpoint{3.618155in}{4.974917in}}%
\pgfpathlineto{\pgfqpoint{3.619419in}{4.969176in}}%
\pgfpathlineto{\pgfqpoint{3.622368in}{4.966851in}}%
\pgfpathlineto{\pgfqpoint{3.624474in}{4.946256in}}%
\pgfpathlineto{\pgfqpoint{3.626160in}{4.938818in}}%
\pgfpathlineto{\pgfqpoint{3.628687in}{4.938881in}}%
\pgfpathlineto{\pgfqpoint{3.629109in}{4.936975in}}%
\pgfpathlineto{\pgfqpoint{3.629530in}{4.937834in}}%
\pgfpathlineto{\pgfqpoint{3.630372in}{4.976635in}}%
\pgfpathlineto{\pgfqpoint{3.631215in}{4.971209in}}%
\pgfpathlineto{\pgfqpoint{3.632479in}{4.968234in}}%
\pgfpathlineto{\pgfqpoint{3.635006in}{4.968988in}}%
\pgfpathlineto{\pgfqpoint{3.637955in}{4.941542in}}%
\pgfpathlineto{\pgfqpoint{3.640483in}{4.942003in}}%
\pgfpathlineto{\pgfqpoint{3.641326in}{4.942485in}}%
\pgfpathlineto{\pgfqpoint{3.642168in}{4.938441in}}%
\pgfpathlineto{\pgfqpoint{3.644696in}{4.974540in}}%
\pgfpathlineto{\pgfqpoint{3.645960in}{4.966662in}}%
\pgfpathlineto{\pgfqpoint{3.649751in}{4.859037in}}%
\pgfpathlineto{\pgfqpoint{3.650594in}{4.883131in}}%
\pgfpathlineto{\pgfqpoint{3.654807in}{5.089058in}}%
\pgfpathlineto{\pgfqpoint{3.656913in}{5.176549in}}%
\pgfpathlineto{\pgfqpoint{3.658598in}{5.194714in}}%
\pgfpathlineto{\pgfqpoint{3.659862in}{5.230917in}}%
\pgfpathlineto{\pgfqpoint{3.663653in}{5.436593in}}%
\pgfpathlineto{\pgfqpoint{3.664496in}{5.433052in}}%
\pgfpathlineto{\pgfqpoint{3.666181in}{5.431167in}}%
\pgfpathlineto{\pgfqpoint{3.668287in}{5.413442in}}%
\pgfpathlineto{\pgfqpoint{3.668709in}{5.413526in}}%
\pgfpathlineto{\pgfqpoint{3.669551in}{5.413819in}}%
\pgfpathlineto{\pgfqpoint{3.669973in}{5.409902in}}%
\pgfpathlineto{\pgfqpoint{3.670394in}{5.414511in}}%
\pgfpathlineto{\pgfqpoint{3.670815in}{5.415579in}}%
\pgfpathlineto{\pgfqpoint{3.671236in}{5.415118in}}%
\pgfpathlineto{\pgfqpoint{3.671658in}{5.413065in}}%
\pgfpathlineto{\pgfqpoint{3.672079in}{5.416166in}}%
\pgfpathlineto{\pgfqpoint{3.672921in}{5.416166in}}%
\pgfpathlineto{\pgfqpoint{3.674185in}{5.415202in}}%
\pgfpathlineto{\pgfqpoint{3.674607in}{5.415181in}}%
\pgfpathlineto{\pgfqpoint{3.675449in}{5.413421in}}%
\pgfpathlineto{\pgfqpoint{3.677556in}{5.423038in}}%
\pgfpathlineto{\pgfqpoint{3.677977in}{5.420189in}}%
\pgfpathlineto{\pgfqpoint{3.678398in}{5.419015in}}%
\pgfpathlineto{\pgfqpoint{3.678819in}{5.421676in}}%
\pgfpathlineto{\pgfqpoint{3.679241in}{5.418869in}}%
\pgfpathlineto{\pgfqpoint{3.681768in}{5.403700in}}%
\pgfpathlineto{\pgfqpoint{3.683875in}{5.431921in}}%
\pgfpathlineto{\pgfqpoint{3.684296in}{5.430077in}}%
\pgfpathlineto{\pgfqpoint{3.688509in}{5.369885in}}%
\pgfpathlineto{\pgfqpoint{3.688930in}{5.370807in}}%
\pgfpathlineto{\pgfqpoint{3.691036in}{5.377616in}}%
\pgfpathlineto{\pgfqpoint{3.692300in}{5.378014in}}%
\pgfpathlineto{\pgfqpoint{3.693564in}{5.379774in}}%
\pgfpathlineto{\pgfqpoint{3.694828in}{5.381471in}}%
\pgfpathlineto{\pgfqpoint{3.695670in}{5.372420in}}%
\pgfpathlineto{\pgfqpoint{3.696513in}{5.374159in}}%
\pgfpathlineto{\pgfqpoint{3.696934in}{5.374851in}}%
\pgfpathlineto{\pgfqpoint{3.698198in}{5.365423in}}%
\pgfpathlineto{\pgfqpoint{3.698619in}{5.377407in}}%
\pgfpathlineto{\pgfqpoint{3.699041in}{5.369780in}}%
\pgfpathlineto{\pgfqpoint{3.699462in}{5.367895in}}%
\pgfpathlineto{\pgfqpoint{3.699883in}{5.369906in}}%
\pgfpathlineto{\pgfqpoint{3.700305in}{5.378726in}}%
\pgfpathlineto{\pgfqpoint{3.700726in}{5.370053in}}%
\pgfpathlineto{\pgfqpoint{3.701147in}{5.368083in}}%
\pgfpathlineto{\pgfqpoint{3.701990in}{5.394922in}}%
\pgfpathlineto{\pgfqpoint{3.702832in}{5.385557in}}%
\pgfpathlineto{\pgfqpoint{3.703253in}{5.384509in}}%
\pgfpathlineto{\pgfqpoint{3.703675in}{5.380298in}}%
\pgfpathlineto{\pgfqpoint{3.704096in}{5.360688in}}%
\pgfpathlineto{\pgfqpoint{3.704939in}{5.372672in}}%
\pgfpathlineto{\pgfqpoint{3.708730in}{5.440804in}}%
\pgfpathlineto{\pgfqpoint{3.710415in}{5.422053in}}%
\pgfpathlineto{\pgfqpoint{3.710836in}{5.406927in}}%
\pgfpathlineto{\pgfqpoint{3.711258in}{5.415014in}}%
\pgfpathlineto{\pgfqpoint{3.713364in}{5.446314in}}%
\pgfpathlineto{\pgfqpoint{3.713785in}{5.448577in}}%
\pgfpathlineto{\pgfqpoint{3.715049in}{5.471435in}}%
\pgfpathlineto{\pgfqpoint{3.715471in}{5.462111in}}%
\pgfpathlineto{\pgfqpoint{3.716734in}{5.454883in}}%
\pgfpathlineto{\pgfqpoint{3.717156in}{5.454841in}}%
\pgfpathlineto{\pgfqpoint{3.717577in}{5.452809in}}%
\pgfpathlineto{\pgfqpoint{3.719683in}{5.391255in}}%
\pgfpathlineto{\pgfqpoint{3.723475in}{5.214345in}}%
\pgfpathlineto{\pgfqpoint{3.726003in}{5.155368in}}%
\pgfpathlineto{\pgfqpoint{3.726424in}{5.154739in}}%
\pgfpathlineto{\pgfqpoint{3.728109in}{5.106426in}}%
\pgfpathlineto{\pgfqpoint{3.729373in}{5.114744in}}%
\pgfpathlineto{\pgfqpoint{3.730637in}{5.114723in}}%
\pgfpathlineto{\pgfqpoint{3.731479in}{5.125136in}}%
\pgfpathlineto{\pgfqpoint{3.732743in}{5.160585in}}%
\pgfpathlineto{\pgfqpoint{3.734849in}{5.292869in}}%
\pgfpathlineto{\pgfqpoint{3.736534in}{5.344304in}}%
\pgfpathlineto{\pgfqpoint{3.741590in}{5.462949in}}%
\pgfpathlineto{\pgfqpoint{3.742854in}{5.454318in}}%
\pgfpathlineto{\pgfqpoint{3.747488in}{5.454192in}}%
\pgfpathlineto{\pgfqpoint{3.750437in}{5.453857in}}%
\pgfpathlineto{\pgfqpoint{3.757177in}{5.453270in}}%
\pgfpathlineto{\pgfqpoint{3.758862in}{5.395068in}}%
\pgfpathlineto{\pgfqpoint{3.762232in}{5.145039in}}%
\pgfpathlineto{\pgfqpoint{3.771922in}{4.478818in}}%
\pgfpathlineto{\pgfqpoint{3.772764in}{4.472512in}}%
\pgfpathlineto{\pgfqpoint{3.773186in}{4.477059in}}%
\pgfpathlineto{\pgfqpoint{3.774028in}{4.508317in}}%
\pgfpathlineto{\pgfqpoint{3.784560in}{5.071480in}}%
\pgfpathlineto{\pgfqpoint{3.786667in}{5.176612in}}%
\pgfpathlineto{\pgfqpoint{3.787930in}{5.215560in}}%
\pgfpathlineto{\pgfqpoint{3.788352in}{5.214534in}}%
\pgfpathlineto{\pgfqpoint{3.789194in}{5.188177in}}%
\pgfpathlineto{\pgfqpoint{3.793407in}{5.052289in}}%
\pgfpathlineto{\pgfqpoint{3.797620in}{4.985434in}}%
\pgfpathlineto{\pgfqpoint{3.799305in}{5.016693in}}%
\pgfpathlineto{\pgfqpoint{3.799726in}{5.015038in}}%
\pgfpathlineto{\pgfqpoint{3.802254in}{4.993731in}}%
\pgfpathlineto{\pgfqpoint{3.802675in}{4.995219in}}%
\pgfpathlineto{\pgfqpoint{3.804360in}{5.037791in}}%
\pgfpathlineto{\pgfqpoint{3.807730in}{5.115980in}}%
\pgfpathlineto{\pgfqpoint{3.809837in}{5.150088in}}%
\pgfpathlineto{\pgfqpoint{3.811943in}{5.208227in}}%
\pgfpathlineto{\pgfqpoint{3.814050in}{5.182164in}}%
\pgfpathlineto{\pgfqpoint{3.816156in}{5.143070in}}%
\pgfpathlineto{\pgfqpoint{3.817841in}{5.129012in}}%
\pgfpathlineto{\pgfqpoint{3.821211in}{5.074476in}}%
\pgfpathlineto{\pgfqpoint{3.822054in}{5.088492in}}%
\pgfpathlineto{\pgfqpoint{3.823318in}{5.106887in}}%
\pgfpathlineto{\pgfqpoint{3.823739in}{5.103975in}}%
\pgfpathlineto{\pgfqpoint{3.827109in}{5.061675in}}%
\pgfpathlineto{\pgfqpoint{3.828794in}{5.141079in}}%
\pgfpathlineto{\pgfqpoint{3.831322in}{5.216524in}}%
\pgfpathlineto{\pgfqpoint{3.835535in}{5.323311in}}%
\pgfpathlineto{\pgfqpoint{3.835956in}{5.320399in}}%
\pgfpathlineto{\pgfqpoint{3.837641in}{5.313066in}}%
\pgfpathlineto{\pgfqpoint{3.838905in}{5.300076in}}%
\pgfpathlineto{\pgfqpoint{3.839326in}{5.300265in}}%
\pgfpathlineto{\pgfqpoint{3.841011in}{5.354591in}}%
\pgfpathlineto{\pgfqpoint{3.841433in}{5.347887in}}%
\pgfpathlineto{\pgfqpoint{3.843539in}{5.310301in}}%
\pgfpathlineto{\pgfqpoint{3.843960in}{5.309672in}}%
\pgfpathlineto{\pgfqpoint{3.845224in}{5.300056in}}%
\pgfpathlineto{\pgfqpoint{3.846909in}{5.358739in}}%
\pgfpathlineto{\pgfqpoint{3.847752in}{5.353502in}}%
\pgfpathlineto{\pgfqpoint{3.849858in}{5.288134in}}%
\pgfpathlineto{\pgfqpoint{3.851122in}{5.229576in}}%
\pgfpathlineto{\pgfqpoint{3.851543in}{5.230310in}}%
\pgfpathlineto{\pgfqpoint{3.852807in}{5.254068in}}%
\pgfpathlineto{\pgfqpoint{3.853229in}{5.242922in}}%
\pgfpathlineto{\pgfqpoint{3.857441in}{5.137622in}}%
\pgfpathlineto{\pgfqpoint{3.859969in}{5.004374in}}%
\pgfpathlineto{\pgfqpoint{3.861233in}{4.941458in}}%
\pgfpathlineto{\pgfqpoint{3.861654in}{4.945732in}}%
\pgfpathlineto{\pgfqpoint{3.862075in}{4.947157in}}%
\pgfpathlineto{\pgfqpoint{3.864182in}{5.019962in}}%
\pgfpathlineto{\pgfqpoint{3.864603in}{5.036974in}}%
\pgfpathlineto{\pgfqpoint{3.865446in}{5.031296in}}%
\pgfpathlineto{\pgfqpoint{3.874292in}{4.937247in}}%
\pgfpathlineto{\pgfqpoint{3.883561in}{4.541315in}}%
\pgfpathlineto{\pgfqpoint{3.885246in}{4.541776in}}%
\pgfpathlineto{\pgfqpoint{3.886509in}{4.541127in}}%
\pgfpathlineto{\pgfqpoint{3.888616in}{4.540184in}}%
\pgfpathlineto{\pgfqpoint{3.891144in}{4.539870in}}%
\pgfpathlineto{\pgfqpoint{3.892829in}{4.538948in}}%
\pgfpathlineto{\pgfqpoint{3.895356in}{4.550031in}}%
\pgfpathlineto{\pgfqpoint{3.896199in}{4.549570in}}%
\pgfpathlineto{\pgfqpoint{3.899148in}{4.549654in}}%
\pgfpathlineto{\pgfqpoint{3.900833in}{4.550408in}}%
\pgfpathlineto{\pgfqpoint{3.903361in}{4.561931in}}%
\pgfpathlineto{\pgfqpoint{3.904203in}{4.561742in}}%
\pgfpathlineto{\pgfqpoint{3.905888in}{4.561491in}}%
\pgfpathlineto{\pgfqpoint{3.906310in}{4.561575in}}%
\pgfpathlineto{\pgfqpoint{3.908416in}{4.550597in}}%
\pgfpathlineto{\pgfqpoint{3.909258in}{4.553111in}}%
\pgfpathlineto{\pgfqpoint{3.911365in}{4.562413in}}%
\pgfpathlineto{\pgfqpoint{3.913893in}{4.562141in}}%
\pgfpathlineto{\pgfqpoint{3.914314in}{4.562203in}}%
\pgfpathlineto{\pgfqpoint{3.916841in}{4.551099in}}%
\pgfpathlineto{\pgfqpoint{3.917684in}{4.551476in}}%
\pgfpathlineto{\pgfqpoint{3.919369in}{4.552084in}}%
\pgfpathlineto{\pgfqpoint{3.922318in}{4.552063in}}%
\pgfpathlineto{\pgfqpoint{3.925688in}{4.540854in}}%
\pgfpathlineto{\pgfqpoint{3.930322in}{4.542007in}}%
\pgfpathlineto{\pgfqpoint{3.930744in}{4.542551in}}%
\pgfpathlineto{\pgfqpoint{3.935378in}{4.564843in}}%
\pgfpathlineto{\pgfqpoint{3.936220in}{4.565367in}}%
\pgfpathlineto{\pgfqpoint{3.941276in}{4.589796in}}%
\pgfpathlineto{\pgfqpoint{3.941697in}{4.588895in}}%
\pgfpathlineto{\pgfqpoint{3.942118in}{4.589607in}}%
\pgfpathlineto{\pgfqpoint{3.943803in}{4.599098in}}%
\pgfpathlineto{\pgfqpoint{3.944225in}{4.598511in}}%
\pgfpathlineto{\pgfqpoint{3.949280in}{4.575780in}}%
\pgfpathlineto{\pgfqpoint{3.949701in}{4.575361in}}%
\pgfpathlineto{\pgfqpoint{3.954335in}{4.554787in}}%
\pgfpathlineto{\pgfqpoint{3.954757in}{4.554326in}}%
\pgfpathlineto{\pgfqpoint{3.956442in}{4.556505in}}%
\pgfpathlineto{\pgfqpoint{3.956863in}{4.556190in}}%
\pgfpathlineto{\pgfqpoint{3.958127in}{4.554158in}}%
\pgfpathlineto{\pgfqpoint{3.959391in}{4.554053in}}%
\pgfpathlineto{\pgfqpoint{3.963603in}{4.553362in}}%
\pgfpathlineto{\pgfqpoint{3.964867in}{4.552838in}}%
\pgfpathlineto{\pgfqpoint{3.967816in}{4.549863in}}%
\pgfpathlineto{\pgfqpoint{3.968237in}{4.550932in}}%
\pgfpathlineto{\pgfqpoint{3.969923in}{4.551854in}}%
\pgfpathlineto{\pgfqpoint{3.971186in}{4.551267in}}%
\pgfpathlineto{\pgfqpoint{3.981718in}{4.551204in}}%
\pgfpathlineto{\pgfqpoint{3.986352in}{4.513576in}}%
\pgfpathlineto{\pgfqpoint{3.986774in}{4.516635in}}%
\pgfpathlineto{\pgfqpoint{3.991829in}{4.539304in}}%
\pgfpathlineto{\pgfqpoint{3.993935in}{4.586779in}}%
\pgfpathlineto{\pgfqpoint{3.999412in}{4.773620in}}%
\pgfpathlineto{\pgfqpoint{4.005731in}{4.986943in}}%
\pgfpathlineto{\pgfqpoint{4.013314in}{5.148077in}}%
\pgfpathlineto{\pgfqpoint{4.014157in}{5.156038in}}%
\pgfpathlineto{\pgfqpoint{4.014578in}{5.152204in}}%
\pgfpathlineto{\pgfqpoint{4.018791in}{5.101105in}}%
\pgfpathlineto{\pgfqpoint{4.019212in}{5.106154in}}%
\pgfpathlineto{\pgfqpoint{4.020055in}{5.109841in}}%
\pgfpathlineto{\pgfqpoint{4.022161in}{5.084491in}}%
\pgfpathlineto{\pgfqpoint{4.025531in}{5.030521in}}%
\pgfpathlineto{\pgfqpoint{4.026374in}{5.039509in}}%
\pgfpathlineto{\pgfqpoint{4.039855in}{5.227775in}}%
\pgfpathlineto{\pgfqpoint{4.040697in}{5.220484in}}%
\pgfpathlineto{\pgfqpoint{4.041119in}{5.213423in}}%
\pgfpathlineto{\pgfqpoint{4.041540in}{5.216231in}}%
\pgfpathlineto{\pgfqpoint{4.043646in}{5.224255in}}%
\pgfpathlineto{\pgfqpoint{4.044910in}{5.217990in}}%
\pgfpathlineto{\pgfqpoint{4.047859in}{5.201209in}}%
\pgfpathlineto{\pgfqpoint{4.052072in}{5.224946in}}%
\pgfpathlineto{\pgfqpoint{4.052493in}{5.228634in}}%
\pgfpathlineto{\pgfqpoint{4.052914in}{5.223312in}}%
\pgfpathlineto{\pgfqpoint{4.053336in}{5.222202in}}%
\pgfpathlineto{\pgfqpoint{4.054599in}{5.235945in}}%
\pgfpathlineto{\pgfqpoint{4.055863in}{5.245499in}}%
\pgfpathlineto{\pgfqpoint{4.056285in}{5.243278in}}%
\pgfpathlineto{\pgfqpoint{4.057548in}{5.250485in}}%
\pgfpathlineto{\pgfqpoint{4.057970in}{5.247322in}}%
\pgfpathlineto{\pgfqpoint{4.058391in}{5.251261in}}%
\pgfpathlineto{\pgfqpoint{4.059234in}{5.252581in}}%
\pgfpathlineto{\pgfqpoint{4.060497in}{5.248872in}}%
\pgfpathlineto{\pgfqpoint{4.061340in}{5.249417in}}%
\pgfpathlineto{\pgfqpoint{4.065974in}{5.281870in}}%
\pgfpathlineto{\pgfqpoint{4.066395in}{5.280781in}}%
\pgfpathlineto{\pgfqpoint{4.067238in}{5.281430in}}%
\pgfpathlineto{\pgfqpoint{4.067659in}{5.281556in}}%
\pgfpathlineto{\pgfqpoint{4.068502in}{5.274663in}}%
\pgfpathlineto{\pgfqpoint{4.068923in}{5.275208in}}%
\pgfpathlineto{\pgfqpoint{4.069344in}{5.276276in}}%
\pgfpathlineto{\pgfqpoint{4.069765in}{5.272631in}}%
\pgfpathlineto{\pgfqpoint{4.070608in}{5.273113in}}%
\pgfpathlineto{\pgfqpoint{4.071029in}{5.274370in}}%
\pgfpathlineto{\pgfqpoint{4.071451in}{5.272233in}}%
\pgfpathlineto{\pgfqpoint{4.072714in}{5.273553in}}%
\pgfpathlineto{\pgfqpoint{4.073136in}{5.272421in}}%
\pgfpathlineto{\pgfqpoint{4.075663in}{5.300516in}}%
\pgfpathlineto{\pgfqpoint{4.077348in}{5.271625in}}%
\pgfpathlineto{\pgfqpoint{4.078612in}{5.230687in}}%
\pgfpathlineto{\pgfqpoint{4.080297in}{5.179252in}}%
\pgfpathlineto{\pgfqpoint{4.080719in}{5.180761in}}%
\pgfpathlineto{\pgfqpoint{4.081561in}{5.185684in}}%
\pgfpathlineto{\pgfqpoint{4.083246in}{5.195908in}}%
\pgfpathlineto{\pgfqpoint{4.085774in}{5.178393in}}%
\pgfpathlineto{\pgfqpoint{4.086195in}{5.173784in}}%
\pgfpathlineto{\pgfqpoint{4.086617in}{5.177786in}}%
\pgfpathlineto{\pgfqpoint{4.088302in}{5.181578in}}%
\pgfpathlineto{\pgfqpoint{4.089566in}{5.168001in}}%
\pgfpathlineto{\pgfqpoint{4.090408in}{5.171354in}}%
\pgfpathlineto{\pgfqpoint{4.092514in}{5.161046in}}%
\pgfpathlineto{\pgfqpoint{4.092936in}{5.163623in}}%
\pgfpathlineto{\pgfqpoint{4.093778in}{5.182835in}}%
\pgfpathlineto{\pgfqpoint{4.094200in}{5.182751in}}%
\pgfpathlineto{\pgfqpoint{4.095042in}{5.187926in}}%
\pgfpathlineto{\pgfqpoint{4.096306in}{5.173993in}}%
\pgfpathlineto{\pgfqpoint{4.096727in}{5.179545in}}%
\pgfpathlineto{\pgfqpoint{4.097149in}{5.175062in}}%
\pgfpathlineto{\pgfqpoint{4.099255in}{5.161842in}}%
\pgfpathlineto{\pgfqpoint{4.100097in}{5.188491in}}%
\pgfpathlineto{\pgfqpoint{4.100940in}{5.187821in}}%
\pgfpathlineto{\pgfqpoint{4.101361in}{5.192703in}}%
\pgfpathlineto{\pgfqpoint{4.101783in}{5.188156in}}%
\pgfpathlineto{\pgfqpoint{4.102625in}{5.176759in}}%
\pgfpathlineto{\pgfqpoint{4.103046in}{5.182039in}}%
\pgfpathlineto{\pgfqpoint{4.105574in}{5.162847in}}%
\pgfpathlineto{\pgfqpoint{4.105995in}{5.169426in}}%
\pgfpathlineto{\pgfqpoint{4.106417in}{5.187591in}}%
\pgfpathlineto{\pgfqpoint{4.107259in}{5.186836in}}%
\pgfpathlineto{\pgfqpoint{4.108102in}{5.189895in}}%
\pgfpathlineto{\pgfqpoint{4.108944in}{5.173658in}}%
\pgfpathlineto{\pgfqpoint{4.109366in}{5.176319in}}%
\pgfpathlineto{\pgfqpoint{4.109787in}{5.177597in}}%
\pgfpathlineto{\pgfqpoint{4.112315in}{5.158196in}}%
\pgfpathlineto{\pgfqpoint{4.113157in}{5.175544in}}%
\pgfpathlineto{\pgfqpoint{4.113578in}{5.171228in}}%
\pgfpathlineto{\pgfqpoint{4.115263in}{5.155892in}}%
\pgfpathlineto{\pgfqpoint{4.115685in}{5.156876in}}%
\pgfpathlineto{\pgfqpoint{4.119476in}{5.244054in}}%
\pgfpathlineto{\pgfqpoint{4.120319in}{5.234165in}}%
\pgfpathlineto{\pgfqpoint{4.121583in}{5.218368in}}%
\pgfpathlineto{\pgfqpoint{4.122004in}{5.219185in}}%
\pgfpathlineto{\pgfqpoint{4.124532in}{5.259117in}}%
\pgfpathlineto{\pgfqpoint{4.126217in}{5.307745in}}%
\pgfpathlineto{\pgfqpoint{4.127059in}{5.303554in}}%
\pgfpathlineto{\pgfqpoint{4.127902in}{5.301836in}}%
\pgfpathlineto{\pgfqpoint{4.128744in}{5.299008in}}%
\pgfpathlineto{\pgfqpoint{4.129587in}{5.301313in}}%
\pgfpathlineto{\pgfqpoint{4.130008in}{5.299825in}}%
\pgfpathlineto{\pgfqpoint{4.130430in}{5.298715in}}%
\pgfpathlineto{\pgfqpoint{4.131272in}{5.303387in}}%
\pgfpathlineto{\pgfqpoint{4.131693in}{5.302193in}}%
\pgfpathlineto{\pgfqpoint{4.132115in}{5.300475in}}%
\pgfpathlineto{\pgfqpoint{4.132536in}{5.304811in}}%
\pgfpathlineto{\pgfqpoint{4.133378in}{5.302172in}}%
\pgfpathlineto{\pgfqpoint{4.133800in}{5.301250in}}%
\pgfpathlineto{\pgfqpoint{4.134221in}{5.305943in}}%
\pgfpathlineto{\pgfqpoint{4.134642in}{5.301522in}}%
\pgfpathlineto{\pgfqpoint{4.137170in}{5.291361in}}%
\pgfpathlineto{\pgfqpoint{4.137591in}{5.295446in}}%
\pgfpathlineto{\pgfqpoint{4.138013in}{5.290439in}}%
\pgfpathlineto{\pgfqpoint{4.138855in}{5.282059in}}%
\pgfpathlineto{\pgfqpoint{4.140119in}{5.299972in}}%
\pgfpathlineto{\pgfqpoint{4.140540in}{5.299197in}}%
\pgfpathlineto{\pgfqpoint{4.140961in}{5.303533in}}%
\pgfpathlineto{\pgfqpoint{4.141383in}{5.300537in}}%
\pgfpathlineto{\pgfqpoint{4.142225in}{5.300349in}}%
\pgfpathlineto{\pgfqpoint{4.142647in}{5.304246in}}%
\pgfpathlineto{\pgfqpoint{4.143489in}{5.301899in}}%
\pgfpathlineto{\pgfqpoint{4.143910in}{5.302193in}}%
\pgfpathlineto{\pgfqpoint{4.144332in}{5.304015in}}%
\pgfpathlineto{\pgfqpoint{4.144753in}{5.300202in}}%
\pgfpathlineto{\pgfqpoint{4.145596in}{5.301962in}}%
\pgfpathlineto{\pgfqpoint{4.146017in}{5.303638in}}%
\pgfpathlineto{\pgfqpoint{4.146438in}{5.301292in}}%
\pgfpathlineto{\pgfqpoint{4.147281in}{5.301983in}}%
\pgfpathlineto{\pgfqpoint{4.151493in}{5.316313in}}%
\pgfpathlineto{\pgfqpoint{4.152336in}{5.325637in}}%
\pgfpathlineto{\pgfqpoint{4.152757in}{5.319519in}}%
\pgfpathlineto{\pgfqpoint{4.153600in}{5.306508in}}%
\pgfpathlineto{\pgfqpoint{4.154442in}{5.308205in}}%
\pgfpathlineto{\pgfqpoint{4.155285in}{5.305461in}}%
\pgfpathlineto{\pgfqpoint{4.155706in}{5.309588in}}%
\pgfpathlineto{\pgfqpoint{4.156127in}{5.307095in}}%
\pgfpathlineto{\pgfqpoint{4.157391in}{5.303072in}}%
\pgfpathlineto{\pgfqpoint{4.160340in}{5.262050in}}%
\pgfpathlineto{\pgfqpoint{4.160762in}{5.259809in}}%
\pgfpathlineto{\pgfqpoint{4.162868in}{5.233264in}}%
\pgfpathlineto{\pgfqpoint{4.164132in}{5.223228in}}%
\pgfpathlineto{\pgfqpoint{4.165396in}{5.198820in}}%
\pgfpathlineto{\pgfqpoint{4.165817in}{5.207683in}}%
\pgfpathlineto{\pgfqpoint{4.167081in}{5.217404in}}%
\pgfpathlineto{\pgfqpoint{4.167502in}{5.216629in}}%
\pgfpathlineto{\pgfqpoint{4.170872in}{5.167289in}}%
\pgfpathlineto{\pgfqpoint{4.171715in}{5.161548in}}%
\pgfpathlineto{\pgfqpoint{4.172136in}{5.163371in}}%
\pgfpathlineto{\pgfqpoint{4.172557in}{5.167163in}}%
\pgfpathlineto{\pgfqpoint{4.172979in}{5.163811in}}%
\pgfpathlineto{\pgfqpoint{4.173400in}{5.160857in}}%
\pgfpathlineto{\pgfqpoint{4.173821in}{5.163392in}}%
\pgfpathlineto{\pgfqpoint{4.175506in}{5.168986in}}%
\pgfpathlineto{\pgfqpoint{4.177613in}{5.147260in}}%
\pgfpathlineto{\pgfqpoint{4.178034in}{5.145626in}}%
\pgfpathlineto{\pgfqpoint{4.178455in}{5.146108in}}%
\pgfpathlineto{\pgfqpoint{4.178876in}{5.152183in}}%
\pgfpathlineto{\pgfqpoint{4.181404in}{5.108668in}}%
\pgfpathlineto{\pgfqpoint{4.181825in}{5.108605in}}%
\pgfpathlineto{\pgfqpoint{4.182247in}{5.112188in}}%
\pgfpathlineto{\pgfqpoint{4.182668in}{5.109569in}}%
\pgfpathlineto{\pgfqpoint{4.183511in}{5.109674in}}%
\pgfpathlineto{\pgfqpoint{4.183932in}{5.113173in}}%
\pgfpathlineto{\pgfqpoint{4.184353in}{5.110449in}}%
\pgfpathlineto{\pgfqpoint{4.185196in}{5.111161in}}%
\pgfpathlineto{\pgfqpoint{4.185617in}{5.116755in}}%
\pgfpathlineto{\pgfqpoint{4.186038in}{5.111413in}}%
\pgfpathlineto{\pgfqpoint{4.186460in}{5.111182in}}%
\pgfpathlineto{\pgfqpoint{4.186881in}{5.112418in}}%
\pgfpathlineto{\pgfqpoint{4.187302in}{5.117174in}}%
\pgfpathlineto{\pgfqpoint{4.187723in}{5.111643in}}%
\pgfpathlineto{\pgfqpoint{4.188566in}{5.111706in}}%
\pgfpathlineto{\pgfqpoint{4.188987in}{5.114283in}}%
\pgfpathlineto{\pgfqpoint{4.189408in}{5.111329in}}%
\pgfpathlineto{\pgfqpoint{4.190251in}{5.112104in}}%
\pgfpathlineto{\pgfqpoint{4.190672in}{5.113696in}}%
\pgfpathlineto{\pgfqpoint{4.191515in}{5.107683in}}%
\pgfpathlineto{\pgfqpoint{4.191936in}{5.111685in}}%
\pgfpathlineto{\pgfqpoint{4.195306in}{5.180048in}}%
\pgfpathlineto{\pgfqpoint{4.196570in}{5.152456in}}%
\pgfpathlineto{\pgfqpoint{4.202889in}{4.826856in}}%
\pgfpathlineto{\pgfqpoint{4.206681in}{4.699726in}}%
\pgfpathlineto{\pgfqpoint{4.210472in}{4.650805in}}%
\pgfpathlineto{\pgfqpoint{4.213000in}{4.651224in}}%
\pgfpathlineto{\pgfqpoint{4.213421in}{4.652020in}}%
\pgfpathlineto{\pgfqpoint{4.218477in}{4.685730in}}%
\pgfpathlineto{\pgfqpoint{4.223532in}{4.503792in}}%
\pgfpathlineto{\pgfqpoint{4.228587in}{4.540037in}}%
\pgfpathlineto{\pgfqpoint{4.230272in}{4.542279in}}%
\pgfpathlineto{\pgfqpoint{4.232379in}{4.546993in}}%
\pgfpathlineto{\pgfqpoint{4.233221in}{4.551497in}}%
\pgfpathlineto{\pgfqpoint{4.234064in}{4.551288in}}%
\pgfpathlineto{\pgfqpoint{4.236592in}{4.550387in}}%
\pgfpathlineto{\pgfqpoint{4.237855in}{4.551141in}}%
\pgfpathlineto{\pgfqpoint{4.238698in}{4.552880in}}%
\pgfpathlineto{\pgfqpoint{4.240804in}{4.546742in}}%
\pgfpathlineto{\pgfqpoint{4.241226in}{4.550198in}}%
\pgfpathlineto{\pgfqpoint{4.248809in}{4.890191in}}%
\pgfpathlineto{\pgfqpoint{4.253021in}{5.084910in}}%
\pgfpathlineto{\pgfqpoint{4.255549in}{5.151199in}}%
\pgfpathlineto{\pgfqpoint{4.261447in}{5.221364in}}%
\pgfpathlineto{\pgfqpoint{4.261868in}{5.216943in}}%
\pgfpathlineto{\pgfqpoint{4.262290in}{5.215120in}}%
\pgfpathlineto{\pgfqpoint{4.267345in}{5.293184in}}%
\pgfpathlineto{\pgfqpoint{4.268187in}{5.296180in}}%
\pgfpathlineto{\pgfqpoint{4.269030in}{5.280697in}}%
\pgfpathlineto{\pgfqpoint{4.271558in}{5.185139in}}%
\pgfpathlineto{\pgfqpoint{4.283353in}{4.687511in}}%
\pgfpathlineto{\pgfqpoint{4.283775in}{4.688224in}}%
\pgfpathlineto{\pgfqpoint{4.285460in}{4.690340in}}%
\pgfpathlineto{\pgfqpoint{4.286302in}{4.689837in}}%
\pgfpathlineto{\pgfqpoint{4.286724in}{4.690298in}}%
\pgfpathlineto{\pgfqpoint{4.289251in}{4.691115in}}%
\pgfpathlineto{\pgfqpoint{4.291358in}{4.686212in}}%
\pgfpathlineto{\pgfqpoint{4.292200in}{4.690193in}}%
\pgfpathlineto{\pgfqpoint{4.292622in}{4.690067in}}%
\pgfpathlineto{\pgfqpoint{4.293464in}{4.689711in}}%
\pgfpathlineto{\pgfqpoint{4.294728in}{4.690507in}}%
\pgfpathlineto{\pgfqpoint{4.295571in}{4.691031in}}%
\pgfpathlineto{\pgfqpoint{4.295992in}{4.692393in}}%
\pgfpathlineto{\pgfqpoint{4.296413in}{4.691827in}}%
\pgfpathlineto{\pgfqpoint{4.296834in}{4.688978in}}%
\pgfpathlineto{\pgfqpoint{4.303996in}{4.936577in}}%
\pgfpathlineto{\pgfqpoint{4.307366in}{5.066054in}}%
\pgfpathlineto{\pgfqpoint{4.310315in}{5.171982in}}%
\pgfpathlineto{\pgfqpoint{4.310737in}{5.166975in}}%
\pgfpathlineto{\pgfqpoint{4.311158in}{5.167666in}}%
\pgfpathlineto{\pgfqpoint{4.314949in}{5.208248in}}%
\pgfpathlineto{\pgfqpoint{4.317056in}{5.240408in}}%
\pgfpathlineto{\pgfqpoint{4.317477in}{5.238690in}}%
\pgfpathlineto{\pgfqpoint{4.317898in}{5.231839in}}%
\pgfpathlineto{\pgfqpoint{4.318320in}{5.237747in}}%
\pgfpathlineto{\pgfqpoint{4.320847in}{5.290188in}}%
\pgfpathlineto{\pgfqpoint{4.321690in}{5.296117in}}%
\pgfpathlineto{\pgfqpoint{4.323796in}{5.317906in}}%
\pgfpathlineto{\pgfqpoint{4.324217in}{5.317382in}}%
\pgfpathlineto{\pgfqpoint{4.327166in}{5.201083in}}%
\pgfpathlineto{\pgfqpoint{4.338962in}{4.689334in}}%
\pgfpathlineto{\pgfqpoint{4.339383in}{4.691597in}}%
\pgfpathlineto{\pgfqpoint{4.341069in}{4.694886in}}%
\pgfpathlineto{\pgfqpoint{4.343596in}{4.694509in}}%
\pgfpathlineto{\pgfqpoint{4.345281in}{4.693671in}}%
\pgfpathlineto{\pgfqpoint{4.346966in}{4.688224in}}%
\pgfpathlineto{\pgfqpoint{4.348230in}{4.692791in}}%
\pgfpathlineto{\pgfqpoint{4.348652in}{4.692644in}}%
\pgfpathlineto{\pgfqpoint{4.351179in}{4.691701in}}%
\pgfpathlineto{\pgfqpoint{4.352022in}{4.693692in}}%
\pgfpathlineto{\pgfqpoint{4.352443in}{4.690277in}}%
\pgfpathlineto{\pgfqpoint{4.354128in}{4.747578in}}%
\pgfpathlineto{\pgfqpoint{4.363396in}{5.078897in}}%
\pgfpathlineto{\pgfqpoint{4.365924in}{5.169238in}}%
\pgfpathlineto{\pgfqpoint{4.366345in}{5.164880in}}%
\pgfpathlineto{\pgfqpoint{4.367609in}{5.144913in}}%
\pgfpathlineto{\pgfqpoint{4.368452in}{5.145751in}}%
\pgfpathlineto{\pgfqpoint{4.368873in}{5.145395in}}%
\pgfpathlineto{\pgfqpoint{4.370137in}{5.133244in}}%
\pgfpathlineto{\pgfqpoint{4.371822in}{5.122391in}}%
\pgfpathlineto{\pgfqpoint{4.372243in}{5.122601in}}%
\pgfpathlineto{\pgfqpoint{4.373086in}{5.116734in}}%
\pgfpathlineto{\pgfqpoint{4.373928in}{5.099471in}}%
\pgfpathlineto{\pgfqpoint{4.374350in}{5.105274in}}%
\pgfpathlineto{\pgfqpoint{4.376035in}{5.111182in}}%
\pgfpathlineto{\pgfqpoint{4.376456in}{5.110617in}}%
\pgfpathlineto{\pgfqpoint{4.380669in}{5.062911in}}%
\pgfpathlineto{\pgfqpoint{4.381933in}{5.072172in}}%
\pgfpathlineto{\pgfqpoint{4.382775in}{5.074853in}}%
\pgfpathlineto{\pgfqpoint{4.384460in}{5.086879in}}%
\pgfpathlineto{\pgfqpoint{4.386145in}{5.097480in}}%
\pgfpathlineto{\pgfqpoint{4.386988in}{5.099429in}}%
\pgfpathlineto{\pgfqpoint{4.387830in}{5.086921in}}%
\pgfpathlineto{\pgfqpoint{4.388252in}{5.091928in}}%
\pgfpathlineto{\pgfqpoint{4.392043in}{5.120568in}}%
\pgfpathlineto{\pgfqpoint{4.393728in}{5.154027in}}%
\pgfpathlineto{\pgfqpoint{4.394571in}{5.144830in}}%
\pgfpathlineto{\pgfqpoint{4.399626in}{5.100895in}}%
\pgfpathlineto{\pgfqpoint{4.400048in}{5.101147in}}%
\pgfpathlineto{\pgfqpoint{4.400890in}{5.096684in}}%
\pgfpathlineto{\pgfqpoint{4.401733in}{5.086942in}}%
\pgfpathlineto{\pgfqpoint{4.402154in}{5.092913in}}%
\pgfpathlineto{\pgfqpoint{4.403839in}{5.098297in}}%
\pgfpathlineto{\pgfqpoint{4.404260in}{5.097753in}}%
\pgfpathlineto{\pgfqpoint{4.408473in}{5.047512in}}%
\pgfpathlineto{\pgfqpoint{4.411001in}{5.078624in}}%
\pgfpathlineto{\pgfqpoint{4.411843in}{5.083401in}}%
\pgfpathlineto{\pgfqpoint{4.413950in}{5.124507in}}%
\pgfpathlineto{\pgfqpoint{4.414792in}{5.128970in}}%
\pgfpathlineto{\pgfqpoint{4.415635in}{5.119647in}}%
\pgfpathlineto{\pgfqpoint{4.419426in}{5.198150in}}%
\pgfpathlineto{\pgfqpoint{4.423639in}{5.339129in}}%
\pgfpathlineto{\pgfqpoint{4.425324in}{5.365653in}}%
\pgfpathlineto{\pgfqpoint{4.425745in}{5.360541in}}%
\pgfpathlineto{\pgfqpoint{4.426167in}{5.363055in}}%
\pgfpathlineto{\pgfqpoint{4.427852in}{5.381639in}}%
\pgfpathlineto{\pgfqpoint{4.428273in}{5.381283in}}%
\pgfpathlineto{\pgfqpoint{4.428694in}{5.378287in}}%
\pgfpathlineto{\pgfqpoint{4.429958in}{5.359410in}}%
\pgfpathlineto{\pgfqpoint{4.430380in}{5.361610in}}%
\pgfpathlineto{\pgfqpoint{4.431222in}{5.354591in}}%
\pgfpathlineto{\pgfqpoint{4.431643in}{5.356225in}}%
\pgfpathlineto{\pgfqpoint{4.432065in}{5.356602in}}%
\pgfpathlineto{\pgfqpoint{4.432907in}{5.341685in}}%
\pgfpathlineto{\pgfqpoint{4.433328in}{5.345729in}}%
\pgfpathlineto{\pgfqpoint{4.434592in}{5.353648in}}%
\pgfpathlineto{\pgfqpoint{4.435435in}{5.356141in}}%
\pgfpathlineto{\pgfqpoint{4.436277in}{5.340889in}}%
\pgfpathlineto{\pgfqpoint{4.437120in}{5.348683in}}%
\pgfpathlineto{\pgfqpoint{4.438805in}{5.354675in}}%
\pgfpathlineto{\pgfqpoint{4.439226in}{5.354340in}}%
\pgfpathlineto{\pgfqpoint{4.440069in}{5.344723in}}%
\pgfpathlineto{\pgfqpoint{4.442597in}{5.367706in}}%
\pgfpathlineto{\pgfqpoint{4.443018in}{5.351281in}}%
\pgfpathlineto{\pgfqpoint{4.443860in}{5.355492in}}%
\pgfpathlineto{\pgfqpoint{4.444282in}{5.358467in}}%
\pgfpathlineto{\pgfqpoint{4.444703in}{5.356225in}}%
\pgfpathlineto{\pgfqpoint{4.446809in}{5.331168in}}%
\pgfpathlineto{\pgfqpoint{4.447652in}{5.337914in}}%
\pgfpathlineto{\pgfqpoint{4.449337in}{5.333179in}}%
\pgfpathlineto{\pgfqpoint{4.450180in}{5.315434in}}%
\pgfpathlineto{\pgfqpoint{4.451022in}{5.323269in}}%
\pgfpathlineto{\pgfqpoint{4.453129in}{5.329932in}}%
\pgfpathlineto{\pgfqpoint{4.453550in}{5.322452in}}%
\pgfpathlineto{\pgfqpoint{4.453971in}{5.325406in}}%
\pgfpathlineto{\pgfqpoint{4.456499in}{5.369215in}}%
\pgfpathlineto{\pgfqpoint{4.456920in}{5.355178in}}%
\pgfpathlineto{\pgfqpoint{4.457763in}{5.363013in}}%
\pgfpathlineto{\pgfqpoint{4.459869in}{5.377826in}}%
\pgfpathlineto{\pgfqpoint{4.460712in}{5.371205in}}%
\pgfpathlineto{\pgfqpoint{4.463239in}{5.408330in}}%
\pgfpathlineto{\pgfqpoint{4.463661in}{5.407911in}}%
\pgfpathlineto{\pgfqpoint{4.464082in}{5.397729in}}%
\pgfpathlineto{\pgfqpoint{4.464503in}{5.399405in}}%
\pgfpathlineto{\pgfqpoint{4.466188in}{5.414322in}}%
\pgfpathlineto{\pgfqpoint{4.467031in}{5.414490in}}%
\pgfpathlineto{\pgfqpoint{4.467873in}{5.400704in}}%
\pgfpathlineto{\pgfqpoint{4.468295in}{5.406968in}}%
\pgfpathlineto{\pgfqpoint{4.470401in}{5.427563in}}%
\pgfpathlineto{\pgfqpoint{4.471244in}{5.413170in}}%
\pgfpathlineto{\pgfqpoint{4.471665in}{5.417737in}}%
\pgfpathlineto{\pgfqpoint{4.472086in}{5.418450in}}%
\pgfpathlineto{\pgfqpoint{4.472507in}{5.418031in}}%
\pgfpathlineto{\pgfqpoint{4.472929in}{5.416145in}}%
\pgfpathlineto{\pgfqpoint{4.473350in}{5.417318in}}%
\pgfpathlineto{\pgfqpoint{4.473771in}{5.418575in}}%
\pgfpathlineto{\pgfqpoint{4.474614in}{5.403951in}}%
\pgfpathlineto{\pgfqpoint{4.475035in}{5.406487in}}%
\pgfpathlineto{\pgfqpoint{4.476720in}{5.423268in}}%
\pgfpathlineto{\pgfqpoint{4.477141in}{5.424253in}}%
\pgfpathlineto{\pgfqpoint{4.477563in}{5.422640in}}%
\pgfpathlineto{\pgfqpoint{4.478405in}{5.409671in}}%
\pgfpathlineto{\pgfqpoint{4.480090in}{5.424337in}}%
\pgfpathlineto{\pgfqpoint{4.480933in}{5.424421in}}%
\pgfpathlineto{\pgfqpoint{4.481775in}{5.409671in}}%
\pgfpathlineto{\pgfqpoint{4.482197in}{5.415076in}}%
\pgfpathlineto{\pgfqpoint{4.483882in}{5.429491in}}%
\pgfpathlineto{\pgfqpoint{4.484303in}{5.434268in}}%
\pgfpathlineto{\pgfqpoint{4.485146in}{5.417737in}}%
\pgfpathlineto{\pgfqpoint{4.485567in}{5.423645in}}%
\pgfpathlineto{\pgfqpoint{4.486410in}{5.424588in}}%
\pgfpathlineto{\pgfqpoint{4.488095in}{5.405963in}}%
\pgfpathlineto{\pgfqpoint{4.488516in}{5.392743in}}%
\pgfpathlineto{\pgfqpoint{4.488937in}{5.393748in}}%
\pgfpathlineto{\pgfqpoint{4.490622in}{5.408267in}}%
\pgfpathlineto{\pgfqpoint{4.491044in}{5.408246in}}%
\pgfpathlineto{\pgfqpoint{4.491465in}{5.405711in}}%
\pgfpathlineto{\pgfqpoint{4.492307in}{5.389600in}}%
\pgfpathlineto{\pgfqpoint{4.492729in}{5.396786in}}%
\pgfpathlineto{\pgfqpoint{4.493150in}{5.400264in}}%
\pgfpathlineto{\pgfqpoint{4.493571in}{5.398274in}}%
\pgfpathlineto{\pgfqpoint{4.494835in}{5.391842in}}%
\pgfpathlineto{\pgfqpoint{4.495256in}{5.393287in}}%
\pgfpathlineto{\pgfqpoint{4.497363in}{5.402213in}}%
\pgfpathlineto{\pgfqpoint{4.498205in}{5.431586in}}%
\pgfpathlineto{\pgfqpoint{4.499469in}{5.444974in}}%
\pgfpathlineto{\pgfqpoint{4.500733in}{5.444617in}}%
\pgfpathlineto{\pgfqpoint{4.501997in}{5.484194in}}%
\pgfpathlineto{\pgfqpoint{4.502418in}{5.472419in}}%
\pgfpathlineto{\pgfqpoint{4.503682in}{5.453417in}}%
\pgfpathlineto{\pgfqpoint{4.505367in}{5.484089in}}%
\pgfpathlineto{\pgfqpoint{4.505788in}{5.472545in}}%
\pgfpathlineto{\pgfqpoint{4.507052in}{5.446147in}}%
\pgfpathlineto{\pgfqpoint{4.508316in}{5.486561in}}%
\pgfpathlineto{\pgfqpoint{4.508737in}{5.477846in}}%
\pgfpathlineto{\pgfqpoint{4.510422in}{5.442878in}}%
\pgfpathlineto{\pgfqpoint{4.510844in}{5.444806in}}%
\pgfpathlineto{\pgfqpoint{4.511265in}{5.442459in}}%
\pgfpathlineto{\pgfqpoint{4.513371in}{5.398441in}}%
\pgfpathlineto{\pgfqpoint{4.514214in}{5.398651in}}%
\pgfpathlineto{\pgfqpoint{4.518427in}{5.297437in}}%
\pgfpathlineto{\pgfqpoint{4.521797in}{5.259327in}}%
\pgfpathlineto{\pgfqpoint{4.524325in}{5.228780in}}%
\pgfpathlineto{\pgfqpoint{4.526431in}{5.249166in}}%
\pgfpathlineto{\pgfqpoint{4.528116in}{5.222432in}}%
\pgfpathlineto{\pgfqpoint{4.529801in}{5.167687in}}%
\pgfpathlineto{\pgfqpoint{4.531908in}{5.117028in}}%
\pgfpathlineto{\pgfqpoint{4.532750in}{5.110721in}}%
\pgfpathlineto{\pgfqpoint{4.533171in}{5.114136in}}%
\pgfpathlineto{\pgfqpoint{4.533593in}{5.116441in}}%
\pgfpathlineto{\pgfqpoint{4.534014in}{5.116148in}}%
\pgfpathlineto{\pgfqpoint{4.538227in}{4.980700in}}%
\pgfpathlineto{\pgfqpoint{4.539491in}{4.964190in}}%
\pgfpathlineto{\pgfqpoint{4.539912in}{4.967039in}}%
\pgfpathlineto{\pgfqpoint{4.540754in}{4.974016in}}%
\pgfpathlineto{\pgfqpoint{4.541597in}{4.971712in}}%
\pgfpathlineto{\pgfqpoint{4.542440in}{4.976845in}}%
\pgfpathlineto{\pgfqpoint{4.544125in}{4.979589in}}%
\pgfpathlineto{\pgfqpoint{4.544967in}{4.967060in}}%
\pgfpathlineto{\pgfqpoint{4.545810in}{4.967542in}}%
\pgfpathlineto{\pgfqpoint{4.546652in}{4.965007in}}%
\pgfpathlineto{\pgfqpoint{4.549601in}{4.999409in}}%
\pgfpathlineto{\pgfqpoint{4.550023in}{4.996790in}}%
\pgfpathlineto{\pgfqpoint{4.550444in}{4.999409in}}%
\pgfpathlineto{\pgfqpoint{4.550865in}{4.999765in}}%
\pgfpathlineto{\pgfqpoint{4.551708in}{4.990609in}}%
\pgfpathlineto{\pgfqpoint{4.552129in}{4.997418in}}%
\pgfpathlineto{\pgfqpoint{4.558027in}{5.059978in}}%
\pgfpathlineto{\pgfqpoint{4.558448in}{5.050131in}}%
\pgfpathlineto{\pgfqpoint{4.558869in}{5.056794in}}%
\pgfpathlineto{\pgfqpoint{4.562240in}{5.103116in}}%
\pgfpathlineto{\pgfqpoint{4.564767in}{5.191006in}}%
\pgfpathlineto{\pgfqpoint{4.565189in}{5.189497in}}%
\pgfpathlineto{\pgfqpoint{4.567295in}{5.221699in}}%
\pgfpathlineto{\pgfqpoint{4.570665in}{5.374411in}}%
\pgfpathlineto{\pgfqpoint{4.572772in}{5.480506in}}%
\pgfpathlineto{\pgfqpoint{4.573614in}{5.488677in}}%
\pgfpathlineto{\pgfqpoint{4.576563in}{5.432906in}}%
\pgfpathlineto{\pgfqpoint{4.576984in}{5.440155in}}%
\pgfpathlineto{\pgfqpoint{4.577827in}{5.439861in}}%
\pgfpathlineto{\pgfqpoint{4.578669in}{5.444282in}}%
\pgfpathlineto{\pgfqpoint{4.579933in}{5.409524in}}%
\pgfpathlineto{\pgfqpoint{4.580355in}{5.409629in}}%
\pgfpathlineto{\pgfqpoint{4.588359in}{5.181389in}}%
\pgfpathlineto{\pgfqpoint{4.589623in}{5.167582in}}%
\pgfpathlineto{\pgfqpoint{4.590044in}{5.167918in}}%
\pgfpathlineto{\pgfqpoint{4.592150in}{5.170997in}}%
\pgfpathlineto{\pgfqpoint{4.592572in}{5.181934in}}%
\pgfpathlineto{\pgfqpoint{4.592993in}{5.179462in}}%
\pgfpathlineto{\pgfqpoint{4.595521in}{5.141394in}}%
\pgfpathlineto{\pgfqpoint{4.595942in}{5.150884in}}%
\pgfpathlineto{\pgfqpoint{4.596784in}{5.149669in}}%
\pgfpathlineto{\pgfqpoint{4.598470in}{5.141519in}}%
\pgfpathlineto{\pgfqpoint{4.598891in}{5.142232in}}%
\pgfpathlineto{\pgfqpoint{4.600155in}{5.153839in}}%
\pgfpathlineto{\pgfqpoint{4.601840in}{5.139445in}}%
\pgfpathlineto{\pgfqpoint{4.602682in}{5.163937in}}%
\pgfpathlineto{\pgfqpoint{4.603525in}{5.157861in}}%
\pgfpathlineto{\pgfqpoint{4.605210in}{5.146087in}}%
\pgfpathlineto{\pgfqpoint{4.606053in}{5.154069in}}%
\pgfpathlineto{\pgfqpoint{4.606474in}{5.153545in}}%
\pgfpathlineto{\pgfqpoint{4.608159in}{5.130709in}}%
\pgfpathlineto{\pgfqpoint{4.608580in}{5.133013in}}%
\pgfpathlineto{\pgfqpoint{4.609001in}{5.145668in}}%
\pgfpathlineto{\pgfqpoint{4.609844in}{5.138146in}}%
\pgfpathlineto{\pgfqpoint{4.611950in}{5.125806in}}%
\pgfpathlineto{\pgfqpoint{4.612372in}{5.129933in}}%
\pgfpathlineto{\pgfqpoint{4.612793in}{5.123921in}}%
\pgfpathlineto{\pgfqpoint{4.614899in}{5.087591in}}%
\pgfpathlineto{\pgfqpoint{4.615321in}{5.093814in}}%
\pgfpathlineto{\pgfqpoint{4.615742in}{5.097459in}}%
\pgfpathlineto{\pgfqpoint{4.616163in}{5.093688in}}%
\pgfpathlineto{\pgfqpoint{4.618270in}{5.085329in}}%
\pgfpathlineto{\pgfqpoint{4.632172in}{5.472168in}}%
\pgfpathlineto{\pgfqpoint{4.632593in}{5.468774in}}%
\pgfpathlineto{\pgfqpoint{4.634278in}{5.455323in}}%
\pgfpathlineto{\pgfqpoint{4.634699in}{5.455219in}}%
\pgfpathlineto{\pgfqpoint{4.635121in}{5.457607in}}%
\pgfpathlineto{\pgfqpoint{4.635542in}{5.456203in}}%
\pgfpathlineto{\pgfqpoint{4.635963in}{5.455051in}}%
\pgfpathlineto{\pgfqpoint{4.636385in}{5.455344in}}%
\pgfpathlineto{\pgfqpoint{4.637227in}{5.456371in}}%
\pgfpathlineto{\pgfqpoint{4.637648in}{5.455763in}}%
\pgfpathlineto{\pgfqpoint{4.638070in}{5.455721in}}%
\pgfpathlineto{\pgfqpoint{4.638912in}{5.459933in}}%
\pgfpathlineto{\pgfqpoint{4.639333in}{5.458445in}}%
\pgfpathlineto{\pgfqpoint{4.640597in}{5.456559in}}%
\pgfpathlineto{\pgfqpoint{4.646074in}{5.379187in}}%
\pgfpathlineto{\pgfqpoint{4.646917in}{5.377490in}}%
\pgfpathlineto{\pgfqpoint{4.648602in}{5.372609in}}%
\pgfpathlineto{\pgfqpoint{4.651551in}{5.370325in}}%
\pgfpathlineto{\pgfqpoint{4.652814in}{5.361924in}}%
\pgfpathlineto{\pgfqpoint{4.653236in}{5.362091in}}%
\pgfpathlineto{\pgfqpoint{4.654078in}{5.362112in}}%
\pgfpathlineto{\pgfqpoint{4.659555in}{5.434393in}}%
\pgfpathlineto{\pgfqpoint{4.660397in}{5.435189in}}%
\pgfpathlineto{\pgfqpoint{4.662083in}{5.438835in}}%
\pgfpathlineto{\pgfqpoint{4.663346in}{5.439987in}}%
\pgfpathlineto{\pgfqpoint{4.665453in}{5.440951in}}%
\pgfpathlineto{\pgfqpoint{4.667980in}{5.453899in}}%
\pgfpathlineto{\pgfqpoint{4.672614in}{5.375228in}}%
\pgfpathlineto{\pgfqpoint{4.673878in}{5.382519in}}%
\pgfpathlineto{\pgfqpoint{4.674300in}{5.379166in}}%
\pgfpathlineto{\pgfqpoint{4.674721in}{5.379648in}}%
\pgfpathlineto{\pgfqpoint{4.675142in}{5.382602in}}%
\pgfpathlineto{\pgfqpoint{4.676406in}{5.353166in}}%
\pgfpathlineto{\pgfqpoint{4.678934in}{5.193666in}}%
\pgfpathlineto{\pgfqpoint{4.681461in}{5.011728in}}%
\pgfpathlineto{\pgfqpoint{4.681883in}{5.014556in}}%
\pgfpathlineto{\pgfqpoint{4.682304in}{5.016274in}}%
\pgfpathlineto{\pgfqpoint{4.682725in}{5.005757in}}%
\pgfpathlineto{\pgfqpoint{4.683568in}{5.010471in}}%
\pgfpathlineto{\pgfqpoint{4.683989in}{5.011916in}}%
\pgfpathlineto{\pgfqpoint{4.684832in}{5.003746in}}%
\pgfpathlineto{\pgfqpoint{4.685253in}{5.009088in}}%
\pgfpathlineto{\pgfqpoint{4.685674in}{5.010136in}}%
\pgfpathlineto{\pgfqpoint{4.686517in}{4.993333in}}%
\pgfpathlineto{\pgfqpoint{4.687359in}{4.999241in}}%
\pgfpathlineto{\pgfqpoint{4.688623in}{4.999346in}}%
\pgfpathlineto{\pgfqpoint{4.691151in}{5.026477in}}%
\pgfpathlineto{\pgfqpoint{4.692836in}{5.027148in}}%
\pgfpathlineto{\pgfqpoint{4.694100in}{5.027085in}}%
\pgfpathlineto{\pgfqpoint{4.694521in}{5.023251in}}%
\pgfpathlineto{\pgfqpoint{4.694942in}{5.001252in}}%
\pgfpathlineto{\pgfqpoint{4.695785in}{5.011120in}}%
\pgfpathlineto{\pgfqpoint{4.696206in}{5.013634in}}%
\pgfpathlineto{\pgfqpoint{4.696627in}{5.006511in}}%
\pgfpathlineto{\pgfqpoint{4.697049in}{5.012147in}}%
\pgfpathlineto{\pgfqpoint{4.697891in}{5.019040in}}%
\pgfpathlineto{\pgfqpoint{4.698312in}{5.012398in}}%
\pgfpathlineto{\pgfqpoint{4.698734in}{5.014619in}}%
\pgfpathlineto{\pgfqpoint{4.699576in}{5.021219in}}%
\pgfpathlineto{\pgfqpoint{4.700419in}{5.004709in}}%
\pgfpathlineto{\pgfqpoint{4.701261in}{5.007098in}}%
\pgfpathlineto{\pgfqpoint{4.701683in}{5.005715in}}%
\pgfpathlineto{\pgfqpoint{4.702104in}{5.006783in}}%
\pgfpathlineto{\pgfqpoint{4.702525in}{5.008145in}}%
\pgfpathlineto{\pgfqpoint{4.703789in}{5.038021in}}%
\pgfpathlineto{\pgfqpoint{4.707159in}{5.265905in}}%
\pgfpathlineto{\pgfqpoint{4.709266in}{5.367853in}}%
\pgfpathlineto{\pgfqpoint{4.710108in}{5.373845in}}%
\pgfpathlineto{\pgfqpoint{4.710529in}{5.365904in}}%
\pgfpathlineto{\pgfqpoint{4.710951in}{5.372462in}}%
\pgfpathlineto{\pgfqpoint{4.711793in}{5.379879in}}%
\pgfpathlineto{\pgfqpoint{4.712215in}{5.373489in}}%
\pgfpathlineto{\pgfqpoint{4.712636in}{5.378454in}}%
\pgfpathlineto{\pgfqpoint{4.713478in}{5.385996in}}%
\pgfpathlineto{\pgfqpoint{4.714321in}{5.368775in}}%
\pgfpathlineto{\pgfqpoint{4.714742in}{5.371121in}}%
\pgfpathlineto{\pgfqpoint{4.715164in}{5.372986in}}%
\pgfpathlineto{\pgfqpoint{4.715585in}{5.371876in}}%
\pgfpathlineto{\pgfqpoint{4.716006in}{5.371142in}}%
\pgfpathlineto{\pgfqpoint{4.716427in}{5.372357in}}%
\pgfpathlineto{\pgfqpoint{4.716849in}{5.372546in}}%
\pgfpathlineto{\pgfqpoint{4.717691in}{5.370137in}}%
\pgfpathlineto{\pgfqpoint{4.718534in}{5.372630in}}%
\pgfpathlineto{\pgfqpoint{4.718955in}{5.371289in}}%
\pgfpathlineto{\pgfqpoint{4.722325in}{5.373992in}}%
\pgfpathlineto{\pgfqpoint{4.722747in}{5.372211in}}%
\pgfpathlineto{\pgfqpoint{4.724010in}{5.382791in}}%
\pgfpathlineto{\pgfqpoint{4.724432in}{5.373216in}}%
\pgfpathlineto{\pgfqpoint{4.724853in}{5.378747in}}%
\pgfpathlineto{\pgfqpoint{4.725696in}{5.382267in}}%
\pgfpathlineto{\pgfqpoint{4.726117in}{5.374620in}}%
\pgfpathlineto{\pgfqpoint{4.726538in}{5.379795in}}%
\pgfpathlineto{\pgfqpoint{4.727381in}{5.384446in}}%
\pgfpathlineto{\pgfqpoint{4.727802in}{5.377218in}}%
\pgfpathlineto{\pgfqpoint{4.728644in}{5.381324in}}%
\pgfpathlineto{\pgfqpoint{4.730751in}{5.383713in}}%
\pgfpathlineto{\pgfqpoint{4.731172in}{5.382016in}}%
\pgfpathlineto{\pgfqpoint{4.732015in}{5.383105in}}%
\pgfpathlineto{\pgfqpoint{4.732436in}{5.383063in}}%
\pgfpathlineto{\pgfqpoint{4.733279in}{5.381241in}}%
\pgfpathlineto{\pgfqpoint{4.733700in}{5.382665in}}%
\pgfpathlineto{\pgfqpoint{4.734121in}{5.382477in}}%
\pgfpathlineto{\pgfqpoint{4.735806in}{5.380382in}}%
\pgfpathlineto{\pgfqpoint{4.736649in}{5.382812in}}%
\pgfpathlineto{\pgfqpoint{4.737913in}{5.389328in}}%
\pgfpathlineto{\pgfqpoint{4.738334in}{5.380989in}}%
\pgfpathlineto{\pgfqpoint{4.739176in}{5.385808in}}%
\pgfpathlineto{\pgfqpoint{4.739598in}{5.387589in}}%
\pgfpathlineto{\pgfqpoint{4.740019in}{5.378475in}}%
\pgfpathlineto{\pgfqpoint{4.740440in}{5.383566in}}%
\pgfpathlineto{\pgfqpoint{4.741283in}{5.385305in}}%
\pgfpathlineto{\pgfqpoint{4.742125in}{5.373447in}}%
\pgfpathlineto{\pgfqpoint{4.742547in}{5.376233in}}%
\pgfpathlineto{\pgfqpoint{4.743389in}{5.376108in}}%
\pgfpathlineto{\pgfqpoint{4.744232in}{5.378098in}}%
\pgfpathlineto{\pgfqpoint{4.744653in}{5.377868in}}%
\pgfpathlineto{\pgfqpoint{4.745074in}{5.376212in}}%
\pgfpathlineto{\pgfqpoint{4.745496in}{5.377302in}}%
\pgfpathlineto{\pgfqpoint{4.746338in}{5.377721in}}%
\pgfpathlineto{\pgfqpoint{4.746759in}{5.375961in}}%
\pgfpathlineto{\pgfqpoint{4.747181in}{5.376569in}}%
\pgfpathlineto{\pgfqpoint{4.747602in}{5.377490in}}%
\pgfpathlineto{\pgfqpoint{4.748023in}{5.377218in}}%
\pgfpathlineto{\pgfqpoint{4.749708in}{5.374494in}}%
\pgfpathlineto{\pgfqpoint{4.750130in}{5.374578in}}%
\pgfpathlineto{\pgfqpoint{4.753079in}{5.389726in}}%
\pgfpathlineto{\pgfqpoint{4.753500in}{5.391653in}}%
\pgfpathlineto{\pgfqpoint{4.753921in}{5.383608in}}%
\pgfpathlineto{\pgfqpoint{4.754764in}{5.388301in}}%
\pgfpathlineto{\pgfqpoint{4.755185in}{5.387987in}}%
\pgfpathlineto{\pgfqpoint{4.762768in}{5.325385in}}%
\pgfpathlineto{\pgfqpoint{4.765296in}{5.296766in}}%
\pgfpathlineto{\pgfqpoint{4.765717in}{5.300495in}}%
\pgfpathlineto{\pgfqpoint{4.766138in}{5.296012in}}%
\pgfpathlineto{\pgfqpoint{4.766981in}{5.295174in}}%
\pgfpathlineto{\pgfqpoint{4.767402in}{5.302528in}}%
\pgfpathlineto{\pgfqpoint{4.768245in}{5.298547in}}%
\pgfpathlineto{\pgfqpoint{4.768666in}{5.299092in}}%
\pgfpathlineto{\pgfqpoint{4.776670in}{5.261631in}}%
\pgfpathlineto{\pgfqpoint{4.777513in}{5.263601in}}%
\pgfpathlineto{\pgfqpoint{4.778355in}{5.261799in}}%
\pgfpathlineto{\pgfqpoint{4.779198in}{5.263601in}}%
\pgfpathlineto{\pgfqpoint{4.780040in}{5.261066in}}%
\pgfpathlineto{\pgfqpoint{4.780462in}{5.262868in}}%
\pgfpathlineto{\pgfqpoint{4.780883in}{5.262658in}}%
\pgfpathlineto{\pgfqpoint{4.781726in}{5.256289in}}%
\pgfpathlineto{\pgfqpoint{4.782147in}{5.258216in}}%
\pgfpathlineto{\pgfqpoint{4.782568in}{5.259830in}}%
\pgfpathlineto{\pgfqpoint{4.784674in}{5.290565in}}%
\pgfpathlineto{\pgfqpoint{4.785096in}{5.288763in}}%
\pgfpathlineto{\pgfqpoint{4.785517in}{5.287066in}}%
\pgfpathlineto{\pgfqpoint{4.788466in}{5.134585in}}%
\pgfpathlineto{\pgfqpoint{4.800262in}{4.501781in}}%
\pgfpathlineto{\pgfqpoint{4.803632in}{4.501383in}}%
\pgfpathlineto{\pgfqpoint{4.809530in}{4.453803in}}%
\pgfpathlineto{\pgfqpoint{4.809951in}{4.449194in}}%
\pgfpathlineto{\pgfqpoint{4.812058in}{4.545589in}}%
\pgfpathlineto{\pgfqpoint{4.823853in}{5.252706in}}%
\pgfpathlineto{\pgfqpoint{4.824275in}{5.247029in}}%
\pgfpathlineto{\pgfqpoint{4.824696in}{5.248914in}}%
\pgfpathlineto{\pgfqpoint{4.825538in}{5.260835in}}%
\pgfpathlineto{\pgfqpoint{4.825960in}{5.255262in}}%
\pgfpathlineto{\pgfqpoint{4.826381in}{5.257504in}}%
\pgfpathlineto{\pgfqpoint{4.827224in}{5.270347in}}%
\pgfpathlineto{\pgfqpoint{4.827645in}{5.264627in}}%
\pgfpathlineto{\pgfqpoint{4.828487in}{5.276129in}}%
\pgfpathlineto{\pgfqpoint{4.829330in}{5.269802in}}%
\pgfpathlineto{\pgfqpoint{4.829751in}{5.269970in}}%
\pgfpathlineto{\pgfqpoint{4.830172in}{5.276926in}}%
\pgfpathlineto{\pgfqpoint{4.830594in}{5.270682in}}%
\pgfpathlineto{\pgfqpoint{4.831015in}{5.267875in}}%
\pgfpathlineto{\pgfqpoint{4.831436in}{5.269090in}}%
\pgfpathlineto{\pgfqpoint{4.831858in}{5.275250in}}%
\pgfpathlineto{\pgfqpoint{4.832279in}{5.269006in}}%
\pgfpathlineto{\pgfqpoint{4.832700in}{5.267330in}}%
\pgfpathlineto{\pgfqpoint{4.833121in}{5.269572in}}%
\pgfpathlineto{\pgfqpoint{4.833543in}{5.275836in}}%
\pgfpathlineto{\pgfqpoint{4.833964in}{5.269656in}}%
\pgfpathlineto{\pgfqpoint{4.834385in}{5.268315in}}%
\pgfpathlineto{\pgfqpoint{4.835228in}{5.274516in}}%
\pgfpathlineto{\pgfqpoint{4.836070in}{5.267959in}}%
\pgfpathlineto{\pgfqpoint{4.836492in}{5.273573in}}%
\pgfpathlineto{\pgfqpoint{4.836913in}{5.273511in}}%
\pgfpathlineto{\pgfqpoint{4.837334in}{5.267414in}}%
\pgfpathlineto{\pgfqpoint{4.837755in}{5.267665in}}%
\pgfpathlineto{\pgfqpoint{4.838177in}{5.274747in}}%
\pgfpathlineto{\pgfqpoint{4.838598in}{5.272652in}}%
\pgfpathlineto{\pgfqpoint{4.839441in}{5.269153in}}%
\pgfpathlineto{\pgfqpoint{4.839862in}{5.275648in}}%
\pgfpathlineto{\pgfqpoint{4.840283in}{5.270640in}}%
\pgfpathlineto{\pgfqpoint{4.841126in}{5.264858in}}%
\pgfpathlineto{\pgfqpoint{4.843653in}{5.297793in}}%
\pgfpathlineto{\pgfqpoint{4.844496in}{5.293184in}}%
\pgfpathlineto{\pgfqpoint{4.845760in}{5.246651in}}%
\pgfpathlineto{\pgfqpoint{4.857977in}{4.599622in}}%
\pgfpathlineto{\pgfqpoint{4.859662in}{4.574418in}}%
\pgfpathlineto{\pgfqpoint{4.860083in}{4.574795in}}%
\pgfpathlineto{\pgfqpoint{4.861768in}{4.574229in}}%
\pgfpathlineto{\pgfqpoint{4.865981in}{4.602492in}}%
\pgfpathlineto{\pgfqpoint{4.866402in}{4.602513in}}%
\pgfpathlineto{\pgfqpoint{4.867245in}{4.612088in}}%
\pgfpathlineto{\pgfqpoint{4.870615in}{4.654597in}}%
\pgfpathlineto{\pgfqpoint{4.874407in}{4.739889in}}%
\pgfpathlineto{\pgfqpoint{4.879462in}{4.905946in}}%
\pgfpathlineto{\pgfqpoint{4.885360in}{5.055976in}}%
\pgfpathlineto{\pgfqpoint{4.888730in}{5.146820in}}%
\pgfpathlineto{\pgfqpoint{4.891258in}{5.154383in}}%
\pgfpathlineto{\pgfqpoint{4.892943in}{5.160187in}}%
\pgfpathlineto{\pgfqpoint{4.893785in}{5.164754in}}%
\pgfpathlineto{\pgfqpoint{4.895471in}{5.143049in}}%
\pgfpathlineto{\pgfqpoint{4.903054in}{4.956313in}}%
\pgfpathlineto{\pgfqpoint{4.903896in}{4.956941in}}%
\pgfpathlineto{\pgfqpoint{4.906003in}{4.949022in}}%
\pgfpathlineto{\pgfqpoint{4.907266in}{4.944810in}}%
\pgfpathlineto{\pgfqpoint{4.907688in}{4.947199in}}%
\pgfpathlineto{\pgfqpoint{4.909373in}{4.976132in}}%
\pgfpathlineto{\pgfqpoint{4.916113in}{5.153273in}}%
\pgfpathlineto{\pgfqpoint{4.918220in}{5.160270in}}%
\pgfpathlineto{\pgfqpoint{4.919483in}{5.162575in}}%
\pgfpathlineto{\pgfqpoint{4.920747in}{5.170327in}}%
\pgfpathlineto{\pgfqpoint{4.921169in}{5.168211in}}%
\pgfpathlineto{\pgfqpoint{4.922432in}{5.153315in}}%
\pgfpathlineto{\pgfqpoint{4.932122in}{4.957109in}}%
\pgfpathlineto{\pgfqpoint{4.932543in}{4.958471in}}%
\pgfpathlineto{\pgfqpoint{4.934649in}{4.929265in}}%
\pgfpathlineto{\pgfqpoint{4.935071in}{4.929600in}}%
\pgfpathlineto{\pgfqpoint{4.938862in}{4.952164in}}%
\pgfpathlineto{\pgfqpoint{4.939705in}{4.957632in}}%
\pgfpathlineto{\pgfqpoint{4.940126in}{4.957486in}}%
\pgfpathlineto{\pgfqpoint{4.940969in}{4.957758in}}%
\pgfpathlineto{\pgfqpoint{4.941811in}{4.967982in}}%
\pgfpathlineto{\pgfqpoint{4.943496in}{4.978877in}}%
\pgfpathlineto{\pgfqpoint{4.944339in}{4.979568in}}%
\pgfpathlineto{\pgfqpoint{4.944760in}{4.977850in}}%
\pgfpathlineto{\pgfqpoint{4.945603in}{4.978458in}}%
\pgfpathlineto{\pgfqpoint{4.947709in}{4.979401in}}%
\pgfpathlineto{\pgfqpoint{4.950658in}{5.034900in}}%
\pgfpathlineto{\pgfqpoint{4.951079in}{5.039404in}}%
\pgfpathlineto{\pgfqpoint{4.953607in}{5.101985in}}%
\pgfpathlineto{\pgfqpoint{4.954871in}{5.132783in}}%
\pgfpathlineto{\pgfqpoint{4.961611in}{5.269698in}}%
\pgfpathlineto{\pgfqpoint{4.962454in}{5.276716in}}%
\pgfpathlineto{\pgfqpoint{4.963296in}{5.274579in}}%
\pgfpathlineto{\pgfqpoint{4.963718in}{5.274495in}}%
\pgfpathlineto{\pgfqpoint{4.964139in}{5.273154in}}%
\pgfpathlineto{\pgfqpoint{4.965824in}{5.278183in}}%
\pgfpathlineto{\pgfqpoint{4.967509in}{5.273594in}}%
\pgfpathlineto{\pgfqpoint{4.968352in}{5.280320in}}%
\pgfpathlineto{\pgfqpoint{4.968773in}{5.279272in}}%
\pgfpathlineto{\pgfqpoint{4.970458in}{5.274391in}}%
\pgfpathlineto{\pgfqpoint{4.970879in}{5.272128in}}%
\pgfpathlineto{\pgfqpoint{4.971722in}{5.280676in}}%
\pgfpathlineto{\pgfqpoint{4.972143in}{5.279586in}}%
\pgfpathlineto{\pgfqpoint{4.974250in}{5.264166in}}%
\pgfpathlineto{\pgfqpoint{4.975092in}{5.281807in}}%
\pgfpathlineto{\pgfqpoint{4.975513in}{5.281179in}}%
\pgfpathlineto{\pgfqpoint{4.977620in}{5.272107in}}%
\pgfpathlineto{\pgfqpoint{4.978041in}{5.286710in}}%
\pgfpathlineto{\pgfqpoint{4.978884in}{5.283965in}}%
\pgfpathlineto{\pgfqpoint{4.980569in}{5.270054in}}%
\pgfpathlineto{\pgfqpoint{4.980990in}{5.272589in}}%
\pgfpathlineto{\pgfqpoint{4.981411in}{5.286228in}}%
\pgfpathlineto{\pgfqpoint{4.982254in}{5.282499in}}%
\pgfpathlineto{\pgfqpoint{4.983939in}{5.268315in}}%
\pgfpathlineto{\pgfqpoint{4.984782in}{5.285034in}}%
\pgfpathlineto{\pgfqpoint{4.985624in}{5.280403in}}%
\pgfpathlineto{\pgfqpoint{4.987309in}{5.266345in}}%
\pgfpathlineto{\pgfqpoint{4.988152in}{5.282520in}}%
\pgfpathlineto{\pgfqpoint{4.988573in}{5.281032in}}%
\pgfpathlineto{\pgfqpoint{4.990679in}{5.268126in}}%
\pgfpathlineto{\pgfqpoint{4.991101in}{5.282624in}}%
\pgfpathlineto{\pgfqpoint{4.991943in}{5.280885in}}%
\pgfpathlineto{\pgfqpoint{4.993628in}{5.266869in}}%
\pgfpathlineto{\pgfqpoint{4.994050in}{5.269237in}}%
\pgfpathlineto{\pgfqpoint{4.994471in}{5.284049in}}%
\pgfpathlineto{\pgfqpoint{4.995314in}{5.279565in}}%
\pgfpathlineto{\pgfqpoint{4.996999in}{5.267162in}}%
\pgfpathlineto{\pgfqpoint{4.997841in}{5.280487in}}%
\pgfpathlineto{\pgfqpoint{4.998684in}{5.275375in}}%
\pgfpathlineto{\pgfqpoint{5.000369in}{5.261694in}}%
\pgfpathlineto{\pgfqpoint{5.001211in}{5.279105in}}%
\pgfpathlineto{\pgfqpoint{5.001633in}{5.277387in}}%
\pgfpathlineto{\pgfqpoint{5.003739in}{5.264418in}}%
\pgfpathlineto{\pgfqpoint{5.004160in}{5.279419in}}%
\pgfpathlineto{\pgfqpoint{5.005003in}{5.278162in}}%
\pgfpathlineto{\pgfqpoint{5.006688in}{5.265863in}}%
\pgfpathlineto{\pgfqpoint{5.007109in}{5.267728in}}%
\pgfpathlineto{\pgfqpoint{5.007531in}{5.282582in}}%
\pgfpathlineto{\pgfqpoint{5.008373in}{5.278769in}}%
\pgfpathlineto{\pgfqpoint{5.010058in}{5.265633in}}%
\pgfpathlineto{\pgfqpoint{5.010901in}{5.284447in}}%
\pgfpathlineto{\pgfqpoint{5.011743in}{5.282331in}}%
\pgfpathlineto{\pgfqpoint{5.013428in}{5.270075in}}%
\pgfpathlineto{\pgfqpoint{5.014271in}{5.287653in}}%
\pgfpathlineto{\pgfqpoint{5.014692in}{5.285788in}}%
\pgfpathlineto{\pgfqpoint{5.016799in}{5.272149in}}%
\pgfpathlineto{\pgfqpoint{5.017220in}{5.284719in}}%
\pgfpathlineto{\pgfqpoint{5.018063in}{5.282876in}}%
\pgfpathlineto{\pgfqpoint{5.019748in}{5.271520in}}%
\pgfpathlineto{\pgfqpoint{5.020169in}{5.272924in}}%
\pgfpathlineto{\pgfqpoint{5.020590in}{5.288407in}}%
\pgfpathlineto{\pgfqpoint{5.021433in}{5.286207in}}%
\pgfpathlineto{\pgfqpoint{5.023118in}{5.273657in}}%
\pgfpathlineto{\pgfqpoint{5.023960in}{5.290334in}}%
\pgfpathlineto{\pgfqpoint{5.024803in}{5.285872in}}%
\pgfpathlineto{\pgfqpoint{5.026488in}{5.273113in}}%
\pgfpathlineto{\pgfqpoint{5.028173in}{5.288072in}}%
\pgfpathlineto{\pgfqpoint{5.028594in}{5.284321in}}%
\pgfpathlineto{\pgfqpoint{5.029016in}{5.285180in}}%
\pgfpathlineto{\pgfqpoint{5.029858in}{5.292387in}}%
\pgfpathlineto{\pgfqpoint{5.030280in}{5.304665in}}%
\pgfpathlineto{\pgfqpoint{5.030701in}{5.298903in}}%
\pgfpathlineto{\pgfqpoint{5.032807in}{5.258112in}}%
\pgfpathlineto{\pgfqpoint{5.033229in}{5.263852in}}%
\pgfpathlineto{\pgfqpoint{5.033650in}{5.266157in}}%
\pgfpathlineto{\pgfqpoint{5.035756in}{5.204645in}}%
\pgfpathlineto{\pgfqpoint{5.036177in}{5.206991in}}%
\pgfpathlineto{\pgfqpoint{5.036599in}{5.210993in}}%
\pgfpathlineto{\pgfqpoint{5.037020in}{5.209254in}}%
\pgfpathlineto{\pgfqpoint{5.038705in}{5.150193in}}%
\pgfpathlineto{\pgfqpoint{5.039126in}{5.160040in}}%
\pgfpathlineto{\pgfqpoint{5.040390in}{5.185831in}}%
\pgfpathlineto{\pgfqpoint{5.041654in}{5.158406in}}%
\pgfpathlineto{\pgfqpoint{5.042075in}{5.162449in}}%
\pgfpathlineto{\pgfqpoint{5.042918in}{5.156918in}}%
\pgfpathlineto{\pgfqpoint{5.044603in}{5.121511in}}%
\pgfpathlineto{\pgfqpoint{5.045024in}{5.125659in}}%
\pgfpathlineto{\pgfqpoint{5.045867in}{5.135444in}}%
\pgfpathlineto{\pgfqpoint{5.046288in}{5.132699in}}%
\pgfpathlineto{\pgfqpoint{5.047552in}{5.113068in}}%
\pgfpathlineto{\pgfqpoint{5.048816in}{5.133076in}}%
\pgfpathlineto{\pgfqpoint{5.049237in}{5.132531in}}%
\pgfpathlineto{\pgfqpoint{5.050080in}{5.108857in}}%
\pgfpathlineto{\pgfqpoint{5.050501in}{5.110700in}}%
\pgfpathlineto{\pgfqpoint{5.052186in}{5.169824in}}%
\pgfpathlineto{\pgfqpoint{5.052607in}{5.160375in}}%
\pgfpathlineto{\pgfqpoint{5.053450in}{5.147721in}}%
\pgfpathlineto{\pgfqpoint{5.053871in}{5.151513in}}%
\pgfpathlineto{\pgfqpoint{5.054714in}{5.146129in}}%
\pgfpathlineto{\pgfqpoint{5.056399in}{5.113948in}}%
\pgfpathlineto{\pgfqpoint{5.056820in}{5.119898in}}%
\pgfpathlineto{\pgfqpoint{5.057663in}{5.133432in}}%
\pgfpathlineto{\pgfqpoint{5.058084in}{5.132049in}}%
\pgfpathlineto{\pgfqpoint{5.059348in}{5.112397in}}%
\pgfpathlineto{\pgfqpoint{5.060612in}{5.132406in}}%
\pgfpathlineto{\pgfqpoint{5.061033in}{5.130960in}}%
\pgfpathlineto{\pgfqpoint{5.062297in}{5.106196in}}%
\pgfpathlineto{\pgfqpoint{5.063982in}{5.164544in}}%
\pgfpathlineto{\pgfqpoint{5.064403in}{5.155263in}}%
\pgfpathlineto{\pgfqpoint{5.065246in}{5.142127in}}%
\pgfpathlineto{\pgfqpoint{5.065667in}{5.143489in}}%
\pgfpathlineto{\pgfqpoint{5.066510in}{5.155640in}}%
\pgfpathlineto{\pgfqpoint{5.066931in}{5.151492in}}%
\pgfpathlineto{\pgfqpoint{5.068616in}{5.124340in}}%
\pgfpathlineto{\pgfqpoint{5.069037in}{5.126393in}}%
\pgfpathlineto{\pgfqpoint{5.070722in}{5.166074in}}%
\pgfpathlineto{\pgfqpoint{5.071565in}{5.158364in}}%
\pgfpathlineto{\pgfqpoint{5.072407in}{5.160564in}}%
\pgfpathlineto{\pgfqpoint{5.072829in}{5.163853in}}%
\pgfpathlineto{\pgfqpoint{5.073671in}{5.162952in}}%
\pgfpathlineto{\pgfqpoint{5.074093in}{5.162952in}}%
\pgfpathlineto{\pgfqpoint{5.074514in}{5.161360in}}%
\pgfpathlineto{\pgfqpoint{5.074935in}{5.162449in}}%
\pgfpathlineto{\pgfqpoint{5.075356in}{5.162617in}}%
\pgfpathlineto{\pgfqpoint{5.076620in}{5.148538in}}%
\pgfpathlineto{\pgfqpoint{5.077884in}{5.151974in}}%
\pgfpathlineto{\pgfqpoint{5.078727in}{5.151660in}}%
\pgfpathlineto{\pgfqpoint{5.079569in}{5.148915in}}%
\pgfpathlineto{\pgfqpoint{5.079990in}{5.150759in}}%
\pgfpathlineto{\pgfqpoint{5.081676in}{5.155829in}}%
\pgfpathlineto{\pgfqpoint{5.082097in}{5.155368in}}%
\pgfpathlineto{\pgfqpoint{5.082939in}{5.137706in}}%
\pgfpathlineto{\pgfqpoint{5.083782in}{5.140556in}}%
\pgfpathlineto{\pgfqpoint{5.084203in}{5.140074in}}%
\pgfpathlineto{\pgfqpoint{5.085467in}{5.143258in}}%
\pgfpathlineto{\pgfqpoint{5.086310in}{5.136743in}}%
\pgfpathlineto{\pgfqpoint{5.087152in}{5.140220in}}%
\pgfpathlineto{\pgfqpoint{5.087573in}{5.140556in}}%
\pgfpathlineto{\pgfqpoint{5.088837in}{5.149627in}}%
\pgfpathlineto{\pgfqpoint{5.089259in}{5.148224in}}%
\pgfpathlineto{\pgfqpoint{5.090101in}{5.124067in}}%
\pgfpathlineto{\pgfqpoint{5.090522in}{5.125219in}}%
\pgfpathlineto{\pgfqpoint{5.091365in}{5.118285in}}%
\pgfpathlineto{\pgfqpoint{5.092629in}{5.102006in}}%
\pgfpathlineto{\pgfqpoint{5.096420in}{4.979861in}}%
\pgfpathlineto{\pgfqpoint{5.108637in}{4.718624in}}%
\pgfpathlineto{\pgfqpoint{5.109059in}{4.718540in}}%
\pgfpathlineto{\pgfqpoint{5.109901in}{4.747243in}}%
\pgfpathlineto{\pgfqpoint{5.110744in}{4.737542in}}%
\pgfpathlineto{\pgfqpoint{5.111165in}{4.736180in}}%
\pgfpathlineto{\pgfqpoint{5.112429in}{4.720677in}}%
\pgfpathlineto{\pgfqpoint{5.112850in}{4.725873in}}%
\pgfpathlineto{\pgfqpoint{5.113271in}{4.729329in}}%
\pgfpathlineto{\pgfqpoint{5.113693in}{4.726313in}}%
\pgfpathlineto{\pgfqpoint{5.114535in}{4.722248in}}%
\pgfpathlineto{\pgfqpoint{5.114957in}{4.718225in}}%
\pgfpathlineto{\pgfqpoint{5.115378in}{4.718561in}}%
\pgfpathlineto{\pgfqpoint{5.115799in}{4.722248in}}%
\pgfpathlineto{\pgfqpoint{5.116642in}{4.745189in}}%
\pgfpathlineto{\pgfqpoint{5.117063in}{4.738548in}}%
\pgfpathlineto{\pgfqpoint{5.117484in}{4.738443in}}%
\pgfpathlineto{\pgfqpoint{5.119169in}{4.722667in}}%
\pgfpathlineto{\pgfqpoint{5.119591in}{4.730607in}}%
\pgfpathlineto{\pgfqpoint{5.120433in}{4.726375in}}%
\pgfpathlineto{\pgfqpoint{5.120854in}{4.725307in}}%
\pgfpathlineto{\pgfqpoint{5.122118in}{4.718456in}}%
\pgfpathlineto{\pgfqpoint{5.122961in}{4.746572in}}%
\pgfpathlineto{\pgfqpoint{5.123382in}{4.740203in}}%
\pgfpathlineto{\pgfqpoint{5.125067in}{4.729623in}}%
\pgfpathlineto{\pgfqpoint{5.125488in}{4.722101in}}%
\pgfpathlineto{\pgfqpoint{5.125910in}{4.728345in}}%
\pgfpathlineto{\pgfqpoint{5.126331in}{4.730042in}}%
\pgfpathlineto{\pgfqpoint{5.128437in}{4.718016in}}%
\pgfpathlineto{\pgfqpoint{5.128859in}{4.718665in}}%
\pgfpathlineto{\pgfqpoint{5.130544in}{4.747431in}}%
\pgfpathlineto{\pgfqpoint{5.132229in}{4.795702in}}%
\pgfpathlineto{\pgfqpoint{5.133914in}{4.824196in}}%
\pgfpathlineto{\pgfqpoint{5.134757in}{4.833309in}}%
\pgfpathlineto{\pgfqpoint{5.135599in}{4.844623in}}%
\pgfpathlineto{\pgfqpoint{5.136020in}{4.838358in}}%
\pgfpathlineto{\pgfqpoint{5.138548in}{4.781414in}}%
\pgfpathlineto{\pgfqpoint{5.140654in}{4.772845in}}%
\pgfpathlineto{\pgfqpoint{5.141076in}{4.775673in}}%
\pgfpathlineto{\pgfqpoint{5.141918in}{4.785855in}}%
\pgfpathlineto{\pgfqpoint{5.142340in}{4.785792in}}%
\pgfpathlineto{\pgfqpoint{5.147395in}{4.630147in}}%
\pgfpathlineto{\pgfqpoint{5.149080in}{4.625601in}}%
\pgfpathlineto{\pgfqpoint{5.157506in}{4.730314in}}%
\pgfpathlineto{\pgfqpoint{5.161297in}{4.811499in}}%
\pgfpathlineto{\pgfqpoint{5.161718in}{4.810221in}}%
\pgfpathlineto{\pgfqpoint{5.162561in}{4.805717in}}%
\pgfpathlineto{\pgfqpoint{5.167616in}{4.734986in}}%
\pgfpathlineto{\pgfqpoint{5.168880in}{4.730859in}}%
\pgfpathlineto{\pgfqpoint{5.169301in}{4.731299in}}%
\pgfpathlineto{\pgfqpoint{5.170986in}{4.696478in}}%
\pgfpathlineto{\pgfqpoint{5.174778in}{4.615063in}}%
\pgfpathlineto{\pgfqpoint{5.175621in}{4.616278in}}%
\pgfpathlineto{\pgfqpoint{5.176463in}{4.640288in}}%
\pgfpathlineto{\pgfqpoint{5.178148in}{4.683510in}}%
\pgfpathlineto{\pgfqpoint{5.178570in}{4.682797in}}%
\pgfpathlineto{\pgfqpoint{5.178991in}{4.681394in}}%
\pgfpathlineto{\pgfqpoint{5.179412in}{4.684809in}}%
\pgfpathlineto{\pgfqpoint{5.181518in}{4.740287in}}%
\pgfpathlineto{\pgfqpoint{5.181940in}{4.739931in}}%
\pgfpathlineto{\pgfqpoint{5.182361in}{4.738359in}}%
\pgfpathlineto{\pgfqpoint{5.182782in}{4.742131in}}%
\pgfpathlineto{\pgfqpoint{5.188259in}{4.949923in}}%
\pgfpathlineto{\pgfqpoint{5.188680in}{4.948686in}}%
\pgfpathlineto{\pgfqpoint{5.189523in}{4.942653in}}%
\pgfpathlineto{\pgfqpoint{5.191629in}{4.881413in}}%
\pgfpathlineto{\pgfqpoint{5.192050in}{4.882607in}}%
\pgfpathlineto{\pgfqpoint{5.192893in}{4.879318in}}%
\pgfpathlineto{\pgfqpoint{5.194999in}{4.824342in}}%
\pgfpathlineto{\pgfqpoint{5.195421in}{4.824887in}}%
\pgfpathlineto{\pgfqpoint{5.196263in}{4.822142in}}%
\pgfpathlineto{\pgfqpoint{5.201740in}{4.613617in}}%
\pgfpathlineto{\pgfqpoint{5.202582in}{4.615021in}}%
\pgfpathlineto{\pgfqpoint{5.205110in}{4.681205in}}%
\pgfpathlineto{\pgfqpoint{5.205531in}{4.680053in}}%
\pgfpathlineto{\pgfqpoint{5.205953in}{4.679152in}}%
\pgfpathlineto{\pgfqpoint{5.208059in}{4.737919in}}%
\pgfpathlineto{\pgfqpoint{5.208902in}{4.737354in}}%
\pgfpathlineto{\pgfqpoint{5.209323in}{4.736474in}}%
\pgfpathlineto{\pgfqpoint{5.212272in}{4.849986in}}%
\pgfpathlineto{\pgfqpoint{5.214799in}{4.950216in}}%
\pgfpathlineto{\pgfqpoint{5.215642in}{4.947325in}}%
\pgfpathlineto{\pgfqpoint{5.216063in}{4.944852in}}%
\pgfpathlineto{\pgfqpoint{5.222382in}{4.822939in}}%
\pgfpathlineto{\pgfqpoint{5.222804in}{4.824300in}}%
\pgfpathlineto{\pgfqpoint{5.223646in}{4.800542in}}%
\pgfpathlineto{\pgfqpoint{5.228280in}{4.630420in}}%
\pgfpathlineto{\pgfqpoint{5.228702in}{4.638318in}}%
\pgfpathlineto{\pgfqpoint{5.234599in}{4.729267in}}%
\pgfpathlineto{\pgfqpoint{5.238391in}{4.789731in}}%
\pgfpathlineto{\pgfqpoint{5.242183in}{4.931402in}}%
\pgfpathlineto{\pgfqpoint{5.244710in}{4.909466in}}%
\pgfpathlineto{\pgfqpoint{5.245553in}{4.900415in}}%
\pgfpathlineto{\pgfqpoint{5.245974in}{4.901945in}}%
\pgfpathlineto{\pgfqpoint{5.246395in}{4.903579in}}%
\pgfpathlineto{\pgfqpoint{5.248080in}{4.881476in}}%
\pgfpathlineto{\pgfqpoint{5.252293in}{5.085308in}}%
\pgfpathlineto{\pgfqpoint{5.252714in}{5.077828in}}%
\pgfpathlineto{\pgfqpoint{5.253978in}{5.042337in}}%
\pgfpathlineto{\pgfqpoint{5.254821in}{4.991636in}}%
\pgfpathlineto{\pgfqpoint{5.255663in}{4.994255in}}%
\pgfpathlineto{\pgfqpoint{5.259034in}{5.060753in}}%
\pgfpathlineto{\pgfqpoint{5.259876in}{5.059014in}}%
\pgfpathlineto{\pgfqpoint{5.260297in}{5.060963in}}%
\pgfpathlineto{\pgfqpoint{5.261561in}{5.066955in}}%
\pgfpathlineto{\pgfqpoint{5.263668in}{5.007161in}}%
\pgfpathlineto{\pgfqpoint{5.267459in}{4.860064in}}%
\pgfpathlineto{\pgfqpoint{5.267880in}{4.863123in}}%
\pgfpathlineto{\pgfqpoint{5.270408in}{4.931590in}}%
\pgfpathlineto{\pgfqpoint{5.271672in}{4.969156in}}%
\pgfpathlineto{\pgfqpoint{5.276727in}{5.158636in}}%
\pgfpathlineto{\pgfqpoint{5.278412in}{5.205671in}}%
\pgfpathlineto{\pgfqpoint{5.282204in}{5.353900in}}%
\pgfpathlineto{\pgfqpoint{5.283046in}{5.355932in}}%
\pgfpathlineto{\pgfqpoint{5.285574in}{5.377197in}}%
\pgfpathlineto{\pgfqpoint{5.286838in}{5.373426in}}%
\pgfpathlineto{\pgfqpoint{5.287259in}{5.372860in}}%
\pgfpathlineto{\pgfqpoint{5.288523in}{5.346336in}}%
\pgfpathlineto{\pgfqpoint{5.295685in}{5.016945in}}%
\pgfpathlineto{\pgfqpoint{5.296949in}{5.004835in}}%
\pgfpathlineto{\pgfqpoint{5.297370in}{5.004898in}}%
\pgfpathlineto{\pgfqpoint{5.297791in}{5.005024in}}%
\pgfpathlineto{\pgfqpoint{5.298212in}{5.008418in}}%
\pgfpathlineto{\pgfqpoint{5.299055in}{4.983737in}}%
\pgfpathlineto{\pgfqpoint{5.301161in}{4.899493in}}%
\pgfpathlineto{\pgfqpoint{5.301583in}{4.910933in}}%
\pgfpathlineto{\pgfqpoint{5.303268in}{4.988368in}}%
\pgfpathlineto{\pgfqpoint{5.307902in}{5.200790in}}%
\pgfpathlineto{\pgfqpoint{5.311272in}{5.136344in}}%
\pgfpathlineto{\pgfqpoint{5.313379in}{5.081306in}}%
\pgfpathlineto{\pgfqpoint{5.314642in}{5.087675in}}%
\pgfpathlineto{\pgfqpoint{5.316749in}{5.022476in}}%
\pgfpathlineto{\pgfqpoint{5.318434in}{4.939363in}}%
\pgfpathlineto{\pgfqpoint{5.320119in}{4.872592in}}%
\pgfpathlineto{\pgfqpoint{5.320962in}{4.869010in}}%
\pgfpathlineto{\pgfqpoint{5.323489in}{4.917616in}}%
\pgfpathlineto{\pgfqpoint{5.326017in}{4.990274in}}%
\pgfpathlineto{\pgfqpoint{5.328545in}{5.027734in}}%
\pgfpathlineto{\pgfqpoint{5.330230in}{5.004919in}}%
\pgfpathlineto{\pgfqpoint{5.330651in}{5.010785in}}%
\pgfpathlineto{\pgfqpoint{5.333179in}{5.061528in}}%
\pgfpathlineto{\pgfqpoint{5.335706in}{5.139089in}}%
\pgfpathlineto{\pgfqpoint{5.336549in}{5.128970in}}%
\pgfpathlineto{\pgfqpoint{5.338234in}{5.105756in}}%
\pgfpathlineto{\pgfqpoint{5.339076in}{5.111182in}}%
\pgfpathlineto{\pgfqpoint{5.339919in}{5.115352in}}%
\pgfpathlineto{\pgfqpoint{5.347081in}{5.245332in}}%
\pgfpathlineto{\pgfqpoint{5.347923in}{5.247091in}}%
\pgfpathlineto{\pgfqpoint{5.353400in}{5.295153in}}%
\pgfpathlineto{\pgfqpoint{5.354664in}{5.296347in}}%
\pgfpathlineto{\pgfqpoint{5.358455in}{5.333493in}}%
\pgfpathlineto{\pgfqpoint{5.358877in}{5.333011in}}%
\pgfpathlineto{\pgfqpoint{5.359298in}{5.332194in}}%
\pgfpathlineto{\pgfqpoint{5.360140in}{5.335379in}}%
\pgfpathlineto{\pgfqpoint{5.360983in}{5.326705in}}%
\pgfpathlineto{\pgfqpoint{5.361825in}{5.340868in}}%
\pgfpathlineto{\pgfqpoint{5.362247in}{5.339569in}}%
\pgfpathlineto{\pgfqpoint{5.364353in}{5.327040in}}%
\pgfpathlineto{\pgfqpoint{5.364774in}{5.329932in}}%
\pgfpathlineto{\pgfqpoint{5.366460in}{5.298338in}}%
\pgfpathlineto{\pgfqpoint{5.369830in}{5.231190in}}%
\pgfpathlineto{\pgfqpoint{5.370672in}{5.216000in}}%
\pgfpathlineto{\pgfqpoint{5.371094in}{5.220337in}}%
\pgfpathlineto{\pgfqpoint{5.371515in}{5.217907in}}%
\pgfpathlineto{\pgfqpoint{5.373621in}{5.181242in}}%
\pgfpathlineto{\pgfqpoint{5.374043in}{5.179902in}}%
\pgfpathlineto{\pgfqpoint{5.374464in}{5.189246in}}%
\pgfpathlineto{\pgfqpoint{5.374885in}{5.179357in}}%
\pgfpathlineto{\pgfqpoint{5.376992in}{5.151115in}}%
\pgfpathlineto{\pgfqpoint{5.377413in}{5.152791in}}%
\pgfpathlineto{\pgfqpoint{5.377834in}{5.160836in}}%
\pgfpathlineto{\pgfqpoint{5.378255in}{5.154425in}}%
\pgfpathlineto{\pgfqpoint{5.378677in}{5.152602in}}%
\pgfpathlineto{\pgfqpoint{5.382468in}{5.196160in}}%
\pgfpathlineto{\pgfqpoint{5.382889in}{5.202843in}}%
\pgfpathlineto{\pgfqpoint{5.383311in}{5.195929in}}%
\pgfpathlineto{\pgfqpoint{5.383732in}{5.195761in}}%
\pgfpathlineto{\pgfqpoint{5.384575in}{5.199365in}}%
\pgfpathlineto{\pgfqpoint{5.384996in}{5.195342in}}%
\pgfpathlineto{\pgfqpoint{5.385838in}{5.196600in}}%
\pgfpathlineto{\pgfqpoint{5.386260in}{5.196516in}}%
\pgfpathlineto{\pgfqpoint{5.387102in}{5.193792in}}%
\pgfpathlineto{\pgfqpoint{5.401004in}{5.466281in}}%
\pgfpathlineto{\pgfqpoint{5.402689in}{5.454108in}}%
\pgfpathlineto{\pgfqpoint{5.403532in}{5.454108in}}%
\pgfpathlineto{\pgfqpoint{5.404796in}{5.422242in}}%
\pgfpathlineto{\pgfqpoint{5.407745in}{5.219268in}}%
\pgfpathlineto{\pgfqpoint{5.410272in}{5.105358in}}%
\pgfpathlineto{\pgfqpoint{5.411958in}{5.045124in}}%
\pgfpathlineto{\pgfqpoint{5.413643in}{5.014012in}}%
\pgfpathlineto{\pgfqpoint{5.415328in}{4.947786in}}%
\pgfpathlineto{\pgfqpoint{5.417013in}{4.922812in}}%
\pgfpathlineto{\pgfqpoint{5.417855in}{4.913070in}}%
\pgfpathlineto{\pgfqpoint{5.418277in}{4.904564in}}%
\pgfpathlineto{\pgfqpoint{5.418698in}{4.905423in}}%
\pgfpathlineto{\pgfqpoint{5.420383in}{5.018579in}}%
\pgfpathlineto{\pgfqpoint{5.423753in}{5.197102in}}%
\pgfpathlineto{\pgfqpoint{5.425438in}{5.256771in}}%
\pgfpathlineto{\pgfqpoint{5.427124in}{5.288092in}}%
\pgfpathlineto{\pgfqpoint{5.428809in}{5.354235in}}%
\pgfpathlineto{\pgfqpoint{5.432600in}{5.453689in}}%
\pgfpathlineto{\pgfqpoint{5.435128in}{5.453270in}}%
\pgfpathlineto{\pgfqpoint{5.441447in}{5.454066in}}%
\pgfpathlineto{\pgfqpoint{5.454928in}{5.453731in}}%
\pgfpathlineto{\pgfqpoint{5.456192in}{5.422221in}}%
\pgfpathlineto{\pgfqpoint{5.459141in}{5.219185in}}%
\pgfpathlineto{\pgfqpoint{5.469251in}{4.514666in}}%
\pgfpathlineto{\pgfqpoint{5.470515in}{4.503101in}}%
\pgfpathlineto{\pgfqpoint{5.471779in}{4.533521in}}%
\pgfpathlineto{\pgfqpoint{5.475571in}{4.698573in}}%
\pgfpathlineto{\pgfqpoint{5.483575in}{5.019606in}}%
\pgfpathlineto{\pgfqpoint{5.486945in}{5.080510in}}%
\pgfpathlineto{\pgfqpoint{5.489051in}{5.106657in}}%
\pgfpathlineto{\pgfqpoint{5.492843in}{5.156793in}}%
\pgfpathlineto{\pgfqpoint{5.493264in}{5.152623in}}%
\pgfpathlineto{\pgfqpoint{5.493686in}{5.152079in}}%
\pgfpathlineto{\pgfqpoint{5.494107in}{5.154237in}}%
\pgfpathlineto{\pgfqpoint{5.494528in}{5.151052in}}%
\pgfpathlineto{\pgfqpoint{5.498741in}{5.111685in}}%
\pgfpathlineto{\pgfqpoint{5.500847in}{5.101168in}}%
\pgfpathlineto{\pgfqpoint{5.506324in}{5.033287in}}%
\pgfpathlineto{\pgfqpoint{5.507166in}{5.033873in}}%
\pgfpathlineto{\pgfqpoint{5.507588in}{5.031736in}}%
\pgfpathlineto{\pgfqpoint{5.508009in}{5.034627in}}%
\pgfpathlineto{\pgfqpoint{5.517698in}{5.163162in}}%
\pgfpathlineto{\pgfqpoint{5.519805in}{5.193834in}}%
\pgfpathlineto{\pgfqpoint{5.520647in}{5.197605in}}%
\pgfpathlineto{\pgfqpoint{5.523596in}{5.253461in}}%
\pgfpathlineto{\pgfqpoint{5.524018in}{5.252769in}}%
\pgfpathlineto{\pgfqpoint{5.525703in}{5.192870in}}%
\pgfpathlineto{\pgfqpoint{5.531601in}{4.895450in}}%
\pgfpathlineto{\pgfqpoint{5.532022in}{4.895555in}}%
\pgfpathlineto{\pgfqpoint{5.532443in}{4.903432in}}%
\pgfpathlineto{\pgfqpoint{5.532864in}{4.893041in}}%
\pgfpathlineto{\pgfqpoint{5.533286in}{4.888808in}}%
\pgfpathlineto{\pgfqpoint{5.533707in}{4.890862in}}%
\pgfpathlineto{\pgfqpoint{5.534128in}{4.898257in}}%
\pgfpathlineto{\pgfqpoint{5.534550in}{4.889919in}}%
\pgfpathlineto{\pgfqpoint{5.534971in}{4.887405in}}%
\pgfpathlineto{\pgfqpoint{5.535813in}{4.909403in}}%
\pgfpathlineto{\pgfqpoint{5.536235in}{4.899326in}}%
\pgfpathlineto{\pgfqpoint{5.538341in}{4.864568in}}%
\pgfpathlineto{\pgfqpoint{5.538762in}{4.860127in}}%
\pgfpathlineto{\pgfqpoint{5.539605in}{4.861363in}}%
\pgfpathlineto{\pgfqpoint{5.540447in}{4.860399in}}%
\pgfpathlineto{\pgfqpoint{5.541711in}{4.860189in}}%
\pgfpathlineto{\pgfqpoint{5.543396in}{4.861467in}}%
\pgfpathlineto{\pgfqpoint{5.545081in}{4.865574in}}%
\pgfpathlineto{\pgfqpoint{5.545924in}{4.862871in}}%
\pgfpathlineto{\pgfqpoint{5.546345in}{4.864652in}}%
\pgfpathlineto{\pgfqpoint{5.546767in}{4.864673in}}%
\pgfpathlineto{\pgfqpoint{5.547609in}{4.858450in}}%
\pgfpathlineto{\pgfqpoint{5.548030in}{4.862745in}}%
\pgfpathlineto{\pgfqpoint{5.548452in}{4.862808in}}%
\pgfpathlineto{\pgfqpoint{5.549294in}{4.845482in}}%
\pgfpathlineto{\pgfqpoint{5.549716in}{4.857005in}}%
\pgfpathlineto{\pgfqpoint{5.552664in}{4.895534in}}%
\pgfpathlineto{\pgfqpoint{5.555192in}{5.029222in}}%
\pgfpathlineto{\pgfqpoint{5.556035in}{5.038629in}}%
\pgfpathlineto{\pgfqpoint{5.557299in}{5.113131in}}%
\pgfpathlineto{\pgfqpoint{5.557720in}{5.104520in}}%
\pgfpathlineto{\pgfqpoint{5.558984in}{5.071229in}}%
\pgfpathlineto{\pgfqpoint{5.559826in}{5.039341in}}%
\pgfpathlineto{\pgfqpoint{5.560247in}{5.049817in}}%
\pgfpathlineto{\pgfqpoint{5.560669in}{5.054677in}}%
\pgfpathlineto{\pgfqpoint{5.561090in}{5.048790in}}%
\pgfpathlineto{\pgfqpoint{5.562354in}{5.019145in}}%
\pgfpathlineto{\pgfqpoint{5.563196in}{4.963478in}}%
\pgfpathlineto{\pgfqpoint{5.564039in}{4.975127in}}%
\pgfpathlineto{\pgfqpoint{5.565303in}{4.990463in}}%
\pgfpathlineto{\pgfqpoint{5.565724in}{4.998696in}}%
\pgfpathlineto{\pgfqpoint{5.566145in}{4.989855in}}%
\pgfpathlineto{\pgfqpoint{5.566567in}{4.958240in}}%
\pgfpathlineto{\pgfqpoint{5.567409in}{4.966914in}}%
\pgfpathlineto{\pgfqpoint{5.567830in}{4.968548in}}%
\pgfpathlineto{\pgfqpoint{5.569516in}{4.991887in}}%
\pgfpathlineto{\pgfqpoint{5.570358in}{4.957884in}}%
\pgfpathlineto{\pgfqpoint{5.570779in}{4.967144in}}%
\pgfpathlineto{\pgfqpoint{5.571622in}{4.977473in}}%
\pgfpathlineto{\pgfqpoint{5.572886in}{4.996455in}}%
\pgfpathlineto{\pgfqpoint{5.573728in}{4.956836in}}%
\pgfpathlineto{\pgfqpoint{5.574150in}{4.964777in}}%
\pgfpathlineto{\pgfqpoint{5.576256in}{4.996664in}}%
\pgfpathlineto{\pgfqpoint{5.576677in}{4.988305in}}%
\pgfpathlineto{\pgfqpoint{5.577520in}{4.956459in}}%
\pgfpathlineto{\pgfqpoint{5.577941in}{4.964442in}}%
\pgfpathlineto{\pgfqpoint{5.578362in}{4.965908in}}%
\pgfpathlineto{\pgfqpoint{5.579626in}{4.983758in}}%
\pgfpathlineto{\pgfqpoint{5.580048in}{4.983737in}}%
\pgfpathlineto{\pgfqpoint{5.582997in}{4.848645in}}%
\pgfpathlineto{\pgfqpoint{5.583418in}{4.871838in}}%
\pgfpathlineto{\pgfqpoint{5.583839in}{4.870476in}}%
\pgfpathlineto{\pgfqpoint{5.586367in}{4.813301in}}%
\pgfpathlineto{\pgfqpoint{5.586788in}{4.811751in}}%
\pgfpathlineto{\pgfqpoint{5.588473in}{4.713889in}}%
\pgfpathlineto{\pgfqpoint{5.589316in}{4.686715in}}%
\pgfpathlineto{\pgfqpoint{5.589737in}{4.700815in}}%
\pgfpathlineto{\pgfqpoint{5.590580in}{4.743262in}}%
\pgfpathlineto{\pgfqpoint{5.591422in}{4.734148in}}%
\pgfpathlineto{\pgfqpoint{5.592265in}{4.735971in}}%
\pgfpathlineto{\pgfqpoint{5.593528in}{4.771399in}}%
\pgfpathlineto{\pgfqpoint{5.596477in}{4.917511in}}%
\pgfpathlineto{\pgfqpoint{5.597320in}{4.913908in}}%
\pgfpathlineto{\pgfqpoint{5.598584in}{4.932952in}}%
\pgfpathlineto{\pgfqpoint{5.600269in}{5.002887in}}%
\pgfpathlineto{\pgfqpoint{5.602797in}{5.124759in}}%
\pgfpathlineto{\pgfqpoint{5.603218in}{5.120987in}}%
\pgfpathlineto{\pgfqpoint{5.606167in}{5.107306in}}%
\pgfpathlineto{\pgfqpoint{5.612486in}{5.107181in}}%
\pgfpathlineto{\pgfqpoint{5.614171in}{5.107327in}}%
\pgfpathlineto{\pgfqpoint{5.615014in}{5.108857in}}%
\pgfpathlineto{\pgfqpoint{5.615014in}{5.108857in}}%
\pgfusepath{stroke}%
\end{pgfscope}%
\begin{pgfscope}%
\pgfsetrectcap%
\pgfsetmiterjoin%
\pgfsetlinewidth{0.803000pt}%
\definecolor{currentstroke}{rgb}{0.000000,0.000000,0.000000}%
\pgfsetstrokecolor{currentstroke}%
\pgfsetdash{}{0pt}%
\pgfpathmoveto{\pgfqpoint{0.885050in}{4.360741in}}%
\pgfpathlineto{\pgfqpoint{0.885050in}{5.646667in}}%
\pgfusepath{stroke}%
\end{pgfscope}%
\begin{pgfscope}%
\pgfsetrectcap%
\pgfsetmiterjoin%
\pgfsetlinewidth{0.803000pt}%
\definecolor{currentstroke}{rgb}{0.000000,0.000000,0.000000}%
\pgfsetstrokecolor{currentstroke}%
\pgfsetdash{}{0pt}%
\pgfpathmoveto{\pgfqpoint{5.840250in}{4.360741in}}%
\pgfpathlineto{\pgfqpoint{5.840250in}{5.646667in}}%
\pgfusepath{stroke}%
\end{pgfscope}%
\begin{pgfscope}%
\pgfsetrectcap%
\pgfsetmiterjoin%
\pgfsetlinewidth{0.803000pt}%
\definecolor{currentstroke}{rgb}{0.000000,0.000000,0.000000}%
\pgfsetstrokecolor{currentstroke}%
\pgfsetdash{}{0pt}%
\pgfpathmoveto{\pgfqpoint{0.885050in}{4.360741in}}%
\pgfpathlineto{\pgfqpoint{5.840250in}{4.360741in}}%
\pgfusepath{stroke}%
\end{pgfscope}%
\begin{pgfscope}%
\pgfsetrectcap%
\pgfsetmiterjoin%
\pgfsetlinewidth{0.803000pt}%
\definecolor{currentstroke}{rgb}{0.000000,0.000000,0.000000}%
\pgfsetstrokecolor{currentstroke}%
\pgfsetdash{}{0pt}%
\pgfpathmoveto{\pgfqpoint{0.885050in}{5.646667in}}%
\pgfpathlineto{\pgfqpoint{5.840250in}{5.646667in}}%
\pgfusepath{stroke}%
\end{pgfscope}%
\begin{pgfscope}%
\definecolor{textcolor}{rgb}{0.000000,0.000000,0.000000}%
\pgfsetstrokecolor{textcolor}%
\pgfsetfillcolor{textcolor}%
\pgftext[x=3.362650in,y=5.730000in,,base]{\color{textcolor}{\sffamily\fontsize{12.000000}{14.400000}\selectfont\catcode`\^=\active\def^{\ifmmode\sp\else\^{}\fi}\catcode`\%=\active\def%{\%}czas obliczeń}}%
\end{pgfscope}%
\begin{pgfscope}%
\pgfsetbuttcap%
\pgfsetmiterjoin%
\definecolor{currentfill}{rgb}{1.000000,1.000000,1.000000}%
\pgfsetfillcolor{currentfill}%
\pgfsetlinewidth{0.000000pt}%
\definecolor{currentstroke}{rgb}{0.000000,0.000000,0.000000}%
\pgfsetstrokecolor{currentstroke}%
\pgfsetstrokeopacity{0.000000}%
\pgfsetdash{}{0pt}%
\pgfpathmoveto{\pgfqpoint{0.885050in}{2.474259in}}%
\pgfpathlineto{\pgfqpoint{5.840250in}{2.474259in}}%
\pgfpathlineto{\pgfqpoint{5.840250in}{3.760185in}}%
\pgfpathlineto{\pgfqpoint{0.885050in}{3.760185in}}%
\pgfpathlineto{\pgfqpoint{0.885050in}{2.474259in}}%
\pgfpathclose%
\pgfusepath{fill}%
\end{pgfscope}%
\begin{pgfscope}%
\pgfsetbuttcap%
\pgfsetroundjoin%
\definecolor{currentfill}{rgb}{0.000000,0.000000,0.000000}%
\pgfsetfillcolor{currentfill}%
\pgfsetlinewidth{0.803000pt}%
\definecolor{currentstroke}{rgb}{0.000000,0.000000,0.000000}%
\pgfsetstrokecolor{currentstroke}%
\pgfsetdash{}{0pt}%
\pgfsys@defobject{currentmarker}{\pgfqpoint{0.000000in}{-0.048611in}}{\pgfqpoint{0.000000in}{0.000000in}}{%
\pgfpathmoveto{\pgfqpoint{0.000000in}{0.000000in}}%
\pgfpathlineto{\pgfqpoint{0.000000in}{-0.048611in}}%
\pgfusepath{stroke,fill}%
}%
\begin{pgfscope}%
\pgfsys@transformshift{1.096806in}{2.474259in}%
\pgfsys@useobject{currentmarker}{}%
\end{pgfscope}%
\end{pgfscope}%
\begin{pgfscope}%
\definecolor{textcolor}{rgb}{0.000000,0.000000,0.000000}%
\pgfsetstrokecolor{textcolor}%
\pgfsetfillcolor{textcolor}%
\pgftext[x=1.096806in,y=2.377037in,,top]{\color{textcolor}{\sffamily\fontsize{10.000000}{12.000000}\selectfont\catcode`\^=\active\def^{\ifmmode\sp\else\^{}\fi}\catcode`\%=\active\def%{\%}0}}%
\end{pgfscope}%
\begin{pgfscope}%
\pgfsetbuttcap%
\pgfsetroundjoin%
\definecolor{currentfill}{rgb}{0.000000,0.000000,0.000000}%
\pgfsetfillcolor{currentfill}%
\pgfsetlinewidth{0.803000pt}%
\definecolor{currentstroke}{rgb}{0.000000,0.000000,0.000000}%
\pgfsetstrokecolor{currentstroke}%
\pgfsetdash{}{0pt}%
\pgfsys@defobject{currentmarker}{\pgfqpoint{0.000000in}{-0.048611in}}{\pgfqpoint{0.000000in}{0.000000in}}{%
\pgfpathmoveto{\pgfqpoint{0.000000in}{0.000000in}}%
\pgfpathlineto{\pgfqpoint{0.000000in}{-0.048611in}}%
\pgfusepath{stroke,fill}%
}%
\begin{pgfscope}%
\pgfsys@transformshift{1.939362in}{2.474259in}%
\pgfsys@useobject{currentmarker}{}%
\end{pgfscope}%
\end{pgfscope}%
\begin{pgfscope}%
\definecolor{textcolor}{rgb}{0.000000,0.000000,0.000000}%
\pgfsetstrokecolor{textcolor}%
\pgfsetfillcolor{textcolor}%
\pgftext[x=1.939362in,y=2.377037in,,top]{\color{textcolor}{\sffamily\fontsize{10.000000}{12.000000}\selectfont\catcode`\^=\active\def^{\ifmmode\sp\else\^{}\fi}\catcode`\%=\active\def%{\%}2000}}%
\end{pgfscope}%
\begin{pgfscope}%
\pgfsetbuttcap%
\pgfsetroundjoin%
\definecolor{currentfill}{rgb}{0.000000,0.000000,0.000000}%
\pgfsetfillcolor{currentfill}%
\pgfsetlinewidth{0.803000pt}%
\definecolor{currentstroke}{rgb}{0.000000,0.000000,0.000000}%
\pgfsetstrokecolor{currentstroke}%
\pgfsetdash{}{0pt}%
\pgfsys@defobject{currentmarker}{\pgfqpoint{0.000000in}{-0.048611in}}{\pgfqpoint{0.000000in}{0.000000in}}{%
\pgfpathmoveto{\pgfqpoint{0.000000in}{0.000000in}}%
\pgfpathlineto{\pgfqpoint{0.000000in}{-0.048611in}}%
\pgfusepath{stroke,fill}%
}%
\begin{pgfscope}%
\pgfsys@transformshift{2.781918in}{2.474259in}%
\pgfsys@useobject{currentmarker}{}%
\end{pgfscope}%
\end{pgfscope}%
\begin{pgfscope}%
\definecolor{textcolor}{rgb}{0.000000,0.000000,0.000000}%
\pgfsetstrokecolor{textcolor}%
\pgfsetfillcolor{textcolor}%
\pgftext[x=2.781918in,y=2.377037in,,top]{\color{textcolor}{\sffamily\fontsize{10.000000}{12.000000}\selectfont\catcode`\^=\active\def^{\ifmmode\sp\else\^{}\fi}\catcode`\%=\active\def%{\%}4000}}%
\end{pgfscope}%
\begin{pgfscope}%
\pgfsetbuttcap%
\pgfsetroundjoin%
\definecolor{currentfill}{rgb}{0.000000,0.000000,0.000000}%
\pgfsetfillcolor{currentfill}%
\pgfsetlinewidth{0.803000pt}%
\definecolor{currentstroke}{rgb}{0.000000,0.000000,0.000000}%
\pgfsetstrokecolor{currentstroke}%
\pgfsetdash{}{0pt}%
\pgfsys@defobject{currentmarker}{\pgfqpoint{0.000000in}{-0.048611in}}{\pgfqpoint{0.000000in}{0.000000in}}{%
\pgfpathmoveto{\pgfqpoint{0.000000in}{0.000000in}}%
\pgfpathlineto{\pgfqpoint{0.000000in}{-0.048611in}}%
\pgfusepath{stroke,fill}%
}%
\begin{pgfscope}%
\pgfsys@transformshift{3.624474in}{2.474259in}%
\pgfsys@useobject{currentmarker}{}%
\end{pgfscope}%
\end{pgfscope}%
\begin{pgfscope}%
\definecolor{textcolor}{rgb}{0.000000,0.000000,0.000000}%
\pgfsetstrokecolor{textcolor}%
\pgfsetfillcolor{textcolor}%
\pgftext[x=3.624474in,y=2.377037in,,top]{\color{textcolor}{\sffamily\fontsize{10.000000}{12.000000}\selectfont\catcode`\^=\active\def^{\ifmmode\sp\else\^{}\fi}\catcode`\%=\active\def%{\%}6000}}%
\end{pgfscope}%
\begin{pgfscope}%
\pgfsetbuttcap%
\pgfsetroundjoin%
\definecolor{currentfill}{rgb}{0.000000,0.000000,0.000000}%
\pgfsetfillcolor{currentfill}%
\pgfsetlinewidth{0.803000pt}%
\definecolor{currentstroke}{rgb}{0.000000,0.000000,0.000000}%
\pgfsetstrokecolor{currentstroke}%
\pgfsetdash{}{0pt}%
\pgfsys@defobject{currentmarker}{\pgfqpoint{0.000000in}{-0.048611in}}{\pgfqpoint{0.000000in}{0.000000in}}{%
\pgfpathmoveto{\pgfqpoint{0.000000in}{0.000000in}}%
\pgfpathlineto{\pgfqpoint{0.000000in}{-0.048611in}}%
\pgfusepath{stroke,fill}%
}%
\begin{pgfscope}%
\pgfsys@transformshift{4.467031in}{2.474259in}%
\pgfsys@useobject{currentmarker}{}%
\end{pgfscope}%
\end{pgfscope}%
\begin{pgfscope}%
\definecolor{textcolor}{rgb}{0.000000,0.000000,0.000000}%
\pgfsetstrokecolor{textcolor}%
\pgfsetfillcolor{textcolor}%
\pgftext[x=4.467031in,y=2.377037in,,top]{\color{textcolor}{\sffamily\fontsize{10.000000}{12.000000}\selectfont\catcode`\^=\active\def^{\ifmmode\sp\else\^{}\fi}\catcode`\%=\active\def%{\%}8000}}%
\end{pgfscope}%
\begin{pgfscope}%
\pgfsetbuttcap%
\pgfsetroundjoin%
\definecolor{currentfill}{rgb}{0.000000,0.000000,0.000000}%
\pgfsetfillcolor{currentfill}%
\pgfsetlinewidth{0.803000pt}%
\definecolor{currentstroke}{rgb}{0.000000,0.000000,0.000000}%
\pgfsetstrokecolor{currentstroke}%
\pgfsetdash{}{0pt}%
\pgfsys@defobject{currentmarker}{\pgfqpoint{0.000000in}{-0.048611in}}{\pgfqpoint{0.000000in}{0.000000in}}{%
\pgfpathmoveto{\pgfqpoint{0.000000in}{0.000000in}}%
\pgfpathlineto{\pgfqpoint{0.000000in}{-0.048611in}}%
\pgfusepath{stroke,fill}%
}%
\begin{pgfscope}%
\pgfsys@transformshift{5.309587in}{2.474259in}%
\pgfsys@useobject{currentmarker}{}%
\end{pgfscope}%
\end{pgfscope}%
\begin{pgfscope}%
\definecolor{textcolor}{rgb}{0.000000,0.000000,0.000000}%
\pgfsetstrokecolor{textcolor}%
\pgfsetfillcolor{textcolor}%
\pgftext[x=5.309587in,y=2.377037in,,top]{\color{textcolor}{\sffamily\fontsize{10.000000}{12.000000}\selectfont\catcode`\^=\active\def^{\ifmmode\sp\else\^{}\fi}\catcode`\%=\active\def%{\%}10000}}%
\end{pgfscope}%
\begin{pgfscope}%
\pgfsetbuttcap%
\pgfsetroundjoin%
\definecolor{currentfill}{rgb}{0.000000,0.000000,0.000000}%
\pgfsetfillcolor{currentfill}%
\pgfsetlinewidth{0.803000pt}%
\definecolor{currentstroke}{rgb}{0.000000,0.000000,0.000000}%
\pgfsetstrokecolor{currentstroke}%
\pgfsetdash{}{0pt}%
\pgfsys@defobject{currentmarker}{\pgfqpoint{-0.048611in}{0.000000in}}{\pgfqpoint{-0.000000in}{0.000000in}}{%
\pgfpathmoveto{\pgfqpoint{-0.000000in}{0.000000in}}%
\pgfpathlineto{\pgfqpoint{-0.048611in}{0.000000in}}%
\pgfusepath{stroke,fill}%
}%
\begin{pgfscope}%
\pgfsys@transformshift{0.885050in}{2.474259in}%
\pgfsys@useobject{currentmarker}{}%
\end{pgfscope}%
\end{pgfscope}%
\begin{pgfscope}%
\definecolor{textcolor}{rgb}{0.000000,0.000000,0.000000}%
\pgfsetstrokecolor{textcolor}%
\pgfsetfillcolor{textcolor}%
\pgftext[x=0.699463in, y=2.421498in, left, base]{\color{textcolor}{\sffamily\fontsize{10.000000}{12.000000}\selectfont\catcode`\^=\active\def^{\ifmmode\sp\else\^{}\fi}\catcode`\%=\active\def%{\%}0}}%
\end{pgfscope}%
\begin{pgfscope}%
\pgfsetbuttcap%
\pgfsetroundjoin%
\definecolor{currentfill}{rgb}{0.000000,0.000000,0.000000}%
\pgfsetfillcolor{currentfill}%
\pgfsetlinewidth{0.803000pt}%
\definecolor{currentstroke}{rgb}{0.000000,0.000000,0.000000}%
\pgfsetstrokecolor{currentstroke}%
\pgfsetdash{}{0pt}%
\pgfsys@defobject{currentmarker}{\pgfqpoint{-0.048611in}{0.000000in}}{\pgfqpoint{-0.000000in}{0.000000in}}{%
\pgfpathmoveto{\pgfqpoint{-0.000000in}{0.000000in}}%
\pgfpathlineto{\pgfqpoint{-0.048611in}{0.000000in}}%
\pgfusepath{stroke,fill}%
}%
\begin{pgfscope}%
\pgfsys@transformshift{0.885050in}{3.086605in}%
\pgfsys@useobject{currentmarker}{}%
\end{pgfscope}%
\end{pgfscope}%
\begin{pgfscope}%
\definecolor{textcolor}{rgb}{0.000000,0.000000,0.000000}%
\pgfsetstrokecolor{textcolor}%
\pgfsetfillcolor{textcolor}%
\pgftext[x=0.611097in, y=3.033843in, left, base]{\color{textcolor}{\sffamily\fontsize{10.000000}{12.000000}\selectfont\catcode`\^=\active\def^{\ifmmode\sp\else\^{}\fi}\catcode`\%=\active\def%{\%}50}}%
\end{pgfscope}%
\begin{pgfscope}%
\pgfsetbuttcap%
\pgfsetroundjoin%
\definecolor{currentfill}{rgb}{0.000000,0.000000,0.000000}%
\pgfsetfillcolor{currentfill}%
\pgfsetlinewidth{0.803000pt}%
\definecolor{currentstroke}{rgb}{0.000000,0.000000,0.000000}%
\pgfsetstrokecolor{currentstroke}%
\pgfsetdash{}{0pt}%
\pgfsys@defobject{currentmarker}{\pgfqpoint{-0.048611in}{0.000000in}}{\pgfqpoint{-0.000000in}{0.000000in}}{%
\pgfpathmoveto{\pgfqpoint{-0.000000in}{0.000000in}}%
\pgfpathlineto{\pgfqpoint{-0.048611in}{0.000000in}}%
\pgfusepath{stroke,fill}%
}%
\begin{pgfscope}%
\pgfsys@transformshift{0.885050in}{3.698951in}%
\pgfsys@useobject{currentmarker}{}%
\end{pgfscope}%
\end{pgfscope}%
\begin{pgfscope}%
\definecolor{textcolor}{rgb}{0.000000,0.000000,0.000000}%
\pgfsetstrokecolor{textcolor}%
\pgfsetfillcolor{textcolor}%
\pgftext[x=0.522732in, y=3.646189in, left, base]{\color{textcolor}{\sffamily\fontsize{10.000000}{12.000000}\selectfont\catcode`\^=\active\def^{\ifmmode\sp\else\^{}\fi}\catcode`\%=\active\def%{\%}100}}%
\end{pgfscope}%
\begin{pgfscope}%
\definecolor{textcolor}{rgb}{0.000000,0.000000,0.000000}%
\pgfsetstrokecolor{textcolor}%
\pgfsetfillcolor{textcolor}%
\pgftext[x=0.467176in,y=3.117222in,,bottom,rotate=90.000000]{\color{textcolor}{\sffamily\fontsize{10.000000}{12.000000}\selectfont\catcode`\^=\active\def^{\ifmmode\sp\else\^{}\fi}\catcode`\%=\active\def%{\%}obciążenie (%)}}%
\end{pgfscope}%
\begin{pgfscope}%
\pgfpathrectangle{\pgfqpoint{0.885050in}{2.474259in}}{\pgfqpoint{4.955200in}{1.285926in}}%
\pgfusepath{clip}%
\pgfsetrectcap%
\pgfsetroundjoin%
\pgfsetlinewidth{1.505625pt}%
\definecolor{currentstroke}{rgb}{0.145098,0.145098,1.000000}%
\pgfsetstrokecolor{currentstroke}%
\pgfsetdash{}{0pt}%
\pgfpathmoveto{\pgfqpoint{1.123767in}{3.269068in}}%
\pgfpathlineto{\pgfqpoint{1.143568in}{3.268680in}}%
\pgfpathlineto{\pgfqpoint{1.148202in}{3.269023in}}%
\pgfpathlineto{\pgfqpoint{1.156206in}{3.269205in}}%
\pgfpathlineto{\pgfqpoint{1.168423in}{3.268086in}}%
\pgfpathlineto{\pgfqpoint{1.173057in}{3.266353in}}%
\pgfpathlineto{\pgfqpoint{1.181904in}{3.265908in}}%
\pgfpathlineto{\pgfqpoint{1.185274in}{3.266971in}}%
\pgfpathlineto{\pgfqpoint{1.192015in}{3.267965in}}%
\pgfpathlineto{\pgfqpoint{1.199598in}{3.267658in}}%
\pgfpathlineto{\pgfqpoint{1.203389in}{3.267240in}}%
\pgfpathlineto{\pgfqpoint{1.211393in}{3.267166in}}%
\pgfpathlineto{\pgfqpoint{1.215185in}{3.267180in}}%
\pgfpathlineto{\pgfqpoint{1.223610in}{3.267834in}}%
\pgfpathlineto{\pgfqpoint{1.232879in}{3.268926in}}%
\pgfpathlineto{\pgfqpoint{1.243410in}{3.268916in}}%
\pgfpathlineto{\pgfqpoint{1.253521in}{3.269919in}}%
\pgfpathlineto{\pgfqpoint{1.263211in}{3.270349in}}%
\pgfpathlineto{\pgfqpoint{1.263632in}{3.273875in}}%
\pgfpathlineto{\pgfqpoint{1.264474in}{3.273701in}}%
\pgfpathlineto{\pgfqpoint{1.269108in}{3.274709in}}%
\pgfpathlineto{\pgfqpoint{1.271636in}{3.273767in}}%
\pgfpathlineto{\pgfqpoint{1.275849in}{3.273587in}}%
\pgfpathlineto{\pgfqpoint{1.276691in}{3.273497in}}%
\pgfpathlineto{\pgfqpoint{1.277955in}{3.269563in}}%
\pgfpathlineto{\pgfqpoint{1.282168in}{3.266385in}}%
\pgfpathlineto{\pgfqpoint{1.285538in}{3.265711in}}%
\pgfpathlineto{\pgfqpoint{1.289751in}{3.263410in}}%
\pgfpathlineto{\pgfqpoint{1.297334in}{3.264998in}}%
\pgfpathlineto{\pgfqpoint{1.307445in}{3.268173in}}%
\pgfpathlineto{\pgfqpoint{1.328087in}{3.269269in}}%
\pgfpathlineto{\pgfqpoint{1.332721in}{3.268065in}}%
\pgfpathlineto{\pgfqpoint{1.343675in}{3.265918in}}%
\pgfpathlineto{\pgfqpoint{1.364739in}{3.266284in}}%
\pgfpathlineto{\pgfqpoint{1.371900in}{3.265031in}}%
\pgfpathlineto{\pgfqpoint{1.380326in}{3.266080in}}%
\pgfpathlineto{\pgfqpoint{1.409815in}{3.269732in}}%
\pgfpathlineto{\pgfqpoint{1.415713in}{3.270156in}}%
\pgfpathlineto{\pgfqpoint{1.416977in}{3.272667in}}%
\pgfpathlineto{\pgfqpoint{1.422032in}{3.271146in}}%
\pgfpathlineto{\pgfqpoint{1.427930in}{3.269406in}}%
\pgfpathlineto{\pgfqpoint{1.429194in}{3.269356in}}%
\pgfpathlineto{\pgfqpoint{1.430458in}{3.266757in}}%
\pgfpathlineto{\pgfqpoint{1.447730in}{3.267756in}}%
\pgfpathlineto{\pgfqpoint{1.451101in}{3.267743in}}%
\pgfpathlineto{\pgfqpoint{1.479326in}{3.271896in}}%
\pgfpathlineto{\pgfqpoint{1.495756in}{3.269023in}}%
\pgfpathlineto{\pgfqpoint{1.502075in}{3.267873in}}%
\pgfpathlineto{\pgfqpoint{1.511765in}{3.269276in}}%
\pgfpathlineto{\pgfqpoint{1.528194in}{3.270026in}}%
\pgfpathlineto{\pgfqpoint{1.533250in}{3.270753in}}%
\pgfpathlineto{\pgfqpoint{1.540833in}{3.270549in}}%
\pgfpathlineto{\pgfqpoint{1.547573in}{3.266350in}}%
\pgfpathlineto{\pgfqpoint{1.552629in}{3.266681in}}%
\pgfpathlineto{\pgfqpoint{1.563161in}{3.264488in}}%
\pgfpathlineto{\pgfqpoint{1.566531in}{3.262537in}}%
\pgfpathlineto{\pgfqpoint{1.580012in}{3.262255in}}%
\pgfpathlineto{\pgfqpoint{1.589280in}{3.264726in}}%
\pgfpathlineto{\pgfqpoint{1.603182in}{3.267176in}}%
\pgfpathlineto{\pgfqpoint{1.607395in}{3.266865in}}%
\pgfpathlineto{\pgfqpoint{1.612029in}{3.266101in}}%
\pgfpathlineto{\pgfqpoint{1.624246in}{3.267219in}}%
\pgfpathlineto{\pgfqpoint{1.628037in}{3.267629in}}%
\pgfpathlineto{\pgfqpoint{1.643625in}{3.266495in}}%
\pgfpathlineto{\pgfqpoint{1.646574in}{3.266795in}}%
\pgfpathlineto{\pgfqpoint{1.652472in}{3.267193in}}%
\pgfpathlineto{\pgfqpoint{1.656684in}{3.266769in}}%
\pgfpathlineto{\pgfqpoint{1.660897in}{3.265941in}}%
\pgfpathlineto{\pgfqpoint{1.665531in}{3.266397in}}%
\pgfpathlineto{\pgfqpoint{1.672272in}{3.268380in}}%
\pgfpathlineto{\pgfqpoint{1.677748in}{3.268479in}}%
\pgfpathlineto{\pgfqpoint{1.683646in}{3.267355in}}%
\pgfpathlineto{\pgfqpoint{1.691650in}{3.267974in}}%
\pgfpathlineto{\pgfqpoint{1.705974in}{3.269172in}}%
\pgfpathlineto{\pgfqpoint{1.742625in}{3.269064in}}%
\pgfpathlineto{\pgfqpoint{1.749787in}{3.269529in}}%
\pgfpathlineto{\pgfqpoint{1.757370in}{3.269636in}}%
\pgfpathlineto{\pgfqpoint{1.767902in}{3.269969in}}%
\pgfpathlineto{\pgfqpoint{1.783068in}{3.272486in}}%
\pgfpathlineto{\pgfqpoint{1.785595in}{3.271554in}}%
\pgfpathlineto{\pgfqpoint{1.789387in}{3.269900in}}%
\pgfpathlineto{\pgfqpoint{1.794021in}{3.268172in}}%
\pgfpathlineto{\pgfqpoint{1.814242in}{3.267462in}}%
\pgfpathlineto{\pgfqpoint{1.821825in}{3.267716in}}%
\pgfpathlineto{\pgfqpoint{1.830251in}{3.269291in}}%
\pgfpathlineto{\pgfqpoint{1.834464in}{3.270220in}}%
\pgfpathlineto{\pgfqpoint{1.844153in}{3.270785in}}%
\pgfpathlineto{\pgfqpoint{1.870272in}{3.268786in}}%
\pgfpathlineto{\pgfqpoint{1.884175in}{3.268987in}}%
\pgfpathlineto{\pgfqpoint{1.888387in}{3.268808in}}%
\pgfpathlineto{\pgfqpoint{1.911136in}{3.268663in}}%
\pgfpathlineto{\pgfqpoint{1.914928in}{3.268451in}}%
\pgfpathlineto{\pgfqpoint{1.921668in}{3.268623in}}%
\pgfpathlineto{\pgfqpoint{1.932621in}{3.269821in}}%
\pgfpathlineto{\pgfqpoint{1.937256in}{3.268247in}}%
\pgfpathlineto{\pgfqpoint{1.940626in}{3.266911in}}%
\pgfpathlineto{\pgfqpoint{1.952843in}{3.267939in}}%
\pgfpathlineto{\pgfqpoint{1.956213in}{3.267905in}}%
\pgfpathlineto{\pgfqpoint{1.968430in}{3.266483in}}%
\pgfpathlineto{\pgfqpoint{1.978962in}{3.262469in}}%
\pgfpathlineto{\pgfqpoint{1.985703in}{3.262845in}}%
\pgfpathlineto{\pgfqpoint{1.988651in}{3.263281in}}%
\pgfpathlineto{\pgfqpoint{2.000869in}{3.267885in}}%
\pgfpathlineto{\pgfqpoint{2.020669in}{3.265554in}}%
\pgfpathlineto{\pgfqpoint{2.022354in}{3.267186in}}%
\pgfpathlineto{\pgfqpoint{2.034150in}{3.268129in}}%
\pgfpathlineto{\pgfqpoint{2.035835in}{3.266572in}}%
\pgfpathlineto{\pgfqpoint{2.042575in}{3.266680in}}%
\pgfpathlineto{\pgfqpoint{2.048473in}{3.266282in}}%
\pgfpathlineto{\pgfqpoint{2.055213in}{3.263956in}}%
\pgfpathlineto{\pgfqpoint{2.057320in}{3.263539in}}%
\pgfpathlineto{\pgfqpoint{2.059426in}{3.262952in}}%
\pgfpathlineto{\pgfqpoint{2.065745in}{3.264518in}}%
\pgfpathlineto{\pgfqpoint{2.072486in}{3.266532in}}%
\pgfpathlineto{\pgfqpoint{2.087652in}{3.265211in}}%
\pgfpathlineto{\pgfqpoint{2.092707in}{3.264708in}}%
\pgfpathlineto{\pgfqpoint{2.125146in}{3.266947in}}%
\pgfpathlineto{\pgfqpoint{2.129780in}{3.267226in}}%
\pgfpathlineto{\pgfqpoint{2.133571in}{3.267511in}}%
\pgfpathlineto{\pgfqpoint{2.145367in}{3.269557in}}%
\pgfpathlineto{\pgfqpoint{2.147473in}{3.269362in}}%
\pgfpathlineto{\pgfqpoint{2.150844in}{3.270001in}}%
\pgfpathlineto{\pgfqpoint{2.154635in}{3.268854in}}%
\pgfpathlineto{\pgfqpoint{2.162639in}{3.266311in}}%
\pgfpathlineto{\pgfqpoint{2.164746in}{3.265416in}}%
\pgfpathlineto{\pgfqpoint{2.167273in}{3.265306in}}%
\pgfpathlineto{\pgfqpoint{2.174856in}{3.262359in}}%
\pgfpathlineto{\pgfqpoint{2.181176in}{3.264195in}}%
\pgfpathlineto{\pgfqpoint{2.183703in}{3.265044in}}%
\pgfpathlineto{\pgfqpoint{2.192971in}{3.267515in}}%
\pgfpathlineto{\pgfqpoint{2.197605in}{3.267748in}}%
\pgfpathlineto{\pgfqpoint{2.202661in}{3.268092in}}%
\pgfpathlineto{\pgfqpoint{2.208137in}{3.268098in}}%
\pgfpathlineto{\pgfqpoint{2.214035in}{3.266456in}}%
\pgfpathlineto{\pgfqpoint{2.224146in}{3.262731in}}%
\pgfpathlineto{\pgfqpoint{2.228359in}{3.263130in}}%
\pgfpathlineto{\pgfqpoint{2.243103in}{3.266088in}}%
\pgfpathlineto{\pgfqpoint{2.257006in}{3.264270in}}%
\pgfpathlineto{\pgfqpoint{2.261218in}{3.264827in}}%
\pgfpathlineto{\pgfqpoint{2.273857in}{3.267178in}}%
\pgfpathlineto{\pgfqpoint{2.297027in}{3.268192in}}%
\pgfpathlineto{\pgfqpoint{2.306295in}{3.267082in}}%
\pgfpathlineto{\pgfqpoint{2.312193in}{3.266991in}}%
\pgfpathlineto{\pgfqpoint{2.343789in}{3.268450in}}%
\pgfpathlineto{\pgfqpoint{2.352215in}{3.271005in}}%
\pgfpathlineto{\pgfqpoint{2.355163in}{3.271499in}}%
\pgfpathlineto{\pgfqpoint{2.361904in}{3.271366in}}%
\pgfpathlineto{\pgfqpoint{2.366959in}{3.271827in}}%
\pgfpathlineto{\pgfqpoint{2.372015in}{3.270757in}}%
\pgfpathlineto{\pgfqpoint{2.387181in}{3.267936in}}%
\pgfpathlineto{\pgfqpoint{2.401504in}{3.269793in}}%
\pgfpathlineto{\pgfqpoint{2.404032in}{3.269679in}}%
\pgfpathlineto{\pgfqpoint{2.409508in}{3.269820in}}%
\pgfpathlineto{\pgfqpoint{2.421304in}{3.271357in}}%
\pgfpathlineto{\pgfqpoint{2.433521in}{3.265312in}}%
\pgfpathlineto{\pgfqpoint{2.437734in}{3.265236in}}%
\pgfpathlineto{\pgfqpoint{2.446160in}{3.262680in}}%
\pgfpathlineto{\pgfqpoint{2.449108in}{3.261420in}}%
\pgfpathlineto{\pgfqpoint{2.458377in}{3.261367in}}%
\pgfpathlineto{\pgfqpoint{2.474806in}{3.265583in}}%
\pgfpathlineto{\pgfqpoint{2.481968in}{3.267047in}}%
\pgfpathlineto{\pgfqpoint{2.495870in}{3.266086in}}%
\pgfpathlineto{\pgfqpoint{2.498819in}{3.266103in}}%
\pgfpathlineto{\pgfqpoint{2.513564in}{3.267105in}}%
\pgfpathlineto{\pgfqpoint{2.520726in}{3.266529in}}%
\pgfpathlineto{\pgfqpoint{2.533785in}{3.267226in}}%
\pgfpathlineto{\pgfqpoint{2.540947in}{3.266543in}}%
\pgfpathlineto{\pgfqpoint{2.544317in}{3.266031in}}%
\pgfpathlineto{\pgfqpoint{2.555271in}{3.267984in}}%
\pgfpathlineto{\pgfqpoint{2.556534in}{3.272162in}}%
\pgfpathlineto{\pgfqpoint{2.559905in}{3.273411in}}%
\pgfpathlineto{\pgfqpoint{2.565803in}{3.274616in}}%
\pgfpathlineto{\pgfqpoint{2.568751in}{3.275700in}}%
\pgfpathlineto{\pgfqpoint{2.569173in}{3.271862in}}%
\pgfpathlineto{\pgfqpoint{2.570015in}{3.272271in}}%
\pgfpathlineto{\pgfqpoint{2.573807in}{3.273007in}}%
\pgfpathlineto{\pgfqpoint{2.575492in}{3.273923in}}%
\pgfpathlineto{\pgfqpoint{2.577598in}{3.274175in}}%
\pgfpathlineto{\pgfqpoint{2.588130in}{3.274436in}}%
\pgfpathlineto{\pgfqpoint{2.591922in}{3.273973in}}%
\pgfpathlineto{\pgfqpoint{2.596135in}{3.274057in}}%
\pgfpathlineto{\pgfqpoint{2.635735in}{3.275222in}}%
\pgfpathlineto{\pgfqpoint{2.637420in}{3.276155in}}%
\pgfpathlineto{\pgfqpoint{2.640369in}{3.276315in}}%
\pgfpathlineto{\pgfqpoint{2.653428in}{3.274399in}}%
\pgfpathlineto{\pgfqpoint{2.657220in}{3.274838in}}%
\pgfpathlineto{\pgfqpoint{2.669437in}{3.274877in}}%
\pgfpathlineto{\pgfqpoint{2.673228in}{3.273470in}}%
\pgfpathlineto{\pgfqpoint{2.689658in}{3.271770in}}%
\pgfpathlineto{\pgfqpoint{2.693029in}{3.272208in}}%
\pgfpathlineto{\pgfqpoint{2.703982in}{3.272769in}}%
\pgfpathlineto{\pgfqpoint{2.723361in}{3.276534in}}%
\pgfpathlineto{\pgfqpoint{2.727573in}{3.276021in}}%
\pgfpathlineto{\pgfqpoint{2.730944in}{3.275968in}}%
\pgfpathlineto{\pgfqpoint{2.748637in}{3.273358in}}%
\pgfpathlineto{\pgfqpoint{2.754114in}{3.273891in}}%
\pgfpathlineto{\pgfqpoint{2.774756in}{3.273036in}}%
\pgfpathlineto{\pgfqpoint{2.778969in}{3.273173in}}%
\pgfpathlineto{\pgfqpoint{2.784867in}{3.274042in}}%
\pgfpathlineto{\pgfqpoint{2.810986in}{3.273963in}}%
\pgfpathlineto{\pgfqpoint{2.817306in}{3.273977in}}%
\pgfpathlineto{\pgfqpoint{2.826995in}{3.272958in}}%
\pgfpathlineto{\pgfqpoint{2.832893in}{3.273207in}}%
\pgfpathlineto{\pgfqpoint{2.837106in}{3.272752in}}%
\pgfpathlineto{\pgfqpoint{2.860276in}{3.262482in}}%
\pgfpathlineto{\pgfqpoint{2.866174in}{3.262630in}}%
\pgfpathlineto{\pgfqpoint{2.875863in}{3.264883in}}%
\pgfpathlineto{\pgfqpoint{2.878812in}{3.265777in}}%
\pgfpathlineto{\pgfqpoint{2.896506in}{3.266502in}}%
\pgfpathlineto{\pgfqpoint{2.901561in}{3.266654in}}%
\pgfpathlineto{\pgfqpoint{2.906617in}{3.265278in}}%
\pgfpathlineto{\pgfqpoint{2.912093in}{3.263302in}}%
\pgfpathlineto{\pgfqpoint{2.918834in}{3.263371in}}%
\pgfpathlineto{\pgfqpoint{2.928102in}{3.266453in}}%
\pgfpathlineto{\pgfqpoint{2.933578in}{3.267311in}}%
\pgfpathlineto{\pgfqpoint{2.939476in}{3.268080in}}%
\pgfpathlineto{\pgfqpoint{2.962225in}{3.265158in}}%
\pgfpathlineto{\pgfqpoint{2.978234in}{3.266905in}}%
\pgfpathlineto{\pgfqpoint{2.981183in}{3.266969in}}%
\pgfpathlineto{\pgfqpoint{3.004774in}{3.267246in}}%
\pgfpathlineto{\pgfqpoint{3.009408in}{3.266566in}}%
\pgfpathlineto{\pgfqpoint{3.013621in}{3.266638in}}%
\pgfpathlineto{\pgfqpoint{3.016991in}{3.266408in}}%
\pgfpathlineto{\pgfqpoint{3.021204in}{3.266195in}}%
\pgfpathlineto{\pgfqpoint{3.028366in}{3.265262in}}%
\pgfpathlineto{\pgfqpoint{3.099983in}{3.270013in}}%
\pgfpathlineto{\pgfqpoint{3.110515in}{3.269339in}}%
\pgfpathlineto{\pgfqpoint{3.115149in}{3.269126in}}%
\pgfpathlineto{\pgfqpoint{3.119362in}{3.268196in}}%
\pgfpathlineto{\pgfqpoint{3.134528in}{3.267616in}}%
\pgfpathlineto{\pgfqpoint{3.140005in}{3.267526in}}%
\pgfpathlineto{\pgfqpoint{3.153486in}{3.268146in}}%
\pgfpathlineto{\pgfqpoint{3.172022in}{3.268223in}}%
\pgfpathlineto{\pgfqpoint{3.177920in}{3.268833in}}%
\pgfpathlineto{\pgfqpoint{3.192664in}{3.272069in}}%
\pgfpathlineto{\pgfqpoint{3.195192in}{3.270549in}}%
\pgfpathlineto{\pgfqpoint{3.199826in}{3.268574in}}%
\pgfpathlineto{\pgfqpoint{3.209937in}{3.270017in}}%
\pgfpathlineto{\pgfqpoint{3.212464in}{3.270680in}}%
\pgfpathlineto{\pgfqpoint{3.221311in}{3.270023in}}%
\pgfpathlineto{\pgfqpoint{3.226367in}{3.268338in}}%
\pgfpathlineto{\pgfqpoint{3.231843in}{3.268971in}}%
\pgfpathlineto{\pgfqpoint{3.239848in}{3.268108in}}%
\pgfpathlineto{\pgfqpoint{3.246588in}{3.266439in}}%
\pgfpathlineto{\pgfqpoint{3.256699in}{3.267780in}}%
\pgfpathlineto{\pgfqpoint{3.268494in}{3.263613in}}%
\pgfpathlineto{\pgfqpoint{3.275235in}{3.264059in}}%
\pgfpathlineto{\pgfqpoint{3.286188in}{3.266784in}}%
\pgfpathlineto{\pgfqpoint{3.315678in}{3.266811in}}%
\pgfpathlineto{\pgfqpoint{3.324103in}{3.263959in}}%
\pgfpathlineto{\pgfqpoint{3.333371in}{3.264763in}}%
\pgfpathlineto{\pgfqpoint{3.340954in}{3.266568in}}%
\pgfpathlineto{\pgfqpoint{3.362861in}{3.266676in}}%
\pgfpathlineto{\pgfqpoint{3.367495in}{3.265493in}}%
\pgfpathlineto{\pgfqpoint{3.370444in}{3.264763in}}%
\pgfpathlineto{\pgfqpoint{3.375920in}{3.263614in}}%
\pgfpathlineto{\pgfqpoint{3.380133in}{3.265205in}}%
\pgfpathlineto{\pgfqpoint{3.387295in}{3.268654in}}%
\pgfpathlineto{\pgfqpoint{3.391086in}{3.269194in}}%
\pgfpathlineto{\pgfqpoint{3.403303in}{3.263409in}}%
\pgfpathlineto{\pgfqpoint{3.416784in}{3.266417in}}%
\pgfpathlineto{\pgfqpoint{3.422261in}{3.264730in}}%
\pgfpathlineto{\pgfqpoint{3.428159in}{3.264022in}}%
\pgfpathlineto{\pgfqpoint{3.433635in}{3.265481in}}%
\pgfpathlineto{\pgfqpoint{3.442061in}{3.266464in}}%
\pgfpathlineto{\pgfqpoint{3.454278in}{3.264973in}}%
\pgfpathlineto{\pgfqpoint{3.458070in}{3.265824in}}%
\pgfpathlineto{\pgfqpoint{3.467759in}{3.264545in}}%
\pgfpathlineto{\pgfqpoint{3.474078in}{3.263400in}}%
\pgfpathlineto{\pgfqpoint{3.481240in}{3.263559in}}%
\pgfpathlineto{\pgfqpoint{3.485453in}{3.263703in}}%
\pgfpathlineto{\pgfqpoint{3.511572in}{3.264228in}}%
\pgfpathlineto{\pgfqpoint{3.516206in}{3.264475in}}%
\pgfpathlineto{\pgfqpoint{3.525474in}{3.264619in}}%
\pgfpathlineto{\pgfqpoint{3.529687in}{3.265084in}}%
\pgfpathlineto{\pgfqpoint{3.547381in}{3.263052in}}%
\pgfpathlineto{\pgfqpoint{3.550330in}{3.262573in}}%
\pgfpathlineto{\pgfqpoint{3.556227in}{3.262788in}}%
\pgfpathlineto{\pgfqpoint{3.558334in}{3.262668in}}%
\pgfpathlineto{\pgfqpoint{3.562547in}{3.262545in}}%
\pgfpathlineto{\pgfqpoint{3.565074in}{3.262822in}}%
\pgfpathlineto{\pgfqpoint{3.567602in}{3.263584in}}%
\pgfpathlineto{\pgfqpoint{3.573079in}{3.264871in}}%
\pgfpathlineto{\pgfqpoint{3.579398in}{3.267608in}}%
\pgfpathlineto{\pgfqpoint{3.584032in}{3.267366in}}%
\pgfpathlineto{\pgfqpoint{3.586981in}{3.267222in}}%
\pgfpathlineto{\pgfqpoint{3.591615in}{3.267442in}}%
\pgfpathlineto{\pgfqpoint{3.593721in}{3.267192in}}%
\pgfpathlineto{\pgfqpoint{3.598777in}{3.266801in}}%
\pgfpathlineto{\pgfqpoint{3.608045in}{3.263795in}}%
\pgfpathlineto{\pgfqpoint{3.628266in}{3.261946in}}%
\pgfpathlineto{\pgfqpoint{3.644696in}{3.262758in}}%
\pgfpathlineto{\pgfqpoint{3.646802in}{3.263144in}}%
\pgfpathlineto{\pgfqpoint{3.649330in}{3.263036in}}%
\pgfpathlineto{\pgfqpoint{3.651436in}{3.263630in}}%
\pgfpathlineto{\pgfqpoint{3.656492in}{3.265741in}}%
\pgfpathlineto{\pgfqpoint{3.660704in}{3.266342in}}%
\pgfpathlineto{\pgfqpoint{3.682190in}{3.267413in}}%
\pgfpathlineto{\pgfqpoint{3.684296in}{3.267421in}}%
\pgfpathlineto{\pgfqpoint{3.686402in}{3.267166in}}%
\pgfpathlineto{\pgfqpoint{3.712943in}{3.265598in}}%
\pgfpathlineto{\pgfqpoint{3.715892in}{3.265929in}}%
\pgfpathlineto{\pgfqpoint{3.733586in}{3.264475in}}%
\pgfpathlineto{\pgfqpoint{3.741590in}{3.265205in}}%
\pgfpathlineto{\pgfqpoint{3.759705in}{3.265052in}}%
\pgfpathlineto{\pgfqpoint{3.770658in}{3.259457in}}%
\pgfpathlineto{\pgfqpoint{3.775713in}{3.259895in}}%
\pgfpathlineto{\pgfqpoint{3.779084in}{3.261122in}}%
\pgfpathlineto{\pgfqpoint{3.785824in}{3.263169in}}%
\pgfpathlineto{\pgfqpoint{3.788773in}{3.263866in}}%
\pgfpathlineto{\pgfqpoint{3.794671in}{3.263148in}}%
\pgfpathlineto{\pgfqpoint{3.798884in}{3.263384in}}%
\pgfpathlineto{\pgfqpoint{3.803518in}{3.263392in}}%
\pgfpathlineto{\pgfqpoint{3.807730in}{3.264238in}}%
\pgfpathlineto{\pgfqpoint{3.811101in}{3.264622in}}%
\pgfpathlineto{\pgfqpoint{3.814050in}{3.264751in}}%
\pgfpathlineto{\pgfqpoint{3.817420in}{3.264277in}}%
\pgfpathlineto{\pgfqpoint{3.820790in}{3.264248in}}%
\pgfpathlineto{\pgfqpoint{3.827531in}{3.264444in}}%
\pgfpathlineto{\pgfqpoint{3.830479in}{3.265705in}}%
\pgfpathlineto{\pgfqpoint{3.857863in}{3.265175in}}%
\pgfpathlineto{\pgfqpoint{3.863339in}{3.265285in}}%
\pgfpathlineto{\pgfqpoint{3.868395in}{3.266244in}}%
\pgfpathlineto{\pgfqpoint{3.871765in}{3.265580in}}%
\pgfpathlineto{\pgfqpoint{3.875978in}{3.264437in}}%
\pgfpathlineto{\pgfqpoint{3.881454in}{3.261051in}}%
\pgfpathlineto{\pgfqpoint{3.891565in}{3.261057in}}%
\pgfpathlineto{\pgfqpoint{3.893671in}{3.261348in}}%
\pgfpathlineto{\pgfqpoint{3.897884in}{3.261459in}}%
\pgfpathlineto{\pgfqpoint{3.906310in}{3.262226in}}%
\pgfpathlineto{\pgfqpoint{3.907995in}{3.262051in}}%
\pgfpathlineto{\pgfqpoint{3.908837in}{3.267696in}}%
\pgfpathlineto{\pgfqpoint{3.909258in}{3.267625in}}%
\pgfpathlineto{\pgfqpoint{3.921476in}{3.266687in}}%
\pgfpathlineto{\pgfqpoint{3.922318in}{3.261025in}}%
\pgfpathlineto{\pgfqpoint{3.922739in}{3.261071in}}%
\pgfpathlineto{\pgfqpoint{3.954335in}{3.262571in}}%
\pgfpathlineto{\pgfqpoint{3.962340in}{3.265390in}}%
\pgfpathlineto{\pgfqpoint{3.981297in}{3.267164in}}%
\pgfpathlineto{\pgfqpoint{3.987195in}{3.266587in}}%
\pgfpathlineto{\pgfqpoint{3.992250in}{3.266811in}}%
\pgfpathlineto{\pgfqpoint{3.999412in}{3.268523in}}%
\pgfpathlineto{\pgfqpoint{4.022582in}{3.273630in}}%
\pgfpathlineto{\pgfqpoint{4.031008in}{3.272887in}}%
\pgfpathlineto{\pgfqpoint{4.035642in}{3.273212in}}%
\pgfpathlineto{\pgfqpoint{4.044067in}{3.274154in}}%
\pgfpathlineto{\pgfqpoint{4.049123in}{3.273656in}}%
\pgfpathlineto{\pgfqpoint{4.063025in}{3.272527in}}%
\pgfpathlineto{\pgfqpoint{4.071029in}{3.272170in}}%
\pgfpathlineto{\pgfqpoint{4.076085in}{3.272567in}}%
\pgfpathlineto{\pgfqpoint{4.092936in}{3.272500in}}%
\pgfpathlineto{\pgfqpoint{4.119476in}{3.274452in}}%
\pgfpathlineto{\pgfqpoint{4.130430in}{3.275428in}}%
\pgfpathlineto{\pgfqpoint{4.140540in}{3.273271in}}%
\pgfpathlineto{\pgfqpoint{4.146438in}{3.270866in}}%
\pgfpathlineto{\pgfqpoint{4.161604in}{3.268538in}}%
\pgfpathlineto{\pgfqpoint{4.167502in}{3.269189in}}%
\pgfpathlineto{\pgfqpoint{4.178876in}{3.268486in}}%
\pgfpathlineto{\pgfqpoint{4.185196in}{3.268141in}}%
\pgfpathlineto{\pgfqpoint{4.196991in}{3.268150in}}%
\pgfpathlineto{\pgfqpoint{4.202468in}{3.265605in}}%
\pgfpathlineto{\pgfqpoint{4.206681in}{3.264166in}}%
\pgfpathlineto{\pgfqpoint{4.215106in}{3.263627in}}%
\pgfpathlineto{\pgfqpoint{4.219740in}{3.263359in}}%
\pgfpathlineto{\pgfqpoint{4.223953in}{3.262163in}}%
\pgfpathlineto{\pgfqpoint{4.231115in}{3.262658in}}%
\pgfpathlineto{\pgfqpoint{4.235328in}{3.262847in}}%
\pgfpathlineto{\pgfqpoint{4.243753in}{3.263589in}}%
\pgfpathlineto{\pgfqpoint{4.247124in}{3.264246in}}%
\pgfpathlineto{\pgfqpoint{4.261447in}{3.267660in}}%
\pgfpathlineto{\pgfqpoint{4.264396in}{3.268005in}}%
\pgfpathlineto{\pgfqpoint{4.267766in}{3.268337in}}%
\pgfpathlineto{\pgfqpoint{4.275349in}{3.266093in}}%
\pgfpathlineto{\pgfqpoint{4.280405in}{3.263877in}}%
\pgfpathlineto{\pgfqpoint{4.295571in}{3.263216in}}%
\pgfpathlineto{\pgfqpoint{4.303154in}{3.264984in}}%
\pgfpathlineto{\pgfqpoint{4.314107in}{3.267177in}}%
\pgfpathlineto{\pgfqpoint{4.319162in}{3.268334in}}%
\pgfpathlineto{\pgfqpoint{4.324217in}{3.268634in}}%
\pgfpathlineto{\pgfqpoint{4.346124in}{3.264453in}}%
\pgfpathlineto{\pgfqpoint{4.350337in}{3.264332in}}%
\pgfpathlineto{\pgfqpoint{4.358762in}{3.265030in}}%
\pgfpathlineto{\pgfqpoint{4.362132in}{3.265474in}}%
\pgfpathlineto{\pgfqpoint{4.373086in}{3.266252in}}%
\pgfpathlineto{\pgfqpoint{4.377298in}{3.266714in}}%
\pgfpathlineto{\pgfqpoint{4.386567in}{3.266664in}}%
\pgfpathlineto{\pgfqpoint{4.391201in}{3.266725in}}%
\pgfpathlineto{\pgfqpoint{4.397099in}{3.267521in}}%
\pgfpathlineto{\pgfqpoint{4.400469in}{3.268016in}}%
\pgfpathlineto{\pgfqpoint{4.414792in}{3.268316in}}%
\pgfpathlineto{\pgfqpoint{4.416899in}{3.268251in}}%
\pgfpathlineto{\pgfqpoint{4.421533in}{3.269106in}}%
\pgfpathlineto{\pgfqpoint{4.440069in}{3.268118in}}%
\pgfpathlineto{\pgfqpoint{4.444703in}{3.268374in}}%
\pgfpathlineto{\pgfqpoint{4.448073in}{3.268306in}}%
\pgfpathlineto{\pgfqpoint{4.453129in}{3.268361in}}%
\pgfpathlineto{\pgfqpoint{4.469980in}{3.269515in}}%
\pgfpathlineto{\pgfqpoint{4.472929in}{3.269573in}}%
\pgfpathlineto{\pgfqpoint{4.530222in}{3.268592in}}%
\pgfpathlineto{\pgfqpoint{4.532750in}{3.268464in}}%
\pgfpathlineto{\pgfqpoint{4.550023in}{3.267774in}}%
\pgfpathlineto{\pgfqpoint{4.575720in}{3.268966in}}%
\pgfpathlineto{\pgfqpoint{4.597627in}{3.268344in}}%
\pgfpathlineto{\pgfqpoint{4.602261in}{3.268461in}}%
\pgfpathlineto{\pgfqpoint{4.634699in}{3.269887in}}%
\pgfpathlineto{\pgfqpoint{4.638912in}{3.270157in}}%
\pgfpathlineto{\pgfqpoint{4.641861in}{3.270294in}}%
\pgfpathlineto{\pgfqpoint{4.651129in}{3.268721in}}%
\pgfpathlineto{\pgfqpoint{4.657448in}{3.267933in}}%
\pgfpathlineto{\pgfqpoint{4.678091in}{3.269520in}}%
\pgfpathlineto{\pgfqpoint{4.685674in}{3.267967in}}%
\pgfpathlineto{\pgfqpoint{4.690729in}{3.268263in}}%
\pgfpathlineto{\pgfqpoint{4.697470in}{3.268821in}}%
\pgfpathlineto{\pgfqpoint{4.706317in}{3.270265in}}%
\pgfpathlineto{\pgfqpoint{4.710951in}{3.270987in}}%
\pgfpathlineto{\pgfqpoint{4.714742in}{3.270173in}}%
\pgfpathlineto{\pgfqpoint{4.718534in}{3.270064in}}%
\pgfpathlineto{\pgfqpoint{4.728644in}{3.269291in}}%
\pgfpathlineto{\pgfqpoint{4.734121in}{3.269244in}}%
\pgfpathlineto{\pgfqpoint{4.748866in}{3.270266in}}%
\pgfpathlineto{\pgfqpoint{4.753500in}{3.270402in}}%
\pgfpathlineto{\pgfqpoint{4.759398in}{3.270659in}}%
\pgfpathlineto{\pgfqpoint{4.766559in}{3.270099in}}%
\pgfpathlineto{\pgfqpoint{4.773721in}{3.269427in}}%
\pgfpathlineto{\pgfqpoint{4.781726in}{3.269835in}}%
\pgfpathlineto{\pgfqpoint{4.792679in}{3.265793in}}%
\pgfpathlineto{\pgfqpoint{4.801104in}{3.261319in}}%
\pgfpathlineto{\pgfqpoint{4.806160in}{3.261174in}}%
\pgfpathlineto{\pgfqpoint{4.812058in}{3.262883in}}%
\pgfpathlineto{\pgfqpoint{4.823853in}{3.270398in}}%
\pgfpathlineto{\pgfqpoint{4.839862in}{3.270610in}}%
\pgfpathlineto{\pgfqpoint{4.844496in}{3.270881in}}%
\pgfpathlineto{\pgfqpoint{4.859662in}{3.263583in}}%
\pgfpathlineto{\pgfqpoint{4.866824in}{3.263131in}}%
\pgfpathlineto{\pgfqpoint{4.895892in}{3.266772in}}%
\pgfpathlineto{\pgfqpoint{4.900105in}{3.266104in}}%
\pgfpathlineto{\pgfqpoint{4.915271in}{3.267508in}}%
\pgfpathlineto{\pgfqpoint{4.922432in}{3.268276in}}%
\pgfpathlineto{\pgfqpoint{4.927488in}{3.267462in}}%
\pgfpathlineto{\pgfqpoint{4.945603in}{3.265470in}}%
\pgfpathlineto{\pgfqpoint{4.948552in}{3.265211in}}%
\pgfpathlineto{\pgfqpoint{4.951922in}{3.265831in}}%
\pgfpathlineto{\pgfqpoint{4.967088in}{3.267645in}}%
\pgfpathlineto{\pgfqpoint{4.980148in}{3.268069in}}%
\pgfpathlineto{\pgfqpoint{4.985624in}{3.267838in}}%
\pgfpathlineto{\pgfqpoint{4.998262in}{3.268416in}}%
\pgfpathlineto{\pgfqpoint{5.051344in}{3.267651in}}%
\pgfpathlineto{\pgfqpoint{5.055556in}{3.267541in}}%
\pgfpathlineto{\pgfqpoint{5.060190in}{3.267723in}}%
\pgfpathlineto{\pgfqpoint{5.076620in}{3.268217in}}%
\pgfpathlineto{\pgfqpoint{5.081676in}{3.268007in}}%
\pgfpathlineto{\pgfqpoint{5.114114in}{3.264055in}}%
\pgfpathlineto{\pgfqpoint{5.122961in}{3.264592in}}%
\pgfpathlineto{\pgfqpoint{5.144867in}{3.264864in}}%
\pgfpathlineto{\pgfqpoint{5.147816in}{3.264540in}}%
\pgfpathlineto{\pgfqpoint{5.162982in}{3.265200in}}%
\pgfpathlineto{\pgfqpoint{5.168038in}{3.263888in}}%
\pgfpathlineto{\pgfqpoint{5.176463in}{3.262839in}}%
\pgfpathlineto{\pgfqpoint{5.181097in}{3.263907in}}%
\pgfpathlineto{\pgfqpoint{5.184889in}{3.264559in}}%
\pgfpathlineto{\pgfqpoint{5.195842in}{3.265237in}}%
\pgfpathlineto{\pgfqpoint{5.201319in}{3.263721in}}%
\pgfpathlineto{\pgfqpoint{5.204267in}{3.264061in}}%
\pgfpathlineto{\pgfqpoint{5.222804in}{3.265728in}}%
\pgfpathlineto{\pgfqpoint{5.230387in}{3.264321in}}%
\pgfpathlineto{\pgfqpoint{5.240919in}{3.265221in}}%
\pgfpathlineto{\pgfqpoint{5.244289in}{3.265460in}}%
\pgfpathlineto{\pgfqpoint{5.250608in}{3.266301in}}%
\pgfpathlineto{\pgfqpoint{5.254400in}{3.266703in}}%
\pgfpathlineto{\pgfqpoint{5.257349in}{3.267602in}}%
\pgfpathlineto{\pgfqpoint{5.259876in}{3.268200in}}%
\pgfpathlineto{\pgfqpoint{5.281361in}{3.269737in}}%
\pgfpathlineto{\pgfqpoint{5.285574in}{3.269805in}}%
\pgfpathlineto{\pgfqpoint{5.290629in}{3.268369in}}%
\pgfpathlineto{\pgfqpoint{5.298634in}{3.266244in}}%
\pgfpathlineto{\pgfqpoint{5.306217in}{3.268322in}}%
\pgfpathlineto{\pgfqpoint{5.312957in}{3.268728in}}%
\pgfpathlineto{\pgfqpoint{5.315906in}{3.268737in}}%
\pgfpathlineto{\pgfqpoint{5.319276in}{3.268620in}}%
\pgfpathlineto{\pgfqpoint{5.321804in}{3.268396in}}%
\pgfpathlineto{\pgfqpoint{5.335285in}{3.268109in}}%
\pgfpathlineto{\pgfqpoint{5.340762in}{3.266107in}}%
\pgfpathlineto{\pgfqpoint{5.355928in}{3.267355in}}%
\pgfpathlineto{\pgfqpoint{5.360140in}{3.268370in}}%
\pgfpathlineto{\pgfqpoint{5.374464in}{3.269269in}}%
\pgfpathlineto{\pgfqpoint{5.377413in}{3.268901in}}%
\pgfpathlineto{\pgfqpoint{5.381626in}{3.269348in}}%
\pgfpathlineto{\pgfqpoint{5.403953in}{3.268644in}}%
\pgfpathlineto{\pgfqpoint{5.406060in}{3.268157in}}%
\pgfpathlineto{\pgfqpoint{5.409851in}{3.266729in}}%
\pgfpathlineto{\pgfqpoint{5.414485in}{3.266815in}}%
\pgfpathlineto{\pgfqpoint{5.417855in}{3.266490in}}%
\pgfpathlineto{\pgfqpoint{5.421647in}{3.268185in}}%
\pgfpathlineto{\pgfqpoint{5.424175in}{3.268611in}}%
\pgfpathlineto{\pgfqpoint{5.430073in}{3.268901in}}%
\pgfpathlineto{\pgfqpoint{5.433443in}{3.268628in}}%
\pgfpathlineto{\pgfqpoint{5.446502in}{3.267418in}}%
\pgfpathlineto{\pgfqpoint{5.457877in}{3.267380in}}%
\pgfpathlineto{\pgfqpoint{5.464617in}{3.264615in}}%
\pgfpathlineto{\pgfqpoint{5.470094in}{3.262812in}}%
\pgfpathlineto{\pgfqpoint{5.485681in}{3.266228in}}%
\pgfpathlineto{\pgfqpoint{5.489894in}{3.266246in}}%
\pgfpathlineto{\pgfqpoint{5.496634in}{3.266475in}}%
\pgfpathlineto{\pgfqpoint{5.500847in}{3.266878in}}%
\pgfpathlineto{\pgfqpoint{5.505481in}{3.266855in}}%
\pgfpathlineto{\pgfqpoint{5.516435in}{3.267706in}}%
\pgfpathlineto{\pgfqpoint{5.527388in}{3.268671in}}%
\pgfpathlineto{\pgfqpoint{5.532443in}{3.266462in}}%
\pgfpathlineto{\pgfqpoint{5.536656in}{3.266139in}}%
\pgfpathlineto{\pgfqpoint{5.541290in}{3.265510in}}%
\pgfpathlineto{\pgfqpoint{5.555613in}{3.268501in}}%
\pgfpathlineto{\pgfqpoint{5.558562in}{3.269658in}}%
\pgfpathlineto{\pgfqpoint{5.560669in}{3.269830in}}%
\pgfpathlineto{\pgfqpoint{5.562354in}{3.269427in}}%
\pgfpathlineto{\pgfqpoint{5.564039in}{3.269040in}}%
\pgfpathlineto{\pgfqpoint{5.566145in}{3.269084in}}%
\pgfpathlineto{\pgfqpoint{5.572043in}{3.269766in}}%
\pgfpathlineto{\pgfqpoint{5.573728in}{3.269564in}}%
\pgfpathlineto{\pgfqpoint{5.577941in}{3.270047in}}%
\pgfpathlineto{\pgfqpoint{5.580469in}{3.269564in}}%
\pgfpathlineto{\pgfqpoint{5.582997in}{3.268189in}}%
\pgfpathlineto{\pgfqpoint{5.586788in}{3.266702in}}%
\pgfpathlineto{\pgfqpoint{5.589316in}{3.264903in}}%
\pgfpathlineto{\pgfqpoint{5.591422in}{3.265111in}}%
\pgfpathlineto{\pgfqpoint{5.593950in}{3.265805in}}%
\pgfpathlineto{\pgfqpoint{5.597320in}{3.267074in}}%
\pgfpathlineto{\pgfqpoint{5.599426in}{3.267633in}}%
\pgfpathlineto{\pgfqpoint{5.603639in}{3.269362in}}%
\pgfpathlineto{\pgfqpoint{5.607431in}{3.269356in}}%
\pgfpathlineto{\pgfqpoint{5.613750in}{3.269480in}}%
\pgfpathlineto{\pgfqpoint{5.613750in}{3.269480in}}%
\pgfusepath{stroke}%
\end{pgfscope}%
\begin{pgfscope}%
\pgfpathrectangle{\pgfqpoint{0.885050in}{2.474259in}}{\pgfqpoint{4.955200in}{1.285926in}}%
\pgfusepath{clip}%
\pgfsetrectcap%
\pgfsetroundjoin%
\pgfsetlinewidth{1.505625pt}%
\definecolor{currentstroke}{rgb}{1.000000,0.145098,0.145098}%
\pgfsetstrokecolor{currentstroke}%
\pgfsetdash{}{0pt}%
\pgfpathmoveto{\pgfqpoint{1.110287in}{3.221736in}}%
\pgfpathlineto{\pgfqpoint{1.111550in}{3.214308in}}%
\pgfpathlineto{\pgfqpoint{1.111972in}{3.215052in}}%
\pgfpathlineto{\pgfqpoint{1.112814in}{3.218177in}}%
\pgfpathlineto{\pgfqpoint{1.119133in}{3.260573in}}%
\pgfpathlineto{\pgfqpoint{1.122504in}{3.267180in}}%
\pgfpathlineto{\pgfqpoint{1.123346in}{3.266052in}}%
\pgfpathlineto{\pgfqpoint{1.125453in}{3.266130in}}%
\pgfpathlineto{\pgfqpoint{1.126295in}{3.265907in}}%
\pgfpathlineto{\pgfqpoint{1.127980in}{3.258680in}}%
\pgfpathlineto{\pgfqpoint{1.132614in}{3.233430in}}%
\pgfpathlineto{\pgfqpoint{1.133878in}{3.231284in}}%
\pgfpathlineto{\pgfqpoint{1.136827in}{3.224232in}}%
\pgfpathlineto{\pgfqpoint{1.138512in}{3.224750in}}%
\pgfpathlineto{\pgfqpoint{1.139776in}{3.224945in}}%
\pgfpathlineto{\pgfqpoint{1.141461in}{3.232061in}}%
\pgfpathlineto{\pgfqpoint{1.146095in}{3.257899in}}%
\pgfpathlineto{\pgfqpoint{1.147359in}{3.260232in}}%
\pgfpathlineto{\pgfqpoint{1.151572in}{3.279594in}}%
\pgfpathlineto{\pgfqpoint{1.155363in}{3.292880in}}%
\pgfpathlineto{\pgfqpoint{1.156206in}{3.292664in}}%
\pgfpathlineto{\pgfqpoint{1.158734in}{3.311025in}}%
\pgfpathlineto{\pgfqpoint{1.159155in}{3.310976in}}%
\pgfpathlineto{\pgfqpoint{1.160840in}{3.311137in}}%
\pgfpathlineto{\pgfqpoint{1.162104in}{3.315233in}}%
\pgfpathlineto{\pgfqpoint{1.162525in}{3.314369in}}%
\pgfpathlineto{\pgfqpoint{1.163368in}{3.308232in}}%
\pgfpathlineto{\pgfqpoint{1.165474in}{3.281232in}}%
\pgfpathlineto{\pgfqpoint{1.165895in}{3.286133in}}%
\pgfpathlineto{\pgfqpoint{1.167159in}{3.274575in}}%
\pgfpathlineto{\pgfqpoint{1.171793in}{3.224825in}}%
\pgfpathlineto{\pgfqpoint{1.176427in}{3.198422in}}%
\pgfpathlineto{\pgfqpoint{1.177270in}{3.201979in}}%
\pgfpathlineto{\pgfqpoint{1.178955in}{3.211692in}}%
\pgfpathlineto{\pgfqpoint{1.179376in}{3.205451in}}%
\pgfpathlineto{\pgfqpoint{1.180219in}{3.208675in}}%
\pgfpathlineto{\pgfqpoint{1.183168in}{3.222753in}}%
\pgfpathlineto{\pgfqpoint{1.185274in}{3.227350in}}%
\pgfpathlineto{\pgfqpoint{1.190329in}{3.250605in}}%
\pgfpathlineto{\pgfqpoint{1.192857in}{3.261242in}}%
\pgfpathlineto{\pgfqpoint{1.194121in}{3.268657in}}%
\pgfpathlineto{\pgfqpoint{1.194542in}{3.268516in}}%
\pgfpathlineto{\pgfqpoint{1.195806in}{3.271260in}}%
\pgfpathlineto{\pgfqpoint{1.200019in}{3.280153in}}%
\pgfpathlineto{\pgfqpoint{1.203389in}{3.272198in}}%
\pgfpathlineto{\pgfqpoint{1.204232in}{3.288110in}}%
\pgfpathlineto{\pgfqpoint{1.204653in}{3.286379in}}%
\pgfpathlineto{\pgfqpoint{1.205495in}{3.289435in}}%
\pgfpathlineto{\pgfqpoint{1.205917in}{3.289219in}}%
\pgfpathlineto{\pgfqpoint{1.211393in}{3.269720in}}%
\pgfpathlineto{\pgfqpoint{1.213078in}{3.266271in}}%
\pgfpathlineto{\pgfqpoint{1.213500in}{3.266558in}}%
\pgfpathlineto{\pgfqpoint{1.214764in}{3.267690in}}%
\pgfpathlineto{\pgfqpoint{1.216870in}{3.273441in}}%
\pgfpathlineto{\pgfqpoint{1.218555in}{3.251366in}}%
\pgfpathlineto{\pgfqpoint{1.219819in}{3.248888in}}%
\pgfpathlineto{\pgfqpoint{1.223610in}{3.253465in}}%
\pgfpathlineto{\pgfqpoint{1.227402in}{3.261822in}}%
\pgfpathlineto{\pgfqpoint{1.232457in}{3.263229in}}%
\pgfpathlineto{\pgfqpoint{1.233721in}{3.263154in}}%
\pgfpathlineto{\pgfqpoint{1.237513in}{3.259386in}}%
\pgfpathlineto{\pgfqpoint{1.238776in}{3.257247in}}%
\pgfpathlineto{\pgfqpoint{1.239198in}{3.257786in}}%
\pgfpathlineto{\pgfqpoint{1.240040in}{3.257777in}}%
\pgfpathlineto{\pgfqpoint{1.241725in}{3.256892in}}%
\pgfpathlineto{\pgfqpoint{1.244674in}{3.257331in}}%
\pgfpathlineto{\pgfqpoint{1.248466in}{3.257856in}}%
\pgfpathlineto{\pgfqpoint{1.250572in}{3.257478in}}%
\pgfpathlineto{\pgfqpoint{1.252679in}{3.257066in}}%
\pgfpathlineto{\pgfqpoint{1.253942in}{3.254749in}}%
\pgfpathlineto{\pgfqpoint{1.254364in}{3.255621in}}%
\pgfpathlineto{\pgfqpoint{1.257313in}{3.256868in}}%
\pgfpathlineto{\pgfqpoint{1.258576in}{3.258962in}}%
\pgfpathlineto{\pgfqpoint{1.258998in}{3.258296in}}%
\pgfpathlineto{\pgfqpoint{1.259419in}{3.259331in}}%
\pgfpathlineto{\pgfqpoint{1.261947in}{3.265912in}}%
\pgfpathlineto{\pgfqpoint{1.262368in}{3.266077in}}%
\pgfpathlineto{\pgfqpoint{1.264053in}{3.258656in}}%
\pgfpathlineto{\pgfqpoint{1.264896in}{3.253588in}}%
\pgfpathlineto{\pgfqpoint{1.265317in}{3.255559in}}%
\pgfpathlineto{\pgfqpoint{1.267423in}{3.263031in}}%
\pgfpathlineto{\pgfqpoint{1.267845in}{3.261850in}}%
\pgfpathlineto{\pgfqpoint{1.268687in}{3.260406in}}%
\pgfpathlineto{\pgfqpoint{1.270372in}{3.249036in}}%
\pgfpathlineto{\pgfqpoint{1.271215in}{3.243923in}}%
\pgfpathlineto{\pgfqpoint{1.271636in}{3.245007in}}%
\pgfpathlineto{\pgfqpoint{1.272479in}{3.248156in}}%
\pgfpathlineto{\pgfqpoint{1.272900in}{3.247515in}}%
\pgfpathlineto{\pgfqpoint{1.274585in}{3.245429in}}%
\pgfpathlineto{\pgfqpoint{1.277113in}{3.243921in}}%
\pgfpathlineto{\pgfqpoint{1.277534in}{3.244936in}}%
\pgfpathlineto{\pgfqpoint{1.278377in}{3.246018in}}%
\pgfpathlineto{\pgfqpoint{1.282589in}{3.220693in}}%
\pgfpathlineto{\pgfqpoint{1.283432in}{3.221691in}}%
\pgfpathlineto{\pgfqpoint{1.284274in}{3.223709in}}%
\pgfpathlineto{\pgfqpoint{1.284696in}{3.222292in}}%
\pgfpathlineto{\pgfqpoint{1.290594in}{3.182436in}}%
\pgfpathlineto{\pgfqpoint{1.293543in}{3.176317in}}%
\pgfpathlineto{\pgfqpoint{1.297755in}{3.199786in}}%
\pgfpathlineto{\pgfqpoint{1.300283in}{3.216468in}}%
\pgfpathlineto{\pgfqpoint{1.305338in}{3.267505in}}%
\pgfpathlineto{\pgfqpoint{1.308287in}{3.287809in}}%
\pgfpathlineto{\pgfqpoint{1.310815in}{3.295919in}}%
\pgfpathlineto{\pgfqpoint{1.311658in}{3.297758in}}%
\pgfpathlineto{\pgfqpoint{1.312500in}{3.301421in}}%
\pgfpathlineto{\pgfqpoint{1.313343in}{3.300592in}}%
\pgfpathlineto{\pgfqpoint{1.327245in}{3.299451in}}%
\pgfpathlineto{\pgfqpoint{1.328930in}{3.290225in}}%
\pgfpathlineto{\pgfqpoint{1.339041in}{3.202927in}}%
\pgfpathlineto{\pgfqpoint{1.341990in}{3.190407in}}%
\pgfpathlineto{\pgfqpoint{1.342411in}{3.190531in}}%
\pgfpathlineto{\pgfqpoint{1.359683in}{3.245568in}}%
\pgfpathlineto{\pgfqpoint{1.360105in}{3.245137in}}%
\pgfpathlineto{\pgfqpoint{1.362211in}{3.238895in}}%
\pgfpathlineto{\pgfqpoint{1.371479in}{3.209094in}}%
\pgfpathlineto{\pgfqpoint{1.373164in}{3.207925in}}%
\pgfpathlineto{\pgfqpoint{1.374849in}{3.213876in}}%
\pgfpathlineto{\pgfqpoint{1.381168in}{3.238939in}}%
\pgfpathlineto{\pgfqpoint{1.385802in}{3.255839in}}%
\pgfpathlineto{\pgfqpoint{1.387909in}{3.257918in}}%
\pgfpathlineto{\pgfqpoint{1.388330in}{3.257043in}}%
\pgfpathlineto{\pgfqpoint{1.389173in}{3.259471in}}%
\pgfpathlineto{\pgfqpoint{1.389594in}{3.258857in}}%
\pgfpathlineto{\pgfqpoint{1.392964in}{3.258294in}}%
\pgfpathlineto{\pgfqpoint{1.394228in}{3.260795in}}%
\pgfpathlineto{\pgfqpoint{1.400968in}{3.255565in}}%
\pgfpathlineto{\pgfqpoint{1.403075in}{3.250118in}}%
\pgfpathlineto{\pgfqpoint{1.403496in}{3.250970in}}%
\pgfpathlineto{\pgfqpoint{1.411079in}{3.272735in}}%
\pgfpathlineto{\pgfqpoint{1.412764in}{3.276907in}}%
\pgfpathlineto{\pgfqpoint{1.416135in}{3.280877in}}%
\pgfpathlineto{\pgfqpoint{1.416977in}{3.282326in}}%
\pgfpathlineto{\pgfqpoint{1.418662in}{3.284587in}}%
\pgfpathlineto{\pgfqpoint{1.419083in}{3.283881in}}%
\pgfpathlineto{\pgfqpoint{1.422032in}{3.271900in}}%
\pgfpathlineto{\pgfqpoint{1.424981in}{3.257712in}}%
\pgfpathlineto{\pgfqpoint{1.425403in}{3.258323in}}%
\pgfpathlineto{\pgfqpoint{1.427088in}{3.257563in}}%
\pgfpathlineto{\pgfqpoint{1.430037in}{3.257419in}}%
\pgfpathlineto{\pgfqpoint{1.431722in}{3.257697in}}%
\pgfpathlineto{\pgfqpoint{1.438041in}{3.257396in}}%
\pgfpathlineto{\pgfqpoint{1.440569in}{3.257566in}}%
\pgfpathlineto{\pgfqpoint{1.444781in}{3.257914in}}%
\pgfpathlineto{\pgfqpoint{1.446045in}{3.258165in}}%
\pgfpathlineto{\pgfqpoint{1.449415in}{3.258176in}}%
\pgfpathlineto{\pgfqpoint{1.451522in}{3.258653in}}%
\pgfpathlineto{\pgfqpoint{1.452786in}{3.258563in}}%
\pgfpathlineto{\pgfqpoint{1.454892in}{3.258065in}}%
\pgfpathlineto{\pgfqpoint{1.457420in}{3.258709in}}%
\pgfpathlineto{\pgfqpoint{1.459947in}{3.257397in}}%
\pgfpathlineto{\pgfqpoint{1.462054in}{3.262286in}}%
\pgfpathlineto{\pgfqpoint{1.463739in}{3.269298in}}%
\pgfpathlineto{\pgfqpoint{1.464581in}{3.268489in}}%
\pgfpathlineto{\pgfqpoint{1.466267in}{3.266995in}}%
\pgfpathlineto{\pgfqpoint{1.466688in}{3.267650in}}%
\pgfpathlineto{\pgfqpoint{1.469637in}{3.276655in}}%
\pgfpathlineto{\pgfqpoint{1.470479in}{3.279663in}}%
\pgfpathlineto{\pgfqpoint{1.470901in}{3.279017in}}%
\pgfpathlineto{\pgfqpoint{1.473007in}{3.278768in}}%
\pgfpathlineto{\pgfqpoint{1.473850in}{3.279449in}}%
\pgfpathlineto{\pgfqpoint{1.474271in}{3.278923in}}%
\pgfpathlineto{\pgfqpoint{1.476377in}{3.278870in}}%
\pgfpathlineto{\pgfqpoint{1.476799in}{3.279555in}}%
\pgfpathlineto{\pgfqpoint{1.477641in}{3.278945in}}%
\pgfpathlineto{\pgfqpoint{1.480590in}{3.278153in}}%
\pgfpathlineto{\pgfqpoint{1.482696in}{3.276366in}}%
\pgfpathlineto{\pgfqpoint{1.484803in}{3.279364in}}%
\pgfpathlineto{\pgfqpoint{1.487752in}{3.279094in}}%
\pgfpathlineto{\pgfqpoint{1.488173in}{3.279845in}}%
\pgfpathlineto{\pgfqpoint{1.489016in}{3.279241in}}%
\pgfpathlineto{\pgfqpoint{1.496599in}{3.282094in}}%
\pgfpathlineto{\pgfqpoint{1.498284in}{3.280319in}}%
\pgfpathlineto{\pgfqpoint{1.500811in}{3.280248in}}%
\pgfpathlineto{\pgfqpoint{1.501233in}{3.280733in}}%
\pgfpathlineto{\pgfqpoint{1.501654in}{3.280087in}}%
\pgfpathlineto{\pgfqpoint{1.505445in}{3.278678in}}%
\pgfpathlineto{\pgfqpoint{1.506709in}{3.280369in}}%
\pgfpathlineto{\pgfqpoint{1.510501in}{3.280900in}}%
\pgfpathlineto{\pgfqpoint{1.510922in}{3.281463in}}%
\pgfpathlineto{\pgfqpoint{1.511343in}{3.280608in}}%
\pgfpathlineto{\pgfqpoint{1.513450in}{3.279541in}}%
\pgfpathlineto{\pgfqpoint{1.514292in}{3.279989in}}%
\pgfpathlineto{\pgfqpoint{1.515135in}{3.278962in}}%
\pgfpathlineto{\pgfqpoint{1.515556in}{3.279700in}}%
\pgfpathlineto{\pgfqpoint{1.518084in}{3.278201in}}%
\pgfpathlineto{\pgfqpoint{1.518926in}{3.279630in}}%
\pgfpathlineto{\pgfqpoint{1.519348in}{3.278046in}}%
\pgfpathlineto{\pgfqpoint{1.521033in}{3.276506in}}%
\pgfpathlineto{\pgfqpoint{1.521875in}{3.277259in}}%
\pgfpathlineto{\pgfqpoint{1.522297in}{3.277548in}}%
\pgfpathlineto{\pgfqpoint{1.523982in}{3.275153in}}%
\pgfpathlineto{\pgfqpoint{1.524824in}{3.274356in}}%
\pgfpathlineto{\pgfqpoint{1.525246in}{3.275355in}}%
\pgfpathlineto{\pgfqpoint{1.525667in}{3.274950in}}%
\pgfpathlineto{\pgfqpoint{1.526509in}{3.274386in}}%
\pgfpathlineto{\pgfqpoint{1.526931in}{3.275240in}}%
\pgfpathlineto{\pgfqpoint{1.527352in}{3.274526in}}%
\pgfpathlineto{\pgfqpoint{1.528194in}{3.274338in}}%
\pgfpathlineto{\pgfqpoint{1.528616in}{3.275141in}}%
\pgfpathlineto{\pgfqpoint{1.529037in}{3.274267in}}%
\pgfpathlineto{\pgfqpoint{1.529880in}{3.274242in}}%
\pgfpathlineto{\pgfqpoint{1.530301in}{3.275229in}}%
\pgfpathlineto{\pgfqpoint{1.530722in}{3.274453in}}%
\pgfpathlineto{\pgfqpoint{1.531565in}{3.274662in}}%
\pgfpathlineto{\pgfqpoint{1.531986in}{3.275008in}}%
\pgfpathlineto{\pgfqpoint{1.532407in}{3.274135in}}%
\pgfpathlineto{\pgfqpoint{1.534935in}{3.274162in}}%
\pgfpathlineto{\pgfqpoint{1.537884in}{3.277457in}}%
\pgfpathlineto{\pgfqpoint{1.538305in}{3.276863in}}%
\pgfpathlineto{\pgfqpoint{1.542518in}{3.243764in}}%
\pgfpathlineto{\pgfqpoint{1.547995in}{3.200077in}}%
\pgfpathlineto{\pgfqpoint{1.548837in}{3.200260in}}%
\pgfpathlineto{\pgfqpoint{1.557684in}{3.201139in}}%
\pgfpathlineto{\pgfqpoint{1.561054in}{3.203751in}}%
\pgfpathlineto{\pgfqpoint{1.562318in}{3.193805in}}%
\pgfpathlineto{\pgfqpoint{1.566110in}{3.167172in}}%
\pgfpathlineto{\pgfqpoint{1.567795in}{3.168251in}}%
\pgfpathlineto{\pgfqpoint{1.572007in}{3.171178in}}%
\pgfpathlineto{\pgfqpoint{1.578748in}{3.171486in}}%
\pgfpathlineto{\pgfqpoint{1.582539in}{3.190285in}}%
\pgfpathlineto{\pgfqpoint{1.592650in}{3.243574in}}%
\pgfpathlineto{\pgfqpoint{1.596863in}{3.251840in}}%
\pgfpathlineto{\pgfqpoint{1.598548in}{3.258645in}}%
\pgfpathlineto{\pgfqpoint{1.598969in}{3.258376in}}%
\pgfpathlineto{\pgfqpoint{1.599812in}{3.259087in}}%
\pgfpathlineto{\pgfqpoint{1.601497in}{3.262380in}}%
\pgfpathlineto{\pgfqpoint{1.603182in}{3.261309in}}%
\pgfpathlineto{\pgfqpoint{1.604025in}{3.260875in}}%
\pgfpathlineto{\pgfqpoint{1.606131in}{3.256336in}}%
\pgfpathlineto{\pgfqpoint{1.613293in}{3.232194in}}%
\pgfpathlineto{\pgfqpoint{1.615399in}{3.226816in}}%
\pgfpathlineto{\pgfqpoint{1.617084in}{3.227427in}}%
\pgfpathlineto{\pgfqpoint{1.620033in}{3.232829in}}%
\pgfpathlineto{\pgfqpoint{1.620876in}{3.235534in}}%
\pgfpathlineto{\pgfqpoint{1.622982in}{3.243098in}}%
\pgfpathlineto{\pgfqpoint{1.626352in}{3.244986in}}%
\pgfpathlineto{\pgfqpoint{1.627616in}{3.248123in}}%
\pgfpathlineto{\pgfqpoint{1.630565in}{3.254132in}}%
\pgfpathlineto{\pgfqpoint{1.633093in}{3.257298in}}%
\pgfpathlineto{\pgfqpoint{1.633935in}{3.259999in}}%
\pgfpathlineto{\pgfqpoint{1.634357in}{3.259321in}}%
\pgfpathlineto{\pgfqpoint{1.635620in}{3.255713in}}%
\pgfpathlineto{\pgfqpoint{1.636042in}{3.255978in}}%
\pgfpathlineto{\pgfqpoint{1.639412in}{3.261880in}}%
\pgfpathlineto{\pgfqpoint{1.642782in}{3.269829in}}%
\pgfpathlineto{\pgfqpoint{1.644889in}{3.273880in}}%
\pgfpathlineto{\pgfqpoint{1.646152in}{3.280003in}}%
\pgfpathlineto{\pgfqpoint{1.646574in}{3.279684in}}%
\pgfpathlineto{\pgfqpoint{1.647416in}{3.279166in}}%
\pgfpathlineto{\pgfqpoint{1.649101in}{3.282559in}}%
\pgfpathlineto{\pgfqpoint{1.649523in}{3.281952in}}%
\pgfpathlineto{\pgfqpoint{1.650365in}{3.281279in}}%
\pgfpathlineto{\pgfqpoint{1.650786in}{3.281683in}}%
\pgfpathlineto{\pgfqpoint{1.652472in}{3.281990in}}%
\pgfpathlineto{\pgfqpoint{1.657106in}{3.274309in}}%
\pgfpathlineto{\pgfqpoint{1.657948in}{3.273864in}}%
\pgfpathlineto{\pgfqpoint{1.660897in}{3.263772in}}%
\pgfpathlineto{\pgfqpoint{1.662582in}{3.260007in}}%
\pgfpathlineto{\pgfqpoint{1.663003in}{3.260983in}}%
\pgfpathlineto{\pgfqpoint{1.667216in}{3.268421in}}%
\pgfpathlineto{\pgfqpoint{1.668059in}{3.268733in}}%
\pgfpathlineto{\pgfqpoint{1.668901in}{3.273244in}}%
\pgfpathlineto{\pgfqpoint{1.670586in}{3.279725in}}%
\pgfpathlineto{\pgfqpoint{1.671429in}{3.280363in}}%
\pgfpathlineto{\pgfqpoint{1.673114in}{3.283578in}}%
\pgfpathlineto{\pgfqpoint{1.673535in}{3.282932in}}%
\pgfpathlineto{\pgfqpoint{1.674378in}{3.282563in}}%
\pgfpathlineto{\pgfqpoint{1.676063in}{3.286309in}}%
\pgfpathlineto{\pgfqpoint{1.680276in}{3.278508in}}%
\pgfpathlineto{\pgfqpoint{1.680697in}{3.279140in}}%
\pgfpathlineto{\pgfqpoint{1.681540in}{3.280831in}}%
\pgfpathlineto{\pgfqpoint{1.681961in}{3.279880in}}%
\pgfpathlineto{\pgfqpoint{1.684489in}{3.269775in}}%
\pgfpathlineto{\pgfqpoint{1.686174in}{3.265426in}}%
\pgfpathlineto{\pgfqpoint{1.687016in}{3.266780in}}%
\pgfpathlineto{\pgfqpoint{1.688280in}{3.267664in}}%
\pgfpathlineto{\pgfqpoint{1.689123in}{3.266754in}}%
\pgfpathlineto{\pgfqpoint{1.692914in}{3.277255in}}%
\pgfpathlineto{\pgfqpoint{1.693757in}{3.281376in}}%
\pgfpathlineto{\pgfqpoint{1.694178in}{3.281153in}}%
\pgfpathlineto{\pgfqpoint{1.695021in}{3.279853in}}%
\pgfpathlineto{\pgfqpoint{1.695442in}{3.280450in}}%
\pgfpathlineto{\pgfqpoint{1.696706in}{3.283457in}}%
\pgfpathlineto{\pgfqpoint{1.697127in}{3.283079in}}%
\pgfpathlineto{\pgfqpoint{1.697970in}{3.282034in}}%
\pgfpathlineto{\pgfqpoint{1.698391in}{3.282608in}}%
\pgfpathlineto{\pgfqpoint{1.699655in}{3.285186in}}%
\pgfpathlineto{\pgfqpoint{1.700076in}{3.284851in}}%
\pgfpathlineto{\pgfqpoint{1.704710in}{3.276472in}}%
\pgfpathlineto{\pgfqpoint{1.705553in}{3.279994in}}%
\pgfpathlineto{\pgfqpoint{1.705974in}{3.279641in}}%
\pgfpathlineto{\pgfqpoint{1.707659in}{3.276569in}}%
\pgfpathlineto{\pgfqpoint{1.708502in}{3.279580in}}%
\pgfpathlineto{\pgfqpoint{1.711029in}{3.285583in}}%
\pgfpathlineto{\pgfqpoint{1.715242in}{3.296419in}}%
\pgfpathlineto{\pgfqpoint{1.721982in}{3.303775in}}%
\pgfpathlineto{\pgfqpoint{1.723668in}{3.302471in}}%
\pgfpathlineto{\pgfqpoint{1.724510in}{3.302345in}}%
\pgfpathlineto{\pgfqpoint{1.725353in}{3.302863in}}%
\pgfpathlineto{\pgfqpoint{1.725774in}{3.302459in}}%
\pgfpathlineto{\pgfqpoint{1.730829in}{3.302846in}}%
\pgfpathlineto{\pgfqpoint{1.731672in}{3.304768in}}%
\pgfpathlineto{\pgfqpoint{1.732093in}{3.303931in}}%
\pgfpathlineto{\pgfqpoint{1.734199in}{3.303427in}}%
\pgfpathlineto{\pgfqpoint{1.734621in}{3.304792in}}%
\pgfpathlineto{\pgfqpoint{1.735042in}{3.304462in}}%
\pgfpathlineto{\pgfqpoint{1.736727in}{3.302794in}}%
\pgfpathlineto{\pgfqpoint{1.745153in}{3.301859in}}%
\pgfpathlineto{\pgfqpoint{1.746838in}{3.300235in}}%
\pgfpathlineto{\pgfqpoint{1.748102in}{3.301787in}}%
\pgfpathlineto{\pgfqpoint{1.748944in}{3.300577in}}%
\pgfpathlineto{\pgfqpoint{1.755263in}{3.300043in}}%
\pgfpathlineto{\pgfqpoint{1.756949in}{3.299968in}}%
\pgfpathlineto{\pgfqpoint{1.757791in}{3.301056in}}%
\pgfpathlineto{\pgfqpoint{1.759897in}{3.298742in}}%
\pgfpathlineto{\pgfqpoint{1.762425in}{3.299648in}}%
\pgfpathlineto{\pgfqpoint{1.763268in}{3.299218in}}%
\pgfpathlineto{\pgfqpoint{1.763689in}{3.299852in}}%
\pgfpathlineto{\pgfqpoint{1.765374in}{3.302760in}}%
\pgfpathlineto{\pgfqpoint{1.767059in}{3.303679in}}%
\pgfpathlineto{\pgfqpoint{1.768744in}{3.306105in}}%
\pgfpathlineto{\pgfqpoint{1.770008in}{3.308554in}}%
\pgfpathlineto{\pgfqpoint{1.772115in}{3.317296in}}%
\pgfpathlineto{\pgfqpoint{1.774221in}{3.316568in}}%
\pgfpathlineto{\pgfqpoint{1.775906in}{3.314500in}}%
\pgfpathlineto{\pgfqpoint{1.777170in}{3.314980in}}%
\pgfpathlineto{\pgfqpoint{1.778434in}{3.311531in}}%
\pgfpathlineto{\pgfqpoint{1.778855in}{3.312059in}}%
\pgfpathlineto{\pgfqpoint{1.779276in}{3.312908in}}%
\pgfpathlineto{\pgfqpoint{1.779698in}{3.312339in}}%
\pgfpathlineto{\pgfqpoint{1.781383in}{3.309664in}}%
\pgfpathlineto{\pgfqpoint{1.782646in}{3.309710in}}%
\pgfpathlineto{\pgfqpoint{1.783489in}{3.307936in}}%
\pgfpathlineto{\pgfqpoint{1.784753in}{3.302908in}}%
\pgfpathlineto{\pgfqpoint{1.786438in}{3.303875in}}%
\pgfpathlineto{\pgfqpoint{1.786859in}{3.302706in}}%
\pgfpathlineto{\pgfqpoint{1.790229in}{3.284440in}}%
\pgfpathlineto{\pgfqpoint{1.791493in}{3.280565in}}%
\pgfpathlineto{\pgfqpoint{1.793178in}{3.281000in}}%
\pgfpathlineto{\pgfqpoint{1.793600in}{3.281673in}}%
\pgfpathlineto{\pgfqpoint{1.794021in}{3.280453in}}%
\pgfpathlineto{\pgfqpoint{1.799076in}{3.258162in}}%
\pgfpathlineto{\pgfqpoint{1.799498in}{3.257669in}}%
\pgfpathlineto{\pgfqpoint{1.799919in}{3.258754in}}%
\pgfpathlineto{\pgfqpoint{1.802447in}{3.260763in}}%
\pgfpathlineto{\pgfqpoint{1.804553in}{3.259672in}}%
\pgfpathlineto{\pgfqpoint{1.812557in}{3.260663in}}%
\pgfpathlineto{\pgfqpoint{1.812979in}{3.261036in}}%
\pgfpathlineto{\pgfqpoint{1.813400in}{3.260359in}}%
\pgfpathlineto{\pgfqpoint{1.817191in}{3.259272in}}%
\pgfpathlineto{\pgfqpoint{1.819298in}{3.260449in}}%
\pgfpathlineto{\pgfqpoint{1.819719in}{3.259375in}}%
\pgfpathlineto{\pgfqpoint{1.820140in}{3.258512in}}%
\pgfpathlineto{\pgfqpoint{1.820562in}{3.258831in}}%
\pgfpathlineto{\pgfqpoint{1.824353in}{3.269825in}}%
\pgfpathlineto{\pgfqpoint{1.826038in}{3.273280in}}%
\pgfpathlineto{\pgfqpoint{1.827302in}{3.275000in}}%
\pgfpathlineto{\pgfqpoint{1.829408in}{3.288452in}}%
\pgfpathlineto{\pgfqpoint{1.832357in}{3.307639in}}%
\pgfpathlineto{\pgfqpoint{1.833200in}{3.309464in}}%
\pgfpathlineto{\pgfqpoint{1.833621in}{3.308906in}}%
\pgfpathlineto{\pgfqpoint{1.834464in}{3.308296in}}%
\pgfpathlineto{\pgfqpoint{1.834885in}{3.308729in}}%
\pgfpathlineto{\pgfqpoint{1.835728in}{3.309259in}}%
\pgfpathlineto{\pgfqpoint{1.838255in}{3.314008in}}%
\pgfpathlineto{\pgfqpoint{1.839940in}{3.314311in}}%
\pgfpathlineto{\pgfqpoint{1.844153in}{3.299456in}}%
\pgfpathlineto{\pgfqpoint{1.844996in}{3.297186in}}%
\pgfpathlineto{\pgfqpoint{1.847523in}{3.290378in}}%
\pgfpathlineto{\pgfqpoint{1.847945in}{3.290894in}}%
\pgfpathlineto{\pgfqpoint{1.850051in}{3.289316in}}%
\pgfpathlineto{\pgfqpoint{1.851736in}{3.288762in}}%
\pgfpathlineto{\pgfqpoint{1.855949in}{3.282982in}}%
\pgfpathlineto{\pgfqpoint{1.859740in}{3.283846in}}%
\pgfpathlineto{\pgfqpoint{1.860583in}{3.283162in}}%
\pgfpathlineto{\pgfqpoint{1.861004in}{3.283727in}}%
\pgfpathlineto{\pgfqpoint{1.862268in}{3.283763in}}%
\pgfpathlineto{\pgfqpoint{1.865638in}{3.283334in}}%
\pgfpathlineto{\pgfqpoint{1.866902in}{3.283117in}}%
\pgfpathlineto{\pgfqpoint{1.867323in}{3.282920in}}%
\pgfpathlineto{\pgfqpoint{1.868587in}{3.283830in}}%
\pgfpathlineto{\pgfqpoint{1.870694in}{3.283257in}}%
\pgfpathlineto{\pgfqpoint{1.871115in}{3.284156in}}%
\pgfpathlineto{\pgfqpoint{1.871536in}{3.283755in}}%
\pgfpathlineto{\pgfqpoint{1.874064in}{3.278065in}}%
\pgfpathlineto{\pgfqpoint{1.874906in}{3.279092in}}%
\pgfpathlineto{\pgfqpoint{1.877434in}{3.278761in}}%
\pgfpathlineto{\pgfqpoint{1.879540in}{3.283211in}}%
\pgfpathlineto{\pgfqpoint{1.879962in}{3.283009in}}%
\pgfpathlineto{\pgfqpoint{1.881226in}{3.287407in}}%
\pgfpathlineto{\pgfqpoint{1.882068in}{3.286462in}}%
\pgfpathlineto{\pgfqpoint{1.883753in}{3.287683in}}%
\pgfpathlineto{\pgfqpoint{1.887966in}{3.298909in}}%
\pgfpathlineto{\pgfqpoint{1.888387in}{3.298429in}}%
\pgfpathlineto{\pgfqpoint{1.889651in}{3.298720in}}%
\pgfpathlineto{\pgfqpoint{1.890494in}{3.300108in}}%
\pgfpathlineto{\pgfqpoint{1.891336in}{3.302138in}}%
\pgfpathlineto{\pgfqpoint{1.891758in}{3.301277in}}%
\pgfpathlineto{\pgfqpoint{1.893443in}{3.299521in}}%
\pgfpathlineto{\pgfqpoint{1.897655in}{3.300102in}}%
\pgfpathlineto{\pgfqpoint{1.899762in}{3.302587in}}%
\pgfpathlineto{\pgfqpoint{1.900183in}{3.301632in}}%
\pgfpathlineto{\pgfqpoint{1.904817in}{3.294002in}}%
\pgfpathlineto{\pgfqpoint{1.908187in}{3.292974in}}%
\pgfpathlineto{\pgfqpoint{1.912400in}{3.292492in}}%
\pgfpathlineto{\pgfqpoint{1.913243in}{3.290460in}}%
\pgfpathlineto{\pgfqpoint{1.913664in}{3.291228in}}%
\pgfpathlineto{\pgfqpoint{1.916192in}{3.295464in}}%
\pgfpathlineto{\pgfqpoint{1.917034in}{3.296513in}}%
\pgfpathlineto{\pgfqpoint{1.918719in}{3.298133in}}%
\pgfpathlineto{\pgfqpoint{1.923353in}{3.298613in}}%
\pgfpathlineto{\pgfqpoint{1.927566in}{3.297824in}}%
\pgfpathlineto{\pgfqpoint{1.934307in}{3.291093in}}%
\pgfpathlineto{\pgfqpoint{1.937256in}{3.265995in}}%
\pgfpathlineto{\pgfqpoint{1.940205in}{3.244189in}}%
\pgfpathlineto{\pgfqpoint{1.943996in}{3.241755in}}%
\pgfpathlineto{\pgfqpoint{1.945681in}{3.243344in}}%
\pgfpathlineto{\pgfqpoint{1.947788in}{3.242727in}}%
\pgfpathlineto{\pgfqpoint{1.950315in}{3.241863in}}%
\pgfpathlineto{\pgfqpoint{1.954107in}{3.243875in}}%
\pgfpathlineto{\pgfqpoint{1.956634in}{3.243572in}}%
\pgfpathlineto{\pgfqpoint{1.957477in}{3.244152in}}%
\pgfpathlineto{\pgfqpoint{1.957898in}{3.244591in}}%
\pgfpathlineto{\pgfqpoint{1.958319in}{3.244141in}}%
\pgfpathlineto{\pgfqpoint{1.964217in}{3.241055in}}%
\pgfpathlineto{\pgfqpoint{1.965902in}{3.239990in}}%
\pgfpathlineto{\pgfqpoint{1.974749in}{3.161304in}}%
\pgfpathlineto{\pgfqpoint{1.976434in}{3.158710in}}%
\pgfpathlineto{\pgfqpoint{1.986966in}{3.158604in}}%
\pgfpathlineto{\pgfqpoint{1.997920in}{3.237346in}}%
\pgfpathlineto{\pgfqpoint{2.000447in}{3.259530in}}%
\pgfpathlineto{\pgfqpoint{2.003396in}{3.265727in}}%
\pgfpathlineto{\pgfqpoint{2.005503in}{3.269921in}}%
\pgfpathlineto{\pgfqpoint{2.005924in}{3.269254in}}%
\pgfpathlineto{\pgfqpoint{2.007609in}{3.268801in}}%
\pgfpathlineto{\pgfqpoint{2.009294in}{3.268053in}}%
\pgfpathlineto{\pgfqpoint{2.010979in}{3.260255in}}%
\pgfpathlineto{\pgfqpoint{2.015192in}{3.234605in}}%
\pgfpathlineto{\pgfqpoint{2.018984in}{3.226313in}}%
\pgfpathlineto{\pgfqpoint{2.019405in}{3.226939in}}%
\pgfpathlineto{\pgfqpoint{2.021090in}{3.227635in}}%
\pgfpathlineto{\pgfqpoint{2.022775in}{3.228389in}}%
\pgfpathlineto{\pgfqpoint{2.024460in}{3.236147in}}%
\pgfpathlineto{\pgfqpoint{2.028252in}{3.258303in}}%
\pgfpathlineto{\pgfqpoint{2.031622in}{3.266022in}}%
\pgfpathlineto{\pgfqpoint{2.032464in}{3.268025in}}%
\pgfpathlineto{\pgfqpoint{2.032886in}{3.267396in}}%
\pgfpathlineto{\pgfqpoint{2.034571in}{3.266664in}}%
\pgfpathlineto{\pgfqpoint{2.045103in}{3.264923in}}%
\pgfpathlineto{\pgfqpoint{2.046367in}{3.258612in}}%
\pgfpathlineto{\pgfqpoint{2.059005in}{3.169731in}}%
\pgfpathlineto{\pgfqpoint{2.059426in}{3.170378in}}%
\pgfpathlineto{\pgfqpoint{2.061533in}{3.179893in}}%
\pgfpathlineto{\pgfqpoint{2.075435in}{3.248363in}}%
\pgfpathlineto{\pgfqpoint{2.081754in}{3.263151in}}%
\pgfpathlineto{\pgfqpoint{2.083018in}{3.263151in}}%
\pgfpathlineto{\pgfqpoint{2.085967in}{3.254710in}}%
\pgfpathlineto{\pgfqpoint{2.090180in}{3.242708in}}%
\pgfpathlineto{\pgfqpoint{2.095656in}{3.229490in}}%
\pgfpathlineto{\pgfqpoint{2.096077in}{3.229702in}}%
\pgfpathlineto{\pgfqpoint{2.097341in}{3.231066in}}%
\pgfpathlineto{\pgfqpoint{2.101975in}{3.244638in}}%
\pgfpathlineto{\pgfqpoint{2.110822in}{3.264114in}}%
\pgfpathlineto{\pgfqpoint{2.117141in}{3.263807in}}%
\pgfpathlineto{\pgfqpoint{2.120933in}{3.258581in}}%
\pgfpathlineto{\pgfqpoint{2.121354in}{3.259311in}}%
\pgfpathlineto{\pgfqpoint{2.122197in}{3.259677in}}%
\pgfpathlineto{\pgfqpoint{2.122618in}{3.258962in}}%
\pgfpathlineto{\pgfqpoint{2.124303in}{3.258462in}}%
\pgfpathlineto{\pgfqpoint{2.133150in}{3.259562in}}%
\pgfpathlineto{\pgfqpoint{2.134835in}{3.259263in}}%
\pgfpathlineto{\pgfqpoint{2.136099in}{3.256876in}}%
\pgfpathlineto{\pgfqpoint{2.136520in}{3.257451in}}%
\pgfpathlineto{\pgfqpoint{2.141575in}{3.263740in}}%
\pgfpathlineto{\pgfqpoint{2.143682in}{3.269996in}}%
\pgfpathlineto{\pgfqpoint{2.144524in}{3.270078in}}%
\pgfpathlineto{\pgfqpoint{2.146631in}{3.258307in}}%
\pgfpathlineto{\pgfqpoint{2.147052in}{3.255416in}}%
\pgfpathlineto{\pgfqpoint{2.147473in}{3.256502in}}%
\pgfpathlineto{\pgfqpoint{2.149580in}{3.264550in}}%
\pgfpathlineto{\pgfqpoint{2.150001in}{3.264014in}}%
\pgfpathlineto{\pgfqpoint{2.151265in}{3.261146in}}%
\pgfpathlineto{\pgfqpoint{2.153793in}{3.245831in}}%
\pgfpathlineto{\pgfqpoint{2.154214in}{3.247081in}}%
\pgfpathlineto{\pgfqpoint{2.154635in}{3.247749in}}%
\pgfpathlineto{\pgfqpoint{2.156741in}{3.243721in}}%
\pgfpathlineto{\pgfqpoint{2.159269in}{3.245550in}}%
\pgfpathlineto{\pgfqpoint{2.160533in}{3.248509in}}%
\pgfpathlineto{\pgfqpoint{2.165167in}{3.222731in}}%
\pgfpathlineto{\pgfqpoint{2.165588in}{3.223561in}}%
\pgfpathlineto{\pgfqpoint{2.166852in}{3.225557in}}%
\pgfpathlineto{\pgfqpoint{2.174856in}{3.180385in}}%
\pgfpathlineto{\pgfqpoint{2.175699in}{3.178507in}}%
\pgfpathlineto{\pgfqpoint{2.176120in}{3.179674in}}%
\pgfpathlineto{\pgfqpoint{2.180754in}{3.204521in}}%
\pgfpathlineto{\pgfqpoint{2.182018in}{3.213188in}}%
\pgfpathlineto{\pgfqpoint{2.192129in}{3.292977in}}%
\pgfpathlineto{\pgfqpoint{2.195078in}{3.301943in}}%
\pgfpathlineto{\pgfqpoint{2.195499in}{3.300995in}}%
\pgfpathlineto{\pgfqpoint{2.197605in}{3.300682in}}%
\pgfpathlineto{\pgfqpoint{2.209401in}{3.300341in}}%
\pgfpathlineto{\pgfqpoint{2.211086in}{3.291678in}}%
\pgfpathlineto{\pgfqpoint{2.224146in}{3.190169in}}%
\pgfpathlineto{\pgfqpoint{2.224989in}{3.191041in}}%
\pgfpathlineto{\pgfqpoint{2.231308in}{3.215136in}}%
\pgfpathlineto{\pgfqpoint{2.239733in}{3.242819in}}%
\pgfpathlineto{\pgfqpoint{2.242261in}{3.246101in}}%
\pgfpathlineto{\pgfqpoint{2.243946in}{3.241844in}}%
\pgfpathlineto{\pgfqpoint{2.254478in}{3.207923in}}%
\pgfpathlineto{\pgfqpoint{2.255742in}{3.207413in}}%
\pgfpathlineto{\pgfqpoint{2.257848in}{3.213174in}}%
\pgfpathlineto{\pgfqpoint{2.270065in}{3.255434in}}%
\pgfpathlineto{\pgfqpoint{2.270487in}{3.255284in}}%
\pgfpathlineto{\pgfqpoint{2.272172in}{3.253157in}}%
\pgfpathlineto{\pgfqpoint{2.273436in}{3.251742in}}%
\pgfpathlineto{\pgfqpoint{2.274699in}{3.250617in}}%
\pgfpathlineto{\pgfqpoint{2.279333in}{3.246920in}}%
\pgfpathlineto{\pgfqpoint{2.281440in}{3.245559in}}%
\pgfpathlineto{\pgfqpoint{2.285231in}{3.244039in}}%
\pgfpathlineto{\pgfqpoint{2.286074in}{3.245522in}}%
\pgfpathlineto{\pgfqpoint{2.297870in}{3.271396in}}%
\pgfpathlineto{\pgfqpoint{2.299133in}{3.270993in}}%
\pgfpathlineto{\pgfqpoint{2.301240in}{3.274748in}}%
\pgfpathlineto{\pgfqpoint{2.302082in}{3.274192in}}%
\pgfpathlineto{\pgfqpoint{2.307559in}{3.256923in}}%
\pgfpathlineto{\pgfqpoint{2.308402in}{3.257597in}}%
\pgfpathlineto{\pgfqpoint{2.337049in}{3.258254in}}%
\pgfpathlineto{\pgfqpoint{2.339155in}{3.257746in}}%
\pgfpathlineto{\pgfqpoint{2.339576in}{3.259396in}}%
\pgfpathlineto{\pgfqpoint{2.340419in}{3.258777in}}%
\pgfpathlineto{\pgfqpoint{2.342525in}{3.257288in}}%
\pgfpathlineto{\pgfqpoint{2.346317in}{3.270134in}}%
\pgfpathlineto{\pgfqpoint{2.347159in}{3.269036in}}%
\pgfpathlineto{\pgfqpoint{2.349266in}{3.268043in}}%
\pgfpathlineto{\pgfqpoint{2.354321in}{3.280068in}}%
\pgfpathlineto{\pgfqpoint{2.356849in}{3.279381in}}%
\pgfpathlineto{\pgfqpoint{2.361061in}{3.279904in}}%
\pgfpathlineto{\pgfqpoint{2.361904in}{3.278734in}}%
\pgfpathlineto{\pgfqpoint{2.362325in}{3.279309in}}%
\pgfpathlineto{\pgfqpoint{2.365274in}{3.277037in}}%
\pgfpathlineto{\pgfqpoint{2.365695in}{3.278656in}}%
\pgfpathlineto{\pgfqpoint{2.367381in}{3.279420in}}%
\pgfpathlineto{\pgfqpoint{2.368644in}{3.278948in}}%
\pgfpathlineto{\pgfqpoint{2.369066in}{3.279668in}}%
\pgfpathlineto{\pgfqpoint{2.369908in}{3.278994in}}%
\pgfpathlineto{\pgfqpoint{2.372015in}{3.279811in}}%
\pgfpathlineto{\pgfqpoint{2.378334in}{3.282111in}}%
\pgfpathlineto{\pgfqpoint{2.378755in}{3.283067in}}%
\pgfpathlineto{\pgfqpoint{2.379176in}{3.281530in}}%
\pgfpathlineto{\pgfqpoint{2.380861in}{3.280161in}}%
\pgfpathlineto{\pgfqpoint{2.382968in}{3.279943in}}%
\pgfpathlineto{\pgfqpoint{2.383810in}{3.280550in}}%
\pgfpathlineto{\pgfqpoint{2.384232in}{3.280191in}}%
\pgfpathlineto{\pgfqpoint{2.388866in}{3.280390in}}%
\pgfpathlineto{\pgfqpoint{2.390130in}{3.281205in}}%
\pgfpathlineto{\pgfqpoint{2.390551in}{3.280195in}}%
\pgfpathlineto{\pgfqpoint{2.390972in}{3.279496in}}%
\pgfpathlineto{\pgfqpoint{2.391393in}{3.280398in}}%
\pgfpathlineto{\pgfqpoint{2.393078in}{3.281687in}}%
\pgfpathlineto{\pgfqpoint{2.397713in}{3.279283in}}%
\pgfpathlineto{\pgfqpoint{2.398134in}{3.279913in}}%
\pgfpathlineto{\pgfqpoint{2.398555in}{3.279080in}}%
\pgfpathlineto{\pgfqpoint{2.400662in}{3.278365in}}%
\pgfpathlineto{\pgfqpoint{2.401504in}{3.279282in}}%
\pgfpathlineto{\pgfqpoint{2.403189in}{3.276803in}}%
\pgfpathlineto{\pgfqpoint{2.404032in}{3.276303in}}%
\pgfpathlineto{\pgfqpoint{2.404453in}{3.277547in}}%
\pgfpathlineto{\pgfqpoint{2.404874in}{3.276613in}}%
\pgfpathlineto{\pgfqpoint{2.406981in}{3.274391in}}%
\pgfpathlineto{\pgfqpoint{2.409508in}{3.275267in}}%
\pgfpathlineto{\pgfqpoint{2.410351in}{3.274201in}}%
\pgfpathlineto{\pgfqpoint{2.410772in}{3.274841in}}%
\pgfpathlineto{\pgfqpoint{2.411615in}{3.274104in}}%
\pgfpathlineto{\pgfqpoint{2.417091in}{3.273629in}}%
\pgfpathlineto{\pgfqpoint{2.419619in}{3.277710in}}%
\pgfpathlineto{\pgfqpoint{2.420883in}{3.275296in}}%
\pgfpathlineto{\pgfqpoint{2.425517in}{3.238003in}}%
\pgfpathlineto{\pgfqpoint{2.430572in}{3.198943in}}%
\pgfpathlineto{\pgfqpoint{2.441525in}{3.199752in}}%
\pgfpathlineto{\pgfqpoint{2.443632in}{3.199280in}}%
\pgfpathlineto{\pgfqpoint{2.449108in}{3.157857in}}%
\pgfpathlineto{\pgfqpoint{2.461326in}{3.159157in}}%
\pgfpathlineto{\pgfqpoint{2.473543in}{3.235290in}}%
\pgfpathlineto{\pgfqpoint{2.475649in}{3.244356in}}%
\pgfpathlineto{\pgfqpoint{2.479441in}{3.253140in}}%
\pgfpathlineto{\pgfqpoint{2.480704in}{3.258306in}}%
\pgfpathlineto{\pgfqpoint{2.481126in}{3.258034in}}%
\pgfpathlineto{\pgfqpoint{2.481968in}{3.258189in}}%
\pgfpathlineto{\pgfqpoint{2.483653in}{3.262232in}}%
\pgfpathlineto{\pgfqpoint{2.484075in}{3.261610in}}%
\pgfpathlineto{\pgfqpoint{2.485760in}{3.261029in}}%
\pgfpathlineto{\pgfqpoint{2.486602in}{3.260439in}}%
\pgfpathlineto{\pgfqpoint{2.490394in}{3.249289in}}%
\pgfpathlineto{\pgfqpoint{2.498398in}{3.225769in}}%
\pgfpathlineto{\pgfqpoint{2.502611in}{3.232338in}}%
\pgfpathlineto{\pgfqpoint{2.503453in}{3.236392in}}%
\pgfpathlineto{\pgfqpoint{2.505138in}{3.242204in}}%
\pgfpathlineto{\pgfqpoint{2.506824in}{3.243542in}}%
\pgfpathlineto{\pgfqpoint{2.508509in}{3.244549in}}%
\pgfpathlineto{\pgfqpoint{2.509351in}{3.245880in}}%
\pgfpathlineto{\pgfqpoint{2.512721in}{3.253130in}}%
\pgfpathlineto{\pgfqpoint{2.516513in}{3.259204in}}%
\pgfpathlineto{\pgfqpoint{2.516934in}{3.257590in}}%
\pgfpathlineto{\pgfqpoint{2.517777in}{3.255010in}}%
\pgfpathlineto{\pgfqpoint{2.518198in}{3.255132in}}%
\pgfpathlineto{\pgfqpoint{2.521147in}{3.258624in}}%
\pgfpathlineto{\pgfqpoint{2.523675in}{3.265442in}}%
\pgfpathlineto{\pgfqpoint{2.527466in}{3.275316in}}%
\pgfpathlineto{\pgfqpoint{2.528730in}{3.279403in}}%
\pgfpathlineto{\pgfqpoint{2.529573in}{3.278558in}}%
\pgfpathlineto{\pgfqpoint{2.529994in}{3.279165in}}%
\pgfpathlineto{\pgfqpoint{2.531258in}{3.282115in}}%
\pgfpathlineto{\pgfqpoint{2.531679in}{3.281726in}}%
\pgfpathlineto{\pgfqpoint{2.532943in}{3.280694in}}%
\pgfpathlineto{\pgfqpoint{2.533364in}{3.280975in}}%
\pgfpathlineto{\pgfqpoint{2.534628in}{3.281282in}}%
\pgfpathlineto{\pgfqpoint{2.539683in}{3.273595in}}%
\pgfpathlineto{\pgfqpoint{2.540105in}{3.273353in}}%
\pgfpathlineto{\pgfqpoint{2.544739in}{3.259094in}}%
\pgfpathlineto{\pgfqpoint{2.551058in}{3.271721in}}%
\pgfpathlineto{\pgfqpoint{2.552743in}{3.279162in}}%
\pgfpathlineto{\pgfqpoint{2.554007in}{3.280424in}}%
\pgfpathlineto{\pgfqpoint{2.555271in}{3.283218in}}%
\pgfpathlineto{\pgfqpoint{2.555692in}{3.282633in}}%
\pgfpathlineto{\pgfqpoint{2.556534in}{3.281880in}}%
\pgfpathlineto{\pgfqpoint{2.558220in}{3.285949in}}%
\pgfpathlineto{\pgfqpoint{2.558641in}{3.285409in}}%
\pgfpathlineto{\pgfqpoint{2.562854in}{3.278258in}}%
\pgfpathlineto{\pgfqpoint{2.563696in}{3.280290in}}%
\pgfpathlineto{\pgfqpoint{2.564117in}{3.280000in}}%
\pgfpathlineto{\pgfqpoint{2.567066in}{3.269086in}}%
\pgfpathlineto{\pgfqpoint{2.568751in}{3.265169in}}%
\pgfpathlineto{\pgfqpoint{2.569173in}{3.266098in}}%
\pgfpathlineto{\pgfqpoint{2.570437in}{3.267453in}}%
\pgfpathlineto{\pgfqpoint{2.571279in}{3.266210in}}%
\pgfpathlineto{\pgfqpoint{2.571700in}{3.266807in}}%
\pgfpathlineto{\pgfqpoint{2.574228in}{3.271018in}}%
\pgfpathlineto{\pgfqpoint{2.575071in}{3.275183in}}%
\pgfpathlineto{\pgfqpoint{2.576334in}{3.280537in}}%
\pgfpathlineto{\pgfqpoint{2.576756in}{3.279755in}}%
\pgfpathlineto{\pgfqpoint{2.577177in}{3.278993in}}%
\pgfpathlineto{\pgfqpoint{2.577598in}{3.279321in}}%
\pgfpathlineto{\pgfqpoint{2.579283in}{3.282509in}}%
\pgfpathlineto{\pgfqpoint{2.579705in}{3.281850in}}%
\pgfpathlineto{\pgfqpoint{2.580547in}{3.281312in}}%
\pgfpathlineto{\pgfqpoint{2.582232in}{3.284246in}}%
\pgfpathlineto{\pgfqpoint{2.582654in}{3.283591in}}%
\pgfpathlineto{\pgfqpoint{2.586866in}{3.276283in}}%
\pgfpathlineto{\pgfqpoint{2.587288in}{3.276724in}}%
\pgfpathlineto{\pgfqpoint{2.588130in}{3.279798in}}%
\pgfpathlineto{\pgfqpoint{2.588552in}{3.278084in}}%
\pgfpathlineto{\pgfqpoint{2.590237in}{3.276215in}}%
\pgfpathlineto{\pgfqpoint{2.592764in}{3.284197in}}%
\pgfpathlineto{\pgfqpoint{2.598662in}{3.297913in}}%
\pgfpathlineto{\pgfqpoint{2.600347in}{3.300805in}}%
\pgfpathlineto{\pgfqpoint{2.601190in}{3.302715in}}%
\pgfpathlineto{\pgfqpoint{2.601611in}{3.302163in}}%
\pgfpathlineto{\pgfqpoint{2.602454in}{3.302876in}}%
\pgfpathlineto{\pgfqpoint{2.602875in}{3.302570in}}%
\pgfpathlineto{\pgfqpoint{2.603296in}{3.302598in}}%
\pgfpathlineto{\pgfqpoint{2.604139in}{3.303850in}}%
\pgfpathlineto{\pgfqpoint{2.605824in}{3.301793in}}%
\pgfpathlineto{\pgfqpoint{2.607088in}{3.302368in}}%
\pgfpathlineto{\pgfqpoint{2.608773in}{3.302265in}}%
\pgfpathlineto{\pgfqpoint{2.612564in}{3.301885in}}%
\pgfpathlineto{\pgfqpoint{2.613407in}{3.302767in}}%
\pgfpathlineto{\pgfqpoint{2.613828in}{3.304171in}}%
\pgfpathlineto{\pgfqpoint{2.614250in}{3.304021in}}%
\pgfpathlineto{\pgfqpoint{2.615935in}{3.302453in}}%
\pgfpathlineto{\pgfqpoint{2.616356in}{3.302392in}}%
\pgfpathlineto{\pgfqpoint{2.617198in}{3.304057in}}%
\pgfpathlineto{\pgfqpoint{2.617620in}{3.303514in}}%
\pgfpathlineto{\pgfqpoint{2.619305in}{3.302222in}}%
\pgfpathlineto{\pgfqpoint{2.626467in}{3.301444in}}%
\pgfpathlineto{\pgfqpoint{2.626888in}{3.302567in}}%
\pgfpathlineto{\pgfqpoint{2.627309in}{3.302117in}}%
\pgfpathlineto{\pgfqpoint{2.628994in}{3.299782in}}%
\pgfpathlineto{\pgfqpoint{2.629837in}{3.300766in}}%
\pgfpathlineto{\pgfqpoint{2.630258in}{3.301498in}}%
\pgfpathlineto{\pgfqpoint{2.630679in}{3.301013in}}%
\pgfpathlineto{\pgfqpoint{2.632364in}{3.299676in}}%
\pgfpathlineto{\pgfqpoint{2.639526in}{3.299400in}}%
\pgfpathlineto{\pgfqpoint{2.639947in}{3.300410in}}%
\pgfpathlineto{\pgfqpoint{2.640369in}{3.299962in}}%
\pgfpathlineto{\pgfqpoint{2.642054in}{3.297800in}}%
\pgfpathlineto{\pgfqpoint{2.642896in}{3.298715in}}%
\pgfpathlineto{\pgfqpoint{2.643318in}{3.299619in}}%
\pgfpathlineto{\pgfqpoint{2.643739in}{3.299127in}}%
\pgfpathlineto{\pgfqpoint{2.645424in}{3.297925in}}%
\pgfpathlineto{\pgfqpoint{2.645845in}{3.297912in}}%
\pgfpathlineto{\pgfqpoint{2.647952in}{3.301919in}}%
\pgfpathlineto{\pgfqpoint{2.649637in}{3.303382in}}%
\pgfpathlineto{\pgfqpoint{2.652165in}{3.306648in}}%
\pgfpathlineto{\pgfqpoint{2.652586in}{3.307158in}}%
\pgfpathlineto{\pgfqpoint{2.655113in}{3.315279in}}%
\pgfpathlineto{\pgfqpoint{2.655956in}{3.316096in}}%
\pgfpathlineto{\pgfqpoint{2.656377in}{3.315636in}}%
\pgfpathlineto{\pgfqpoint{2.658062in}{3.313114in}}%
\pgfpathlineto{\pgfqpoint{2.659748in}{3.312978in}}%
\pgfpathlineto{\pgfqpoint{2.661433in}{3.310131in}}%
\pgfpathlineto{\pgfqpoint{2.662275in}{3.311425in}}%
\pgfpathlineto{\pgfqpoint{2.662696in}{3.310434in}}%
\pgfpathlineto{\pgfqpoint{2.663539in}{3.308216in}}%
\pgfpathlineto{\pgfqpoint{2.663960in}{3.308618in}}%
\pgfpathlineto{\pgfqpoint{2.665224in}{3.308800in}}%
\pgfpathlineto{\pgfqpoint{2.666067in}{3.308259in}}%
\pgfpathlineto{\pgfqpoint{2.667331in}{3.302742in}}%
\pgfpathlineto{\pgfqpoint{2.667752in}{3.303045in}}%
\pgfpathlineto{\pgfqpoint{2.668594in}{3.303717in}}%
\pgfpathlineto{\pgfqpoint{2.669016in}{3.304116in}}%
\pgfpathlineto{\pgfqpoint{2.669437in}{3.303296in}}%
\pgfpathlineto{\pgfqpoint{2.672807in}{3.285849in}}%
\pgfpathlineto{\pgfqpoint{2.674492in}{3.280347in}}%
\pgfpathlineto{\pgfqpoint{2.675335in}{3.279296in}}%
\pgfpathlineto{\pgfqpoint{2.675756in}{3.280516in}}%
\pgfpathlineto{\pgfqpoint{2.676599in}{3.281390in}}%
\pgfpathlineto{\pgfqpoint{2.680811in}{3.259602in}}%
\pgfpathlineto{\pgfqpoint{2.682075in}{3.256717in}}%
\pgfpathlineto{\pgfqpoint{2.682497in}{3.257397in}}%
\pgfpathlineto{\pgfqpoint{2.684182in}{3.258994in}}%
\pgfpathlineto{\pgfqpoint{2.687552in}{3.260049in}}%
\pgfpathlineto{\pgfqpoint{2.693871in}{3.260875in}}%
\pgfpathlineto{\pgfqpoint{2.695556in}{3.261770in}}%
\pgfpathlineto{\pgfqpoint{2.700190in}{3.259215in}}%
\pgfpathlineto{\pgfqpoint{2.700612in}{3.260100in}}%
\pgfpathlineto{\pgfqpoint{2.702297in}{3.259601in}}%
\pgfpathlineto{\pgfqpoint{2.703139in}{3.258329in}}%
\pgfpathlineto{\pgfqpoint{2.703560in}{3.258744in}}%
\pgfpathlineto{\pgfqpoint{2.708195in}{3.273043in}}%
\pgfpathlineto{\pgfqpoint{2.709458in}{3.273656in}}%
\pgfpathlineto{\pgfqpoint{2.710722in}{3.278789in}}%
\pgfpathlineto{\pgfqpoint{2.715778in}{3.308748in}}%
\pgfpathlineto{\pgfqpoint{2.718726in}{3.309251in}}%
\pgfpathlineto{\pgfqpoint{2.720833in}{3.313300in}}%
\pgfpathlineto{\pgfqpoint{2.722939in}{3.312383in}}%
\pgfpathlineto{\pgfqpoint{2.727995in}{3.295517in}}%
\pgfpathlineto{\pgfqpoint{2.732207in}{3.289019in}}%
\pgfpathlineto{\pgfqpoint{2.734735in}{3.287419in}}%
\pgfpathlineto{\pgfqpoint{2.736841in}{3.281724in}}%
\pgfpathlineto{\pgfqpoint{2.737684in}{3.282280in}}%
\pgfpathlineto{\pgfqpoint{2.754535in}{3.282845in}}%
\pgfpathlineto{\pgfqpoint{2.756642in}{3.277522in}}%
\pgfpathlineto{\pgfqpoint{2.757484in}{3.278619in}}%
\pgfpathlineto{\pgfqpoint{2.760012in}{3.277882in}}%
\pgfpathlineto{\pgfqpoint{2.763803in}{3.286661in}}%
\pgfpathlineto{\pgfqpoint{2.764225in}{3.286104in}}%
\pgfpathlineto{\pgfqpoint{2.765067in}{3.285616in}}%
\pgfpathlineto{\pgfqpoint{2.765488in}{3.286336in}}%
\pgfpathlineto{\pgfqpoint{2.766331in}{3.286636in}}%
\pgfpathlineto{\pgfqpoint{2.770122in}{3.297073in}}%
\pgfpathlineto{\pgfqpoint{2.770544in}{3.298794in}}%
\pgfpathlineto{\pgfqpoint{2.771386in}{3.297823in}}%
\pgfpathlineto{\pgfqpoint{2.772650in}{3.298309in}}%
\pgfpathlineto{\pgfqpoint{2.773914in}{3.300589in}}%
\pgfpathlineto{\pgfqpoint{2.774335in}{3.299860in}}%
\pgfpathlineto{\pgfqpoint{2.776020in}{3.297717in}}%
\pgfpathlineto{\pgfqpoint{2.780654in}{3.298532in}}%
\pgfpathlineto{\pgfqpoint{2.782339in}{3.300740in}}%
\pgfpathlineto{\pgfqpoint{2.783182in}{3.299234in}}%
\pgfpathlineto{\pgfqpoint{2.787395in}{3.293388in}}%
\pgfpathlineto{\pgfqpoint{2.792450in}{3.292368in}}%
\pgfpathlineto{\pgfqpoint{2.795399in}{3.291618in}}%
\pgfpathlineto{\pgfqpoint{2.795820in}{3.290056in}}%
\pgfpathlineto{\pgfqpoint{2.796663in}{3.291026in}}%
\pgfpathlineto{\pgfqpoint{2.798769in}{3.294757in}}%
\pgfpathlineto{\pgfqpoint{2.799612in}{3.295593in}}%
\pgfpathlineto{\pgfqpoint{2.801718in}{3.297395in}}%
\pgfpathlineto{\pgfqpoint{2.803825in}{3.298134in}}%
\pgfpathlineto{\pgfqpoint{2.806352in}{3.297766in}}%
\pgfpathlineto{\pgfqpoint{2.809723in}{3.297973in}}%
\pgfpathlineto{\pgfqpoint{2.813514in}{3.293267in}}%
\pgfpathlineto{\pgfqpoint{2.815199in}{3.293572in}}%
\pgfpathlineto{\pgfqpoint{2.816884in}{3.291480in}}%
\pgfpathlineto{\pgfqpoint{2.818991in}{3.275664in}}%
\pgfpathlineto{\pgfqpoint{2.823203in}{3.243986in}}%
\pgfpathlineto{\pgfqpoint{2.824046in}{3.243611in}}%
\pgfpathlineto{\pgfqpoint{2.826574in}{3.241966in}}%
\pgfpathlineto{\pgfqpoint{2.828259in}{3.243809in}}%
\pgfpathlineto{\pgfqpoint{2.830365in}{3.243234in}}%
\pgfpathlineto{\pgfqpoint{2.832050in}{3.241972in}}%
\pgfpathlineto{\pgfqpoint{2.834999in}{3.242152in}}%
\pgfpathlineto{\pgfqpoint{2.835421in}{3.243758in}}%
\pgfpathlineto{\pgfqpoint{2.836263in}{3.243018in}}%
\pgfpathlineto{\pgfqpoint{2.840055in}{3.242999in}}%
\pgfpathlineto{\pgfqpoint{2.840897in}{3.243634in}}%
\pgfpathlineto{\pgfqpoint{2.841318in}{3.243125in}}%
\pgfpathlineto{\pgfqpoint{2.844689in}{3.241130in}}%
\pgfpathlineto{\pgfqpoint{2.846374in}{3.240578in}}%
\pgfpathlineto{\pgfqpoint{2.848480in}{3.240442in}}%
\pgfpathlineto{\pgfqpoint{2.858591in}{3.158009in}}%
\pgfpathlineto{\pgfqpoint{2.869544in}{3.159441in}}%
\pgfpathlineto{\pgfqpoint{2.870387in}{3.161263in}}%
\pgfpathlineto{\pgfqpoint{2.883446in}{3.270134in}}%
\pgfpathlineto{\pgfqpoint{2.885553in}{3.276379in}}%
\pgfpathlineto{\pgfqpoint{2.886816in}{3.278752in}}%
\pgfpathlineto{\pgfqpoint{2.889344in}{3.284261in}}%
\pgfpathlineto{\pgfqpoint{2.891029in}{3.283304in}}%
\pgfpathlineto{\pgfqpoint{2.901561in}{3.283405in}}%
\pgfpathlineto{\pgfqpoint{2.902825in}{3.279265in}}%
\pgfpathlineto{\pgfqpoint{2.906617in}{3.246171in}}%
\pgfpathlineto{\pgfqpoint{2.915042in}{3.176417in}}%
\pgfpathlineto{\pgfqpoint{2.915885in}{3.174838in}}%
\pgfpathlineto{\pgfqpoint{2.916306in}{3.175240in}}%
\pgfpathlineto{\pgfqpoint{2.918412in}{3.185373in}}%
\pgfpathlineto{\pgfqpoint{2.930208in}{3.246686in}}%
\pgfpathlineto{\pgfqpoint{2.937791in}{3.262412in}}%
\pgfpathlineto{\pgfqpoint{2.938634in}{3.263284in}}%
\pgfpathlineto{\pgfqpoint{2.939476in}{3.263961in}}%
\pgfpathlineto{\pgfqpoint{2.939898in}{3.263254in}}%
\pgfpathlineto{\pgfqpoint{2.941161in}{3.261549in}}%
\pgfpathlineto{\pgfqpoint{2.946217in}{3.249248in}}%
\pgfpathlineto{\pgfqpoint{2.952536in}{3.235652in}}%
\pgfpathlineto{\pgfqpoint{2.953378in}{3.236466in}}%
\pgfpathlineto{\pgfqpoint{2.954642in}{3.238363in}}%
\pgfpathlineto{\pgfqpoint{2.961804in}{3.254056in}}%
\pgfpathlineto{\pgfqpoint{2.968123in}{3.263485in}}%
\pgfpathlineto{\pgfqpoint{2.970230in}{3.263260in}}%
\pgfpathlineto{\pgfqpoint{2.971493in}{3.263169in}}%
\pgfpathlineto{\pgfqpoint{2.972757in}{3.262537in}}%
\pgfpathlineto{\pgfqpoint{2.977391in}{3.260679in}}%
\pgfpathlineto{\pgfqpoint{2.978655in}{3.260681in}}%
\pgfpathlineto{\pgfqpoint{2.979076in}{3.260049in}}%
\pgfpathlineto{\pgfqpoint{2.979498in}{3.260724in}}%
\pgfpathlineto{\pgfqpoint{2.980340in}{3.260784in}}%
\pgfpathlineto{\pgfqpoint{2.981604in}{3.260616in}}%
\pgfpathlineto{\pgfqpoint{2.984132in}{3.260771in}}%
\pgfpathlineto{\pgfqpoint{2.984974in}{3.261350in}}%
\pgfpathlineto{\pgfqpoint{2.985396in}{3.261811in}}%
\pgfpathlineto{\pgfqpoint{2.985817in}{3.261037in}}%
\pgfpathlineto{\pgfqpoint{2.986659in}{3.259978in}}%
\pgfpathlineto{\pgfqpoint{2.987081in}{3.260749in}}%
\pgfpathlineto{\pgfqpoint{2.990030in}{3.263326in}}%
\pgfpathlineto{\pgfqpoint{2.992136in}{3.263752in}}%
\pgfpathlineto{\pgfqpoint{2.992979in}{3.262136in}}%
\pgfpathlineto{\pgfqpoint{2.994242in}{3.262118in}}%
\pgfpathlineto{\pgfqpoint{2.998876in}{3.262803in}}%
\pgfpathlineto{\pgfqpoint{3.000983in}{3.263384in}}%
\pgfpathlineto{\pgfqpoint{3.003932in}{3.259785in}}%
\pgfpathlineto{\pgfqpoint{3.012357in}{3.233556in}}%
\pgfpathlineto{\pgfqpoint{3.017834in}{3.216997in}}%
\pgfpathlineto{\pgfqpoint{3.019519in}{3.216798in}}%
\pgfpathlineto{\pgfqpoint{3.024153in}{3.212336in}}%
\pgfpathlineto{\pgfqpoint{3.026681in}{3.208991in}}%
\pgfpathlineto{\pgfqpoint{3.027945in}{3.209533in}}%
\pgfpathlineto{\pgfqpoint{3.028787in}{3.211569in}}%
\pgfpathlineto{\pgfqpoint{3.031736in}{3.220366in}}%
\pgfpathlineto{\pgfqpoint{3.033843in}{3.226341in}}%
\pgfpathlineto{\pgfqpoint{3.038898in}{3.252960in}}%
\pgfpathlineto{\pgfqpoint{3.042689in}{3.269820in}}%
\pgfpathlineto{\pgfqpoint{3.043532in}{3.270656in}}%
\pgfpathlineto{\pgfqpoint{3.043953in}{3.270284in}}%
\pgfpathlineto{\pgfqpoint{3.046060in}{3.270230in}}%
\pgfpathlineto{\pgfqpoint{3.046902in}{3.270977in}}%
\pgfpathlineto{\pgfqpoint{3.047323in}{3.270307in}}%
\pgfpathlineto{\pgfqpoint{3.047745in}{3.269263in}}%
\pgfpathlineto{\pgfqpoint{3.048587in}{3.269715in}}%
\pgfpathlineto{\pgfqpoint{3.054906in}{3.269333in}}%
\pgfpathlineto{\pgfqpoint{3.057013in}{3.269032in}}%
\pgfpathlineto{\pgfqpoint{3.059119in}{3.270560in}}%
\pgfpathlineto{\pgfqpoint{3.062489in}{3.269348in}}%
\pgfpathlineto{\pgfqpoint{3.064596in}{3.265796in}}%
\pgfpathlineto{\pgfqpoint{3.065017in}{3.265848in}}%
\pgfpathlineto{\pgfqpoint{3.066281in}{3.267539in}}%
\pgfpathlineto{\pgfqpoint{3.067545in}{3.267796in}}%
\pgfpathlineto{\pgfqpoint{3.068387in}{3.267064in}}%
\pgfpathlineto{\pgfqpoint{3.069230in}{3.269788in}}%
\pgfpathlineto{\pgfqpoint{3.069651in}{3.269506in}}%
\pgfpathlineto{\pgfqpoint{3.070915in}{3.269157in}}%
\pgfpathlineto{\pgfqpoint{3.071758in}{3.269124in}}%
\pgfpathlineto{\pgfqpoint{3.074285in}{3.274731in}}%
\pgfpathlineto{\pgfqpoint{3.075970in}{3.276629in}}%
\pgfpathlineto{\pgfqpoint{3.078498in}{3.282854in}}%
\pgfpathlineto{\pgfqpoint{3.078919in}{3.282166in}}%
\pgfpathlineto{\pgfqpoint{3.080604in}{3.281646in}}%
\pgfpathlineto{\pgfqpoint{3.081868in}{3.280069in}}%
\pgfpathlineto{\pgfqpoint{3.083553in}{3.276842in}}%
\pgfpathlineto{\pgfqpoint{3.085238in}{3.278087in}}%
\pgfpathlineto{\pgfqpoint{3.086502in}{3.276393in}}%
\pgfpathlineto{\pgfqpoint{3.086924in}{3.276662in}}%
\pgfpathlineto{\pgfqpoint{3.090294in}{3.277722in}}%
\pgfpathlineto{\pgfqpoint{3.093664in}{3.276389in}}%
\pgfpathlineto{\pgfqpoint{3.096192in}{3.276844in}}%
\pgfpathlineto{\pgfqpoint{3.101247in}{3.278446in}}%
\pgfpathlineto{\pgfqpoint{3.105460in}{3.279003in}}%
\pgfpathlineto{\pgfqpoint{3.111358in}{3.290099in}}%
\pgfpathlineto{\pgfqpoint{3.118941in}{3.282462in}}%
\pgfpathlineto{\pgfqpoint{3.122732in}{3.262927in}}%
\pgfpathlineto{\pgfqpoint{3.124839in}{3.255182in}}%
\pgfpathlineto{\pgfqpoint{3.129051in}{3.256740in}}%
\pgfpathlineto{\pgfqpoint{3.132000in}{3.255021in}}%
\pgfpathlineto{\pgfqpoint{3.133264in}{3.255187in}}%
\pgfpathlineto{\pgfqpoint{3.136634in}{3.257454in}}%
\pgfpathlineto{\pgfqpoint{3.139583in}{3.258286in}}%
\pgfpathlineto{\pgfqpoint{3.146324in}{3.257497in}}%
\pgfpathlineto{\pgfqpoint{3.149273in}{3.255453in}}%
\pgfpathlineto{\pgfqpoint{3.150537in}{3.255427in}}%
\pgfpathlineto{\pgfqpoint{3.153907in}{3.255212in}}%
\pgfpathlineto{\pgfqpoint{3.156434in}{3.254887in}}%
\pgfpathlineto{\pgfqpoint{3.158541in}{3.254405in}}%
\pgfpathlineto{\pgfqpoint{3.170337in}{3.252557in}}%
\pgfpathlineto{\pgfqpoint{3.172022in}{3.251256in}}%
\pgfpathlineto{\pgfqpoint{3.172864in}{3.252473in}}%
\pgfpathlineto{\pgfqpoint{3.177920in}{3.272699in}}%
\pgfpathlineto{\pgfqpoint{3.178341in}{3.272599in}}%
\pgfpathlineto{\pgfqpoint{3.182132in}{3.273330in}}%
\pgfpathlineto{\pgfqpoint{3.185924in}{3.280209in}}%
\pgfpathlineto{\pgfqpoint{3.186345in}{3.279932in}}%
\pgfpathlineto{\pgfqpoint{3.186766in}{3.279376in}}%
\pgfpathlineto{\pgfqpoint{3.187188in}{3.280001in}}%
\pgfpathlineto{\pgfqpoint{3.187609in}{3.280597in}}%
\pgfpathlineto{\pgfqpoint{3.188030in}{3.280298in}}%
\pgfpathlineto{\pgfqpoint{3.190137in}{3.279044in}}%
\pgfpathlineto{\pgfqpoint{3.192664in}{3.282767in}}%
\pgfpathlineto{\pgfqpoint{3.193507in}{3.279839in}}%
\pgfpathlineto{\pgfqpoint{3.195192in}{3.267400in}}%
\pgfpathlineto{\pgfqpoint{3.198984in}{3.242658in}}%
\pgfpathlineto{\pgfqpoint{3.202354in}{3.239805in}}%
\pgfpathlineto{\pgfqpoint{3.204039in}{3.237851in}}%
\pgfpathlineto{\pgfqpoint{3.206145in}{3.234660in}}%
\pgfpathlineto{\pgfqpoint{3.207409in}{3.239460in}}%
\pgfpathlineto{\pgfqpoint{3.209516in}{3.256327in}}%
\pgfpathlineto{\pgfqpoint{3.212043in}{3.271628in}}%
\pgfpathlineto{\pgfqpoint{3.215835in}{3.273538in}}%
\pgfpathlineto{\pgfqpoint{3.217099in}{3.274323in}}%
\pgfpathlineto{\pgfqpoint{3.219626in}{3.278561in}}%
\pgfpathlineto{\pgfqpoint{3.220890in}{3.273531in}}%
\pgfpathlineto{\pgfqpoint{3.223418in}{3.252452in}}%
\pgfpathlineto{\pgfqpoint{3.225524in}{3.241662in}}%
\pgfpathlineto{\pgfqpoint{3.229737in}{3.241049in}}%
\pgfpathlineto{\pgfqpoint{3.238162in}{3.240836in}}%
\pgfpathlineto{\pgfqpoint{3.240690in}{3.227303in}}%
\pgfpathlineto{\pgfqpoint{3.246588in}{3.226991in}}%
\pgfpathlineto{\pgfqpoint{3.253328in}{3.280944in}}%
\pgfpathlineto{\pgfqpoint{3.254171in}{3.279969in}}%
\pgfpathlineto{\pgfqpoint{3.255014in}{3.278182in}}%
\pgfpathlineto{\pgfqpoint{3.265546in}{3.225069in}}%
\pgfpathlineto{\pgfqpoint{3.268073in}{3.217520in}}%
\pgfpathlineto{\pgfqpoint{3.271022in}{3.218344in}}%
\pgfpathlineto{\pgfqpoint{3.273129in}{3.216692in}}%
\pgfpathlineto{\pgfqpoint{3.273971in}{3.218194in}}%
\pgfpathlineto{\pgfqpoint{3.284503in}{3.255947in}}%
\pgfpathlineto{\pgfqpoint{3.287031in}{3.264228in}}%
\pgfpathlineto{\pgfqpoint{3.288295in}{3.264828in}}%
\pgfpathlineto{\pgfqpoint{3.288716in}{3.264405in}}%
\pgfpathlineto{\pgfqpoint{3.297984in}{3.251877in}}%
\pgfpathlineto{\pgfqpoint{3.302197in}{3.252773in}}%
\pgfpathlineto{\pgfqpoint{3.307252in}{3.264168in}}%
\pgfpathlineto{\pgfqpoint{3.307673in}{3.263455in}}%
\pgfpathlineto{\pgfqpoint{3.309358in}{3.260484in}}%
\pgfpathlineto{\pgfqpoint{3.317784in}{3.232759in}}%
\pgfpathlineto{\pgfqpoint{3.321154in}{3.218447in}}%
\pgfpathlineto{\pgfqpoint{3.321997in}{3.216911in}}%
\pgfpathlineto{\pgfqpoint{3.322839in}{3.217511in}}%
\pgfpathlineto{\pgfqpoint{3.324524in}{3.219173in}}%
\pgfpathlineto{\pgfqpoint{3.324946in}{3.218550in}}%
\pgfpathlineto{\pgfqpoint{3.326631in}{3.217204in}}%
\pgfpathlineto{\pgfqpoint{3.327473in}{3.218117in}}%
\pgfpathlineto{\pgfqpoint{3.337163in}{3.257760in}}%
\pgfpathlineto{\pgfqpoint{3.337584in}{3.257375in}}%
\pgfpathlineto{\pgfqpoint{3.340533in}{3.265881in}}%
\pgfpathlineto{\pgfqpoint{3.340954in}{3.265009in}}%
\pgfpathlineto{\pgfqpoint{3.347273in}{3.250581in}}%
\pgfpathlineto{\pgfqpoint{3.348959in}{3.247874in}}%
\pgfpathlineto{\pgfqpoint{3.351486in}{3.245749in}}%
\pgfpathlineto{\pgfqpoint{3.351908in}{3.246341in}}%
\pgfpathlineto{\pgfqpoint{3.354435in}{3.249740in}}%
\pgfpathlineto{\pgfqpoint{3.354856in}{3.248750in}}%
\pgfpathlineto{\pgfqpoint{3.355699in}{3.249492in}}%
\pgfpathlineto{\pgfqpoint{3.357805in}{3.247642in}}%
\pgfpathlineto{\pgfqpoint{3.358648in}{3.245458in}}%
\pgfpathlineto{\pgfqpoint{3.359069in}{3.245801in}}%
\pgfpathlineto{\pgfqpoint{3.360333in}{3.248456in}}%
\pgfpathlineto{\pgfqpoint{3.361176in}{3.252308in}}%
\pgfpathlineto{\pgfqpoint{3.362018in}{3.250846in}}%
\pgfpathlineto{\pgfqpoint{3.364546in}{3.242916in}}%
\pgfpathlineto{\pgfqpoint{3.365810in}{3.236491in}}%
\pgfpathlineto{\pgfqpoint{3.371286in}{3.214120in}}%
\pgfpathlineto{\pgfqpoint{3.371708in}{3.214613in}}%
\pgfpathlineto{\pgfqpoint{3.372129in}{3.213474in}}%
\pgfpathlineto{\pgfqpoint{3.376763in}{3.192202in}}%
\pgfpathlineto{\pgfqpoint{3.377184in}{3.192316in}}%
\pgfpathlineto{\pgfqpoint{3.386031in}{3.240860in}}%
\pgfpathlineto{\pgfqpoint{3.390665in}{3.260298in}}%
\pgfpathlineto{\pgfqpoint{3.401618in}{3.201100in}}%
\pgfpathlineto{\pgfqpoint{3.402461in}{3.202978in}}%
\pgfpathlineto{\pgfqpoint{3.407516in}{3.225322in}}%
\pgfpathlineto{\pgfqpoint{3.413835in}{3.256985in}}%
\pgfpathlineto{\pgfqpoint{3.415099in}{3.259805in}}%
\pgfpathlineto{\pgfqpoint{3.426052in}{3.200538in}}%
\pgfpathlineto{\pgfqpoint{3.426895in}{3.202546in}}%
\pgfpathlineto{\pgfqpoint{3.432372in}{3.228441in}}%
\pgfpathlineto{\pgfqpoint{3.436163in}{3.249682in}}%
\pgfpathlineto{\pgfqpoint{3.439533in}{3.262096in}}%
\pgfpathlineto{\pgfqpoint{3.450487in}{3.200264in}}%
\pgfpathlineto{\pgfqpoint{3.451329in}{3.202298in}}%
\pgfpathlineto{\pgfqpoint{3.455542in}{3.222573in}}%
\pgfpathlineto{\pgfqpoint{3.457648in}{3.229471in}}%
\pgfpathlineto{\pgfqpoint{3.458491in}{3.228252in}}%
\pgfpathlineto{\pgfqpoint{3.461019in}{3.224127in}}%
\pgfpathlineto{\pgfqpoint{3.463125in}{3.221388in}}%
\pgfpathlineto{\pgfqpoint{3.464389in}{3.225888in}}%
\pgfpathlineto{\pgfqpoint{3.464810in}{3.225204in}}%
\pgfpathlineto{\pgfqpoint{3.466495in}{3.224316in}}%
\pgfpathlineto{\pgfqpoint{3.469023in}{3.223586in}}%
\pgfpathlineto{\pgfqpoint{3.471551in}{3.219406in}}%
\pgfpathlineto{\pgfqpoint{3.475763in}{3.219404in}}%
\pgfpathlineto{\pgfqpoint{3.476185in}{3.220243in}}%
\pgfpathlineto{\pgfqpoint{3.477027in}{3.224828in}}%
\pgfpathlineto{\pgfqpoint{3.477448in}{3.224706in}}%
\pgfpathlineto{\pgfqpoint{3.479555in}{3.224315in}}%
\pgfpathlineto{\pgfqpoint{3.481661in}{3.224159in}}%
\pgfpathlineto{\pgfqpoint{3.484610in}{3.219669in}}%
\pgfpathlineto{\pgfqpoint{3.488823in}{3.219912in}}%
\pgfpathlineto{\pgfqpoint{3.489665in}{3.225219in}}%
\pgfpathlineto{\pgfqpoint{3.490508in}{3.224777in}}%
\pgfpathlineto{\pgfqpoint{3.492614in}{3.224610in}}%
\pgfpathlineto{\pgfqpoint{3.494721in}{3.223827in}}%
\pgfpathlineto{\pgfqpoint{3.496406in}{3.222483in}}%
\pgfpathlineto{\pgfqpoint{3.497670in}{3.222881in}}%
\pgfpathlineto{\pgfqpoint{3.501883in}{3.228967in}}%
\pgfpathlineto{\pgfqpoint{3.502725in}{3.233325in}}%
\pgfpathlineto{\pgfqpoint{3.503146in}{3.233104in}}%
\pgfpathlineto{\pgfqpoint{3.505253in}{3.232785in}}%
\pgfpathlineto{\pgfqpoint{3.507359in}{3.232451in}}%
\pgfpathlineto{\pgfqpoint{3.509466in}{3.229881in}}%
\pgfpathlineto{\pgfqpoint{3.511151in}{3.229167in}}%
\pgfpathlineto{\pgfqpoint{3.514521in}{3.229521in}}%
\pgfpathlineto{\pgfqpoint{3.515363in}{3.234402in}}%
\pgfpathlineto{\pgfqpoint{3.515785in}{3.234136in}}%
\pgfpathlineto{\pgfqpoint{3.517470in}{3.233174in}}%
\pgfpathlineto{\pgfqpoint{3.519997in}{3.232982in}}%
\pgfpathlineto{\pgfqpoint{3.522525in}{3.229399in}}%
\pgfpathlineto{\pgfqpoint{3.524632in}{3.229132in}}%
\pgfpathlineto{\pgfqpoint{3.527159in}{3.228707in}}%
\pgfpathlineto{\pgfqpoint{3.528423in}{3.234083in}}%
\pgfpathlineto{\pgfqpoint{3.528844in}{3.233442in}}%
\pgfpathlineto{\pgfqpoint{3.530529in}{3.233092in}}%
\pgfpathlineto{\pgfqpoint{3.533057in}{3.232458in}}%
\pgfpathlineto{\pgfqpoint{3.535585in}{3.229438in}}%
\pgfpathlineto{\pgfqpoint{3.539798in}{3.229419in}}%
\pgfpathlineto{\pgfqpoint{3.541061in}{3.234956in}}%
\pgfpathlineto{\pgfqpoint{3.541904in}{3.233794in}}%
\pgfpathlineto{\pgfqpoint{3.546959in}{3.231955in}}%
\pgfpathlineto{\pgfqpoint{3.549066in}{3.229940in}}%
\pgfpathlineto{\pgfqpoint{3.552857in}{3.229509in}}%
\pgfpathlineto{\pgfqpoint{3.553700in}{3.234493in}}%
\pgfpathlineto{\pgfqpoint{3.554121in}{3.234241in}}%
\pgfpathlineto{\pgfqpoint{3.555806in}{3.233188in}}%
\pgfpathlineto{\pgfqpoint{3.558755in}{3.232134in}}%
\pgfpathlineto{\pgfqpoint{3.561283in}{3.229392in}}%
\pgfpathlineto{\pgfqpoint{3.565496in}{3.228888in}}%
\pgfpathlineto{\pgfqpoint{3.566759in}{3.234393in}}%
\pgfpathlineto{\pgfqpoint{3.567181in}{3.233827in}}%
\pgfpathlineto{\pgfqpoint{3.568023in}{3.235011in}}%
\pgfpathlineto{\pgfqpoint{3.568444in}{3.234614in}}%
\pgfpathlineto{\pgfqpoint{3.570551in}{3.234689in}}%
\pgfpathlineto{\pgfqpoint{3.572657in}{3.239876in}}%
\pgfpathlineto{\pgfqpoint{3.578555in}{3.284031in}}%
\pgfpathlineto{\pgfqpoint{3.580240in}{3.293029in}}%
\pgfpathlineto{\pgfqpoint{3.585717in}{3.304930in}}%
\pgfpathlineto{\pgfqpoint{3.586559in}{3.303666in}}%
\pgfpathlineto{\pgfqpoint{3.591194in}{3.304125in}}%
\pgfpathlineto{\pgfqpoint{3.591615in}{3.305544in}}%
\pgfpathlineto{\pgfqpoint{3.592036in}{3.305340in}}%
\pgfpathlineto{\pgfqpoint{3.593721in}{3.301906in}}%
\pgfpathlineto{\pgfqpoint{3.596670in}{3.302010in}}%
\pgfpathlineto{\pgfqpoint{3.598777in}{3.294704in}}%
\pgfpathlineto{\pgfqpoint{3.603832in}{3.251107in}}%
\pgfpathlineto{\pgfqpoint{3.606781in}{3.245054in}}%
\pgfpathlineto{\pgfqpoint{3.613943in}{3.229310in}}%
\pgfpathlineto{\pgfqpoint{3.616891in}{3.229771in}}%
\pgfpathlineto{\pgfqpoint{3.617734in}{3.234324in}}%
\pgfpathlineto{\pgfqpoint{3.618155in}{3.233977in}}%
\pgfpathlineto{\pgfqpoint{3.619840in}{3.233159in}}%
\pgfpathlineto{\pgfqpoint{3.622368in}{3.232875in}}%
\pgfpathlineto{\pgfqpoint{3.624474in}{3.230052in}}%
\pgfpathlineto{\pgfqpoint{3.626160in}{3.229030in}}%
\pgfpathlineto{\pgfqpoint{3.629530in}{3.228898in}}%
\pgfpathlineto{\pgfqpoint{3.630372in}{3.234214in}}%
\pgfpathlineto{\pgfqpoint{3.631215in}{3.233468in}}%
\pgfpathlineto{\pgfqpoint{3.633321in}{3.233091in}}%
\pgfpathlineto{\pgfqpoint{3.635428in}{3.232325in}}%
\pgfpathlineto{\pgfqpoint{3.637955in}{3.229406in}}%
\pgfpathlineto{\pgfqpoint{3.642589in}{3.230275in}}%
\pgfpathlineto{\pgfqpoint{3.643432in}{3.233846in}}%
\pgfpathlineto{\pgfqpoint{3.643853in}{3.233322in}}%
\pgfpathlineto{\pgfqpoint{3.645117in}{3.233437in}}%
\pgfpathlineto{\pgfqpoint{3.645960in}{3.232848in}}%
\pgfpathlineto{\pgfqpoint{3.649751in}{3.218101in}}%
\pgfpathlineto{\pgfqpoint{3.650594in}{3.221401in}}%
\pgfpathlineto{\pgfqpoint{3.654807in}{3.249616in}}%
\pgfpathlineto{\pgfqpoint{3.656913in}{3.261602in}}%
\pgfpathlineto{\pgfqpoint{3.659019in}{3.265028in}}%
\pgfpathlineto{\pgfqpoint{3.659441in}{3.265832in}}%
\pgfpathlineto{\pgfqpoint{3.663653in}{3.297230in}}%
\pgfpathlineto{\pgfqpoint{3.664496in}{3.296744in}}%
\pgfpathlineto{\pgfqpoint{3.667866in}{3.295476in}}%
\pgfpathlineto{\pgfqpoint{3.669130in}{3.294130in}}%
\pgfpathlineto{\pgfqpoint{3.676713in}{3.294256in}}%
\pgfpathlineto{\pgfqpoint{3.677556in}{3.295374in}}%
\pgfpathlineto{\pgfqpoint{3.677977in}{3.294983in}}%
\pgfpathlineto{\pgfqpoint{3.682190in}{3.293676in}}%
\pgfpathlineto{\pgfqpoint{3.683875in}{3.296589in}}%
\pgfpathlineto{\pgfqpoint{3.684296in}{3.296338in}}%
\pgfpathlineto{\pgfqpoint{3.688930in}{3.288217in}}%
\pgfpathlineto{\pgfqpoint{3.691458in}{3.289155in}}%
\pgfpathlineto{\pgfqpoint{3.693985in}{3.289469in}}%
\pgfpathlineto{\pgfqpoint{3.695249in}{3.289184in}}%
\pgfpathlineto{\pgfqpoint{3.696513in}{3.288677in}}%
\pgfpathlineto{\pgfqpoint{3.697356in}{3.288191in}}%
\pgfpathlineto{\pgfqpoint{3.698198in}{3.287479in}}%
\pgfpathlineto{\pgfqpoint{3.698619in}{3.289124in}}%
\pgfpathlineto{\pgfqpoint{3.699041in}{3.288074in}}%
\pgfpathlineto{\pgfqpoint{3.699883in}{3.288092in}}%
\pgfpathlineto{\pgfqpoint{3.700305in}{3.289302in}}%
\pgfpathlineto{\pgfqpoint{3.700726in}{3.288115in}}%
\pgfpathlineto{\pgfqpoint{3.701147in}{3.287844in}}%
\pgfpathlineto{\pgfqpoint{3.701990in}{3.291519in}}%
\pgfpathlineto{\pgfqpoint{3.702832in}{3.290239in}}%
\pgfpathlineto{\pgfqpoint{3.703675in}{3.289515in}}%
\pgfpathlineto{\pgfqpoint{3.704096in}{3.286832in}}%
\pgfpathlineto{\pgfqpoint{3.704939in}{3.288471in}}%
\pgfpathlineto{\pgfqpoint{3.708730in}{3.297807in}}%
\pgfpathlineto{\pgfqpoint{3.709151in}{3.297221in}}%
\pgfpathlineto{\pgfqpoint{3.710836in}{3.293167in}}%
\pgfpathlineto{\pgfqpoint{3.711258in}{3.294272in}}%
\pgfpathlineto{\pgfqpoint{3.713785in}{3.298873in}}%
\pgfpathlineto{\pgfqpoint{3.715049in}{3.302004in}}%
\pgfpathlineto{\pgfqpoint{3.715471in}{3.300728in}}%
\pgfpathlineto{\pgfqpoint{3.716734in}{3.299738in}}%
\pgfpathlineto{\pgfqpoint{3.717577in}{3.299451in}}%
\pgfpathlineto{\pgfqpoint{3.719683in}{3.291019in}}%
\pgfpathlineto{\pgfqpoint{3.723475in}{3.266782in}}%
\pgfpathlineto{\pgfqpoint{3.726003in}{3.258701in}}%
\pgfpathlineto{\pgfqpoint{3.726845in}{3.257404in}}%
\pgfpathlineto{\pgfqpoint{3.728109in}{3.251996in}}%
\pgfpathlineto{\pgfqpoint{3.728951in}{3.253138in}}%
\pgfpathlineto{\pgfqpoint{3.731058in}{3.253440in}}%
\pgfpathlineto{\pgfqpoint{3.733164in}{3.261878in}}%
\pgfpathlineto{\pgfqpoint{3.736956in}{3.286115in}}%
\pgfpathlineto{\pgfqpoint{3.739483in}{3.294140in}}%
\pgfpathlineto{\pgfqpoint{3.740326in}{3.295435in}}%
\pgfpathlineto{\pgfqpoint{3.741590in}{3.300843in}}%
\pgfpathlineto{\pgfqpoint{3.742432in}{3.299672in}}%
\pgfpathlineto{\pgfqpoint{3.757598in}{3.298237in}}%
\pgfpathlineto{\pgfqpoint{3.758862in}{3.291541in}}%
\pgfpathlineto{\pgfqpoint{3.762232in}{3.257288in}}%
\pgfpathlineto{\pgfqpoint{3.771922in}{3.166009in}}%
\pgfpathlineto{\pgfqpoint{3.772764in}{3.165144in}}%
\pgfpathlineto{\pgfqpoint{3.773186in}{3.165768in}}%
\pgfpathlineto{\pgfqpoint{3.774028in}{3.170051in}}%
\pgfpathlineto{\pgfqpoint{3.784560in}{3.247209in}}%
\pgfpathlineto{\pgfqpoint{3.786667in}{3.261611in}}%
\pgfpathlineto{\pgfqpoint{3.787930in}{3.266946in}}%
\pgfpathlineto{\pgfqpoint{3.788352in}{3.266806in}}%
\pgfpathlineto{\pgfqpoint{3.789194in}{3.263194in}}%
\pgfpathlineto{\pgfqpoint{3.793407in}{3.244577in}}%
\pgfpathlineto{\pgfqpoint{3.797620in}{3.235419in}}%
\pgfpathlineto{\pgfqpoint{3.799305in}{3.239703in}}%
\pgfpathlineto{\pgfqpoint{3.799726in}{3.239473in}}%
\pgfpathlineto{\pgfqpoint{3.802675in}{3.236758in}}%
\pgfpathlineto{\pgfqpoint{3.804360in}{3.242591in}}%
\pgfpathlineto{\pgfqpoint{3.807730in}{3.253304in}}%
\pgfpathlineto{\pgfqpoint{3.809837in}{3.257977in}}%
\pgfpathlineto{\pgfqpoint{3.811943in}{3.265944in}}%
\pgfpathlineto{\pgfqpoint{3.814050in}{3.262373in}}%
\pgfpathlineto{\pgfqpoint{3.816577in}{3.256723in}}%
\pgfpathlineto{\pgfqpoint{3.817841in}{3.255090in}}%
\pgfpathlineto{\pgfqpoint{3.821211in}{3.247618in}}%
\pgfpathlineto{\pgfqpoint{3.822475in}{3.250994in}}%
\pgfpathlineto{\pgfqpoint{3.823318in}{3.252060in}}%
\pgfpathlineto{\pgfqpoint{3.823739in}{3.251660in}}%
\pgfpathlineto{\pgfqpoint{3.827109in}{3.245863in}}%
\pgfpathlineto{\pgfqpoint{3.828794in}{3.256743in}}%
\pgfpathlineto{\pgfqpoint{3.831322in}{3.267079in}}%
\pgfpathlineto{\pgfqpoint{3.835535in}{3.281709in}}%
\pgfpathlineto{\pgfqpoint{3.835956in}{3.281311in}}%
\pgfpathlineto{\pgfqpoint{3.838484in}{3.279377in}}%
\pgfpathlineto{\pgfqpoint{3.839326in}{3.278554in}}%
\pgfpathlineto{\pgfqpoint{3.841011in}{3.285998in}}%
\pgfpathlineto{\pgfqpoint{3.841433in}{3.285078in}}%
\pgfpathlineto{\pgfqpoint{3.843539in}{3.279929in}}%
\pgfpathlineto{\pgfqpoint{3.845645in}{3.280398in}}%
\pgfpathlineto{\pgfqpoint{3.846909in}{3.286565in}}%
\pgfpathlineto{\pgfqpoint{3.847331in}{3.286456in}}%
\pgfpathlineto{\pgfqpoint{3.848594in}{3.283328in}}%
\pgfpathlineto{\pgfqpoint{3.851122in}{3.268868in}}%
\pgfpathlineto{\pgfqpoint{3.851965in}{3.270108in}}%
\pgfpathlineto{\pgfqpoint{3.852807in}{3.272223in}}%
\pgfpathlineto{\pgfqpoint{3.853229in}{3.270696in}}%
\pgfpathlineto{\pgfqpoint{3.857441in}{3.256270in}}%
\pgfpathlineto{\pgfqpoint{3.859969in}{3.238014in}}%
\pgfpathlineto{\pgfqpoint{3.861233in}{3.229393in}}%
\pgfpathlineto{\pgfqpoint{3.861654in}{3.229978in}}%
\pgfpathlineto{\pgfqpoint{3.862497in}{3.231516in}}%
\pgfpathlineto{\pgfqpoint{3.864603in}{3.242479in}}%
\pgfpathlineto{\pgfqpoint{3.865867in}{3.241088in}}%
\pgfpathlineto{\pgfqpoint{3.868816in}{3.237526in}}%
\pgfpathlineto{\pgfqpoint{3.873029in}{3.229556in}}%
\pgfpathlineto{\pgfqpoint{3.874292in}{3.228817in}}%
\pgfpathlineto{\pgfqpoint{3.883561in}{3.174568in}}%
\pgfpathlineto{\pgfqpoint{3.894092in}{3.175135in}}%
\pgfpathlineto{\pgfqpoint{3.896199in}{3.175704in}}%
\pgfpathlineto{\pgfqpoint{3.901675in}{3.176445in}}%
\pgfpathlineto{\pgfqpoint{3.904624in}{3.177329in}}%
\pgfpathlineto{\pgfqpoint{3.907573in}{3.176487in}}%
\pgfpathlineto{\pgfqpoint{3.909258in}{3.176187in}}%
\pgfpathlineto{\pgfqpoint{3.912207in}{3.177384in}}%
\pgfpathlineto{\pgfqpoint{3.915999in}{3.176223in}}%
\pgfpathlineto{\pgfqpoint{3.918105in}{3.176042in}}%
\pgfpathlineto{\pgfqpoint{3.923161in}{3.175417in}}%
\pgfpathlineto{\pgfqpoint{3.926110in}{3.174511in}}%
\pgfpathlineto{\pgfqpoint{3.931586in}{3.175368in}}%
\pgfpathlineto{\pgfqpoint{3.937905in}{3.179149in}}%
\pgfpathlineto{\pgfqpoint{3.944646in}{3.182080in}}%
\pgfpathlineto{\pgfqpoint{3.950122in}{3.178901in}}%
\pgfpathlineto{\pgfqpoint{3.954757in}{3.176356in}}%
\pgfpathlineto{\pgfqpoint{3.956863in}{3.176609in}}%
\pgfpathlineto{\pgfqpoint{3.959391in}{3.176319in}}%
\pgfpathlineto{\pgfqpoint{3.978769in}{3.176009in}}%
\pgfpathlineto{\pgfqpoint{3.981718in}{3.175925in}}%
\pgfpathlineto{\pgfqpoint{3.987195in}{3.171413in}}%
\pgfpathlineto{\pgfqpoint{3.991408in}{3.173337in}}%
\pgfpathlineto{\pgfqpoint{3.994357in}{3.182452in}}%
\pgfpathlineto{\pgfqpoint{3.998991in}{3.203949in}}%
\pgfpathlineto{\pgfqpoint{4.005310in}{3.234243in}}%
\pgfpathlineto{\pgfqpoint{4.013735in}{3.258113in}}%
\pgfpathlineto{\pgfqpoint{4.014157in}{3.258792in}}%
\pgfpathlineto{\pgfqpoint{4.014578in}{3.258269in}}%
\pgfpathlineto{\pgfqpoint{4.018791in}{3.251266in}}%
\pgfpathlineto{\pgfqpoint{4.019633in}{3.252404in}}%
\pgfpathlineto{\pgfqpoint{4.020897in}{3.251873in}}%
\pgfpathlineto{\pgfqpoint{4.025531in}{3.241596in}}%
\pgfpathlineto{\pgfqpoint{4.026374in}{3.242827in}}%
\pgfpathlineto{\pgfqpoint{4.040276in}{3.268576in}}%
\pgfpathlineto{\pgfqpoint{4.040697in}{3.267624in}}%
\pgfpathlineto{\pgfqpoint{4.041119in}{3.266657in}}%
\pgfpathlineto{\pgfqpoint{4.041961in}{3.267497in}}%
\pgfpathlineto{\pgfqpoint{4.045331in}{3.266864in}}%
\pgfpathlineto{\pgfqpoint{4.047859in}{3.264981in}}%
\pgfpathlineto{\pgfqpoint{4.060076in}{3.271524in}}%
\pgfpathlineto{\pgfqpoint{4.062182in}{3.272466in}}%
\pgfpathlineto{\pgfqpoint{4.067659in}{3.275990in}}%
\pgfpathlineto{\pgfqpoint{4.069765in}{3.274766in}}%
\pgfpathlineto{\pgfqpoint{4.073557in}{3.275703in}}%
\pgfpathlineto{\pgfqpoint{4.075663in}{3.278586in}}%
\pgfpathlineto{\pgfqpoint{4.077770in}{3.273484in}}%
\pgfpathlineto{\pgfqpoint{4.080297in}{3.261972in}}%
\pgfpathlineto{\pgfqpoint{4.080719in}{3.262178in}}%
\pgfpathlineto{\pgfqpoint{4.084931in}{3.263592in}}%
\pgfpathlineto{\pgfqpoint{4.086195in}{3.261224in}}%
\pgfpathlineto{\pgfqpoint{4.086617in}{3.261771in}}%
\pgfpathlineto{\pgfqpoint{4.088723in}{3.261747in}}%
\pgfpathlineto{\pgfqpoint{4.090408in}{3.260889in}}%
\pgfpathlineto{\pgfqpoint{4.092936in}{3.259832in}}%
\pgfpathlineto{\pgfqpoint{4.094200in}{3.262452in}}%
\pgfpathlineto{\pgfqpoint{4.095042in}{3.263161in}}%
\pgfpathlineto{\pgfqpoint{4.095463in}{3.262002in}}%
\pgfpathlineto{\pgfqpoint{4.096306in}{3.261253in}}%
\pgfpathlineto{\pgfqpoint{4.096727in}{3.262012in}}%
\pgfpathlineto{\pgfqpoint{4.099255in}{3.259587in}}%
\pgfpathlineto{\pgfqpoint{4.100940in}{3.263147in}}%
\pgfpathlineto{\pgfqpoint{4.101361in}{3.263815in}}%
\pgfpathlineto{\pgfqpoint{4.101783in}{3.263193in}}%
\pgfpathlineto{\pgfqpoint{4.102625in}{3.261632in}}%
\pgfpathlineto{\pgfqpoint{4.103468in}{3.261989in}}%
\pgfpathlineto{\pgfqpoint{4.104732in}{3.260544in}}%
\pgfpathlineto{\pgfqpoint{4.105574in}{3.259727in}}%
\pgfpathlineto{\pgfqpoint{4.105995in}{3.260626in}}%
\pgfpathlineto{\pgfqpoint{4.106838in}{3.263090in}}%
\pgfpathlineto{\pgfqpoint{4.107259in}{3.263014in}}%
\pgfpathlineto{\pgfqpoint{4.108102in}{3.263433in}}%
\pgfpathlineto{\pgfqpoint{4.108944in}{3.261208in}}%
\pgfpathlineto{\pgfqpoint{4.109787in}{3.261749in}}%
\pgfpathlineto{\pgfqpoint{4.112315in}{3.259087in}}%
\pgfpathlineto{\pgfqpoint{4.113157in}{3.261465in}}%
\pgfpathlineto{\pgfqpoint{4.113578in}{3.260873in}}%
\pgfpathlineto{\pgfqpoint{4.115685in}{3.258909in}}%
\pgfpathlineto{\pgfqpoint{4.119476in}{3.270851in}}%
\pgfpathlineto{\pgfqpoint{4.120319in}{3.269494in}}%
\pgfpathlineto{\pgfqpoint{4.122004in}{3.267445in}}%
\pgfpathlineto{\pgfqpoint{4.124532in}{3.272915in}}%
\pgfpathlineto{\pgfqpoint{4.126217in}{3.279576in}}%
\pgfpathlineto{\pgfqpoint{4.127059in}{3.279003in}}%
\pgfpathlineto{\pgfqpoint{4.138013in}{3.277206in}}%
\pgfpathlineto{\pgfqpoint{4.138855in}{3.276058in}}%
\pgfpathlineto{\pgfqpoint{4.140540in}{3.278405in}}%
\pgfpathlineto{\pgfqpoint{4.141804in}{3.278590in}}%
\pgfpathlineto{\pgfqpoint{4.148544in}{3.279254in}}%
\pgfpathlineto{\pgfqpoint{4.152336in}{3.282027in}}%
\pgfpathlineto{\pgfqpoint{4.152757in}{3.281192in}}%
\pgfpathlineto{\pgfqpoint{4.154442in}{3.279640in}}%
\pgfpathlineto{\pgfqpoint{4.158234in}{3.276988in}}%
\pgfpathlineto{\pgfqpoint{4.165396in}{3.264652in}}%
\pgfpathlineto{\pgfqpoint{4.167081in}{3.267200in}}%
\pgfpathlineto{\pgfqpoint{4.167923in}{3.265883in}}%
\pgfpathlineto{\pgfqpoint{4.171715in}{3.259546in}}%
\pgfpathlineto{\pgfqpoint{4.174664in}{3.260045in}}%
\pgfpathlineto{\pgfqpoint{4.175928in}{3.259561in}}%
\pgfpathlineto{\pgfqpoint{4.178034in}{3.257366in}}%
\pgfpathlineto{\pgfqpoint{4.179298in}{3.256584in}}%
\pgfpathlineto{\pgfqpoint{4.181404in}{3.252301in}}%
\pgfpathlineto{\pgfqpoint{4.185196in}{3.252642in}}%
\pgfpathlineto{\pgfqpoint{4.185617in}{3.253411in}}%
\pgfpathlineto{\pgfqpoint{4.186460in}{3.252647in}}%
\pgfpathlineto{\pgfqpoint{4.189408in}{3.252668in}}%
\pgfpathlineto{\pgfqpoint{4.191936in}{3.252717in}}%
\pgfpathlineto{\pgfqpoint{4.195306in}{3.262081in}}%
\pgfpathlineto{\pgfqpoint{4.196570in}{3.258302in}}%
\pgfpathlineto{\pgfqpoint{4.203732in}{3.208497in}}%
\pgfpathlineto{\pgfqpoint{4.206260in}{3.197253in}}%
\pgfpathlineto{\pgfqpoint{4.210472in}{3.189573in}}%
\pgfpathlineto{\pgfqpoint{4.214264in}{3.190576in}}%
\pgfpathlineto{\pgfqpoint{4.218477in}{3.194356in}}%
\pgfpathlineto{\pgfqpoint{4.223532in}{3.169431in}}%
\pgfpathlineto{\pgfqpoint{4.223953in}{3.169913in}}%
\pgfpathlineto{\pgfqpoint{4.229851in}{3.174693in}}%
\pgfpathlineto{\pgfqpoint{4.232800in}{3.175725in}}%
\pgfpathlineto{\pgfqpoint{4.234906in}{3.175846in}}%
\pgfpathlineto{\pgfqpoint{4.241226in}{3.175789in}}%
\pgfpathlineto{\pgfqpoint{4.248809in}{3.222370in}}%
\pgfpathlineto{\pgfqpoint{4.253021in}{3.249049in}}%
\pgfpathlineto{\pgfqpoint{4.255549in}{3.258130in}}%
\pgfpathlineto{\pgfqpoint{4.261447in}{3.267741in}}%
\pgfpathlineto{\pgfqpoint{4.261868in}{3.267135in}}%
\pgfpathlineto{\pgfqpoint{4.262290in}{3.266886in}}%
\pgfpathlineto{\pgfqpoint{4.267345in}{3.277581in}}%
\pgfpathlineto{\pgfqpoint{4.268609in}{3.277376in}}%
\pgfpathlineto{\pgfqpoint{4.271558in}{3.262781in}}%
\pgfpathlineto{\pgfqpoint{4.283353in}{3.194603in}}%
\pgfpathlineto{\pgfqpoint{4.283775in}{3.194699in}}%
\pgfpathlineto{\pgfqpoint{4.286724in}{3.194983in}}%
\pgfpathlineto{\pgfqpoint{4.297256in}{3.196560in}}%
\pgfpathlineto{\pgfqpoint{4.308630in}{3.253530in}}%
\pgfpathlineto{\pgfqpoint{4.310315in}{3.260978in}}%
\pgfpathlineto{\pgfqpoint{4.311158in}{3.260387in}}%
\pgfpathlineto{\pgfqpoint{4.315792in}{3.267865in}}%
\pgfpathlineto{\pgfqpoint{4.317056in}{3.270351in}}%
\pgfpathlineto{\pgfqpoint{4.317477in}{3.270116in}}%
\pgfpathlineto{\pgfqpoint{4.317898in}{3.269177in}}%
\pgfpathlineto{\pgfqpoint{4.318320in}{3.269986in}}%
\pgfpathlineto{\pgfqpoint{4.320847in}{3.277174in}}%
\pgfpathlineto{\pgfqpoint{4.321690in}{3.277984in}}%
\pgfpathlineto{\pgfqpoint{4.323796in}{3.280971in}}%
\pgfpathlineto{\pgfqpoint{4.324217in}{3.280898in}}%
\pgfpathlineto{\pgfqpoint{4.327166in}{3.264963in}}%
\pgfpathlineto{\pgfqpoint{4.338962in}{3.194851in}}%
\pgfpathlineto{\pgfqpoint{4.339383in}{3.195161in}}%
\pgfpathlineto{\pgfqpoint{4.341490in}{3.195628in}}%
\pgfpathlineto{\pgfqpoint{4.352443in}{3.194979in}}%
\pgfpathlineto{\pgfqpoint{4.354128in}{3.202829in}}%
\pgfpathlineto{\pgfqpoint{4.363396in}{3.248224in}}%
\pgfpathlineto{\pgfqpoint{4.365924in}{3.260600in}}%
\pgfpathlineto{\pgfqpoint{4.366345in}{3.260002in}}%
\pgfpathlineto{\pgfqpoint{4.368452in}{3.257382in}}%
\pgfpathlineto{\pgfqpoint{4.369715in}{3.256498in}}%
\pgfpathlineto{\pgfqpoint{4.371822in}{3.254183in}}%
\pgfpathlineto{\pgfqpoint{4.373086in}{3.253407in}}%
\pgfpathlineto{\pgfqpoint{4.373928in}{3.251042in}}%
\pgfpathlineto{\pgfqpoint{4.374350in}{3.251838in}}%
\pgfpathlineto{\pgfqpoint{4.376456in}{3.252571in}}%
\pgfpathlineto{\pgfqpoint{4.380669in}{3.246033in}}%
\pgfpathlineto{\pgfqpoint{4.382354in}{3.247376in}}%
\pgfpathlineto{\pgfqpoint{4.386988in}{3.251036in}}%
\pgfpathlineto{\pgfqpoint{4.387409in}{3.249749in}}%
\pgfpathlineto{\pgfqpoint{4.387830in}{3.249325in}}%
\pgfpathlineto{\pgfqpoint{4.388252in}{3.250010in}}%
\pgfpathlineto{\pgfqpoint{4.392465in}{3.254894in}}%
\pgfpathlineto{\pgfqpoint{4.393728in}{3.258517in}}%
\pgfpathlineto{\pgfqpoint{4.394571in}{3.257256in}}%
\pgfpathlineto{\pgfqpoint{4.399626in}{3.251238in}}%
\pgfpathlineto{\pgfqpoint{4.400890in}{3.250660in}}%
\pgfpathlineto{\pgfqpoint{4.401733in}{3.249327in}}%
\pgfpathlineto{\pgfqpoint{4.402154in}{3.250145in}}%
\pgfpathlineto{\pgfqpoint{4.404260in}{3.250805in}}%
\pgfpathlineto{\pgfqpoint{4.408473in}{3.243926in}}%
\pgfpathlineto{\pgfqpoint{4.411001in}{3.248189in}}%
\pgfpathlineto{\pgfqpoint{4.411843in}{3.248840in}}%
\pgfpathlineto{\pgfqpoint{4.414371in}{3.254846in}}%
\pgfpathlineto{\pgfqpoint{4.414792in}{3.255084in}}%
\pgfpathlineto{\pgfqpoint{4.415635in}{3.253808in}}%
\pgfpathlineto{\pgfqpoint{4.416056in}{3.254784in}}%
\pgfpathlineto{\pgfqpoint{4.419005in}{3.261661in}}%
\pgfpathlineto{\pgfqpoint{4.422375in}{3.277825in}}%
\pgfpathlineto{\pgfqpoint{4.425324in}{3.287511in}}%
\pgfpathlineto{\pgfqpoint{4.425745in}{3.286811in}}%
\pgfpathlineto{\pgfqpoint{4.426167in}{3.287155in}}%
\pgfpathlineto{\pgfqpoint{4.427852in}{3.289701in}}%
\pgfpathlineto{\pgfqpoint{4.428694in}{3.289242in}}%
\pgfpathlineto{\pgfqpoint{4.429958in}{3.286655in}}%
\pgfpathlineto{\pgfqpoint{4.430380in}{3.286958in}}%
\pgfpathlineto{\pgfqpoint{4.430801in}{3.286246in}}%
\pgfpathlineto{\pgfqpoint{4.433328in}{3.284783in}}%
\pgfpathlineto{\pgfqpoint{4.435014in}{3.286206in}}%
\pgfpathlineto{\pgfqpoint{4.435856in}{3.285691in}}%
\pgfpathlineto{\pgfqpoint{4.436699in}{3.284212in}}%
\pgfpathlineto{\pgfqpoint{4.437120in}{3.285188in}}%
\pgfpathlineto{\pgfqpoint{4.439226in}{3.285960in}}%
\pgfpathlineto{\pgfqpoint{4.440069in}{3.284642in}}%
\pgfpathlineto{\pgfqpoint{4.440490in}{3.285558in}}%
\pgfpathlineto{\pgfqpoint{4.442597in}{3.287793in}}%
\pgfpathlineto{\pgfqpoint{4.443018in}{3.285543in}}%
\pgfpathlineto{\pgfqpoint{4.443860in}{3.286119in}}%
\pgfpathlineto{\pgfqpoint{4.444703in}{3.286220in}}%
\pgfpathlineto{\pgfqpoint{4.448073in}{3.283558in}}%
\pgfpathlineto{\pgfqpoint{4.449758in}{3.282393in}}%
\pgfpathlineto{\pgfqpoint{4.450601in}{3.280690in}}%
\pgfpathlineto{\pgfqpoint{4.451022in}{3.281701in}}%
\pgfpathlineto{\pgfqpoint{4.453129in}{3.282616in}}%
\pgfpathlineto{\pgfqpoint{4.453550in}{3.281592in}}%
\pgfpathlineto{\pgfqpoint{4.453971in}{3.281999in}}%
\pgfpathlineto{\pgfqpoint{4.456499in}{3.287997in}}%
\pgfpathlineto{\pgfqpoint{4.456920in}{3.286077in}}%
\pgfpathlineto{\pgfqpoint{4.457763in}{3.287150in}}%
\pgfpathlineto{\pgfqpoint{4.459869in}{3.289179in}}%
\pgfpathlineto{\pgfqpoint{4.460290in}{3.288808in}}%
\pgfpathlineto{\pgfqpoint{4.460712in}{3.288272in}}%
\pgfpathlineto{\pgfqpoint{4.461133in}{3.289375in}}%
\pgfpathlineto{\pgfqpoint{4.463239in}{3.293356in}}%
\pgfpathlineto{\pgfqpoint{4.463661in}{3.293303in}}%
\pgfpathlineto{\pgfqpoint{4.464082in}{3.291907in}}%
\pgfpathlineto{\pgfqpoint{4.464503in}{3.292136in}}%
\pgfpathlineto{\pgfqpoint{4.466188in}{3.294178in}}%
\pgfpathlineto{\pgfqpoint{4.467031in}{3.294202in}}%
\pgfpathlineto{\pgfqpoint{4.467873in}{3.292311in}}%
\pgfpathlineto{\pgfqpoint{4.468295in}{3.293173in}}%
\pgfpathlineto{\pgfqpoint{4.470401in}{3.295993in}}%
\pgfpathlineto{\pgfqpoint{4.471244in}{3.294021in}}%
\pgfpathlineto{\pgfqpoint{4.471665in}{3.294646in}}%
\pgfpathlineto{\pgfqpoint{4.474192in}{3.294125in}}%
\pgfpathlineto{\pgfqpoint{4.474614in}{3.292758in}}%
\pgfpathlineto{\pgfqpoint{4.475035in}{3.293106in}}%
\pgfpathlineto{\pgfqpoint{4.476720in}{3.295405in}}%
\pgfpathlineto{\pgfqpoint{4.477563in}{3.295319in}}%
\pgfpathlineto{\pgfqpoint{4.478405in}{3.293543in}}%
\pgfpathlineto{\pgfqpoint{4.478827in}{3.294779in}}%
\pgfpathlineto{\pgfqpoint{4.480933in}{3.295560in}}%
\pgfpathlineto{\pgfqpoint{4.481775in}{3.293543in}}%
\pgfpathlineto{\pgfqpoint{4.482197in}{3.294283in}}%
\pgfpathlineto{\pgfqpoint{4.484303in}{3.296912in}}%
\pgfpathlineto{\pgfqpoint{4.485146in}{3.294649in}}%
\pgfpathlineto{\pgfqpoint{4.485567in}{3.295458in}}%
\pgfpathlineto{\pgfqpoint{4.486831in}{3.295115in}}%
\pgfpathlineto{\pgfqpoint{4.488095in}{3.293034in}}%
\pgfpathlineto{\pgfqpoint{4.488516in}{3.291221in}}%
\pgfpathlineto{\pgfqpoint{4.488937in}{3.291360in}}%
\pgfpathlineto{\pgfqpoint{4.490622in}{3.293350in}}%
\pgfpathlineto{\pgfqpoint{4.491465in}{3.292997in}}%
\pgfpathlineto{\pgfqpoint{4.492307in}{3.290790in}}%
\pgfpathlineto{\pgfqpoint{4.492729in}{3.291777in}}%
\pgfpathlineto{\pgfqpoint{4.493993in}{3.291713in}}%
\pgfpathlineto{\pgfqpoint{4.495678in}{3.291489in}}%
\pgfpathlineto{\pgfqpoint{4.496941in}{3.292138in}}%
\pgfpathlineto{\pgfqpoint{4.497784in}{3.293403in}}%
\pgfpathlineto{\pgfqpoint{4.499469in}{3.298377in}}%
\pgfpathlineto{\pgfqpoint{4.500733in}{3.298331in}}%
\pgfpathlineto{\pgfqpoint{4.501997in}{3.303752in}}%
\pgfpathlineto{\pgfqpoint{4.502418in}{3.302141in}}%
\pgfpathlineto{\pgfqpoint{4.503682in}{3.299535in}}%
\pgfpathlineto{\pgfqpoint{4.505367in}{3.303737in}}%
\pgfpathlineto{\pgfqpoint{4.505788in}{3.302156in}}%
\pgfpathlineto{\pgfqpoint{4.507052in}{3.298538in}}%
\pgfpathlineto{\pgfqpoint{4.508316in}{3.304077in}}%
\pgfpathlineto{\pgfqpoint{4.508737in}{3.302879in}}%
\pgfpathlineto{\pgfqpoint{4.510422in}{3.298090in}}%
\pgfpathlineto{\pgfqpoint{4.510844in}{3.298354in}}%
\pgfpathlineto{\pgfqpoint{4.511265in}{3.298034in}}%
\pgfpathlineto{\pgfqpoint{4.513371in}{3.292003in}}%
\pgfpathlineto{\pgfqpoint{4.514214in}{3.292030in}}%
\pgfpathlineto{\pgfqpoint{4.518848in}{3.277664in}}%
\pgfpathlineto{\pgfqpoint{4.520112in}{3.277141in}}%
\pgfpathlineto{\pgfqpoint{4.523903in}{3.268819in}}%
\pgfpathlineto{\pgfqpoint{4.524746in}{3.269622in}}%
\pgfpathlineto{\pgfqpoint{4.526431in}{3.271552in}}%
\pgfpathlineto{\pgfqpoint{4.528537in}{3.266715in}}%
\pgfpathlineto{\pgfqpoint{4.532750in}{3.252583in}}%
\pgfpathlineto{\pgfqpoint{4.534014in}{3.253327in}}%
\pgfpathlineto{\pgfqpoint{4.538227in}{3.234770in}}%
\pgfpathlineto{\pgfqpoint{4.539491in}{3.232509in}}%
\pgfpathlineto{\pgfqpoint{4.539912in}{3.232897in}}%
\pgfpathlineto{\pgfqpoint{4.542440in}{3.234242in}}%
\pgfpathlineto{\pgfqpoint{4.544125in}{3.234620in}}%
\pgfpathlineto{\pgfqpoint{4.545810in}{3.232968in}}%
\pgfpathlineto{\pgfqpoint{4.546652in}{3.232619in}}%
\pgfpathlineto{\pgfqpoint{4.549601in}{3.237333in}}%
\pgfpathlineto{\pgfqpoint{4.551708in}{3.236128in}}%
\pgfpathlineto{\pgfqpoint{4.556763in}{3.245128in}}%
\pgfpathlineto{\pgfqpoint{4.558027in}{3.245634in}}%
\pgfpathlineto{\pgfqpoint{4.558448in}{3.244283in}}%
\pgfpathlineto{\pgfqpoint{4.558869in}{3.245194in}}%
\pgfpathlineto{\pgfqpoint{4.562240in}{3.251543in}}%
\pgfpathlineto{\pgfqpoint{4.564767in}{3.263584in}}%
\pgfpathlineto{\pgfqpoint{4.565189in}{3.263375in}}%
\pgfpathlineto{\pgfqpoint{4.567295in}{3.267789in}}%
\pgfpathlineto{\pgfqpoint{4.570665in}{3.288711in}}%
\pgfpathlineto{\pgfqpoint{4.572772in}{3.303247in}}%
\pgfpathlineto{\pgfqpoint{4.573614in}{3.304366in}}%
\pgfpathlineto{\pgfqpoint{4.576563in}{3.296723in}}%
\pgfpathlineto{\pgfqpoint{4.576984in}{3.297719in}}%
\pgfpathlineto{\pgfqpoint{4.578669in}{3.298285in}}%
\pgfpathlineto{\pgfqpoint{4.580776in}{3.291655in}}%
\pgfpathlineto{\pgfqpoint{4.589201in}{3.260381in}}%
\pgfpathlineto{\pgfqpoint{4.592150in}{3.260840in}}%
\pgfpathlineto{\pgfqpoint{4.592572in}{3.262340in}}%
\pgfpathlineto{\pgfqpoint{4.592993in}{3.262003in}}%
\pgfpathlineto{\pgfqpoint{4.595521in}{3.256786in}}%
\pgfpathlineto{\pgfqpoint{4.595942in}{3.258086in}}%
\pgfpathlineto{\pgfqpoint{4.598048in}{3.257400in}}%
\pgfpathlineto{\pgfqpoint{4.598891in}{3.256900in}}%
\pgfpathlineto{\pgfqpoint{4.600155in}{3.258490in}}%
\pgfpathlineto{\pgfqpoint{4.601840in}{3.256519in}}%
\pgfpathlineto{\pgfqpoint{4.602682in}{3.259874in}}%
\pgfpathlineto{\pgfqpoint{4.603525in}{3.259043in}}%
\pgfpathlineto{\pgfqpoint{4.605210in}{3.257427in}}%
\pgfpathlineto{\pgfqpoint{4.605631in}{3.258383in}}%
\pgfpathlineto{\pgfqpoint{4.606895in}{3.257860in}}%
\pgfpathlineto{\pgfqpoint{4.608580in}{3.255639in}}%
\pgfpathlineto{\pgfqpoint{4.609001in}{3.257369in}}%
\pgfpathlineto{\pgfqpoint{4.609844in}{3.256342in}}%
\pgfpathlineto{\pgfqpoint{4.611950in}{3.254650in}}%
\pgfpathlineto{\pgfqpoint{4.612372in}{3.255218in}}%
\pgfpathlineto{\pgfqpoint{4.612793in}{3.254393in}}%
\pgfpathlineto{\pgfqpoint{4.614899in}{3.249415in}}%
\pgfpathlineto{\pgfqpoint{4.615321in}{3.250269in}}%
\pgfpathlineto{\pgfqpoint{4.615742in}{3.250766in}}%
\pgfpathlineto{\pgfqpoint{4.616163in}{3.250252in}}%
\pgfpathlineto{\pgfqpoint{4.618270in}{3.249102in}}%
\pgfpathlineto{\pgfqpoint{4.632172in}{3.302104in}}%
\pgfpathlineto{\pgfqpoint{4.632593in}{3.301639in}}%
\pgfpathlineto{\pgfqpoint{4.634278in}{3.299796in}}%
\pgfpathlineto{\pgfqpoint{4.641019in}{3.298988in}}%
\pgfpathlineto{\pgfqpoint{4.646917in}{3.289134in}}%
\pgfpathlineto{\pgfqpoint{4.649444in}{3.288413in}}%
\pgfpathlineto{\pgfqpoint{4.654500in}{3.287945in}}%
\pgfpathlineto{\pgfqpoint{4.659976in}{3.296983in}}%
\pgfpathlineto{\pgfqpoint{4.667138in}{3.298340in}}%
\pgfpathlineto{\pgfqpoint{4.667980in}{3.299602in}}%
\pgfpathlineto{\pgfqpoint{4.672614in}{3.288823in}}%
\pgfpathlineto{\pgfqpoint{4.673878in}{3.289820in}}%
\pgfpathlineto{\pgfqpoint{4.674300in}{3.289361in}}%
\pgfpathlineto{\pgfqpoint{4.675563in}{3.289048in}}%
\pgfpathlineto{\pgfqpoint{4.677670in}{3.276221in}}%
\pgfpathlineto{\pgfqpoint{4.682725in}{3.238205in}}%
\pgfpathlineto{\pgfqpoint{4.685674in}{3.238802in}}%
\pgfpathlineto{\pgfqpoint{4.686517in}{3.236502in}}%
\pgfpathlineto{\pgfqpoint{4.687359in}{3.237309in}}%
\pgfpathlineto{\pgfqpoint{4.689044in}{3.238288in}}%
\pgfpathlineto{\pgfqpoint{4.694100in}{3.241124in}}%
\pgfpathlineto{\pgfqpoint{4.694521in}{3.240602in}}%
\pgfpathlineto{\pgfqpoint{4.694942in}{3.237587in}}%
\pgfpathlineto{\pgfqpoint{4.695785in}{3.238935in}}%
\pgfpathlineto{\pgfqpoint{4.697891in}{3.240023in}}%
\pgfpathlineto{\pgfqpoint{4.698312in}{3.239112in}}%
\pgfpathlineto{\pgfqpoint{4.698734in}{3.239416in}}%
\pgfpathlineto{\pgfqpoint{4.699576in}{3.240321in}}%
\pgfpathlineto{\pgfqpoint{4.700419in}{3.238059in}}%
\pgfpathlineto{\pgfqpoint{4.701261in}{3.238387in}}%
\pgfpathlineto{\pgfqpoint{4.702525in}{3.238532in}}%
\pgfpathlineto{\pgfqpoint{4.703789in}{3.242625in}}%
\pgfpathlineto{\pgfqpoint{4.707159in}{3.273843in}}%
\pgfpathlineto{\pgfqpoint{4.709266in}{3.287813in}}%
\pgfpathlineto{\pgfqpoint{4.711793in}{3.289460in}}%
\pgfpathlineto{\pgfqpoint{4.712215in}{3.288585in}}%
\pgfpathlineto{\pgfqpoint{4.712636in}{3.289263in}}%
\pgfpathlineto{\pgfqpoint{4.713478in}{3.290296in}}%
\pgfpathlineto{\pgfqpoint{4.714321in}{3.287939in}}%
\pgfpathlineto{\pgfqpoint{4.714742in}{3.288261in}}%
\pgfpathlineto{\pgfqpoint{4.716849in}{3.288454in}}%
\pgfpathlineto{\pgfqpoint{4.718534in}{3.288466in}}%
\pgfpathlineto{\pgfqpoint{4.721061in}{3.288443in}}%
\pgfpathlineto{\pgfqpoint{4.739598in}{3.290517in}}%
\pgfpathlineto{\pgfqpoint{4.740019in}{3.289266in}}%
\pgfpathlineto{\pgfqpoint{4.740862in}{3.290070in}}%
\pgfpathlineto{\pgfqpoint{4.741283in}{3.290203in}}%
\pgfpathlineto{\pgfqpoint{4.742125in}{3.288580in}}%
\pgfpathlineto{\pgfqpoint{4.742547in}{3.288963in}}%
\pgfpathlineto{\pgfqpoint{4.750551in}{3.289174in}}%
\pgfpathlineto{\pgfqpoint{4.751815in}{3.290441in}}%
\pgfpathlineto{\pgfqpoint{4.752236in}{3.289882in}}%
\pgfpathlineto{\pgfqpoint{4.752657in}{3.290784in}}%
\pgfpathlineto{\pgfqpoint{4.755606in}{3.290149in}}%
\pgfpathlineto{\pgfqpoint{4.763189in}{3.280870in}}%
\pgfpathlineto{\pgfqpoint{4.765296in}{3.278074in}}%
\pgfpathlineto{\pgfqpoint{4.766559in}{3.277933in}}%
\pgfpathlineto{\pgfqpoint{4.770772in}{3.276584in}}%
\pgfpathlineto{\pgfqpoint{4.773300in}{3.274494in}}%
\pgfpathlineto{\pgfqpoint{4.776249in}{3.273276in}}%
\pgfpathlineto{\pgfqpoint{4.782989in}{3.273927in}}%
\pgfpathlineto{\pgfqpoint{4.784674in}{3.277223in}}%
\pgfpathlineto{\pgfqpoint{4.785096in}{3.276976in}}%
\pgfpathlineto{\pgfqpoint{4.785517in}{3.276745in}}%
\pgfpathlineto{\pgfqpoint{4.788466in}{3.255854in}}%
\pgfpathlineto{\pgfqpoint{4.800262in}{3.169154in}}%
\pgfpathlineto{\pgfqpoint{4.804475in}{3.168288in}}%
\pgfpathlineto{\pgfqpoint{4.809530in}{3.162582in}}%
\pgfpathlineto{\pgfqpoint{4.809951in}{3.161952in}}%
\pgfpathlineto{\pgfqpoint{4.812058in}{3.175159in}}%
\pgfpathlineto{\pgfqpoint{4.823853in}{3.272037in}}%
\pgfpathlineto{\pgfqpoint{4.824275in}{3.271260in}}%
\pgfpathlineto{\pgfqpoint{4.824696in}{3.271516in}}%
\pgfpathlineto{\pgfqpoint{4.826802in}{3.273828in}}%
\pgfpathlineto{\pgfqpoint{4.828487in}{3.275248in}}%
\pgfpathlineto{\pgfqpoint{4.829751in}{3.274402in}}%
\pgfpathlineto{\pgfqpoint{4.830172in}{3.275354in}}%
\pgfpathlineto{\pgfqpoint{4.830594in}{3.274500in}}%
\pgfpathlineto{\pgfqpoint{4.831436in}{3.274283in}}%
\pgfpathlineto{\pgfqpoint{4.831858in}{3.275126in}}%
\pgfpathlineto{\pgfqpoint{4.832279in}{3.274270in}}%
\pgfpathlineto{\pgfqpoint{4.833121in}{3.274347in}}%
\pgfpathlineto{\pgfqpoint{4.833543in}{3.275204in}}%
\pgfpathlineto{\pgfqpoint{4.833964in}{3.274358in}}%
\pgfpathlineto{\pgfqpoint{4.834807in}{3.274699in}}%
\pgfpathlineto{\pgfqpoint{4.835228in}{3.275026in}}%
\pgfpathlineto{\pgfqpoint{4.835649in}{3.274177in}}%
\pgfpathlineto{\pgfqpoint{4.837755in}{3.274087in}}%
\pgfpathlineto{\pgfqpoint{4.838177in}{3.275055in}}%
\pgfpathlineto{\pgfqpoint{4.839019in}{3.274301in}}%
\pgfpathlineto{\pgfqpoint{4.841968in}{3.275070in}}%
\pgfpathlineto{\pgfqpoint{4.843653in}{3.278212in}}%
\pgfpathlineto{\pgfqpoint{4.844075in}{3.277917in}}%
\pgfpathlineto{\pgfqpoint{4.844917in}{3.276262in}}%
\pgfpathlineto{\pgfqpoint{4.856713in}{3.190922in}}%
\pgfpathlineto{\pgfqpoint{4.858819in}{3.180190in}}%
\pgfpathlineto{\pgfqpoint{4.860083in}{3.179159in}}%
\pgfpathlineto{\pgfqpoint{4.862190in}{3.179598in}}%
\pgfpathlineto{\pgfqpoint{4.863875in}{3.181469in}}%
\pgfpathlineto{\pgfqpoint{4.866402in}{3.182956in}}%
\pgfpathlineto{\pgfqpoint{4.868509in}{3.185543in}}%
\pgfpathlineto{\pgfqpoint{4.873985in}{3.199406in}}%
\pgfpathlineto{\pgfqpoint{4.879462in}{3.224530in}}%
\pgfpathlineto{\pgfqpoint{4.885360in}{3.245083in}}%
\pgfpathlineto{\pgfqpoint{4.888730in}{3.257528in}}%
\pgfpathlineto{\pgfqpoint{4.893785in}{3.259989in}}%
\pgfpathlineto{\pgfqpoint{4.895892in}{3.255437in}}%
\pgfpathlineto{\pgfqpoint{4.902632in}{3.231541in}}%
\pgfpathlineto{\pgfqpoint{4.908109in}{3.230872in}}%
\pgfpathlineto{\pgfqpoint{4.909794in}{3.235871in}}%
\pgfpathlineto{\pgfqpoint{4.916956in}{3.258799in}}%
\pgfpathlineto{\pgfqpoint{4.919483in}{3.259688in}}%
\pgfpathlineto{\pgfqpoint{4.921169in}{3.260463in}}%
\pgfpathlineto{\pgfqpoint{4.922432in}{3.258418in}}%
\pgfpathlineto{\pgfqpoint{4.932122in}{3.231539in}}%
\pgfpathlineto{\pgfqpoint{4.932964in}{3.230769in}}%
\pgfpathlineto{\pgfqpoint{4.935071in}{3.227768in}}%
\pgfpathlineto{\pgfqpoint{4.942232in}{3.233771in}}%
\pgfpathlineto{\pgfqpoint{4.943918in}{3.234537in}}%
\pgfpathlineto{\pgfqpoint{4.948130in}{3.236269in}}%
\pgfpathlineto{\pgfqpoint{4.952764in}{3.249320in}}%
\pgfpathlineto{\pgfqpoint{4.962454in}{3.275327in}}%
\pgfpathlineto{\pgfqpoint{4.962875in}{3.275086in}}%
\pgfpathlineto{\pgfqpoint{4.967930in}{3.275492in}}%
\pgfpathlineto{\pgfqpoint{4.969194in}{3.275564in}}%
\pgfpathlineto{\pgfqpoint{4.970879in}{3.274697in}}%
\pgfpathlineto{\pgfqpoint{4.972143in}{3.275722in}}%
\pgfpathlineto{\pgfqpoint{4.974250in}{3.273605in}}%
\pgfpathlineto{\pgfqpoint{4.975092in}{3.276023in}}%
\pgfpathlineto{\pgfqpoint{4.975513in}{3.275936in}}%
\pgfpathlineto{\pgfqpoint{4.977620in}{3.274696in}}%
\pgfpathlineto{\pgfqpoint{4.978041in}{3.276692in}}%
\pgfpathlineto{\pgfqpoint{4.978884in}{3.276320in}}%
\pgfpathlineto{\pgfqpoint{4.980569in}{3.274411in}}%
\pgfpathlineto{\pgfqpoint{4.980990in}{3.274763in}}%
\pgfpathlineto{\pgfqpoint{4.981411in}{3.276627in}}%
\pgfpathlineto{\pgfqpoint{4.982254in}{3.276120in}}%
\pgfpathlineto{\pgfqpoint{4.983939in}{3.274176in}}%
\pgfpathlineto{\pgfqpoint{4.984782in}{3.276468in}}%
\pgfpathlineto{\pgfqpoint{4.985624in}{3.275833in}}%
\pgfpathlineto{\pgfqpoint{4.987309in}{3.273906in}}%
\pgfpathlineto{\pgfqpoint{4.988152in}{3.276122in}}%
\pgfpathlineto{\pgfqpoint{4.988573in}{3.275917in}}%
\pgfpathlineto{\pgfqpoint{4.990679in}{3.274149in}}%
\pgfpathlineto{\pgfqpoint{4.991101in}{3.276137in}}%
\pgfpathlineto{\pgfqpoint{4.991943in}{3.275898in}}%
\pgfpathlineto{\pgfqpoint{4.993628in}{3.273978in}}%
\pgfpathlineto{\pgfqpoint{4.994050in}{3.274299in}}%
\pgfpathlineto{\pgfqpoint{4.994471in}{3.276332in}}%
\pgfpathlineto{\pgfqpoint{4.995314in}{3.275717in}}%
\pgfpathlineto{\pgfqpoint{4.996999in}{3.274015in}}%
\pgfpathlineto{\pgfqpoint{4.997841in}{3.275841in}}%
\pgfpathlineto{\pgfqpoint{4.998684in}{3.275142in}}%
\pgfpathlineto{\pgfqpoint{5.000369in}{3.273269in}}%
\pgfpathlineto{\pgfqpoint{5.001211in}{3.275651in}}%
\pgfpathlineto{\pgfqpoint{5.001633in}{3.275419in}}%
\pgfpathlineto{\pgfqpoint{5.003739in}{3.273642in}}%
\pgfpathlineto{\pgfqpoint{5.004582in}{3.275669in}}%
\pgfpathlineto{\pgfqpoint{5.005003in}{3.275526in}}%
\pgfpathlineto{\pgfqpoint{5.007109in}{3.274097in}}%
\pgfpathlineto{\pgfqpoint{5.007531in}{3.276130in}}%
\pgfpathlineto{\pgfqpoint{5.008373in}{3.275606in}}%
\pgfpathlineto{\pgfqpoint{5.010058in}{3.273807in}}%
\pgfpathlineto{\pgfqpoint{5.011743in}{3.276095in}}%
\pgfpathlineto{\pgfqpoint{5.013428in}{3.274416in}}%
\pgfpathlineto{\pgfqpoint{5.014271in}{3.276826in}}%
\pgfpathlineto{\pgfqpoint{5.014692in}{3.276569in}}%
\pgfpathlineto{\pgfqpoint{5.016799in}{3.274699in}}%
\pgfpathlineto{\pgfqpoint{5.017641in}{3.276356in}}%
\pgfpathlineto{\pgfqpoint{5.018063in}{3.276171in}}%
\pgfpathlineto{\pgfqpoint{5.020169in}{3.274806in}}%
\pgfpathlineto{\pgfqpoint{5.020590in}{3.276928in}}%
\pgfpathlineto{\pgfqpoint{5.021433in}{3.276626in}}%
\pgfpathlineto{\pgfqpoint{5.023118in}{3.274907in}}%
\pgfpathlineto{\pgfqpoint{5.023960in}{3.277192in}}%
\pgfpathlineto{\pgfqpoint{5.024803in}{3.276579in}}%
\pgfpathlineto{\pgfqpoint{5.026488in}{3.274831in}}%
\pgfpathlineto{\pgfqpoint{5.028173in}{3.276884in}}%
\pgfpathlineto{\pgfqpoint{5.029437in}{3.276708in}}%
\pgfpathlineto{\pgfqpoint{5.030280in}{3.279155in}}%
\pgfpathlineto{\pgfqpoint{5.030701in}{3.278368in}}%
\pgfpathlineto{\pgfqpoint{5.032807in}{3.272779in}}%
\pgfpathlineto{\pgfqpoint{5.033229in}{3.273564in}}%
\pgfpathlineto{\pgfqpoint{5.033650in}{3.273880in}}%
\pgfpathlineto{\pgfqpoint{5.035756in}{3.265453in}}%
\pgfpathlineto{\pgfqpoint{5.036177in}{3.265772in}}%
\pgfpathlineto{\pgfqpoint{5.037020in}{3.266083in}}%
\pgfpathlineto{\pgfqpoint{5.038705in}{3.257994in}}%
\pgfpathlineto{\pgfqpoint{5.039126in}{3.259342in}}%
\pgfpathlineto{\pgfqpoint{5.040390in}{3.262875in}}%
\pgfpathlineto{\pgfqpoint{5.041654in}{3.259115in}}%
\pgfpathlineto{\pgfqpoint{5.042075in}{3.259670in}}%
\pgfpathlineto{\pgfqpoint{5.043339in}{3.257831in}}%
\pgfpathlineto{\pgfqpoint{5.044603in}{3.254062in}}%
\pgfpathlineto{\pgfqpoint{5.045024in}{3.254630in}}%
\pgfpathlineto{\pgfqpoint{5.045867in}{3.255971in}}%
\pgfpathlineto{\pgfqpoint{5.046288in}{3.255593in}}%
\pgfpathlineto{\pgfqpoint{5.047552in}{3.252906in}}%
\pgfpathlineto{\pgfqpoint{5.047973in}{3.253695in}}%
\pgfpathlineto{\pgfqpoint{5.049237in}{3.255571in}}%
\pgfpathlineto{\pgfqpoint{5.050080in}{3.252329in}}%
\pgfpathlineto{\pgfqpoint{5.050501in}{3.252581in}}%
\pgfpathlineto{\pgfqpoint{5.052186in}{3.260681in}}%
\pgfpathlineto{\pgfqpoint{5.052607in}{3.259386in}}%
\pgfpathlineto{\pgfqpoint{5.053450in}{3.257652in}}%
\pgfpathlineto{\pgfqpoint{5.053871in}{3.258173in}}%
\pgfpathlineto{\pgfqpoint{5.055135in}{3.256369in}}%
\pgfpathlineto{\pgfqpoint{5.056399in}{3.253026in}}%
\pgfpathlineto{\pgfqpoint{5.058084in}{3.255506in}}%
\pgfpathlineto{\pgfqpoint{5.059348in}{3.252815in}}%
\pgfpathlineto{\pgfqpoint{5.059769in}{3.253656in}}%
\pgfpathlineto{\pgfqpoint{5.060612in}{3.255555in}}%
\pgfpathlineto{\pgfqpoint{5.061033in}{3.255354in}}%
\pgfpathlineto{\pgfqpoint{5.062297in}{3.251965in}}%
\pgfpathlineto{\pgfqpoint{5.063982in}{3.259957in}}%
\pgfpathlineto{\pgfqpoint{5.064403in}{3.258688in}}%
\pgfpathlineto{\pgfqpoint{5.065246in}{3.256885in}}%
\pgfpathlineto{\pgfqpoint{5.065667in}{3.257075in}}%
\pgfpathlineto{\pgfqpoint{5.066510in}{3.258740in}}%
\pgfpathlineto{\pgfqpoint{5.066931in}{3.258167in}}%
\pgfpathlineto{\pgfqpoint{5.068616in}{3.254448in}}%
\pgfpathlineto{\pgfqpoint{5.069037in}{3.254731in}}%
\pgfpathlineto{\pgfqpoint{5.070722in}{3.260168in}}%
\pgfpathlineto{\pgfqpoint{5.072407in}{3.259416in}}%
\pgfpathlineto{\pgfqpoint{5.074093in}{3.259738in}}%
\pgfpathlineto{\pgfqpoint{5.079569in}{3.257818in}}%
\pgfpathlineto{\pgfqpoint{5.082097in}{3.258702in}}%
\pgfpathlineto{\pgfqpoint{5.084203in}{3.256607in}}%
\pgfpathlineto{\pgfqpoint{5.085888in}{3.256836in}}%
\pgfpathlineto{\pgfqpoint{5.086731in}{3.256296in}}%
\pgfpathlineto{\pgfqpoint{5.087152in}{3.256625in}}%
\pgfpathlineto{\pgfqpoint{5.089259in}{3.257722in}}%
\pgfpathlineto{\pgfqpoint{5.090101in}{3.254412in}}%
\pgfpathlineto{\pgfqpoint{5.090522in}{3.254571in}}%
\pgfpathlineto{\pgfqpoint{5.092207in}{3.252320in}}%
\pgfpathlineto{\pgfqpoint{5.093050in}{3.248826in}}%
\pgfpathlineto{\pgfqpoint{5.097263in}{3.231739in}}%
\pgfpathlineto{\pgfqpoint{5.100212in}{3.223553in}}%
\pgfpathlineto{\pgfqpoint{5.102739in}{3.211624in}}%
\pgfpathlineto{\pgfqpoint{5.103161in}{3.212953in}}%
\pgfpathlineto{\pgfqpoint{5.105267in}{3.206489in}}%
\pgfpathlineto{\pgfqpoint{5.107795in}{3.199810in}}%
\pgfpathlineto{\pgfqpoint{5.109059in}{3.198852in}}%
\pgfpathlineto{\pgfqpoint{5.109901in}{3.202783in}}%
\pgfpathlineto{\pgfqpoint{5.110744in}{3.201455in}}%
\pgfpathlineto{\pgfqpoint{5.112008in}{3.200294in}}%
\pgfpathlineto{\pgfqpoint{5.112429in}{3.199147in}}%
\pgfpathlineto{\pgfqpoint{5.112850in}{3.199858in}}%
\pgfpathlineto{\pgfqpoint{5.114114in}{3.199798in}}%
\pgfpathlineto{\pgfqpoint{5.115378in}{3.198854in}}%
\pgfpathlineto{\pgfqpoint{5.115799in}{3.199360in}}%
\pgfpathlineto{\pgfqpoint{5.116642in}{3.202504in}}%
\pgfpathlineto{\pgfqpoint{5.117063in}{3.201595in}}%
\pgfpathlineto{\pgfqpoint{5.118327in}{3.200592in}}%
\pgfpathlineto{\pgfqpoint{5.119169in}{3.199417in}}%
\pgfpathlineto{\pgfqpoint{5.119591in}{3.200505in}}%
\pgfpathlineto{\pgfqpoint{5.122118in}{3.198842in}}%
\pgfpathlineto{\pgfqpoint{5.122961in}{3.202692in}}%
\pgfpathlineto{\pgfqpoint{5.123803in}{3.201516in}}%
\pgfpathlineto{\pgfqpoint{5.128859in}{3.198869in}}%
\pgfpathlineto{\pgfqpoint{5.130544in}{3.202810in}}%
\pgfpathlineto{\pgfqpoint{5.132229in}{3.209424in}}%
\pgfpathlineto{\pgfqpoint{5.133914in}{3.213330in}}%
\pgfpathlineto{\pgfqpoint{5.135178in}{3.215398in}}%
\pgfpathlineto{\pgfqpoint{5.135599in}{3.216125in}}%
\pgfpathlineto{\pgfqpoint{5.136020in}{3.215269in}}%
\pgfpathlineto{\pgfqpoint{5.138548in}{3.207467in}}%
\pgfpathlineto{\pgfqpoint{5.141497in}{3.207397in}}%
\pgfpathlineto{\pgfqpoint{5.142340in}{3.208066in}}%
\pgfpathlineto{\pgfqpoint{5.147395in}{3.186743in}}%
\pgfpathlineto{\pgfqpoint{5.149501in}{3.186776in}}%
\pgfpathlineto{\pgfqpoint{5.151608in}{3.190841in}}%
\pgfpathlineto{\pgfqpoint{5.152872in}{3.191767in}}%
\pgfpathlineto{\pgfqpoint{5.154978in}{3.195346in}}%
\pgfpathlineto{\pgfqpoint{5.155820in}{3.195270in}}%
\pgfpathlineto{\pgfqpoint{5.161297in}{3.211588in}}%
\pgfpathlineto{\pgfqpoint{5.162561in}{3.210795in}}%
\pgfpathlineto{\pgfqpoint{5.168038in}{3.200531in}}%
\pgfpathlineto{\pgfqpoint{5.169723in}{3.199806in}}%
\pgfpathlineto{\pgfqpoint{5.174778in}{3.184676in}}%
\pgfpathlineto{\pgfqpoint{5.175199in}{3.184812in}}%
\pgfpathlineto{\pgfqpoint{5.176042in}{3.185612in}}%
\pgfpathlineto{\pgfqpoint{5.178148in}{3.194054in}}%
\pgfpathlineto{\pgfqpoint{5.179412in}{3.194232in}}%
\pgfpathlineto{\pgfqpoint{5.181940in}{3.201782in}}%
\pgfpathlineto{\pgfqpoint{5.182782in}{3.202084in}}%
\pgfpathlineto{\pgfqpoint{5.188259in}{3.230552in}}%
\pgfpathlineto{\pgfqpoint{5.188680in}{3.230387in}}%
\pgfpathlineto{\pgfqpoint{5.189523in}{3.229557in}}%
\pgfpathlineto{\pgfqpoint{5.191629in}{3.221168in}}%
\pgfpathlineto{\pgfqpoint{5.192050in}{3.221329in}}%
\pgfpathlineto{\pgfqpoint{5.192893in}{3.220880in}}%
\pgfpathlineto{\pgfqpoint{5.195421in}{3.213423in}}%
\pgfpathlineto{\pgfqpoint{5.196263in}{3.213045in}}%
\pgfpathlineto{\pgfqpoint{5.201740in}{3.184477in}}%
\pgfpathlineto{\pgfqpoint{5.202582in}{3.184669in}}%
\pgfpathlineto{\pgfqpoint{5.205110in}{3.193737in}}%
\pgfpathlineto{\pgfqpoint{5.205531in}{3.193582in}}%
\pgfpathlineto{\pgfqpoint{5.205953in}{3.193456in}}%
\pgfpathlineto{\pgfqpoint{5.212693in}{3.219162in}}%
\pgfpathlineto{\pgfqpoint{5.214799in}{3.230595in}}%
\pgfpathlineto{\pgfqpoint{5.216485in}{3.228593in}}%
\pgfpathlineto{\pgfqpoint{5.218591in}{3.220992in}}%
\pgfpathlineto{\pgfqpoint{5.219855in}{3.219966in}}%
\pgfpathlineto{\pgfqpoint{5.221961in}{3.213212in}}%
\pgfpathlineto{\pgfqpoint{5.223225in}{3.212396in}}%
\pgfpathlineto{\pgfqpoint{5.228280in}{3.186779in}}%
\pgfpathlineto{\pgfqpoint{5.228702in}{3.187861in}}%
\pgfpathlineto{\pgfqpoint{5.235021in}{3.201047in}}%
\pgfpathlineto{\pgfqpoint{5.236285in}{3.203972in}}%
\pgfpathlineto{\pgfqpoint{5.238391in}{3.208606in}}%
\pgfpathlineto{\pgfqpoint{5.242183in}{3.228015in}}%
\pgfpathlineto{\pgfqpoint{5.245131in}{3.224362in}}%
\pgfpathlineto{\pgfqpoint{5.248080in}{3.221174in}}%
\pgfpathlineto{\pgfqpoint{5.252293in}{3.249102in}}%
\pgfpathlineto{\pgfqpoint{5.252714in}{3.248076in}}%
\pgfpathlineto{\pgfqpoint{5.253978in}{3.243217in}}%
\pgfpathlineto{\pgfqpoint{5.255242in}{3.236271in}}%
\pgfpathlineto{\pgfqpoint{5.255663in}{3.236629in}}%
\pgfpathlineto{\pgfqpoint{5.259034in}{3.245736in}}%
\pgfpathlineto{\pgfqpoint{5.261983in}{3.245480in}}%
\pgfpathlineto{\pgfqpoint{5.264089in}{3.236123in}}%
\pgfpathlineto{\pgfqpoint{5.267459in}{3.218242in}}%
\pgfpathlineto{\pgfqpoint{5.267880in}{3.218662in}}%
\pgfpathlineto{\pgfqpoint{5.270408in}{3.228042in}}%
\pgfpathlineto{\pgfqpoint{5.271672in}{3.233189in}}%
\pgfpathlineto{\pgfqpoint{5.276727in}{3.259147in}}%
\pgfpathlineto{\pgfqpoint{5.278412in}{3.265593in}}%
\pgfpathlineto{\pgfqpoint{5.282204in}{3.285900in}}%
\pgfpathlineto{\pgfqpoint{5.284732in}{3.287352in}}%
\pgfpathlineto{\pgfqpoint{5.286417in}{3.288851in}}%
\pgfpathlineto{\pgfqpoint{5.288102in}{3.286767in}}%
\pgfpathlineto{\pgfqpoint{5.291472in}{3.266681in}}%
\pgfpathlineto{\pgfqpoint{5.296527in}{3.238334in}}%
\pgfpathlineto{\pgfqpoint{5.298634in}{3.237794in}}%
\pgfpathlineto{\pgfqpoint{5.301161in}{3.223645in}}%
\pgfpathlineto{\pgfqpoint{5.301583in}{3.225211in}}%
\pgfpathlineto{\pgfqpoint{5.303268in}{3.235819in}}%
\pgfpathlineto{\pgfqpoint{5.307902in}{3.264924in}}%
\pgfpathlineto{\pgfqpoint{5.311272in}{3.256095in}}%
\pgfpathlineto{\pgfqpoint{5.313379in}{3.248554in}}%
\pgfpathlineto{\pgfqpoint{5.314642in}{3.249426in}}%
\pgfpathlineto{\pgfqpoint{5.316749in}{3.240495in}}%
\pgfpathlineto{\pgfqpoint{5.318434in}{3.229106in}}%
\pgfpathlineto{\pgfqpoint{5.320119in}{3.219958in}}%
\pgfpathlineto{\pgfqpoint{5.320962in}{3.219468in}}%
\pgfpathlineto{\pgfqpoint{5.323489in}{3.226127in}}%
\pgfpathlineto{\pgfqpoint{5.326438in}{3.237375in}}%
\pgfpathlineto{\pgfqpoint{5.328545in}{3.241214in}}%
\pgfpathlineto{\pgfqpoint{5.330230in}{3.238088in}}%
\pgfpathlineto{\pgfqpoint{5.330651in}{3.238891in}}%
\pgfpathlineto{\pgfqpoint{5.333600in}{3.247403in}}%
\pgfpathlineto{\pgfqpoint{5.335706in}{3.256470in}}%
\pgfpathlineto{\pgfqpoint{5.336128in}{3.255796in}}%
\pgfpathlineto{\pgfqpoint{5.338234in}{3.251903in}}%
\pgfpathlineto{\pgfqpoint{5.339076in}{3.252646in}}%
\pgfpathlineto{\pgfqpoint{5.339919in}{3.253217in}}%
\pgfpathlineto{\pgfqpoint{5.347081in}{3.271026in}}%
\pgfpathlineto{\pgfqpoint{5.348345in}{3.271875in}}%
\pgfpathlineto{\pgfqpoint{5.350872in}{3.274972in}}%
\pgfpathlineto{\pgfqpoint{5.351715in}{3.275466in}}%
\pgfpathlineto{\pgfqpoint{5.357191in}{3.281624in}}%
\pgfpathlineto{\pgfqpoint{5.358034in}{3.281824in}}%
\pgfpathlineto{\pgfqpoint{5.359298in}{3.282925in}}%
\pgfpathlineto{\pgfqpoint{5.360140in}{3.283364in}}%
\pgfpathlineto{\pgfqpoint{5.360562in}{3.282777in}}%
\pgfpathlineto{\pgfqpoint{5.360983in}{3.282177in}}%
\pgfpathlineto{\pgfqpoint{5.361404in}{3.283389in}}%
\pgfpathlineto{\pgfqpoint{5.362247in}{3.283939in}}%
\pgfpathlineto{\pgfqpoint{5.362668in}{3.283472in}}%
\pgfpathlineto{\pgfqpoint{5.363511in}{3.283183in}}%
\pgfpathlineto{\pgfqpoint{5.366460in}{3.278290in}}%
\pgfpathlineto{\pgfqpoint{5.370251in}{3.267500in}}%
\pgfpathlineto{\pgfqpoint{5.378677in}{3.258324in}}%
\pgfpathlineto{\pgfqpoint{5.379098in}{3.258850in}}%
\pgfpathlineto{\pgfqpoint{5.381204in}{3.262749in}}%
\pgfpathlineto{\pgfqpoint{5.382047in}{3.263408in}}%
\pgfpathlineto{\pgfqpoint{5.382889in}{3.265204in}}%
\pgfpathlineto{\pgfqpoint{5.383311in}{3.264257in}}%
\pgfpathlineto{\pgfqpoint{5.387523in}{3.265143in}}%
\pgfpathlineto{\pgfqpoint{5.397634in}{3.293437in}}%
\pgfpathlineto{\pgfqpoint{5.398477in}{3.295229in}}%
\pgfpathlineto{\pgfqpoint{5.401004in}{3.301298in}}%
\pgfpathlineto{\pgfqpoint{5.402689in}{3.299631in}}%
\pgfpathlineto{\pgfqpoint{5.403953in}{3.298570in}}%
\pgfpathlineto{\pgfqpoint{5.405638in}{3.289215in}}%
\pgfpathlineto{\pgfqpoint{5.410694in}{3.250478in}}%
\pgfpathlineto{\pgfqpoint{5.411958in}{3.243597in}}%
\pgfpathlineto{\pgfqpoint{5.413643in}{3.239334in}}%
\pgfpathlineto{\pgfqpoint{5.418698in}{3.224456in}}%
\pgfpathlineto{\pgfqpoint{5.420383in}{3.239960in}}%
\pgfpathlineto{\pgfqpoint{5.423753in}{3.264418in}}%
\pgfpathlineto{\pgfqpoint{5.425438in}{3.272593in}}%
\pgfpathlineto{\pgfqpoint{5.427124in}{3.276884in}}%
\pgfpathlineto{\pgfqpoint{5.428809in}{3.285946in}}%
\pgfpathlineto{\pgfqpoint{5.432600in}{3.299571in}}%
\pgfpathlineto{\pgfqpoint{5.444817in}{3.299668in}}%
\pgfpathlineto{\pgfqpoint{5.454928in}{3.299578in}}%
\pgfpathlineto{\pgfqpoint{5.456192in}{3.295263in}}%
\pgfpathlineto{\pgfqpoint{5.459141in}{3.267445in}}%
\pgfpathlineto{\pgfqpoint{5.469251in}{3.170921in}}%
\pgfpathlineto{\pgfqpoint{5.470515in}{3.169336in}}%
\pgfpathlineto{\pgfqpoint{5.472200in}{3.175467in}}%
\pgfpathlineto{\pgfqpoint{5.475571in}{3.196116in}}%
\pgfpathlineto{\pgfqpoint{5.483575in}{3.240100in}}%
\pgfpathlineto{\pgfqpoint{5.487366in}{3.248924in}}%
\pgfpathlineto{\pgfqpoint{5.492843in}{3.258898in}}%
\pgfpathlineto{\pgfqpoint{5.495371in}{3.257032in}}%
\pgfpathlineto{\pgfqpoint{5.496213in}{3.256181in}}%
\pgfpathlineto{\pgfqpoint{5.499583in}{3.252130in}}%
\pgfpathlineto{\pgfqpoint{5.502111in}{3.248974in}}%
\pgfpathlineto{\pgfqpoint{5.506745in}{3.241956in}}%
\pgfpathlineto{\pgfqpoint{5.509273in}{3.243467in}}%
\pgfpathlineto{\pgfqpoint{5.513907in}{3.250739in}}%
\pgfpathlineto{\pgfqpoint{5.517698in}{3.259771in}}%
\pgfpathlineto{\pgfqpoint{5.519805in}{3.263970in}}%
\pgfpathlineto{\pgfqpoint{5.520647in}{3.264487in}}%
\pgfpathlineto{\pgfqpoint{5.523596in}{3.272141in}}%
\pgfpathlineto{\pgfqpoint{5.524018in}{3.272043in}}%
\pgfpathlineto{\pgfqpoint{5.525703in}{3.263839in}}%
\pgfpathlineto{\pgfqpoint{5.532022in}{3.223103in}}%
\pgfpathlineto{\pgfqpoint{5.532443in}{3.224185in}}%
\pgfpathlineto{\pgfqpoint{5.533286in}{3.222180in}}%
\pgfpathlineto{\pgfqpoint{5.533707in}{3.222462in}}%
\pgfpathlineto{\pgfqpoint{5.534128in}{3.223475in}}%
\pgfpathlineto{\pgfqpoint{5.534550in}{3.222333in}}%
\pgfpathlineto{\pgfqpoint{5.534971in}{3.221987in}}%
\pgfpathlineto{\pgfqpoint{5.535392in}{3.222869in}}%
\pgfpathlineto{\pgfqpoint{5.535813in}{3.225002in}}%
\pgfpathlineto{\pgfqpoint{5.536235in}{3.223622in}}%
\pgfpathlineto{\pgfqpoint{5.538341in}{3.218860in}}%
\pgfpathlineto{\pgfqpoint{5.539605in}{3.218420in}}%
\pgfpathlineto{\pgfqpoint{5.548873in}{3.217532in}}%
\pgfpathlineto{\pgfqpoint{5.549294in}{3.216243in}}%
\pgfpathlineto{\pgfqpoint{5.549716in}{3.217825in}}%
\pgfpathlineto{\pgfqpoint{5.552664in}{3.223104in}}%
\pgfpathlineto{\pgfqpoint{5.555192in}{3.241416in}}%
\pgfpathlineto{\pgfqpoint{5.556035in}{3.242707in}}%
\pgfpathlineto{\pgfqpoint{5.557299in}{3.252912in}}%
\pgfpathlineto{\pgfqpoint{5.557720in}{3.251733in}}%
\pgfpathlineto{\pgfqpoint{5.563196in}{3.232410in}}%
\pgfpathlineto{\pgfqpoint{5.564460in}{3.234393in}}%
\pgfpathlineto{\pgfqpoint{5.565724in}{3.237238in}}%
\pgfpathlineto{\pgfqpoint{5.566988in}{3.231717in}}%
\pgfpathlineto{\pgfqpoint{5.567830in}{3.233103in}}%
\pgfpathlineto{\pgfqpoint{5.569516in}{3.236303in}}%
\pgfpathlineto{\pgfqpoint{5.570358in}{3.231646in}}%
\pgfpathlineto{\pgfqpoint{5.570779in}{3.232913in}}%
\pgfpathlineto{\pgfqpoint{5.572465in}{3.235678in}}%
\pgfpathlineto{\pgfqpoint{5.572886in}{3.236930in}}%
\pgfpathlineto{\pgfqpoint{5.573728in}{3.231502in}}%
\pgfpathlineto{\pgfqpoint{5.574150in}{3.232588in}}%
\pgfpathlineto{\pgfqpoint{5.576256in}{3.236958in}}%
\pgfpathlineto{\pgfqpoint{5.576677in}{3.235810in}}%
\pgfpathlineto{\pgfqpoint{5.577520in}{3.231450in}}%
\pgfpathlineto{\pgfqpoint{5.577941in}{3.232544in}}%
\pgfpathlineto{\pgfqpoint{5.578784in}{3.234083in}}%
\pgfpathlineto{\pgfqpoint{5.580048in}{3.235186in}}%
\pgfpathlineto{\pgfqpoint{5.582997in}{3.216678in}}%
\pgfpathlineto{\pgfqpoint{5.583418in}{3.219857in}}%
\pgfpathlineto{\pgfqpoint{5.583839in}{3.219669in}}%
\pgfpathlineto{\pgfqpoint{5.586367in}{3.211835in}}%
\pgfpathlineto{\pgfqpoint{5.586788in}{3.211624in}}%
\pgfpathlineto{\pgfqpoint{5.588473in}{3.198215in}}%
\pgfpathlineto{\pgfqpoint{5.589316in}{3.194492in}}%
\pgfpathlineto{\pgfqpoint{5.589737in}{3.196424in}}%
\pgfpathlineto{\pgfqpoint{5.590580in}{3.202240in}}%
\pgfpathlineto{\pgfqpoint{5.591422in}{3.200990in}}%
\pgfpathlineto{\pgfqpoint{5.592686in}{3.202417in}}%
\pgfpathlineto{\pgfqpoint{5.594371in}{3.211963in}}%
\pgfpathlineto{\pgfqpoint{5.596477in}{3.226113in}}%
\pgfpathlineto{\pgfqpoint{5.597320in}{3.225621in}}%
\pgfpathlineto{\pgfqpoint{5.599005in}{3.229703in}}%
\pgfpathlineto{\pgfqpoint{5.602797in}{3.254505in}}%
\pgfpathlineto{\pgfqpoint{5.604482in}{3.252930in}}%
\pgfpathlineto{\pgfqpoint{5.607009in}{3.252124in}}%
\pgfpathlineto{\pgfqpoint{5.615014in}{3.252326in}}%
\pgfpathlineto{\pgfqpoint{5.615014in}{3.252326in}}%
\pgfusepath{stroke}%
\end{pgfscope}%
\begin{pgfscope}%
\pgfsetrectcap%
\pgfsetmiterjoin%
\pgfsetlinewidth{0.803000pt}%
\definecolor{currentstroke}{rgb}{0.000000,0.000000,0.000000}%
\pgfsetstrokecolor{currentstroke}%
\pgfsetdash{}{0pt}%
\pgfpathmoveto{\pgfqpoint{0.885050in}{2.474259in}}%
\pgfpathlineto{\pgfqpoint{0.885050in}{3.760185in}}%
\pgfusepath{stroke}%
\end{pgfscope}%
\begin{pgfscope}%
\pgfsetrectcap%
\pgfsetmiterjoin%
\pgfsetlinewidth{0.803000pt}%
\definecolor{currentstroke}{rgb}{0.000000,0.000000,0.000000}%
\pgfsetstrokecolor{currentstroke}%
\pgfsetdash{}{0pt}%
\pgfpathmoveto{\pgfqpoint{5.840250in}{2.474259in}}%
\pgfpathlineto{\pgfqpoint{5.840250in}{3.760185in}}%
\pgfusepath{stroke}%
\end{pgfscope}%
\begin{pgfscope}%
\pgfsetrectcap%
\pgfsetmiterjoin%
\pgfsetlinewidth{0.803000pt}%
\definecolor{currentstroke}{rgb}{0.000000,0.000000,0.000000}%
\pgfsetstrokecolor{currentstroke}%
\pgfsetdash{}{0pt}%
\pgfpathmoveto{\pgfqpoint{0.885050in}{2.474259in}}%
\pgfpathlineto{\pgfqpoint{5.840250in}{2.474259in}}%
\pgfusepath{stroke}%
\end{pgfscope}%
\begin{pgfscope}%
\pgfsetrectcap%
\pgfsetmiterjoin%
\pgfsetlinewidth{0.803000pt}%
\definecolor{currentstroke}{rgb}{0.000000,0.000000,0.000000}%
\pgfsetstrokecolor{currentstroke}%
\pgfsetdash{}{0pt}%
\pgfpathmoveto{\pgfqpoint{0.885050in}{3.760185in}}%
\pgfpathlineto{\pgfqpoint{5.840250in}{3.760185in}}%
\pgfusepath{stroke}%
\end{pgfscope}%
\begin{pgfscope}%
\definecolor{textcolor}{rgb}{0.000000,0.000000,0.000000}%
\pgfsetstrokecolor{textcolor}%
\pgfsetfillcolor{textcolor}%
\pgftext[x=3.362650in,y=3.843519in,,base]{\color{textcolor}{\sffamily\fontsize{12.000000}{14.400000}\selectfont\catcode`\^=\active\def^{\ifmmode\sp\else\^{}\fi}\catcode`\%=\active\def%{\%}obciążenie}}%
\end{pgfscope}%
\begin{pgfscope}%
\pgfsetbuttcap%
\pgfsetmiterjoin%
\definecolor{currentfill}{rgb}{1.000000,1.000000,1.000000}%
\pgfsetfillcolor{currentfill}%
\pgfsetlinewidth{0.000000pt}%
\definecolor{currentstroke}{rgb}{0.000000,0.000000,0.000000}%
\pgfsetstrokecolor{currentstroke}%
\pgfsetstrokeopacity{0.000000}%
\pgfsetdash{}{0pt}%
\pgfpathmoveto{\pgfqpoint{0.885050in}{0.587778in}}%
\pgfpathlineto{\pgfqpoint{5.840250in}{0.587778in}}%
\pgfpathlineto{\pgfqpoint{5.840250in}{1.873704in}}%
\pgfpathlineto{\pgfqpoint{0.885050in}{1.873704in}}%
\pgfpathlineto{\pgfqpoint{0.885050in}{0.587778in}}%
\pgfpathclose%
\pgfusepath{fill}%
\end{pgfscope}%
\begin{pgfscope}%
\pgfsetbuttcap%
\pgfsetroundjoin%
\definecolor{currentfill}{rgb}{0.000000,0.000000,0.000000}%
\pgfsetfillcolor{currentfill}%
\pgfsetlinewidth{0.803000pt}%
\definecolor{currentstroke}{rgb}{0.000000,0.000000,0.000000}%
\pgfsetstrokecolor{currentstroke}%
\pgfsetdash{}{0pt}%
\pgfsys@defobject{currentmarker}{\pgfqpoint{0.000000in}{-0.048611in}}{\pgfqpoint{0.000000in}{0.000000in}}{%
\pgfpathmoveto{\pgfqpoint{0.000000in}{0.000000in}}%
\pgfpathlineto{\pgfqpoint{0.000000in}{-0.048611in}}%
\pgfusepath{stroke,fill}%
}%
\begin{pgfscope}%
\pgfsys@transformshift{1.096806in}{0.587778in}%
\pgfsys@useobject{currentmarker}{}%
\end{pgfscope}%
\end{pgfscope}%
\begin{pgfscope}%
\definecolor{textcolor}{rgb}{0.000000,0.000000,0.000000}%
\pgfsetstrokecolor{textcolor}%
\pgfsetfillcolor{textcolor}%
\pgftext[x=1.096806in,y=0.490556in,,top]{\color{textcolor}{\sffamily\fontsize{10.000000}{12.000000}\selectfont\catcode`\^=\active\def^{\ifmmode\sp\else\^{}\fi}\catcode`\%=\active\def%{\%}0}}%
\end{pgfscope}%
\begin{pgfscope}%
\pgfsetbuttcap%
\pgfsetroundjoin%
\definecolor{currentfill}{rgb}{0.000000,0.000000,0.000000}%
\pgfsetfillcolor{currentfill}%
\pgfsetlinewidth{0.803000pt}%
\definecolor{currentstroke}{rgb}{0.000000,0.000000,0.000000}%
\pgfsetstrokecolor{currentstroke}%
\pgfsetdash{}{0pt}%
\pgfsys@defobject{currentmarker}{\pgfqpoint{0.000000in}{-0.048611in}}{\pgfqpoint{0.000000in}{0.000000in}}{%
\pgfpathmoveto{\pgfqpoint{0.000000in}{0.000000in}}%
\pgfpathlineto{\pgfqpoint{0.000000in}{-0.048611in}}%
\pgfusepath{stroke,fill}%
}%
\begin{pgfscope}%
\pgfsys@transformshift{1.939362in}{0.587778in}%
\pgfsys@useobject{currentmarker}{}%
\end{pgfscope}%
\end{pgfscope}%
\begin{pgfscope}%
\definecolor{textcolor}{rgb}{0.000000,0.000000,0.000000}%
\pgfsetstrokecolor{textcolor}%
\pgfsetfillcolor{textcolor}%
\pgftext[x=1.939362in,y=0.490556in,,top]{\color{textcolor}{\sffamily\fontsize{10.000000}{12.000000}\selectfont\catcode`\^=\active\def^{\ifmmode\sp\else\^{}\fi}\catcode`\%=\active\def%{\%}2000}}%
\end{pgfscope}%
\begin{pgfscope}%
\pgfsetbuttcap%
\pgfsetroundjoin%
\definecolor{currentfill}{rgb}{0.000000,0.000000,0.000000}%
\pgfsetfillcolor{currentfill}%
\pgfsetlinewidth{0.803000pt}%
\definecolor{currentstroke}{rgb}{0.000000,0.000000,0.000000}%
\pgfsetstrokecolor{currentstroke}%
\pgfsetdash{}{0pt}%
\pgfsys@defobject{currentmarker}{\pgfqpoint{0.000000in}{-0.048611in}}{\pgfqpoint{0.000000in}{0.000000in}}{%
\pgfpathmoveto{\pgfqpoint{0.000000in}{0.000000in}}%
\pgfpathlineto{\pgfqpoint{0.000000in}{-0.048611in}}%
\pgfusepath{stroke,fill}%
}%
\begin{pgfscope}%
\pgfsys@transformshift{2.781918in}{0.587778in}%
\pgfsys@useobject{currentmarker}{}%
\end{pgfscope}%
\end{pgfscope}%
\begin{pgfscope}%
\definecolor{textcolor}{rgb}{0.000000,0.000000,0.000000}%
\pgfsetstrokecolor{textcolor}%
\pgfsetfillcolor{textcolor}%
\pgftext[x=2.781918in,y=0.490556in,,top]{\color{textcolor}{\sffamily\fontsize{10.000000}{12.000000}\selectfont\catcode`\^=\active\def^{\ifmmode\sp\else\^{}\fi}\catcode`\%=\active\def%{\%}4000}}%
\end{pgfscope}%
\begin{pgfscope}%
\pgfsetbuttcap%
\pgfsetroundjoin%
\definecolor{currentfill}{rgb}{0.000000,0.000000,0.000000}%
\pgfsetfillcolor{currentfill}%
\pgfsetlinewidth{0.803000pt}%
\definecolor{currentstroke}{rgb}{0.000000,0.000000,0.000000}%
\pgfsetstrokecolor{currentstroke}%
\pgfsetdash{}{0pt}%
\pgfsys@defobject{currentmarker}{\pgfqpoint{0.000000in}{-0.048611in}}{\pgfqpoint{0.000000in}{0.000000in}}{%
\pgfpathmoveto{\pgfqpoint{0.000000in}{0.000000in}}%
\pgfpathlineto{\pgfqpoint{0.000000in}{-0.048611in}}%
\pgfusepath{stroke,fill}%
}%
\begin{pgfscope}%
\pgfsys@transformshift{3.624474in}{0.587778in}%
\pgfsys@useobject{currentmarker}{}%
\end{pgfscope}%
\end{pgfscope}%
\begin{pgfscope}%
\definecolor{textcolor}{rgb}{0.000000,0.000000,0.000000}%
\pgfsetstrokecolor{textcolor}%
\pgfsetfillcolor{textcolor}%
\pgftext[x=3.624474in,y=0.490556in,,top]{\color{textcolor}{\sffamily\fontsize{10.000000}{12.000000}\selectfont\catcode`\^=\active\def^{\ifmmode\sp\else\^{}\fi}\catcode`\%=\active\def%{\%}6000}}%
\end{pgfscope}%
\begin{pgfscope}%
\pgfsetbuttcap%
\pgfsetroundjoin%
\definecolor{currentfill}{rgb}{0.000000,0.000000,0.000000}%
\pgfsetfillcolor{currentfill}%
\pgfsetlinewidth{0.803000pt}%
\definecolor{currentstroke}{rgb}{0.000000,0.000000,0.000000}%
\pgfsetstrokecolor{currentstroke}%
\pgfsetdash{}{0pt}%
\pgfsys@defobject{currentmarker}{\pgfqpoint{0.000000in}{-0.048611in}}{\pgfqpoint{0.000000in}{0.000000in}}{%
\pgfpathmoveto{\pgfqpoint{0.000000in}{0.000000in}}%
\pgfpathlineto{\pgfqpoint{0.000000in}{-0.048611in}}%
\pgfusepath{stroke,fill}%
}%
\begin{pgfscope}%
\pgfsys@transformshift{4.467031in}{0.587778in}%
\pgfsys@useobject{currentmarker}{}%
\end{pgfscope}%
\end{pgfscope}%
\begin{pgfscope}%
\definecolor{textcolor}{rgb}{0.000000,0.000000,0.000000}%
\pgfsetstrokecolor{textcolor}%
\pgfsetfillcolor{textcolor}%
\pgftext[x=4.467031in,y=0.490556in,,top]{\color{textcolor}{\sffamily\fontsize{10.000000}{12.000000}\selectfont\catcode`\^=\active\def^{\ifmmode\sp\else\^{}\fi}\catcode`\%=\active\def%{\%}8000}}%
\end{pgfscope}%
\begin{pgfscope}%
\pgfsetbuttcap%
\pgfsetroundjoin%
\definecolor{currentfill}{rgb}{0.000000,0.000000,0.000000}%
\pgfsetfillcolor{currentfill}%
\pgfsetlinewidth{0.803000pt}%
\definecolor{currentstroke}{rgb}{0.000000,0.000000,0.000000}%
\pgfsetstrokecolor{currentstroke}%
\pgfsetdash{}{0pt}%
\pgfsys@defobject{currentmarker}{\pgfqpoint{0.000000in}{-0.048611in}}{\pgfqpoint{0.000000in}{0.000000in}}{%
\pgfpathmoveto{\pgfqpoint{0.000000in}{0.000000in}}%
\pgfpathlineto{\pgfqpoint{0.000000in}{-0.048611in}}%
\pgfusepath{stroke,fill}%
}%
\begin{pgfscope}%
\pgfsys@transformshift{5.309587in}{0.587778in}%
\pgfsys@useobject{currentmarker}{}%
\end{pgfscope}%
\end{pgfscope}%
\begin{pgfscope}%
\definecolor{textcolor}{rgb}{0.000000,0.000000,0.000000}%
\pgfsetstrokecolor{textcolor}%
\pgfsetfillcolor{textcolor}%
\pgftext[x=5.309587in,y=0.490556in,,top]{\color{textcolor}{\sffamily\fontsize{10.000000}{12.000000}\selectfont\catcode`\^=\active\def^{\ifmmode\sp\else\^{}\fi}\catcode`\%=\active\def%{\%}10000}}%
\end{pgfscope}%
\begin{pgfscope}%
\definecolor{textcolor}{rgb}{0.000000,0.000000,0.000000}%
\pgfsetstrokecolor{textcolor}%
\pgfsetfillcolor{textcolor}%
\pgftext[x=3.362650in,y=0.300587in,,top]{\color{textcolor}{\sffamily\fontsize{10.000000}{12.000000}\selectfont\catcode`\^=\active\def^{\ifmmode\sp\else\^{}\fi}\catcode`\%=\active\def%{\%}czas (próbki)}}%
\end{pgfscope}%
\begin{pgfscope}%
\pgfsetbuttcap%
\pgfsetroundjoin%
\definecolor{currentfill}{rgb}{0.000000,0.000000,0.000000}%
\pgfsetfillcolor{currentfill}%
\pgfsetlinewidth{0.803000pt}%
\definecolor{currentstroke}{rgb}{0.000000,0.000000,0.000000}%
\pgfsetstrokecolor{currentstroke}%
\pgfsetdash{}{0pt}%
\pgfsys@defobject{currentmarker}{\pgfqpoint{-0.048611in}{0.000000in}}{\pgfqpoint{-0.000000in}{0.000000in}}{%
\pgfpathmoveto{\pgfqpoint{-0.000000in}{0.000000in}}%
\pgfpathlineto{\pgfqpoint{-0.048611in}{0.000000in}}%
\pgfusepath{stroke,fill}%
}%
\begin{pgfscope}%
\pgfsys@transformshift{0.885050in}{0.803850in}%
\pgfsys@useobject{currentmarker}{}%
\end{pgfscope}%
\end{pgfscope}%
\begin{pgfscope}%
\definecolor{textcolor}{rgb}{0.000000,0.000000,0.000000}%
\pgfsetstrokecolor{textcolor}%
\pgfsetfillcolor{textcolor}%
\pgftext[x=0.458924in, y=0.751089in, left, base]{\color{textcolor}{\sffamily\fontsize{10.000000}{12.000000}\selectfont\catcode`\^=\active\def^{\ifmmode\sp\else\^{}\fi}\catcode`\%=\active\def%{\%}\ensuremath{-}0.5}}%
\end{pgfscope}%
\begin{pgfscope}%
\pgfsetbuttcap%
\pgfsetroundjoin%
\definecolor{currentfill}{rgb}{0.000000,0.000000,0.000000}%
\pgfsetfillcolor{currentfill}%
\pgfsetlinewidth{0.803000pt}%
\definecolor{currentstroke}{rgb}{0.000000,0.000000,0.000000}%
\pgfsetstrokecolor{currentstroke}%
\pgfsetdash{}{0pt}%
\pgfsys@defobject{currentmarker}{\pgfqpoint{-0.048611in}{0.000000in}}{\pgfqpoint{-0.000000in}{0.000000in}}{%
\pgfpathmoveto{\pgfqpoint{-0.000000in}{0.000000in}}%
\pgfpathlineto{\pgfqpoint{-0.048611in}{0.000000in}}%
\pgfusepath{stroke,fill}%
}%
\begin{pgfscope}%
\pgfsys@transformshift{0.885050in}{1.224173in}%
\pgfsys@useobject{currentmarker}{}%
\end{pgfscope}%
\end{pgfscope}%
\begin{pgfscope}%
\definecolor{textcolor}{rgb}{0.000000,0.000000,0.000000}%
\pgfsetstrokecolor{textcolor}%
\pgfsetfillcolor{textcolor}%
\pgftext[x=0.566949in, y=1.171412in, left, base]{\color{textcolor}{\sffamily\fontsize{10.000000}{12.000000}\selectfont\catcode`\^=\active\def^{\ifmmode\sp\else\^{}\fi}\catcode`\%=\active\def%{\%}0.0}}%
\end{pgfscope}%
\begin{pgfscope}%
\pgfsetbuttcap%
\pgfsetroundjoin%
\definecolor{currentfill}{rgb}{0.000000,0.000000,0.000000}%
\pgfsetfillcolor{currentfill}%
\pgfsetlinewidth{0.803000pt}%
\definecolor{currentstroke}{rgb}{0.000000,0.000000,0.000000}%
\pgfsetstrokecolor{currentstroke}%
\pgfsetdash{}{0pt}%
\pgfsys@defobject{currentmarker}{\pgfqpoint{-0.048611in}{0.000000in}}{\pgfqpoint{-0.000000in}{0.000000in}}{%
\pgfpathmoveto{\pgfqpoint{-0.000000in}{0.000000in}}%
\pgfpathlineto{\pgfqpoint{-0.048611in}{0.000000in}}%
\pgfusepath{stroke,fill}%
}%
\begin{pgfscope}%
\pgfsys@transformshift{0.885050in}{1.644496in}%
\pgfsys@useobject{currentmarker}{}%
\end{pgfscope}%
\end{pgfscope}%
\begin{pgfscope}%
\definecolor{textcolor}{rgb}{0.000000,0.000000,0.000000}%
\pgfsetstrokecolor{textcolor}%
\pgfsetfillcolor{textcolor}%
\pgftext[x=0.566949in, y=1.591735in, left, base]{\color{textcolor}{\sffamily\fontsize{10.000000}{12.000000}\selectfont\catcode`\^=\active\def^{\ifmmode\sp\else\^{}\fi}\catcode`\%=\active\def%{\%}0.5}}%
\end{pgfscope}%
\begin{pgfscope}%
\definecolor{textcolor}{rgb}{0.000000,0.000000,0.000000}%
\pgfsetstrokecolor{textcolor}%
\pgfsetfillcolor{textcolor}%
\pgftext[x=0.403368in,y=1.230741in,,bottom,rotate=90.000000]{\color{textcolor}{\sffamily\fontsize{10.000000}{12.000000}\selectfont\catcode`\^=\active\def^{\ifmmode\sp\else\^{}\fi}\catcode`\%=\active\def%{\%}czas (ns)}}%
\end{pgfscope}%
\begin{pgfscope}%
\pgfpathrectangle{\pgfqpoint{0.885050in}{0.587778in}}{\pgfqpoint{4.955200in}{1.285926in}}%
\pgfusepath{clip}%
\pgfsetrectcap%
\pgfsetroundjoin%
\pgfsetlinewidth{1.505625pt}%
\definecolor{currentstroke}{rgb}{0.145098,0.145098,1.000000}%
\pgfsetstrokecolor{currentstroke}%
\pgfsetdash{}{0pt}%
\pgfpathmoveto{\pgfqpoint{1.123767in}{1.224173in}}%
\pgfpathlineto{\pgfqpoint{1.124189in}{1.224173in}}%
\pgfpathlineto{\pgfqpoint{1.125453in}{1.250443in}}%
\pgfpathlineto{\pgfqpoint{1.127138in}{1.211038in}}%
\pgfpathlineto{\pgfqpoint{1.128402in}{1.237308in}}%
\pgfpathlineto{\pgfqpoint{1.130087in}{1.197903in}}%
\pgfpathlineto{\pgfqpoint{1.131350in}{1.224173in}}%
\pgfpathlineto{\pgfqpoint{1.131772in}{1.224173in}}%
\pgfpathlineto{\pgfqpoint{1.132193in}{1.237308in}}%
\pgfpathlineto{\pgfqpoint{1.132614in}{1.224173in}}%
\pgfpathlineto{\pgfqpoint{1.133878in}{1.224173in}}%
\pgfpathlineto{\pgfqpoint{1.134299in}{1.197903in}}%
\pgfpathlineto{\pgfqpoint{1.134721in}{1.237308in}}%
\pgfpathlineto{\pgfqpoint{1.135985in}{1.211038in}}%
\pgfpathlineto{\pgfqpoint{1.136827in}{1.237308in}}%
\pgfpathlineto{\pgfqpoint{1.137248in}{1.224173in}}%
\pgfpathlineto{\pgfqpoint{1.138091in}{1.224173in}}%
\pgfpathlineto{\pgfqpoint{1.138512in}{1.237308in}}%
\pgfpathlineto{\pgfqpoint{1.138933in}{1.224173in}}%
\pgfpathlineto{\pgfqpoint{1.139776in}{1.224173in}}%
\pgfpathlineto{\pgfqpoint{1.140619in}{1.211038in}}%
\pgfpathlineto{\pgfqpoint{1.142304in}{1.237308in}}%
\pgfpathlineto{\pgfqpoint{1.142725in}{1.211038in}}%
\pgfpathlineto{\pgfqpoint{1.143568in}{1.224173in}}%
\pgfpathlineto{\pgfqpoint{1.143989in}{1.211038in}}%
\pgfpathlineto{\pgfqpoint{1.144410in}{1.250443in}}%
\pgfpathlineto{\pgfqpoint{1.145253in}{1.237308in}}%
\pgfpathlineto{\pgfqpoint{1.145674in}{1.197903in}}%
\pgfpathlineto{\pgfqpoint{1.146095in}{1.224173in}}%
\pgfpathlineto{\pgfqpoint{1.146517in}{1.224173in}}%
\pgfpathlineto{\pgfqpoint{1.146938in}{1.211038in}}%
\pgfpathlineto{\pgfqpoint{1.148202in}{1.237308in}}%
\pgfpathlineto{\pgfqpoint{1.149887in}{1.211038in}}%
\pgfpathlineto{\pgfqpoint{1.150308in}{1.211038in}}%
\pgfpathlineto{\pgfqpoint{1.151151in}{1.250443in}}%
\pgfpathlineto{\pgfqpoint{1.151993in}{1.197903in}}%
\pgfpathlineto{\pgfqpoint{1.152414in}{1.237308in}}%
\pgfpathlineto{\pgfqpoint{1.153257in}{1.224173in}}%
\pgfpathlineto{\pgfqpoint{1.154100in}{1.250443in}}%
\pgfpathlineto{\pgfqpoint{1.155363in}{1.197903in}}%
\pgfpathlineto{\pgfqpoint{1.157048in}{1.250443in}}%
\pgfpathlineto{\pgfqpoint{1.157470in}{1.250443in}}%
\pgfpathlineto{\pgfqpoint{1.158312in}{1.211038in}}%
\pgfpathlineto{\pgfqpoint{1.158734in}{1.224173in}}%
\pgfpathlineto{\pgfqpoint{1.159155in}{1.263578in}}%
\pgfpathlineto{\pgfqpoint{1.159576in}{1.211038in}}%
\pgfpathlineto{\pgfqpoint{1.159997in}{1.250443in}}%
\pgfpathlineto{\pgfqpoint{1.161683in}{1.211038in}}%
\pgfpathlineto{\pgfqpoint{1.162946in}{1.237308in}}%
\pgfpathlineto{\pgfqpoint{1.163368in}{1.237308in}}%
\pgfpathlineto{\pgfqpoint{1.164210in}{1.250443in}}%
\pgfpathlineto{\pgfqpoint{1.164631in}{1.211038in}}%
\pgfpathlineto{\pgfqpoint{1.165053in}{1.250443in}}%
\pgfpathlineto{\pgfqpoint{1.165474in}{1.250443in}}%
\pgfpathlineto{\pgfqpoint{1.165895in}{1.211038in}}%
\pgfpathlineto{\pgfqpoint{1.166317in}{1.237308in}}%
\pgfpathlineto{\pgfqpoint{1.166738in}{1.237308in}}%
\pgfpathlineto{\pgfqpoint{1.167580in}{1.211038in}}%
\pgfpathlineto{\pgfqpoint{1.168844in}{1.237308in}}%
\pgfpathlineto{\pgfqpoint{1.170951in}{1.184768in}}%
\pgfpathlineto{\pgfqpoint{1.172214in}{1.224173in}}%
\pgfpathlineto{\pgfqpoint{1.173478in}{1.211038in}}%
\pgfpathlineto{\pgfqpoint{1.173900in}{1.211038in}}%
\pgfpathlineto{\pgfqpoint{1.174321in}{1.224173in}}%
\pgfpathlineto{\pgfqpoint{1.174742in}{1.211038in}}%
\pgfpathlineto{\pgfqpoint{1.175163in}{1.197903in}}%
\pgfpathlineto{\pgfqpoint{1.175585in}{1.211038in}}%
\pgfpathlineto{\pgfqpoint{1.176849in}{1.224173in}}%
\pgfpathlineto{\pgfqpoint{1.177691in}{1.197903in}}%
\pgfpathlineto{\pgfqpoint{1.179376in}{1.276714in}}%
\pgfpathlineto{\pgfqpoint{1.180219in}{1.197903in}}%
\pgfpathlineto{\pgfqpoint{1.180640in}{1.224173in}}%
\pgfpathlineto{\pgfqpoint{1.181061in}{1.224173in}}%
\pgfpathlineto{\pgfqpoint{1.181483in}{1.237308in}}%
\pgfpathlineto{\pgfqpoint{1.181904in}{1.211038in}}%
\pgfpathlineto{\pgfqpoint{1.182325in}{1.224173in}}%
\pgfpathlineto{\pgfqpoint{1.183168in}{1.237308in}}%
\pgfpathlineto{\pgfqpoint{1.183589in}{1.197903in}}%
\pgfpathlineto{\pgfqpoint{1.184010in}{1.237308in}}%
\pgfpathlineto{\pgfqpoint{1.184432in}{1.250443in}}%
\pgfpathlineto{\pgfqpoint{1.185695in}{1.224173in}}%
\pgfpathlineto{\pgfqpoint{1.186117in}{1.224173in}}%
\pgfpathlineto{\pgfqpoint{1.186538in}{1.250443in}}%
\pgfpathlineto{\pgfqpoint{1.187380in}{1.237308in}}%
\pgfpathlineto{\pgfqpoint{1.187802in}{1.211038in}}%
\pgfpathlineto{\pgfqpoint{1.188223in}{1.224173in}}%
\pgfpathlineto{\pgfqpoint{1.188644in}{1.263578in}}%
\pgfpathlineto{\pgfqpoint{1.189066in}{1.224173in}}%
\pgfpathlineto{\pgfqpoint{1.189487in}{1.224173in}}%
\pgfpathlineto{\pgfqpoint{1.189908in}{1.197903in}}%
\pgfpathlineto{\pgfqpoint{1.190329in}{1.224173in}}%
\pgfpathlineto{\pgfqpoint{1.190751in}{1.224173in}}%
\pgfpathlineto{\pgfqpoint{1.191593in}{1.237308in}}%
\pgfpathlineto{\pgfqpoint{1.192857in}{1.158498in}}%
\pgfpathlineto{\pgfqpoint{1.193700in}{1.237308in}}%
\pgfpathlineto{\pgfqpoint{1.194121in}{1.197903in}}%
\pgfpathlineto{\pgfqpoint{1.195385in}{1.250443in}}%
\pgfpathlineto{\pgfqpoint{1.195806in}{1.224173in}}%
\pgfpathlineto{\pgfqpoint{1.196649in}{1.211038in}}%
\pgfpathlineto{\pgfqpoint{1.197070in}{1.250443in}}%
\pgfpathlineto{\pgfqpoint{1.197491in}{1.211038in}}%
\pgfpathlineto{\pgfqpoint{1.198755in}{1.224173in}}%
\pgfpathlineto{\pgfqpoint{1.199598in}{1.224173in}}%
\pgfpathlineto{\pgfqpoint{1.200861in}{1.211038in}}%
\pgfpathlineto{\pgfqpoint{1.201704in}{1.237308in}}%
\pgfpathlineto{\pgfqpoint{1.202546in}{1.211038in}}%
\pgfpathlineto{\pgfqpoint{1.202968in}{1.224173in}}%
\pgfpathlineto{\pgfqpoint{1.203389in}{1.237308in}}%
\pgfpathlineto{\pgfqpoint{1.203810in}{1.224173in}}%
\pgfpathlineto{\pgfqpoint{1.204232in}{1.224173in}}%
\pgfpathlineto{\pgfqpoint{1.205074in}{1.211038in}}%
\pgfpathlineto{\pgfqpoint{1.205495in}{1.250443in}}%
\pgfpathlineto{\pgfqpoint{1.206338in}{1.237308in}}%
\pgfpathlineto{\pgfqpoint{1.207181in}{1.224173in}}%
\pgfpathlineto{\pgfqpoint{1.207602in}{1.250443in}}%
\pgfpathlineto{\pgfqpoint{1.208023in}{1.237308in}}%
\pgfpathlineto{\pgfqpoint{1.208866in}{1.197903in}}%
\pgfpathlineto{\pgfqpoint{1.209708in}{1.211038in}}%
\pgfpathlineto{\pgfqpoint{1.210130in}{1.211038in}}%
\pgfpathlineto{\pgfqpoint{1.210551in}{1.224173in}}%
\pgfpathlineto{\pgfqpoint{1.210972in}{1.197903in}}%
\pgfpathlineto{\pgfqpoint{1.211393in}{1.224173in}}%
\pgfpathlineto{\pgfqpoint{1.211815in}{1.237308in}}%
\pgfpathlineto{\pgfqpoint{1.212236in}{1.224173in}}%
\pgfpathlineto{\pgfqpoint{1.213078in}{1.224173in}}%
\pgfpathlineto{\pgfqpoint{1.214342in}{1.237308in}}%
\pgfpathlineto{\pgfqpoint{1.214764in}{1.237308in}}%
\pgfpathlineto{\pgfqpoint{1.215185in}{1.211038in}}%
\pgfpathlineto{\pgfqpoint{1.215606in}{1.224173in}}%
\pgfpathlineto{\pgfqpoint{1.216027in}{1.237308in}}%
\pgfpathlineto{\pgfqpoint{1.216449in}{1.211038in}}%
\pgfpathlineto{\pgfqpoint{1.216870in}{1.224173in}}%
\pgfpathlineto{\pgfqpoint{1.218134in}{1.237308in}}%
\pgfpathlineto{\pgfqpoint{1.218976in}{1.224173in}}%
\pgfpathlineto{\pgfqpoint{1.219398in}{1.237308in}}%
\pgfpathlineto{\pgfqpoint{1.219819in}{1.224173in}}%
\pgfpathlineto{\pgfqpoint{1.221925in}{1.224173in}}%
\pgfpathlineto{\pgfqpoint{1.223189in}{1.237308in}}%
\pgfpathlineto{\pgfqpoint{1.223610in}{1.211038in}}%
\pgfpathlineto{\pgfqpoint{1.224032in}{1.224173in}}%
\pgfpathlineto{\pgfqpoint{1.224453in}{1.250443in}}%
\pgfpathlineto{\pgfqpoint{1.224874in}{1.224173in}}%
\pgfpathlineto{\pgfqpoint{1.226138in}{1.211038in}}%
\pgfpathlineto{\pgfqpoint{1.226981in}{1.211038in}}%
\pgfpathlineto{\pgfqpoint{1.227402in}{1.224173in}}%
\pgfpathlineto{\pgfqpoint{1.227823in}{1.211038in}}%
\pgfpathlineto{\pgfqpoint{1.228244in}{1.211038in}}%
\pgfpathlineto{\pgfqpoint{1.229930in}{1.237308in}}%
\pgfpathlineto{\pgfqpoint{1.230351in}{1.237308in}}%
\pgfpathlineto{\pgfqpoint{1.231615in}{1.211038in}}%
\pgfpathlineto{\pgfqpoint{1.232036in}{1.237308in}}%
\pgfpathlineto{\pgfqpoint{1.232457in}{1.211038in}}%
\pgfpathlineto{\pgfqpoint{1.232879in}{1.211038in}}%
\pgfpathlineto{\pgfqpoint{1.234142in}{1.237308in}}%
\pgfpathlineto{\pgfqpoint{1.235406in}{1.211038in}}%
\pgfpathlineto{\pgfqpoint{1.237091in}{1.250443in}}%
\pgfpathlineto{\pgfqpoint{1.238355in}{1.224173in}}%
\pgfpathlineto{\pgfqpoint{1.238776in}{1.224173in}}%
\pgfpathlineto{\pgfqpoint{1.240040in}{1.237308in}}%
\pgfpathlineto{\pgfqpoint{1.241304in}{1.224173in}}%
\pgfpathlineto{\pgfqpoint{1.241725in}{1.224173in}}%
\pgfpathlineto{\pgfqpoint{1.242147in}{1.250443in}}%
\pgfpathlineto{\pgfqpoint{1.242568in}{1.211038in}}%
\pgfpathlineto{\pgfqpoint{1.243832in}{1.224173in}}%
\pgfpathlineto{\pgfqpoint{1.245096in}{1.224173in}}%
\pgfpathlineto{\pgfqpoint{1.245517in}{1.211038in}}%
\pgfpathlineto{\pgfqpoint{1.246359in}{1.250443in}}%
\pgfpathlineto{\pgfqpoint{1.246781in}{1.224173in}}%
\pgfpathlineto{\pgfqpoint{1.247623in}{1.211038in}}%
\pgfpathlineto{\pgfqpoint{1.248466in}{1.250443in}}%
\pgfpathlineto{\pgfqpoint{1.248887in}{1.237308in}}%
\pgfpathlineto{\pgfqpoint{1.250572in}{1.211038in}}%
\pgfpathlineto{\pgfqpoint{1.250993in}{1.211038in}}%
\pgfpathlineto{\pgfqpoint{1.252257in}{1.237308in}}%
\pgfpathlineto{\pgfqpoint{1.253521in}{1.224173in}}%
\pgfpathlineto{\pgfqpoint{1.254785in}{1.237308in}}%
\pgfpathlineto{\pgfqpoint{1.255206in}{0.646229in}}%
\pgfpathlineto{\pgfqpoint{1.255628in}{1.197903in}}%
\pgfpathlineto{\pgfqpoint{1.256049in}{1.237308in}}%
\pgfpathlineto{\pgfqpoint{1.256470in}{1.224173in}}%
\pgfpathlineto{\pgfqpoint{1.257734in}{1.211038in}}%
\pgfpathlineto{\pgfqpoint{1.258998in}{1.224173in}}%
\pgfpathlineto{\pgfqpoint{1.259419in}{1.224173in}}%
\pgfpathlineto{\pgfqpoint{1.259840in}{1.197903in}}%
\pgfpathlineto{\pgfqpoint{1.260683in}{1.211038in}}%
\pgfpathlineto{\pgfqpoint{1.261104in}{1.224173in}}%
\pgfpathlineto{\pgfqpoint{1.261525in}{1.211038in}}%
\pgfpathlineto{\pgfqpoint{1.261947in}{1.211038in}}%
\pgfpathlineto{\pgfqpoint{1.263211in}{1.224173in}}%
\pgfpathlineto{\pgfqpoint{1.263632in}{1.040282in}}%
\pgfpathlineto{\pgfqpoint{1.264053in}{1.237308in}}%
\pgfpathlineto{\pgfqpoint{1.264474in}{1.237308in}}%
\pgfpathlineto{\pgfqpoint{1.264896in}{1.224173in}}%
\pgfpathlineto{\pgfqpoint{1.265317in}{1.237308in}}%
\pgfpathlineto{\pgfqpoint{1.265738in}{0.948336in}}%
\pgfpathlineto{\pgfqpoint{1.266159in}{1.224173in}}%
\pgfpathlineto{\pgfqpoint{1.267002in}{1.211038in}}%
\pgfpathlineto{\pgfqpoint{1.267423in}{1.224173in}}%
\pgfpathlineto{\pgfqpoint{1.267845in}{0.895796in}}%
\pgfpathlineto{\pgfqpoint{1.268266in}{1.224173in}}%
\pgfpathlineto{\pgfqpoint{1.268687in}{1.788982in}}%
\pgfpathlineto{\pgfqpoint{1.269108in}{1.237308in}}%
\pgfpathlineto{\pgfqpoint{1.269951in}{1.211038in}}%
\pgfpathlineto{\pgfqpoint{1.270372in}{1.224173in}}%
\pgfpathlineto{\pgfqpoint{1.270794in}{1.224173in}}%
\pgfpathlineto{\pgfqpoint{1.271215in}{1.237308in}}%
\pgfpathlineto{\pgfqpoint{1.271636in}{1.224173in}}%
\pgfpathlineto{\pgfqpoint{1.272900in}{1.211038in}}%
\pgfpathlineto{\pgfqpoint{1.273321in}{1.224173in}}%
\pgfpathlineto{\pgfqpoint{1.273742in}{1.211038in}}%
\pgfpathlineto{\pgfqpoint{1.274164in}{1.066552in}}%
\pgfpathlineto{\pgfqpoint{1.274585in}{1.224173in}}%
\pgfpathlineto{\pgfqpoint{1.275849in}{1.211038in}}%
\pgfpathlineto{\pgfqpoint{1.276270in}{1.224173in}}%
\pgfpathlineto{\pgfqpoint{1.276691in}{1.211038in}}%
\pgfpathlineto{\pgfqpoint{1.277113in}{1.394929in}}%
\pgfpathlineto{\pgfqpoint{1.277534in}{1.211038in}}%
\pgfpathlineto{\pgfqpoint{1.278377in}{1.224173in}}%
\pgfpathlineto{\pgfqpoint{1.278798in}{1.211038in}}%
\pgfpathlineto{\pgfqpoint{1.279219in}{1.500010in}}%
\pgfpathlineto{\pgfqpoint{1.279640in}{1.211038in}}%
\pgfpathlineto{\pgfqpoint{1.280062in}{1.237308in}}%
\pgfpathlineto{\pgfqpoint{1.280483in}{1.197903in}}%
\pgfpathlineto{\pgfqpoint{1.280904in}{1.224173in}}%
\pgfpathlineto{\pgfqpoint{1.281326in}{1.552551in}}%
\pgfpathlineto{\pgfqpoint{1.281747in}{1.211038in}}%
\pgfpathlineto{\pgfqpoint{1.282589in}{1.237308in}}%
\pgfpathlineto{\pgfqpoint{1.283011in}{1.224173in}}%
\pgfpathlineto{\pgfqpoint{1.283432in}{1.224173in}}%
\pgfpathlineto{\pgfqpoint{1.283853in}{1.237308in}}%
\pgfpathlineto{\pgfqpoint{1.284274in}{1.224173in}}%
\pgfpathlineto{\pgfqpoint{1.285538in}{1.211038in}}%
\pgfpathlineto{\pgfqpoint{1.285960in}{1.211038in}}%
\pgfpathlineto{\pgfqpoint{1.286381in}{1.224173in}}%
\pgfpathlineto{\pgfqpoint{1.287645in}{1.394929in}}%
\pgfpathlineto{\pgfqpoint{1.288909in}{1.224173in}}%
\pgfpathlineto{\pgfqpoint{1.290172in}{1.224173in}}%
\pgfpathlineto{\pgfqpoint{1.290594in}{1.237308in}}%
\pgfpathlineto{\pgfqpoint{1.291015in}{1.224173in}}%
\pgfpathlineto{\pgfqpoint{1.291857in}{1.224173in}}%
\pgfpathlineto{\pgfqpoint{1.292279in}{1.211038in}}%
\pgfpathlineto{\pgfqpoint{1.292700in}{1.224173in}}%
\pgfpathlineto{\pgfqpoint{1.293543in}{1.224173in}}%
\pgfpathlineto{\pgfqpoint{1.293964in}{1.250443in}}%
\pgfpathlineto{\pgfqpoint{1.294385in}{1.224173in}}%
\pgfpathlineto{\pgfqpoint{1.294806in}{1.211038in}}%
\pgfpathlineto{\pgfqpoint{1.295228in}{1.224173in}}%
\pgfpathlineto{\pgfqpoint{1.295649in}{1.224173in}}%
\pgfpathlineto{\pgfqpoint{1.296070in}{1.197903in}}%
\pgfpathlineto{\pgfqpoint{1.296492in}{1.211038in}}%
\pgfpathlineto{\pgfqpoint{1.297755in}{1.224173in}}%
\pgfpathlineto{\pgfqpoint{1.298177in}{1.224173in}}%
\pgfpathlineto{\pgfqpoint{1.299440in}{1.237308in}}%
\pgfpathlineto{\pgfqpoint{1.299862in}{1.224173in}}%
\pgfpathlineto{\pgfqpoint{1.300283in}{1.066552in}}%
\pgfpathlineto{\pgfqpoint{1.300704in}{1.237308in}}%
\pgfpathlineto{\pgfqpoint{1.301126in}{1.211038in}}%
\pgfpathlineto{\pgfqpoint{1.301547in}{1.224173in}}%
\pgfpathlineto{\pgfqpoint{1.302811in}{1.237308in}}%
\pgfpathlineto{\pgfqpoint{1.304075in}{1.211038in}}%
\pgfpathlineto{\pgfqpoint{1.304496in}{1.237308in}}%
\pgfpathlineto{\pgfqpoint{1.305338in}{1.224173in}}%
\pgfpathlineto{\pgfqpoint{1.305760in}{1.237308in}}%
\pgfpathlineto{\pgfqpoint{1.306181in}{1.211038in}}%
\pgfpathlineto{\pgfqpoint{1.307023in}{1.224173in}}%
\pgfpathlineto{\pgfqpoint{1.307445in}{1.237308in}}%
\pgfpathlineto{\pgfqpoint{1.307866in}{1.224173in}}%
\pgfpathlineto{\pgfqpoint{1.308287in}{1.224173in}}%
\pgfpathlineto{\pgfqpoint{1.308709in}{1.237308in}}%
\pgfpathlineto{\pgfqpoint{1.309130in}{1.224173in}}%
\pgfpathlineto{\pgfqpoint{1.310394in}{1.224173in}}%
\pgfpathlineto{\pgfqpoint{1.311236in}{1.211038in}}%
\pgfpathlineto{\pgfqpoint{1.311658in}{1.250443in}}%
\pgfpathlineto{\pgfqpoint{1.312500in}{1.237308in}}%
\pgfpathlineto{\pgfqpoint{1.314185in}{1.211038in}}%
\pgfpathlineto{\pgfqpoint{1.315449in}{1.224173in}}%
\pgfpathlineto{\pgfqpoint{1.315870in}{1.224173in}}%
\pgfpathlineto{\pgfqpoint{1.316292in}{1.211038in}}%
\pgfpathlineto{\pgfqpoint{1.317555in}{1.237308in}}%
\pgfpathlineto{\pgfqpoint{1.318819in}{1.224173in}}%
\pgfpathlineto{\pgfqpoint{1.319241in}{1.224173in}}%
\pgfpathlineto{\pgfqpoint{1.320083in}{1.237308in}}%
\pgfpathlineto{\pgfqpoint{1.321347in}{1.224173in}}%
\pgfpathlineto{\pgfqpoint{1.322611in}{1.237308in}}%
\pgfpathlineto{\pgfqpoint{1.323032in}{1.237308in}}%
\pgfpathlineto{\pgfqpoint{1.323453in}{1.224173in}}%
\pgfpathlineto{\pgfqpoint{1.323875in}{1.237308in}}%
\pgfpathlineto{\pgfqpoint{1.324717in}{1.237308in}}%
\pgfpathlineto{\pgfqpoint{1.325138in}{1.211038in}}%
\pgfpathlineto{\pgfqpoint{1.325981in}{1.224173in}}%
\pgfpathlineto{\pgfqpoint{1.326402in}{1.211038in}}%
\pgfpathlineto{\pgfqpoint{1.326824in}{1.224173in}}%
\pgfpathlineto{\pgfqpoint{1.327666in}{1.224173in}}%
\pgfpathlineto{\pgfqpoint{1.328087in}{1.237308in}}%
\pgfpathlineto{\pgfqpoint{1.328509in}{1.224173in}}%
\pgfpathlineto{\pgfqpoint{1.329351in}{1.224173in}}%
\pgfpathlineto{\pgfqpoint{1.329772in}{1.237308in}}%
\pgfpathlineto{\pgfqpoint{1.330194in}{1.224173in}}%
\pgfpathlineto{\pgfqpoint{1.331036in}{1.224173in}}%
\pgfpathlineto{\pgfqpoint{1.331458in}{1.211038in}}%
\pgfpathlineto{\pgfqpoint{1.331879in}{1.224173in}}%
\pgfpathlineto{\pgfqpoint{1.333564in}{1.224173in}}%
\pgfpathlineto{\pgfqpoint{1.334407in}{1.211038in}}%
\pgfpathlineto{\pgfqpoint{1.335670in}{1.224173in}}%
\pgfpathlineto{\pgfqpoint{1.336092in}{1.224173in}}%
\pgfpathlineto{\pgfqpoint{1.336513in}{1.211038in}}%
\pgfpathlineto{\pgfqpoint{1.336934in}{1.224173in}}%
\pgfpathlineto{\pgfqpoint{1.337355in}{1.237308in}}%
\pgfpathlineto{\pgfqpoint{1.337777in}{1.224173in}}%
\pgfpathlineto{\pgfqpoint{1.339041in}{1.224173in}}%
\pgfpathlineto{\pgfqpoint{1.339883in}{1.237308in}}%
\pgfpathlineto{\pgfqpoint{1.340726in}{1.224173in}}%
\pgfpathlineto{\pgfqpoint{1.341147in}{1.237308in}}%
\pgfpathlineto{\pgfqpoint{1.341568in}{1.224173in}}%
\pgfpathlineto{\pgfqpoint{1.343253in}{1.224173in}}%
\pgfpathlineto{\pgfqpoint{1.344096in}{1.237308in}}%
\pgfpathlineto{\pgfqpoint{1.345781in}{1.211038in}}%
\pgfpathlineto{\pgfqpoint{1.346624in}{1.211038in}}%
\pgfpathlineto{\pgfqpoint{1.347466in}{1.237308in}}%
\pgfpathlineto{\pgfqpoint{1.347887in}{1.224173in}}%
\pgfpathlineto{\pgfqpoint{1.349151in}{1.224173in}}%
\pgfpathlineto{\pgfqpoint{1.349573in}{1.211038in}}%
\pgfpathlineto{\pgfqpoint{1.349994in}{1.250443in}}%
\pgfpathlineto{\pgfqpoint{1.350415in}{1.237308in}}%
\pgfpathlineto{\pgfqpoint{1.351679in}{1.211038in}}%
\pgfpathlineto{\pgfqpoint{1.352943in}{1.224173in}}%
\pgfpathlineto{\pgfqpoint{1.356313in}{1.224173in}}%
\pgfpathlineto{\pgfqpoint{1.357156in}{1.184768in}}%
\pgfpathlineto{\pgfqpoint{1.358419in}{1.224173in}}%
\pgfpathlineto{\pgfqpoint{1.358841in}{1.224173in}}%
\pgfpathlineto{\pgfqpoint{1.360105in}{1.250443in}}%
\pgfpathlineto{\pgfqpoint{1.361790in}{1.211038in}}%
\pgfpathlineto{\pgfqpoint{1.363053in}{1.237308in}}%
\pgfpathlineto{\pgfqpoint{1.363475in}{1.211038in}}%
\pgfpathlineto{\pgfqpoint{1.364317in}{1.224173in}}%
\pgfpathlineto{\pgfqpoint{1.366424in}{1.224173in}}%
\pgfpathlineto{\pgfqpoint{1.366845in}{1.237308in}}%
\pgfpathlineto{\pgfqpoint{1.367266in}{1.224173in}}%
\pgfpathlineto{\pgfqpoint{1.368530in}{1.224173in}}%
\pgfpathlineto{\pgfqpoint{1.368951in}{1.237308in}}%
\pgfpathlineto{\pgfqpoint{1.369373in}{1.224173in}}%
\pgfpathlineto{\pgfqpoint{1.369794in}{1.224173in}}%
\pgfpathlineto{\pgfqpoint{1.370636in}{1.237308in}}%
\pgfpathlineto{\pgfqpoint{1.371058in}{1.211038in}}%
\pgfpathlineto{\pgfqpoint{1.371900in}{1.224173in}}%
\pgfpathlineto{\pgfqpoint{1.373164in}{1.224173in}}%
\pgfpathlineto{\pgfqpoint{1.373585in}{1.211038in}}%
\pgfpathlineto{\pgfqpoint{1.374428in}{1.237308in}}%
\pgfpathlineto{\pgfqpoint{1.374849in}{1.211038in}}%
\pgfpathlineto{\pgfqpoint{1.375692in}{1.224173in}}%
\pgfpathlineto{\pgfqpoint{1.376956in}{1.211038in}}%
\pgfpathlineto{\pgfqpoint{1.378641in}{1.237308in}}%
\pgfpathlineto{\pgfqpoint{1.380326in}{1.211038in}}%
\pgfpathlineto{\pgfqpoint{1.381590in}{1.211038in}}%
\pgfpathlineto{\pgfqpoint{1.382854in}{1.224173in}}%
\pgfpathlineto{\pgfqpoint{1.383275in}{1.224173in}}%
\pgfpathlineto{\pgfqpoint{1.383696in}{1.211038in}}%
\pgfpathlineto{\pgfqpoint{1.384117in}{1.224173in}}%
\pgfpathlineto{\pgfqpoint{1.384539in}{1.224173in}}%
\pgfpathlineto{\pgfqpoint{1.385381in}{1.211038in}}%
\pgfpathlineto{\pgfqpoint{1.386224in}{1.224173in}}%
\pgfpathlineto{\pgfqpoint{1.386645in}{1.473740in}}%
\pgfpathlineto{\pgfqpoint{1.387066in}{1.211038in}}%
\pgfpathlineto{\pgfqpoint{1.387488in}{1.211038in}}%
\pgfpathlineto{\pgfqpoint{1.388751in}{1.237308in}}%
\pgfpathlineto{\pgfqpoint{1.390015in}{1.224173in}}%
\pgfpathlineto{\pgfqpoint{1.391700in}{1.250443in}}%
\pgfpathlineto{\pgfqpoint{1.392964in}{1.224173in}}%
\pgfpathlineto{\pgfqpoint{1.394228in}{1.250443in}}%
\pgfpathlineto{\pgfqpoint{1.395492in}{1.211038in}}%
\pgfpathlineto{\pgfqpoint{1.395913in}{1.224173in}}%
\pgfpathlineto{\pgfqpoint{1.396756in}{1.224173in}}%
\pgfpathlineto{\pgfqpoint{1.398441in}{1.250443in}}%
\pgfpathlineto{\pgfqpoint{1.399283in}{1.224173in}}%
\pgfpathlineto{\pgfqpoint{1.399705in}{1.237308in}}%
\pgfpathlineto{\pgfqpoint{1.400126in}{0.974606in}}%
\pgfpathlineto{\pgfqpoint{1.400547in}{1.237308in}}%
\pgfpathlineto{\pgfqpoint{1.402232in}{1.211038in}}%
\pgfpathlineto{\pgfqpoint{1.402654in}{1.237308in}}%
\pgfpathlineto{\pgfqpoint{1.403496in}{1.224173in}}%
\pgfpathlineto{\pgfqpoint{1.403917in}{1.224173in}}%
\pgfpathlineto{\pgfqpoint{1.404339in}{1.237308in}}%
\pgfpathlineto{\pgfqpoint{1.405603in}{1.211038in}}%
\pgfpathlineto{\pgfqpoint{1.406024in}{1.250443in}}%
\pgfpathlineto{\pgfqpoint{1.406866in}{1.237308in}}%
\pgfpathlineto{\pgfqpoint{1.407709in}{1.211038in}}%
\pgfpathlineto{\pgfqpoint{1.408973in}{1.237308in}}%
\pgfpathlineto{\pgfqpoint{1.410237in}{1.224173in}}%
\pgfpathlineto{\pgfqpoint{1.411079in}{1.224173in}}%
\pgfpathlineto{\pgfqpoint{1.411922in}{1.211038in}}%
\pgfpathlineto{\pgfqpoint{1.412343in}{1.237308in}}%
\pgfpathlineto{\pgfqpoint{1.413186in}{1.224173in}}%
\pgfpathlineto{\pgfqpoint{1.414449in}{1.237308in}}%
\pgfpathlineto{\pgfqpoint{1.415292in}{1.224173in}}%
\pgfpathlineto{\pgfqpoint{1.416135in}{1.237308in}}%
\pgfpathlineto{\pgfqpoint{1.416556in}{1.224173in}}%
\pgfpathlineto{\pgfqpoint{1.416977in}{1.263578in}}%
\pgfpathlineto{\pgfqpoint{1.417398in}{1.211038in}}%
\pgfpathlineto{\pgfqpoint{1.417820in}{1.211038in}}%
\pgfpathlineto{\pgfqpoint{1.419083in}{1.237308in}}%
\pgfpathlineto{\pgfqpoint{1.419505in}{1.197903in}}%
\pgfpathlineto{\pgfqpoint{1.420347in}{1.211038in}}%
\pgfpathlineto{\pgfqpoint{1.422032in}{1.211038in}}%
\pgfpathlineto{\pgfqpoint{1.423296in}{1.224173in}}%
\pgfpathlineto{\pgfqpoint{1.424560in}{1.224173in}}%
\pgfpathlineto{\pgfqpoint{1.424981in}{1.237308in}}%
\pgfpathlineto{\pgfqpoint{1.425824in}{1.197903in}}%
\pgfpathlineto{\pgfqpoint{1.426245in}{1.211038in}}%
\pgfpathlineto{\pgfqpoint{1.428773in}{1.211038in}}%
\pgfpathlineto{\pgfqpoint{1.429194in}{1.224173in}}%
\pgfpathlineto{\pgfqpoint{1.430458in}{1.184768in}}%
\pgfpathlineto{\pgfqpoint{1.431722in}{1.224173in}}%
\pgfpathlineto{\pgfqpoint{1.432143in}{1.224173in}}%
\pgfpathlineto{\pgfqpoint{1.432986in}{1.211038in}}%
\pgfpathlineto{\pgfqpoint{1.434671in}{1.237308in}}%
\pgfpathlineto{\pgfqpoint{1.435935in}{1.224173in}}%
\pgfpathlineto{\pgfqpoint{1.436777in}{1.224173in}}%
\pgfpathlineto{\pgfqpoint{1.437620in}{1.211038in}}%
\pgfpathlineto{\pgfqpoint{1.438884in}{1.224173in}}%
\pgfpathlineto{\pgfqpoint{1.439305in}{1.224173in}}%
\pgfpathlineto{\pgfqpoint{1.440147in}{1.237308in}}%
\pgfpathlineto{\pgfqpoint{1.441411in}{1.211038in}}%
\pgfpathlineto{\pgfqpoint{1.442254in}{1.250443in}}%
\pgfpathlineto{\pgfqpoint{1.442675in}{1.224173in}}%
\pgfpathlineto{\pgfqpoint{1.443096in}{1.224173in}}%
\pgfpathlineto{\pgfqpoint{1.443518in}{1.211038in}}%
\pgfpathlineto{\pgfqpoint{1.443939in}{1.224173in}}%
\pgfpathlineto{\pgfqpoint{1.444360in}{1.237308in}}%
\pgfpathlineto{\pgfqpoint{1.444781in}{1.211038in}}%
\pgfpathlineto{\pgfqpoint{1.445624in}{1.224173in}}%
\pgfpathlineto{\pgfqpoint{1.446045in}{1.224173in}}%
\pgfpathlineto{\pgfqpoint{1.446467in}{1.237308in}}%
\pgfpathlineto{\pgfqpoint{1.446888in}{1.224173in}}%
\pgfpathlineto{\pgfqpoint{1.447309in}{1.224173in}}%
\pgfpathlineto{\pgfqpoint{1.447730in}{1.211038in}}%
\pgfpathlineto{\pgfqpoint{1.448152in}{1.224173in}}%
\pgfpathlineto{\pgfqpoint{1.448573in}{1.224173in}}%
\pgfpathlineto{\pgfqpoint{1.449837in}{1.211038in}}%
\pgfpathlineto{\pgfqpoint{1.451101in}{1.250443in}}%
\pgfpathlineto{\pgfqpoint{1.452786in}{1.211038in}}%
\pgfpathlineto{\pgfqpoint{1.454471in}{1.237308in}}%
\pgfpathlineto{\pgfqpoint{1.455735in}{1.197903in}}%
\pgfpathlineto{\pgfqpoint{1.456998in}{1.237308in}}%
\pgfpathlineto{\pgfqpoint{1.457420in}{1.211038in}}%
\pgfpathlineto{\pgfqpoint{1.457841in}{1.224173in}}%
\pgfpathlineto{\pgfqpoint{1.459105in}{1.237308in}}%
\pgfpathlineto{\pgfqpoint{1.459526in}{1.237308in}}%
\pgfpathlineto{\pgfqpoint{1.460790in}{1.224173in}}%
\pgfpathlineto{\pgfqpoint{1.461211in}{1.237308in}}%
\pgfpathlineto{\pgfqpoint{1.461633in}{1.211038in}}%
\pgfpathlineto{\pgfqpoint{1.462054in}{1.224173in}}%
\pgfpathlineto{\pgfqpoint{1.462475in}{1.237308in}}%
\pgfpathlineto{\pgfqpoint{1.462896in}{1.211038in}}%
\pgfpathlineto{\pgfqpoint{1.463739in}{1.224173in}}%
\pgfpathlineto{\pgfqpoint{1.464581in}{1.211038in}}%
\pgfpathlineto{\pgfqpoint{1.465424in}{1.237308in}}%
\pgfpathlineto{\pgfqpoint{1.465845in}{1.211038in}}%
\pgfpathlineto{\pgfqpoint{1.466688in}{1.224173in}}%
\pgfpathlineto{\pgfqpoint{1.467952in}{1.224173in}}%
\pgfpathlineto{\pgfqpoint{1.468373in}{1.250443in}}%
\pgfpathlineto{\pgfqpoint{1.469216in}{1.237308in}}%
\pgfpathlineto{\pgfqpoint{1.470479in}{1.224173in}}%
\pgfpathlineto{\pgfqpoint{1.472164in}{1.224173in}}%
\pgfpathlineto{\pgfqpoint{1.473007in}{1.211038in}}%
\pgfpathlineto{\pgfqpoint{1.474271in}{1.224173in}}%
\pgfpathlineto{\pgfqpoint{1.474692in}{1.224173in}}%
\pgfpathlineto{\pgfqpoint{1.475113in}{1.237308in}}%
\pgfpathlineto{\pgfqpoint{1.475535in}{1.224173in}}%
\pgfpathlineto{\pgfqpoint{1.476799in}{1.224173in}}%
\pgfpathlineto{\pgfqpoint{1.477220in}{1.211038in}}%
\pgfpathlineto{\pgfqpoint{1.477641in}{1.237308in}}%
\pgfpathlineto{\pgfqpoint{1.478062in}{1.197903in}}%
\pgfpathlineto{\pgfqpoint{1.478484in}{1.224173in}}%
\pgfpathlineto{\pgfqpoint{1.479326in}{1.224173in}}%
\pgfpathlineto{\pgfqpoint{1.479748in}{1.237308in}}%
\pgfpathlineto{\pgfqpoint{1.480169in}{1.224173in}}%
\pgfpathlineto{\pgfqpoint{1.480590in}{1.224173in}}%
\pgfpathlineto{\pgfqpoint{1.481011in}{1.211038in}}%
\pgfpathlineto{\pgfqpoint{1.481433in}{1.224173in}}%
\pgfpathlineto{\pgfqpoint{1.482275in}{1.224173in}}%
\pgfpathlineto{\pgfqpoint{1.482696in}{1.237308in}}%
\pgfpathlineto{\pgfqpoint{1.483118in}{1.211038in}}%
\pgfpathlineto{\pgfqpoint{1.483960in}{1.224173in}}%
\pgfpathlineto{\pgfqpoint{1.484382in}{1.224173in}}%
\pgfpathlineto{\pgfqpoint{1.484803in}{1.197903in}}%
\pgfpathlineto{\pgfqpoint{1.485224in}{1.211038in}}%
\pgfpathlineto{\pgfqpoint{1.486488in}{1.237308in}}%
\pgfpathlineto{\pgfqpoint{1.488173in}{1.211038in}}%
\pgfpathlineto{\pgfqpoint{1.488594in}{1.224173in}}%
\pgfpathlineto{\pgfqpoint{1.489016in}{1.211038in}}%
\pgfpathlineto{\pgfqpoint{1.489437in}{1.211038in}}%
\pgfpathlineto{\pgfqpoint{1.490701in}{1.237308in}}%
\pgfpathlineto{\pgfqpoint{1.491122in}{1.211038in}}%
\pgfpathlineto{\pgfqpoint{1.491965in}{1.224173in}}%
\pgfpathlineto{\pgfqpoint{1.492807in}{1.224173in}}%
\pgfpathlineto{\pgfqpoint{1.493228in}{1.211038in}}%
\pgfpathlineto{\pgfqpoint{1.493650in}{1.224173in}}%
\pgfpathlineto{\pgfqpoint{1.494071in}{1.224173in}}%
\pgfpathlineto{\pgfqpoint{1.494492in}{1.237308in}}%
\pgfpathlineto{\pgfqpoint{1.494914in}{1.224173in}}%
\pgfpathlineto{\pgfqpoint{1.495756in}{1.224173in}}%
\pgfpathlineto{\pgfqpoint{1.496177in}{1.197903in}}%
\pgfpathlineto{\pgfqpoint{1.496599in}{1.237308in}}%
\pgfpathlineto{\pgfqpoint{1.497862in}{1.224173in}}%
\pgfpathlineto{\pgfqpoint{1.498284in}{1.250443in}}%
\pgfpathlineto{\pgfqpoint{1.499126in}{1.237308in}}%
\pgfpathlineto{\pgfqpoint{1.499548in}{1.237308in}}%
\pgfpathlineto{\pgfqpoint{1.499969in}{1.211038in}}%
\pgfpathlineto{\pgfqpoint{1.500390in}{1.224173in}}%
\pgfpathlineto{\pgfqpoint{1.500811in}{1.237308in}}%
\pgfpathlineto{\pgfqpoint{1.501233in}{1.211038in}}%
\pgfpathlineto{\pgfqpoint{1.502075in}{1.224173in}}%
\pgfpathlineto{\pgfqpoint{1.502918in}{1.237308in}}%
\pgfpathlineto{\pgfqpoint{1.504182in}{1.211038in}}%
\pgfpathlineto{\pgfqpoint{1.505445in}{1.224173in}}%
\pgfpathlineto{\pgfqpoint{1.507131in}{1.224173in}}%
\pgfpathlineto{\pgfqpoint{1.507552in}{1.211038in}}%
\pgfpathlineto{\pgfqpoint{1.507973in}{1.224173in}}%
\pgfpathlineto{\pgfqpoint{1.508816in}{1.224173in}}%
\pgfpathlineto{\pgfqpoint{1.509237in}{1.211038in}}%
\pgfpathlineto{\pgfqpoint{1.509658in}{1.237308in}}%
\pgfpathlineto{\pgfqpoint{1.510080in}{1.211038in}}%
\pgfpathlineto{\pgfqpoint{1.510501in}{1.211038in}}%
\pgfpathlineto{\pgfqpoint{1.512186in}{1.237308in}}%
\pgfpathlineto{\pgfqpoint{1.513028in}{1.211038in}}%
\pgfpathlineto{\pgfqpoint{1.513450in}{1.224173in}}%
\pgfpathlineto{\pgfqpoint{1.513871in}{1.224173in}}%
\pgfpathlineto{\pgfqpoint{1.514292in}{1.211038in}}%
\pgfpathlineto{\pgfqpoint{1.514714in}{1.237308in}}%
\pgfpathlineto{\pgfqpoint{1.515556in}{1.224173in}}%
\pgfpathlineto{\pgfqpoint{1.515977in}{1.224173in}}%
\pgfpathlineto{\pgfqpoint{1.517241in}{1.211038in}}%
\pgfpathlineto{\pgfqpoint{1.518505in}{1.237308in}}%
\pgfpathlineto{\pgfqpoint{1.518926in}{1.211038in}}%
\pgfpathlineto{\pgfqpoint{1.519769in}{1.224173in}}%
\pgfpathlineto{\pgfqpoint{1.520190in}{1.250443in}}%
\pgfpathlineto{\pgfqpoint{1.520611in}{1.211038in}}%
\pgfpathlineto{\pgfqpoint{1.521033in}{1.237308in}}%
\pgfpathlineto{\pgfqpoint{1.521454in}{1.211038in}}%
\pgfpathlineto{\pgfqpoint{1.522297in}{1.224173in}}%
\pgfpathlineto{\pgfqpoint{1.522718in}{1.224173in}}%
\pgfpathlineto{\pgfqpoint{1.523139in}{1.211038in}}%
\pgfpathlineto{\pgfqpoint{1.523982in}{1.237308in}}%
\pgfpathlineto{\pgfqpoint{1.524403in}{1.224173in}}%
\pgfpathlineto{\pgfqpoint{1.526509in}{1.224173in}}%
\pgfpathlineto{\pgfqpoint{1.527773in}{1.237308in}}%
\pgfpathlineto{\pgfqpoint{1.528616in}{1.197903in}}%
\pgfpathlineto{\pgfqpoint{1.529037in}{1.237308in}}%
\pgfpathlineto{\pgfqpoint{1.529880in}{1.224173in}}%
\pgfpathlineto{\pgfqpoint{1.530722in}{1.250443in}}%
\pgfpathlineto{\pgfqpoint{1.531986in}{1.211038in}}%
\pgfpathlineto{\pgfqpoint{1.532829in}{1.237308in}}%
\pgfpathlineto{\pgfqpoint{1.533250in}{1.224173in}}%
\pgfpathlineto{\pgfqpoint{1.533671in}{1.197903in}}%
\pgfpathlineto{\pgfqpoint{1.534092in}{1.237308in}}%
\pgfpathlineto{\pgfqpoint{1.535777in}{1.211038in}}%
\pgfpathlineto{\pgfqpoint{1.537041in}{1.237308in}}%
\pgfpathlineto{\pgfqpoint{1.538726in}{1.211038in}}%
\pgfpathlineto{\pgfqpoint{1.539990in}{1.237308in}}%
\pgfpathlineto{\pgfqpoint{1.541675in}{1.211038in}}%
\pgfpathlineto{\pgfqpoint{1.542097in}{1.237308in}}%
\pgfpathlineto{\pgfqpoint{1.542939in}{1.224173in}}%
\pgfpathlineto{\pgfqpoint{1.543361in}{1.237308in}}%
\pgfpathlineto{\pgfqpoint{1.544624in}{1.211038in}}%
\pgfpathlineto{\pgfqpoint{1.545046in}{1.237308in}}%
\pgfpathlineto{\pgfqpoint{1.545467in}{1.224173in}}%
\pgfpathlineto{\pgfqpoint{1.545888in}{1.197903in}}%
\pgfpathlineto{\pgfqpoint{1.546309in}{1.211038in}}%
\pgfpathlineto{\pgfqpoint{1.547152in}{1.237308in}}%
\pgfpathlineto{\pgfqpoint{1.547573in}{1.224173in}}%
\pgfpathlineto{\pgfqpoint{1.549680in}{1.224173in}}%
\pgfpathlineto{\pgfqpoint{1.550101in}{1.237308in}}%
\pgfpathlineto{\pgfqpoint{1.550522in}{1.197903in}}%
\pgfpathlineto{\pgfqpoint{1.550944in}{1.224173in}}%
\pgfpathlineto{\pgfqpoint{1.551365in}{1.224173in}}%
\pgfpathlineto{\pgfqpoint{1.551786in}{1.250443in}}%
\pgfpathlineto{\pgfqpoint{1.552207in}{1.224173in}}%
\pgfpathlineto{\pgfqpoint{1.552629in}{1.224173in}}%
\pgfpathlineto{\pgfqpoint{1.553050in}{1.211038in}}%
\pgfpathlineto{\pgfqpoint{1.553471in}{1.237308in}}%
\pgfpathlineto{\pgfqpoint{1.553892in}{1.197903in}}%
\pgfpathlineto{\pgfqpoint{1.554314in}{1.224173in}}%
\pgfpathlineto{\pgfqpoint{1.554735in}{1.211038in}}%
\pgfpathlineto{\pgfqpoint{1.555156in}{1.237308in}}%
\pgfpathlineto{\pgfqpoint{1.555578in}{1.224173in}}%
\pgfpathlineto{\pgfqpoint{1.556841in}{1.197903in}}%
\pgfpathlineto{\pgfqpoint{1.558105in}{1.237308in}}%
\pgfpathlineto{\pgfqpoint{1.558527in}{1.211038in}}%
\pgfpathlineto{\pgfqpoint{1.558948in}{1.224173in}}%
\pgfpathlineto{\pgfqpoint{1.560212in}{1.237308in}}%
\pgfpathlineto{\pgfqpoint{1.561054in}{1.211038in}}%
\pgfpathlineto{\pgfqpoint{1.561475in}{1.224173in}}%
\pgfpathlineto{\pgfqpoint{1.561897in}{1.250443in}}%
\pgfpathlineto{\pgfqpoint{1.562739in}{1.237308in}}%
\pgfpathlineto{\pgfqpoint{1.563161in}{1.211038in}}%
\pgfpathlineto{\pgfqpoint{1.564003in}{1.224173in}}%
\pgfpathlineto{\pgfqpoint{1.564424in}{1.224173in}}%
\pgfpathlineto{\pgfqpoint{1.565267in}{1.197903in}}%
\pgfpathlineto{\pgfqpoint{1.565688in}{1.237308in}}%
\pgfpathlineto{\pgfqpoint{1.566531in}{1.224173in}}%
\pgfpathlineto{\pgfqpoint{1.566952in}{1.197903in}}%
\pgfpathlineto{\pgfqpoint{1.567373in}{1.237308in}}%
\pgfpathlineto{\pgfqpoint{1.567795in}{1.224173in}}%
\pgfpathlineto{\pgfqpoint{1.568216in}{1.250443in}}%
\pgfpathlineto{\pgfqpoint{1.568637in}{1.224173in}}%
\pgfpathlineto{\pgfqpoint{1.569901in}{1.224173in}}%
\pgfpathlineto{\pgfqpoint{1.570322in}{1.250443in}}%
\pgfpathlineto{\pgfqpoint{1.570744in}{1.237308in}}%
\pgfpathlineto{\pgfqpoint{1.571165in}{1.211038in}}%
\pgfpathlineto{\pgfqpoint{1.571586in}{1.224173in}}%
\pgfpathlineto{\pgfqpoint{1.572850in}{1.237308in}}%
\pgfpathlineto{\pgfqpoint{1.573693in}{1.211038in}}%
\pgfpathlineto{\pgfqpoint{1.574535in}{1.237308in}}%
\pgfpathlineto{\pgfqpoint{1.574956in}{1.224173in}}%
\pgfpathlineto{\pgfqpoint{1.575378in}{1.211038in}}%
\pgfpathlineto{\pgfqpoint{1.575799in}{1.224173in}}%
\pgfpathlineto{\pgfqpoint{1.576220in}{1.224173in}}%
\pgfpathlineto{\pgfqpoint{1.576641in}{1.237308in}}%
\pgfpathlineto{\pgfqpoint{1.577063in}{1.211038in}}%
\pgfpathlineto{\pgfqpoint{1.577905in}{1.224173in}}%
\pgfpathlineto{\pgfqpoint{1.578327in}{1.224173in}}%
\pgfpathlineto{\pgfqpoint{1.579169in}{1.211038in}}%
\pgfpathlineto{\pgfqpoint{1.580854in}{1.237308in}}%
\pgfpathlineto{\pgfqpoint{1.581697in}{1.211038in}}%
\pgfpathlineto{\pgfqpoint{1.582118in}{1.224173in}}%
\pgfpathlineto{\pgfqpoint{1.582961in}{1.224173in}}%
\pgfpathlineto{\pgfqpoint{1.583382in}{1.237308in}}%
\pgfpathlineto{\pgfqpoint{1.583803in}{1.224173in}}%
\pgfpathlineto{\pgfqpoint{1.585910in}{1.224173in}}%
\pgfpathlineto{\pgfqpoint{1.586331in}{1.211038in}}%
\pgfpathlineto{\pgfqpoint{1.586752in}{1.237308in}}%
\pgfpathlineto{\pgfqpoint{1.587595in}{1.224173in}}%
\pgfpathlineto{\pgfqpoint{1.588859in}{1.224173in}}%
\pgfpathlineto{\pgfqpoint{1.590122in}{1.211038in}}%
\pgfpathlineto{\pgfqpoint{1.591807in}{1.237308in}}%
\pgfpathlineto{\pgfqpoint{1.593071in}{1.211038in}}%
\pgfpathlineto{\pgfqpoint{1.594756in}{1.237308in}}%
\pgfpathlineto{\pgfqpoint{1.595599in}{1.224173in}}%
\pgfpathlineto{\pgfqpoint{1.596020in}{1.237308in}}%
\pgfpathlineto{\pgfqpoint{1.596442in}{1.224173in}}%
\pgfpathlineto{\pgfqpoint{1.596863in}{1.224173in}}%
\pgfpathlineto{\pgfqpoint{1.597284in}{1.211038in}}%
\pgfpathlineto{\pgfqpoint{1.597705in}{1.224173in}}%
\pgfpathlineto{\pgfqpoint{1.598969in}{1.224173in}}%
\pgfpathlineto{\pgfqpoint{1.599390in}{1.211038in}}%
\pgfpathlineto{\pgfqpoint{1.600654in}{1.237308in}}%
\pgfpathlineto{\pgfqpoint{1.602339in}{1.211038in}}%
\pgfpathlineto{\pgfqpoint{1.604025in}{1.237308in}}%
\pgfpathlineto{\pgfqpoint{1.605288in}{1.211038in}}%
\pgfpathlineto{\pgfqpoint{1.606131in}{1.250443in}}%
\pgfpathlineto{\pgfqpoint{1.606552in}{1.224173in}}%
\pgfpathlineto{\pgfqpoint{1.606974in}{1.224173in}}%
\pgfpathlineto{\pgfqpoint{1.607816in}{1.211038in}}%
\pgfpathlineto{\pgfqpoint{1.609080in}{1.224173in}}%
\pgfpathlineto{\pgfqpoint{1.609501in}{1.224173in}}%
\pgfpathlineto{\pgfqpoint{1.610765in}{1.237308in}}%
\pgfpathlineto{\pgfqpoint{1.612029in}{1.224173in}}%
\pgfpathlineto{\pgfqpoint{1.612450in}{1.224173in}}%
\pgfpathlineto{\pgfqpoint{1.612871in}{1.237308in}}%
\pgfpathlineto{\pgfqpoint{1.613293in}{1.224173in}}%
\pgfpathlineto{\pgfqpoint{1.614135in}{1.224173in}}%
\pgfpathlineto{\pgfqpoint{1.614978in}{1.197903in}}%
\pgfpathlineto{\pgfqpoint{1.615399in}{1.211038in}}%
\pgfpathlineto{\pgfqpoint{1.616663in}{1.224173in}}%
\pgfpathlineto{\pgfqpoint{1.617084in}{1.224173in}}%
\pgfpathlineto{\pgfqpoint{1.618348in}{1.211038in}}%
\pgfpathlineto{\pgfqpoint{1.619191in}{1.237308in}}%
\pgfpathlineto{\pgfqpoint{1.619612in}{1.211038in}}%
\pgfpathlineto{\pgfqpoint{1.620033in}{1.237308in}}%
\pgfpathlineto{\pgfqpoint{1.620454in}{1.237308in}}%
\pgfpathlineto{\pgfqpoint{1.621718in}{1.211038in}}%
\pgfpathlineto{\pgfqpoint{1.622982in}{1.224173in}}%
\pgfpathlineto{\pgfqpoint{1.623403in}{1.224173in}}%
\pgfpathlineto{\pgfqpoint{1.624246in}{1.211038in}}%
\pgfpathlineto{\pgfqpoint{1.625510in}{1.237308in}}%
\pgfpathlineto{\pgfqpoint{1.626352in}{1.211038in}}%
\pgfpathlineto{\pgfqpoint{1.626774in}{1.224173in}}%
\pgfpathlineto{\pgfqpoint{1.627616in}{1.211038in}}%
\pgfpathlineto{\pgfqpoint{1.628880in}{1.237308in}}%
\pgfpathlineto{\pgfqpoint{1.629301in}{1.237308in}}%
\pgfpathlineto{\pgfqpoint{1.630565in}{1.211038in}}%
\pgfpathlineto{\pgfqpoint{1.631408in}{1.250443in}}%
\pgfpathlineto{\pgfqpoint{1.631829in}{1.224173in}}%
\pgfpathlineto{\pgfqpoint{1.632250in}{1.224173in}}%
\pgfpathlineto{\pgfqpoint{1.632671in}{1.197903in}}%
\pgfpathlineto{\pgfqpoint{1.633093in}{1.224173in}}%
\pgfpathlineto{\pgfqpoint{1.636463in}{1.224173in}}%
\pgfpathlineto{\pgfqpoint{1.637306in}{1.211038in}}%
\pgfpathlineto{\pgfqpoint{1.637727in}{1.250443in}}%
\pgfpathlineto{\pgfqpoint{1.638148in}{1.224173in}}%
\pgfpathlineto{\pgfqpoint{1.638991in}{1.211038in}}%
\pgfpathlineto{\pgfqpoint{1.639412in}{1.237308in}}%
\pgfpathlineto{\pgfqpoint{1.640254in}{1.224173in}}%
\pgfpathlineto{\pgfqpoint{1.640676in}{1.211038in}}%
\pgfpathlineto{\pgfqpoint{1.641097in}{1.237308in}}%
\pgfpathlineto{\pgfqpoint{1.641518in}{1.224173in}}%
\pgfpathlineto{\pgfqpoint{1.641940in}{1.197903in}}%
\pgfpathlineto{\pgfqpoint{1.642782in}{1.211038in}}%
\pgfpathlineto{\pgfqpoint{1.643203in}{1.237308in}}%
\pgfpathlineto{\pgfqpoint{1.644046in}{1.224173in}}%
\pgfpathlineto{\pgfqpoint{1.644889in}{1.197903in}}%
\pgfpathlineto{\pgfqpoint{1.646152in}{1.237308in}}%
\pgfpathlineto{\pgfqpoint{1.647416in}{1.211038in}}%
\pgfpathlineto{\pgfqpoint{1.648680in}{1.237308in}}%
\pgfpathlineto{\pgfqpoint{1.649101in}{1.211038in}}%
\pgfpathlineto{\pgfqpoint{1.649944in}{1.224173in}}%
\pgfpathlineto{\pgfqpoint{1.650786in}{1.237308in}}%
\pgfpathlineto{\pgfqpoint{1.651208in}{1.197903in}}%
\pgfpathlineto{\pgfqpoint{1.651629in}{1.224173in}}%
\pgfpathlineto{\pgfqpoint{1.652050in}{1.237308in}}%
\pgfpathlineto{\pgfqpoint{1.652472in}{1.224173in}}%
\pgfpathlineto{\pgfqpoint{1.652893in}{1.211038in}}%
\pgfpathlineto{\pgfqpoint{1.653314in}{1.224173in}}%
\pgfpathlineto{\pgfqpoint{1.653735in}{1.224173in}}%
\pgfpathlineto{\pgfqpoint{1.654157in}{1.250443in}}%
\pgfpathlineto{\pgfqpoint{1.654578in}{1.224173in}}%
\pgfpathlineto{\pgfqpoint{1.654999in}{1.211038in}}%
\pgfpathlineto{\pgfqpoint{1.655420in}{1.250443in}}%
\pgfpathlineto{\pgfqpoint{1.655842in}{1.211038in}}%
\pgfpathlineto{\pgfqpoint{1.656684in}{1.224173in}}%
\pgfpathlineto{\pgfqpoint{1.657106in}{1.211038in}}%
\pgfpathlineto{\pgfqpoint{1.658369in}{1.237308in}}%
\pgfpathlineto{\pgfqpoint{1.658791in}{1.237308in}}%
\pgfpathlineto{\pgfqpoint{1.659633in}{1.211038in}}%
\pgfpathlineto{\pgfqpoint{1.660055in}{1.224173in}}%
\pgfpathlineto{\pgfqpoint{1.660476in}{1.224173in}}%
\pgfpathlineto{\pgfqpoint{1.661318in}{1.237308in}}%
\pgfpathlineto{\pgfqpoint{1.662582in}{1.211038in}}%
\pgfpathlineto{\pgfqpoint{1.663003in}{1.211038in}}%
\pgfpathlineto{\pgfqpoint{1.663846in}{1.224173in}}%
\pgfpathlineto{\pgfqpoint{1.664267in}{1.211038in}}%
\pgfpathlineto{\pgfqpoint{1.664689in}{1.237308in}}%
\pgfpathlineto{\pgfqpoint{1.665110in}{1.211038in}}%
\pgfpathlineto{\pgfqpoint{1.665952in}{1.211038in}}%
\pgfpathlineto{\pgfqpoint{1.666374in}{1.224173in}}%
\pgfpathlineto{\pgfqpoint{1.666795in}{1.211038in}}%
\pgfpathlineto{\pgfqpoint{1.667216in}{1.211038in}}%
\pgfpathlineto{\pgfqpoint{1.668480in}{1.263578in}}%
\pgfpathlineto{\pgfqpoint{1.668901in}{1.211038in}}%
\pgfpathlineto{\pgfqpoint{1.669744in}{1.237308in}}%
\pgfpathlineto{\pgfqpoint{1.671008in}{1.224173in}}%
\pgfpathlineto{\pgfqpoint{1.671429in}{1.237308in}}%
\pgfpathlineto{\pgfqpoint{1.671850in}{1.224173in}}%
\pgfpathlineto{\pgfqpoint{1.672272in}{1.211038in}}%
\pgfpathlineto{\pgfqpoint{1.672693in}{1.224173in}}%
\pgfpathlineto{\pgfqpoint{1.673957in}{1.224173in}}%
\pgfpathlineto{\pgfqpoint{1.674799in}{1.211038in}}%
\pgfpathlineto{\pgfqpoint{1.676484in}{1.250443in}}%
\pgfpathlineto{\pgfqpoint{1.677327in}{1.211038in}}%
\pgfpathlineto{\pgfqpoint{1.677748in}{1.237308in}}%
\pgfpathlineto{\pgfqpoint{1.678591in}{1.224173in}}%
\pgfpathlineto{\pgfqpoint{1.679855in}{1.224173in}}%
\pgfpathlineto{\pgfqpoint{1.680276in}{0.777580in}}%
\pgfpathlineto{\pgfqpoint{1.680697in}{1.224173in}}%
\pgfpathlineto{\pgfqpoint{1.681118in}{1.211038in}}%
\pgfpathlineto{\pgfqpoint{1.681540in}{1.237308in}}%
\pgfpathlineto{\pgfqpoint{1.682382in}{1.224173in}}%
\pgfpathlineto{\pgfqpoint{1.682804in}{1.224173in}}%
\pgfpathlineto{\pgfqpoint{1.683225in}{1.211038in}}%
\pgfpathlineto{\pgfqpoint{1.684489in}{1.237308in}}%
\pgfpathlineto{\pgfqpoint{1.684910in}{1.211038in}}%
\pgfpathlineto{\pgfqpoint{1.685331in}{1.224173in}}%
\pgfpathlineto{\pgfqpoint{1.685753in}{1.237308in}}%
\pgfpathlineto{\pgfqpoint{1.686174in}{1.224173in}}%
\pgfpathlineto{\pgfqpoint{1.686595in}{1.224173in}}%
\pgfpathlineto{\pgfqpoint{1.687016in}{1.211038in}}%
\pgfpathlineto{\pgfqpoint{1.687438in}{1.224173in}}%
\pgfpathlineto{\pgfqpoint{1.688280in}{1.237308in}}%
\pgfpathlineto{\pgfqpoint{1.689544in}{1.211038in}}%
\pgfpathlineto{\pgfqpoint{1.690808in}{1.237308in}}%
\pgfpathlineto{\pgfqpoint{1.691229in}{1.224173in}}%
\pgfpathlineto{\pgfqpoint{1.691650in}{1.237308in}}%
\pgfpathlineto{\pgfqpoint{1.692072in}{1.237308in}}%
\pgfpathlineto{\pgfqpoint{1.693336in}{1.224173in}}%
\pgfpathlineto{\pgfqpoint{1.693757in}{1.683902in}}%
\pgfpathlineto{\pgfqpoint{1.694178in}{1.224173in}}%
\pgfpathlineto{\pgfqpoint{1.694599in}{1.224173in}}%
\pgfpathlineto{\pgfqpoint{1.695021in}{1.237308in}}%
\pgfpathlineto{\pgfqpoint{1.695442in}{1.211038in}}%
\pgfpathlineto{\pgfqpoint{1.695863in}{1.224173in}}%
\pgfpathlineto{\pgfqpoint{1.696706in}{1.237308in}}%
\pgfpathlineto{\pgfqpoint{1.697970in}{1.197903in}}%
\pgfpathlineto{\pgfqpoint{1.699233in}{1.224173in}}%
\pgfpathlineto{\pgfqpoint{1.699655in}{1.224173in}}%
\pgfpathlineto{\pgfqpoint{1.700919in}{1.237308in}}%
\pgfpathlineto{\pgfqpoint{1.701761in}{1.224173in}}%
\pgfpathlineto{\pgfqpoint{1.703025in}{1.237308in}}%
\pgfpathlineto{\pgfqpoint{1.703446in}{1.211038in}}%
\pgfpathlineto{\pgfqpoint{1.704289in}{1.224173in}}%
\pgfpathlineto{\pgfqpoint{1.704710in}{1.211038in}}%
\pgfpathlineto{\pgfqpoint{1.705131in}{1.224173in}}%
\pgfpathlineto{\pgfqpoint{1.706395in}{1.237308in}}%
\pgfpathlineto{\pgfqpoint{1.707238in}{1.237308in}}%
\pgfpathlineto{\pgfqpoint{1.708502in}{1.197903in}}%
\pgfpathlineto{\pgfqpoint{1.709765in}{1.224173in}}%
\pgfpathlineto{\pgfqpoint{1.710187in}{1.211038in}}%
\pgfpathlineto{\pgfqpoint{1.710608in}{1.224173in}}%
\pgfpathlineto{\pgfqpoint{1.711029in}{1.224173in}}%
\pgfpathlineto{\pgfqpoint{1.711872in}{1.237308in}}%
\pgfpathlineto{\pgfqpoint{1.713557in}{1.211038in}}%
\pgfpathlineto{\pgfqpoint{1.713978in}{1.211038in}}%
\pgfpathlineto{\pgfqpoint{1.714821in}{1.237308in}}%
\pgfpathlineto{\pgfqpoint{1.716085in}{1.211038in}}%
\pgfpathlineto{\pgfqpoint{1.716927in}{1.224173in}}%
\pgfpathlineto{\pgfqpoint{1.717348in}{1.211038in}}%
\pgfpathlineto{\pgfqpoint{1.717770in}{1.224173in}}%
\pgfpathlineto{\pgfqpoint{1.718191in}{1.237308in}}%
\pgfpathlineto{\pgfqpoint{1.719455in}{1.211038in}}%
\pgfpathlineto{\pgfqpoint{1.719876in}{1.211038in}}%
\pgfpathlineto{\pgfqpoint{1.721140in}{1.224173in}}%
\pgfpathlineto{\pgfqpoint{1.721561in}{1.224173in}}%
\pgfpathlineto{\pgfqpoint{1.722404in}{1.237308in}}%
\pgfpathlineto{\pgfqpoint{1.723668in}{1.224173in}}%
\pgfpathlineto{\pgfqpoint{1.724510in}{1.224173in}}%
\pgfpathlineto{\pgfqpoint{1.725774in}{1.211038in}}%
\pgfpathlineto{\pgfqpoint{1.727459in}{1.237308in}}%
\pgfpathlineto{\pgfqpoint{1.728302in}{1.211038in}}%
\pgfpathlineto{\pgfqpoint{1.728723in}{1.224173in}}%
\pgfpathlineto{\pgfqpoint{1.729144in}{1.224173in}}%
\pgfpathlineto{\pgfqpoint{1.730408in}{1.237308in}}%
\pgfpathlineto{\pgfqpoint{1.730829in}{1.237308in}}%
\pgfpathlineto{\pgfqpoint{1.731672in}{1.197903in}}%
\pgfpathlineto{\pgfqpoint{1.732093in}{1.237308in}}%
\pgfpathlineto{\pgfqpoint{1.732936in}{1.224173in}}%
\pgfpathlineto{\pgfqpoint{1.733778in}{1.211038in}}%
\pgfpathlineto{\pgfqpoint{1.734621in}{1.237308in}}%
\pgfpathlineto{\pgfqpoint{1.735042in}{1.224173in}}%
\pgfpathlineto{\pgfqpoint{1.735885in}{1.211038in}}%
\pgfpathlineto{\pgfqpoint{1.737148in}{1.237308in}}%
\pgfpathlineto{\pgfqpoint{1.737991in}{1.211038in}}%
\pgfpathlineto{\pgfqpoint{1.738412in}{1.224173in}}%
\pgfpathlineto{\pgfqpoint{1.738834in}{1.224173in}}%
\pgfpathlineto{\pgfqpoint{1.739255in}{1.237308in}}%
\pgfpathlineto{\pgfqpoint{1.739676in}{1.224173in}}%
\pgfpathlineto{\pgfqpoint{1.740097in}{1.211038in}}%
\pgfpathlineto{\pgfqpoint{1.741361in}{1.237308in}}%
\pgfpathlineto{\pgfqpoint{1.741783in}{1.211038in}}%
\pgfpathlineto{\pgfqpoint{1.742204in}{1.237308in}}%
\pgfpathlineto{\pgfqpoint{1.742625in}{1.237308in}}%
\pgfpathlineto{\pgfqpoint{1.743889in}{1.211038in}}%
\pgfpathlineto{\pgfqpoint{1.744731in}{1.211038in}}%
\pgfpathlineto{\pgfqpoint{1.745153in}{1.237308in}}%
\pgfpathlineto{\pgfqpoint{1.745574in}{1.211038in}}%
\pgfpathlineto{\pgfqpoint{1.745995in}{1.211038in}}%
\pgfpathlineto{\pgfqpoint{1.747259in}{1.237308in}}%
\pgfpathlineto{\pgfqpoint{1.748523in}{1.224173in}}%
\pgfpathlineto{\pgfqpoint{1.749366in}{1.237308in}}%
\pgfpathlineto{\pgfqpoint{1.750629in}{1.211038in}}%
\pgfpathlineto{\pgfqpoint{1.752314in}{1.237308in}}%
\pgfpathlineto{\pgfqpoint{1.754000in}{1.211038in}}%
\pgfpathlineto{\pgfqpoint{1.754421in}{1.250443in}}%
\pgfpathlineto{\pgfqpoint{1.755263in}{1.237308in}}%
\pgfpathlineto{\pgfqpoint{1.756106in}{1.197903in}}%
\pgfpathlineto{\pgfqpoint{1.756527in}{1.237308in}}%
\pgfpathlineto{\pgfqpoint{1.757370in}{1.224173in}}%
\pgfpathlineto{\pgfqpoint{1.758634in}{1.237308in}}%
\pgfpathlineto{\pgfqpoint{1.759476in}{1.237308in}}%
\pgfpathlineto{\pgfqpoint{1.760740in}{1.224173in}}%
\pgfpathlineto{\pgfqpoint{1.763268in}{1.224173in}}%
\pgfpathlineto{\pgfqpoint{1.764953in}{1.197903in}}%
\pgfpathlineto{\pgfqpoint{1.765374in}{1.237308in}}%
\pgfpathlineto{\pgfqpoint{1.766217in}{1.224173in}}%
\pgfpathlineto{\pgfqpoint{1.766638in}{1.211038in}}%
\pgfpathlineto{\pgfqpoint{1.767480in}{1.237308in}}%
\pgfpathlineto{\pgfqpoint{1.767902in}{1.197903in}}%
\pgfpathlineto{\pgfqpoint{1.768323in}{1.224173in}}%
\pgfpathlineto{\pgfqpoint{1.769166in}{1.224173in}}%
\pgfpathlineto{\pgfqpoint{1.769587in}{1.237308in}}%
\pgfpathlineto{\pgfqpoint{1.770008in}{1.224173in}}%
\pgfpathlineto{\pgfqpoint{1.770429in}{1.224173in}}%
\pgfpathlineto{\pgfqpoint{1.771693in}{1.237308in}}%
\pgfpathlineto{\pgfqpoint{1.772536in}{1.224173in}}%
\pgfpathlineto{\pgfqpoint{1.772957in}{1.237308in}}%
\pgfpathlineto{\pgfqpoint{1.773800in}{1.211038in}}%
\pgfpathlineto{\pgfqpoint{1.774221in}{1.224173in}}%
\pgfpathlineto{\pgfqpoint{1.774642in}{1.237308in}}%
\pgfpathlineto{\pgfqpoint{1.775906in}{1.211038in}}%
\pgfpathlineto{\pgfqpoint{1.776327in}{1.211038in}}%
\pgfpathlineto{\pgfqpoint{1.776749in}{1.250443in}}%
\pgfpathlineto{\pgfqpoint{1.777591in}{1.237308in}}%
\pgfpathlineto{\pgfqpoint{1.778434in}{1.250443in}}%
\pgfpathlineto{\pgfqpoint{1.779698in}{1.211038in}}%
\pgfpathlineto{\pgfqpoint{1.780540in}{1.224173in}}%
\pgfpathlineto{\pgfqpoint{1.780961in}{1.211038in}}%
\pgfpathlineto{\pgfqpoint{1.781383in}{1.237308in}}%
\pgfpathlineto{\pgfqpoint{1.781804in}{1.224173in}}%
\pgfpathlineto{\pgfqpoint{1.782225in}{1.211038in}}%
\pgfpathlineto{\pgfqpoint{1.783068in}{1.237308in}}%
\pgfpathlineto{\pgfqpoint{1.784332in}{1.197903in}}%
\pgfpathlineto{\pgfqpoint{1.785595in}{1.224173in}}%
\pgfpathlineto{\pgfqpoint{1.786438in}{1.211038in}}%
\pgfpathlineto{\pgfqpoint{1.787281in}{1.237308in}}%
\pgfpathlineto{\pgfqpoint{1.787702in}{1.224173in}}%
\pgfpathlineto{\pgfqpoint{1.788123in}{1.211038in}}%
\pgfpathlineto{\pgfqpoint{1.788966in}{1.237308in}}%
\pgfpathlineto{\pgfqpoint{1.789387in}{1.224173in}}%
\pgfpathlineto{\pgfqpoint{1.789808in}{1.224173in}}%
\pgfpathlineto{\pgfqpoint{1.790229in}{1.211038in}}%
\pgfpathlineto{\pgfqpoint{1.790651in}{1.237308in}}%
\pgfpathlineto{\pgfqpoint{1.791072in}{1.224173in}}%
\pgfpathlineto{\pgfqpoint{1.791493in}{1.197903in}}%
\pgfpathlineto{\pgfqpoint{1.792336in}{1.211038in}}%
\pgfpathlineto{\pgfqpoint{1.793600in}{1.237308in}}%
\pgfpathlineto{\pgfqpoint{1.795285in}{1.211038in}}%
\pgfpathlineto{\pgfqpoint{1.795706in}{1.237308in}}%
\pgfpathlineto{\pgfqpoint{1.796127in}{1.211038in}}%
\pgfpathlineto{\pgfqpoint{1.796549in}{1.211038in}}%
\pgfpathlineto{\pgfqpoint{1.797391in}{1.197903in}}%
\pgfpathlineto{\pgfqpoint{1.797812in}{1.250443in}}%
\pgfpathlineto{\pgfqpoint{1.799076in}{1.211038in}}%
\pgfpathlineto{\pgfqpoint{1.799498in}{1.224173in}}%
\pgfpathlineto{\pgfqpoint{1.800340in}{1.224173in}}%
\pgfpathlineto{\pgfqpoint{1.800761in}{1.237308in}}%
\pgfpathlineto{\pgfqpoint{1.801604in}{1.211038in}}%
\pgfpathlineto{\pgfqpoint{1.802025in}{1.224173in}}%
\pgfpathlineto{\pgfqpoint{1.802447in}{1.211038in}}%
\pgfpathlineto{\pgfqpoint{1.803289in}{1.237308in}}%
\pgfpathlineto{\pgfqpoint{1.803710in}{1.224173in}}%
\pgfpathlineto{\pgfqpoint{1.804132in}{1.224173in}}%
\pgfpathlineto{\pgfqpoint{1.804553in}{1.211038in}}%
\pgfpathlineto{\pgfqpoint{1.805817in}{1.237308in}}%
\pgfpathlineto{\pgfqpoint{1.806238in}{1.237308in}}%
\pgfpathlineto{\pgfqpoint{1.807502in}{1.211038in}}%
\pgfpathlineto{\pgfqpoint{1.808766in}{1.250443in}}%
\pgfpathlineto{\pgfqpoint{1.809187in}{1.224173in}}%
\pgfpathlineto{\pgfqpoint{1.810030in}{1.237308in}}%
\pgfpathlineto{\pgfqpoint{1.810451in}{1.224173in}}%
\pgfpathlineto{\pgfqpoint{1.810872in}{1.250443in}}%
\pgfpathlineto{\pgfqpoint{1.811293in}{1.211038in}}%
\pgfpathlineto{\pgfqpoint{1.812979in}{1.237308in}}%
\pgfpathlineto{\pgfqpoint{1.814664in}{1.211038in}}%
\pgfpathlineto{\pgfqpoint{1.815927in}{1.237308in}}%
\pgfpathlineto{\pgfqpoint{1.817191in}{1.224173in}}%
\pgfpathlineto{\pgfqpoint{1.818455in}{1.224173in}}%
\pgfpathlineto{\pgfqpoint{1.818876in}{1.250443in}}%
\pgfpathlineto{\pgfqpoint{1.819298in}{1.224173in}}%
\pgfpathlineto{\pgfqpoint{1.820562in}{1.197903in}}%
\pgfpathlineto{\pgfqpoint{1.820983in}{1.237308in}}%
\pgfpathlineto{\pgfqpoint{1.821825in}{1.224173in}}%
\pgfpathlineto{\pgfqpoint{1.822247in}{1.211038in}}%
\pgfpathlineto{\pgfqpoint{1.822668in}{1.224173in}}%
\pgfpathlineto{\pgfqpoint{1.823510in}{1.224173in}}%
\pgfpathlineto{\pgfqpoint{1.823932in}{1.211038in}}%
\pgfpathlineto{\pgfqpoint{1.824353in}{1.224173in}}%
\pgfpathlineto{\pgfqpoint{1.824774in}{1.237308in}}%
\pgfpathlineto{\pgfqpoint{1.825196in}{1.211038in}}%
\pgfpathlineto{\pgfqpoint{1.825617in}{1.224173in}}%
\pgfpathlineto{\pgfqpoint{1.826038in}{1.237308in}}%
\pgfpathlineto{\pgfqpoint{1.826459in}{1.224173in}}%
\pgfpathlineto{\pgfqpoint{1.826881in}{1.211038in}}%
\pgfpathlineto{\pgfqpoint{1.828145in}{1.237308in}}%
\pgfpathlineto{\pgfqpoint{1.828566in}{1.237308in}}%
\pgfpathlineto{\pgfqpoint{1.830251in}{1.211038in}}%
\pgfpathlineto{\pgfqpoint{1.831515in}{1.237308in}}%
\pgfpathlineto{\pgfqpoint{1.832779in}{1.211038in}}%
\pgfpathlineto{\pgfqpoint{1.833621in}{1.237308in}}%
\pgfpathlineto{\pgfqpoint{1.834042in}{1.224173in}}%
\pgfpathlineto{\pgfqpoint{1.834464in}{1.224173in}}%
\pgfpathlineto{\pgfqpoint{1.835306in}{1.211038in}}%
\pgfpathlineto{\pgfqpoint{1.836570in}{1.224173in}}%
\pgfpathlineto{\pgfqpoint{1.837413in}{1.224173in}}%
\pgfpathlineto{\pgfqpoint{1.837834in}{1.211038in}}%
\pgfpathlineto{\pgfqpoint{1.838255in}{1.224173in}}%
\pgfpathlineto{\pgfqpoint{1.838676in}{1.237308in}}%
\pgfpathlineto{\pgfqpoint{1.839940in}{1.211038in}}%
\pgfpathlineto{\pgfqpoint{1.840362in}{1.250443in}}%
\pgfpathlineto{\pgfqpoint{1.841204in}{1.237308in}}%
\pgfpathlineto{\pgfqpoint{1.842468in}{1.224173in}}%
\pgfpathlineto{\pgfqpoint{1.842889in}{1.224173in}}%
\pgfpathlineto{\pgfqpoint{1.843732in}{1.237308in}}%
\pgfpathlineto{\pgfqpoint{1.844996in}{1.211038in}}%
\pgfpathlineto{\pgfqpoint{1.845417in}{1.211038in}}%
\pgfpathlineto{\pgfqpoint{1.846259in}{1.263578in}}%
\pgfpathlineto{\pgfqpoint{1.846681in}{1.211038in}}%
\pgfpathlineto{\pgfqpoint{1.847523in}{1.237308in}}%
\pgfpathlineto{\pgfqpoint{1.847945in}{1.224173in}}%
\pgfpathlineto{\pgfqpoint{1.848366in}{1.263578in}}%
\pgfpathlineto{\pgfqpoint{1.848787in}{1.237308in}}%
\pgfpathlineto{\pgfqpoint{1.850051in}{1.211038in}}%
\pgfpathlineto{\pgfqpoint{1.850472in}{1.211038in}}%
\pgfpathlineto{\pgfqpoint{1.851736in}{1.237308in}}%
\pgfpathlineto{\pgfqpoint{1.852157in}{1.211038in}}%
\pgfpathlineto{\pgfqpoint{1.853000in}{1.224173in}}%
\pgfpathlineto{\pgfqpoint{1.853421in}{1.237308in}}%
\pgfpathlineto{\pgfqpoint{1.854264in}{1.211038in}}%
\pgfpathlineto{\pgfqpoint{1.854685in}{1.224173in}}%
\pgfpathlineto{\pgfqpoint{1.855528in}{1.224173in}}%
\pgfpathlineto{\pgfqpoint{1.855949in}{1.211038in}}%
\pgfpathlineto{\pgfqpoint{1.857634in}{1.250443in}}%
\pgfpathlineto{\pgfqpoint{1.858055in}{1.211038in}}%
\pgfpathlineto{\pgfqpoint{1.858898in}{1.224173in}}%
\pgfpathlineto{\pgfqpoint{1.859740in}{1.197903in}}%
\pgfpathlineto{\pgfqpoint{1.861004in}{1.224173in}}%
\pgfpathlineto{\pgfqpoint{1.861425in}{1.224173in}}%
\pgfpathlineto{\pgfqpoint{1.861847in}{1.197903in}}%
\pgfpathlineto{\pgfqpoint{1.862268in}{1.211038in}}%
\pgfpathlineto{\pgfqpoint{1.863953in}{1.237308in}}%
\pgfpathlineto{\pgfqpoint{1.865217in}{1.197903in}}%
\pgfpathlineto{\pgfqpoint{1.866481in}{1.237308in}}%
\pgfpathlineto{\pgfqpoint{1.866902in}{1.211038in}}%
\pgfpathlineto{\pgfqpoint{1.867323in}{1.224173in}}%
\pgfpathlineto{\pgfqpoint{1.867745in}{1.237308in}}%
\pgfpathlineto{\pgfqpoint{1.868166in}{1.211038in}}%
\pgfpathlineto{\pgfqpoint{1.869008in}{1.224173in}}%
\pgfpathlineto{\pgfqpoint{1.869851in}{1.224173in}}%
\pgfpathlineto{\pgfqpoint{1.871115in}{1.211038in}}%
\pgfpathlineto{\pgfqpoint{1.871957in}{1.237308in}}%
\pgfpathlineto{\pgfqpoint{1.872379in}{1.224173in}}%
\pgfpathlineto{\pgfqpoint{1.872800in}{1.237308in}}%
\pgfpathlineto{\pgfqpoint{1.873221in}{1.211038in}}%
\pgfpathlineto{\pgfqpoint{1.873643in}{1.237308in}}%
\pgfpathlineto{\pgfqpoint{1.874064in}{1.237308in}}%
\pgfpathlineto{\pgfqpoint{1.874906in}{1.211038in}}%
\pgfpathlineto{\pgfqpoint{1.875328in}{1.250443in}}%
\pgfpathlineto{\pgfqpoint{1.876170in}{1.237308in}}%
\pgfpathlineto{\pgfqpoint{1.877013in}{1.211038in}}%
\pgfpathlineto{\pgfqpoint{1.877434in}{1.224173in}}%
\pgfpathlineto{\pgfqpoint{1.877855in}{1.250443in}}%
\pgfpathlineto{\pgfqpoint{1.878277in}{1.224173in}}%
\pgfpathlineto{\pgfqpoint{1.879540in}{1.224173in}}%
\pgfpathlineto{\pgfqpoint{1.879962in}{1.211038in}}%
\pgfpathlineto{\pgfqpoint{1.880383in}{1.224173in}}%
\pgfpathlineto{\pgfqpoint{1.880804in}{1.237308in}}%
\pgfpathlineto{\pgfqpoint{1.881226in}{1.211038in}}%
\pgfpathlineto{\pgfqpoint{1.881647in}{1.237308in}}%
\pgfpathlineto{\pgfqpoint{1.882068in}{1.237308in}}%
\pgfpathlineto{\pgfqpoint{1.883332in}{1.211038in}}%
\pgfpathlineto{\pgfqpoint{1.884175in}{1.237308in}}%
\pgfpathlineto{\pgfqpoint{1.885438in}{1.197903in}}%
\pgfpathlineto{\pgfqpoint{1.886702in}{1.237308in}}%
\pgfpathlineto{\pgfqpoint{1.887545in}{1.211038in}}%
\pgfpathlineto{\pgfqpoint{1.888387in}{1.237308in}}%
\pgfpathlineto{\pgfqpoint{1.889230in}{1.211038in}}%
\pgfpathlineto{\pgfqpoint{1.889651in}{1.224173in}}%
\pgfpathlineto{\pgfqpoint{1.891336in}{1.197903in}}%
\pgfpathlineto{\pgfqpoint{1.891758in}{1.224173in}}%
\pgfpathlineto{\pgfqpoint{1.892600in}{1.211038in}}%
\pgfpathlineto{\pgfqpoint{1.893021in}{1.211038in}}%
\pgfpathlineto{\pgfqpoint{1.893864in}{1.237308in}}%
\pgfpathlineto{\pgfqpoint{1.894285in}{1.184768in}}%
\pgfpathlineto{\pgfqpoint{1.895128in}{1.211038in}}%
\pgfpathlineto{\pgfqpoint{1.895549in}{1.211038in}}%
\pgfpathlineto{\pgfqpoint{1.896813in}{1.237308in}}%
\pgfpathlineto{\pgfqpoint{1.897655in}{1.211038in}}%
\pgfpathlineto{\pgfqpoint{1.898077in}{1.237308in}}%
\pgfpathlineto{\pgfqpoint{1.898919in}{1.224173in}}%
\pgfpathlineto{\pgfqpoint{1.899762in}{1.224173in}}%
\pgfpathlineto{\pgfqpoint{1.900183in}{1.211038in}}%
\pgfpathlineto{\pgfqpoint{1.900604in}{1.224173in}}%
\pgfpathlineto{\pgfqpoint{1.901447in}{1.224173in}}%
\pgfpathlineto{\pgfqpoint{1.902289in}{1.211038in}}%
\pgfpathlineto{\pgfqpoint{1.903975in}{1.250443in}}%
\pgfpathlineto{\pgfqpoint{1.904817in}{1.211038in}}%
\pgfpathlineto{\pgfqpoint{1.905238in}{1.224173in}}%
\pgfpathlineto{\pgfqpoint{1.906502in}{1.237308in}}%
\pgfpathlineto{\pgfqpoint{1.906924in}{1.237308in}}%
\pgfpathlineto{\pgfqpoint{1.907345in}{1.211038in}}%
\pgfpathlineto{\pgfqpoint{1.907766in}{1.250443in}}%
\pgfpathlineto{\pgfqpoint{1.908609in}{1.224173in}}%
\pgfpathlineto{\pgfqpoint{1.909030in}{1.237308in}}%
\pgfpathlineto{\pgfqpoint{1.909451in}{1.237308in}}%
\pgfpathlineto{\pgfqpoint{1.910294in}{1.211038in}}%
\pgfpathlineto{\pgfqpoint{1.910715in}{1.224173in}}%
\pgfpathlineto{\pgfqpoint{1.912400in}{1.224173in}}%
\pgfpathlineto{\pgfqpoint{1.912821in}{1.211038in}}%
\pgfpathlineto{\pgfqpoint{1.913243in}{1.224173in}}%
\pgfpathlineto{\pgfqpoint{1.914507in}{1.224173in}}%
\pgfpathlineto{\pgfqpoint{1.914928in}{1.211038in}}%
\pgfpathlineto{\pgfqpoint{1.915349in}{1.224173in}}%
\pgfpathlineto{\pgfqpoint{1.916192in}{1.224173in}}%
\pgfpathlineto{\pgfqpoint{1.916613in}{1.211038in}}%
\pgfpathlineto{\pgfqpoint{1.917034in}{1.224173in}}%
\pgfpathlineto{\pgfqpoint{1.917455in}{1.224173in}}%
\pgfpathlineto{\pgfqpoint{1.918298in}{1.237308in}}%
\pgfpathlineto{\pgfqpoint{1.918719in}{1.211038in}}%
\pgfpathlineto{\pgfqpoint{1.919562in}{1.224173in}}%
\pgfpathlineto{\pgfqpoint{1.919983in}{1.224173in}}%
\pgfpathlineto{\pgfqpoint{1.921247in}{1.211038in}}%
\pgfpathlineto{\pgfqpoint{1.922090in}{1.237308in}}%
\pgfpathlineto{\pgfqpoint{1.922932in}{1.211038in}}%
\pgfpathlineto{\pgfqpoint{1.923353in}{1.237308in}}%
\pgfpathlineto{\pgfqpoint{1.924196in}{1.224173in}}%
\pgfpathlineto{\pgfqpoint{1.925038in}{1.224173in}}%
\pgfpathlineto{\pgfqpoint{1.926302in}{1.237308in}}%
\pgfpathlineto{\pgfqpoint{1.926724in}{1.237308in}}%
\pgfpathlineto{\pgfqpoint{1.927566in}{1.224173in}}%
\pgfpathlineto{\pgfqpoint{1.928830in}{1.237308in}}%
\pgfpathlineto{\pgfqpoint{1.929673in}{1.237308in}}%
\pgfpathlineto{\pgfqpoint{1.931358in}{1.211038in}}%
\pgfpathlineto{\pgfqpoint{1.933043in}{1.237308in}}%
\pgfpathlineto{\pgfqpoint{1.933464in}{1.224173in}}%
\pgfpathlineto{\pgfqpoint{1.933885in}{1.237308in}}%
\pgfpathlineto{\pgfqpoint{1.934728in}{1.237308in}}%
\pgfpathlineto{\pgfqpoint{1.935570in}{1.211038in}}%
\pgfpathlineto{\pgfqpoint{1.935992in}{1.224173in}}%
\pgfpathlineto{\pgfqpoint{1.936834in}{1.224173in}}%
\pgfpathlineto{\pgfqpoint{1.938098in}{1.250443in}}%
\pgfpathlineto{\pgfqpoint{1.938941in}{1.211038in}}%
\pgfpathlineto{\pgfqpoint{1.939362in}{1.237308in}}%
\pgfpathlineto{\pgfqpoint{1.939783in}{1.224173in}}%
\pgfpathlineto{\pgfqpoint{1.940205in}{1.237308in}}%
\pgfpathlineto{\pgfqpoint{1.940626in}{1.237308in}}%
\pgfpathlineto{\pgfqpoint{1.942311in}{1.211038in}}%
\pgfpathlineto{\pgfqpoint{1.942732in}{1.211038in}}%
\pgfpathlineto{\pgfqpoint{1.943575in}{1.237308in}}%
\pgfpathlineto{\pgfqpoint{1.943996in}{1.211038in}}%
\pgfpathlineto{\pgfqpoint{1.944839in}{1.224173in}}%
\pgfpathlineto{\pgfqpoint{1.945681in}{1.224173in}}%
\pgfpathlineto{\pgfqpoint{1.946102in}{1.211038in}}%
\pgfpathlineto{\pgfqpoint{1.946524in}{1.224173in}}%
\pgfpathlineto{\pgfqpoint{1.947366in}{1.224173in}}%
\pgfpathlineto{\pgfqpoint{1.948209in}{1.211038in}}%
\pgfpathlineto{\pgfqpoint{1.949894in}{1.237308in}}%
\pgfpathlineto{\pgfqpoint{1.951579in}{1.197903in}}%
\pgfpathlineto{\pgfqpoint{1.952000in}{1.237308in}}%
\pgfpathlineto{\pgfqpoint{1.952422in}{1.224173in}}%
\pgfpathlineto{\pgfqpoint{1.953685in}{1.211038in}}%
\pgfpathlineto{\pgfqpoint{1.954107in}{1.211038in}}%
\pgfpathlineto{\pgfqpoint{1.954949in}{1.237308in}}%
\pgfpathlineto{\pgfqpoint{1.955371in}{1.211038in}}%
\pgfpathlineto{\pgfqpoint{1.956213in}{1.224173in}}%
\pgfpathlineto{\pgfqpoint{1.956634in}{1.211038in}}%
\pgfpathlineto{\pgfqpoint{1.957056in}{1.224173in}}%
\pgfpathlineto{\pgfqpoint{1.957477in}{1.224173in}}%
\pgfpathlineto{\pgfqpoint{1.958319in}{1.237308in}}%
\pgfpathlineto{\pgfqpoint{1.959583in}{1.224173in}}%
\pgfpathlineto{\pgfqpoint{1.961268in}{1.224173in}}%
\pgfpathlineto{\pgfqpoint{1.962111in}{1.237308in}}%
\pgfpathlineto{\pgfqpoint{1.962954in}{1.211038in}}%
\pgfpathlineto{\pgfqpoint{1.963375in}{1.224173in}}%
\pgfpathlineto{\pgfqpoint{1.964639in}{1.237308in}}%
\pgfpathlineto{\pgfqpoint{1.965902in}{1.237308in}}%
\pgfpathlineto{\pgfqpoint{1.966745in}{1.211038in}}%
\pgfpathlineto{\pgfqpoint{1.967166in}{1.224173in}}%
\pgfpathlineto{\pgfqpoint{1.967588in}{1.237308in}}%
\pgfpathlineto{\pgfqpoint{1.968009in}{1.211038in}}%
\pgfpathlineto{\pgfqpoint{1.968430in}{1.224173in}}%
\pgfpathlineto{\pgfqpoint{1.969694in}{1.237308in}}%
\pgfpathlineto{\pgfqpoint{1.970115in}{1.237308in}}%
\pgfpathlineto{\pgfqpoint{1.971379in}{1.211038in}}%
\pgfpathlineto{\pgfqpoint{1.971800in}{1.211038in}}%
\pgfpathlineto{\pgfqpoint{1.972643in}{1.237308in}}%
\pgfpathlineto{\pgfqpoint{1.973064in}{1.224173in}}%
\pgfpathlineto{\pgfqpoint{1.973907in}{1.224173in}}%
\pgfpathlineto{\pgfqpoint{1.974328in}{1.211038in}}%
\pgfpathlineto{\pgfqpoint{1.974749in}{1.237308in}}%
\pgfpathlineto{\pgfqpoint{1.975171in}{1.224173in}}%
\pgfpathlineto{\pgfqpoint{1.975592in}{1.211038in}}%
\pgfpathlineto{\pgfqpoint{1.976013in}{1.224173in}}%
\pgfpathlineto{\pgfqpoint{1.977698in}{1.224173in}}%
\pgfpathlineto{\pgfqpoint{1.978962in}{1.211038in}}%
\pgfpathlineto{\pgfqpoint{1.979383in}{1.211038in}}%
\pgfpathlineto{\pgfqpoint{1.980226in}{1.237308in}}%
\pgfpathlineto{\pgfqpoint{1.980647in}{1.224173in}}%
\pgfpathlineto{\pgfqpoint{1.981490in}{1.224173in}}%
\pgfpathlineto{\pgfqpoint{1.982332in}{1.211038in}}%
\pgfpathlineto{\pgfqpoint{1.983596in}{1.224173in}}%
\pgfpathlineto{\pgfqpoint{1.984017in}{1.224173in}}%
\pgfpathlineto{\pgfqpoint{1.984439in}{1.211038in}}%
\pgfpathlineto{\pgfqpoint{1.984860in}{1.224173in}}%
\pgfpathlineto{\pgfqpoint{1.985281in}{1.237308in}}%
\pgfpathlineto{\pgfqpoint{1.986124in}{1.211038in}}%
\pgfpathlineto{\pgfqpoint{1.986545in}{1.224173in}}%
\pgfpathlineto{\pgfqpoint{1.986966in}{1.211038in}}%
\pgfpathlineto{\pgfqpoint{1.987388in}{1.224173in}}%
\pgfpathlineto{\pgfqpoint{1.987809in}{1.237308in}}%
\pgfpathlineto{\pgfqpoint{1.988230in}{1.224173in}}%
\pgfpathlineto{\pgfqpoint{1.988651in}{1.224173in}}%
\pgfpathlineto{\pgfqpoint{1.989915in}{1.237308in}}%
\pgfpathlineto{\pgfqpoint{1.991179in}{1.211038in}}%
\pgfpathlineto{\pgfqpoint{1.991600in}{1.250443in}}%
\pgfpathlineto{\pgfqpoint{1.992022in}{1.237308in}}%
\pgfpathlineto{\pgfqpoint{1.992443in}{1.211038in}}%
\pgfpathlineto{\pgfqpoint{1.993286in}{1.224173in}}%
\pgfpathlineto{\pgfqpoint{1.993707in}{1.224173in}}%
\pgfpathlineto{\pgfqpoint{1.994128in}{1.250443in}}%
\pgfpathlineto{\pgfqpoint{1.994549in}{1.211038in}}%
\pgfpathlineto{\pgfqpoint{1.994971in}{1.237308in}}%
\pgfpathlineto{\pgfqpoint{1.995392in}{1.237308in}}%
\pgfpathlineto{\pgfqpoint{1.997077in}{1.211038in}}%
\pgfpathlineto{\pgfqpoint{1.998341in}{1.224173in}}%
\pgfpathlineto{\pgfqpoint{1.998762in}{1.224173in}}%
\pgfpathlineto{\pgfqpoint{2.000026in}{1.237308in}}%
\pgfpathlineto{\pgfqpoint{2.000447in}{1.237308in}}%
\pgfpathlineto{\pgfqpoint{2.001711in}{1.197903in}}%
\pgfpathlineto{\pgfqpoint{2.002554in}{1.224173in}}%
\pgfpathlineto{\pgfqpoint{2.002975in}{1.211038in}}%
\pgfpathlineto{\pgfqpoint{2.004239in}{1.237308in}}%
\pgfpathlineto{\pgfqpoint{2.005081in}{1.197903in}}%
\pgfpathlineto{\pgfqpoint{2.005503in}{1.211038in}}%
\pgfpathlineto{\pgfqpoint{2.006345in}{1.237308in}}%
\pgfpathlineto{\pgfqpoint{2.007609in}{1.197903in}}%
\pgfpathlineto{\pgfqpoint{2.008030in}{1.237308in}}%
\pgfpathlineto{\pgfqpoint{2.008873in}{1.224173in}}%
\pgfpathlineto{\pgfqpoint{2.009715in}{1.237308in}}%
\pgfpathlineto{\pgfqpoint{2.010979in}{1.224173in}}%
\pgfpathlineto{\pgfqpoint{2.011401in}{1.237308in}}%
\pgfpathlineto{\pgfqpoint{2.011822in}{1.224173in}}%
\pgfpathlineto{\pgfqpoint{2.012664in}{1.224173in}}%
\pgfpathlineto{\pgfqpoint{2.013928in}{1.211038in}}%
\pgfpathlineto{\pgfqpoint{2.015192in}{1.250443in}}%
\pgfpathlineto{\pgfqpoint{2.016877in}{1.224173in}}%
\pgfpathlineto{\pgfqpoint{2.018141in}{1.250443in}}%
\pgfpathlineto{\pgfqpoint{2.019826in}{1.224173in}}%
\pgfpathlineto{\pgfqpoint{2.021090in}{1.289849in}}%
\pgfpathlineto{\pgfqpoint{2.020669in}{1.211038in}}%
\pgfpathlineto{\pgfqpoint{2.021511in}{1.276714in}}%
\pgfpathlineto{\pgfqpoint{2.022775in}{1.224173in}}%
\pgfpathlineto{\pgfqpoint{2.023196in}{1.224173in}}%
\pgfpathlineto{\pgfqpoint{2.024039in}{1.211038in}}%
\pgfpathlineto{\pgfqpoint{2.024460in}{1.237308in}}%
\pgfpathlineto{\pgfqpoint{2.025303in}{1.224173in}}%
\pgfpathlineto{\pgfqpoint{2.025724in}{1.224173in}}%
\pgfpathlineto{\pgfqpoint{2.026988in}{1.211038in}}%
\pgfpathlineto{\pgfqpoint{2.027409in}{1.237308in}}%
\pgfpathlineto{\pgfqpoint{2.028252in}{1.224173in}}%
\pgfpathlineto{\pgfqpoint{2.029094in}{1.211038in}}%
\pgfpathlineto{\pgfqpoint{2.030358in}{1.224173in}}%
\pgfpathlineto{\pgfqpoint{2.031622in}{1.197903in}}%
\pgfpathlineto{\pgfqpoint{2.032043in}{1.224173in}}%
\pgfpathlineto{\pgfqpoint{2.032464in}{1.211038in}}%
\pgfpathlineto{\pgfqpoint{2.032886in}{1.197903in}}%
\pgfpathlineto{\pgfqpoint{2.034150in}{1.237308in}}%
\pgfpathlineto{\pgfqpoint{2.034992in}{1.158498in}}%
\pgfpathlineto{\pgfqpoint{2.036256in}{1.211038in}}%
\pgfpathlineto{\pgfqpoint{2.037520in}{1.237308in}}%
\pgfpathlineto{\pgfqpoint{2.038784in}{1.224173in}}%
\pgfpathlineto{\pgfqpoint{2.039205in}{1.224173in}}%
\pgfpathlineto{\pgfqpoint{2.040469in}{1.237308in}}%
\pgfpathlineto{\pgfqpoint{2.041311in}{1.211038in}}%
\pgfpathlineto{\pgfqpoint{2.041733in}{1.224173in}}%
\pgfpathlineto{\pgfqpoint{2.042154in}{1.224173in}}%
\pgfpathlineto{\pgfqpoint{2.042996in}{1.197903in}}%
\pgfpathlineto{\pgfqpoint{2.043418in}{1.237308in}}%
\pgfpathlineto{\pgfqpoint{2.044260in}{1.224173in}}%
\pgfpathlineto{\pgfqpoint{2.044681in}{1.237308in}}%
\pgfpathlineto{\pgfqpoint{2.045103in}{1.224173in}}%
\pgfpathlineto{\pgfqpoint{2.045524in}{1.211038in}}%
\pgfpathlineto{\pgfqpoint{2.045945in}{1.224173in}}%
\pgfpathlineto{\pgfqpoint{2.046367in}{1.237308in}}%
\pgfpathlineto{\pgfqpoint{2.046788in}{1.224173in}}%
\pgfpathlineto{\pgfqpoint{2.047209in}{1.224173in}}%
\pgfpathlineto{\pgfqpoint{2.047630in}{1.211038in}}%
\pgfpathlineto{\pgfqpoint{2.048052in}{1.224173in}}%
\pgfpathlineto{\pgfqpoint{2.048894in}{1.224173in}}%
\pgfpathlineto{\pgfqpoint{2.049737in}{1.237308in}}%
\pgfpathlineto{\pgfqpoint{2.050158in}{1.211038in}}%
\pgfpathlineto{\pgfqpoint{2.051001in}{1.224173in}}%
\pgfpathlineto{\pgfqpoint{2.051422in}{1.211038in}}%
\pgfpathlineto{\pgfqpoint{2.051843in}{1.224173in}}%
\pgfpathlineto{\pgfqpoint{2.053528in}{1.224173in}}%
\pgfpathlineto{\pgfqpoint{2.053950in}{1.211038in}}%
\pgfpathlineto{\pgfqpoint{2.054371in}{1.224173in}}%
\pgfpathlineto{\pgfqpoint{2.054792in}{1.237308in}}%
\pgfpathlineto{\pgfqpoint{2.055213in}{1.224173in}}%
\pgfpathlineto{\pgfqpoint{2.056056in}{1.224173in}}%
\pgfpathlineto{\pgfqpoint{2.056477in}{1.237308in}}%
\pgfpathlineto{\pgfqpoint{2.056899in}{1.224173in}}%
\pgfpathlineto{\pgfqpoint{2.057741in}{1.211038in}}%
\pgfpathlineto{\pgfqpoint{2.059426in}{1.237308in}}%
\pgfpathlineto{\pgfqpoint{2.059847in}{1.237308in}}%
\pgfpathlineto{\pgfqpoint{2.060269in}{1.211038in}}%
\pgfpathlineto{\pgfqpoint{2.060690in}{1.237308in}}%
\pgfpathlineto{\pgfqpoint{2.061111in}{1.237308in}}%
\pgfpathlineto{\pgfqpoint{2.061533in}{1.211038in}}%
\pgfpathlineto{\pgfqpoint{2.062375in}{1.224173in}}%
\pgfpathlineto{\pgfqpoint{2.062796in}{1.211038in}}%
\pgfpathlineto{\pgfqpoint{2.063218in}{1.224173in}}%
\pgfpathlineto{\pgfqpoint{2.063639in}{1.237308in}}%
\pgfpathlineto{\pgfqpoint{2.064060in}{1.211038in}}%
\pgfpathlineto{\pgfqpoint{2.064903in}{1.224173in}}%
\pgfpathlineto{\pgfqpoint{2.065324in}{1.211038in}}%
\pgfpathlineto{\pgfqpoint{2.065745in}{1.224173in}}%
\pgfpathlineto{\pgfqpoint{2.066588in}{1.224173in}}%
\pgfpathlineto{\pgfqpoint{2.067430in}{1.237308in}}%
\pgfpathlineto{\pgfqpoint{2.067852in}{1.211038in}}%
\pgfpathlineto{\pgfqpoint{2.068273in}{1.224173in}}%
\pgfpathlineto{\pgfqpoint{2.069537in}{1.263578in}}%
\pgfpathlineto{\pgfqpoint{2.070379in}{1.211038in}}%
\pgfpathlineto{\pgfqpoint{2.071222in}{1.224173in}}%
\pgfpathlineto{\pgfqpoint{2.071643in}{1.211038in}}%
\pgfpathlineto{\pgfqpoint{2.072065in}{1.224173in}}%
\pgfpathlineto{\pgfqpoint{2.072486in}{1.224173in}}%
\pgfpathlineto{\pgfqpoint{2.073328in}{1.211038in}}%
\pgfpathlineto{\pgfqpoint{2.073750in}{1.237308in}}%
\pgfpathlineto{\pgfqpoint{2.074592in}{1.224173in}}%
\pgfpathlineto{\pgfqpoint{2.076277in}{1.224173in}}%
\pgfpathlineto{\pgfqpoint{2.077541in}{1.211038in}}%
\pgfpathlineto{\pgfqpoint{2.077962in}{1.211038in}}%
\pgfpathlineto{\pgfqpoint{2.078805in}{1.237308in}}%
\pgfpathlineto{\pgfqpoint{2.079226in}{1.224173in}}%
\pgfpathlineto{\pgfqpoint{2.079648in}{1.224173in}}%
\pgfpathlineto{\pgfqpoint{2.080490in}{1.197903in}}%
\pgfpathlineto{\pgfqpoint{2.081333in}{1.237308in}}%
\pgfpathlineto{\pgfqpoint{2.081754in}{1.211038in}}%
\pgfpathlineto{\pgfqpoint{2.082597in}{1.237308in}}%
\pgfpathlineto{\pgfqpoint{2.083018in}{1.184768in}}%
\pgfpathlineto{\pgfqpoint{2.083439in}{1.224173in}}%
\pgfpathlineto{\pgfqpoint{2.083860in}{1.237308in}}%
\pgfpathlineto{\pgfqpoint{2.084282in}{1.211038in}}%
\pgfpathlineto{\pgfqpoint{2.084703in}{1.224173in}}%
\pgfpathlineto{\pgfqpoint{2.085124in}{1.237308in}}%
\pgfpathlineto{\pgfqpoint{2.085545in}{1.211038in}}%
\pgfpathlineto{\pgfqpoint{2.085967in}{1.237308in}}%
\pgfpathlineto{\pgfqpoint{2.086388in}{1.237308in}}%
\pgfpathlineto{\pgfqpoint{2.087652in}{1.224173in}}%
\pgfpathlineto{\pgfqpoint{2.088073in}{1.224173in}}%
\pgfpathlineto{\pgfqpoint{2.088494in}{1.237308in}}%
\pgfpathlineto{\pgfqpoint{2.088916in}{1.224173in}}%
\pgfpathlineto{\pgfqpoint{2.089337in}{1.224173in}}%
\pgfpathlineto{\pgfqpoint{2.090180in}{1.237308in}}%
\pgfpathlineto{\pgfqpoint{2.090601in}{1.211038in}}%
\pgfpathlineto{\pgfqpoint{2.091022in}{1.250443in}}%
\pgfpathlineto{\pgfqpoint{2.091865in}{1.197903in}}%
\pgfpathlineto{\pgfqpoint{2.093128in}{1.211038in}}%
\pgfpathlineto{\pgfqpoint{2.093550in}{1.237308in}}%
\pgfpathlineto{\pgfqpoint{2.093971in}{1.224173in}}%
\pgfpathlineto{\pgfqpoint{2.094814in}{1.211038in}}%
\pgfpathlineto{\pgfqpoint{2.095235in}{1.237308in}}%
\pgfpathlineto{\pgfqpoint{2.095656in}{1.211038in}}%
\pgfpathlineto{\pgfqpoint{2.096077in}{1.211038in}}%
\pgfpathlineto{\pgfqpoint{2.096499in}{1.237308in}}%
\pgfpathlineto{\pgfqpoint{2.097341in}{1.224173in}}%
\pgfpathlineto{\pgfqpoint{2.099026in}{1.224173in}}%
\pgfpathlineto{\pgfqpoint{2.099448in}{1.197903in}}%
\pgfpathlineto{\pgfqpoint{2.099869in}{1.224173in}}%
\pgfpathlineto{\pgfqpoint{2.100290in}{1.237308in}}%
\pgfpathlineto{\pgfqpoint{2.101554in}{1.211038in}}%
\pgfpathlineto{\pgfqpoint{2.101975in}{1.211038in}}%
\pgfpathlineto{\pgfqpoint{2.103239in}{1.237308in}}%
\pgfpathlineto{\pgfqpoint{2.104503in}{1.197903in}}%
\pgfpathlineto{\pgfqpoint{2.104924in}{1.211038in}}%
\pgfpathlineto{\pgfqpoint{2.106188in}{1.237308in}}%
\pgfpathlineto{\pgfqpoint{2.106609in}{1.237308in}}%
\pgfpathlineto{\pgfqpoint{2.107031in}{1.211038in}}%
\pgfpathlineto{\pgfqpoint{2.107452in}{1.237308in}}%
\pgfpathlineto{\pgfqpoint{2.107873in}{1.237308in}}%
\pgfpathlineto{\pgfqpoint{2.109137in}{1.224173in}}%
\pgfpathlineto{\pgfqpoint{2.109980in}{1.224173in}}%
\pgfpathlineto{\pgfqpoint{2.110401in}{1.250443in}}%
\pgfpathlineto{\pgfqpoint{2.110822in}{1.211038in}}%
\pgfpathlineto{\pgfqpoint{2.111243in}{1.237308in}}%
\pgfpathlineto{\pgfqpoint{2.112086in}{1.211038in}}%
\pgfpathlineto{\pgfqpoint{2.112507in}{1.237308in}}%
\pgfpathlineto{\pgfqpoint{2.112929in}{1.224173in}}%
\pgfpathlineto{\pgfqpoint{2.113350in}{1.211038in}}%
\pgfpathlineto{\pgfqpoint{2.113771in}{1.224173in}}%
\pgfpathlineto{\pgfqpoint{2.114192in}{1.224173in}}%
\pgfpathlineto{\pgfqpoint{2.114614in}{1.211038in}}%
\pgfpathlineto{\pgfqpoint{2.115035in}{1.224173in}}%
\pgfpathlineto{\pgfqpoint{2.115456in}{1.237308in}}%
\pgfpathlineto{\pgfqpoint{2.116299in}{1.250443in}}%
\pgfpathlineto{\pgfqpoint{2.116720in}{1.197903in}}%
\pgfpathlineto{\pgfqpoint{2.117984in}{1.250443in}}%
\pgfpathlineto{\pgfqpoint{2.118826in}{1.237308in}}%
\pgfpathlineto{\pgfqpoint{2.119248in}{1.237308in}}%
\pgfpathlineto{\pgfqpoint{2.119669in}{1.211038in}}%
\pgfpathlineto{\pgfqpoint{2.120512in}{1.224173in}}%
\pgfpathlineto{\pgfqpoint{2.120933in}{1.211038in}}%
\pgfpathlineto{\pgfqpoint{2.121354in}{1.224173in}}%
\pgfpathlineto{\pgfqpoint{2.121775in}{1.224173in}}%
\pgfpathlineto{\pgfqpoint{2.122197in}{1.211038in}}%
\pgfpathlineto{\pgfqpoint{2.122618in}{1.237308in}}%
\pgfpathlineto{\pgfqpoint{2.123039in}{1.224173in}}%
\pgfpathlineto{\pgfqpoint{2.123460in}{1.211038in}}%
\pgfpathlineto{\pgfqpoint{2.123882in}{1.224173in}}%
\pgfpathlineto{\pgfqpoint{2.124303in}{1.224173in}}%
\pgfpathlineto{\pgfqpoint{2.124724in}{1.211038in}}%
\pgfpathlineto{\pgfqpoint{2.125146in}{1.224173in}}%
\pgfpathlineto{\pgfqpoint{2.125567in}{1.237308in}}%
\pgfpathlineto{\pgfqpoint{2.125988in}{1.211038in}}%
\pgfpathlineto{\pgfqpoint{2.126409in}{1.237308in}}%
\pgfpathlineto{\pgfqpoint{2.126831in}{1.250443in}}%
\pgfpathlineto{\pgfqpoint{2.127252in}{1.211038in}}%
\pgfpathlineto{\pgfqpoint{2.127673in}{1.237308in}}%
\pgfpathlineto{\pgfqpoint{2.128095in}{1.237308in}}%
\pgfpathlineto{\pgfqpoint{2.129780in}{1.211038in}}%
\pgfpathlineto{\pgfqpoint{2.130201in}{1.237308in}}%
\pgfpathlineto{\pgfqpoint{2.131043in}{1.224173in}}%
\pgfpathlineto{\pgfqpoint{2.131465in}{1.224173in}}%
\pgfpathlineto{\pgfqpoint{2.131886in}{1.237308in}}%
\pgfpathlineto{\pgfqpoint{2.132307in}{1.211038in}}%
\pgfpathlineto{\pgfqpoint{2.133150in}{1.224173in}}%
\pgfpathlineto{\pgfqpoint{2.133571in}{1.211038in}}%
\pgfpathlineto{\pgfqpoint{2.133992in}{1.237308in}}%
\pgfpathlineto{\pgfqpoint{2.134414in}{1.224173in}}%
\pgfpathlineto{\pgfqpoint{2.134835in}{1.211038in}}%
\pgfpathlineto{\pgfqpoint{2.135256in}{1.224173in}}%
\pgfpathlineto{\pgfqpoint{2.135678in}{1.224173in}}%
\pgfpathlineto{\pgfqpoint{2.136099in}{1.211038in}}%
\pgfpathlineto{\pgfqpoint{2.136520in}{1.224173in}}%
\pgfpathlineto{\pgfqpoint{2.136941in}{1.237308in}}%
\pgfpathlineto{\pgfqpoint{2.137363in}{1.184768in}}%
\pgfpathlineto{\pgfqpoint{2.137784in}{1.237308in}}%
\pgfpathlineto{\pgfqpoint{2.139048in}{1.211038in}}%
\pgfpathlineto{\pgfqpoint{2.139469in}{1.224173in}}%
\pgfpathlineto{\pgfqpoint{2.139890in}{1.211038in}}%
\pgfpathlineto{\pgfqpoint{2.141575in}{1.211038in}}%
\pgfpathlineto{\pgfqpoint{2.142839in}{1.237308in}}%
\pgfpathlineto{\pgfqpoint{2.143682in}{1.211038in}}%
\pgfpathlineto{\pgfqpoint{2.144524in}{1.237308in}}%
\pgfpathlineto{\pgfqpoint{2.144946in}{1.211038in}}%
\pgfpathlineto{\pgfqpoint{2.145788in}{1.224173in}}%
\pgfpathlineto{\pgfqpoint{2.146210in}{1.211038in}}%
\pgfpathlineto{\pgfqpoint{2.146631in}{1.237308in}}%
\pgfpathlineto{\pgfqpoint{2.147052in}{1.224173in}}%
\pgfpathlineto{\pgfqpoint{2.147473in}{1.211038in}}%
\pgfpathlineto{\pgfqpoint{2.147895in}{1.237308in}}%
\pgfpathlineto{\pgfqpoint{2.148737in}{1.224173in}}%
\pgfpathlineto{\pgfqpoint{2.149158in}{1.224173in}}%
\pgfpathlineto{\pgfqpoint{2.149580in}{1.237308in}}%
\pgfpathlineto{\pgfqpoint{2.150001in}{1.224173in}}%
\pgfpathlineto{\pgfqpoint{2.150422in}{1.224173in}}%
\pgfpathlineto{\pgfqpoint{2.150844in}{1.237308in}}%
\pgfpathlineto{\pgfqpoint{2.151265in}{1.211038in}}%
\pgfpathlineto{\pgfqpoint{2.151686in}{1.224173in}}%
\pgfpathlineto{\pgfqpoint{2.152107in}{1.250443in}}%
\pgfpathlineto{\pgfqpoint{2.152950in}{1.237308in}}%
\pgfpathlineto{\pgfqpoint{2.153793in}{1.211038in}}%
\pgfpathlineto{\pgfqpoint{2.154214in}{1.250443in}}%
\pgfpathlineto{\pgfqpoint{2.154635in}{1.237308in}}%
\pgfpathlineto{\pgfqpoint{2.156320in}{1.197903in}}%
\pgfpathlineto{\pgfqpoint{2.157163in}{1.237308in}}%
\pgfpathlineto{\pgfqpoint{2.157584in}{1.197903in}}%
\pgfpathlineto{\pgfqpoint{2.158005in}{1.224173in}}%
\pgfpathlineto{\pgfqpoint{2.158427in}{1.224173in}}%
\pgfpathlineto{\pgfqpoint{2.158848in}{1.211038in}}%
\pgfpathlineto{\pgfqpoint{2.159269in}{1.237308in}}%
\pgfpathlineto{\pgfqpoint{2.159690in}{1.224173in}}%
\pgfpathlineto{\pgfqpoint{2.160112in}{1.211038in}}%
\pgfpathlineto{\pgfqpoint{2.160533in}{1.237308in}}%
\pgfpathlineto{\pgfqpoint{2.160954in}{1.224173in}}%
\pgfpathlineto{\pgfqpoint{2.161376in}{1.197903in}}%
\pgfpathlineto{\pgfqpoint{2.161797in}{1.237308in}}%
\pgfpathlineto{\pgfqpoint{2.162218in}{1.237308in}}%
\pgfpathlineto{\pgfqpoint{2.163903in}{1.211038in}}%
\pgfpathlineto{\pgfqpoint{2.164324in}{1.237308in}}%
\pgfpathlineto{\pgfqpoint{2.165167in}{1.224173in}}%
\pgfpathlineto{\pgfqpoint{2.166010in}{1.224173in}}%
\pgfpathlineto{\pgfqpoint{2.166431in}{1.211038in}}%
\pgfpathlineto{\pgfqpoint{2.166852in}{1.237308in}}%
\pgfpathlineto{\pgfqpoint{2.167273in}{1.224173in}}%
\pgfpathlineto{\pgfqpoint{2.167695in}{1.211038in}}%
\pgfpathlineto{\pgfqpoint{2.168116in}{1.224173in}}%
\pgfpathlineto{\pgfqpoint{2.168537in}{1.224173in}}%
\pgfpathlineto{\pgfqpoint{2.168959in}{1.211038in}}%
\pgfpathlineto{\pgfqpoint{2.169380in}{1.224173in}}%
\pgfpathlineto{\pgfqpoint{2.169801in}{1.237308in}}%
\pgfpathlineto{\pgfqpoint{2.170222in}{1.224173in}}%
\pgfpathlineto{\pgfqpoint{2.170644in}{1.224173in}}%
\pgfpathlineto{\pgfqpoint{2.171065in}{1.250443in}}%
\pgfpathlineto{\pgfqpoint{2.171486in}{1.211038in}}%
\pgfpathlineto{\pgfqpoint{2.171907in}{1.224173in}}%
\pgfpathlineto{\pgfqpoint{2.172329in}{1.211038in}}%
\pgfpathlineto{\pgfqpoint{2.172750in}{1.211038in}}%
\pgfpathlineto{\pgfqpoint{2.173171in}{1.237308in}}%
\pgfpathlineto{\pgfqpoint{2.173593in}{1.224173in}}%
\pgfpathlineto{\pgfqpoint{2.174014in}{1.211038in}}%
\pgfpathlineto{\pgfqpoint{2.174856in}{1.237308in}}%
\pgfpathlineto{\pgfqpoint{2.175278in}{1.224173in}}%
\pgfpathlineto{\pgfqpoint{2.175699in}{1.224173in}}%
\pgfpathlineto{\pgfqpoint{2.176120in}{1.237308in}}%
\pgfpathlineto{\pgfqpoint{2.176542in}{1.211038in}}%
\pgfpathlineto{\pgfqpoint{2.177384in}{1.224173in}}%
\pgfpathlineto{\pgfqpoint{2.177805in}{1.211038in}}%
\pgfpathlineto{\pgfqpoint{2.178227in}{1.237308in}}%
\pgfpathlineto{\pgfqpoint{2.178648in}{1.224173in}}%
\pgfpathlineto{\pgfqpoint{2.179069in}{1.211038in}}%
\pgfpathlineto{\pgfqpoint{2.179490in}{1.237308in}}%
\pgfpathlineto{\pgfqpoint{2.179912in}{1.224173in}}%
\pgfpathlineto{\pgfqpoint{2.180333in}{1.211038in}}%
\pgfpathlineto{\pgfqpoint{2.180754in}{1.237308in}}%
\pgfpathlineto{\pgfqpoint{2.181176in}{1.224173in}}%
\pgfpathlineto{\pgfqpoint{2.181597in}{1.211038in}}%
\pgfpathlineto{\pgfqpoint{2.182018in}{1.224173in}}%
\pgfpathlineto{\pgfqpoint{2.183282in}{1.237308in}}%
\pgfpathlineto{\pgfqpoint{2.184125in}{1.211038in}}%
\pgfpathlineto{\pgfqpoint{2.184546in}{1.224173in}}%
\pgfpathlineto{\pgfqpoint{2.185810in}{1.250443in}}%
\pgfpathlineto{\pgfqpoint{2.186652in}{1.211038in}}%
\pgfpathlineto{\pgfqpoint{2.187073in}{1.237308in}}%
\pgfpathlineto{\pgfqpoint{2.187495in}{1.237308in}}%
\pgfpathlineto{\pgfqpoint{2.188759in}{1.211038in}}%
\pgfpathlineto{\pgfqpoint{2.189601in}{1.211038in}}%
\pgfpathlineto{\pgfqpoint{2.190865in}{1.224173in}}%
\pgfpathlineto{\pgfqpoint{2.191286in}{1.224173in}}%
\pgfpathlineto{\pgfqpoint{2.191708in}{1.211038in}}%
\pgfpathlineto{\pgfqpoint{2.192129in}{1.224173in}}%
\pgfpathlineto{\pgfqpoint{2.192550in}{1.237308in}}%
\pgfpathlineto{\pgfqpoint{2.192971in}{1.211038in}}%
\pgfpathlineto{\pgfqpoint{2.193814in}{1.224173in}}%
\pgfpathlineto{\pgfqpoint{2.194235in}{1.211038in}}%
\pgfpathlineto{\pgfqpoint{2.195078in}{1.237308in}}%
\pgfpathlineto{\pgfqpoint{2.195499in}{1.224173in}}%
\pgfpathlineto{\pgfqpoint{2.195920in}{1.224173in}}%
\pgfpathlineto{\pgfqpoint{2.196342in}{1.237308in}}%
\pgfpathlineto{\pgfqpoint{2.197605in}{1.211038in}}%
\pgfpathlineto{\pgfqpoint{2.198027in}{1.211038in}}%
\pgfpathlineto{\pgfqpoint{2.198869in}{1.224173in}}%
\pgfpathlineto{\pgfqpoint{2.199712in}{1.211038in}}%
\pgfpathlineto{\pgfqpoint{2.200133in}{1.224173in}}%
\pgfpathlineto{\pgfqpoint{2.200554in}{1.211038in}}%
\pgfpathlineto{\pgfqpoint{2.200976in}{1.211038in}}%
\pgfpathlineto{\pgfqpoint{2.202240in}{1.250443in}}%
\pgfpathlineto{\pgfqpoint{2.203503in}{1.211038in}}%
\pgfpathlineto{\pgfqpoint{2.204767in}{1.237308in}}%
\pgfpathlineto{\pgfqpoint{2.206031in}{1.224173in}}%
\pgfpathlineto{\pgfqpoint{2.206452in}{1.224173in}}%
\pgfpathlineto{\pgfqpoint{2.206874in}{1.211038in}}%
\pgfpathlineto{\pgfqpoint{2.207716in}{1.237308in}}%
\pgfpathlineto{\pgfqpoint{2.208137in}{1.211038in}}%
\pgfpathlineto{\pgfqpoint{2.208980in}{1.224173in}}%
\pgfpathlineto{\pgfqpoint{2.209823in}{1.211038in}}%
\pgfpathlineto{\pgfqpoint{2.211086in}{1.237308in}}%
\pgfpathlineto{\pgfqpoint{2.211929in}{1.211038in}}%
\pgfpathlineto{\pgfqpoint{2.212350in}{1.237308in}}%
\pgfpathlineto{\pgfqpoint{2.213193in}{1.224173in}}%
\pgfpathlineto{\pgfqpoint{2.213614in}{1.237308in}}%
\pgfpathlineto{\pgfqpoint{2.214035in}{1.224173in}}%
\pgfpathlineto{\pgfqpoint{2.215299in}{1.224173in}}%
\pgfpathlineto{\pgfqpoint{2.215720in}{1.211038in}}%
\pgfpathlineto{\pgfqpoint{2.216984in}{1.237308in}}%
\pgfpathlineto{\pgfqpoint{2.218248in}{1.211038in}}%
\pgfpathlineto{\pgfqpoint{2.218669in}{1.237308in}}%
\pgfpathlineto{\pgfqpoint{2.219091in}{1.224173in}}%
\pgfpathlineto{\pgfqpoint{2.219512in}{1.211038in}}%
\pgfpathlineto{\pgfqpoint{2.219933in}{1.237308in}}%
\pgfpathlineto{\pgfqpoint{2.220354in}{1.224173in}}%
\pgfpathlineto{\pgfqpoint{2.220776in}{1.211038in}}%
\pgfpathlineto{\pgfqpoint{2.221197in}{1.224173in}}%
\pgfpathlineto{\pgfqpoint{2.221618in}{1.237308in}}%
\pgfpathlineto{\pgfqpoint{2.222882in}{1.211038in}}%
\pgfpathlineto{\pgfqpoint{2.224146in}{1.224173in}}%
\pgfpathlineto{\pgfqpoint{2.224567in}{1.224173in}}%
\pgfpathlineto{\pgfqpoint{2.224989in}{1.237308in}}%
\pgfpathlineto{\pgfqpoint{2.225410in}{1.224173in}}%
\pgfpathlineto{\pgfqpoint{2.225831in}{1.211038in}}%
\pgfpathlineto{\pgfqpoint{2.226674in}{1.237308in}}%
\pgfpathlineto{\pgfqpoint{2.227095in}{1.211038in}}%
\pgfpathlineto{\pgfqpoint{2.227937in}{1.224173in}}%
\pgfpathlineto{\pgfqpoint{2.228359in}{1.224173in}}%
\pgfpathlineto{\pgfqpoint{2.228780in}{1.237308in}}%
\pgfpathlineto{\pgfqpoint{2.229201in}{1.224173in}}%
\pgfpathlineto{\pgfqpoint{2.230044in}{1.211038in}}%
\pgfpathlineto{\pgfqpoint{2.230465in}{1.250443in}}%
\pgfpathlineto{\pgfqpoint{2.230886in}{1.211038in}}%
\pgfpathlineto{\pgfqpoint{2.231308in}{1.211038in}}%
\pgfpathlineto{\pgfqpoint{2.232572in}{1.237308in}}%
\pgfpathlineto{\pgfqpoint{2.233835in}{1.211038in}}%
\pgfpathlineto{\pgfqpoint{2.235520in}{1.237308in}}%
\pgfpathlineto{\pgfqpoint{2.235942in}{1.224173in}}%
\pgfpathlineto{\pgfqpoint{2.236363in}{1.250443in}}%
\pgfpathlineto{\pgfqpoint{2.236784in}{1.237308in}}%
\pgfpathlineto{\pgfqpoint{2.237206in}{1.211038in}}%
\pgfpathlineto{\pgfqpoint{2.238048in}{1.224173in}}%
\pgfpathlineto{\pgfqpoint{2.238469in}{1.211038in}}%
\pgfpathlineto{\pgfqpoint{2.238891in}{1.237308in}}%
\pgfpathlineto{\pgfqpoint{2.239312in}{1.224173in}}%
\pgfpathlineto{\pgfqpoint{2.239733in}{1.211038in}}%
\pgfpathlineto{\pgfqpoint{2.240155in}{1.224173in}}%
\pgfpathlineto{\pgfqpoint{2.240576in}{1.250443in}}%
\pgfpathlineto{\pgfqpoint{2.240997in}{1.224173in}}%
\pgfpathlineto{\pgfqpoint{2.241840in}{1.224173in}}%
\pgfpathlineto{\pgfqpoint{2.242261in}{1.211038in}}%
\pgfpathlineto{\pgfqpoint{2.242682in}{1.224173in}}%
\pgfpathlineto{\pgfqpoint{2.243525in}{1.224173in}}%
\pgfpathlineto{\pgfqpoint{2.243946in}{1.211038in}}%
\pgfpathlineto{\pgfqpoint{2.244367in}{1.224173in}}%
\pgfpathlineto{\pgfqpoint{2.245210in}{1.224173in}}%
\pgfpathlineto{\pgfqpoint{2.245631in}{1.237308in}}%
\pgfpathlineto{\pgfqpoint{2.246052in}{1.211038in}}%
\pgfpathlineto{\pgfqpoint{2.246474in}{1.224173in}}%
\pgfpathlineto{\pgfqpoint{2.247738in}{1.237308in}}%
\pgfpathlineto{\pgfqpoint{2.248580in}{1.211038in}}%
\pgfpathlineto{\pgfqpoint{2.249001in}{1.224173in}}%
\pgfpathlineto{\pgfqpoint{2.249423in}{1.224173in}}%
\pgfpathlineto{\pgfqpoint{2.250265in}{1.211038in}}%
\pgfpathlineto{\pgfqpoint{2.251950in}{1.237308in}}%
\pgfpathlineto{\pgfqpoint{2.252372in}{1.211038in}}%
\pgfpathlineto{\pgfqpoint{2.253214in}{1.224173in}}%
\pgfpathlineto{\pgfqpoint{2.254057in}{1.211038in}}%
\pgfpathlineto{\pgfqpoint{2.255321in}{1.237308in}}%
\pgfpathlineto{\pgfqpoint{2.256163in}{1.224173in}}%
\pgfpathlineto{\pgfqpoint{2.256584in}{1.237308in}}%
\pgfpathlineto{\pgfqpoint{2.257006in}{1.224173in}}%
\pgfpathlineto{\pgfqpoint{2.257427in}{1.211038in}}%
\pgfpathlineto{\pgfqpoint{2.257848in}{1.224173in}}%
\pgfpathlineto{\pgfqpoint{2.258269in}{1.237308in}}%
\pgfpathlineto{\pgfqpoint{2.259112in}{1.211038in}}%
\pgfpathlineto{\pgfqpoint{2.259533in}{1.224173in}}%
\pgfpathlineto{\pgfqpoint{2.260797in}{1.224173in}}%
\pgfpathlineto{\pgfqpoint{2.261218in}{1.211038in}}%
\pgfpathlineto{\pgfqpoint{2.261640in}{1.224173in}}%
\pgfpathlineto{\pgfqpoint{2.262061in}{1.224173in}}%
\pgfpathlineto{\pgfqpoint{2.262904in}{1.211038in}}%
\pgfpathlineto{\pgfqpoint{2.264167in}{1.224173in}}%
\pgfpathlineto{\pgfqpoint{2.264589in}{1.224173in}}%
\pgfpathlineto{\pgfqpoint{2.265010in}{1.237308in}}%
\pgfpathlineto{\pgfqpoint{2.266274in}{1.211038in}}%
\pgfpathlineto{\pgfqpoint{2.267116in}{1.237308in}}%
\pgfpathlineto{\pgfqpoint{2.268801in}{1.197903in}}%
\pgfpathlineto{\pgfqpoint{2.270487in}{1.237308in}}%
\pgfpathlineto{\pgfqpoint{2.271750in}{1.211038in}}%
\pgfpathlineto{\pgfqpoint{2.272172in}{1.237308in}}%
\pgfpathlineto{\pgfqpoint{2.273014in}{1.224173in}}%
\pgfpathlineto{\pgfqpoint{2.273436in}{1.224173in}}%
\pgfpathlineto{\pgfqpoint{2.273857in}{1.211038in}}%
\pgfpathlineto{\pgfqpoint{2.274278in}{1.237308in}}%
\pgfpathlineto{\pgfqpoint{2.275121in}{1.224173in}}%
\pgfpathlineto{\pgfqpoint{2.275542in}{1.224173in}}%
\pgfpathlineto{\pgfqpoint{2.276384in}{1.237308in}}%
\pgfpathlineto{\pgfqpoint{2.276806in}{1.224173in}}%
\pgfpathlineto{\pgfqpoint{2.277227in}{1.237308in}}%
\pgfpathlineto{\pgfqpoint{2.278070in}{1.237308in}}%
\pgfpathlineto{\pgfqpoint{2.278912in}{1.211038in}}%
\pgfpathlineto{\pgfqpoint{2.279333in}{1.237308in}}%
\pgfpathlineto{\pgfqpoint{2.280176in}{1.224173in}}%
\pgfpathlineto{\pgfqpoint{2.280597in}{1.224173in}}%
\pgfpathlineto{\pgfqpoint{2.281019in}{1.237308in}}%
\pgfpathlineto{\pgfqpoint{2.281440in}{1.224173in}}%
\pgfpathlineto{\pgfqpoint{2.281861in}{1.224173in}}%
\pgfpathlineto{\pgfqpoint{2.282282in}{1.250443in}}%
\pgfpathlineto{\pgfqpoint{2.282704in}{1.224173in}}%
\pgfpathlineto{\pgfqpoint{2.283125in}{1.224173in}}%
\pgfpathlineto{\pgfqpoint{2.283546in}{1.237308in}}%
\pgfpathlineto{\pgfqpoint{2.283967in}{1.211038in}}%
\pgfpathlineto{\pgfqpoint{2.284810in}{1.224173in}}%
\pgfpathlineto{\pgfqpoint{2.285231in}{1.224173in}}%
\pgfpathlineto{\pgfqpoint{2.285653in}{1.237308in}}%
\pgfpathlineto{\pgfqpoint{2.286074in}{1.224173in}}%
\pgfpathlineto{\pgfqpoint{2.286495in}{1.224173in}}%
\pgfpathlineto{\pgfqpoint{2.286916in}{1.237308in}}%
\pgfpathlineto{\pgfqpoint{2.287338in}{1.224173in}}%
\pgfpathlineto{\pgfqpoint{2.287759in}{1.211038in}}%
\pgfpathlineto{\pgfqpoint{2.288602in}{1.237308in}}%
\pgfpathlineto{\pgfqpoint{2.289023in}{1.224173in}}%
\pgfpathlineto{\pgfqpoint{2.290708in}{1.224173in}}%
\pgfpathlineto{\pgfqpoint{2.291550in}{1.211038in}}%
\pgfpathlineto{\pgfqpoint{2.292393in}{1.237308in}}%
\pgfpathlineto{\pgfqpoint{2.292814in}{1.211038in}}%
\pgfpathlineto{\pgfqpoint{2.293236in}{1.237308in}}%
\pgfpathlineto{\pgfqpoint{2.293657in}{1.237308in}}%
\pgfpathlineto{\pgfqpoint{2.294078in}{1.197903in}}%
\pgfpathlineto{\pgfqpoint{2.294499in}{1.224173in}}%
\pgfpathlineto{\pgfqpoint{2.295763in}{1.224173in}}%
\pgfpathlineto{\pgfqpoint{2.296185in}{1.237308in}}%
\pgfpathlineto{\pgfqpoint{2.296606in}{1.224173in}}%
\pgfpathlineto{\pgfqpoint{2.297448in}{1.224173in}}%
\pgfpathlineto{\pgfqpoint{2.297870in}{1.211038in}}%
\pgfpathlineto{\pgfqpoint{2.298712in}{1.237308in}}%
\pgfpathlineto{\pgfqpoint{2.299133in}{1.211038in}}%
\pgfpathlineto{\pgfqpoint{2.299976in}{1.224173in}}%
\pgfpathlineto{\pgfqpoint{2.300397in}{1.211038in}}%
\pgfpathlineto{\pgfqpoint{2.300819in}{1.224173in}}%
\pgfpathlineto{\pgfqpoint{2.301240in}{1.224173in}}%
\pgfpathlineto{\pgfqpoint{2.302082in}{1.211038in}}%
\pgfpathlineto{\pgfqpoint{2.303768in}{1.237308in}}%
\pgfpathlineto{\pgfqpoint{2.304189in}{1.211038in}}%
\pgfpathlineto{\pgfqpoint{2.304610in}{1.237308in}}%
\pgfpathlineto{\pgfqpoint{2.305031in}{1.237308in}}%
\pgfpathlineto{\pgfqpoint{2.305453in}{1.224173in}}%
\pgfpathlineto{\pgfqpoint{2.305874in}{1.237308in}}%
\pgfpathlineto{\pgfqpoint{2.306295in}{1.237308in}}%
\pgfpathlineto{\pgfqpoint{2.306716in}{1.211038in}}%
\pgfpathlineto{\pgfqpoint{2.307138in}{1.224173in}}%
\pgfpathlineto{\pgfqpoint{2.307559in}{1.237308in}}%
\pgfpathlineto{\pgfqpoint{2.309244in}{1.197903in}}%
\pgfpathlineto{\pgfqpoint{2.309665in}{1.237308in}}%
\pgfpathlineto{\pgfqpoint{2.310087in}{1.224173in}}%
\pgfpathlineto{\pgfqpoint{2.310508in}{1.211038in}}%
\pgfpathlineto{\pgfqpoint{2.311351in}{1.237308in}}%
\pgfpathlineto{\pgfqpoint{2.312193in}{1.211038in}}%
\pgfpathlineto{\pgfqpoint{2.313457in}{1.237308in}}%
\pgfpathlineto{\pgfqpoint{2.314299in}{1.224173in}}%
\pgfpathlineto{\pgfqpoint{2.315563in}{1.237308in}}%
\pgfpathlineto{\pgfqpoint{2.316827in}{1.211038in}}%
\pgfpathlineto{\pgfqpoint{2.317670in}{1.224173in}}%
\pgfpathlineto{\pgfqpoint{2.318091in}{1.211038in}}%
\pgfpathlineto{\pgfqpoint{2.318512in}{1.224173in}}%
\pgfpathlineto{\pgfqpoint{2.318934in}{1.224173in}}%
\pgfpathlineto{\pgfqpoint{2.319355in}{1.197903in}}%
\pgfpathlineto{\pgfqpoint{2.319776in}{1.211038in}}%
\pgfpathlineto{\pgfqpoint{2.321461in}{1.237308in}}%
\pgfpathlineto{\pgfqpoint{2.322304in}{1.224173in}}%
\pgfpathlineto{\pgfqpoint{2.322725in}{1.237308in}}%
\pgfpathlineto{\pgfqpoint{2.323146in}{1.197903in}}%
\pgfpathlineto{\pgfqpoint{2.323568in}{1.237308in}}%
\pgfpathlineto{\pgfqpoint{2.323989in}{1.237308in}}%
\pgfpathlineto{\pgfqpoint{2.324410in}{1.211038in}}%
\pgfpathlineto{\pgfqpoint{2.325253in}{1.224173in}}%
\pgfpathlineto{\pgfqpoint{2.326095in}{1.224173in}}%
\pgfpathlineto{\pgfqpoint{2.326517in}{1.237308in}}%
\pgfpathlineto{\pgfqpoint{2.326938in}{1.211038in}}%
\pgfpathlineto{\pgfqpoint{2.327359in}{1.237308in}}%
\pgfpathlineto{\pgfqpoint{2.327780in}{1.237308in}}%
\pgfpathlineto{\pgfqpoint{2.328202in}{1.211038in}}%
\pgfpathlineto{\pgfqpoint{2.329044in}{1.224173in}}%
\pgfpathlineto{\pgfqpoint{2.330308in}{1.224173in}}%
\pgfpathlineto{\pgfqpoint{2.330729in}{1.211038in}}%
\pgfpathlineto{\pgfqpoint{2.331151in}{1.237308in}}%
\pgfpathlineto{\pgfqpoint{2.331572in}{1.224173in}}%
\pgfpathlineto{\pgfqpoint{2.331993in}{1.211038in}}%
\pgfpathlineto{\pgfqpoint{2.332836in}{1.237308in}}%
\pgfpathlineto{\pgfqpoint{2.333257in}{1.224173in}}%
\pgfpathlineto{\pgfqpoint{2.334100in}{1.237308in}}%
\pgfpathlineto{\pgfqpoint{2.334942in}{1.211038in}}%
\pgfpathlineto{\pgfqpoint{2.335363in}{1.224173in}}%
\pgfpathlineto{\pgfqpoint{2.335785in}{1.224173in}}%
\pgfpathlineto{\pgfqpoint{2.336627in}{1.237308in}}%
\pgfpathlineto{\pgfqpoint{2.337049in}{1.211038in}}%
\pgfpathlineto{\pgfqpoint{2.337470in}{1.224173in}}%
\pgfpathlineto{\pgfqpoint{2.337891in}{1.237308in}}%
\pgfpathlineto{\pgfqpoint{2.338312in}{1.211038in}}%
\pgfpathlineto{\pgfqpoint{2.338734in}{1.224173in}}%
\pgfpathlineto{\pgfqpoint{2.339155in}{1.237308in}}%
\pgfpathlineto{\pgfqpoint{2.339997in}{1.211038in}}%
\pgfpathlineto{\pgfqpoint{2.340419in}{1.224173in}}%
\pgfpathlineto{\pgfqpoint{2.340840in}{1.211038in}}%
\pgfpathlineto{\pgfqpoint{2.341261in}{1.224173in}}%
\pgfpathlineto{\pgfqpoint{2.341683in}{1.224173in}}%
\pgfpathlineto{\pgfqpoint{2.342525in}{1.211038in}}%
\pgfpathlineto{\pgfqpoint{2.344210in}{1.263578in}}%
\pgfpathlineto{\pgfqpoint{2.344632in}{1.197903in}}%
\pgfpathlineto{\pgfqpoint{2.345474in}{1.224173in}}%
\pgfpathlineto{\pgfqpoint{2.345895in}{1.211038in}}%
\pgfpathlineto{\pgfqpoint{2.346317in}{1.237308in}}%
\pgfpathlineto{\pgfqpoint{2.347159in}{1.224173in}}%
\pgfpathlineto{\pgfqpoint{2.348423in}{1.224173in}}%
\pgfpathlineto{\pgfqpoint{2.349266in}{1.237308in}}%
\pgfpathlineto{\pgfqpoint{2.349687in}{1.211038in}}%
\pgfpathlineto{\pgfqpoint{2.350529in}{1.224173in}}%
\pgfpathlineto{\pgfqpoint{2.350951in}{1.211038in}}%
\pgfpathlineto{\pgfqpoint{2.351372in}{1.224173in}}%
\pgfpathlineto{\pgfqpoint{2.352636in}{1.224173in}}%
\pgfpathlineto{\pgfqpoint{2.353900in}{1.237308in}}%
\pgfpathlineto{\pgfqpoint{2.354742in}{1.197903in}}%
\pgfpathlineto{\pgfqpoint{2.355163in}{1.224173in}}%
\pgfpathlineto{\pgfqpoint{2.356849in}{1.224173in}}%
\pgfpathlineto{\pgfqpoint{2.357691in}{1.197903in}}%
\pgfpathlineto{\pgfqpoint{2.358112in}{1.250443in}}%
\pgfpathlineto{\pgfqpoint{2.358955in}{1.224173in}}%
\pgfpathlineto{\pgfqpoint{2.359376in}{1.237308in}}%
\pgfpathlineto{\pgfqpoint{2.360640in}{1.211038in}}%
\pgfpathlineto{\pgfqpoint{2.361483in}{1.237308in}}%
\pgfpathlineto{\pgfqpoint{2.361904in}{1.224173in}}%
\pgfpathlineto{\pgfqpoint{2.362325in}{1.211038in}}%
\pgfpathlineto{\pgfqpoint{2.362746in}{1.224173in}}%
\pgfpathlineto{\pgfqpoint{2.364010in}{1.237308in}}%
\pgfpathlineto{\pgfqpoint{2.364853in}{1.211038in}}%
\pgfpathlineto{\pgfqpoint{2.365274in}{1.224173in}}%
\pgfpathlineto{\pgfqpoint{2.366538in}{1.224173in}}%
\pgfpathlineto{\pgfqpoint{2.366959in}{1.250443in}}%
\pgfpathlineto{\pgfqpoint{2.367381in}{1.211038in}}%
\pgfpathlineto{\pgfqpoint{2.367802in}{1.237308in}}%
\pgfpathlineto{\pgfqpoint{2.368223in}{1.237308in}}%
\pgfpathlineto{\pgfqpoint{2.368644in}{1.211038in}}%
\pgfpathlineto{\pgfqpoint{2.369487in}{1.224173in}}%
\pgfpathlineto{\pgfqpoint{2.369908in}{1.224173in}}%
\pgfpathlineto{\pgfqpoint{2.370329in}{1.237308in}}%
\pgfpathlineto{\pgfqpoint{2.370751in}{1.224173in}}%
\pgfpathlineto{\pgfqpoint{2.371593in}{1.211038in}}%
\pgfpathlineto{\pgfqpoint{2.372015in}{1.237308in}}%
\pgfpathlineto{\pgfqpoint{2.372857in}{1.224173in}}%
\pgfpathlineto{\pgfqpoint{2.373278in}{1.237308in}}%
\pgfpathlineto{\pgfqpoint{2.374542in}{1.211038in}}%
\pgfpathlineto{\pgfqpoint{2.374964in}{1.211038in}}%
\pgfpathlineto{\pgfqpoint{2.375806in}{1.224173in}}%
\pgfpathlineto{\pgfqpoint{2.377070in}{1.211038in}}%
\pgfpathlineto{\pgfqpoint{2.377491in}{1.211038in}}%
\pgfpathlineto{\pgfqpoint{2.378334in}{1.237308in}}%
\pgfpathlineto{\pgfqpoint{2.378755in}{1.224173in}}%
\pgfpathlineto{\pgfqpoint{2.379176in}{1.237308in}}%
\pgfpathlineto{\pgfqpoint{2.380440in}{1.211038in}}%
\pgfpathlineto{\pgfqpoint{2.382125in}{1.237308in}}%
\pgfpathlineto{\pgfqpoint{2.383389in}{1.211038in}}%
\pgfpathlineto{\pgfqpoint{2.383810in}{1.211038in}}%
\pgfpathlineto{\pgfqpoint{2.385074in}{1.237308in}}%
\pgfpathlineto{\pgfqpoint{2.386338in}{1.211038in}}%
\pgfpathlineto{\pgfqpoint{2.387181in}{1.224173in}}%
\pgfpathlineto{\pgfqpoint{2.387602in}{1.211038in}}%
\pgfpathlineto{\pgfqpoint{2.388023in}{1.224173in}}%
\pgfpathlineto{\pgfqpoint{2.389287in}{1.237308in}}%
\pgfpathlineto{\pgfqpoint{2.389708in}{1.237308in}}%
\pgfpathlineto{\pgfqpoint{2.390130in}{1.211038in}}%
\pgfpathlineto{\pgfqpoint{2.390972in}{1.224173in}}%
\pgfpathlineto{\pgfqpoint{2.391393in}{1.211038in}}%
\pgfpathlineto{\pgfqpoint{2.391815in}{1.224173in}}%
\pgfpathlineto{\pgfqpoint{2.392236in}{1.224173in}}%
\pgfpathlineto{\pgfqpoint{2.392657in}{1.211038in}}%
\pgfpathlineto{\pgfqpoint{2.393078in}{1.237308in}}%
\pgfpathlineto{\pgfqpoint{2.393500in}{1.224173in}}%
\pgfpathlineto{\pgfqpoint{2.393921in}{1.211038in}}%
\pgfpathlineto{\pgfqpoint{2.394764in}{1.237308in}}%
\pgfpathlineto{\pgfqpoint{2.395185in}{1.211038in}}%
\pgfpathlineto{\pgfqpoint{2.396027in}{1.224173in}}%
\pgfpathlineto{\pgfqpoint{2.396449in}{1.211038in}}%
\pgfpathlineto{\pgfqpoint{2.397291in}{1.237308in}}%
\pgfpathlineto{\pgfqpoint{2.397713in}{1.224173in}}%
\pgfpathlineto{\pgfqpoint{2.398976in}{1.224173in}}%
\pgfpathlineto{\pgfqpoint{2.399398in}{1.237308in}}%
\pgfpathlineto{\pgfqpoint{2.399819in}{1.224173in}}%
\pgfpathlineto{\pgfqpoint{2.400240in}{1.211038in}}%
\pgfpathlineto{\pgfqpoint{2.400662in}{1.237308in}}%
\pgfpathlineto{\pgfqpoint{2.401504in}{1.224173in}}%
\pgfpathlineto{\pgfqpoint{2.402768in}{1.211038in}}%
\pgfpathlineto{\pgfqpoint{2.403610in}{1.237308in}}%
\pgfpathlineto{\pgfqpoint{2.404032in}{1.211038in}}%
\pgfpathlineto{\pgfqpoint{2.404453in}{1.237308in}}%
\pgfpathlineto{\pgfqpoint{2.404874in}{1.237308in}}%
\pgfpathlineto{\pgfqpoint{2.405717in}{1.211038in}}%
\pgfpathlineto{\pgfqpoint{2.406138in}{1.224173in}}%
\pgfpathlineto{\pgfqpoint{2.406559in}{1.211038in}}%
\pgfpathlineto{\pgfqpoint{2.407402in}{1.237308in}}%
\pgfpathlineto{\pgfqpoint{2.407823in}{1.197903in}}%
\pgfpathlineto{\pgfqpoint{2.408245in}{1.211038in}}%
\pgfpathlineto{\pgfqpoint{2.409508in}{1.237308in}}%
\pgfpathlineto{\pgfqpoint{2.411193in}{1.211038in}}%
\pgfpathlineto{\pgfqpoint{2.411615in}{1.211038in}}%
\pgfpathlineto{\pgfqpoint{2.412457in}{1.224173in}}%
\pgfpathlineto{\pgfqpoint{2.412879in}{1.211038in}}%
\pgfpathlineto{\pgfqpoint{2.413300in}{1.224173in}}%
\pgfpathlineto{\pgfqpoint{2.414564in}{1.237308in}}%
\pgfpathlineto{\pgfqpoint{2.414985in}{1.237308in}}%
\pgfpathlineto{\pgfqpoint{2.416670in}{1.211038in}}%
\pgfpathlineto{\pgfqpoint{2.417091in}{1.211038in}}%
\pgfpathlineto{\pgfqpoint{2.417513in}{1.237308in}}%
\pgfpathlineto{\pgfqpoint{2.418355in}{1.224173in}}%
\pgfpathlineto{\pgfqpoint{2.419619in}{1.237308in}}%
\pgfpathlineto{\pgfqpoint{2.420040in}{1.237308in}}%
\pgfpathlineto{\pgfqpoint{2.420462in}{1.197903in}}%
\pgfpathlineto{\pgfqpoint{2.420883in}{1.224173in}}%
\pgfpathlineto{\pgfqpoint{2.422147in}{1.237308in}}%
\pgfpathlineto{\pgfqpoint{2.422568in}{1.197903in}}%
\pgfpathlineto{\pgfqpoint{2.422989in}{1.211038in}}%
\pgfpathlineto{\pgfqpoint{2.423411in}{1.237308in}}%
\pgfpathlineto{\pgfqpoint{2.423832in}{1.224173in}}%
\pgfpathlineto{\pgfqpoint{2.424253in}{1.211038in}}%
\pgfpathlineto{\pgfqpoint{2.425096in}{1.237308in}}%
\pgfpathlineto{\pgfqpoint{2.425517in}{1.211038in}}%
\pgfpathlineto{\pgfqpoint{2.426359in}{1.224173in}}%
\pgfpathlineto{\pgfqpoint{2.427202in}{1.224173in}}%
\pgfpathlineto{\pgfqpoint{2.428045in}{1.211038in}}%
\pgfpathlineto{\pgfqpoint{2.429730in}{1.237308in}}%
\pgfpathlineto{\pgfqpoint{2.430151in}{1.237308in}}%
\pgfpathlineto{\pgfqpoint{2.430572in}{1.211038in}}%
\pgfpathlineto{\pgfqpoint{2.431415in}{1.224173in}}%
\pgfpathlineto{\pgfqpoint{2.431836in}{1.211038in}}%
\pgfpathlineto{\pgfqpoint{2.432257in}{1.224173in}}%
\pgfpathlineto{\pgfqpoint{2.432679in}{1.237308in}}%
\pgfpathlineto{\pgfqpoint{2.433100in}{1.211038in}}%
\pgfpathlineto{\pgfqpoint{2.433521in}{1.224173in}}%
\pgfpathlineto{\pgfqpoint{2.433942in}{1.237308in}}%
\pgfpathlineto{\pgfqpoint{2.434364in}{1.211038in}}%
\pgfpathlineto{\pgfqpoint{2.435206in}{1.224173in}}%
\pgfpathlineto{\pgfqpoint{2.435628in}{1.211038in}}%
\pgfpathlineto{\pgfqpoint{2.436049in}{1.224173in}}%
\pgfpathlineto{\pgfqpoint{2.436470in}{1.224173in}}%
\pgfpathlineto{\pgfqpoint{2.436891in}{1.211038in}}%
\pgfpathlineto{\pgfqpoint{2.437313in}{1.224173in}}%
\pgfpathlineto{\pgfqpoint{2.437734in}{1.224173in}}%
\pgfpathlineto{\pgfqpoint{2.438155in}{1.211038in}}%
\pgfpathlineto{\pgfqpoint{2.438577in}{1.224173in}}%
\pgfpathlineto{\pgfqpoint{2.438998in}{1.237308in}}%
\pgfpathlineto{\pgfqpoint{2.439419in}{1.224173in}}%
\pgfpathlineto{\pgfqpoint{2.439840in}{1.224173in}}%
\pgfpathlineto{\pgfqpoint{2.440262in}{1.237308in}}%
\pgfpathlineto{\pgfqpoint{2.440683in}{1.211038in}}%
\pgfpathlineto{\pgfqpoint{2.441525in}{1.224173in}}%
\pgfpathlineto{\pgfqpoint{2.441947in}{1.211038in}}%
\pgfpathlineto{\pgfqpoint{2.442368in}{1.224173in}}%
\pgfpathlineto{\pgfqpoint{2.443632in}{1.224173in}}%
\pgfpathlineto{\pgfqpoint{2.444053in}{1.250443in}}%
\pgfpathlineto{\pgfqpoint{2.444474in}{1.211038in}}%
\pgfpathlineto{\pgfqpoint{2.444896in}{1.237308in}}%
\pgfpathlineto{\pgfqpoint{2.445317in}{1.237308in}}%
\pgfpathlineto{\pgfqpoint{2.445738in}{1.197903in}}%
\pgfpathlineto{\pgfqpoint{2.446581in}{1.211038in}}%
\pgfpathlineto{\pgfqpoint{2.447002in}{1.211038in}}%
\pgfpathlineto{\pgfqpoint{2.447845in}{1.237308in}}%
\pgfpathlineto{\pgfqpoint{2.448266in}{1.211038in}}%
\pgfpathlineto{\pgfqpoint{2.449108in}{1.224173in}}%
\pgfpathlineto{\pgfqpoint{2.449530in}{1.224173in}}%
\pgfpathlineto{\pgfqpoint{2.450372in}{1.237308in}}%
\pgfpathlineto{\pgfqpoint{2.452057in}{1.211038in}}%
\pgfpathlineto{\pgfqpoint{2.453321in}{1.224173in}}%
\pgfpathlineto{\pgfqpoint{2.453743in}{1.224173in}}%
\pgfpathlineto{\pgfqpoint{2.455006in}{1.237308in}}%
\pgfpathlineto{\pgfqpoint{2.455428in}{1.237308in}}%
\pgfpathlineto{\pgfqpoint{2.456691in}{1.197903in}}%
\pgfpathlineto{\pgfqpoint{2.457113in}{1.211038in}}%
\pgfpathlineto{\pgfqpoint{2.457955in}{1.237308in}}%
\pgfpathlineto{\pgfqpoint{2.458377in}{1.224173in}}%
\pgfpathlineto{\pgfqpoint{2.458798in}{1.224173in}}%
\pgfpathlineto{\pgfqpoint{2.460062in}{1.250443in}}%
\pgfpathlineto{\pgfqpoint{2.461326in}{1.224173in}}%
\pgfpathlineto{\pgfqpoint{2.461747in}{1.224173in}}%
\pgfpathlineto{\pgfqpoint{2.462168in}{1.211038in}}%
\pgfpathlineto{\pgfqpoint{2.462589in}{1.237308in}}%
\pgfpathlineto{\pgfqpoint{2.463011in}{1.224173in}}%
\pgfpathlineto{\pgfqpoint{2.463432in}{1.211038in}}%
\pgfpathlineto{\pgfqpoint{2.463853in}{1.224173in}}%
\pgfpathlineto{\pgfqpoint{2.464274in}{1.237308in}}%
\pgfpathlineto{\pgfqpoint{2.464696in}{1.211038in}}%
\pgfpathlineto{\pgfqpoint{2.465117in}{1.237308in}}%
\pgfpathlineto{\pgfqpoint{2.465538in}{1.237308in}}%
\pgfpathlineto{\pgfqpoint{2.465960in}{1.211038in}}%
\pgfpathlineto{\pgfqpoint{2.466802in}{1.224173in}}%
\pgfpathlineto{\pgfqpoint{2.467223in}{1.237308in}}%
\pgfpathlineto{\pgfqpoint{2.467645in}{1.224173in}}%
\pgfpathlineto{\pgfqpoint{2.469330in}{1.224173in}}%
\pgfpathlineto{\pgfqpoint{2.470172in}{1.250443in}}%
\pgfpathlineto{\pgfqpoint{2.470594in}{1.237308in}}%
\pgfpathlineto{\pgfqpoint{2.471015in}{1.211038in}}%
\pgfpathlineto{\pgfqpoint{2.471858in}{1.224173in}}%
\pgfpathlineto{\pgfqpoint{2.472279in}{1.224173in}}%
\pgfpathlineto{\pgfqpoint{2.473543in}{1.237308in}}%
\pgfpathlineto{\pgfqpoint{2.473964in}{1.237308in}}%
\pgfpathlineto{\pgfqpoint{2.475228in}{1.224173in}}%
\pgfpathlineto{\pgfqpoint{2.475649in}{1.237308in}}%
\pgfpathlineto{\pgfqpoint{2.476070in}{1.211038in}}%
\pgfpathlineto{\pgfqpoint{2.476913in}{1.224173in}}%
\pgfpathlineto{\pgfqpoint{2.478177in}{1.237308in}}%
\pgfpathlineto{\pgfqpoint{2.479019in}{1.211038in}}%
\pgfpathlineto{\pgfqpoint{2.480283in}{1.237308in}}%
\pgfpathlineto{\pgfqpoint{2.480704in}{1.197903in}}%
\pgfpathlineto{\pgfqpoint{2.481126in}{1.211038in}}%
\pgfpathlineto{\pgfqpoint{2.481547in}{1.263578in}}%
\pgfpathlineto{\pgfqpoint{2.481968in}{1.224173in}}%
\pgfpathlineto{\pgfqpoint{2.482811in}{1.211038in}}%
\pgfpathlineto{\pgfqpoint{2.483232in}{1.224173in}}%
\pgfpathlineto{\pgfqpoint{2.483653in}{1.211038in}}%
\pgfpathlineto{\pgfqpoint{2.484075in}{1.211038in}}%
\pgfpathlineto{\pgfqpoint{2.484496in}{1.237308in}}%
\pgfpathlineto{\pgfqpoint{2.485338in}{1.224173in}}%
\pgfpathlineto{\pgfqpoint{2.487024in}{1.197903in}}%
\pgfpathlineto{\pgfqpoint{2.488709in}{1.237308in}}%
\pgfpathlineto{\pgfqpoint{2.489972in}{1.224173in}}%
\pgfpathlineto{\pgfqpoint{2.490394in}{1.237308in}}%
\pgfpathlineto{\pgfqpoint{2.491236in}{1.211038in}}%
\pgfpathlineto{\pgfqpoint{2.491658in}{1.224173in}}%
\pgfpathlineto{\pgfqpoint{2.493343in}{1.224173in}}%
\pgfpathlineto{\pgfqpoint{2.493764in}{1.211038in}}%
\pgfpathlineto{\pgfqpoint{2.494185in}{1.224173in}}%
\pgfpathlineto{\pgfqpoint{2.494607in}{1.224173in}}%
\pgfpathlineto{\pgfqpoint{2.495028in}{1.171633in}}%
\pgfpathlineto{\pgfqpoint{2.495449in}{1.211038in}}%
\pgfpathlineto{\pgfqpoint{2.495870in}{1.237308in}}%
\pgfpathlineto{\pgfqpoint{2.496713in}{1.224173in}}%
\pgfpathlineto{\pgfqpoint{2.497555in}{1.224173in}}%
\pgfpathlineto{\pgfqpoint{2.498819in}{1.211038in}}%
\pgfpathlineto{\pgfqpoint{2.499241in}{1.237308in}}%
\pgfpathlineto{\pgfqpoint{2.500083in}{1.224173in}}%
\pgfpathlineto{\pgfqpoint{2.500504in}{1.237308in}}%
\pgfpathlineto{\pgfqpoint{2.500926in}{1.224173in}}%
\pgfpathlineto{\pgfqpoint{2.502190in}{1.211038in}}%
\pgfpathlineto{\pgfqpoint{2.502611in}{1.211038in}}%
\pgfpathlineto{\pgfqpoint{2.503032in}{1.237308in}}%
\pgfpathlineto{\pgfqpoint{2.503453in}{1.224173in}}%
\pgfpathlineto{\pgfqpoint{2.503875in}{1.211038in}}%
\pgfpathlineto{\pgfqpoint{2.504296in}{1.224173in}}%
\pgfpathlineto{\pgfqpoint{2.505981in}{1.224173in}}%
\pgfpathlineto{\pgfqpoint{2.506402in}{1.197903in}}%
\pgfpathlineto{\pgfqpoint{2.506824in}{1.211038in}}%
\pgfpathlineto{\pgfqpoint{2.508087in}{1.237308in}}%
\pgfpathlineto{\pgfqpoint{2.508930in}{1.224173in}}%
\pgfpathlineto{\pgfqpoint{2.509773in}{1.237308in}}%
\pgfpathlineto{\pgfqpoint{2.510194in}{1.211038in}}%
\pgfpathlineto{\pgfqpoint{2.510615in}{1.224173in}}%
\pgfpathlineto{\pgfqpoint{2.511458in}{1.237308in}}%
\pgfpathlineto{\pgfqpoint{2.512721in}{1.211038in}}%
\pgfpathlineto{\pgfqpoint{2.513564in}{1.237308in}}%
\pgfpathlineto{\pgfqpoint{2.513985in}{1.211038in}}%
\pgfpathlineto{\pgfqpoint{2.514407in}{1.224173in}}%
\pgfpathlineto{\pgfqpoint{2.514828in}{1.237308in}}%
\pgfpathlineto{\pgfqpoint{2.515670in}{1.211038in}}%
\pgfpathlineto{\pgfqpoint{2.516092in}{1.237308in}}%
\pgfpathlineto{\pgfqpoint{2.516934in}{1.224173in}}%
\pgfpathlineto{\pgfqpoint{2.517356in}{1.224173in}}%
\pgfpathlineto{\pgfqpoint{2.517777in}{1.211038in}}%
\pgfpathlineto{\pgfqpoint{2.519041in}{1.237308in}}%
\pgfpathlineto{\pgfqpoint{2.519462in}{1.237308in}}%
\pgfpathlineto{\pgfqpoint{2.519883in}{1.250443in}}%
\pgfpathlineto{\pgfqpoint{2.521568in}{1.197903in}}%
\pgfpathlineto{\pgfqpoint{2.521990in}{1.237308in}}%
\pgfpathlineto{\pgfqpoint{2.522411in}{1.224173in}}%
\pgfpathlineto{\pgfqpoint{2.522832in}{1.197903in}}%
\pgfpathlineto{\pgfqpoint{2.523253in}{1.237308in}}%
\pgfpathlineto{\pgfqpoint{2.524517in}{1.224173in}}%
\pgfpathlineto{\pgfqpoint{2.525781in}{1.224173in}}%
\pgfpathlineto{\pgfqpoint{2.526202in}{0.646229in}}%
\pgfpathlineto{\pgfqpoint{2.526624in}{1.211038in}}%
\pgfpathlineto{\pgfqpoint{2.527887in}{1.224173in}}%
\pgfpathlineto{\pgfqpoint{2.528309in}{1.224173in}}%
\pgfpathlineto{\pgfqpoint{2.529151in}{1.237308in}}%
\pgfpathlineto{\pgfqpoint{2.529573in}{1.211038in}}%
\pgfpathlineto{\pgfqpoint{2.529994in}{1.250443in}}%
\pgfpathlineto{\pgfqpoint{2.531679in}{1.211038in}}%
\pgfpathlineto{\pgfqpoint{2.532100in}{1.224173in}}%
\pgfpathlineto{\pgfqpoint{2.532522in}{0.711904in}}%
\pgfpathlineto{\pgfqpoint{2.532943in}{1.224173in}}%
\pgfpathlineto{\pgfqpoint{2.533364in}{1.224173in}}%
\pgfpathlineto{\pgfqpoint{2.533785in}{1.237308in}}%
\pgfpathlineto{\pgfqpoint{2.534628in}{1.211038in}}%
\pgfpathlineto{\pgfqpoint{2.535049in}{1.237308in}}%
\pgfpathlineto{\pgfqpoint{2.535892in}{1.224173in}}%
\pgfpathlineto{\pgfqpoint{2.536313in}{1.224173in}}%
\pgfpathlineto{\pgfqpoint{2.536734in}{1.211038in}}%
\pgfpathlineto{\pgfqpoint{2.537156in}{1.237308in}}%
\pgfpathlineto{\pgfqpoint{2.537577in}{1.224173in}}%
\pgfpathlineto{\pgfqpoint{2.537998in}{1.211038in}}%
\pgfpathlineto{\pgfqpoint{2.538419in}{1.224173in}}%
\pgfpathlineto{\pgfqpoint{2.539262in}{1.224173in}}%
\pgfpathlineto{\pgfqpoint{2.539683in}{1.815253in}}%
\pgfpathlineto{\pgfqpoint{2.540105in}{1.237308in}}%
\pgfpathlineto{\pgfqpoint{2.540526in}{1.211038in}}%
\pgfpathlineto{\pgfqpoint{2.541368in}{1.224173in}}%
\pgfpathlineto{\pgfqpoint{2.541790in}{1.211038in}}%
\pgfpathlineto{\pgfqpoint{2.542211in}{1.237308in}}%
\pgfpathlineto{\pgfqpoint{2.542632in}{1.224173in}}%
\pgfpathlineto{\pgfqpoint{2.543054in}{0.948336in}}%
\pgfpathlineto{\pgfqpoint{2.543475in}{1.211038in}}%
\pgfpathlineto{\pgfqpoint{2.544739in}{1.237308in}}%
\pgfpathlineto{\pgfqpoint{2.545581in}{1.211038in}}%
\pgfpathlineto{\pgfqpoint{2.546002in}{1.723307in}}%
\pgfpathlineto{\pgfqpoint{2.546424in}{1.224173in}}%
\pgfpathlineto{\pgfqpoint{2.546845in}{1.197903in}}%
\pgfpathlineto{\pgfqpoint{2.547266in}{1.224173in}}%
\pgfpathlineto{\pgfqpoint{2.547688in}{1.224173in}}%
\pgfpathlineto{\pgfqpoint{2.548109in}{1.237308in}}%
\pgfpathlineto{\pgfqpoint{2.548530in}{1.224173in}}%
\pgfpathlineto{\pgfqpoint{2.549373in}{1.224173in}}%
\pgfpathlineto{\pgfqpoint{2.549794in}{1.237308in}}%
\pgfpathlineto{\pgfqpoint{2.550215in}{1.224173in}}%
\pgfpathlineto{\pgfqpoint{2.550637in}{1.211038in}}%
\pgfpathlineto{\pgfqpoint{2.551058in}{1.237308in}}%
\pgfpathlineto{\pgfqpoint{2.551479in}{1.224173in}}%
\pgfpathlineto{\pgfqpoint{2.551900in}{1.211038in}}%
\pgfpathlineto{\pgfqpoint{2.552322in}{1.237308in}}%
\pgfpathlineto{\pgfqpoint{2.552743in}{1.224173in}}%
\pgfpathlineto{\pgfqpoint{2.553585in}{1.211038in}}%
\pgfpathlineto{\pgfqpoint{2.554849in}{1.237308in}}%
\pgfpathlineto{\pgfqpoint{2.555692in}{1.197903in}}%
\pgfpathlineto{\pgfqpoint{2.556113in}{1.237308in}}%
\pgfpathlineto{\pgfqpoint{2.556534in}{1.500010in}}%
\pgfpathlineto{\pgfqpoint{2.556956in}{1.211038in}}%
\pgfpathlineto{\pgfqpoint{2.557798in}{1.237308in}}%
\pgfpathlineto{\pgfqpoint{2.558220in}{1.211038in}}%
\pgfpathlineto{\pgfqpoint{2.558641in}{1.224173in}}%
\pgfpathlineto{\pgfqpoint{2.559483in}{1.237308in}}%
\pgfpathlineto{\pgfqpoint{2.560326in}{1.250443in}}%
\pgfpathlineto{\pgfqpoint{2.560747in}{1.211038in}}%
\pgfpathlineto{\pgfqpoint{2.562432in}{1.250443in}}%
\pgfpathlineto{\pgfqpoint{2.562854in}{1.250443in}}%
\pgfpathlineto{\pgfqpoint{2.563275in}{1.211038in}}%
\pgfpathlineto{\pgfqpoint{2.563696in}{1.224173in}}%
\pgfpathlineto{\pgfqpoint{2.564117in}{1.250443in}}%
\pgfpathlineto{\pgfqpoint{2.564539in}{1.224173in}}%
\pgfpathlineto{\pgfqpoint{2.564960in}{1.224173in}}%
\pgfpathlineto{\pgfqpoint{2.565381in}{1.237308in}}%
\pgfpathlineto{\pgfqpoint{2.565803in}{1.224173in}}%
\pgfpathlineto{\pgfqpoint{2.566224in}{1.224173in}}%
\pgfpathlineto{\pgfqpoint{2.566645in}{1.237308in}}%
\pgfpathlineto{\pgfqpoint{2.567066in}{1.224173in}}%
\pgfpathlineto{\pgfqpoint{2.567488in}{1.211038in}}%
\pgfpathlineto{\pgfqpoint{2.567909in}{1.237308in}}%
\pgfpathlineto{\pgfqpoint{2.568330in}{1.211038in}}%
\pgfpathlineto{\pgfqpoint{2.568751in}{1.211038in}}%
\pgfpathlineto{\pgfqpoint{2.569173in}{1.250443in}}%
\pgfpathlineto{\pgfqpoint{2.569594in}{1.211038in}}%
\pgfpathlineto{\pgfqpoint{2.570437in}{1.250443in}}%
\pgfpathlineto{\pgfqpoint{2.570858in}{1.197903in}}%
\pgfpathlineto{\pgfqpoint{2.571279in}{1.211038in}}%
\pgfpathlineto{\pgfqpoint{2.572122in}{1.237308in}}%
\pgfpathlineto{\pgfqpoint{2.572543in}{1.211038in}}%
\pgfpathlineto{\pgfqpoint{2.573386in}{1.224173in}}%
\pgfpathlineto{\pgfqpoint{2.574649in}{1.211038in}}%
\pgfpathlineto{\pgfqpoint{2.575071in}{1.211038in}}%
\pgfpathlineto{\pgfqpoint{2.575492in}{1.237308in}}%
\pgfpathlineto{\pgfqpoint{2.575913in}{1.197903in}}%
\pgfpathlineto{\pgfqpoint{2.576334in}{1.197903in}}%
\pgfpathlineto{\pgfqpoint{2.576756in}{1.250443in}}%
\pgfpathlineto{\pgfqpoint{2.577177in}{1.224173in}}%
\pgfpathlineto{\pgfqpoint{2.578020in}{1.211038in}}%
\pgfpathlineto{\pgfqpoint{2.578441in}{1.237308in}}%
\pgfpathlineto{\pgfqpoint{2.579283in}{1.224173in}}%
\pgfpathlineto{\pgfqpoint{2.579705in}{1.224173in}}%
\pgfpathlineto{\pgfqpoint{2.580126in}{1.211038in}}%
\pgfpathlineto{\pgfqpoint{2.580969in}{1.237308in}}%
\pgfpathlineto{\pgfqpoint{2.581390in}{1.211038in}}%
\pgfpathlineto{\pgfqpoint{2.581811in}{1.237308in}}%
\pgfpathlineto{\pgfqpoint{2.582232in}{1.237308in}}%
\pgfpathlineto{\pgfqpoint{2.583075in}{1.250443in}}%
\pgfpathlineto{\pgfqpoint{2.583917in}{1.197903in}}%
\pgfpathlineto{\pgfqpoint{2.584760in}{1.224173in}}%
\pgfpathlineto{\pgfqpoint{2.585181in}{1.211038in}}%
\pgfpathlineto{\pgfqpoint{2.586024in}{1.237308in}}%
\pgfpathlineto{\pgfqpoint{2.586445in}{1.211038in}}%
\pgfpathlineto{\pgfqpoint{2.586866in}{1.250443in}}%
\pgfpathlineto{\pgfqpoint{2.587288in}{1.224173in}}%
\pgfpathlineto{\pgfqpoint{2.587709in}{1.224173in}}%
\pgfpathlineto{\pgfqpoint{2.588552in}{1.237308in}}%
\pgfpathlineto{\pgfqpoint{2.590237in}{1.197903in}}%
\pgfpathlineto{\pgfqpoint{2.590658in}{1.237308in}}%
\pgfpathlineto{\pgfqpoint{2.591500in}{1.224173in}}%
\pgfpathlineto{\pgfqpoint{2.592343in}{1.224173in}}%
\pgfpathlineto{\pgfqpoint{2.592764in}{1.211038in}}%
\pgfpathlineto{\pgfqpoint{2.593186in}{1.237308in}}%
\pgfpathlineto{\pgfqpoint{2.593607in}{1.224173in}}%
\pgfpathlineto{\pgfqpoint{2.594449in}{1.211038in}}%
\pgfpathlineto{\pgfqpoint{2.595713in}{1.237308in}}%
\pgfpathlineto{\pgfqpoint{2.596556in}{1.197903in}}%
\pgfpathlineto{\pgfqpoint{2.596977in}{1.237308in}}%
\pgfpathlineto{\pgfqpoint{2.597820in}{1.224173in}}%
\pgfpathlineto{\pgfqpoint{2.598241in}{1.237308in}}%
\pgfpathlineto{\pgfqpoint{2.598662in}{1.224173in}}%
\pgfpathlineto{\pgfqpoint{2.599084in}{1.211038in}}%
\pgfpathlineto{\pgfqpoint{2.599505in}{1.224173in}}%
\pgfpathlineto{\pgfqpoint{2.599926in}{1.237308in}}%
\pgfpathlineto{\pgfqpoint{2.600347in}{1.211038in}}%
\pgfpathlineto{\pgfqpoint{2.601190in}{1.224173in}}%
\pgfpathlineto{\pgfqpoint{2.602454in}{1.211038in}}%
\pgfpathlineto{\pgfqpoint{2.602875in}{1.211038in}}%
\pgfpathlineto{\pgfqpoint{2.603718in}{1.237308in}}%
\pgfpathlineto{\pgfqpoint{2.605403in}{1.197903in}}%
\pgfpathlineto{\pgfqpoint{2.605824in}{1.237308in}}%
\pgfpathlineto{\pgfqpoint{2.606245in}{1.224173in}}%
\pgfpathlineto{\pgfqpoint{2.606667in}{1.211038in}}%
\pgfpathlineto{\pgfqpoint{2.607930in}{1.250443in}}%
\pgfpathlineto{\pgfqpoint{2.609194in}{1.211038in}}%
\pgfpathlineto{\pgfqpoint{2.610037in}{1.237308in}}%
\pgfpathlineto{\pgfqpoint{2.610458in}{1.211038in}}%
\pgfpathlineto{\pgfqpoint{2.610879in}{1.250443in}}%
\pgfpathlineto{\pgfqpoint{2.611301in}{1.224173in}}%
\pgfpathlineto{\pgfqpoint{2.611722in}{1.211038in}}%
\pgfpathlineto{\pgfqpoint{2.612143in}{1.237308in}}%
\pgfpathlineto{\pgfqpoint{2.612986in}{1.224173in}}%
\pgfpathlineto{\pgfqpoint{2.613407in}{1.224173in}}%
\pgfpathlineto{\pgfqpoint{2.613828in}{1.211038in}}%
\pgfpathlineto{\pgfqpoint{2.614250in}{1.224173in}}%
\pgfpathlineto{\pgfqpoint{2.614671in}{1.237308in}}%
\pgfpathlineto{\pgfqpoint{2.615092in}{1.224173in}}%
\pgfpathlineto{\pgfqpoint{2.615513in}{1.211038in}}%
\pgfpathlineto{\pgfqpoint{2.615935in}{1.224173in}}%
\pgfpathlineto{\pgfqpoint{2.616356in}{1.224173in}}%
\pgfpathlineto{\pgfqpoint{2.617198in}{1.211038in}}%
\pgfpathlineto{\pgfqpoint{2.618041in}{1.224173in}}%
\pgfpathlineto{\pgfqpoint{2.618462in}{1.211038in}}%
\pgfpathlineto{\pgfqpoint{2.619726in}{1.237308in}}%
\pgfpathlineto{\pgfqpoint{2.620147in}{1.224173in}}%
\pgfpathlineto{\pgfqpoint{2.620569in}{1.237308in}}%
\pgfpathlineto{\pgfqpoint{2.620990in}{1.237308in}}%
\pgfpathlineto{\pgfqpoint{2.621411in}{1.197903in}}%
\pgfpathlineto{\pgfqpoint{2.621833in}{1.211038in}}%
\pgfpathlineto{\pgfqpoint{2.622254in}{1.237308in}}%
\pgfpathlineto{\pgfqpoint{2.622675in}{1.224173in}}%
\pgfpathlineto{\pgfqpoint{2.623096in}{1.211038in}}%
\pgfpathlineto{\pgfqpoint{2.623518in}{1.224173in}}%
\pgfpathlineto{\pgfqpoint{2.623939in}{1.224173in}}%
\pgfpathlineto{\pgfqpoint{2.624360in}{1.197903in}}%
\pgfpathlineto{\pgfqpoint{2.624781in}{1.224173in}}%
\pgfpathlineto{\pgfqpoint{2.625203in}{1.237308in}}%
\pgfpathlineto{\pgfqpoint{2.626467in}{1.211038in}}%
\pgfpathlineto{\pgfqpoint{2.627309in}{1.237308in}}%
\pgfpathlineto{\pgfqpoint{2.627730in}{1.211038in}}%
\pgfpathlineto{\pgfqpoint{2.628152in}{1.224173in}}%
\pgfpathlineto{\pgfqpoint{2.628573in}{1.237308in}}%
\pgfpathlineto{\pgfqpoint{2.628994in}{1.224173in}}%
\pgfpathlineto{\pgfqpoint{2.629416in}{1.224173in}}%
\pgfpathlineto{\pgfqpoint{2.629837in}{1.237308in}}%
\pgfpathlineto{\pgfqpoint{2.630258in}{1.224173in}}%
\pgfpathlineto{\pgfqpoint{2.630679in}{1.224173in}}%
\pgfpathlineto{\pgfqpoint{2.631943in}{1.250443in}}%
\pgfpathlineto{\pgfqpoint{2.632786in}{1.224173in}}%
\pgfpathlineto{\pgfqpoint{2.633207in}{1.237308in}}%
\pgfpathlineto{\pgfqpoint{2.633628in}{1.237308in}}%
\pgfpathlineto{\pgfqpoint{2.634050in}{1.197903in}}%
\pgfpathlineto{\pgfqpoint{2.634471in}{1.211038in}}%
\pgfpathlineto{\pgfqpoint{2.635735in}{1.263578in}}%
\pgfpathlineto{\pgfqpoint{2.636156in}{1.250443in}}%
\pgfpathlineto{\pgfqpoint{2.636577in}{1.224173in}}%
\pgfpathlineto{\pgfqpoint{2.637420in}{1.237308in}}%
\pgfpathlineto{\pgfqpoint{2.637841in}{1.237308in}}%
\pgfpathlineto{\pgfqpoint{2.639526in}{1.211038in}}%
\pgfpathlineto{\pgfqpoint{2.639947in}{1.237308in}}%
\pgfpathlineto{\pgfqpoint{2.640369in}{1.211038in}}%
\pgfpathlineto{\pgfqpoint{2.640790in}{1.211038in}}%
\pgfpathlineto{\pgfqpoint{2.641211in}{1.224173in}}%
\pgfpathlineto{\pgfqpoint{2.641633in}{1.211038in}}%
\pgfpathlineto{\pgfqpoint{2.642054in}{1.211038in}}%
\pgfpathlineto{\pgfqpoint{2.642475in}{1.237308in}}%
\pgfpathlineto{\pgfqpoint{2.642896in}{1.224173in}}%
\pgfpathlineto{\pgfqpoint{2.643318in}{1.211038in}}%
\pgfpathlineto{\pgfqpoint{2.643739in}{1.224173in}}%
\pgfpathlineto{\pgfqpoint{2.644160in}{1.224173in}}%
\pgfpathlineto{\pgfqpoint{2.644582in}{1.211038in}}%
\pgfpathlineto{\pgfqpoint{2.645003in}{1.237308in}}%
\pgfpathlineto{\pgfqpoint{2.645424in}{1.211038in}}%
\pgfpathlineto{\pgfqpoint{2.645845in}{1.211038in}}%
\pgfpathlineto{\pgfqpoint{2.646267in}{1.250443in}}%
\pgfpathlineto{\pgfqpoint{2.646688in}{1.197903in}}%
\pgfpathlineto{\pgfqpoint{2.647530in}{1.237308in}}%
\pgfpathlineto{\pgfqpoint{2.647952in}{1.224173in}}%
\pgfpathlineto{\pgfqpoint{2.648373in}{1.211038in}}%
\pgfpathlineto{\pgfqpoint{2.648794in}{1.237308in}}%
\pgfpathlineto{\pgfqpoint{2.649216in}{1.171633in}}%
\pgfpathlineto{\pgfqpoint{2.649637in}{1.197903in}}%
\pgfpathlineto{\pgfqpoint{2.650058in}{1.237308in}}%
\pgfpathlineto{\pgfqpoint{2.650479in}{1.197903in}}%
\pgfpathlineto{\pgfqpoint{2.652586in}{1.237308in}}%
\pgfpathlineto{\pgfqpoint{2.653428in}{1.211038in}}%
\pgfpathlineto{\pgfqpoint{2.654271in}{1.250443in}}%
\pgfpathlineto{\pgfqpoint{2.654692in}{1.237308in}}%
\pgfpathlineto{\pgfqpoint{2.655113in}{1.237308in}}%
\pgfpathlineto{\pgfqpoint{2.655956in}{1.211038in}}%
\pgfpathlineto{\pgfqpoint{2.656799in}{1.237308in}}%
\pgfpathlineto{\pgfqpoint{2.657220in}{1.224173in}}%
\pgfpathlineto{\pgfqpoint{2.658062in}{1.237308in}}%
\pgfpathlineto{\pgfqpoint{2.659748in}{1.197903in}}%
\pgfpathlineto{\pgfqpoint{2.660169in}{1.237308in}}%
\pgfpathlineto{\pgfqpoint{2.660590in}{1.224173in}}%
\pgfpathlineto{\pgfqpoint{2.661011in}{1.211038in}}%
\pgfpathlineto{\pgfqpoint{2.661433in}{1.237308in}}%
\pgfpathlineto{\pgfqpoint{2.661854in}{1.224173in}}%
\pgfpathlineto{\pgfqpoint{2.662275in}{1.211038in}}%
\pgfpathlineto{\pgfqpoint{2.662696in}{1.237308in}}%
\pgfpathlineto{\pgfqpoint{2.663539in}{1.224173in}}%
\pgfpathlineto{\pgfqpoint{2.664803in}{1.224173in}}%
\pgfpathlineto{\pgfqpoint{2.665645in}{1.237308in}}%
\pgfpathlineto{\pgfqpoint{2.667331in}{1.197903in}}%
\pgfpathlineto{\pgfqpoint{2.669437in}{1.237308in}}%
\pgfpathlineto{\pgfqpoint{2.670280in}{1.211038in}}%
\pgfpathlineto{\pgfqpoint{2.670701in}{1.224173in}}%
\pgfpathlineto{\pgfqpoint{2.671543in}{1.211038in}}%
\pgfpathlineto{\pgfqpoint{2.672807in}{1.237308in}}%
\pgfpathlineto{\pgfqpoint{2.674071in}{1.211038in}}%
\pgfpathlineto{\pgfqpoint{2.675335in}{1.237308in}}%
\pgfpathlineto{\pgfqpoint{2.676177in}{1.211038in}}%
\pgfpathlineto{\pgfqpoint{2.677441in}{1.237308in}}%
\pgfpathlineto{\pgfqpoint{2.678705in}{1.211038in}}%
\pgfpathlineto{\pgfqpoint{2.679548in}{1.224173in}}%
\pgfpathlineto{\pgfqpoint{2.679969in}{1.211038in}}%
\pgfpathlineto{\pgfqpoint{2.680390in}{1.237308in}}%
\pgfpathlineto{\pgfqpoint{2.680811in}{1.224173in}}%
\pgfpathlineto{\pgfqpoint{2.681233in}{1.211038in}}%
\pgfpathlineto{\pgfqpoint{2.681654in}{1.250443in}}%
\pgfpathlineto{\pgfqpoint{2.682075in}{1.224173in}}%
\pgfpathlineto{\pgfqpoint{2.683339in}{1.211038in}}%
\pgfpathlineto{\pgfqpoint{2.684603in}{1.250443in}}%
\pgfpathlineto{\pgfqpoint{2.685024in}{1.237308in}}%
\pgfpathlineto{\pgfqpoint{2.686288in}{1.224173in}}%
\pgfpathlineto{\pgfqpoint{2.686709in}{1.250443in}}%
\pgfpathlineto{\pgfqpoint{2.687552in}{1.237308in}}%
\pgfpathlineto{\pgfqpoint{2.687973in}{1.237308in}}%
\pgfpathlineto{\pgfqpoint{2.688816in}{1.224173in}}%
\pgfpathlineto{\pgfqpoint{2.689658in}{1.237308in}}%
\pgfpathlineto{\pgfqpoint{2.690080in}{1.224173in}}%
\pgfpathlineto{\pgfqpoint{2.690501in}{1.250443in}}%
\pgfpathlineto{\pgfqpoint{2.691343in}{1.237308in}}%
\pgfpathlineto{\pgfqpoint{2.692186in}{1.237308in}}%
\pgfpathlineto{\pgfqpoint{2.693450in}{1.224173in}}%
\pgfpathlineto{\pgfqpoint{2.693871in}{1.224173in}}%
\pgfpathlineto{\pgfqpoint{2.694714in}{1.237308in}}%
\pgfpathlineto{\pgfqpoint{2.695135in}{1.197903in}}%
\pgfpathlineto{\pgfqpoint{2.695556in}{1.237308in}}%
\pgfpathlineto{\pgfqpoint{2.695977in}{1.237308in}}%
\pgfpathlineto{\pgfqpoint{2.696399in}{1.224173in}}%
\pgfpathlineto{\pgfqpoint{2.696820in}{1.250443in}}%
\pgfpathlineto{\pgfqpoint{2.697241in}{1.224173in}}%
\pgfpathlineto{\pgfqpoint{2.698084in}{1.197903in}}%
\pgfpathlineto{\pgfqpoint{2.698505in}{1.211038in}}%
\pgfpathlineto{\pgfqpoint{2.698926in}{1.224173in}}%
\pgfpathlineto{\pgfqpoint{2.699348in}{1.211038in}}%
\pgfpathlineto{\pgfqpoint{2.699769in}{1.211038in}}%
\pgfpathlineto{\pgfqpoint{2.700190in}{1.197903in}}%
\pgfpathlineto{\pgfqpoint{2.700612in}{1.224173in}}%
\pgfpathlineto{\pgfqpoint{2.701454in}{1.211038in}}%
\pgfpathlineto{\pgfqpoint{2.702718in}{1.224173in}}%
\pgfpathlineto{\pgfqpoint{2.703139in}{1.224173in}}%
\pgfpathlineto{\pgfqpoint{2.703982in}{1.197903in}}%
\pgfpathlineto{\pgfqpoint{2.704403in}{1.211038in}}%
\pgfpathlineto{\pgfqpoint{2.705246in}{1.211038in}}%
\pgfpathlineto{\pgfqpoint{2.705667in}{1.224173in}}%
\pgfpathlineto{\pgfqpoint{2.706088in}{1.211038in}}%
\pgfpathlineto{\pgfqpoint{2.706509in}{1.211038in}}%
\pgfpathlineto{\pgfqpoint{2.706931in}{1.237308in}}%
\pgfpathlineto{\pgfqpoint{2.707352in}{1.211038in}}%
\pgfpathlineto{\pgfqpoint{2.707773in}{1.211038in}}%
\pgfpathlineto{\pgfqpoint{2.708616in}{1.224173in}}%
\pgfpathlineto{\pgfqpoint{2.709037in}{1.211038in}}%
\pgfpathlineto{\pgfqpoint{2.709458in}{1.237308in}}%
\pgfpathlineto{\pgfqpoint{2.710301in}{1.224173in}}%
\pgfpathlineto{\pgfqpoint{2.711565in}{1.224173in}}%
\pgfpathlineto{\pgfqpoint{2.711986in}{1.237308in}}%
\pgfpathlineto{\pgfqpoint{2.712407in}{1.224173in}}%
\pgfpathlineto{\pgfqpoint{2.712829in}{1.224173in}}%
\pgfpathlineto{\pgfqpoint{2.713250in}{1.250443in}}%
\pgfpathlineto{\pgfqpoint{2.713671in}{1.224173in}}%
\pgfpathlineto{\pgfqpoint{2.714514in}{1.211038in}}%
\pgfpathlineto{\pgfqpoint{2.715778in}{1.224173in}}%
\pgfpathlineto{\pgfqpoint{2.716199in}{1.197903in}}%
\pgfpathlineto{\pgfqpoint{2.716620in}{1.224173in}}%
\pgfpathlineto{\pgfqpoint{2.717463in}{1.237308in}}%
\pgfpathlineto{\pgfqpoint{2.718726in}{1.224173in}}%
\pgfpathlineto{\pgfqpoint{2.719569in}{1.237308in}}%
\pgfpathlineto{\pgfqpoint{2.720412in}{1.224173in}}%
\pgfpathlineto{\pgfqpoint{2.720833in}{1.237308in}}%
\pgfpathlineto{\pgfqpoint{2.721254in}{1.224173in}}%
\pgfpathlineto{\pgfqpoint{2.721675in}{1.211038in}}%
\pgfpathlineto{\pgfqpoint{2.722097in}{1.237308in}}%
\pgfpathlineto{\pgfqpoint{2.722518in}{1.224173in}}%
\pgfpathlineto{\pgfqpoint{2.722939in}{1.197903in}}%
\pgfpathlineto{\pgfqpoint{2.723361in}{1.224173in}}%
\pgfpathlineto{\pgfqpoint{2.725046in}{1.224173in}}%
\pgfpathlineto{\pgfqpoint{2.725467in}{1.211038in}}%
\pgfpathlineto{\pgfqpoint{2.725888in}{1.224173in}}%
\pgfpathlineto{\pgfqpoint{2.726731in}{1.224173in}}%
\pgfpathlineto{\pgfqpoint{2.727995in}{1.237308in}}%
\pgfpathlineto{\pgfqpoint{2.728416in}{1.237308in}}%
\pgfpathlineto{\pgfqpoint{2.728837in}{1.211038in}}%
\pgfpathlineto{\pgfqpoint{2.729258in}{1.224173in}}%
\pgfpathlineto{\pgfqpoint{2.729680in}{1.250443in}}%
\pgfpathlineto{\pgfqpoint{2.730101in}{1.224173in}}%
\pgfpathlineto{\pgfqpoint{2.730944in}{1.211038in}}%
\pgfpathlineto{\pgfqpoint{2.732207in}{1.237308in}}%
\pgfpathlineto{\pgfqpoint{2.732629in}{1.197903in}}%
\pgfpathlineto{\pgfqpoint{2.733050in}{1.237308in}}%
\pgfpathlineto{\pgfqpoint{2.733893in}{1.197903in}}%
\pgfpathlineto{\pgfqpoint{2.734314in}{1.211038in}}%
\pgfpathlineto{\pgfqpoint{2.734735in}{1.237308in}}%
\pgfpathlineto{\pgfqpoint{2.735156in}{1.224173in}}%
\pgfpathlineto{\pgfqpoint{2.735578in}{1.211038in}}%
\pgfpathlineto{\pgfqpoint{2.735999in}{1.224173in}}%
\pgfpathlineto{\pgfqpoint{2.738105in}{1.224173in}}%
\pgfpathlineto{\pgfqpoint{2.738527in}{1.250443in}}%
\pgfpathlineto{\pgfqpoint{2.738948in}{1.224173in}}%
\pgfpathlineto{\pgfqpoint{2.739369in}{1.224173in}}%
\pgfpathlineto{\pgfqpoint{2.739790in}{1.237308in}}%
\pgfpathlineto{\pgfqpoint{2.740212in}{1.197903in}}%
\pgfpathlineto{\pgfqpoint{2.740633in}{1.211038in}}%
\pgfpathlineto{\pgfqpoint{2.741054in}{1.237308in}}%
\pgfpathlineto{\pgfqpoint{2.741476in}{1.224173in}}%
\pgfpathlineto{\pgfqpoint{2.741897in}{1.211038in}}%
\pgfpathlineto{\pgfqpoint{2.742318in}{1.224173in}}%
\pgfpathlineto{\pgfqpoint{2.742739in}{1.237308in}}%
\pgfpathlineto{\pgfqpoint{2.743161in}{1.211038in}}%
\pgfpathlineto{\pgfqpoint{2.744003in}{1.224173in}}%
\pgfpathlineto{\pgfqpoint{2.744424in}{1.224173in}}%
\pgfpathlineto{\pgfqpoint{2.744846in}{1.250443in}}%
\pgfpathlineto{\pgfqpoint{2.745267in}{1.211038in}}%
\pgfpathlineto{\pgfqpoint{2.745688in}{1.211038in}}%
\pgfpathlineto{\pgfqpoint{2.746110in}{1.224173in}}%
\pgfpathlineto{\pgfqpoint{2.746531in}{1.197903in}}%
\pgfpathlineto{\pgfqpoint{2.746952in}{1.211038in}}%
\pgfpathlineto{\pgfqpoint{2.747373in}{1.250443in}}%
\pgfpathlineto{\pgfqpoint{2.747795in}{1.224173in}}%
\pgfpathlineto{\pgfqpoint{2.748216in}{1.211038in}}%
\pgfpathlineto{\pgfqpoint{2.748637in}{1.224173in}}%
\pgfpathlineto{\pgfqpoint{2.749480in}{1.224173in}}%
\pgfpathlineto{\pgfqpoint{2.749901in}{1.250443in}}%
\pgfpathlineto{\pgfqpoint{2.750322in}{1.211038in}}%
\pgfpathlineto{\pgfqpoint{2.750744in}{1.211038in}}%
\pgfpathlineto{\pgfqpoint{2.751586in}{1.224173in}}%
\pgfpathlineto{\pgfqpoint{2.752007in}{1.197903in}}%
\pgfpathlineto{\pgfqpoint{2.752429in}{1.237308in}}%
\pgfpathlineto{\pgfqpoint{2.753271in}{1.211038in}}%
\pgfpathlineto{\pgfqpoint{2.753693in}{1.250443in}}%
\pgfpathlineto{\pgfqpoint{2.754114in}{1.237308in}}%
\pgfpathlineto{\pgfqpoint{2.754535in}{1.211038in}}%
\pgfpathlineto{\pgfqpoint{2.755378in}{1.224173in}}%
\pgfpathlineto{\pgfqpoint{2.755799in}{1.224173in}}%
\pgfpathlineto{\pgfqpoint{2.757063in}{1.211038in}}%
\pgfpathlineto{\pgfqpoint{2.757905in}{1.224173in}}%
\pgfpathlineto{\pgfqpoint{2.758327in}{1.197903in}}%
\pgfpathlineto{\pgfqpoint{2.758748in}{1.224173in}}%
\pgfpathlineto{\pgfqpoint{2.760012in}{1.224173in}}%
\pgfpathlineto{\pgfqpoint{2.760433in}{1.250443in}}%
\pgfpathlineto{\pgfqpoint{2.760854in}{1.211038in}}%
\pgfpathlineto{\pgfqpoint{2.761276in}{1.237308in}}%
\pgfpathlineto{\pgfqpoint{2.762118in}{1.224173in}}%
\pgfpathlineto{\pgfqpoint{2.762539in}{1.237308in}}%
\pgfpathlineto{\pgfqpoint{2.762961in}{1.224173in}}%
\pgfpathlineto{\pgfqpoint{2.763382in}{1.197903in}}%
\pgfpathlineto{\pgfqpoint{2.763803in}{1.237308in}}%
\pgfpathlineto{\pgfqpoint{2.765067in}{1.224173in}}%
\pgfpathlineto{\pgfqpoint{2.765488in}{1.224173in}}%
\pgfpathlineto{\pgfqpoint{2.765910in}{1.211038in}}%
\pgfpathlineto{\pgfqpoint{2.766331in}{1.237308in}}%
\pgfpathlineto{\pgfqpoint{2.766752in}{1.224173in}}%
\pgfpathlineto{\pgfqpoint{2.767173in}{1.197903in}}%
\pgfpathlineto{\pgfqpoint{2.767595in}{1.224173in}}%
\pgfpathlineto{\pgfqpoint{2.769701in}{1.224173in}}%
\pgfpathlineto{\pgfqpoint{2.770122in}{1.237308in}}%
\pgfpathlineto{\pgfqpoint{2.770544in}{1.224173in}}%
\pgfpathlineto{\pgfqpoint{2.770965in}{1.224173in}}%
\pgfpathlineto{\pgfqpoint{2.771386in}{1.237308in}}%
\pgfpathlineto{\pgfqpoint{2.771808in}{1.224173in}}%
\pgfpathlineto{\pgfqpoint{2.772229in}{1.224173in}}%
\pgfpathlineto{\pgfqpoint{2.772650in}{1.237308in}}%
\pgfpathlineto{\pgfqpoint{2.773071in}{1.224173in}}%
\pgfpathlineto{\pgfqpoint{2.774335in}{1.224173in}}%
\pgfpathlineto{\pgfqpoint{2.774756in}{1.211038in}}%
\pgfpathlineto{\pgfqpoint{2.775178in}{1.224173in}}%
\pgfpathlineto{\pgfqpoint{2.775599in}{1.224173in}}%
\pgfpathlineto{\pgfqpoint{2.776020in}{1.211038in}}%
\pgfpathlineto{\pgfqpoint{2.776863in}{1.237308in}}%
\pgfpathlineto{\pgfqpoint{2.777284in}{1.211038in}}%
\pgfpathlineto{\pgfqpoint{2.778127in}{1.224173in}}%
\pgfpathlineto{\pgfqpoint{2.778548in}{1.224173in}}%
\pgfpathlineto{\pgfqpoint{2.778969in}{1.237308in}}%
\pgfpathlineto{\pgfqpoint{2.779391in}{1.224173in}}%
\pgfpathlineto{\pgfqpoint{2.779812in}{1.211038in}}%
\pgfpathlineto{\pgfqpoint{2.780233in}{1.237308in}}%
\pgfpathlineto{\pgfqpoint{2.780654in}{1.224173in}}%
\pgfpathlineto{\pgfqpoint{2.781076in}{1.211038in}}%
\pgfpathlineto{\pgfqpoint{2.781497in}{1.237308in}}%
\pgfpathlineto{\pgfqpoint{2.782339in}{1.224173in}}%
\pgfpathlineto{\pgfqpoint{2.782761in}{1.237308in}}%
\pgfpathlineto{\pgfqpoint{2.783182in}{1.224173in}}%
\pgfpathlineto{\pgfqpoint{2.783603in}{1.211038in}}%
\pgfpathlineto{\pgfqpoint{2.784025in}{1.237308in}}%
\pgfpathlineto{\pgfqpoint{2.784446in}{1.224173in}}%
\pgfpathlineto{\pgfqpoint{2.784867in}{1.211038in}}%
\pgfpathlineto{\pgfqpoint{2.785288in}{1.224173in}}%
\pgfpathlineto{\pgfqpoint{2.785710in}{1.224173in}}%
\pgfpathlineto{\pgfqpoint{2.786131in}{1.211038in}}%
\pgfpathlineto{\pgfqpoint{2.786552in}{1.237308in}}%
\pgfpathlineto{\pgfqpoint{2.786974in}{1.224173in}}%
\pgfpathlineto{\pgfqpoint{2.787395in}{1.211038in}}%
\pgfpathlineto{\pgfqpoint{2.787816in}{1.250443in}}%
\pgfpathlineto{\pgfqpoint{2.788659in}{1.237308in}}%
\pgfpathlineto{\pgfqpoint{2.789080in}{1.237308in}}%
\pgfpathlineto{\pgfqpoint{2.789922in}{1.211038in}}%
\pgfpathlineto{\pgfqpoint{2.790344in}{1.224173in}}%
\pgfpathlineto{\pgfqpoint{2.791186in}{1.224173in}}%
\pgfpathlineto{\pgfqpoint{2.791608in}{1.250443in}}%
\pgfpathlineto{\pgfqpoint{2.792029in}{1.224173in}}%
\pgfpathlineto{\pgfqpoint{2.792450in}{1.211038in}}%
\pgfpathlineto{\pgfqpoint{2.792871in}{1.237308in}}%
\pgfpathlineto{\pgfqpoint{2.793293in}{1.211038in}}%
\pgfpathlineto{\pgfqpoint{2.793714in}{1.211038in}}%
\pgfpathlineto{\pgfqpoint{2.794135in}{1.237308in}}%
\pgfpathlineto{\pgfqpoint{2.794557in}{1.224173in}}%
\pgfpathlineto{\pgfqpoint{2.794978in}{1.211038in}}%
\pgfpathlineto{\pgfqpoint{2.795399in}{1.237308in}}%
\pgfpathlineto{\pgfqpoint{2.795820in}{1.224173in}}%
\pgfpathlineto{\pgfqpoint{2.796242in}{1.211038in}}%
\pgfpathlineto{\pgfqpoint{2.796663in}{1.224173in}}%
\pgfpathlineto{\pgfqpoint{2.797084in}{1.224173in}}%
\pgfpathlineto{\pgfqpoint{2.797927in}{1.211038in}}%
\pgfpathlineto{\pgfqpoint{2.799191in}{1.224173in}}%
\pgfpathlineto{\pgfqpoint{2.799612in}{1.224173in}}%
\pgfpathlineto{\pgfqpoint{2.800033in}{1.211038in}}%
\pgfpathlineto{\pgfqpoint{2.800454in}{1.237308in}}%
\pgfpathlineto{\pgfqpoint{2.800876in}{1.224173in}}%
\pgfpathlineto{\pgfqpoint{2.801297in}{1.197903in}}%
\pgfpathlineto{\pgfqpoint{2.801718in}{1.237308in}}%
\pgfpathlineto{\pgfqpoint{2.802561in}{1.211038in}}%
\pgfpathlineto{\pgfqpoint{2.802982in}{1.224173in}}%
\pgfpathlineto{\pgfqpoint{2.803403in}{1.224173in}}%
\pgfpathlineto{\pgfqpoint{2.803825in}{1.211038in}}%
\pgfpathlineto{\pgfqpoint{2.804246in}{1.237308in}}%
\pgfpathlineto{\pgfqpoint{2.804667in}{1.224173in}}%
\pgfpathlineto{\pgfqpoint{2.805089in}{1.197903in}}%
\pgfpathlineto{\pgfqpoint{2.805510in}{1.224173in}}%
\pgfpathlineto{\pgfqpoint{2.805931in}{1.224173in}}%
\pgfpathlineto{\pgfqpoint{2.806352in}{1.211038in}}%
\pgfpathlineto{\pgfqpoint{2.806774in}{1.263578in}}%
\pgfpathlineto{\pgfqpoint{2.807195in}{1.224173in}}%
\pgfpathlineto{\pgfqpoint{2.807616in}{1.211038in}}%
\pgfpathlineto{\pgfqpoint{2.808037in}{1.237308in}}%
\pgfpathlineto{\pgfqpoint{2.808459in}{1.224173in}}%
\pgfpathlineto{\pgfqpoint{2.808880in}{1.211038in}}%
\pgfpathlineto{\pgfqpoint{2.810565in}{1.250443in}}%
\pgfpathlineto{\pgfqpoint{2.812250in}{1.224173in}}%
\pgfpathlineto{\pgfqpoint{2.812672in}{1.224173in}}%
\pgfpathlineto{\pgfqpoint{2.813093in}{1.237308in}}%
\pgfpathlineto{\pgfqpoint{2.813514in}{1.224173in}}%
\pgfpathlineto{\pgfqpoint{2.813935in}{1.224173in}}%
\pgfpathlineto{\pgfqpoint{2.814357in}{1.237308in}}%
\pgfpathlineto{\pgfqpoint{2.814778in}{1.224173in}}%
\pgfpathlineto{\pgfqpoint{2.815199in}{1.211038in}}%
\pgfpathlineto{\pgfqpoint{2.815620in}{1.224173in}}%
\pgfpathlineto{\pgfqpoint{2.816042in}{1.224173in}}%
\pgfpathlineto{\pgfqpoint{2.816884in}{1.237308in}}%
\pgfpathlineto{\pgfqpoint{2.817727in}{1.211038in}}%
\pgfpathlineto{\pgfqpoint{2.818148in}{1.237308in}}%
\pgfpathlineto{\pgfqpoint{2.818991in}{1.224173in}}%
\pgfpathlineto{\pgfqpoint{2.819412in}{1.237308in}}%
\pgfpathlineto{\pgfqpoint{2.819833in}{1.224173in}}%
\pgfpathlineto{\pgfqpoint{2.820255in}{1.197903in}}%
\pgfpathlineto{\pgfqpoint{2.820676in}{1.237308in}}%
\pgfpathlineto{\pgfqpoint{2.821518in}{1.211038in}}%
\pgfpathlineto{\pgfqpoint{2.821940in}{1.237308in}}%
\pgfpathlineto{\pgfqpoint{2.822782in}{1.224173in}}%
\pgfpathlineto{\pgfqpoint{2.823203in}{1.237308in}}%
\pgfpathlineto{\pgfqpoint{2.823625in}{1.224173in}}%
\pgfpathlineto{\pgfqpoint{2.824046in}{1.197903in}}%
\pgfpathlineto{\pgfqpoint{2.824889in}{1.211038in}}%
\pgfpathlineto{\pgfqpoint{2.825310in}{1.211038in}}%
\pgfpathlineto{\pgfqpoint{2.825731in}{1.237308in}}%
\pgfpathlineto{\pgfqpoint{2.826152in}{1.224173in}}%
\pgfpathlineto{\pgfqpoint{2.826574in}{1.211038in}}%
\pgfpathlineto{\pgfqpoint{2.826995in}{1.237308in}}%
\pgfpathlineto{\pgfqpoint{2.827416in}{1.211038in}}%
\pgfpathlineto{\pgfqpoint{2.827838in}{1.211038in}}%
\pgfpathlineto{\pgfqpoint{2.828259in}{1.250443in}}%
\pgfpathlineto{\pgfqpoint{2.828680in}{1.224173in}}%
\pgfpathlineto{\pgfqpoint{2.829523in}{1.224173in}}%
\pgfpathlineto{\pgfqpoint{2.830365in}{1.211038in}}%
\pgfpathlineto{\pgfqpoint{2.831208in}{1.237308in}}%
\pgfpathlineto{\pgfqpoint{2.831629in}{1.211038in}}%
\pgfpathlineto{\pgfqpoint{2.832472in}{1.224173in}}%
\pgfpathlineto{\pgfqpoint{2.832893in}{1.211038in}}%
\pgfpathlineto{\pgfqpoint{2.833314in}{1.224173in}}%
\pgfpathlineto{\pgfqpoint{2.833735in}{1.224173in}}%
\pgfpathlineto{\pgfqpoint{2.834157in}{1.211038in}}%
\pgfpathlineto{\pgfqpoint{2.834578in}{1.224173in}}%
\pgfpathlineto{\pgfqpoint{2.834999in}{1.224173in}}%
\pgfpathlineto{\pgfqpoint{2.835421in}{1.552551in}}%
\pgfpathlineto{\pgfqpoint{2.835842in}{1.250443in}}%
\pgfpathlineto{\pgfqpoint{2.836684in}{1.224173in}}%
\pgfpathlineto{\pgfqpoint{2.837106in}{1.237308in}}%
\pgfpathlineto{\pgfqpoint{2.837527in}{1.237308in}}%
\pgfpathlineto{\pgfqpoint{2.837948in}{1.224173in}}%
\pgfpathlineto{\pgfqpoint{2.838369in}{1.237308in}}%
\pgfpathlineto{\pgfqpoint{2.838791in}{1.250443in}}%
\pgfpathlineto{\pgfqpoint{2.840055in}{1.224173in}}%
\pgfpathlineto{\pgfqpoint{2.840476in}{1.224173in}}%
\pgfpathlineto{\pgfqpoint{2.840897in}{1.250443in}}%
\pgfpathlineto{\pgfqpoint{2.841318in}{1.237308in}}%
\pgfpathlineto{\pgfqpoint{2.841740in}{1.211038in}}%
\pgfpathlineto{\pgfqpoint{2.842582in}{1.224173in}}%
\pgfpathlineto{\pgfqpoint{2.843846in}{1.237308in}}%
\pgfpathlineto{\pgfqpoint{2.845110in}{1.224173in}}%
\pgfpathlineto{\pgfqpoint{2.846374in}{1.237308in}}%
\pgfpathlineto{\pgfqpoint{2.846795in}{1.224173in}}%
\pgfpathlineto{\pgfqpoint{2.847216in}{1.237308in}}%
\pgfpathlineto{\pgfqpoint{2.848480in}{1.237308in}}%
\pgfpathlineto{\pgfqpoint{2.848901in}{0.895796in}}%
\pgfpathlineto{\pgfqpoint{2.849323in}{1.224173in}}%
\pgfpathlineto{\pgfqpoint{2.849744in}{1.224173in}}%
\pgfpathlineto{\pgfqpoint{2.850165in}{1.211038in}}%
\pgfpathlineto{\pgfqpoint{2.850587in}{1.224173in}}%
\pgfpathlineto{\pgfqpoint{2.851008in}{1.224173in}}%
\pgfpathlineto{\pgfqpoint{2.851429in}{1.237308in}}%
\pgfpathlineto{\pgfqpoint{2.851850in}{1.224173in}}%
\pgfpathlineto{\pgfqpoint{2.852272in}{1.224173in}}%
\pgfpathlineto{\pgfqpoint{2.852693in}{1.211038in}}%
\pgfpathlineto{\pgfqpoint{2.853114in}{1.224173in}}%
\pgfpathlineto{\pgfqpoint{2.853957in}{1.237308in}}%
\pgfpathlineto{\pgfqpoint{2.855221in}{1.211038in}}%
\pgfpathlineto{\pgfqpoint{2.856063in}{1.237308in}}%
\pgfpathlineto{\pgfqpoint{2.856484in}{1.224173in}}%
\pgfpathlineto{\pgfqpoint{2.858170in}{1.224173in}}%
\pgfpathlineto{\pgfqpoint{2.858591in}{1.237308in}}%
\pgfpathlineto{\pgfqpoint{2.859012in}{1.197903in}}%
\pgfpathlineto{\pgfqpoint{2.859433in}{1.250443in}}%
\pgfpathlineto{\pgfqpoint{2.860276in}{1.224173in}}%
\pgfpathlineto{\pgfqpoint{2.860697in}{1.237308in}}%
\pgfpathlineto{\pgfqpoint{2.862804in}{1.197903in}}%
\pgfpathlineto{\pgfqpoint{2.863646in}{1.250443in}}%
\pgfpathlineto{\pgfqpoint{2.864067in}{1.224173in}}%
\pgfpathlineto{\pgfqpoint{2.864489in}{1.224173in}}%
\pgfpathlineto{\pgfqpoint{2.865753in}{1.211038in}}%
\pgfpathlineto{\pgfqpoint{2.866174in}{1.250443in}}%
\pgfpathlineto{\pgfqpoint{2.866595in}{1.224173in}}%
\pgfpathlineto{\pgfqpoint{2.867859in}{1.224173in}}%
\pgfpathlineto{\pgfqpoint{2.868702in}{1.237308in}}%
\pgfpathlineto{\pgfqpoint{2.869544in}{1.211038in}}%
\pgfpathlineto{\pgfqpoint{2.869965in}{1.224173in}}%
\pgfpathlineto{\pgfqpoint{2.872072in}{1.224173in}}%
\pgfpathlineto{\pgfqpoint{2.872493in}{1.237308in}}%
\pgfpathlineto{\pgfqpoint{2.872914in}{1.197903in}}%
\pgfpathlineto{\pgfqpoint{2.873336in}{1.224173in}}%
\pgfpathlineto{\pgfqpoint{2.873757in}{1.237308in}}%
\pgfpathlineto{\pgfqpoint{2.874178in}{1.211038in}}%
\pgfpathlineto{\pgfqpoint{2.875021in}{1.224173in}}%
\pgfpathlineto{\pgfqpoint{2.875863in}{1.224173in}}%
\pgfpathlineto{\pgfqpoint{2.876285in}{1.237308in}}%
\pgfpathlineto{\pgfqpoint{2.876706in}{1.224173in}}%
\pgfpathlineto{\pgfqpoint{2.877127in}{1.211038in}}%
\pgfpathlineto{\pgfqpoint{2.878391in}{1.237308in}}%
\pgfpathlineto{\pgfqpoint{2.879233in}{1.237308in}}%
\pgfpathlineto{\pgfqpoint{2.880919in}{1.211038in}}%
\pgfpathlineto{\pgfqpoint{2.882604in}{1.237308in}}%
\pgfpathlineto{\pgfqpoint{2.883868in}{1.224173in}}%
\pgfpathlineto{\pgfqpoint{2.888502in}{1.224173in}}%
\pgfpathlineto{\pgfqpoint{2.888923in}{1.250443in}}%
\pgfpathlineto{\pgfqpoint{2.889344in}{1.237308in}}%
\pgfpathlineto{\pgfqpoint{2.891029in}{1.211038in}}%
\pgfpathlineto{\pgfqpoint{2.891451in}{1.237308in}}%
\pgfpathlineto{\pgfqpoint{2.892293in}{1.224173in}}%
\pgfpathlineto{\pgfqpoint{2.893136in}{1.197903in}}%
\pgfpathlineto{\pgfqpoint{2.893978in}{1.263578in}}%
\pgfpathlineto{\pgfqpoint{2.894399in}{1.224173in}}%
\pgfpathlineto{\pgfqpoint{2.894821in}{1.224173in}}%
\pgfpathlineto{\pgfqpoint{2.896085in}{1.211038in}}%
\pgfpathlineto{\pgfqpoint{2.897348in}{1.224173in}}%
\pgfpathlineto{\pgfqpoint{2.900719in}{1.224173in}}%
\pgfpathlineto{\pgfqpoint{2.901140in}{1.237308in}}%
\pgfpathlineto{\pgfqpoint{2.901561in}{1.224173in}}%
\pgfpathlineto{\pgfqpoint{2.901982in}{1.224173in}}%
\pgfpathlineto{\pgfqpoint{2.902404in}{1.211038in}}%
\pgfpathlineto{\pgfqpoint{2.902825in}{1.237308in}}%
\pgfpathlineto{\pgfqpoint{2.903668in}{1.224173in}}%
\pgfpathlineto{\pgfqpoint{2.904931in}{1.224173in}}%
\pgfpathlineto{\pgfqpoint{2.905774in}{1.211038in}}%
\pgfpathlineto{\pgfqpoint{2.906617in}{1.250443in}}%
\pgfpathlineto{\pgfqpoint{2.907038in}{1.224173in}}%
\pgfpathlineto{\pgfqpoint{2.907459in}{1.184768in}}%
\pgfpathlineto{\pgfqpoint{2.907880in}{1.224173in}}%
\pgfpathlineto{\pgfqpoint{2.908302in}{1.224173in}}%
\pgfpathlineto{\pgfqpoint{2.909144in}{1.237308in}}%
\pgfpathlineto{\pgfqpoint{2.910408in}{1.224173in}}%
\pgfpathlineto{\pgfqpoint{2.912514in}{1.224173in}}%
\pgfpathlineto{\pgfqpoint{2.912936in}{1.237308in}}%
\pgfpathlineto{\pgfqpoint{2.913357in}{1.224173in}}%
\pgfpathlineto{\pgfqpoint{2.914200in}{1.224173in}}%
\pgfpathlineto{\pgfqpoint{2.914621in}{1.211038in}}%
\pgfpathlineto{\pgfqpoint{2.915042in}{1.224173in}}%
\pgfpathlineto{\pgfqpoint{2.915463in}{1.237308in}}%
\pgfpathlineto{\pgfqpoint{2.916306in}{1.211038in}}%
\pgfpathlineto{\pgfqpoint{2.916727in}{1.224173in}}%
\pgfpathlineto{\pgfqpoint{2.918412in}{1.224173in}}%
\pgfpathlineto{\pgfqpoint{2.919255in}{1.237308in}}%
\pgfpathlineto{\pgfqpoint{2.920097in}{1.211038in}}%
\pgfpathlineto{\pgfqpoint{2.920519in}{1.224173in}}%
\pgfpathlineto{\pgfqpoint{2.920940in}{1.224173in}}%
\pgfpathlineto{\pgfqpoint{2.921361in}{1.237308in}}%
\pgfpathlineto{\pgfqpoint{2.921783in}{1.224173in}}%
\pgfpathlineto{\pgfqpoint{2.923889in}{1.224173in}}%
\pgfpathlineto{\pgfqpoint{2.924310in}{1.237308in}}%
\pgfpathlineto{\pgfqpoint{2.924731in}{1.224173in}}%
\pgfpathlineto{\pgfqpoint{2.925574in}{1.224173in}}%
\pgfpathlineto{\pgfqpoint{2.926417in}{1.197903in}}%
\pgfpathlineto{\pgfqpoint{2.926838in}{1.211038in}}%
\pgfpathlineto{\pgfqpoint{2.928102in}{1.224173in}}%
\pgfpathlineto{\pgfqpoint{2.928523in}{1.224173in}}%
\pgfpathlineto{\pgfqpoint{2.928944in}{1.211038in}}%
\pgfpathlineto{\pgfqpoint{2.929366in}{1.237308in}}%
\pgfpathlineto{\pgfqpoint{2.930208in}{1.224173in}}%
\pgfpathlineto{\pgfqpoint{2.930629in}{1.237308in}}%
\pgfpathlineto{\pgfqpoint{2.931051in}{1.224173in}}%
\pgfpathlineto{\pgfqpoint{2.934000in}{1.224173in}}%
\pgfpathlineto{\pgfqpoint{2.934421in}{1.237308in}}%
\pgfpathlineto{\pgfqpoint{2.934842in}{1.184768in}}%
\pgfpathlineto{\pgfqpoint{2.935263in}{1.211038in}}%
\pgfpathlineto{\pgfqpoint{2.936527in}{1.224173in}}%
\pgfpathlineto{\pgfqpoint{2.937370in}{1.224173in}}%
\pgfpathlineto{\pgfqpoint{2.937791in}{1.211038in}}%
\pgfpathlineto{\pgfqpoint{2.938212in}{1.224173in}}%
\pgfpathlineto{\pgfqpoint{2.938634in}{1.224173in}}%
\pgfpathlineto{\pgfqpoint{2.939055in}{1.211038in}}%
\pgfpathlineto{\pgfqpoint{2.940319in}{1.237308in}}%
\pgfpathlineto{\pgfqpoint{2.940740in}{1.237308in}}%
\pgfpathlineto{\pgfqpoint{2.942004in}{1.211038in}}%
\pgfpathlineto{\pgfqpoint{2.943268in}{1.224173in}}%
\pgfpathlineto{\pgfqpoint{2.943689in}{1.224173in}}%
\pgfpathlineto{\pgfqpoint{2.944110in}{1.211038in}}%
\pgfpathlineto{\pgfqpoint{2.944532in}{1.237308in}}%
\pgfpathlineto{\pgfqpoint{2.945374in}{1.224173in}}%
\pgfpathlineto{\pgfqpoint{2.946638in}{1.224173in}}%
\pgfpathlineto{\pgfqpoint{2.947059in}{1.197903in}}%
\pgfpathlineto{\pgfqpoint{2.947481in}{1.224173in}}%
\pgfpathlineto{\pgfqpoint{2.948323in}{1.263578in}}%
\pgfpathlineto{\pgfqpoint{2.948744in}{1.237308in}}%
\pgfpathlineto{\pgfqpoint{2.950008in}{1.224173in}}%
\pgfpathlineto{\pgfqpoint{2.950851in}{1.237308in}}%
\pgfpathlineto{\pgfqpoint{2.951272in}{1.211038in}}%
\pgfpathlineto{\pgfqpoint{2.952115in}{1.224173in}}%
\pgfpathlineto{\pgfqpoint{2.952536in}{1.237308in}}%
\pgfpathlineto{\pgfqpoint{2.952957in}{1.224173in}}%
\pgfpathlineto{\pgfqpoint{2.953800in}{1.224173in}}%
\pgfpathlineto{\pgfqpoint{2.954221in}{1.211038in}}%
\pgfpathlineto{\pgfqpoint{2.954642in}{1.224173in}}%
\pgfpathlineto{\pgfqpoint{2.955906in}{1.237308in}}%
\pgfpathlineto{\pgfqpoint{2.956749in}{1.211038in}}%
\pgfpathlineto{\pgfqpoint{2.957170in}{1.224173in}}%
\pgfpathlineto{\pgfqpoint{2.957591in}{1.224173in}}%
\pgfpathlineto{\pgfqpoint{2.958012in}{1.211038in}}%
\pgfpathlineto{\pgfqpoint{2.958434in}{1.224173in}}%
\pgfpathlineto{\pgfqpoint{2.960119in}{1.224173in}}%
\pgfpathlineto{\pgfqpoint{2.960540in}{1.250443in}}%
\pgfpathlineto{\pgfqpoint{2.960961in}{1.224173in}}%
\pgfpathlineto{\pgfqpoint{2.961383in}{1.184768in}}%
\pgfpathlineto{\pgfqpoint{2.961804in}{1.211038in}}%
\pgfpathlineto{\pgfqpoint{2.963068in}{1.224173in}}%
\pgfpathlineto{\pgfqpoint{2.963489in}{1.224173in}}%
\pgfpathlineto{\pgfqpoint{2.964332in}{1.211038in}}%
\pgfpathlineto{\pgfqpoint{2.964753in}{1.237308in}}%
\pgfpathlineto{\pgfqpoint{2.965595in}{1.224173in}}%
\pgfpathlineto{\pgfqpoint{2.966017in}{1.237308in}}%
\pgfpathlineto{\pgfqpoint{2.967281in}{1.211038in}}%
\pgfpathlineto{\pgfqpoint{2.968123in}{1.224173in}}%
\pgfpathlineto{\pgfqpoint{2.969387in}{1.211038in}}%
\pgfpathlineto{\pgfqpoint{2.970230in}{1.237308in}}%
\pgfpathlineto{\pgfqpoint{2.970651in}{1.197903in}}%
\pgfpathlineto{\pgfqpoint{2.971072in}{1.237308in}}%
\pgfpathlineto{\pgfqpoint{2.972336in}{1.224173in}}%
\pgfpathlineto{\pgfqpoint{2.974442in}{1.224173in}}%
\pgfpathlineto{\pgfqpoint{2.974864in}{1.263578in}}%
\pgfpathlineto{\pgfqpoint{2.975285in}{1.224173in}}%
\pgfpathlineto{\pgfqpoint{2.976127in}{1.224173in}}%
\pgfpathlineto{\pgfqpoint{2.976549in}{1.132228in}}%
\pgfpathlineto{\pgfqpoint{2.976970in}{1.211038in}}%
\pgfpathlineto{\pgfqpoint{2.978234in}{1.224173in}}%
\pgfpathlineto{\pgfqpoint{2.979076in}{1.224173in}}%
\pgfpathlineto{\pgfqpoint{2.979498in}{1.211038in}}%
\pgfpathlineto{\pgfqpoint{2.980761in}{1.237308in}}%
\pgfpathlineto{\pgfqpoint{2.981604in}{1.211038in}}%
\pgfpathlineto{\pgfqpoint{2.982025in}{1.224173in}}%
\pgfpathlineto{\pgfqpoint{2.983710in}{1.224173in}}%
\pgfpathlineto{\pgfqpoint{2.984553in}{1.211038in}}%
\pgfpathlineto{\pgfqpoint{2.984974in}{1.250443in}}%
\pgfpathlineto{\pgfqpoint{2.985817in}{1.224173in}}%
\pgfpathlineto{\pgfqpoint{2.986238in}{1.237308in}}%
\pgfpathlineto{\pgfqpoint{2.986659in}{1.237308in}}%
\pgfpathlineto{\pgfqpoint{2.987502in}{1.211038in}}%
\pgfpathlineto{\pgfqpoint{2.987923in}{1.224173in}}%
\pgfpathlineto{\pgfqpoint{2.988344in}{1.211038in}}%
\pgfpathlineto{\pgfqpoint{2.990030in}{1.316119in}}%
\pgfpathlineto{\pgfqpoint{2.991293in}{1.224173in}}%
\pgfpathlineto{\pgfqpoint{2.992557in}{1.211038in}}%
\pgfpathlineto{\pgfqpoint{2.993821in}{1.224173in}}%
\pgfpathlineto{\pgfqpoint{2.994242in}{1.224173in}}%
\pgfpathlineto{\pgfqpoint{2.994664in}{1.211038in}}%
\pgfpathlineto{\pgfqpoint{2.995085in}{1.224173in}}%
\pgfpathlineto{\pgfqpoint{2.995506in}{1.224173in}}%
\pgfpathlineto{\pgfqpoint{2.995928in}{1.197903in}}%
\pgfpathlineto{\pgfqpoint{2.996770in}{1.211038in}}%
\pgfpathlineto{\pgfqpoint{2.998034in}{1.224173in}}%
\pgfpathlineto{\pgfqpoint{2.998455in}{1.197903in}}%
\pgfpathlineto{\pgfqpoint{2.998876in}{1.224173in}}%
\pgfpathlineto{\pgfqpoint{2.999298in}{1.224173in}}%
\pgfpathlineto{\pgfqpoint{2.999719in}{1.211038in}}%
\pgfpathlineto{\pgfqpoint{3.000140in}{1.224173in}}%
\pgfpathlineto{\pgfqpoint{3.000983in}{1.237308in}}%
\pgfpathlineto{\pgfqpoint{3.002247in}{1.211038in}}%
\pgfpathlineto{\pgfqpoint{3.003932in}{1.237308in}}%
\pgfpathlineto{\pgfqpoint{3.004774in}{1.224173in}}%
\pgfpathlineto{\pgfqpoint{3.005196in}{1.237308in}}%
\pgfpathlineto{\pgfqpoint{3.005617in}{1.224173in}}%
\pgfpathlineto{\pgfqpoint{3.007302in}{1.224173in}}%
\pgfpathlineto{\pgfqpoint{3.007723in}{1.250443in}}%
\pgfpathlineto{\pgfqpoint{3.008566in}{1.237308in}}%
\pgfpathlineto{\pgfqpoint{3.009830in}{1.224173in}}%
\pgfpathlineto{\pgfqpoint{3.010251in}{1.237308in}}%
\pgfpathlineto{\pgfqpoint{3.010672in}{1.224173in}}%
\pgfpathlineto{\pgfqpoint{3.011515in}{1.224173in}}%
\pgfpathlineto{\pgfqpoint{3.011936in}{1.211038in}}%
\pgfpathlineto{\pgfqpoint{3.012357in}{1.224173in}}%
\pgfpathlineto{\pgfqpoint{3.012779in}{1.250443in}}%
\pgfpathlineto{\pgfqpoint{3.013200in}{1.224173in}}%
\pgfpathlineto{\pgfqpoint{3.013621in}{1.211038in}}%
\pgfpathlineto{\pgfqpoint{3.014042in}{1.224173in}}%
\pgfpathlineto{\pgfqpoint{3.014885in}{1.224173in}}%
\pgfpathlineto{\pgfqpoint{3.015306in}{1.237308in}}%
\pgfpathlineto{\pgfqpoint{3.015728in}{1.224173in}}%
\pgfpathlineto{\pgfqpoint{3.016991in}{1.224173in}}%
\pgfpathlineto{\pgfqpoint{3.017413in}{1.211038in}}%
\pgfpathlineto{\pgfqpoint{3.017834in}{1.224173in}}%
\pgfpathlineto{\pgfqpoint{3.019098in}{1.224173in}}%
\pgfpathlineto{\pgfqpoint{3.020783in}{1.250443in}}%
\pgfpathlineto{\pgfqpoint{3.021204in}{1.197903in}}%
\pgfpathlineto{\pgfqpoint{3.022047in}{1.224173in}}%
\pgfpathlineto{\pgfqpoint{3.022889in}{1.237308in}}%
\pgfpathlineto{\pgfqpoint{3.024153in}{1.224173in}}%
\pgfpathlineto{\pgfqpoint{3.025417in}{1.224173in}}%
\pgfpathlineto{\pgfqpoint{3.026260in}{1.184768in}}%
\pgfpathlineto{\pgfqpoint{3.027523in}{1.224173in}}%
\pgfpathlineto{\pgfqpoint{3.028787in}{1.224173in}}%
\pgfpathlineto{\pgfqpoint{3.029208in}{1.250443in}}%
\pgfpathlineto{\pgfqpoint{3.029630in}{1.224173in}}%
\pgfpathlineto{\pgfqpoint{3.030051in}{1.224173in}}%
\pgfpathlineto{\pgfqpoint{3.030472in}{1.237308in}}%
\pgfpathlineto{\pgfqpoint{3.030894in}{1.224173in}}%
\pgfpathlineto{\pgfqpoint{3.031315in}{1.224173in}}%
\pgfpathlineto{\pgfqpoint{3.032157in}{1.211038in}}%
\pgfpathlineto{\pgfqpoint{3.033421in}{1.224173in}}%
\pgfpathlineto{\pgfqpoint{3.033843in}{1.211038in}}%
\pgfpathlineto{\pgfqpoint{3.034264in}{1.224173in}}%
\pgfpathlineto{\pgfqpoint{3.035528in}{1.224173in}}%
\pgfpathlineto{\pgfqpoint{3.035949in}{1.197903in}}%
\pgfpathlineto{\pgfqpoint{3.036370in}{1.211038in}}%
\pgfpathlineto{\pgfqpoint{3.036791in}{1.224173in}}%
\pgfpathlineto{\pgfqpoint{3.037213in}{1.197903in}}%
\pgfpathlineto{\pgfqpoint{3.037634in}{1.224173in}}%
\pgfpathlineto{\pgfqpoint{3.038898in}{1.237308in}}%
\pgfpathlineto{\pgfqpoint{3.039740in}{1.237308in}}%
\pgfpathlineto{\pgfqpoint{3.041004in}{1.224173in}}%
\pgfpathlineto{\pgfqpoint{3.041426in}{1.224173in}}%
\pgfpathlineto{\pgfqpoint{3.041847in}{1.237308in}}%
\pgfpathlineto{\pgfqpoint{3.042689in}{1.197903in}}%
\pgfpathlineto{\pgfqpoint{3.043111in}{1.224173in}}%
\pgfpathlineto{\pgfqpoint{3.043532in}{1.224173in}}%
\pgfpathlineto{\pgfqpoint{3.043953in}{1.211038in}}%
\pgfpathlineto{\pgfqpoint{3.044374in}{1.224173in}}%
\pgfpathlineto{\pgfqpoint{3.044796in}{1.224173in}}%
\pgfpathlineto{\pgfqpoint{3.045638in}{1.237308in}}%
\pgfpathlineto{\pgfqpoint{3.046481in}{1.211038in}}%
\pgfpathlineto{\pgfqpoint{3.046902in}{1.224173in}}%
\pgfpathlineto{\pgfqpoint{3.047745in}{1.197903in}}%
\pgfpathlineto{\pgfqpoint{3.048166in}{1.237308in}}%
\pgfpathlineto{\pgfqpoint{3.049009in}{1.224173in}}%
\pgfpathlineto{\pgfqpoint{3.049851in}{1.237308in}}%
\pgfpathlineto{\pgfqpoint{3.050272in}{1.224173in}}%
\pgfpathlineto{\pgfqpoint{3.050694in}{1.250443in}}%
\pgfpathlineto{\pgfqpoint{3.051115in}{1.224173in}}%
\pgfpathlineto{\pgfqpoint{3.051536in}{1.211038in}}%
\pgfpathlineto{\pgfqpoint{3.051957in}{1.237308in}}%
\pgfpathlineto{\pgfqpoint{3.052379in}{1.197903in}}%
\pgfpathlineto{\pgfqpoint{3.052800in}{1.224173in}}%
\pgfpathlineto{\pgfqpoint{3.054485in}{1.224173in}}%
\pgfpathlineto{\pgfqpoint{3.055328in}{1.211038in}}%
\pgfpathlineto{\pgfqpoint{3.055749in}{1.237308in}}%
\pgfpathlineto{\pgfqpoint{3.056592in}{1.224173in}}%
\pgfpathlineto{\pgfqpoint{3.057013in}{1.224173in}}%
\pgfpathlineto{\pgfqpoint{3.057434in}{1.211038in}}%
\pgfpathlineto{\pgfqpoint{3.057855in}{1.224173in}}%
\pgfpathlineto{\pgfqpoint{3.059540in}{1.224173in}}%
\pgfpathlineto{\pgfqpoint{3.059962in}{1.263578in}}%
\pgfpathlineto{\pgfqpoint{3.060383in}{1.224173in}}%
\pgfpathlineto{\pgfqpoint{3.060804in}{1.237308in}}%
\pgfpathlineto{\pgfqpoint{3.061647in}{1.211038in}}%
\pgfpathlineto{\pgfqpoint{3.062068in}{1.224173in}}%
\pgfpathlineto{\pgfqpoint{3.062489in}{1.211038in}}%
\pgfpathlineto{\pgfqpoint{3.062911in}{1.224173in}}%
\pgfpathlineto{\pgfqpoint{3.063332in}{1.224173in}}%
\pgfpathlineto{\pgfqpoint{3.063753in}{1.237308in}}%
\pgfpathlineto{\pgfqpoint{3.064175in}{1.224173in}}%
\pgfpathlineto{\pgfqpoint{3.064596in}{1.224173in}}%
\pgfpathlineto{\pgfqpoint{3.065438in}{1.197903in}}%
\pgfpathlineto{\pgfqpoint{3.065860in}{1.237308in}}%
\pgfpathlineto{\pgfqpoint{3.066702in}{1.224173in}}%
\pgfpathlineto{\pgfqpoint{3.067124in}{1.224173in}}%
\pgfpathlineto{\pgfqpoint{3.067545in}{1.211038in}}%
\pgfpathlineto{\pgfqpoint{3.067966in}{1.224173in}}%
\pgfpathlineto{\pgfqpoint{3.068387in}{1.250443in}}%
\pgfpathlineto{\pgfqpoint{3.068809in}{1.224173in}}%
\pgfpathlineto{\pgfqpoint{3.070494in}{1.224173in}}%
\pgfpathlineto{\pgfqpoint{3.071336in}{1.237308in}}%
\pgfpathlineto{\pgfqpoint{3.072600in}{1.211038in}}%
\pgfpathlineto{\pgfqpoint{3.073021in}{1.250443in}}%
\pgfpathlineto{\pgfqpoint{3.073443in}{1.197903in}}%
\pgfpathlineto{\pgfqpoint{3.073864in}{1.197903in}}%
\pgfpathlineto{\pgfqpoint{3.074707in}{1.237308in}}%
\pgfpathlineto{\pgfqpoint{3.075128in}{1.224173in}}%
\pgfpathlineto{\pgfqpoint{3.075549in}{1.224173in}}%
\pgfpathlineto{\pgfqpoint{3.075970in}{1.237308in}}%
\pgfpathlineto{\pgfqpoint{3.077655in}{1.197903in}}%
\pgfpathlineto{\pgfqpoint{3.078919in}{1.250443in}}%
\pgfpathlineto{\pgfqpoint{3.079341in}{1.237308in}}%
\pgfpathlineto{\pgfqpoint{3.080604in}{1.224173in}}%
\pgfpathlineto{\pgfqpoint{3.081026in}{1.250443in}}%
\pgfpathlineto{\pgfqpoint{3.081447in}{1.224173in}}%
\pgfpathlineto{\pgfqpoint{3.083132in}{1.224173in}}%
\pgfpathlineto{\pgfqpoint{3.083553in}{1.237308in}}%
\pgfpathlineto{\pgfqpoint{3.083975in}{1.224173in}}%
\pgfpathlineto{\pgfqpoint{3.084396in}{1.224173in}}%
\pgfpathlineto{\pgfqpoint{3.085660in}{1.211038in}}%
\pgfpathlineto{\pgfqpoint{3.086081in}{1.250443in}}%
\pgfpathlineto{\pgfqpoint{3.086502in}{1.224173in}}%
\pgfpathlineto{\pgfqpoint{3.086924in}{1.224173in}}%
\pgfpathlineto{\pgfqpoint{3.087345in}{1.250443in}}%
\pgfpathlineto{\pgfqpoint{3.087766in}{1.211038in}}%
\pgfpathlineto{\pgfqpoint{3.088187in}{1.237308in}}%
\pgfpathlineto{\pgfqpoint{3.089030in}{1.224173in}}%
\pgfpathlineto{\pgfqpoint{3.090294in}{1.237308in}}%
\pgfpathlineto{\pgfqpoint{3.090715in}{1.211038in}}%
\pgfpathlineto{\pgfqpoint{3.091136in}{1.250443in}}%
\pgfpathlineto{\pgfqpoint{3.091558in}{1.250443in}}%
\pgfpathlineto{\pgfqpoint{3.092821in}{1.211038in}}%
\pgfpathlineto{\pgfqpoint{3.094085in}{1.237308in}}%
\pgfpathlineto{\pgfqpoint{3.094507in}{1.184768in}}%
\pgfpathlineto{\pgfqpoint{3.095349in}{1.197903in}}%
\pgfpathlineto{\pgfqpoint{3.096613in}{1.224173in}}%
\pgfpathlineto{\pgfqpoint{3.097034in}{1.211038in}}%
\pgfpathlineto{\pgfqpoint{3.097456in}{1.237308in}}%
\pgfpathlineto{\pgfqpoint{3.097877in}{1.211038in}}%
\pgfpathlineto{\pgfqpoint{3.098298in}{1.211038in}}%
\pgfpathlineto{\pgfqpoint{3.098719in}{1.237308in}}%
\pgfpathlineto{\pgfqpoint{3.099141in}{1.224173in}}%
\pgfpathlineto{\pgfqpoint{3.099562in}{1.197903in}}%
\pgfpathlineto{\pgfqpoint{3.099983in}{1.211038in}}%
\pgfpathlineto{\pgfqpoint{3.101247in}{1.237308in}}%
\pgfpathlineto{\pgfqpoint{3.101668in}{1.197903in}}%
\pgfpathlineto{\pgfqpoint{3.102090in}{1.224173in}}%
\pgfpathlineto{\pgfqpoint{3.102511in}{1.224173in}}%
\pgfpathlineto{\pgfqpoint{3.102932in}{1.197903in}}%
\pgfpathlineto{\pgfqpoint{3.103353in}{1.224173in}}%
\pgfpathlineto{\pgfqpoint{3.103775in}{1.237308in}}%
\pgfpathlineto{\pgfqpoint{3.104196in}{1.224173in}}%
\pgfpathlineto{\pgfqpoint{3.104617in}{1.224173in}}%
\pgfpathlineto{\pgfqpoint{3.105039in}{1.184768in}}%
\pgfpathlineto{\pgfqpoint{3.105460in}{1.211038in}}%
\pgfpathlineto{\pgfqpoint{3.105881in}{1.211038in}}%
\pgfpathlineto{\pgfqpoint{3.106724in}{1.237308in}}%
\pgfpathlineto{\pgfqpoint{3.107145in}{1.224173in}}%
\pgfpathlineto{\pgfqpoint{3.107566in}{1.211038in}}%
\pgfpathlineto{\pgfqpoint{3.108830in}{1.237308in}}%
\pgfpathlineto{\pgfqpoint{3.109673in}{1.211038in}}%
\pgfpathlineto{\pgfqpoint{3.110094in}{1.224173in}}%
\pgfpathlineto{\pgfqpoint{3.110515in}{1.224173in}}%
\pgfpathlineto{\pgfqpoint{3.110936in}{1.211038in}}%
\pgfpathlineto{\pgfqpoint{3.111358in}{1.237308in}}%
\pgfpathlineto{\pgfqpoint{3.111779in}{1.224173in}}%
\pgfpathlineto{\pgfqpoint{3.112200in}{1.211038in}}%
\pgfpathlineto{\pgfqpoint{3.112622in}{1.237308in}}%
\pgfpathlineto{\pgfqpoint{3.113464in}{1.224173in}}%
\pgfpathlineto{\pgfqpoint{3.113885in}{1.224173in}}%
\pgfpathlineto{\pgfqpoint{3.114307in}{1.211038in}}%
\pgfpathlineto{\pgfqpoint{3.114728in}{1.224173in}}%
\pgfpathlineto{\pgfqpoint{3.115992in}{1.224173in}}%
\pgfpathlineto{\pgfqpoint{3.116413in}{1.237308in}}%
\pgfpathlineto{\pgfqpoint{3.116834in}{1.224173in}}%
\pgfpathlineto{\pgfqpoint{3.118098in}{1.211038in}}%
\pgfpathlineto{\pgfqpoint{3.119362in}{1.237308in}}%
\pgfpathlineto{\pgfqpoint{3.120626in}{1.224173in}}%
\pgfpathlineto{\pgfqpoint{3.121047in}{1.224173in}}%
\pgfpathlineto{\pgfqpoint{3.122311in}{1.237308in}}%
\pgfpathlineto{\pgfqpoint{3.123153in}{1.237308in}}%
\pgfpathlineto{\pgfqpoint{3.124417in}{1.224173in}}%
\pgfpathlineto{\pgfqpoint{3.126102in}{1.224173in}}%
\pgfpathlineto{\pgfqpoint{3.126524in}{1.237308in}}%
\pgfpathlineto{\pgfqpoint{3.126945in}{1.224173in}}%
\pgfpathlineto{\pgfqpoint{3.127788in}{1.224173in}}%
\pgfpathlineto{\pgfqpoint{3.128209in}{1.211038in}}%
\pgfpathlineto{\pgfqpoint{3.128630in}{1.237308in}}%
\pgfpathlineto{\pgfqpoint{3.129473in}{1.224173in}}%
\pgfpathlineto{\pgfqpoint{3.129894in}{1.224173in}}%
\pgfpathlineto{\pgfqpoint{3.130315in}{1.237308in}}%
\pgfpathlineto{\pgfqpoint{3.130737in}{1.224173in}}%
\pgfpathlineto{\pgfqpoint{3.131158in}{1.224173in}}%
\pgfpathlineto{\pgfqpoint{3.131579in}{1.237308in}}%
\pgfpathlineto{\pgfqpoint{3.132000in}{1.211038in}}%
\pgfpathlineto{\pgfqpoint{3.132843in}{1.224173in}}%
\pgfpathlineto{\pgfqpoint{3.133685in}{1.211038in}}%
\pgfpathlineto{\pgfqpoint{3.134949in}{1.224173in}}%
\pgfpathlineto{\pgfqpoint{3.135371in}{1.224173in}}%
\pgfpathlineto{\pgfqpoint{3.135792in}{1.197903in}}%
\pgfpathlineto{\pgfqpoint{3.136213in}{1.211038in}}%
\pgfpathlineto{\pgfqpoint{3.137056in}{1.224173in}}%
\pgfpathlineto{\pgfqpoint{3.138320in}{1.197903in}}%
\pgfpathlineto{\pgfqpoint{3.138741in}{1.250443in}}%
\pgfpathlineto{\pgfqpoint{3.139583in}{1.224173in}}%
\pgfpathlineto{\pgfqpoint{3.141268in}{1.224173in}}%
\pgfpathlineto{\pgfqpoint{3.141690in}{1.237308in}}%
\pgfpathlineto{\pgfqpoint{3.142111in}{1.224173in}}%
\pgfpathlineto{\pgfqpoint{3.142954in}{1.224173in}}%
\pgfpathlineto{\pgfqpoint{3.143796in}{1.211038in}}%
\pgfpathlineto{\pgfqpoint{3.145060in}{1.224173in}}%
\pgfpathlineto{\pgfqpoint{3.146324in}{1.224173in}}%
\pgfpathlineto{\pgfqpoint{3.146745in}{1.250443in}}%
\pgfpathlineto{\pgfqpoint{3.147166in}{1.224173in}}%
\pgfpathlineto{\pgfqpoint{3.148009in}{1.224173in}}%
\pgfpathlineto{\pgfqpoint{3.148430in}{1.211038in}}%
\pgfpathlineto{\pgfqpoint{3.148851in}{1.224173in}}%
\pgfpathlineto{\pgfqpoint{3.149694in}{1.237308in}}%
\pgfpathlineto{\pgfqpoint{3.150958in}{1.224173in}}%
\pgfpathlineto{\pgfqpoint{3.151379in}{1.224173in}}%
\pgfpathlineto{\pgfqpoint{3.151800in}{1.263578in}}%
\pgfpathlineto{\pgfqpoint{3.152222in}{1.211038in}}%
\pgfpathlineto{\pgfqpoint{3.153064in}{1.224173in}}%
\pgfpathlineto{\pgfqpoint{3.153486in}{1.197903in}}%
\pgfpathlineto{\pgfqpoint{3.153907in}{1.224173in}}%
\pgfpathlineto{\pgfqpoint{3.154328in}{1.224173in}}%
\pgfpathlineto{\pgfqpoint{3.155592in}{1.237308in}}%
\pgfpathlineto{\pgfqpoint{3.156434in}{1.224173in}}%
\pgfpathlineto{\pgfqpoint{3.156856in}{1.250443in}}%
\pgfpathlineto{\pgfqpoint{3.157277in}{1.224173in}}%
\pgfpathlineto{\pgfqpoint{3.158541in}{1.224173in}}%
\pgfpathlineto{\pgfqpoint{3.158962in}{1.237308in}}%
\pgfpathlineto{\pgfqpoint{3.159383in}{1.224173in}}%
\pgfpathlineto{\pgfqpoint{3.159805in}{1.224173in}}%
\pgfpathlineto{\pgfqpoint{3.160226in}{1.211038in}}%
\pgfpathlineto{\pgfqpoint{3.160647in}{1.224173in}}%
\pgfpathlineto{\pgfqpoint{3.161490in}{1.224173in}}%
\pgfpathlineto{\pgfqpoint{3.161911in}{1.237308in}}%
\pgfpathlineto{\pgfqpoint{3.162332in}{1.224173in}}%
\pgfpathlineto{\pgfqpoint{3.163175in}{1.224173in}}%
\pgfpathlineto{\pgfqpoint{3.164439in}{1.211038in}}%
\pgfpathlineto{\pgfqpoint{3.165703in}{1.237308in}}%
\pgfpathlineto{\pgfqpoint{3.166545in}{1.224173in}}%
\pgfpathlineto{\pgfqpoint{3.166966in}{1.250443in}}%
\pgfpathlineto{\pgfqpoint{3.167388in}{1.224173in}}%
\pgfpathlineto{\pgfqpoint{3.167809in}{1.224173in}}%
\pgfpathlineto{\pgfqpoint{3.168230in}{1.237308in}}%
\pgfpathlineto{\pgfqpoint{3.168652in}{1.197903in}}%
\pgfpathlineto{\pgfqpoint{3.169073in}{1.224173in}}%
\pgfpathlineto{\pgfqpoint{3.169915in}{1.224173in}}%
\pgfpathlineto{\pgfqpoint{3.170758in}{1.211038in}}%
\pgfpathlineto{\pgfqpoint{3.172022in}{1.237308in}}%
\pgfpathlineto{\pgfqpoint{3.173707in}{1.211038in}}%
\pgfpathlineto{\pgfqpoint{3.174971in}{1.224173in}}%
\pgfpathlineto{\pgfqpoint{3.175392in}{1.224173in}}%
\pgfpathlineto{\pgfqpoint{3.175813in}{1.237308in}}%
\pgfpathlineto{\pgfqpoint{3.176235in}{1.224173in}}%
\pgfpathlineto{\pgfqpoint{3.176656in}{1.224173in}}%
\pgfpathlineto{\pgfqpoint{3.177920in}{1.237308in}}%
\pgfpathlineto{\pgfqpoint{3.178341in}{1.237308in}}%
\pgfpathlineto{\pgfqpoint{3.179183in}{1.211038in}}%
\pgfpathlineto{\pgfqpoint{3.179605in}{1.224173in}}%
\pgfpathlineto{\pgfqpoint{3.181290in}{1.224173in}}%
\pgfpathlineto{\pgfqpoint{3.181711in}{1.197903in}}%
\pgfpathlineto{\pgfqpoint{3.182132in}{1.250443in}}%
\pgfpathlineto{\pgfqpoint{3.182975in}{1.224173in}}%
\pgfpathlineto{\pgfqpoint{3.183396in}{1.224173in}}%
\pgfpathlineto{\pgfqpoint{3.183818in}{1.211038in}}%
\pgfpathlineto{\pgfqpoint{3.184239in}{1.237308in}}%
\pgfpathlineto{\pgfqpoint{3.185081in}{1.224173in}}%
\pgfpathlineto{\pgfqpoint{3.185503in}{1.211038in}}%
\pgfpathlineto{\pgfqpoint{3.185924in}{1.224173in}}%
\pgfpathlineto{\pgfqpoint{3.186766in}{1.224173in}}%
\pgfpathlineto{\pgfqpoint{3.188030in}{1.237308in}}%
\pgfpathlineto{\pgfqpoint{3.188873in}{1.197903in}}%
\pgfpathlineto{\pgfqpoint{3.189294in}{1.211038in}}%
\pgfpathlineto{\pgfqpoint{3.190558in}{1.237308in}}%
\pgfpathlineto{\pgfqpoint{3.191822in}{1.211038in}}%
\pgfpathlineto{\pgfqpoint{3.192243in}{1.237308in}}%
\pgfpathlineto{\pgfqpoint{3.193086in}{1.224173in}}%
\pgfpathlineto{\pgfqpoint{3.193507in}{1.224173in}}%
\pgfpathlineto{\pgfqpoint{3.194771in}{1.211038in}}%
\pgfpathlineto{\pgfqpoint{3.196035in}{1.224173in}}%
\pgfpathlineto{\pgfqpoint{3.196877in}{1.224173in}}%
\pgfpathlineto{\pgfqpoint{3.198141in}{1.237308in}}%
\pgfpathlineto{\pgfqpoint{3.198984in}{1.211038in}}%
\pgfpathlineto{\pgfqpoint{3.199405in}{1.224173in}}%
\pgfpathlineto{\pgfqpoint{3.199826in}{1.237308in}}%
\pgfpathlineto{\pgfqpoint{3.200247in}{1.224173in}}%
\pgfpathlineto{\pgfqpoint{3.201933in}{1.224173in}}%
\pgfpathlineto{\pgfqpoint{3.202354in}{1.250443in}}%
\pgfpathlineto{\pgfqpoint{3.202775in}{1.224173in}}%
\pgfpathlineto{\pgfqpoint{3.203196in}{1.224173in}}%
\pgfpathlineto{\pgfqpoint{3.203618in}{1.237308in}}%
\pgfpathlineto{\pgfqpoint{3.204039in}{1.197903in}}%
\pgfpathlineto{\pgfqpoint{3.204881in}{1.211038in}}%
\pgfpathlineto{\pgfqpoint{3.206145in}{1.224173in}}%
\pgfpathlineto{\pgfqpoint{3.206988in}{1.224173in}}%
\pgfpathlineto{\pgfqpoint{3.208252in}{1.237308in}}%
\pgfpathlineto{\pgfqpoint{3.208673in}{1.237308in}}%
\pgfpathlineto{\pgfqpoint{3.209094in}{1.211038in}}%
\pgfpathlineto{\pgfqpoint{3.209937in}{1.224173in}}%
\pgfpathlineto{\pgfqpoint{3.211622in}{1.224173in}}%
\pgfpathlineto{\pgfqpoint{3.212043in}{1.197903in}}%
\pgfpathlineto{\pgfqpoint{3.212464in}{1.237308in}}%
\pgfpathlineto{\pgfqpoint{3.213307in}{1.197903in}}%
\pgfpathlineto{\pgfqpoint{3.214150in}{1.211038in}}%
\pgfpathlineto{\pgfqpoint{3.215413in}{1.224173in}}%
\pgfpathlineto{\pgfqpoint{3.215835in}{1.224173in}}%
\pgfpathlineto{\pgfqpoint{3.216256in}{1.237308in}}%
\pgfpathlineto{\pgfqpoint{3.216677in}{1.224173in}}%
\pgfpathlineto{\pgfqpoint{3.217099in}{1.197903in}}%
\pgfpathlineto{\pgfqpoint{3.217520in}{1.237308in}}%
\pgfpathlineto{\pgfqpoint{3.218362in}{1.237308in}}%
\pgfpathlineto{\pgfqpoint{3.219626in}{1.211038in}}%
\pgfpathlineto{\pgfqpoint{3.220890in}{1.224173in}}%
\pgfpathlineto{\pgfqpoint{3.221311in}{1.224173in}}%
\pgfpathlineto{\pgfqpoint{3.222154in}{1.197903in}}%
\pgfpathlineto{\pgfqpoint{3.222575in}{1.237308in}}%
\pgfpathlineto{\pgfqpoint{3.223418in}{1.224173in}}%
\pgfpathlineto{\pgfqpoint{3.223839in}{1.224173in}}%
\pgfpathlineto{\pgfqpoint{3.224260in}{1.197903in}}%
\pgfpathlineto{\pgfqpoint{3.224682in}{1.224173in}}%
\pgfpathlineto{\pgfqpoint{3.225524in}{1.224173in}}%
\pgfpathlineto{\pgfqpoint{3.225945in}{1.197903in}}%
\pgfpathlineto{\pgfqpoint{3.226367in}{1.224173in}}%
\pgfpathlineto{\pgfqpoint{3.226788in}{1.237308in}}%
\pgfpathlineto{\pgfqpoint{3.227209in}{1.224173in}}%
\pgfpathlineto{\pgfqpoint{3.228052in}{1.224173in}}%
\pgfpathlineto{\pgfqpoint{3.228473in}{1.237308in}}%
\pgfpathlineto{\pgfqpoint{3.228894in}{1.224173in}}%
\pgfpathlineto{\pgfqpoint{3.229316in}{1.197903in}}%
\pgfpathlineto{\pgfqpoint{3.230158in}{1.211038in}}%
\pgfpathlineto{\pgfqpoint{3.231422in}{1.237308in}}%
\pgfpathlineto{\pgfqpoint{3.231843in}{1.197903in}}%
\pgfpathlineto{\pgfqpoint{3.232265in}{1.237308in}}%
\pgfpathlineto{\pgfqpoint{3.233107in}{1.237308in}}%
\pgfpathlineto{\pgfqpoint{3.234371in}{1.197903in}}%
\pgfpathlineto{\pgfqpoint{3.235635in}{1.237308in}}%
\pgfpathlineto{\pgfqpoint{3.236899in}{1.224173in}}%
\pgfpathlineto{\pgfqpoint{3.237320in}{1.224173in}}%
\pgfpathlineto{\pgfqpoint{3.237741in}{1.250443in}}%
\pgfpathlineto{\pgfqpoint{3.238162in}{1.197903in}}%
\pgfpathlineto{\pgfqpoint{3.238584in}{1.224173in}}%
\pgfpathlineto{\pgfqpoint{3.239005in}{1.250443in}}%
\pgfpathlineto{\pgfqpoint{3.239426in}{1.224173in}}%
\pgfpathlineto{\pgfqpoint{3.239848in}{1.224173in}}%
\pgfpathlineto{\pgfqpoint{3.241111in}{1.237308in}}%
\pgfpathlineto{\pgfqpoint{3.242375in}{1.197903in}}%
\pgfpathlineto{\pgfqpoint{3.242796in}{1.250443in}}%
\pgfpathlineto{\pgfqpoint{3.243639in}{1.237308in}}%
\pgfpathlineto{\pgfqpoint{3.244060in}{1.237308in}}%
\pgfpathlineto{\pgfqpoint{3.244903in}{1.197903in}}%
\pgfpathlineto{\pgfqpoint{3.245324in}{1.250443in}}%
\pgfpathlineto{\pgfqpoint{3.245745in}{1.237308in}}%
\pgfpathlineto{\pgfqpoint{3.246167in}{1.211038in}}%
\pgfpathlineto{\pgfqpoint{3.246588in}{1.224173in}}%
\pgfpathlineto{\pgfqpoint{3.247852in}{1.263578in}}%
\pgfpathlineto{\pgfqpoint{3.249116in}{1.224173in}}%
\pgfpathlineto{\pgfqpoint{3.249537in}{1.211038in}}%
\pgfpathlineto{\pgfqpoint{3.249958in}{1.224173in}}%
\pgfpathlineto{\pgfqpoint{3.250379in}{1.224173in}}%
\pgfpathlineto{\pgfqpoint{3.250801in}{1.211038in}}%
\pgfpathlineto{\pgfqpoint{3.251222in}{1.224173in}}%
\pgfpathlineto{\pgfqpoint{3.251643in}{1.250443in}}%
\pgfpathlineto{\pgfqpoint{3.252065in}{1.224173in}}%
\pgfpathlineto{\pgfqpoint{3.252486in}{1.211038in}}%
\pgfpathlineto{\pgfqpoint{3.252907in}{1.250443in}}%
\pgfpathlineto{\pgfqpoint{3.253328in}{1.211038in}}%
\pgfpathlineto{\pgfqpoint{3.253750in}{1.211038in}}%
\pgfpathlineto{\pgfqpoint{3.254171in}{1.237308in}}%
\pgfpathlineto{\pgfqpoint{3.254592in}{1.197903in}}%
\pgfpathlineto{\pgfqpoint{3.255014in}{1.224173in}}%
\pgfpathlineto{\pgfqpoint{3.255435in}{1.224173in}}%
\pgfpathlineto{\pgfqpoint{3.255856in}{1.250443in}}%
\pgfpathlineto{\pgfqpoint{3.256277in}{1.211038in}}%
\pgfpathlineto{\pgfqpoint{3.257120in}{1.211038in}}%
\pgfpathlineto{\pgfqpoint{3.257541in}{1.197903in}}%
\pgfpathlineto{\pgfqpoint{3.258384in}{1.237308in}}%
\pgfpathlineto{\pgfqpoint{3.258805in}{1.224173in}}%
\pgfpathlineto{\pgfqpoint{3.259226in}{1.211038in}}%
\pgfpathlineto{\pgfqpoint{3.259648in}{1.224173in}}%
\pgfpathlineto{\pgfqpoint{3.260069in}{1.237308in}}%
\pgfpathlineto{\pgfqpoint{3.260490in}{1.224173in}}%
\pgfpathlineto{\pgfqpoint{3.260911in}{1.224173in}}%
\pgfpathlineto{\pgfqpoint{3.261333in}{1.197903in}}%
\pgfpathlineto{\pgfqpoint{3.261754in}{1.237308in}}%
\pgfpathlineto{\pgfqpoint{3.262175in}{1.211038in}}%
\pgfpathlineto{\pgfqpoint{3.262597in}{1.197903in}}%
\pgfpathlineto{\pgfqpoint{3.263018in}{1.237308in}}%
\pgfpathlineto{\pgfqpoint{3.263860in}{1.224173in}}%
\pgfpathlineto{\pgfqpoint{3.264282in}{1.237308in}}%
\pgfpathlineto{\pgfqpoint{3.265546in}{1.211038in}}%
\pgfpathlineto{\pgfqpoint{3.266809in}{1.237308in}}%
\pgfpathlineto{\pgfqpoint{3.267652in}{1.211038in}}%
\pgfpathlineto{\pgfqpoint{3.268073in}{1.250443in}}%
\pgfpathlineto{\pgfqpoint{3.268494in}{1.224173in}}%
\pgfpathlineto{\pgfqpoint{3.269337in}{1.224173in}}%
\pgfpathlineto{\pgfqpoint{3.270180in}{1.263578in}}%
\pgfpathlineto{\pgfqpoint{3.270601in}{1.237308in}}%
\pgfpathlineto{\pgfqpoint{3.271022in}{1.237308in}}%
\pgfpathlineto{\pgfqpoint{3.272707in}{1.197903in}}%
\pgfpathlineto{\pgfqpoint{3.273129in}{1.237308in}}%
\pgfpathlineto{\pgfqpoint{3.273971in}{1.224173in}}%
\pgfpathlineto{\pgfqpoint{3.274392in}{1.224173in}}%
\pgfpathlineto{\pgfqpoint{3.274814in}{1.237308in}}%
\pgfpathlineto{\pgfqpoint{3.275235in}{1.211038in}}%
\pgfpathlineto{\pgfqpoint{3.275656in}{1.237308in}}%
\pgfpathlineto{\pgfqpoint{3.276077in}{1.250443in}}%
\pgfpathlineto{\pgfqpoint{3.277341in}{1.224173in}}%
\pgfpathlineto{\pgfqpoint{3.278605in}{1.224173in}}%
\pgfpathlineto{\pgfqpoint{3.279026in}{1.237308in}}%
\pgfpathlineto{\pgfqpoint{3.279448in}{1.197903in}}%
\pgfpathlineto{\pgfqpoint{3.279869in}{1.211038in}}%
\pgfpathlineto{\pgfqpoint{3.280712in}{1.237308in}}%
\pgfpathlineto{\pgfqpoint{3.281133in}{1.224173in}}%
\pgfpathlineto{\pgfqpoint{3.281554in}{1.224173in}}%
\pgfpathlineto{\pgfqpoint{3.281975in}{1.211038in}}%
\pgfpathlineto{\pgfqpoint{3.282397in}{1.224173in}}%
\pgfpathlineto{\pgfqpoint{3.282818in}{1.224173in}}%
\pgfpathlineto{\pgfqpoint{3.283660in}{1.197903in}}%
\pgfpathlineto{\pgfqpoint{3.284503in}{1.224173in}}%
\pgfpathlineto{\pgfqpoint{3.284924in}{1.211038in}}%
\pgfpathlineto{\pgfqpoint{3.285346in}{1.211038in}}%
\pgfpathlineto{\pgfqpoint{3.286188in}{1.250443in}}%
\pgfpathlineto{\pgfqpoint{3.286609in}{1.224173in}}%
\pgfpathlineto{\pgfqpoint{3.287031in}{1.224173in}}%
\pgfpathlineto{\pgfqpoint{3.287452in}{1.237308in}}%
\pgfpathlineto{\pgfqpoint{3.288295in}{1.211038in}}%
\pgfpathlineto{\pgfqpoint{3.288716in}{1.224173in}}%
\pgfpathlineto{\pgfqpoint{3.289980in}{1.211038in}}%
\pgfpathlineto{\pgfqpoint{3.291665in}{1.237308in}}%
\pgfpathlineto{\pgfqpoint{3.292507in}{1.224173in}}%
\pgfpathlineto{\pgfqpoint{3.292929in}{1.250443in}}%
\pgfpathlineto{\pgfqpoint{3.293350in}{1.237308in}}%
\pgfpathlineto{\pgfqpoint{3.295035in}{1.211038in}}%
\pgfpathlineto{\pgfqpoint{3.295456in}{1.237308in}}%
\pgfpathlineto{\pgfqpoint{3.296299in}{1.224173in}}%
\pgfpathlineto{\pgfqpoint{3.296720in}{1.211038in}}%
\pgfpathlineto{\pgfqpoint{3.297141in}{1.224173in}}%
\pgfpathlineto{\pgfqpoint{3.297984in}{1.224173in}}%
\pgfpathlineto{\pgfqpoint{3.298405in}{1.237308in}}%
\pgfpathlineto{\pgfqpoint{3.298826in}{1.224173in}}%
\pgfpathlineto{\pgfqpoint{3.299248in}{1.224173in}}%
\pgfpathlineto{\pgfqpoint{3.299669in}{1.211038in}}%
\pgfpathlineto{\pgfqpoint{3.300090in}{1.224173in}}%
\pgfpathlineto{\pgfqpoint{3.300512in}{1.224173in}}%
\pgfpathlineto{\pgfqpoint{3.300933in}{1.211038in}}%
\pgfpathlineto{\pgfqpoint{3.302618in}{1.250443in}}%
\pgfpathlineto{\pgfqpoint{3.303882in}{1.224173in}}%
\pgfpathlineto{\pgfqpoint{3.304303in}{1.224173in}}%
\pgfpathlineto{\pgfqpoint{3.305567in}{1.211038in}}%
\pgfpathlineto{\pgfqpoint{3.306831in}{1.224173in}}%
\pgfpathlineto{\pgfqpoint{3.308095in}{1.224173in}}%
\pgfpathlineto{\pgfqpoint{3.308516in}{1.237308in}}%
\pgfpathlineto{\pgfqpoint{3.308937in}{1.224173in}}%
\pgfpathlineto{\pgfqpoint{3.310201in}{1.224173in}}%
\pgfpathlineto{\pgfqpoint{3.310622in}{1.237308in}}%
\pgfpathlineto{\pgfqpoint{3.311044in}{1.224173in}}%
\pgfpathlineto{\pgfqpoint{3.312729in}{1.224173in}}%
\pgfpathlineto{\pgfqpoint{3.313150in}{1.250443in}}%
\pgfpathlineto{\pgfqpoint{3.313571in}{1.211038in}}%
\pgfpathlineto{\pgfqpoint{3.314835in}{1.224173in}}%
\pgfpathlineto{\pgfqpoint{3.315256in}{1.197903in}}%
\pgfpathlineto{\pgfqpoint{3.316099in}{1.211038in}}%
\pgfpathlineto{\pgfqpoint{3.316520in}{1.224173in}}%
\pgfpathlineto{\pgfqpoint{3.316941in}{1.211038in}}%
\pgfpathlineto{\pgfqpoint{3.317363in}{1.211038in}}%
\pgfpathlineto{\pgfqpoint{3.319048in}{1.237308in}}%
\pgfpathlineto{\pgfqpoint{3.320312in}{1.224173in}}%
\pgfpathlineto{\pgfqpoint{3.320733in}{1.224173in}}%
\pgfpathlineto{\pgfqpoint{3.321154in}{1.237308in}}%
\pgfpathlineto{\pgfqpoint{3.321575in}{1.224173in}}%
\pgfpathlineto{\pgfqpoint{3.323261in}{1.224173in}}%
\pgfpathlineto{\pgfqpoint{3.323682in}{1.237308in}}%
\pgfpathlineto{\pgfqpoint{3.324103in}{1.211038in}}%
\pgfpathlineto{\pgfqpoint{3.324946in}{1.224173in}}%
\pgfpathlineto{\pgfqpoint{3.325367in}{1.224173in}}%
\pgfpathlineto{\pgfqpoint{3.325788in}{1.197903in}}%
\pgfpathlineto{\pgfqpoint{3.326631in}{1.211038in}}%
\pgfpathlineto{\pgfqpoint{3.327052in}{1.237308in}}%
\pgfpathlineto{\pgfqpoint{3.327895in}{1.224173in}}%
\pgfpathlineto{\pgfqpoint{3.328316in}{1.211038in}}%
\pgfpathlineto{\pgfqpoint{3.328737in}{1.250443in}}%
\pgfpathlineto{\pgfqpoint{3.329580in}{1.237308in}}%
\pgfpathlineto{\pgfqpoint{3.330422in}{1.211038in}}%
\pgfpathlineto{\pgfqpoint{3.330844in}{1.237308in}}%
\pgfpathlineto{\pgfqpoint{3.331686in}{1.224173in}}%
\pgfpathlineto{\pgfqpoint{3.332950in}{1.197903in}}%
\pgfpathlineto{\pgfqpoint{3.334214in}{1.224173in}}%
\pgfpathlineto{\pgfqpoint{3.335478in}{1.197903in}}%
\pgfpathlineto{\pgfqpoint{3.337163in}{1.224173in}}%
\pgfpathlineto{\pgfqpoint{3.337584in}{1.224173in}}%
\pgfpathlineto{\pgfqpoint{3.338848in}{1.211038in}}%
\pgfpathlineto{\pgfqpoint{3.339269in}{1.237308in}}%
\pgfpathlineto{\pgfqpoint{3.340112in}{1.224173in}}%
\pgfpathlineto{\pgfqpoint{3.340533in}{1.197903in}}%
\pgfpathlineto{\pgfqpoint{3.341376in}{1.211038in}}%
\pgfpathlineto{\pgfqpoint{3.342639in}{1.237308in}}%
\pgfpathlineto{\pgfqpoint{3.343061in}{1.211038in}}%
\pgfpathlineto{\pgfqpoint{3.343482in}{1.237308in}}%
\pgfpathlineto{\pgfqpoint{3.343903in}{1.250443in}}%
\pgfpathlineto{\pgfqpoint{3.344325in}{1.211038in}}%
\pgfpathlineto{\pgfqpoint{3.345167in}{1.224173in}}%
\pgfpathlineto{\pgfqpoint{3.345588in}{1.224173in}}%
\pgfpathlineto{\pgfqpoint{3.346010in}{1.197903in}}%
\pgfpathlineto{\pgfqpoint{3.346431in}{1.237308in}}%
\pgfpathlineto{\pgfqpoint{3.347695in}{1.211038in}}%
\pgfpathlineto{\pgfqpoint{3.348537in}{1.237308in}}%
\pgfpathlineto{\pgfqpoint{3.348959in}{1.224173in}}%
\pgfpathlineto{\pgfqpoint{3.350222in}{1.224173in}}%
\pgfpathlineto{\pgfqpoint{3.350644in}{1.197903in}}%
\pgfpathlineto{\pgfqpoint{3.351065in}{1.224173in}}%
\pgfpathlineto{\pgfqpoint{3.351486in}{1.237308in}}%
\pgfpathlineto{\pgfqpoint{3.351908in}{1.211038in}}%
\pgfpathlineto{\pgfqpoint{3.352750in}{1.224173in}}%
\pgfpathlineto{\pgfqpoint{3.353171in}{1.250443in}}%
\pgfpathlineto{\pgfqpoint{3.354014in}{1.237308in}}%
\pgfpathlineto{\pgfqpoint{3.354435in}{1.237308in}}%
\pgfpathlineto{\pgfqpoint{3.356120in}{1.211038in}}%
\pgfpathlineto{\pgfqpoint{3.356542in}{1.211038in}}%
\pgfpathlineto{\pgfqpoint{3.357805in}{1.224173in}}%
\pgfpathlineto{\pgfqpoint{3.358227in}{1.211038in}}%
\pgfpathlineto{\pgfqpoint{3.358648in}{1.224173in}}%
\pgfpathlineto{\pgfqpoint{3.359491in}{1.250443in}}%
\pgfpathlineto{\pgfqpoint{3.359912in}{1.237308in}}%
\pgfpathlineto{\pgfqpoint{3.360754in}{1.211038in}}%
\pgfpathlineto{\pgfqpoint{3.361176in}{1.237308in}}%
\pgfpathlineto{\pgfqpoint{3.362018in}{1.224173in}}%
\pgfpathlineto{\pgfqpoint{3.362439in}{1.237308in}}%
\pgfpathlineto{\pgfqpoint{3.362861in}{1.224173in}}%
\pgfpathlineto{\pgfqpoint{3.363282in}{1.224173in}}%
\pgfpathlineto{\pgfqpoint{3.363703in}{1.211038in}}%
\pgfpathlineto{\pgfqpoint{3.364125in}{1.250443in}}%
\pgfpathlineto{\pgfqpoint{3.364546in}{1.224173in}}%
\pgfpathlineto{\pgfqpoint{3.365388in}{1.224173in}}%
\pgfpathlineto{\pgfqpoint{3.366652in}{1.211038in}}%
\pgfpathlineto{\pgfqpoint{3.368337in}{1.237308in}}%
\pgfpathlineto{\pgfqpoint{3.370022in}{1.237308in}}%
\pgfpathlineto{\pgfqpoint{3.370865in}{1.224173in}}%
\pgfpathlineto{\pgfqpoint{3.371708in}{1.237308in}}%
\pgfpathlineto{\pgfqpoint{3.372971in}{1.211038in}}%
\pgfpathlineto{\pgfqpoint{3.373393in}{1.211038in}}%
\pgfpathlineto{\pgfqpoint{3.374235in}{1.250443in}}%
\pgfpathlineto{\pgfqpoint{3.374657in}{1.224173in}}%
\pgfpathlineto{\pgfqpoint{3.375078in}{1.237308in}}%
\pgfpathlineto{\pgfqpoint{3.375499in}{1.224173in}}%
\pgfpathlineto{\pgfqpoint{3.375920in}{1.224173in}}%
\pgfpathlineto{\pgfqpoint{3.376342in}{0.646229in}}%
\pgfpathlineto{\pgfqpoint{3.376763in}{1.237308in}}%
\pgfpathlineto{\pgfqpoint{3.377605in}{1.211038in}}%
\pgfpathlineto{\pgfqpoint{3.378027in}{1.224173in}}%
\pgfpathlineto{\pgfqpoint{3.379291in}{1.237308in}}%
\pgfpathlineto{\pgfqpoint{3.379712in}{1.237308in}}%
\pgfpathlineto{\pgfqpoint{3.380976in}{1.211038in}}%
\pgfpathlineto{\pgfqpoint{3.382240in}{1.224173in}}%
\pgfpathlineto{\pgfqpoint{3.382661in}{1.224173in}}%
\pgfpathlineto{\pgfqpoint{3.383082in}{1.237308in}}%
\pgfpathlineto{\pgfqpoint{3.383503in}{1.224173in}}%
\pgfpathlineto{\pgfqpoint{3.383925in}{1.224173in}}%
\pgfpathlineto{\pgfqpoint{3.384346in}{1.237308in}}%
\pgfpathlineto{\pgfqpoint{3.384767in}{1.224173in}}%
\pgfpathlineto{\pgfqpoint{3.385188in}{1.224173in}}%
\pgfpathlineto{\pgfqpoint{3.385610in}{1.211038in}}%
\pgfpathlineto{\pgfqpoint{3.386874in}{1.237308in}}%
\pgfpathlineto{\pgfqpoint{3.387716in}{1.224173in}}%
\pgfpathlineto{\pgfqpoint{3.388137in}{1.237308in}}%
\pgfpathlineto{\pgfqpoint{3.388559in}{1.224173in}}%
\pgfpathlineto{\pgfqpoint{3.388980in}{1.224173in}}%
\pgfpathlineto{\pgfqpoint{3.389401in}{1.237308in}}%
\pgfpathlineto{\pgfqpoint{3.389823in}{1.815253in}}%
\pgfpathlineto{\pgfqpoint{3.390244in}{1.237308in}}%
\pgfpathlineto{\pgfqpoint{3.391929in}{1.211038in}}%
\pgfpathlineto{\pgfqpoint{3.392772in}{1.237308in}}%
\pgfpathlineto{\pgfqpoint{3.393193in}{1.224173in}}%
\pgfpathlineto{\pgfqpoint{3.393614in}{1.211038in}}%
\pgfpathlineto{\pgfqpoint{3.394457in}{1.237308in}}%
\pgfpathlineto{\pgfqpoint{3.394878in}{1.224173in}}%
\pgfpathlineto{\pgfqpoint{3.395720in}{1.224173in}}%
\pgfpathlineto{\pgfqpoint{3.396563in}{1.197903in}}%
\pgfpathlineto{\pgfqpoint{3.397827in}{1.224173in}}%
\pgfpathlineto{\pgfqpoint{3.398669in}{1.224173in}}%
\pgfpathlineto{\pgfqpoint{3.399091in}{1.250443in}}%
\pgfpathlineto{\pgfqpoint{3.399933in}{1.237308in}}%
\pgfpathlineto{\pgfqpoint{3.401618in}{1.211038in}}%
\pgfpathlineto{\pgfqpoint{3.402882in}{1.224173in}}%
\pgfpathlineto{\pgfqpoint{3.403303in}{1.224173in}}%
\pgfpathlineto{\pgfqpoint{3.403725in}{1.211038in}}%
\pgfpathlineto{\pgfqpoint{3.404567in}{1.237308in}}%
\pgfpathlineto{\pgfqpoint{3.404989in}{1.224173in}}%
\pgfpathlineto{\pgfqpoint{3.405410in}{1.224173in}}%
\pgfpathlineto{\pgfqpoint{3.406674in}{1.211038in}}%
\pgfpathlineto{\pgfqpoint{3.407095in}{1.237308in}}%
\pgfpathlineto{\pgfqpoint{3.407938in}{1.224173in}}%
\pgfpathlineto{\pgfqpoint{3.409201in}{1.224173in}}%
\pgfpathlineto{\pgfqpoint{3.409623in}{1.250443in}}%
\pgfpathlineto{\pgfqpoint{3.410044in}{1.237308in}}%
\pgfpathlineto{\pgfqpoint{3.411308in}{1.211038in}}%
\pgfpathlineto{\pgfqpoint{3.412572in}{1.224173in}}%
\pgfpathlineto{\pgfqpoint{3.412993in}{1.224173in}}%
\pgfpathlineto{\pgfqpoint{3.413414in}{1.237308in}}%
\pgfpathlineto{\pgfqpoint{3.413835in}{1.224173in}}%
\pgfpathlineto{\pgfqpoint{3.414257in}{1.224173in}}%
\pgfpathlineto{\pgfqpoint{3.414678in}{1.237308in}}%
\pgfpathlineto{\pgfqpoint{3.415099in}{1.224173in}}%
\pgfpathlineto{\pgfqpoint{3.415942in}{1.224173in}}%
\pgfpathlineto{\pgfqpoint{3.416363in}{1.211038in}}%
\pgfpathlineto{\pgfqpoint{3.416784in}{1.224173in}}%
\pgfpathlineto{\pgfqpoint{3.419312in}{1.224173in}}%
\pgfpathlineto{\pgfqpoint{3.420155in}{1.237308in}}%
\pgfpathlineto{\pgfqpoint{3.421418in}{1.211038in}}%
\pgfpathlineto{\pgfqpoint{3.422682in}{1.224173in}}%
\pgfpathlineto{\pgfqpoint{3.424367in}{1.224173in}}%
\pgfpathlineto{\pgfqpoint{3.424789in}{1.237308in}}%
\pgfpathlineto{\pgfqpoint{3.425210in}{1.224173in}}%
\pgfpathlineto{\pgfqpoint{3.425631in}{1.211038in}}%
\pgfpathlineto{\pgfqpoint{3.426052in}{1.224173in}}%
\pgfpathlineto{\pgfqpoint{3.426474in}{1.224173in}}%
\pgfpathlineto{\pgfqpoint{3.426895in}{1.211038in}}%
\pgfpathlineto{\pgfqpoint{3.427316in}{1.224173in}}%
\pgfpathlineto{\pgfqpoint{3.429423in}{1.224173in}}%
\pgfpathlineto{\pgfqpoint{3.430265in}{1.237308in}}%
\pgfpathlineto{\pgfqpoint{3.431529in}{1.211038in}}%
\pgfpathlineto{\pgfqpoint{3.433214in}{1.237308in}}%
\pgfpathlineto{\pgfqpoint{3.434057in}{1.224173in}}%
\pgfpathlineto{\pgfqpoint{3.434899in}{1.237308in}}%
\pgfpathlineto{\pgfqpoint{3.436584in}{1.211038in}}%
\pgfpathlineto{\pgfqpoint{3.437427in}{1.237308in}}%
\pgfpathlineto{\pgfqpoint{3.437848in}{1.224173in}}%
\pgfpathlineto{\pgfqpoint{3.438270in}{1.197903in}}%
\pgfpathlineto{\pgfqpoint{3.438691in}{1.224173in}}%
\pgfpathlineto{\pgfqpoint{3.439112in}{1.237308in}}%
\pgfpathlineto{\pgfqpoint{3.439533in}{1.224173in}}%
\pgfpathlineto{\pgfqpoint{3.441218in}{1.224173in}}%
\pgfpathlineto{\pgfqpoint{3.441640in}{1.211038in}}%
\pgfpathlineto{\pgfqpoint{3.442061in}{1.224173in}}%
\pgfpathlineto{\pgfqpoint{3.443325in}{1.224173in}}%
\pgfpathlineto{\pgfqpoint{3.443746in}{1.211038in}}%
\pgfpathlineto{\pgfqpoint{3.444167in}{1.224173in}}%
\pgfpathlineto{\pgfqpoint{3.444589in}{1.224173in}}%
\pgfpathlineto{\pgfqpoint{3.445010in}{1.237308in}}%
\pgfpathlineto{\pgfqpoint{3.446695in}{1.197903in}}%
\pgfpathlineto{\pgfqpoint{3.447959in}{1.224173in}}%
\pgfpathlineto{\pgfqpoint{3.449644in}{1.224173in}}%
\pgfpathlineto{\pgfqpoint{3.450065in}{1.237308in}}%
\pgfpathlineto{\pgfqpoint{3.450487in}{1.211038in}}%
\pgfpathlineto{\pgfqpoint{3.451329in}{1.224173in}}%
\pgfpathlineto{\pgfqpoint{3.451750in}{1.237308in}}%
\pgfpathlineto{\pgfqpoint{3.452172in}{1.224173in}}%
\pgfpathlineto{\pgfqpoint{3.453857in}{1.224173in}}%
\pgfpathlineto{\pgfqpoint{3.454278in}{1.237308in}}%
\pgfpathlineto{\pgfqpoint{3.454699in}{1.224173in}}%
\pgfpathlineto{\pgfqpoint{3.455121in}{1.224173in}}%
\pgfpathlineto{\pgfqpoint{3.455963in}{1.211038in}}%
\pgfpathlineto{\pgfqpoint{3.457227in}{1.224173in}}%
\pgfpathlineto{\pgfqpoint{3.457648in}{1.224173in}}%
\pgfpathlineto{\pgfqpoint{3.458912in}{1.237308in}}%
\pgfpathlineto{\pgfqpoint{3.459755in}{1.224173in}}%
\pgfpathlineto{\pgfqpoint{3.460176in}{1.237308in}}%
\pgfpathlineto{\pgfqpoint{3.460597in}{1.211038in}}%
\pgfpathlineto{\pgfqpoint{3.461440in}{1.224173in}}%
\pgfpathlineto{\pgfqpoint{3.464810in}{1.224173in}}%
\pgfpathlineto{\pgfqpoint{3.465231in}{1.237308in}}%
\pgfpathlineto{\pgfqpoint{3.465653in}{1.211038in}}%
\pgfpathlineto{\pgfqpoint{3.466495in}{1.224173in}}%
\pgfpathlineto{\pgfqpoint{3.467759in}{1.211038in}}%
\pgfpathlineto{\pgfqpoint{3.469444in}{1.237308in}}%
\pgfpathlineto{\pgfqpoint{3.470708in}{1.224173in}}%
\pgfpathlineto{\pgfqpoint{3.471129in}{1.224173in}}%
\pgfpathlineto{\pgfqpoint{3.471972in}{1.211038in}}%
\pgfpathlineto{\pgfqpoint{3.473236in}{1.224173in}}%
\pgfpathlineto{\pgfqpoint{3.473657in}{1.224173in}}%
\pgfpathlineto{\pgfqpoint{3.474078in}{1.237308in}}%
\pgfpathlineto{\pgfqpoint{3.474499in}{1.224173in}}%
\pgfpathlineto{\pgfqpoint{3.475342in}{1.224173in}}%
\pgfpathlineto{\pgfqpoint{3.475763in}{1.211038in}}%
\pgfpathlineto{\pgfqpoint{3.476185in}{1.224173in}}%
\pgfpathlineto{\pgfqpoint{3.477027in}{1.224173in}}%
\pgfpathlineto{\pgfqpoint{3.478291in}{1.197903in}}%
\pgfpathlineto{\pgfqpoint{3.479134in}{1.237308in}}%
\pgfpathlineto{\pgfqpoint{3.479555in}{1.224173in}}%
\pgfpathlineto{\pgfqpoint{3.479976in}{1.224173in}}%
\pgfpathlineto{\pgfqpoint{3.480397in}{1.237308in}}%
\pgfpathlineto{\pgfqpoint{3.480819in}{1.224173in}}%
\pgfpathlineto{\pgfqpoint{3.481661in}{1.224173in}}%
\pgfpathlineto{\pgfqpoint{3.482082in}{1.211038in}}%
\pgfpathlineto{\pgfqpoint{3.482504in}{1.224173in}}%
\pgfpathlineto{\pgfqpoint{3.485031in}{1.224173in}}%
\pgfpathlineto{\pgfqpoint{3.485453in}{1.237308in}}%
\pgfpathlineto{\pgfqpoint{3.485874in}{1.224173in}}%
\pgfpathlineto{\pgfqpoint{3.486717in}{1.197903in}}%
\pgfpathlineto{\pgfqpoint{3.487980in}{1.224173in}}%
\pgfpathlineto{\pgfqpoint{3.488402in}{1.224173in}}%
\pgfpathlineto{\pgfqpoint{3.488823in}{1.197903in}}%
\pgfpathlineto{\pgfqpoint{3.489244in}{1.237308in}}%
\pgfpathlineto{\pgfqpoint{3.490508in}{1.224173in}}%
\pgfpathlineto{\pgfqpoint{3.491772in}{1.250443in}}%
\pgfpathlineto{\pgfqpoint{3.492193in}{1.211038in}}%
\pgfpathlineto{\pgfqpoint{3.493036in}{1.224173in}}%
\pgfpathlineto{\pgfqpoint{3.493878in}{1.224173in}}%
\pgfpathlineto{\pgfqpoint{3.494300in}{1.237308in}}%
\pgfpathlineto{\pgfqpoint{3.494721in}{1.224173in}}%
\pgfpathlineto{\pgfqpoint{3.495142in}{1.224173in}}%
\pgfpathlineto{\pgfqpoint{3.495563in}{1.237308in}}%
\pgfpathlineto{\pgfqpoint{3.495985in}{1.224173in}}%
\pgfpathlineto{\pgfqpoint{3.497248in}{1.211038in}}%
\pgfpathlineto{\pgfqpoint{3.497670in}{1.211038in}}%
\pgfpathlineto{\pgfqpoint{3.498934in}{1.224173in}}%
\pgfpathlineto{\pgfqpoint{3.499355in}{1.211038in}}%
\pgfpathlineto{\pgfqpoint{3.500197in}{1.250443in}}%
\pgfpathlineto{\pgfqpoint{3.500619in}{1.224173in}}%
\pgfpathlineto{\pgfqpoint{3.501883in}{1.224173in}}%
\pgfpathlineto{\pgfqpoint{3.502304in}{1.237308in}}%
\pgfpathlineto{\pgfqpoint{3.502725in}{1.224173in}}%
\pgfpathlineto{\pgfqpoint{3.506517in}{1.224173in}}%
\pgfpathlineto{\pgfqpoint{3.507780in}{1.197903in}}%
\pgfpathlineto{\pgfqpoint{3.509044in}{1.224173in}}%
\pgfpathlineto{\pgfqpoint{3.509466in}{1.224173in}}%
\pgfpathlineto{\pgfqpoint{3.510729in}{1.237308in}}%
\pgfpathlineto{\pgfqpoint{3.511151in}{1.237308in}}%
\pgfpathlineto{\pgfqpoint{3.512414in}{1.211038in}}%
\pgfpathlineto{\pgfqpoint{3.512836in}{1.237308in}}%
\pgfpathlineto{\pgfqpoint{3.513678in}{1.224173in}}%
\pgfpathlineto{\pgfqpoint{3.515363in}{1.224173in}}%
\pgfpathlineto{\pgfqpoint{3.515785in}{1.237308in}}%
\pgfpathlineto{\pgfqpoint{3.516206in}{1.224173in}}%
\pgfpathlineto{\pgfqpoint{3.516627in}{1.224173in}}%
\pgfpathlineto{\pgfqpoint{3.517891in}{1.211038in}}%
\pgfpathlineto{\pgfqpoint{3.519155in}{1.237308in}}%
\pgfpathlineto{\pgfqpoint{3.519997in}{1.224173in}}%
\pgfpathlineto{\pgfqpoint{3.521261in}{1.237308in}}%
\pgfpathlineto{\pgfqpoint{3.522525in}{1.211038in}}%
\pgfpathlineto{\pgfqpoint{3.523789in}{1.224173in}}%
\pgfpathlineto{\pgfqpoint{3.524210in}{1.224173in}}%
\pgfpathlineto{\pgfqpoint{3.524632in}{1.197903in}}%
\pgfpathlineto{\pgfqpoint{3.525053in}{1.224173in}}%
\pgfpathlineto{\pgfqpoint{3.525474in}{1.224173in}}%
\pgfpathlineto{\pgfqpoint{3.525895in}{1.237308in}}%
\pgfpathlineto{\pgfqpoint{3.526317in}{1.224173in}}%
\pgfpathlineto{\pgfqpoint{3.527581in}{1.224173in}}%
\pgfpathlineto{\pgfqpoint{3.528844in}{1.211038in}}%
\pgfpathlineto{\pgfqpoint{3.530529in}{1.237308in}}%
\pgfpathlineto{\pgfqpoint{3.531372in}{1.237308in}}%
\pgfpathlineto{\pgfqpoint{3.532636in}{1.211038in}}%
\pgfpathlineto{\pgfqpoint{3.533900in}{1.224173in}}%
\pgfpathlineto{\pgfqpoint{3.535585in}{1.224173in}}%
\pgfpathlineto{\pgfqpoint{3.536006in}{1.237308in}}%
\pgfpathlineto{\pgfqpoint{3.536427in}{1.211038in}}%
\pgfpathlineto{\pgfqpoint{3.537270in}{1.224173in}}%
\pgfpathlineto{\pgfqpoint{3.537691in}{1.211038in}}%
\pgfpathlineto{\pgfqpoint{3.538112in}{1.237308in}}%
\pgfpathlineto{\pgfqpoint{3.538955in}{1.224173in}}%
\pgfpathlineto{\pgfqpoint{3.541061in}{1.224173in}}%
\pgfpathlineto{\pgfqpoint{3.542325in}{1.237308in}}%
\pgfpathlineto{\pgfqpoint{3.542747in}{1.211038in}}%
\pgfpathlineto{\pgfqpoint{3.543589in}{1.224173in}}%
\pgfpathlineto{\pgfqpoint{3.545274in}{1.224173in}}%
\pgfpathlineto{\pgfqpoint{3.545695in}{1.211038in}}%
\pgfpathlineto{\pgfqpoint{3.546117in}{1.237308in}}%
\pgfpathlineto{\pgfqpoint{3.546959in}{1.224173in}}%
\pgfpathlineto{\pgfqpoint{3.547381in}{1.224173in}}%
\pgfpathlineto{\pgfqpoint{3.547802in}{1.211038in}}%
\pgfpathlineto{\pgfqpoint{3.548223in}{1.224173in}}%
\pgfpathlineto{\pgfqpoint{3.549066in}{1.224173in}}%
\pgfpathlineto{\pgfqpoint{3.549487in}{1.211038in}}%
\pgfpathlineto{\pgfqpoint{3.549908in}{1.237308in}}%
\pgfpathlineto{\pgfqpoint{3.550751in}{1.224173in}}%
\pgfpathlineto{\pgfqpoint{3.551593in}{1.237308in}}%
\pgfpathlineto{\pgfqpoint{3.552857in}{1.211038in}}%
\pgfpathlineto{\pgfqpoint{3.554121in}{1.224173in}}%
\pgfpathlineto{\pgfqpoint{3.554542in}{1.224173in}}%
\pgfpathlineto{\pgfqpoint{3.554964in}{1.250443in}}%
\pgfpathlineto{\pgfqpoint{3.555385in}{1.224173in}}%
\pgfpathlineto{\pgfqpoint{3.556227in}{1.224173in}}%
\pgfpathlineto{\pgfqpoint{3.557070in}{1.211038in}}%
\pgfpathlineto{\pgfqpoint{3.557491in}{1.224173in}}%
\pgfpathlineto{\pgfqpoint{3.557913in}{1.211038in}}%
\pgfpathlineto{\pgfqpoint{3.558334in}{1.211038in}}%
\pgfpathlineto{\pgfqpoint{3.560019in}{1.237308in}}%
\pgfpathlineto{\pgfqpoint{3.560861in}{1.197903in}}%
\pgfpathlineto{\pgfqpoint{3.561283in}{1.237308in}}%
\pgfpathlineto{\pgfqpoint{3.562125in}{1.224173in}}%
\pgfpathlineto{\pgfqpoint{3.563389in}{1.224173in}}%
\pgfpathlineto{\pgfqpoint{3.563810in}{1.211038in}}%
\pgfpathlineto{\pgfqpoint{3.564232in}{1.224173in}}%
\pgfpathlineto{\pgfqpoint{3.565074in}{1.224173in}}%
\pgfpathlineto{\pgfqpoint{3.565917in}{1.211038in}}%
\pgfpathlineto{\pgfqpoint{3.567181in}{1.237308in}}%
\pgfpathlineto{\pgfqpoint{3.568444in}{1.197903in}}%
\pgfpathlineto{\pgfqpoint{3.570130in}{1.237308in}}%
\pgfpathlineto{\pgfqpoint{3.570551in}{1.237308in}}%
\pgfpathlineto{\pgfqpoint{3.570972in}{1.224173in}}%
\pgfpathlineto{\pgfqpoint{3.571393in}{1.237308in}}%
\pgfpathlineto{\pgfqpoint{3.571815in}{1.237308in}}%
\pgfpathlineto{\pgfqpoint{3.573500in}{1.211038in}}%
\pgfpathlineto{\pgfqpoint{3.574342in}{1.250443in}}%
\pgfpathlineto{\pgfqpoint{3.574764in}{1.224173in}}%
\pgfpathlineto{\pgfqpoint{3.576449in}{1.224173in}}%
\pgfpathlineto{\pgfqpoint{3.576870in}{1.211038in}}%
\pgfpathlineto{\pgfqpoint{3.577291in}{1.224173in}}%
\pgfpathlineto{\pgfqpoint{3.577713in}{1.224173in}}%
\pgfpathlineto{\pgfqpoint{3.578134in}{1.211038in}}%
\pgfpathlineto{\pgfqpoint{3.578555in}{1.224173in}}%
\pgfpathlineto{\pgfqpoint{3.579398in}{1.237308in}}%
\pgfpathlineto{\pgfqpoint{3.581083in}{1.211038in}}%
\pgfpathlineto{\pgfqpoint{3.581504in}{1.237308in}}%
\pgfpathlineto{\pgfqpoint{3.582347in}{1.224173in}}%
\pgfpathlineto{\pgfqpoint{3.582768in}{1.224173in}}%
\pgfpathlineto{\pgfqpoint{3.584032in}{1.211038in}}%
\pgfpathlineto{\pgfqpoint{3.584874in}{1.224173in}}%
\pgfpathlineto{\pgfqpoint{3.585296in}{1.211038in}}%
\pgfpathlineto{\pgfqpoint{3.585717in}{1.250443in}}%
\pgfpathlineto{\pgfqpoint{3.586559in}{1.237308in}}%
\pgfpathlineto{\pgfqpoint{3.586981in}{1.237308in}}%
\pgfpathlineto{\pgfqpoint{3.588245in}{1.211038in}}%
\pgfpathlineto{\pgfqpoint{3.589930in}{1.237308in}}%
\pgfpathlineto{\pgfqpoint{3.590772in}{1.237308in}}%
\pgfpathlineto{\pgfqpoint{3.592036in}{1.224173in}}%
\pgfpathlineto{\pgfqpoint{3.592879in}{1.224173in}}%
\pgfpathlineto{\pgfqpoint{3.593300in}{1.211038in}}%
\pgfpathlineto{\pgfqpoint{3.593721in}{1.224173in}}%
\pgfpathlineto{\pgfqpoint{3.594142in}{1.224173in}}%
\pgfpathlineto{\pgfqpoint{3.594564in}{1.237308in}}%
\pgfpathlineto{\pgfqpoint{3.594985in}{1.224173in}}%
\pgfpathlineto{\pgfqpoint{3.596249in}{1.224173in}}%
\pgfpathlineto{\pgfqpoint{3.596670in}{1.237308in}}%
\pgfpathlineto{\pgfqpoint{3.597091in}{1.211038in}}%
\pgfpathlineto{\pgfqpoint{3.597934in}{1.224173in}}%
\pgfpathlineto{\pgfqpoint{3.598355in}{1.211038in}}%
\pgfpathlineto{\pgfqpoint{3.598777in}{1.237308in}}%
\pgfpathlineto{\pgfqpoint{3.599198in}{1.197903in}}%
\pgfpathlineto{\pgfqpoint{3.599619in}{1.224173in}}%
\pgfpathlineto{\pgfqpoint{3.600040in}{1.224173in}}%
\pgfpathlineto{\pgfqpoint{3.600883in}{1.237308in}}%
\pgfpathlineto{\pgfqpoint{3.602147in}{1.224173in}}%
\pgfpathlineto{\pgfqpoint{3.602989in}{1.224173in}}%
\pgfpathlineto{\pgfqpoint{3.604253in}{1.211038in}}%
\pgfpathlineto{\pgfqpoint{3.605517in}{1.224173in}}%
\pgfpathlineto{\pgfqpoint{3.606781in}{1.224173in}}%
\pgfpathlineto{\pgfqpoint{3.608045in}{1.211038in}}%
\pgfpathlineto{\pgfqpoint{3.608466in}{1.211038in}}%
\pgfpathlineto{\pgfqpoint{3.610151in}{1.237308in}}%
\pgfpathlineto{\pgfqpoint{3.610572in}{1.237308in}}%
\pgfpathlineto{\pgfqpoint{3.610994in}{1.211038in}}%
\pgfpathlineto{\pgfqpoint{3.611415in}{1.224173in}}%
\pgfpathlineto{\pgfqpoint{3.611836in}{1.237308in}}%
\pgfpathlineto{\pgfqpoint{3.612257in}{1.224173in}}%
\pgfpathlineto{\pgfqpoint{3.613100in}{1.224173in}}%
\pgfpathlineto{\pgfqpoint{3.614364in}{1.211038in}}%
\pgfpathlineto{\pgfqpoint{3.615628in}{1.224173in}}%
\pgfpathlineto{\pgfqpoint{3.616470in}{1.224173in}}%
\pgfpathlineto{\pgfqpoint{3.616891in}{1.237308in}}%
\pgfpathlineto{\pgfqpoint{3.617313in}{1.211038in}}%
\pgfpathlineto{\pgfqpoint{3.618155in}{1.224173in}}%
\pgfpathlineto{\pgfqpoint{3.618577in}{1.211038in}}%
\pgfpathlineto{\pgfqpoint{3.618998in}{1.224173in}}%
\pgfpathlineto{\pgfqpoint{3.619840in}{1.224173in}}%
\pgfpathlineto{\pgfqpoint{3.621104in}{1.237308in}}%
\pgfpathlineto{\pgfqpoint{3.621947in}{1.237308in}}%
\pgfpathlineto{\pgfqpoint{3.623632in}{1.197903in}}%
\pgfpathlineto{\pgfqpoint{3.624474in}{1.237308in}}%
\pgfpathlineto{\pgfqpoint{3.624896in}{1.224173in}}%
\pgfpathlineto{\pgfqpoint{3.625317in}{1.224173in}}%
\pgfpathlineto{\pgfqpoint{3.625738in}{1.211038in}}%
\pgfpathlineto{\pgfqpoint{3.626160in}{1.224173in}}%
\pgfpathlineto{\pgfqpoint{3.627845in}{1.224173in}}%
\pgfpathlineto{\pgfqpoint{3.628266in}{1.237308in}}%
\pgfpathlineto{\pgfqpoint{3.628687in}{1.211038in}}%
\pgfpathlineto{\pgfqpoint{3.629530in}{1.224173in}}%
\pgfpathlineto{\pgfqpoint{3.630794in}{1.224173in}}%
\pgfpathlineto{\pgfqpoint{3.632057in}{1.237308in}}%
\pgfpathlineto{\pgfqpoint{3.633743in}{1.211038in}}%
\pgfpathlineto{\pgfqpoint{3.634585in}{1.237308in}}%
\pgfpathlineto{\pgfqpoint{3.635006in}{1.224173in}}%
\pgfpathlineto{\pgfqpoint{3.635428in}{1.224173in}}%
\pgfpathlineto{\pgfqpoint{3.635849in}{1.211038in}}%
\pgfpathlineto{\pgfqpoint{3.636270in}{1.224173in}}%
\pgfpathlineto{\pgfqpoint{3.637113in}{1.237308in}}%
\pgfpathlineto{\pgfqpoint{3.638377in}{1.197903in}}%
\pgfpathlineto{\pgfqpoint{3.638798in}{1.211038in}}%
\pgfpathlineto{\pgfqpoint{3.639219in}{1.237308in}}%
\pgfpathlineto{\pgfqpoint{3.640062in}{1.224173in}}%
\pgfpathlineto{\pgfqpoint{3.640483in}{1.237308in}}%
\pgfpathlineto{\pgfqpoint{3.640904in}{1.224173in}}%
\pgfpathlineto{\pgfqpoint{3.641747in}{1.211038in}}%
\pgfpathlineto{\pgfqpoint{3.642168in}{1.237308in}}%
\pgfpathlineto{\pgfqpoint{3.643011in}{1.224173in}}%
\pgfpathlineto{\pgfqpoint{3.643432in}{1.211038in}}%
\pgfpathlineto{\pgfqpoint{3.643853in}{1.224173in}}%
\pgfpathlineto{\pgfqpoint{3.644275in}{1.237308in}}%
\pgfpathlineto{\pgfqpoint{3.644696in}{1.211038in}}%
\pgfpathlineto{\pgfqpoint{3.645538in}{1.224173in}}%
\pgfpathlineto{\pgfqpoint{3.646802in}{1.224173in}}%
\pgfpathlineto{\pgfqpoint{3.647223in}{1.250443in}}%
\pgfpathlineto{\pgfqpoint{3.647645in}{1.224173in}}%
\pgfpathlineto{\pgfqpoint{3.648909in}{1.211038in}}%
\pgfpathlineto{\pgfqpoint{3.649330in}{1.237308in}}%
\pgfpathlineto{\pgfqpoint{3.649751in}{1.224173in}}%
\pgfpathlineto{\pgfqpoint{3.650172in}{1.211038in}}%
\pgfpathlineto{\pgfqpoint{3.650594in}{1.224173in}}%
\pgfpathlineto{\pgfqpoint{3.651436in}{1.224173in}}%
\pgfpathlineto{\pgfqpoint{3.652279in}{1.276714in}}%
\pgfpathlineto{\pgfqpoint{3.653543in}{1.224173in}}%
\pgfpathlineto{\pgfqpoint{3.654385in}{1.224173in}}%
\pgfpathlineto{\pgfqpoint{3.654807in}{1.237308in}}%
\pgfpathlineto{\pgfqpoint{3.655228in}{1.224173in}}%
\pgfpathlineto{\pgfqpoint{3.656492in}{1.224173in}}%
\pgfpathlineto{\pgfqpoint{3.656913in}{1.237308in}}%
\pgfpathlineto{\pgfqpoint{3.657334in}{1.224173in}}%
\pgfpathlineto{\pgfqpoint{3.660704in}{1.224173in}}%
\pgfpathlineto{\pgfqpoint{3.661126in}{1.211038in}}%
\pgfpathlineto{\pgfqpoint{3.661547in}{1.224173in}}%
\pgfpathlineto{\pgfqpoint{3.662390in}{1.224173in}}%
\pgfpathlineto{\pgfqpoint{3.662811in}{1.237308in}}%
\pgfpathlineto{\pgfqpoint{3.663232in}{1.224173in}}%
\pgfpathlineto{\pgfqpoint{3.664075in}{1.224173in}}%
\pgfpathlineto{\pgfqpoint{3.664496in}{1.237308in}}%
\pgfpathlineto{\pgfqpoint{3.664917in}{1.224173in}}%
\pgfpathlineto{\pgfqpoint{3.665338in}{1.224173in}}%
\pgfpathlineto{\pgfqpoint{3.665760in}{1.171633in}}%
\pgfpathlineto{\pgfqpoint{3.666181in}{1.211038in}}%
\pgfpathlineto{\pgfqpoint{3.667445in}{1.224173in}}%
\pgfpathlineto{\pgfqpoint{3.668709in}{1.224173in}}%
\pgfpathlineto{\pgfqpoint{3.669551in}{1.211038in}}%
\pgfpathlineto{\pgfqpoint{3.670394in}{1.250443in}}%
\pgfpathlineto{\pgfqpoint{3.670815in}{1.224173in}}%
\pgfpathlineto{\pgfqpoint{3.671236in}{1.224173in}}%
\pgfpathlineto{\pgfqpoint{3.671658in}{1.237308in}}%
\pgfpathlineto{\pgfqpoint{3.672079in}{1.224173in}}%
\pgfpathlineto{\pgfqpoint{3.673764in}{1.224173in}}%
\pgfpathlineto{\pgfqpoint{3.674185in}{1.211038in}}%
\pgfpathlineto{\pgfqpoint{3.674607in}{1.250443in}}%
\pgfpathlineto{\pgfqpoint{3.675028in}{1.224173in}}%
\pgfpathlineto{\pgfqpoint{3.675449in}{1.224173in}}%
\pgfpathlineto{\pgfqpoint{3.675870in}{1.237308in}}%
\pgfpathlineto{\pgfqpoint{3.676292in}{1.197903in}}%
\pgfpathlineto{\pgfqpoint{3.676713in}{1.224173in}}%
\pgfpathlineto{\pgfqpoint{3.677556in}{1.224173in}}%
\pgfpathlineto{\pgfqpoint{3.678398in}{1.197903in}}%
\pgfpathlineto{\pgfqpoint{3.678819in}{1.250443in}}%
\pgfpathlineto{\pgfqpoint{3.679662in}{1.237308in}}%
\pgfpathlineto{\pgfqpoint{3.680926in}{1.224173in}}%
\pgfpathlineto{\pgfqpoint{3.681347in}{1.224173in}}%
\pgfpathlineto{\pgfqpoint{3.682611in}{1.237308in}}%
\pgfpathlineto{\pgfqpoint{3.683032in}{1.237308in}}%
\pgfpathlineto{\pgfqpoint{3.684296in}{1.211038in}}%
\pgfpathlineto{\pgfqpoint{3.685560in}{1.237308in}}%
\pgfpathlineto{\pgfqpoint{3.686402in}{1.211038in}}%
\pgfpathlineto{\pgfqpoint{3.687245in}{1.237308in}}%
\pgfpathlineto{\pgfqpoint{3.687666in}{1.224173in}}%
\pgfpathlineto{\pgfqpoint{3.688087in}{1.211038in}}%
\pgfpathlineto{\pgfqpoint{3.688509in}{1.224173in}}%
\pgfpathlineto{\pgfqpoint{3.688930in}{1.224173in}}%
\pgfpathlineto{\pgfqpoint{3.689351in}{1.211038in}}%
\pgfpathlineto{\pgfqpoint{3.690615in}{1.237308in}}%
\pgfpathlineto{\pgfqpoint{3.691036in}{1.224173in}}%
\pgfpathlineto{\pgfqpoint{3.691458in}{1.237308in}}%
\pgfpathlineto{\pgfqpoint{3.691879in}{1.250443in}}%
\pgfpathlineto{\pgfqpoint{3.692300in}{1.197903in}}%
\pgfpathlineto{\pgfqpoint{3.693143in}{1.224173in}}%
\pgfpathlineto{\pgfqpoint{3.694828in}{1.224173in}}%
\pgfpathlineto{\pgfqpoint{3.695249in}{1.197903in}}%
\pgfpathlineto{\pgfqpoint{3.695670in}{1.224173in}}%
\pgfpathlineto{\pgfqpoint{3.696092in}{1.237308in}}%
\pgfpathlineto{\pgfqpoint{3.696513in}{1.224173in}}%
\pgfpathlineto{\pgfqpoint{3.696934in}{1.224173in}}%
\pgfpathlineto{\pgfqpoint{3.697356in}{1.211038in}}%
\pgfpathlineto{\pgfqpoint{3.697777in}{1.237308in}}%
\pgfpathlineto{\pgfqpoint{3.698619in}{1.224173in}}%
\pgfpathlineto{\pgfqpoint{3.699462in}{1.211038in}}%
\pgfpathlineto{\pgfqpoint{3.699883in}{1.250443in}}%
\pgfpathlineto{\pgfqpoint{3.700305in}{1.211038in}}%
\pgfpathlineto{\pgfqpoint{3.700726in}{1.211038in}}%
\pgfpathlineto{\pgfqpoint{3.701147in}{1.237308in}}%
\pgfpathlineto{\pgfqpoint{3.701568in}{1.224173in}}%
\pgfpathlineto{\pgfqpoint{3.701990in}{1.211038in}}%
\pgfpathlineto{\pgfqpoint{3.702411in}{1.224173in}}%
\pgfpathlineto{\pgfqpoint{3.702832in}{1.237308in}}%
\pgfpathlineto{\pgfqpoint{3.703253in}{1.224173in}}%
\pgfpathlineto{\pgfqpoint{3.703675in}{1.224173in}}%
\pgfpathlineto{\pgfqpoint{3.704939in}{1.211038in}}%
\pgfpathlineto{\pgfqpoint{3.706202in}{1.250443in}}%
\pgfpathlineto{\pgfqpoint{3.707466in}{1.211038in}}%
\pgfpathlineto{\pgfqpoint{3.708730in}{1.237308in}}%
\pgfpathlineto{\pgfqpoint{3.709573in}{1.197903in}}%
\pgfpathlineto{\pgfqpoint{3.709994in}{1.224173in}}%
\pgfpathlineto{\pgfqpoint{3.710836in}{1.224173in}}%
\pgfpathlineto{\pgfqpoint{3.711679in}{1.211038in}}%
\pgfpathlineto{\pgfqpoint{3.712943in}{1.237308in}}%
\pgfpathlineto{\pgfqpoint{3.713364in}{1.197903in}}%
\pgfpathlineto{\pgfqpoint{3.713785in}{1.211038in}}%
\pgfpathlineto{\pgfqpoint{3.714207in}{1.224173in}}%
\pgfpathlineto{\pgfqpoint{3.714628in}{1.197903in}}%
\pgfpathlineto{\pgfqpoint{3.715049in}{1.224173in}}%
\pgfpathlineto{\pgfqpoint{3.715471in}{1.237308in}}%
\pgfpathlineto{\pgfqpoint{3.715892in}{1.224173in}}%
\pgfpathlineto{\pgfqpoint{3.717998in}{1.224173in}}%
\pgfpathlineto{\pgfqpoint{3.718419in}{1.197903in}}%
\pgfpathlineto{\pgfqpoint{3.718841in}{1.224173in}}%
\pgfpathlineto{\pgfqpoint{3.719262in}{1.237308in}}%
\pgfpathlineto{\pgfqpoint{3.719683in}{1.184768in}}%
\pgfpathlineto{\pgfqpoint{3.720105in}{1.224173in}}%
\pgfpathlineto{\pgfqpoint{3.720526in}{1.224173in}}%
\pgfpathlineto{\pgfqpoint{3.720947in}{1.237308in}}%
\pgfpathlineto{\pgfqpoint{3.721368in}{1.224173in}}%
\pgfpathlineto{\pgfqpoint{3.721790in}{1.211038in}}%
\pgfpathlineto{\pgfqpoint{3.722211in}{1.224173in}}%
\pgfpathlineto{\pgfqpoint{3.722632in}{1.224173in}}%
\pgfpathlineto{\pgfqpoint{3.723054in}{1.237308in}}%
\pgfpathlineto{\pgfqpoint{3.723475in}{1.224173in}}%
\pgfpathlineto{\pgfqpoint{3.725160in}{1.224173in}}%
\pgfpathlineto{\pgfqpoint{3.725581in}{1.211038in}}%
\pgfpathlineto{\pgfqpoint{3.726003in}{1.224173in}}%
\pgfpathlineto{\pgfqpoint{3.726424in}{1.224173in}}%
\pgfpathlineto{\pgfqpoint{3.727266in}{1.237308in}}%
\pgfpathlineto{\pgfqpoint{3.728530in}{1.224173in}}%
\pgfpathlineto{\pgfqpoint{3.731479in}{1.224173in}}%
\pgfpathlineto{\pgfqpoint{3.731900in}{1.250443in}}%
\pgfpathlineto{\pgfqpoint{3.732322in}{1.224173in}}%
\pgfpathlineto{\pgfqpoint{3.732743in}{1.211038in}}%
\pgfpathlineto{\pgfqpoint{3.733164in}{1.237308in}}%
\pgfpathlineto{\pgfqpoint{3.734007in}{1.224173in}}%
\pgfpathlineto{\pgfqpoint{3.734428in}{1.224173in}}%
\pgfpathlineto{\pgfqpoint{3.735692in}{1.211038in}}%
\pgfpathlineto{\pgfqpoint{3.736956in}{1.224173in}}%
\pgfpathlineto{\pgfqpoint{3.737798in}{1.224173in}}%
\pgfpathlineto{\pgfqpoint{3.739062in}{1.237308in}}%
\pgfpathlineto{\pgfqpoint{3.739905in}{1.211038in}}%
\pgfpathlineto{\pgfqpoint{3.740326in}{1.224173in}}%
\pgfpathlineto{\pgfqpoint{3.740747in}{1.224173in}}%
\pgfpathlineto{\pgfqpoint{3.741169in}{1.211038in}}%
\pgfpathlineto{\pgfqpoint{3.741590in}{1.224173in}}%
\pgfpathlineto{\pgfqpoint{3.742011in}{1.237308in}}%
\pgfpathlineto{\pgfqpoint{3.742432in}{1.211038in}}%
\pgfpathlineto{\pgfqpoint{3.743275in}{1.224173in}}%
\pgfpathlineto{\pgfqpoint{3.744539in}{1.224173in}}%
\pgfpathlineto{\pgfqpoint{3.745803in}{1.211038in}}%
\pgfpathlineto{\pgfqpoint{3.746224in}{1.237308in}}%
\pgfpathlineto{\pgfqpoint{3.747066in}{1.224173in}}%
\pgfpathlineto{\pgfqpoint{3.747909in}{1.224173in}}%
\pgfpathlineto{\pgfqpoint{3.749173in}{1.237308in}}%
\pgfpathlineto{\pgfqpoint{3.750015in}{1.211038in}}%
\pgfpathlineto{\pgfqpoint{3.750437in}{1.224173in}}%
\pgfpathlineto{\pgfqpoint{3.754228in}{1.224173in}}%
\pgfpathlineto{\pgfqpoint{3.754649in}{1.237308in}}%
\pgfpathlineto{\pgfqpoint{3.755492in}{1.211038in}}%
\pgfpathlineto{\pgfqpoint{3.755913in}{1.237308in}}%
\pgfpathlineto{\pgfqpoint{3.756756in}{1.224173in}}%
\pgfpathlineto{\pgfqpoint{3.758862in}{1.224173in}}%
\pgfpathlineto{\pgfqpoint{3.759283in}{1.237308in}}%
\pgfpathlineto{\pgfqpoint{3.760126in}{1.211038in}}%
\pgfpathlineto{\pgfqpoint{3.760547in}{1.224173in}}%
\pgfpathlineto{\pgfqpoint{3.760969in}{1.224173in}}%
\pgfpathlineto{\pgfqpoint{3.762232in}{1.211038in}}%
\pgfpathlineto{\pgfqpoint{3.763496in}{1.224173in}}%
\pgfpathlineto{\pgfqpoint{3.764760in}{1.224173in}}%
\pgfpathlineto{\pgfqpoint{3.766024in}{1.211038in}}%
\pgfpathlineto{\pgfqpoint{3.766445in}{1.211038in}}%
\pgfpathlineto{\pgfqpoint{3.766866in}{1.237308in}}%
\pgfpathlineto{\pgfqpoint{3.767709in}{1.224173in}}%
\pgfpathlineto{\pgfqpoint{3.768130in}{1.211038in}}%
\pgfpathlineto{\pgfqpoint{3.768552in}{1.237308in}}%
\pgfpathlineto{\pgfqpoint{3.769394in}{1.224173in}}%
\pgfpathlineto{\pgfqpoint{3.770237in}{1.224173in}}%
\pgfpathlineto{\pgfqpoint{3.770658in}{1.211038in}}%
\pgfpathlineto{\pgfqpoint{3.771079in}{1.224173in}}%
\pgfpathlineto{\pgfqpoint{3.771501in}{1.224173in}}%
\pgfpathlineto{\pgfqpoint{3.771922in}{1.237308in}}%
\pgfpathlineto{\pgfqpoint{3.772343in}{1.224173in}}%
\pgfpathlineto{\pgfqpoint{3.772764in}{1.224173in}}%
\pgfpathlineto{\pgfqpoint{3.773607in}{1.237308in}}%
\pgfpathlineto{\pgfqpoint{3.774449in}{1.224173in}}%
\pgfpathlineto{\pgfqpoint{3.774871in}{1.237308in}}%
\pgfpathlineto{\pgfqpoint{3.776135in}{1.211038in}}%
\pgfpathlineto{\pgfqpoint{3.776977in}{1.237308in}}%
\pgfpathlineto{\pgfqpoint{3.777398in}{1.224173in}}%
\pgfpathlineto{\pgfqpoint{3.778241in}{1.224173in}}%
\pgfpathlineto{\pgfqpoint{3.778662in}{1.237308in}}%
\pgfpathlineto{\pgfqpoint{3.780347in}{1.197903in}}%
\pgfpathlineto{\pgfqpoint{3.781611in}{1.237308in}}%
\pgfpathlineto{\pgfqpoint{3.782032in}{1.211038in}}%
\pgfpathlineto{\pgfqpoint{3.782875in}{1.224173in}}%
\pgfpathlineto{\pgfqpoint{3.783296in}{1.237308in}}%
\pgfpathlineto{\pgfqpoint{3.784560in}{1.211038in}}%
\pgfpathlineto{\pgfqpoint{3.784981in}{1.224173in}}%
\pgfpathlineto{\pgfqpoint{3.785403in}{1.197903in}}%
\pgfpathlineto{\pgfqpoint{3.786245in}{1.211038in}}%
\pgfpathlineto{\pgfqpoint{3.787088in}{1.211038in}}%
\pgfpathlineto{\pgfqpoint{3.788773in}{1.237308in}}%
\pgfpathlineto{\pgfqpoint{3.790458in}{1.211038in}}%
\pgfpathlineto{\pgfqpoint{3.791722in}{1.224173in}}%
\pgfpathlineto{\pgfqpoint{3.792143in}{1.224173in}}%
\pgfpathlineto{\pgfqpoint{3.793407in}{1.237308in}}%
\pgfpathlineto{\pgfqpoint{3.794250in}{1.237308in}}%
\pgfpathlineto{\pgfqpoint{3.795513in}{1.211038in}}%
\pgfpathlineto{\pgfqpoint{3.796356in}{1.224173in}}%
\pgfpathlineto{\pgfqpoint{3.797620in}{1.211038in}}%
\pgfpathlineto{\pgfqpoint{3.798884in}{1.250443in}}%
\pgfpathlineto{\pgfqpoint{3.800147in}{1.224173in}}%
\pgfpathlineto{\pgfqpoint{3.801833in}{1.224173in}}%
\pgfpathlineto{\pgfqpoint{3.802254in}{1.211038in}}%
\pgfpathlineto{\pgfqpoint{3.802675in}{1.224173in}}%
\pgfpathlineto{\pgfqpoint{3.803518in}{1.224173in}}%
\pgfpathlineto{\pgfqpoint{3.803939in}{1.237308in}}%
\pgfpathlineto{\pgfqpoint{3.805203in}{1.211038in}}%
\pgfpathlineto{\pgfqpoint{3.805624in}{0.738175in}}%
\pgfpathlineto{\pgfqpoint{3.806045in}{1.224173in}}%
\pgfpathlineto{\pgfqpoint{3.806467in}{1.224173in}}%
\pgfpathlineto{\pgfqpoint{3.807309in}{1.211038in}}%
\pgfpathlineto{\pgfqpoint{3.808152in}{1.224173in}}%
\pgfpathlineto{\pgfqpoint{3.808573in}{1.211038in}}%
\pgfpathlineto{\pgfqpoint{3.808994in}{1.237308in}}%
\pgfpathlineto{\pgfqpoint{3.809837in}{1.224173in}}%
\pgfpathlineto{\pgfqpoint{3.810258in}{1.224173in}}%
\pgfpathlineto{\pgfqpoint{3.811101in}{1.237308in}}%
\pgfpathlineto{\pgfqpoint{3.811943in}{1.250443in}}%
\pgfpathlineto{\pgfqpoint{3.812365in}{1.211038in}}%
\pgfpathlineto{\pgfqpoint{3.813628in}{1.237308in}}%
\pgfpathlineto{\pgfqpoint{3.814050in}{1.237308in}}%
\pgfpathlineto{\pgfqpoint{3.814471in}{1.197903in}}%
\pgfpathlineto{\pgfqpoint{3.814892in}{1.211038in}}%
\pgfpathlineto{\pgfqpoint{3.815735in}{1.224173in}}%
\pgfpathlineto{\pgfqpoint{3.816156in}{1.027147in}}%
\pgfpathlineto{\pgfqpoint{3.816577in}{1.237308in}}%
\pgfpathlineto{\pgfqpoint{3.817420in}{1.211038in}}%
\pgfpathlineto{\pgfqpoint{3.817841in}{1.224173in}}%
\pgfpathlineto{\pgfqpoint{3.818262in}{1.224173in}}%
\pgfpathlineto{\pgfqpoint{3.818684in}{1.250443in}}%
\pgfpathlineto{\pgfqpoint{3.819105in}{1.710172in}}%
\pgfpathlineto{\pgfqpoint{3.819526in}{1.211038in}}%
\pgfpathlineto{\pgfqpoint{3.820790in}{1.237308in}}%
\pgfpathlineto{\pgfqpoint{3.821633in}{1.224173in}}%
\pgfpathlineto{\pgfqpoint{3.822054in}{1.237308in}}%
\pgfpathlineto{\pgfqpoint{3.822475in}{1.224173in}}%
\pgfpathlineto{\pgfqpoint{3.822896in}{1.224173in}}%
\pgfpathlineto{\pgfqpoint{3.824160in}{1.237308in}}%
\pgfpathlineto{\pgfqpoint{3.825424in}{1.211038in}}%
\pgfpathlineto{\pgfqpoint{3.826267in}{1.211038in}}%
\pgfpathlineto{\pgfqpoint{3.827952in}{1.250443in}}%
\pgfpathlineto{\pgfqpoint{3.828373in}{1.211038in}}%
\pgfpathlineto{\pgfqpoint{3.828794in}{1.224173in}}%
\pgfpathlineto{\pgfqpoint{3.829216in}{1.237308in}}%
\pgfpathlineto{\pgfqpoint{3.829637in}{1.421200in}}%
\pgfpathlineto{\pgfqpoint{3.830058in}{1.211038in}}%
\pgfpathlineto{\pgfqpoint{3.830901in}{1.250443in}}%
\pgfpathlineto{\pgfqpoint{3.831743in}{1.237308in}}%
\pgfpathlineto{\pgfqpoint{3.832165in}{1.211038in}}%
\pgfpathlineto{\pgfqpoint{3.832586in}{1.224173in}}%
\pgfpathlineto{\pgfqpoint{3.833007in}{1.237308in}}%
\pgfpathlineto{\pgfqpoint{3.833428in}{1.211038in}}%
\pgfpathlineto{\pgfqpoint{3.833850in}{1.224173in}}%
\pgfpathlineto{\pgfqpoint{3.834271in}{1.237308in}}%
\pgfpathlineto{\pgfqpoint{3.835114in}{1.250443in}}%
\pgfpathlineto{\pgfqpoint{3.835535in}{1.211038in}}%
\pgfpathlineto{\pgfqpoint{3.836377in}{1.224173in}}%
\pgfpathlineto{\pgfqpoint{3.837220in}{1.197903in}}%
\pgfpathlineto{\pgfqpoint{3.837641in}{1.211038in}}%
\pgfpathlineto{\pgfqpoint{3.838062in}{1.237308in}}%
\pgfpathlineto{\pgfqpoint{3.838905in}{1.224173in}}%
\pgfpathlineto{\pgfqpoint{3.839326in}{1.237308in}}%
\pgfpathlineto{\pgfqpoint{3.839748in}{1.224173in}}%
\pgfpathlineto{\pgfqpoint{3.840169in}{1.224173in}}%
\pgfpathlineto{\pgfqpoint{3.840590in}{1.237308in}}%
\pgfpathlineto{\pgfqpoint{3.841011in}{1.211038in}}%
\pgfpathlineto{\pgfqpoint{3.841433in}{1.224173in}}%
\pgfpathlineto{\pgfqpoint{3.841854in}{1.250443in}}%
\pgfpathlineto{\pgfqpoint{3.842275in}{1.224173in}}%
\pgfpathlineto{\pgfqpoint{3.842697in}{1.224173in}}%
\pgfpathlineto{\pgfqpoint{3.843118in}{1.211038in}}%
\pgfpathlineto{\pgfqpoint{3.843539in}{1.237308in}}%
\pgfpathlineto{\pgfqpoint{3.843960in}{1.224173in}}%
\pgfpathlineto{\pgfqpoint{3.844803in}{1.197903in}}%
\pgfpathlineto{\pgfqpoint{3.845224in}{1.211038in}}%
\pgfpathlineto{\pgfqpoint{3.846067in}{1.211038in}}%
\pgfpathlineto{\pgfqpoint{3.846488in}{1.224173in}}%
\pgfpathlineto{\pgfqpoint{3.846909in}{1.211038in}}%
\pgfpathlineto{\pgfqpoint{3.847331in}{1.211038in}}%
\pgfpathlineto{\pgfqpoint{3.847752in}{1.224173in}}%
\pgfpathlineto{\pgfqpoint{3.848173in}{1.211038in}}%
\pgfpathlineto{\pgfqpoint{3.848594in}{1.211038in}}%
\pgfpathlineto{\pgfqpoint{3.849016in}{1.237308in}}%
\pgfpathlineto{\pgfqpoint{3.849858in}{1.224173in}}%
\pgfpathlineto{\pgfqpoint{3.850701in}{1.250443in}}%
\pgfpathlineto{\pgfqpoint{3.851965in}{1.224173in}}%
\pgfpathlineto{\pgfqpoint{3.853229in}{1.224173in}}%
\pgfpathlineto{\pgfqpoint{3.853650in}{1.211038in}}%
\pgfpathlineto{\pgfqpoint{3.854071in}{1.224173in}}%
\pgfpathlineto{\pgfqpoint{3.854492in}{1.237308in}}%
\pgfpathlineto{\pgfqpoint{3.856177in}{1.197903in}}%
\pgfpathlineto{\pgfqpoint{3.857020in}{1.224173in}}%
\pgfpathlineto{\pgfqpoint{3.857441in}{1.211038in}}%
\pgfpathlineto{\pgfqpoint{3.857863in}{1.211038in}}%
\pgfpathlineto{\pgfqpoint{3.859126in}{1.237308in}}%
\pgfpathlineto{\pgfqpoint{3.859969in}{1.224173in}}%
\pgfpathlineto{\pgfqpoint{3.860390in}{1.237308in}}%
\pgfpathlineto{\pgfqpoint{3.860812in}{1.224173in}}%
\pgfpathlineto{\pgfqpoint{3.861233in}{1.211038in}}%
\pgfpathlineto{\pgfqpoint{3.861654in}{1.237308in}}%
\pgfpathlineto{\pgfqpoint{3.862497in}{1.224173in}}%
\pgfpathlineto{\pgfqpoint{3.865446in}{1.224173in}}%
\pgfpathlineto{\pgfqpoint{3.865867in}{1.211038in}}%
\pgfpathlineto{\pgfqpoint{3.866288in}{1.224173in}}%
\pgfpathlineto{\pgfqpoint{3.866709in}{1.224173in}}%
\pgfpathlineto{\pgfqpoint{3.867131in}{1.237308in}}%
\pgfpathlineto{\pgfqpoint{3.867552in}{1.224173in}}%
\pgfpathlineto{\pgfqpoint{3.867973in}{1.211038in}}%
\pgfpathlineto{\pgfqpoint{3.868395in}{1.224173in}}%
\pgfpathlineto{\pgfqpoint{3.869658in}{1.237308in}}%
\pgfpathlineto{\pgfqpoint{3.870080in}{1.237308in}}%
\pgfpathlineto{\pgfqpoint{3.871343in}{1.211038in}}%
\pgfpathlineto{\pgfqpoint{3.873029in}{1.237308in}}%
\pgfpathlineto{\pgfqpoint{3.873871in}{1.211038in}}%
\pgfpathlineto{\pgfqpoint{3.875135in}{1.237308in}}%
\pgfpathlineto{\pgfqpoint{3.875978in}{1.211038in}}%
\pgfpathlineto{\pgfqpoint{3.876399in}{1.224173in}}%
\pgfpathlineto{\pgfqpoint{3.877663in}{1.211038in}}%
\pgfpathlineto{\pgfqpoint{3.878505in}{1.197903in}}%
\pgfpathlineto{\pgfqpoint{3.878926in}{1.237308in}}%
\pgfpathlineto{\pgfqpoint{3.879769in}{1.237308in}}%
\pgfpathlineto{\pgfqpoint{3.881454in}{1.211038in}}%
\pgfpathlineto{\pgfqpoint{3.883139in}{1.237308in}}%
\pgfpathlineto{\pgfqpoint{3.883982in}{1.211038in}}%
\pgfpathlineto{\pgfqpoint{3.884824in}{1.250443in}}%
\pgfpathlineto{\pgfqpoint{3.885246in}{1.237308in}}%
\pgfpathlineto{\pgfqpoint{3.885667in}{1.237308in}}%
\pgfpathlineto{\pgfqpoint{3.886509in}{1.211038in}}%
\pgfpathlineto{\pgfqpoint{3.886931in}{1.224173in}}%
\pgfpathlineto{\pgfqpoint{3.887352in}{1.224173in}}%
\pgfpathlineto{\pgfqpoint{3.887773in}{1.211038in}}%
\pgfpathlineto{\pgfqpoint{3.888195in}{1.224173in}}%
\pgfpathlineto{\pgfqpoint{3.888616in}{1.224173in}}%
\pgfpathlineto{\pgfqpoint{3.889880in}{1.237308in}}%
\pgfpathlineto{\pgfqpoint{3.890722in}{1.237308in}}%
\pgfpathlineto{\pgfqpoint{3.891565in}{1.211038in}}%
\pgfpathlineto{\pgfqpoint{3.891986in}{1.237308in}}%
\pgfpathlineto{\pgfqpoint{3.892829in}{1.224173in}}%
\pgfpathlineto{\pgfqpoint{3.893250in}{1.224173in}}%
\pgfpathlineto{\pgfqpoint{3.894514in}{1.237308in}}%
\pgfpathlineto{\pgfqpoint{3.895778in}{1.237308in}}%
\pgfpathlineto{\pgfqpoint{3.897463in}{1.211038in}}%
\pgfpathlineto{\pgfqpoint{3.898727in}{1.211038in}}%
\pgfpathlineto{\pgfqpoint{3.899148in}{1.237308in}}%
\pgfpathlineto{\pgfqpoint{3.899990in}{1.224173in}}%
\pgfpathlineto{\pgfqpoint{3.900833in}{1.224173in}}%
\pgfpathlineto{\pgfqpoint{3.901254in}{1.237308in}}%
\pgfpathlineto{\pgfqpoint{3.901675in}{1.224173in}}%
\pgfpathlineto{\pgfqpoint{3.902518in}{1.224173in}}%
\pgfpathlineto{\pgfqpoint{3.902939in}{1.263578in}}%
\pgfpathlineto{\pgfqpoint{3.903361in}{1.224173in}}%
\pgfpathlineto{\pgfqpoint{3.904624in}{1.237308in}}%
\pgfpathlineto{\pgfqpoint{3.905467in}{1.237308in}}%
\pgfpathlineto{\pgfqpoint{3.906731in}{1.211038in}}%
\pgfpathlineto{\pgfqpoint{3.908416in}{1.276714in}}%
\pgfpathlineto{\pgfqpoint{3.908837in}{1.263578in}}%
\pgfpathlineto{\pgfqpoint{3.910522in}{1.211038in}}%
\pgfpathlineto{\pgfqpoint{3.910944in}{1.250443in}}%
\pgfpathlineto{\pgfqpoint{3.911786in}{1.237308in}}%
\pgfpathlineto{\pgfqpoint{3.912207in}{1.237308in}}%
\pgfpathlineto{\pgfqpoint{3.912629in}{1.211038in}}%
\pgfpathlineto{\pgfqpoint{3.913050in}{1.237308in}}%
\pgfpathlineto{\pgfqpoint{3.913471in}{1.237308in}}%
\pgfpathlineto{\pgfqpoint{3.913893in}{1.224173in}}%
\pgfpathlineto{\pgfqpoint{3.914314in}{1.237308in}}%
\pgfpathlineto{\pgfqpoint{3.915156in}{1.237308in}}%
\pgfpathlineto{\pgfqpoint{3.916420in}{1.197903in}}%
\pgfpathlineto{\pgfqpoint{3.917684in}{1.224173in}}%
\pgfpathlineto{\pgfqpoint{3.918948in}{1.224173in}}%
\pgfpathlineto{\pgfqpoint{3.920212in}{1.237308in}}%
\pgfpathlineto{\pgfqpoint{3.921897in}{1.171633in}}%
\pgfpathlineto{\pgfqpoint{3.922318in}{1.184768in}}%
\pgfpathlineto{\pgfqpoint{3.924003in}{1.250443in}}%
\pgfpathlineto{\pgfqpoint{3.924425in}{1.224173in}}%
\pgfpathlineto{\pgfqpoint{3.924846in}{1.224173in}}%
\pgfpathlineto{\pgfqpoint{3.925267in}{1.237308in}}%
\pgfpathlineto{\pgfqpoint{3.925688in}{1.211038in}}%
\pgfpathlineto{\pgfqpoint{3.926531in}{1.224173in}}%
\pgfpathlineto{\pgfqpoint{3.926952in}{1.211038in}}%
\pgfpathlineto{\pgfqpoint{3.927373in}{1.237308in}}%
\pgfpathlineto{\pgfqpoint{3.927795in}{1.224173in}}%
\pgfpathlineto{\pgfqpoint{3.928637in}{1.211038in}}%
\pgfpathlineto{\pgfqpoint{3.929901in}{1.224173in}}%
\pgfpathlineto{\pgfqpoint{3.931586in}{1.224173in}}%
\pgfpathlineto{\pgfqpoint{3.932008in}{1.211038in}}%
\pgfpathlineto{\pgfqpoint{3.932429in}{1.224173in}}%
\pgfpathlineto{\pgfqpoint{3.933693in}{1.224173in}}%
\pgfpathlineto{\pgfqpoint{3.934114in}{1.237308in}}%
\pgfpathlineto{\pgfqpoint{3.934535in}{1.224173in}}%
\pgfpathlineto{\pgfqpoint{3.934956in}{1.211038in}}%
\pgfpathlineto{\pgfqpoint{3.936220in}{1.237308in}}%
\pgfpathlineto{\pgfqpoint{3.937905in}{1.211038in}}%
\pgfpathlineto{\pgfqpoint{3.939169in}{1.250443in}}%
\pgfpathlineto{\pgfqpoint{3.940854in}{1.211038in}}%
\pgfpathlineto{\pgfqpoint{3.941276in}{1.211038in}}%
\pgfpathlineto{\pgfqpoint{3.942539in}{1.224173in}}%
\pgfpathlineto{\pgfqpoint{3.943382in}{1.224173in}}%
\pgfpathlineto{\pgfqpoint{3.943803in}{1.197903in}}%
\pgfpathlineto{\pgfqpoint{3.944225in}{1.224173in}}%
\pgfpathlineto{\pgfqpoint{3.945067in}{1.224173in}}%
\pgfpathlineto{\pgfqpoint{3.945488in}{1.237308in}}%
\pgfpathlineto{\pgfqpoint{3.945910in}{1.224173in}}%
\pgfpathlineto{\pgfqpoint{3.946752in}{1.224173in}}%
\pgfpathlineto{\pgfqpoint{3.947595in}{1.211038in}}%
\pgfpathlineto{\pgfqpoint{3.948437in}{1.224173in}}%
\pgfpathlineto{\pgfqpoint{3.948859in}{1.211038in}}%
\pgfpathlineto{\pgfqpoint{3.949280in}{1.224173in}}%
\pgfpathlineto{\pgfqpoint{3.950122in}{1.224173in}}%
\pgfpathlineto{\pgfqpoint{3.951808in}{1.250443in}}%
\pgfpathlineto{\pgfqpoint{3.952650in}{1.211038in}}%
\pgfpathlineto{\pgfqpoint{3.953071in}{1.224173in}}%
\pgfpathlineto{\pgfqpoint{3.953493in}{1.224173in}}%
\pgfpathlineto{\pgfqpoint{3.953914in}{1.211038in}}%
\pgfpathlineto{\pgfqpoint{3.955599in}{1.250443in}}%
\pgfpathlineto{\pgfqpoint{3.956863in}{1.224173in}}%
\pgfpathlineto{\pgfqpoint{3.957284in}{1.250443in}}%
\pgfpathlineto{\pgfqpoint{3.957705in}{1.224173in}}%
\pgfpathlineto{\pgfqpoint{3.960654in}{1.224173in}}%
\pgfpathlineto{\pgfqpoint{3.961497in}{1.211038in}}%
\pgfpathlineto{\pgfqpoint{3.963182in}{1.237308in}}%
\pgfpathlineto{\pgfqpoint{3.964446in}{1.224173in}}%
\pgfpathlineto{\pgfqpoint{3.964867in}{1.224173in}}%
\pgfpathlineto{\pgfqpoint{3.965288in}{1.197903in}}%
\pgfpathlineto{\pgfqpoint{3.965710in}{1.237308in}}%
\pgfpathlineto{\pgfqpoint{3.966974in}{1.224173in}}%
\pgfpathlineto{\pgfqpoint{3.967816in}{1.224173in}}%
\pgfpathlineto{\pgfqpoint{3.968237in}{1.237308in}}%
\pgfpathlineto{\pgfqpoint{3.968659in}{1.224173in}}%
\pgfpathlineto{\pgfqpoint{3.969923in}{1.211038in}}%
\pgfpathlineto{\pgfqpoint{3.971186in}{1.224173in}}%
\pgfpathlineto{\pgfqpoint{3.972029in}{1.224173in}}%
\pgfpathlineto{\pgfqpoint{3.972450in}{1.211038in}}%
\pgfpathlineto{\pgfqpoint{3.972871in}{1.224173in}}%
\pgfpathlineto{\pgfqpoint{3.973714in}{1.224173in}}%
\pgfpathlineto{\pgfqpoint{3.974978in}{1.237308in}}%
\pgfpathlineto{\pgfqpoint{3.976663in}{1.211038in}}%
\pgfpathlineto{\pgfqpoint{3.977084in}{1.250443in}}%
\pgfpathlineto{\pgfqpoint{3.977506in}{1.211038in}}%
\pgfpathlineto{\pgfqpoint{3.977927in}{1.224173in}}%
\pgfpathlineto{\pgfqpoint{3.978348in}{1.197903in}}%
\pgfpathlineto{\pgfqpoint{3.978769in}{1.224173in}}%
\pgfpathlineto{\pgfqpoint{3.980454in}{1.224173in}}%
\pgfpathlineto{\pgfqpoint{3.981297in}{1.237308in}}%
\pgfpathlineto{\pgfqpoint{3.981718in}{1.211038in}}%
\pgfpathlineto{\pgfqpoint{3.982140in}{1.263578in}}%
\pgfpathlineto{\pgfqpoint{3.982561in}{1.211038in}}%
\pgfpathlineto{\pgfqpoint{3.983825in}{1.224173in}}%
\pgfpathlineto{\pgfqpoint{3.984667in}{1.224173in}}%
\pgfpathlineto{\pgfqpoint{3.985931in}{1.237308in}}%
\pgfpathlineto{\pgfqpoint{3.987616in}{1.211038in}}%
\pgfpathlineto{\pgfqpoint{3.988038in}{1.237308in}}%
\pgfpathlineto{\pgfqpoint{3.988880in}{1.224173in}}%
\pgfpathlineto{\pgfqpoint{3.989301in}{1.224173in}}%
\pgfpathlineto{\pgfqpoint{3.989723in}{1.237308in}}%
\pgfpathlineto{\pgfqpoint{3.990144in}{1.224173in}}%
\pgfpathlineto{\pgfqpoint{3.990565in}{1.197903in}}%
\pgfpathlineto{\pgfqpoint{3.990986in}{1.237308in}}%
\pgfpathlineto{\pgfqpoint{3.991408in}{1.224173in}}%
\pgfpathlineto{\pgfqpoint{3.991829in}{1.250443in}}%
\pgfpathlineto{\pgfqpoint{3.992250in}{1.224173in}}%
\pgfpathlineto{\pgfqpoint{3.992672in}{1.211038in}}%
\pgfpathlineto{\pgfqpoint{3.993093in}{1.224173in}}%
\pgfpathlineto{\pgfqpoint{3.994357in}{1.224173in}}%
\pgfpathlineto{\pgfqpoint{3.995621in}{1.184768in}}%
\pgfpathlineto{\pgfqpoint{3.996042in}{1.237308in}}%
\pgfpathlineto{\pgfqpoint{3.996884in}{1.224173in}}%
\pgfpathlineto{\pgfqpoint{3.997306in}{1.224173in}}%
\pgfpathlineto{\pgfqpoint{3.998569in}{1.211038in}}%
\pgfpathlineto{\pgfqpoint{3.999833in}{1.224173in}}%
\pgfpathlineto{\pgfqpoint{4.000676in}{1.224173in}}%
\pgfpathlineto{\pgfqpoint{4.001097in}{1.250443in}}%
\pgfpathlineto{\pgfqpoint{4.001518in}{1.211038in}}%
\pgfpathlineto{\pgfqpoint{4.002361in}{1.224173in}}%
\pgfpathlineto{\pgfqpoint{4.003204in}{1.211038in}}%
\pgfpathlineto{\pgfqpoint{4.003625in}{1.237308in}}%
\pgfpathlineto{\pgfqpoint{4.004467in}{1.224173in}}%
\pgfpathlineto{\pgfqpoint{4.005310in}{1.224173in}}%
\pgfpathlineto{\pgfqpoint{4.006152in}{1.237308in}}%
\pgfpathlineto{\pgfqpoint{4.007416in}{1.224173in}}%
\pgfpathlineto{\pgfqpoint{4.007838in}{1.237308in}}%
\pgfpathlineto{\pgfqpoint{4.008259in}{1.224173in}}%
\pgfpathlineto{\pgfqpoint{4.008680in}{1.224173in}}%
\pgfpathlineto{\pgfqpoint{4.009944in}{1.211038in}}%
\pgfpathlineto{\pgfqpoint{4.011629in}{1.237308in}}%
\pgfpathlineto{\pgfqpoint{4.012472in}{1.224173in}}%
\pgfpathlineto{\pgfqpoint{4.012893in}{1.237308in}}%
\pgfpathlineto{\pgfqpoint{4.013314in}{1.224173in}}%
\pgfpathlineto{\pgfqpoint{4.014578in}{1.197903in}}%
\pgfpathlineto{\pgfqpoint{4.016684in}{1.237308in}}%
\pgfpathlineto{\pgfqpoint{4.017948in}{1.211038in}}%
\pgfpathlineto{\pgfqpoint{4.018791in}{1.237308in}}%
\pgfpathlineto{\pgfqpoint{4.019212in}{1.211038in}}%
\pgfpathlineto{\pgfqpoint{4.020055in}{1.224173in}}%
\pgfpathlineto{\pgfqpoint{4.020897in}{1.224173in}}%
\pgfpathlineto{\pgfqpoint{4.021318in}{1.211038in}}%
\pgfpathlineto{\pgfqpoint{4.022582in}{1.237308in}}%
\pgfpathlineto{\pgfqpoint{4.024267in}{1.211038in}}%
\pgfpathlineto{\pgfqpoint{4.025531in}{1.237308in}}%
\pgfpathlineto{\pgfqpoint{4.025953in}{1.197903in}}%
\pgfpathlineto{\pgfqpoint{4.026374in}{1.224173in}}%
\pgfpathlineto{\pgfqpoint{4.026795in}{1.224173in}}%
\pgfpathlineto{\pgfqpoint{4.027216in}{1.237308in}}%
\pgfpathlineto{\pgfqpoint{4.027638in}{1.224173in}}%
\pgfpathlineto{\pgfqpoint{4.028059in}{1.224173in}}%
\pgfpathlineto{\pgfqpoint{4.028901in}{1.237308in}}%
\pgfpathlineto{\pgfqpoint{4.030165in}{1.211038in}}%
\pgfpathlineto{\pgfqpoint{4.031429in}{1.237308in}}%
\pgfpathlineto{\pgfqpoint{4.033114in}{1.197903in}}%
\pgfpathlineto{\pgfqpoint{4.033957in}{1.237308in}}%
\pgfpathlineto{\pgfqpoint{4.034378in}{1.224173in}}%
\pgfpathlineto{\pgfqpoint{4.035642in}{1.224173in}}%
\pgfpathlineto{\pgfqpoint{4.036484in}{1.237308in}}%
\pgfpathlineto{\pgfqpoint{4.036906in}{1.211038in}}%
\pgfpathlineto{\pgfqpoint{4.037748in}{1.224173in}}%
\pgfpathlineto{\pgfqpoint{4.038170in}{1.211038in}}%
\pgfpathlineto{\pgfqpoint{4.038591in}{1.224173in}}%
\pgfpathlineto{\pgfqpoint{4.039012in}{1.224173in}}%
\pgfpathlineto{\pgfqpoint{4.039433in}{1.250443in}}%
\pgfpathlineto{\pgfqpoint{4.039855in}{1.224173in}}%
\pgfpathlineto{\pgfqpoint{4.041119in}{1.224173in}}%
\pgfpathlineto{\pgfqpoint{4.041540in}{1.250443in}}%
\pgfpathlineto{\pgfqpoint{4.041961in}{1.224173in}}%
\pgfpathlineto{\pgfqpoint{4.043225in}{1.224173in}}%
\pgfpathlineto{\pgfqpoint{4.044067in}{1.211038in}}%
\pgfpathlineto{\pgfqpoint{4.045331in}{1.224173in}}%
\pgfpathlineto{\pgfqpoint{4.047016in}{1.224173in}}%
\pgfpathlineto{\pgfqpoint{4.048280in}{1.211038in}}%
\pgfpathlineto{\pgfqpoint{4.048702in}{1.211038in}}%
\pgfpathlineto{\pgfqpoint{4.050387in}{1.250443in}}%
\pgfpathlineto{\pgfqpoint{4.051229in}{1.224173in}}%
\pgfpathlineto{\pgfqpoint{4.051651in}{1.237308in}}%
\pgfpathlineto{\pgfqpoint{4.053336in}{1.211038in}}%
\pgfpathlineto{\pgfqpoint{4.054599in}{1.224173in}}%
\pgfpathlineto{\pgfqpoint{4.055021in}{1.211038in}}%
\pgfpathlineto{\pgfqpoint{4.055442in}{1.224173in}}%
\pgfpathlineto{\pgfqpoint{4.057548in}{1.224173in}}%
\pgfpathlineto{\pgfqpoint{4.057970in}{1.237308in}}%
\pgfpathlineto{\pgfqpoint{4.058391in}{1.211038in}}%
\pgfpathlineto{\pgfqpoint{4.059234in}{1.224173in}}%
\pgfpathlineto{\pgfqpoint{4.059655in}{1.224173in}}%
\pgfpathlineto{\pgfqpoint{4.060076in}{1.237308in}}%
\pgfpathlineto{\pgfqpoint{4.060497in}{1.224173in}}%
\pgfpathlineto{\pgfqpoint{4.061340in}{1.224173in}}%
\pgfpathlineto{\pgfqpoint{4.061761in}{1.237308in}}%
\pgfpathlineto{\pgfqpoint{4.062604in}{1.211038in}}%
\pgfpathlineto{\pgfqpoint{4.063025in}{1.224173in}}%
\pgfpathlineto{\pgfqpoint{4.063868in}{1.211038in}}%
\pgfpathlineto{\pgfqpoint{4.065131in}{1.224173in}}%
\pgfpathlineto{\pgfqpoint{4.065553in}{1.224173in}}%
\pgfpathlineto{\pgfqpoint{4.065974in}{1.211038in}}%
\pgfpathlineto{\pgfqpoint{4.066395in}{1.224173in}}%
\pgfpathlineto{\pgfqpoint{4.066817in}{1.237308in}}%
\pgfpathlineto{\pgfqpoint{4.067238in}{1.224173in}}%
\pgfpathlineto{\pgfqpoint{4.068080in}{1.224173in}}%
\pgfpathlineto{\pgfqpoint{4.068502in}{1.211038in}}%
\pgfpathlineto{\pgfqpoint{4.068923in}{1.237308in}}%
\pgfpathlineto{\pgfqpoint{4.069765in}{1.224173in}}%
\pgfpathlineto{\pgfqpoint{4.070187in}{1.224173in}}%
\pgfpathlineto{\pgfqpoint{4.070608in}{1.237308in}}%
\pgfpathlineto{\pgfqpoint{4.071029in}{1.224173in}}%
\pgfpathlineto{\pgfqpoint{4.071451in}{1.211038in}}%
\pgfpathlineto{\pgfqpoint{4.071872in}{1.250443in}}%
\pgfpathlineto{\pgfqpoint{4.072293in}{1.224173in}}%
\pgfpathlineto{\pgfqpoint{4.073136in}{1.224173in}}%
\pgfpathlineto{\pgfqpoint{4.073557in}{1.211038in}}%
\pgfpathlineto{\pgfqpoint{4.073978in}{1.224173in}}%
\pgfpathlineto{\pgfqpoint{4.075242in}{1.224173in}}%
\pgfpathlineto{\pgfqpoint{4.075663in}{1.250443in}}%
\pgfpathlineto{\pgfqpoint{4.076085in}{1.224173in}}%
\pgfpathlineto{\pgfqpoint{4.076506in}{1.224173in}}%
\pgfpathlineto{\pgfqpoint{4.076927in}{1.237308in}}%
\pgfpathlineto{\pgfqpoint{4.077348in}{1.224173in}}%
\pgfpathlineto{\pgfqpoint{4.078191in}{1.224173in}}%
\pgfpathlineto{\pgfqpoint{4.078612in}{1.197903in}}%
\pgfpathlineto{\pgfqpoint{4.079034in}{1.224173in}}%
\pgfpathlineto{\pgfqpoint{4.079876in}{1.237308in}}%
\pgfpathlineto{\pgfqpoint{4.081140in}{1.224173in}}%
\pgfpathlineto{\pgfqpoint{4.081561in}{1.224173in}}%
\pgfpathlineto{\pgfqpoint{4.081983in}{1.237308in}}%
\pgfpathlineto{\pgfqpoint{4.082404in}{1.211038in}}%
\pgfpathlineto{\pgfqpoint{4.083246in}{1.224173in}}%
\pgfpathlineto{\pgfqpoint{4.083668in}{1.211038in}}%
\pgfpathlineto{\pgfqpoint{4.084089in}{1.224173in}}%
\pgfpathlineto{\pgfqpoint{4.084931in}{1.224173in}}%
\pgfpathlineto{\pgfqpoint{4.085353in}{1.211038in}}%
\pgfpathlineto{\pgfqpoint{4.085774in}{1.224173in}}%
\pgfpathlineto{\pgfqpoint{4.086195in}{1.224173in}}%
\pgfpathlineto{\pgfqpoint{4.086617in}{1.211038in}}%
\pgfpathlineto{\pgfqpoint{4.087038in}{1.237308in}}%
\pgfpathlineto{\pgfqpoint{4.087880in}{1.224173in}}%
\pgfpathlineto{\pgfqpoint{4.088302in}{1.224173in}}%
\pgfpathlineto{\pgfqpoint{4.089144in}{1.211038in}}%
\pgfpathlineto{\pgfqpoint{4.090408in}{1.224173in}}%
\pgfpathlineto{\pgfqpoint{4.091251in}{1.224173in}}%
\pgfpathlineto{\pgfqpoint{4.092093in}{1.250443in}}%
\pgfpathlineto{\pgfqpoint{4.093778in}{1.211038in}}%
\pgfpathlineto{\pgfqpoint{4.095042in}{1.224173in}}%
\pgfpathlineto{\pgfqpoint{4.095885in}{1.224173in}}%
\pgfpathlineto{\pgfqpoint{4.097149in}{1.237308in}}%
\pgfpathlineto{\pgfqpoint{4.098834in}{1.211038in}}%
\pgfpathlineto{\pgfqpoint{4.100097in}{1.237308in}}%
\pgfpathlineto{\pgfqpoint{4.101361in}{1.224173in}}%
\pgfpathlineto{\pgfqpoint{4.101783in}{1.224173in}}%
\pgfpathlineto{\pgfqpoint{4.102204in}{1.250443in}}%
\pgfpathlineto{\pgfqpoint{4.102625in}{1.224173in}}%
\pgfpathlineto{\pgfqpoint{4.103468in}{1.224173in}}%
\pgfpathlineto{\pgfqpoint{4.103889in}{1.211038in}}%
\pgfpathlineto{\pgfqpoint{4.104310in}{1.237308in}}%
\pgfpathlineto{\pgfqpoint{4.104732in}{1.224173in}}%
\pgfpathlineto{\pgfqpoint{4.105153in}{1.211038in}}%
\pgfpathlineto{\pgfqpoint{4.105574in}{1.224173in}}%
\pgfpathlineto{\pgfqpoint{4.106838in}{1.224173in}}%
\pgfpathlineto{\pgfqpoint{4.107259in}{1.237308in}}%
\pgfpathlineto{\pgfqpoint{4.107680in}{1.224173in}}%
\pgfpathlineto{\pgfqpoint{4.108523in}{1.224173in}}%
\pgfpathlineto{\pgfqpoint{4.110208in}{1.197903in}}%
\pgfpathlineto{\pgfqpoint{4.111472in}{1.224173in}}%
\pgfpathlineto{\pgfqpoint{4.111893in}{1.224173in}}%
\pgfpathlineto{\pgfqpoint{4.112315in}{1.237308in}}%
\pgfpathlineto{\pgfqpoint{4.112736in}{1.224173in}}%
\pgfpathlineto{\pgfqpoint{4.113578in}{1.224173in}}%
\pgfpathlineto{\pgfqpoint{4.114000in}{1.211038in}}%
\pgfpathlineto{\pgfqpoint{4.114421in}{1.224173in}}%
\pgfpathlineto{\pgfqpoint{4.115263in}{1.224173in}}%
\pgfpathlineto{\pgfqpoint{4.115685in}{1.211038in}}%
\pgfpathlineto{\pgfqpoint{4.116106in}{1.237308in}}%
\pgfpathlineto{\pgfqpoint{4.116949in}{1.224173in}}%
\pgfpathlineto{\pgfqpoint{4.117370in}{1.237308in}}%
\pgfpathlineto{\pgfqpoint{4.117791in}{1.211038in}}%
\pgfpathlineto{\pgfqpoint{4.118634in}{1.224173in}}%
\pgfpathlineto{\pgfqpoint{4.119055in}{1.211038in}}%
\pgfpathlineto{\pgfqpoint{4.119476in}{1.224173in}}%
\pgfpathlineto{\pgfqpoint{4.119898in}{1.224173in}}%
\pgfpathlineto{\pgfqpoint{4.120319in}{1.263578in}}%
\pgfpathlineto{\pgfqpoint{4.121161in}{1.250443in}}%
\pgfpathlineto{\pgfqpoint{4.121583in}{1.211038in}}%
\pgfpathlineto{\pgfqpoint{4.122425in}{1.224173in}}%
\pgfpathlineto{\pgfqpoint{4.123268in}{1.224173in}}%
\pgfpathlineto{\pgfqpoint{4.123689in}{1.237308in}}%
\pgfpathlineto{\pgfqpoint{4.124532in}{1.211038in}}%
\pgfpathlineto{\pgfqpoint{4.124953in}{1.224173in}}%
\pgfpathlineto{\pgfqpoint{4.126217in}{1.211038in}}%
\pgfpathlineto{\pgfqpoint{4.127059in}{1.211038in}}%
\pgfpathlineto{\pgfqpoint{4.127481in}{1.250443in}}%
\pgfpathlineto{\pgfqpoint{4.127902in}{1.211038in}}%
\pgfpathlineto{\pgfqpoint{4.129166in}{1.224173in}}%
\pgfpathlineto{\pgfqpoint{4.129587in}{1.197903in}}%
\pgfpathlineto{\pgfqpoint{4.130430in}{1.211038in}}%
\pgfpathlineto{\pgfqpoint{4.130851in}{1.237308in}}%
\pgfpathlineto{\pgfqpoint{4.131272in}{1.224173in}}%
\pgfpathlineto{\pgfqpoint{4.131693in}{1.211038in}}%
\pgfpathlineto{\pgfqpoint{4.132115in}{1.224173in}}%
\pgfpathlineto{\pgfqpoint{4.132536in}{1.237308in}}%
\pgfpathlineto{\pgfqpoint{4.133800in}{1.184768in}}%
\pgfpathlineto{\pgfqpoint{4.135064in}{1.237308in}}%
\pgfpathlineto{\pgfqpoint{4.135485in}{1.211038in}}%
\pgfpathlineto{\pgfqpoint{4.135906in}{1.224173in}}%
\pgfpathlineto{\pgfqpoint{4.136327in}{1.263578in}}%
\pgfpathlineto{\pgfqpoint{4.136749in}{1.211038in}}%
\pgfpathlineto{\pgfqpoint{4.137170in}{1.250443in}}%
\pgfpathlineto{\pgfqpoint{4.138434in}{1.224173in}}%
\pgfpathlineto{\pgfqpoint{4.139698in}{1.237308in}}%
\pgfpathlineto{\pgfqpoint{4.140540in}{1.237308in}}%
\pgfpathlineto{\pgfqpoint{4.140961in}{1.211038in}}%
\pgfpathlineto{\pgfqpoint{4.141804in}{1.224173in}}%
\pgfpathlineto{\pgfqpoint{4.142647in}{1.224173in}}%
\pgfpathlineto{\pgfqpoint{4.143910in}{1.237308in}}%
\pgfpathlineto{\pgfqpoint{4.144332in}{1.197903in}}%
\pgfpathlineto{\pgfqpoint{4.144753in}{1.224173in}}%
\pgfpathlineto{\pgfqpoint{4.145174in}{1.237308in}}%
\pgfpathlineto{\pgfqpoint{4.145596in}{1.224173in}}%
\pgfpathlineto{\pgfqpoint{4.146017in}{1.224173in}}%
\pgfpathlineto{\pgfqpoint{4.146859in}{1.237308in}}%
\pgfpathlineto{\pgfqpoint{4.147281in}{1.224173in}}%
\pgfpathlineto{\pgfqpoint{4.147702in}{1.237308in}}%
\pgfpathlineto{\pgfqpoint{4.148123in}{1.250443in}}%
\pgfpathlineto{\pgfqpoint{4.149808in}{1.197903in}}%
\pgfpathlineto{\pgfqpoint{4.150230in}{1.237308in}}%
\pgfpathlineto{\pgfqpoint{4.151072in}{1.224173in}}%
\pgfpathlineto{\pgfqpoint{4.152336in}{1.224173in}}%
\pgfpathlineto{\pgfqpoint{4.153600in}{1.237308in}}%
\pgfpathlineto{\pgfqpoint{4.154442in}{1.211038in}}%
\pgfpathlineto{\pgfqpoint{4.154864in}{1.224173in}}%
\pgfpathlineto{\pgfqpoint{4.157391in}{1.224173in}}%
\pgfpathlineto{\pgfqpoint{4.158655in}{1.263578in}}%
\pgfpathlineto{\pgfqpoint{4.159498in}{1.211038in}}%
\pgfpathlineto{\pgfqpoint{4.160340in}{1.224173in}}%
\pgfpathlineto{\pgfqpoint{4.160762in}{1.237308in}}%
\pgfpathlineto{\pgfqpoint{4.161183in}{1.224173in}}%
\pgfpathlineto{\pgfqpoint{4.161604in}{1.211038in}}%
\pgfpathlineto{\pgfqpoint{4.162025in}{1.224173in}}%
\pgfpathlineto{\pgfqpoint{4.162447in}{1.224173in}}%
\pgfpathlineto{\pgfqpoint{4.162868in}{1.237308in}}%
\pgfpathlineto{\pgfqpoint{4.163289in}{1.224173in}}%
\pgfpathlineto{\pgfqpoint{4.164132in}{1.224173in}}%
\pgfpathlineto{\pgfqpoint{4.164553in}{1.211038in}}%
\pgfpathlineto{\pgfqpoint{4.164974in}{1.224173in}}%
\pgfpathlineto{\pgfqpoint{4.165396in}{1.237308in}}%
\pgfpathlineto{\pgfqpoint{4.165817in}{1.224173in}}%
\pgfpathlineto{\pgfqpoint{4.166659in}{1.224173in}}%
\pgfpathlineto{\pgfqpoint{4.167081in}{1.211038in}}%
\pgfpathlineto{\pgfqpoint{4.167502in}{1.224173in}}%
\pgfpathlineto{\pgfqpoint{4.167923in}{1.237308in}}%
\pgfpathlineto{\pgfqpoint{4.168345in}{1.224173in}}%
\pgfpathlineto{\pgfqpoint{4.169187in}{1.224173in}}%
\pgfpathlineto{\pgfqpoint{4.169608in}{1.211038in}}%
\pgfpathlineto{\pgfqpoint{4.170030in}{1.224173in}}%
\pgfpathlineto{\pgfqpoint{4.170872in}{1.224173in}}%
\pgfpathlineto{\pgfqpoint{4.171293in}{1.237308in}}%
\pgfpathlineto{\pgfqpoint{4.171715in}{1.224173in}}%
\pgfpathlineto{\pgfqpoint{4.172136in}{1.197903in}}%
\pgfpathlineto{\pgfqpoint{4.172557in}{1.224173in}}%
\pgfpathlineto{\pgfqpoint{4.172979in}{1.237308in}}%
\pgfpathlineto{\pgfqpoint{4.173400in}{1.224173in}}%
\pgfpathlineto{\pgfqpoint{4.173821in}{1.224173in}}%
\pgfpathlineto{\pgfqpoint{4.174664in}{1.211038in}}%
\pgfpathlineto{\pgfqpoint{4.175928in}{1.224173in}}%
\pgfpathlineto{\pgfqpoint{4.177613in}{1.224173in}}%
\pgfpathlineto{\pgfqpoint{4.178034in}{1.237308in}}%
\pgfpathlineto{\pgfqpoint{4.178876in}{1.211038in}}%
\pgfpathlineto{\pgfqpoint{4.179298in}{1.237308in}}%
\pgfpathlineto{\pgfqpoint{4.180140in}{1.224173in}}%
\pgfpathlineto{\pgfqpoint{4.180562in}{1.211038in}}%
\pgfpathlineto{\pgfqpoint{4.180983in}{1.224173in}}%
\pgfpathlineto{\pgfqpoint{4.182668in}{1.224173in}}%
\pgfpathlineto{\pgfqpoint{4.183089in}{1.237308in}}%
\pgfpathlineto{\pgfqpoint{4.183511in}{1.224173in}}%
\pgfpathlineto{\pgfqpoint{4.184353in}{1.224173in}}%
\pgfpathlineto{\pgfqpoint{4.184774in}{1.197903in}}%
\pgfpathlineto{\pgfqpoint{4.185617in}{1.211038in}}%
\pgfpathlineto{\pgfqpoint{4.186038in}{1.211038in}}%
\pgfpathlineto{\pgfqpoint{4.187302in}{1.224173in}}%
\pgfpathlineto{\pgfqpoint{4.187723in}{1.224173in}}%
\pgfpathlineto{\pgfqpoint{4.188145in}{1.237308in}}%
\pgfpathlineto{\pgfqpoint{4.188566in}{1.224173in}}%
\pgfpathlineto{\pgfqpoint{4.189408in}{1.224173in}}%
\pgfpathlineto{\pgfqpoint{4.189830in}{1.211038in}}%
\pgfpathlineto{\pgfqpoint{4.190251in}{1.237308in}}%
\pgfpathlineto{\pgfqpoint{4.191094in}{1.224173in}}%
\pgfpathlineto{\pgfqpoint{4.191515in}{1.224173in}}%
\pgfpathlineto{\pgfqpoint{4.191936in}{1.237308in}}%
\pgfpathlineto{\pgfqpoint{4.192357in}{1.224173in}}%
\pgfpathlineto{\pgfqpoint{4.192779in}{1.211038in}}%
\pgfpathlineto{\pgfqpoint{4.194043in}{1.237308in}}%
\pgfpathlineto{\pgfqpoint{4.194885in}{1.211038in}}%
\pgfpathlineto{\pgfqpoint{4.195306in}{1.224173in}}%
\pgfpathlineto{\pgfqpoint{4.197834in}{1.224173in}}%
\pgfpathlineto{\pgfqpoint{4.198255in}{1.237308in}}%
\pgfpathlineto{\pgfqpoint{4.198677in}{1.211038in}}%
\pgfpathlineto{\pgfqpoint{4.199098in}{1.224173in}}%
\pgfpathlineto{\pgfqpoint{4.199519in}{1.237308in}}%
\pgfpathlineto{\pgfqpoint{4.200783in}{1.211038in}}%
\pgfpathlineto{\pgfqpoint{4.202047in}{1.224173in}}%
\pgfpathlineto{\pgfqpoint{4.202889in}{1.224173in}}%
\pgfpathlineto{\pgfqpoint{4.203311in}{1.250443in}}%
\pgfpathlineto{\pgfqpoint{4.203732in}{1.197903in}}%
\pgfpathlineto{\pgfqpoint{4.204574in}{1.224173in}}%
\pgfpathlineto{\pgfqpoint{4.204996in}{1.211038in}}%
\pgfpathlineto{\pgfqpoint{4.205417in}{1.224173in}}%
\pgfpathlineto{\pgfqpoint{4.207523in}{1.224173in}}%
\pgfpathlineto{\pgfqpoint{4.207945in}{1.211038in}}%
\pgfpathlineto{\pgfqpoint{4.208366in}{1.237308in}}%
\pgfpathlineto{\pgfqpoint{4.209209in}{1.224173in}}%
\pgfpathlineto{\pgfqpoint{4.209630in}{1.224173in}}%
\pgfpathlineto{\pgfqpoint{4.210051in}{1.211038in}}%
\pgfpathlineto{\pgfqpoint{4.210472in}{1.224173in}}%
\pgfpathlineto{\pgfqpoint{4.211736in}{1.224173in}}%
\pgfpathlineto{\pgfqpoint{4.212157in}{1.250443in}}%
\pgfpathlineto{\pgfqpoint{4.212579in}{1.224173in}}%
\pgfpathlineto{\pgfqpoint{4.213000in}{1.211038in}}%
\pgfpathlineto{\pgfqpoint{4.214264in}{1.237308in}}%
\pgfpathlineto{\pgfqpoint{4.215106in}{1.211038in}}%
\pgfpathlineto{\pgfqpoint{4.215528in}{1.224173in}}%
\pgfpathlineto{\pgfqpoint{4.215949in}{1.237308in}}%
\pgfpathlineto{\pgfqpoint{4.216370in}{1.224173in}}%
\pgfpathlineto{\pgfqpoint{4.216792in}{1.211038in}}%
\pgfpathlineto{\pgfqpoint{4.217213in}{1.237308in}}%
\pgfpathlineto{\pgfqpoint{4.218055in}{1.224173in}}%
\pgfpathlineto{\pgfqpoint{4.219740in}{1.224173in}}%
\pgfpathlineto{\pgfqpoint{4.220162in}{1.211038in}}%
\pgfpathlineto{\pgfqpoint{4.220583in}{1.224173in}}%
\pgfpathlineto{\pgfqpoint{4.221004in}{1.224173in}}%
\pgfpathlineto{\pgfqpoint{4.221847in}{1.237308in}}%
\pgfpathlineto{\pgfqpoint{4.222689in}{1.224173in}}%
\pgfpathlineto{\pgfqpoint{4.223532in}{1.237308in}}%
\pgfpathlineto{\pgfqpoint{4.225217in}{1.211038in}}%
\pgfpathlineto{\pgfqpoint{4.225638in}{1.211038in}}%
\pgfpathlineto{\pgfqpoint{4.226902in}{1.224173in}}%
\pgfpathlineto{\pgfqpoint{4.227745in}{1.224173in}}%
\pgfpathlineto{\pgfqpoint{4.228587in}{1.237308in}}%
\pgfpathlineto{\pgfqpoint{4.230272in}{1.211038in}}%
\pgfpathlineto{\pgfqpoint{4.231958in}{1.237308in}}%
\pgfpathlineto{\pgfqpoint{4.233221in}{1.224173in}}%
\pgfpathlineto{\pgfqpoint{4.233643in}{1.237308in}}%
\pgfpathlineto{\pgfqpoint{4.234064in}{1.224173in}}%
\pgfpathlineto{\pgfqpoint{4.234485in}{1.224173in}}%
\pgfpathlineto{\pgfqpoint{4.234906in}{1.250443in}}%
\pgfpathlineto{\pgfqpoint{4.235328in}{1.197903in}}%
\pgfpathlineto{\pgfqpoint{4.236170in}{1.224173in}}%
\pgfpathlineto{\pgfqpoint{4.236592in}{1.211038in}}%
\pgfpathlineto{\pgfqpoint{4.237013in}{1.224173in}}%
\pgfpathlineto{\pgfqpoint{4.238277in}{1.224173in}}%
\pgfpathlineto{\pgfqpoint{4.238698in}{1.250443in}}%
\pgfpathlineto{\pgfqpoint{4.239119in}{1.224173in}}%
\pgfpathlineto{\pgfqpoint{4.239962in}{1.224173in}}%
\pgfpathlineto{\pgfqpoint{4.240383in}{1.211038in}}%
\pgfpathlineto{\pgfqpoint{4.240804in}{1.224173in}}%
\pgfpathlineto{\pgfqpoint{4.243332in}{1.224173in}}%
\pgfpathlineto{\pgfqpoint{4.243753in}{1.250443in}}%
\pgfpathlineto{\pgfqpoint{4.244175in}{1.224173in}}%
\pgfpathlineto{\pgfqpoint{4.244596in}{1.197903in}}%
\pgfpathlineto{\pgfqpoint{4.245017in}{1.237308in}}%
\pgfpathlineto{\pgfqpoint{4.245438in}{1.250443in}}%
\pgfpathlineto{\pgfqpoint{4.246281in}{1.211038in}}%
\pgfpathlineto{\pgfqpoint{4.246702in}{1.237308in}}%
\pgfpathlineto{\pgfqpoint{4.248387in}{1.211038in}}%
\pgfpathlineto{\pgfqpoint{4.248809in}{1.237308in}}%
\pgfpathlineto{\pgfqpoint{4.249230in}{1.197903in}}%
\pgfpathlineto{\pgfqpoint{4.249651in}{1.224173in}}%
\pgfpathlineto{\pgfqpoint{4.250073in}{1.224173in}}%
\pgfpathlineto{\pgfqpoint{4.250494in}{1.211038in}}%
\pgfpathlineto{\pgfqpoint{4.250915in}{1.224173in}}%
\pgfpathlineto{\pgfqpoint{4.253443in}{1.224173in}}%
\pgfpathlineto{\pgfqpoint{4.253864in}{1.237308in}}%
\pgfpathlineto{\pgfqpoint{4.254285in}{1.224173in}}%
\pgfpathlineto{\pgfqpoint{4.254707in}{1.224173in}}%
\pgfpathlineto{\pgfqpoint{4.255549in}{1.211038in}}%
\pgfpathlineto{\pgfqpoint{4.255970in}{1.250443in}}%
\pgfpathlineto{\pgfqpoint{4.257234in}{1.211038in}}%
\pgfpathlineto{\pgfqpoint{4.258077in}{1.237308in}}%
\pgfpathlineto{\pgfqpoint{4.258498in}{1.224173in}}%
\pgfpathlineto{\pgfqpoint{4.260183in}{1.224173in}}%
\pgfpathlineto{\pgfqpoint{4.261026in}{1.211038in}}%
\pgfpathlineto{\pgfqpoint{4.262711in}{1.250443in}}%
\pgfpathlineto{\pgfqpoint{4.263553in}{1.224173in}}%
\pgfpathlineto{\pgfqpoint{4.263975in}{1.237308in}}%
\pgfpathlineto{\pgfqpoint{4.264817in}{1.237308in}}%
\pgfpathlineto{\pgfqpoint{4.265660in}{1.211038in}}%
\pgfpathlineto{\pgfqpoint{4.266081in}{1.224173in}}%
\pgfpathlineto{\pgfqpoint{4.268187in}{1.224173in}}%
\pgfpathlineto{\pgfqpoint{4.268609in}{1.197903in}}%
\pgfpathlineto{\pgfqpoint{4.269030in}{1.237308in}}%
\pgfpathlineto{\pgfqpoint{4.269451in}{1.211038in}}%
\pgfpathlineto{\pgfqpoint{4.269873in}{1.237308in}}%
\pgfpathlineto{\pgfqpoint{4.270715in}{1.224173in}}%
\pgfpathlineto{\pgfqpoint{4.271136in}{1.237308in}}%
\pgfpathlineto{\pgfqpoint{4.271558in}{1.224173in}}%
\pgfpathlineto{\pgfqpoint{4.272400in}{1.211038in}}%
\pgfpathlineto{\pgfqpoint{4.273243in}{1.224173in}}%
\pgfpathlineto{\pgfqpoint{4.273664in}{1.211038in}}%
\pgfpathlineto{\pgfqpoint{4.274085in}{1.237308in}}%
\pgfpathlineto{\pgfqpoint{4.274928in}{1.224173in}}%
\pgfpathlineto{\pgfqpoint{4.275770in}{1.211038in}}%
\pgfpathlineto{\pgfqpoint{4.276192in}{1.237308in}}%
\pgfpathlineto{\pgfqpoint{4.277034in}{1.224173in}}%
\pgfpathlineto{\pgfqpoint{4.277456in}{1.224173in}}%
\pgfpathlineto{\pgfqpoint{4.278298in}{1.211038in}}%
\pgfpathlineto{\pgfqpoint{4.279141in}{1.237308in}}%
\pgfpathlineto{\pgfqpoint{4.279562in}{1.224173in}}%
\pgfpathlineto{\pgfqpoint{4.280405in}{1.224173in}}%
\pgfpathlineto{\pgfqpoint{4.280826in}{1.211038in}}%
\pgfpathlineto{\pgfqpoint{4.281247in}{1.224173in}}%
\pgfpathlineto{\pgfqpoint{4.281668in}{1.224173in}}%
\pgfpathlineto{\pgfqpoint{4.282090in}{1.237308in}}%
\pgfpathlineto{\pgfqpoint{4.282511in}{1.224173in}}%
\pgfpathlineto{\pgfqpoint{4.283353in}{1.211038in}}%
\pgfpathlineto{\pgfqpoint{4.283775in}{1.250443in}}%
\pgfpathlineto{\pgfqpoint{4.284196in}{1.224173in}}%
\pgfpathlineto{\pgfqpoint{4.284617in}{1.211038in}}%
\pgfpathlineto{\pgfqpoint{4.285039in}{1.224173in}}%
\pgfpathlineto{\pgfqpoint{4.285460in}{1.224173in}}%
\pgfpathlineto{\pgfqpoint{4.285881in}{1.211038in}}%
\pgfpathlineto{\pgfqpoint{4.286302in}{1.224173in}}%
\pgfpathlineto{\pgfqpoint{4.286724in}{1.250443in}}%
\pgfpathlineto{\pgfqpoint{4.287566in}{1.237308in}}%
\pgfpathlineto{\pgfqpoint{4.288409in}{1.211038in}}%
\pgfpathlineto{\pgfqpoint{4.288830in}{1.224173in}}%
\pgfpathlineto{\pgfqpoint{4.289251in}{1.237308in}}%
\pgfpathlineto{\pgfqpoint{4.289673in}{1.224173in}}%
\pgfpathlineto{\pgfqpoint{4.290515in}{1.224173in}}%
\pgfpathlineto{\pgfqpoint{4.290936in}{1.211038in}}%
\pgfpathlineto{\pgfqpoint{4.291358in}{1.237308in}}%
\pgfpathlineto{\pgfqpoint{4.292200in}{1.224173in}}%
\pgfpathlineto{\pgfqpoint{4.292622in}{1.211038in}}%
\pgfpathlineto{\pgfqpoint{4.293043in}{1.224173in}}%
\pgfpathlineto{\pgfqpoint{4.293885in}{1.224173in}}%
\pgfpathlineto{\pgfqpoint{4.294307in}{1.250443in}}%
\pgfpathlineto{\pgfqpoint{4.294728in}{1.224173in}}%
\pgfpathlineto{\pgfqpoint{4.295149in}{1.224173in}}%
\pgfpathlineto{\pgfqpoint{4.295571in}{1.211038in}}%
\pgfpathlineto{\pgfqpoint{4.295992in}{1.224173in}}%
\pgfpathlineto{\pgfqpoint{4.296834in}{1.224173in}}%
\pgfpathlineto{\pgfqpoint{4.297256in}{1.197903in}}%
\pgfpathlineto{\pgfqpoint{4.297677in}{1.224173in}}%
\pgfpathlineto{\pgfqpoint{4.298098in}{1.237308in}}%
\pgfpathlineto{\pgfqpoint{4.298519in}{1.224173in}}%
\pgfpathlineto{\pgfqpoint{4.299783in}{1.224173in}}%
\pgfpathlineto{\pgfqpoint{4.301047in}{1.197903in}}%
\pgfpathlineto{\pgfqpoint{4.302311in}{1.224173in}}%
\pgfpathlineto{\pgfqpoint{4.303154in}{1.211038in}}%
\pgfpathlineto{\pgfqpoint{4.304417in}{1.237308in}}%
\pgfpathlineto{\pgfqpoint{4.305681in}{1.211038in}}%
\pgfpathlineto{\pgfqpoint{4.306102in}{1.211038in}}%
\pgfpathlineto{\pgfqpoint{4.307366in}{1.224173in}}%
\pgfpathlineto{\pgfqpoint{4.308209in}{1.197903in}}%
\pgfpathlineto{\pgfqpoint{4.309473in}{1.224173in}}%
\pgfpathlineto{\pgfqpoint{4.310315in}{1.224173in}}%
\pgfpathlineto{\pgfqpoint{4.311579in}{1.197903in}}%
\pgfpathlineto{\pgfqpoint{4.313264in}{1.224173in}}%
\pgfpathlineto{\pgfqpoint{4.313685in}{1.224173in}}%
\pgfpathlineto{\pgfqpoint{4.314528in}{1.237308in}}%
\pgfpathlineto{\pgfqpoint{4.315792in}{1.224173in}}%
\pgfpathlineto{\pgfqpoint{4.316634in}{1.224173in}}%
\pgfpathlineto{\pgfqpoint{4.317056in}{1.211038in}}%
\pgfpathlineto{\pgfqpoint{4.317477in}{1.224173in}}%
\pgfpathlineto{\pgfqpoint{4.318320in}{1.224173in}}%
\pgfpathlineto{\pgfqpoint{4.318741in}{1.211038in}}%
\pgfpathlineto{\pgfqpoint{4.319583in}{1.237308in}}%
\pgfpathlineto{\pgfqpoint{4.320005in}{1.224173in}}%
\pgfpathlineto{\pgfqpoint{4.321269in}{1.224173in}}%
\pgfpathlineto{\pgfqpoint{4.321690in}{1.250443in}}%
\pgfpathlineto{\pgfqpoint{4.322111in}{1.224173in}}%
\pgfpathlineto{\pgfqpoint{4.322954in}{1.224173in}}%
\pgfpathlineto{\pgfqpoint{4.324217in}{1.237308in}}%
\pgfpathlineto{\pgfqpoint{4.325481in}{1.237308in}}%
\pgfpathlineto{\pgfqpoint{4.326745in}{1.224173in}}%
\pgfpathlineto{\pgfqpoint{4.329273in}{1.224173in}}%
\pgfpathlineto{\pgfqpoint{4.329694in}{1.250443in}}%
\pgfpathlineto{\pgfqpoint{4.330115in}{1.224173in}}%
\pgfpathlineto{\pgfqpoint{4.330958in}{1.224173in}}%
\pgfpathlineto{\pgfqpoint{4.331379in}{1.211038in}}%
\pgfpathlineto{\pgfqpoint{4.331800in}{1.224173in}}%
\pgfpathlineto{\pgfqpoint{4.332222in}{1.237308in}}%
\pgfpathlineto{\pgfqpoint{4.332643in}{1.224173in}}%
\pgfpathlineto{\pgfqpoint{4.336013in}{1.224173in}}%
\pgfpathlineto{\pgfqpoint{4.336856in}{1.211038in}}%
\pgfpathlineto{\pgfqpoint{4.338120in}{1.224173in}}%
\pgfpathlineto{\pgfqpoint{4.338541in}{1.224173in}}%
\pgfpathlineto{\pgfqpoint{4.338962in}{1.237308in}}%
\pgfpathlineto{\pgfqpoint{4.339383in}{1.224173in}}%
\pgfpathlineto{\pgfqpoint{4.341069in}{1.224173in}}%
\pgfpathlineto{\pgfqpoint{4.341490in}{1.211038in}}%
\pgfpathlineto{\pgfqpoint{4.341911in}{1.237308in}}%
\pgfpathlineto{\pgfqpoint{4.342754in}{1.224173in}}%
\pgfpathlineto{\pgfqpoint{4.343596in}{1.211038in}}%
\pgfpathlineto{\pgfqpoint{4.344860in}{1.250443in}}%
\pgfpathlineto{\pgfqpoint{4.346124in}{1.224173in}}%
\pgfpathlineto{\pgfqpoint{4.349073in}{1.224173in}}%
\pgfpathlineto{\pgfqpoint{4.349915in}{1.237308in}}%
\pgfpathlineto{\pgfqpoint{4.350758in}{1.224173in}}%
\pgfpathlineto{\pgfqpoint{4.351179in}{1.237308in}}%
\pgfpathlineto{\pgfqpoint{4.352443in}{1.211038in}}%
\pgfpathlineto{\pgfqpoint{4.353286in}{1.237308in}}%
\pgfpathlineto{\pgfqpoint{4.353707in}{1.224173in}}%
\pgfpathlineto{\pgfqpoint{4.354549in}{1.224173in}}%
\pgfpathlineto{\pgfqpoint{4.354971in}{1.237308in}}%
\pgfpathlineto{\pgfqpoint{4.355392in}{1.197903in}}%
\pgfpathlineto{\pgfqpoint{4.355813in}{1.224173in}}%
\pgfpathlineto{\pgfqpoint{4.356235in}{1.224173in}}%
\pgfpathlineto{\pgfqpoint{4.356656in}{1.211038in}}%
\pgfpathlineto{\pgfqpoint{4.357077in}{1.237308in}}%
\pgfpathlineto{\pgfqpoint{4.357920in}{1.224173in}}%
\pgfpathlineto{\pgfqpoint{4.358341in}{1.211038in}}%
\pgfpathlineto{\pgfqpoint{4.358762in}{1.224173in}}%
\pgfpathlineto{\pgfqpoint{4.359184in}{1.237308in}}%
\pgfpathlineto{\pgfqpoint{4.359605in}{1.224173in}}%
\pgfpathlineto{\pgfqpoint{4.361290in}{1.224173in}}%
\pgfpathlineto{\pgfqpoint{4.361711in}{1.211038in}}%
\pgfpathlineto{\pgfqpoint{4.362132in}{1.224173in}}%
\pgfpathlineto{\pgfqpoint{4.362554in}{1.224173in}}%
\pgfpathlineto{\pgfqpoint{4.362975in}{1.197903in}}%
\pgfpathlineto{\pgfqpoint{4.363396in}{1.224173in}}%
\pgfpathlineto{\pgfqpoint{4.364239in}{1.224173in}}%
\pgfpathlineto{\pgfqpoint{4.364660in}{1.211038in}}%
\pgfpathlineto{\pgfqpoint{4.365081in}{1.237308in}}%
\pgfpathlineto{\pgfqpoint{4.365924in}{1.224173in}}%
\pgfpathlineto{\pgfqpoint{4.366345in}{1.224173in}}%
\pgfpathlineto{\pgfqpoint{4.366767in}{1.197903in}}%
\pgfpathlineto{\pgfqpoint{4.367188in}{1.224173in}}%
\pgfpathlineto{\pgfqpoint{4.367609in}{1.224173in}}%
\pgfpathlineto{\pgfqpoint{4.368030in}{1.211038in}}%
\pgfpathlineto{\pgfqpoint{4.368452in}{1.224173in}}%
\pgfpathlineto{\pgfqpoint{4.368873in}{1.237308in}}%
\pgfpathlineto{\pgfqpoint{4.369294in}{1.224173in}}%
\pgfpathlineto{\pgfqpoint{4.369715in}{1.224173in}}%
\pgfpathlineto{\pgfqpoint{4.370137in}{1.237308in}}%
\pgfpathlineto{\pgfqpoint{4.370558in}{1.224173in}}%
\pgfpathlineto{\pgfqpoint{4.370979in}{1.224173in}}%
\pgfpathlineto{\pgfqpoint{4.371401in}{1.250443in}}%
\pgfpathlineto{\pgfqpoint{4.371822in}{1.224173in}}%
\pgfpathlineto{\pgfqpoint{4.372243in}{1.224173in}}%
\pgfpathlineto{\pgfqpoint{4.372664in}{1.211038in}}%
\pgfpathlineto{\pgfqpoint{4.373086in}{1.224173in}}%
\pgfpathlineto{\pgfqpoint{4.374771in}{1.224173in}}%
\pgfpathlineto{\pgfqpoint{4.375192in}{1.237308in}}%
\pgfpathlineto{\pgfqpoint{4.375613in}{1.224173in}}%
\pgfpathlineto{\pgfqpoint{4.376035in}{1.224173in}}%
\pgfpathlineto{\pgfqpoint{4.376456in}{1.237308in}}%
\pgfpathlineto{\pgfqpoint{4.376877in}{1.211038in}}%
\pgfpathlineto{\pgfqpoint{4.377720in}{1.224173in}}%
\pgfpathlineto{\pgfqpoint{4.378562in}{1.224173in}}%
\pgfpathlineto{\pgfqpoint{4.378984in}{1.237308in}}%
\pgfpathlineto{\pgfqpoint{4.379405in}{1.224173in}}%
\pgfpathlineto{\pgfqpoint{4.379826in}{1.224173in}}%
\pgfpathlineto{\pgfqpoint{4.380247in}{1.237308in}}%
\pgfpathlineto{\pgfqpoint{4.380669in}{1.224173in}}%
\pgfpathlineto{\pgfqpoint{4.381090in}{1.224173in}}%
\pgfpathlineto{\pgfqpoint{4.381511in}{1.237308in}}%
\pgfpathlineto{\pgfqpoint{4.381933in}{1.211038in}}%
\pgfpathlineto{\pgfqpoint{4.382775in}{1.224173in}}%
\pgfpathlineto{\pgfqpoint{4.383196in}{1.237308in}}%
\pgfpathlineto{\pgfqpoint{4.383618in}{1.224173in}}%
\pgfpathlineto{\pgfqpoint{4.384460in}{1.224173in}}%
\pgfpathlineto{\pgfqpoint{4.384882in}{1.197903in}}%
\pgfpathlineto{\pgfqpoint{4.385724in}{1.211038in}}%
\pgfpathlineto{\pgfqpoint{4.386567in}{1.224173in}}%
\pgfpathlineto{\pgfqpoint{4.386988in}{1.211038in}}%
\pgfpathlineto{\pgfqpoint{4.387409in}{1.224173in}}%
\pgfpathlineto{\pgfqpoint{4.387830in}{1.237308in}}%
\pgfpathlineto{\pgfqpoint{4.388252in}{1.224173in}}%
\pgfpathlineto{\pgfqpoint{4.389937in}{1.224173in}}%
\pgfpathlineto{\pgfqpoint{4.390358in}{1.237308in}}%
\pgfpathlineto{\pgfqpoint{4.390779in}{1.224173in}}%
\pgfpathlineto{\pgfqpoint{4.391622in}{1.224173in}}%
\pgfpathlineto{\pgfqpoint{4.392043in}{1.211038in}}%
\pgfpathlineto{\pgfqpoint{4.392465in}{1.224173in}}%
\pgfpathlineto{\pgfqpoint{4.394992in}{1.224173in}}%
\pgfpathlineto{\pgfqpoint{4.395413in}{1.250443in}}%
\pgfpathlineto{\pgfqpoint{4.395835in}{1.224173in}}%
\pgfpathlineto{\pgfqpoint{4.396256in}{1.224173in}}%
\pgfpathlineto{\pgfqpoint{4.396677in}{1.211038in}}%
\pgfpathlineto{\pgfqpoint{4.397099in}{1.224173in}}%
\pgfpathlineto{\pgfqpoint{4.397520in}{1.224173in}}%
\pgfpathlineto{\pgfqpoint{4.397941in}{1.237308in}}%
\pgfpathlineto{\pgfqpoint{4.398362in}{1.224173in}}%
\pgfpathlineto{\pgfqpoint{4.398784in}{1.224173in}}%
\pgfpathlineto{\pgfqpoint{4.399205in}{1.237308in}}%
\pgfpathlineto{\pgfqpoint{4.399626in}{1.224173in}}%
\pgfpathlineto{\pgfqpoint{4.400048in}{1.224173in}}%
\pgfpathlineto{\pgfqpoint{4.400469in}{1.237308in}}%
\pgfpathlineto{\pgfqpoint{4.400890in}{1.224173in}}%
\pgfpathlineto{\pgfqpoint{4.401311in}{1.211038in}}%
\pgfpathlineto{\pgfqpoint{4.401733in}{1.224173in}}%
\pgfpathlineto{\pgfqpoint{4.405103in}{1.224173in}}%
\pgfpathlineto{\pgfqpoint{4.405524in}{1.237308in}}%
\pgfpathlineto{\pgfqpoint{4.405945in}{1.211038in}}%
\pgfpathlineto{\pgfqpoint{4.406788in}{1.224173in}}%
\pgfpathlineto{\pgfqpoint{4.407631in}{1.211038in}}%
\pgfpathlineto{\pgfqpoint{4.408052in}{1.237308in}}%
\pgfpathlineto{\pgfqpoint{4.408473in}{1.224173in}}%
\pgfpathlineto{\pgfqpoint{4.408894in}{1.211038in}}%
\pgfpathlineto{\pgfqpoint{4.409316in}{1.224173in}}%
\pgfpathlineto{\pgfqpoint{4.409737in}{1.224173in}}%
\pgfpathlineto{\pgfqpoint{4.410158in}{1.237308in}}%
\pgfpathlineto{\pgfqpoint{4.410579in}{1.224173in}}%
\pgfpathlineto{\pgfqpoint{4.411001in}{1.224173in}}%
\pgfpathlineto{\pgfqpoint{4.411422in}{1.211038in}}%
\pgfpathlineto{\pgfqpoint{4.411843in}{1.237308in}}%
\pgfpathlineto{\pgfqpoint{4.412686in}{1.224173in}}%
\pgfpathlineto{\pgfqpoint{4.414792in}{1.224173in}}%
\pgfpathlineto{\pgfqpoint{4.415214in}{1.211038in}}%
\pgfpathlineto{\pgfqpoint{4.415635in}{1.250443in}}%
\pgfpathlineto{\pgfqpoint{4.416056in}{1.224173in}}%
\pgfpathlineto{\pgfqpoint{4.417320in}{1.211038in}}%
\pgfpathlineto{\pgfqpoint{4.418584in}{1.224173in}}%
\pgfpathlineto{\pgfqpoint{4.420269in}{1.224173in}}%
\pgfpathlineto{\pgfqpoint{4.421111in}{1.237308in}}%
\pgfpathlineto{\pgfqpoint{4.421533in}{1.211038in}}%
\pgfpathlineto{\pgfqpoint{4.421954in}{1.250443in}}%
\pgfpathlineto{\pgfqpoint{4.422375in}{1.224173in}}%
\pgfpathlineto{\pgfqpoint{4.423218in}{1.224173in}}%
\pgfpathlineto{\pgfqpoint{4.423639in}{1.211038in}}%
\pgfpathlineto{\pgfqpoint{4.424060in}{1.224173in}}%
\pgfpathlineto{\pgfqpoint{4.424903in}{1.224173in}}%
\pgfpathlineto{\pgfqpoint{4.425324in}{1.211038in}}%
\pgfpathlineto{\pgfqpoint{4.425745in}{1.237308in}}%
\pgfpathlineto{\pgfqpoint{4.426588in}{1.224173in}}%
\pgfpathlineto{\pgfqpoint{4.427009in}{1.224173in}}%
\pgfpathlineto{\pgfqpoint{4.427431in}{1.197903in}}%
\pgfpathlineto{\pgfqpoint{4.427852in}{1.224173in}}%
\pgfpathlineto{\pgfqpoint{4.428694in}{1.237308in}}%
\pgfpathlineto{\pgfqpoint{4.429116in}{1.197903in}}%
\pgfpathlineto{\pgfqpoint{4.429537in}{1.224173in}}%
\pgfpathlineto{\pgfqpoint{4.430801in}{1.237308in}}%
\pgfpathlineto{\pgfqpoint{4.431643in}{1.224173in}}%
\pgfpathlineto{\pgfqpoint{4.432065in}{1.237308in}}%
\pgfpathlineto{\pgfqpoint{4.432486in}{1.211038in}}%
\pgfpathlineto{\pgfqpoint{4.433328in}{1.224173in}}%
\pgfpathlineto{\pgfqpoint{4.434171in}{1.224173in}}%
\pgfpathlineto{\pgfqpoint{4.435435in}{1.197903in}}%
\pgfpathlineto{\pgfqpoint{4.437120in}{1.237308in}}%
\pgfpathlineto{\pgfqpoint{4.437541in}{1.211038in}}%
\pgfpathlineto{\pgfqpoint{4.438384in}{1.224173in}}%
\pgfpathlineto{\pgfqpoint{4.439226in}{1.224173in}}%
\pgfpathlineto{\pgfqpoint{4.439648in}{1.250443in}}%
\pgfpathlineto{\pgfqpoint{4.440069in}{1.211038in}}%
\pgfpathlineto{\pgfqpoint{4.440911in}{1.250443in}}%
\pgfpathlineto{\pgfqpoint{4.441333in}{1.224173in}}%
\pgfpathlineto{\pgfqpoint{4.442175in}{1.224173in}}%
\pgfpathlineto{\pgfqpoint{4.442597in}{1.211038in}}%
\pgfpathlineto{\pgfqpoint{4.443018in}{1.224173in}}%
\pgfpathlineto{\pgfqpoint{4.443439in}{1.237308in}}%
\pgfpathlineto{\pgfqpoint{4.443860in}{1.224173in}}%
\pgfpathlineto{\pgfqpoint{4.444282in}{1.224173in}}%
\pgfpathlineto{\pgfqpoint{4.444703in}{1.237308in}}%
\pgfpathlineto{\pgfqpoint{4.445124in}{1.224173in}}%
\pgfpathlineto{\pgfqpoint{4.445546in}{1.211038in}}%
\pgfpathlineto{\pgfqpoint{4.445967in}{1.237308in}}%
\pgfpathlineto{\pgfqpoint{4.446809in}{1.224173in}}%
\pgfpathlineto{\pgfqpoint{4.448073in}{1.250443in}}%
\pgfpathlineto{\pgfqpoint{4.449337in}{1.224173in}}%
\pgfpathlineto{\pgfqpoint{4.450180in}{1.224173in}}%
\pgfpathlineto{\pgfqpoint{4.450601in}{1.211038in}}%
\pgfpathlineto{\pgfqpoint{4.451022in}{1.237308in}}%
\pgfpathlineto{\pgfqpoint{4.451865in}{1.224173in}}%
\pgfpathlineto{\pgfqpoint{4.452286in}{1.237308in}}%
\pgfpathlineto{\pgfqpoint{4.453129in}{1.211038in}}%
\pgfpathlineto{\pgfqpoint{4.453550in}{1.224173in}}%
\pgfpathlineto{\pgfqpoint{4.454814in}{1.224173in}}%
\pgfpathlineto{\pgfqpoint{4.455235in}{1.211038in}}%
\pgfpathlineto{\pgfqpoint{4.455656in}{1.224173in}}%
\pgfpathlineto{\pgfqpoint{4.456078in}{1.250443in}}%
\pgfpathlineto{\pgfqpoint{4.456499in}{1.224173in}}%
\pgfpathlineto{\pgfqpoint{4.457763in}{1.211038in}}%
\pgfpathlineto{\pgfqpoint{4.458184in}{1.211038in}}%
\pgfpathlineto{\pgfqpoint{4.459448in}{1.224173in}}%
\pgfpathlineto{\pgfqpoint{4.460290in}{1.224173in}}%
\pgfpathlineto{\pgfqpoint{4.460712in}{1.211038in}}%
\pgfpathlineto{\pgfqpoint{4.461133in}{1.224173in}}%
\pgfpathlineto{\pgfqpoint{4.462397in}{1.237308in}}%
\pgfpathlineto{\pgfqpoint{4.463239in}{1.211038in}}%
\pgfpathlineto{\pgfqpoint{4.463661in}{1.224173in}}%
\pgfpathlineto{\pgfqpoint{4.464924in}{1.224173in}}%
\pgfpathlineto{\pgfqpoint{4.465767in}{1.197903in}}%
\pgfpathlineto{\pgfqpoint{4.467031in}{1.237308in}}%
\pgfpathlineto{\pgfqpoint{4.467873in}{1.211038in}}%
\pgfpathlineto{\pgfqpoint{4.468295in}{1.237308in}}%
\pgfpathlineto{\pgfqpoint{4.469137in}{1.224173in}}%
\pgfpathlineto{\pgfqpoint{4.469558in}{1.211038in}}%
\pgfpathlineto{\pgfqpoint{4.469980in}{1.224173in}}%
\pgfpathlineto{\pgfqpoint{4.471244in}{1.237308in}}%
\pgfpathlineto{\pgfqpoint{4.472507in}{1.224173in}}%
\pgfpathlineto{\pgfqpoint{4.474614in}{1.224173in}}%
\pgfpathlineto{\pgfqpoint{4.475456in}{1.211038in}}%
\pgfpathlineto{\pgfqpoint{4.476720in}{1.237308in}}%
\pgfpathlineto{\pgfqpoint{4.477984in}{1.211038in}}%
\pgfpathlineto{\pgfqpoint{4.479669in}{1.237308in}}%
\pgfpathlineto{\pgfqpoint{4.480933in}{1.224173in}}%
\pgfpathlineto{\pgfqpoint{4.482197in}{1.237308in}}%
\pgfpathlineto{\pgfqpoint{4.483882in}{1.211038in}}%
\pgfpathlineto{\pgfqpoint{4.485146in}{1.224173in}}%
\pgfpathlineto{\pgfqpoint{4.488516in}{1.224173in}}%
\pgfpathlineto{\pgfqpoint{4.488937in}{1.237308in}}%
\pgfpathlineto{\pgfqpoint{4.489358in}{1.197903in}}%
\pgfpathlineto{\pgfqpoint{4.490201in}{1.211038in}}%
\pgfpathlineto{\pgfqpoint{4.491886in}{1.250443in}}%
\pgfpathlineto{\pgfqpoint{4.493150in}{1.211038in}}%
\pgfpathlineto{\pgfqpoint{4.493993in}{1.237308in}}%
\pgfpathlineto{\pgfqpoint{4.494414in}{1.224173in}}%
\pgfpathlineto{\pgfqpoint{4.494835in}{1.224173in}}%
\pgfpathlineto{\pgfqpoint{4.495678in}{1.211038in}}%
\pgfpathlineto{\pgfqpoint{4.497363in}{1.250443in}}%
\pgfpathlineto{\pgfqpoint{4.497784in}{1.250443in}}%
\pgfpathlineto{\pgfqpoint{4.499048in}{1.224173in}}%
\pgfpathlineto{\pgfqpoint{4.502418in}{1.224173in}}%
\pgfpathlineto{\pgfqpoint{4.502839in}{1.237308in}}%
\pgfpathlineto{\pgfqpoint{4.503261in}{1.211038in}}%
\pgfpathlineto{\pgfqpoint{4.504103in}{1.224173in}}%
\pgfpathlineto{\pgfqpoint{4.504946in}{1.224173in}}%
\pgfpathlineto{\pgfqpoint{4.505367in}{1.197903in}}%
\pgfpathlineto{\pgfqpoint{4.506210in}{1.211038in}}%
\pgfpathlineto{\pgfqpoint{4.506631in}{1.237308in}}%
\pgfpathlineto{\pgfqpoint{4.507052in}{1.211038in}}%
\pgfpathlineto{\pgfqpoint{4.507473in}{1.211038in}}%
\pgfpathlineto{\pgfqpoint{4.508737in}{1.237308in}}%
\pgfpathlineto{\pgfqpoint{4.509580in}{1.211038in}}%
\pgfpathlineto{\pgfqpoint{4.510001in}{1.224173in}}%
\pgfpathlineto{\pgfqpoint{4.510422in}{1.224173in}}%
\pgfpathlineto{\pgfqpoint{4.511265in}{1.197903in}}%
\pgfpathlineto{\pgfqpoint{4.512529in}{1.224173in}}%
\pgfpathlineto{\pgfqpoint{4.513371in}{1.211038in}}%
\pgfpathlineto{\pgfqpoint{4.513793in}{1.237308in}}%
\pgfpathlineto{\pgfqpoint{4.514635in}{1.224173in}}%
\pgfpathlineto{\pgfqpoint{4.515899in}{1.224173in}}%
\pgfpathlineto{\pgfqpoint{4.516742in}{1.237308in}}%
\pgfpathlineto{\pgfqpoint{4.518427in}{1.211038in}}%
\pgfpathlineto{\pgfqpoint{4.519691in}{1.237308in}}%
\pgfpathlineto{\pgfqpoint{4.520954in}{1.224173in}}%
\pgfpathlineto{\pgfqpoint{4.521376in}{1.237308in}}%
\pgfpathlineto{\pgfqpoint{4.521797in}{1.224173in}}%
\pgfpathlineto{\pgfqpoint{4.522218in}{1.211038in}}%
\pgfpathlineto{\pgfqpoint{4.522639in}{1.224173in}}%
\pgfpathlineto{\pgfqpoint{4.523482in}{1.224173in}}%
\pgfpathlineto{\pgfqpoint{4.523903in}{1.237308in}}%
\pgfpathlineto{\pgfqpoint{4.524325in}{1.224173in}}%
\pgfpathlineto{\pgfqpoint{4.525167in}{1.224173in}}%
\pgfpathlineto{\pgfqpoint{4.525588in}{1.211038in}}%
\pgfpathlineto{\pgfqpoint{4.526010in}{1.224173in}}%
\pgfpathlineto{\pgfqpoint{4.526852in}{1.237308in}}%
\pgfpathlineto{\pgfqpoint{4.527274in}{1.211038in}}%
\pgfpathlineto{\pgfqpoint{4.528116in}{1.224173in}}%
\pgfpathlineto{\pgfqpoint{4.528537in}{1.211038in}}%
\pgfpathlineto{\pgfqpoint{4.528959in}{1.237308in}}%
\pgfpathlineto{\pgfqpoint{4.529380in}{1.224173in}}%
\pgfpathlineto{\pgfqpoint{4.529801in}{1.197903in}}%
\pgfpathlineto{\pgfqpoint{4.530222in}{1.211038in}}%
\pgfpathlineto{\pgfqpoint{4.531486in}{1.224173in}}%
\pgfpathlineto{\pgfqpoint{4.532329in}{1.224173in}}%
\pgfpathlineto{\pgfqpoint{4.532750in}{1.237308in}}%
\pgfpathlineto{\pgfqpoint{4.533171in}{1.224173in}}%
\pgfpathlineto{\pgfqpoint{4.534435in}{1.211038in}}%
\pgfpathlineto{\pgfqpoint{4.534857in}{1.211038in}}%
\pgfpathlineto{\pgfqpoint{4.536120in}{1.224173in}}%
\pgfpathlineto{\pgfqpoint{4.536963in}{1.224173in}}%
\pgfpathlineto{\pgfqpoint{4.537384in}{1.211038in}}%
\pgfpathlineto{\pgfqpoint{4.537805in}{1.237308in}}%
\pgfpathlineto{\pgfqpoint{4.538648in}{1.224173in}}%
\pgfpathlineto{\pgfqpoint{4.539912in}{1.237308in}}%
\pgfpathlineto{\pgfqpoint{4.541176in}{1.224173in}}%
\pgfpathlineto{\pgfqpoint{4.541597in}{1.224173in}}%
\pgfpathlineto{\pgfqpoint{4.542018in}{1.237308in}}%
\pgfpathlineto{\pgfqpoint{4.542440in}{1.211038in}}%
\pgfpathlineto{\pgfqpoint{4.542861in}{1.224173in}}%
\pgfpathlineto{\pgfqpoint{4.543282in}{1.237308in}}%
\pgfpathlineto{\pgfqpoint{4.543703in}{1.224173in}}%
\pgfpathlineto{\pgfqpoint{4.544967in}{1.224173in}}%
\pgfpathlineto{\pgfqpoint{4.545388in}{1.237308in}}%
\pgfpathlineto{\pgfqpoint{4.545810in}{1.224173in}}%
\pgfpathlineto{\pgfqpoint{4.546231in}{1.211038in}}%
\pgfpathlineto{\pgfqpoint{4.546652in}{1.224173in}}%
\pgfpathlineto{\pgfqpoint{4.547916in}{1.237308in}}%
\pgfpathlineto{\pgfqpoint{4.549180in}{1.224173in}}%
\pgfpathlineto{\pgfqpoint{4.550023in}{1.224173in}}%
\pgfpathlineto{\pgfqpoint{4.551286in}{1.211038in}}%
\pgfpathlineto{\pgfqpoint{4.552550in}{1.224173in}}%
\pgfpathlineto{\pgfqpoint{4.552971in}{1.224173in}}%
\pgfpathlineto{\pgfqpoint{4.553814in}{1.211038in}}%
\pgfpathlineto{\pgfqpoint{4.555078in}{1.224173in}}%
\pgfpathlineto{\pgfqpoint{4.556342in}{1.224173in}}%
\pgfpathlineto{\pgfqpoint{4.557184in}{1.237308in}}%
\pgfpathlineto{\pgfqpoint{4.558869in}{1.211038in}}%
\pgfpathlineto{\pgfqpoint{4.560133in}{1.224173in}}%
\pgfpathlineto{\pgfqpoint{4.563503in}{1.224173in}}%
\pgfpathlineto{\pgfqpoint{4.564346in}{1.237308in}}%
\pgfpathlineto{\pgfqpoint{4.565610in}{1.224173in}}%
\pgfpathlineto{\pgfqpoint{4.566874in}{1.224173in}}%
\pgfpathlineto{\pgfqpoint{4.567295in}{1.237308in}}%
\pgfpathlineto{\pgfqpoint{4.567716in}{1.224173in}}%
\pgfpathlineto{\pgfqpoint{4.568559in}{1.224173in}}%
\pgfpathlineto{\pgfqpoint{4.568980in}{1.237308in}}%
\pgfpathlineto{\pgfqpoint{4.569401in}{1.224173in}}%
\pgfpathlineto{\pgfqpoint{4.570244in}{1.197903in}}%
\pgfpathlineto{\pgfqpoint{4.571508in}{1.224173in}}%
\pgfpathlineto{\pgfqpoint{4.571929in}{1.224173in}}%
\pgfpathlineto{\pgfqpoint{4.572350in}{1.237308in}}%
\pgfpathlineto{\pgfqpoint{4.572772in}{1.224173in}}%
\pgfpathlineto{\pgfqpoint{4.575720in}{1.224173in}}%
\pgfpathlineto{\pgfqpoint{4.576142in}{1.211038in}}%
\pgfpathlineto{\pgfqpoint{4.576563in}{1.224173in}}%
\pgfpathlineto{\pgfqpoint{4.578669in}{1.224173in}}%
\pgfpathlineto{\pgfqpoint{4.579091in}{1.211038in}}%
\pgfpathlineto{\pgfqpoint{4.579512in}{1.224173in}}%
\pgfpathlineto{\pgfqpoint{4.582040in}{1.224173in}}%
\pgfpathlineto{\pgfqpoint{4.582461in}{1.211038in}}%
\pgfpathlineto{\pgfqpoint{4.582882in}{1.224173in}}%
\pgfpathlineto{\pgfqpoint{4.583304in}{1.250443in}}%
\pgfpathlineto{\pgfqpoint{4.583725in}{1.237308in}}%
\pgfpathlineto{\pgfqpoint{4.584146in}{1.211038in}}%
\pgfpathlineto{\pgfqpoint{4.584989in}{1.224173in}}%
\pgfpathlineto{\pgfqpoint{4.588359in}{1.224173in}}%
\pgfpathlineto{\pgfqpoint{4.588780in}{1.211038in}}%
\pgfpathlineto{\pgfqpoint{4.589201in}{1.224173in}}%
\pgfpathlineto{\pgfqpoint{4.590044in}{1.224173in}}%
\pgfpathlineto{\pgfqpoint{4.590465in}{1.211038in}}%
\pgfpathlineto{\pgfqpoint{4.590887in}{1.224173in}}%
\pgfpathlineto{\pgfqpoint{4.592150in}{1.224173in}}%
\pgfpathlineto{\pgfqpoint{4.592572in}{1.237308in}}%
\pgfpathlineto{\pgfqpoint{4.593835in}{1.211038in}}%
\pgfpathlineto{\pgfqpoint{4.594257in}{1.211038in}}%
\pgfpathlineto{\pgfqpoint{4.595521in}{1.237308in}}%
\pgfpathlineto{\pgfqpoint{4.596784in}{1.211038in}}%
\pgfpathlineto{\pgfqpoint{4.597627in}{1.237308in}}%
\pgfpathlineto{\pgfqpoint{4.598048in}{1.224173in}}%
\pgfpathlineto{\pgfqpoint{4.599733in}{1.224173in}}%
\pgfpathlineto{\pgfqpoint{4.600155in}{1.237308in}}%
\pgfpathlineto{\pgfqpoint{4.600576in}{1.224173in}}%
\pgfpathlineto{\pgfqpoint{4.601840in}{1.224173in}}%
\pgfpathlineto{\pgfqpoint{4.603104in}{1.237308in}}%
\pgfpathlineto{\pgfqpoint{4.604367in}{1.211038in}}%
\pgfpathlineto{\pgfqpoint{4.605631in}{1.224173in}}%
\pgfpathlineto{\pgfqpoint{4.606053in}{1.224173in}}%
\pgfpathlineto{\pgfqpoint{4.607316in}{1.250443in}}%
\pgfpathlineto{\pgfqpoint{4.609001in}{1.211038in}}%
\pgfpathlineto{\pgfqpoint{4.609423in}{1.211038in}}%
\pgfpathlineto{\pgfqpoint{4.610687in}{1.224173in}}%
\pgfpathlineto{\pgfqpoint{4.612372in}{1.224173in}}%
\pgfpathlineto{\pgfqpoint{4.612793in}{1.211038in}}%
\pgfpathlineto{\pgfqpoint{4.613214in}{1.237308in}}%
\pgfpathlineto{\pgfqpoint{4.613636in}{1.197903in}}%
\pgfpathlineto{\pgfqpoint{4.614057in}{1.224173in}}%
\pgfpathlineto{\pgfqpoint{4.617427in}{1.224173in}}%
\pgfpathlineto{\pgfqpoint{4.617848in}{1.250443in}}%
\pgfpathlineto{\pgfqpoint{4.618691in}{1.237308in}}%
\pgfpathlineto{\pgfqpoint{4.619533in}{1.211038in}}%
\pgfpathlineto{\pgfqpoint{4.619955in}{1.224173in}}%
\pgfpathlineto{\pgfqpoint{4.620797in}{1.224173in}}%
\pgfpathlineto{\pgfqpoint{4.621219in}{1.237308in}}%
\pgfpathlineto{\pgfqpoint{4.621640in}{1.211038in}}%
\pgfpathlineto{\pgfqpoint{4.622482in}{1.224173in}}%
\pgfpathlineto{\pgfqpoint{4.623746in}{1.250443in}}%
\pgfpathlineto{\pgfqpoint{4.624589in}{1.211038in}}%
\pgfpathlineto{\pgfqpoint{4.625010in}{1.224173in}}%
\pgfpathlineto{\pgfqpoint{4.625853in}{1.224173in}}%
\pgfpathlineto{\pgfqpoint{4.626274in}{1.237308in}}%
\pgfpathlineto{\pgfqpoint{4.626695in}{1.211038in}}%
\pgfpathlineto{\pgfqpoint{4.627538in}{1.224173in}}%
\pgfpathlineto{\pgfqpoint{4.629223in}{1.224173in}}%
\pgfpathlineto{\pgfqpoint{4.630487in}{1.211038in}}%
\pgfpathlineto{\pgfqpoint{4.631750in}{1.224173in}}%
\pgfpathlineto{\pgfqpoint{4.632593in}{1.224173in}}%
\pgfpathlineto{\pgfqpoint{4.633014in}{1.237308in}}%
\pgfpathlineto{\pgfqpoint{4.633436in}{1.224173in}}%
\pgfpathlineto{\pgfqpoint{4.633857in}{1.224173in}}%
\pgfpathlineto{\pgfqpoint{4.634699in}{1.211038in}}%
\pgfpathlineto{\pgfqpoint{4.636385in}{1.237308in}}%
\pgfpathlineto{\pgfqpoint{4.637227in}{1.211038in}}%
\pgfpathlineto{\pgfqpoint{4.638491in}{1.237308in}}%
\pgfpathlineto{\pgfqpoint{4.638912in}{1.237308in}}%
\pgfpathlineto{\pgfqpoint{4.640597in}{1.211038in}}%
\pgfpathlineto{\pgfqpoint{4.641861in}{1.224173in}}%
\pgfpathlineto{\pgfqpoint{4.642282in}{1.224173in}}%
\pgfpathlineto{\pgfqpoint{4.643546in}{1.250443in}}%
\pgfpathlineto{\pgfqpoint{4.645653in}{1.211038in}}%
\pgfpathlineto{\pgfqpoint{4.647338in}{1.250443in}}%
\pgfpathlineto{\pgfqpoint{4.648602in}{1.224173in}}%
\pgfpathlineto{\pgfqpoint{4.649444in}{1.224173in}}%
\pgfpathlineto{\pgfqpoint{4.649865in}{1.197903in}}%
\pgfpathlineto{\pgfqpoint{4.650287in}{1.211038in}}%
\pgfpathlineto{\pgfqpoint{4.651551in}{1.224173in}}%
\pgfpathlineto{\pgfqpoint{4.651972in}{1.197903in}}%
\pgfpathlineto{\pgfqpoint{4.652393in}{1.211038in}}%
\pgfpathlineto{\pgfqpoint{4.654078in}{1.237308in}}%
\pgfpathlineto{\pgfqpoint{4.654921in}{1.211038in}}%
\pgfpathlineto{\pgfqpoint{4.655342in}{1.224173in}}%
\pgfpathlineto{\pgfqpoint{4.655763in}{1.224173in}}%
\pgfpathlineto{\pgfqpoint{4.657027in}{1.197903in}}%
\pgfpathlineto{\pgfqpoint{4.657448in}{1.224173in}}%
\pgfpathlineto{\pgfqpoint{4.658291in}{1.211038in}}%
\pgfpathlineto{\pgfqpoint{4.659555in}{1.224173in}}%
\pgfpathlineto{\pgfqpoint{4.659976in}{0.987741in}}%
\pgfpathlineto{\pgfqpoint{4.660397in}{1.224173in}}%
\pgfpathlineto{\pgfqpoint{4.660819in}{1.197903in}}%
\pgfpathlineto{\pgfqpoint{4.661240in}{1.224173in}}%
\pgfpathlineto{\pgfqpoint{4.662925in}{1.224173in}}%
\pgfpathlineto{\pgfqpoint{4.663768in}{1.237308in}}%
\pgfpathlineto{\pgfqpoint{4.665031in}{1.224173in}}%
\pgfpathlineto{\pgfqpoint{4.665453in}{1.237308in}}%
\pgfpathlineto{\pgfqpoint{4.665874in}{1.224173in}}%
\pgfpathlineto{\pgfqpoint{4.666295in}{1.224173in}}%
\pgfpathlineto{\pgfqpoint{4.666717in}{1.237308in}}%
\pgfpathlineto{\pgfqpoint{4.667138in}{1.224173in}}%
\pgfpathlineto{\pgfqpoint{4.667559in}{1.224173in}}%
\pgfpathlineto{\pgfqpoint{4.668402in}{1.237308in}}%
\pgfpathlineto{\pgfqpoint{4.670508in}{1.171633in}}%
\pgfpathlineto{\pgfqpoint{4.671772in}{1.237308in}}%
\pgfpathlineto{\pgfqpoint{4.672193in}{1.224173in}}%
\pgfpathlineto{\pgfqpoint{4.673457in}{1.460605in}}%
\pgfpathlineto{\pgfqpoint{4.673878in}{1.224173in}}%
\pgfpathlineto{\pgfqpoint{4.674721in}{1.237308in}}%
\pgfpathlineto{\pgfqpoint{4.675142in}{1.237308in}}%
\pgfpathlineto{\pgfqpoint{4.675563in}{1.211038in}}%
\pgfpathlineto{\pgfqpoint{4.676406in}{1.224173in}}%
\pgfpathlineto{\pgfqpoint{4.677670in}{1.224173in}}%
\pgfpathlineto{\pgfqpoint{4.678512in}{1.250443in}}%
\pgfpathlineto{\pgfqpoint{4.680197in}{1.211038in}}%
\pgfpathlineto{\pgfqpoint{4.681461in}{1.224173in}}%
\pgfpathlineto{\pgfqpoint{4.683146in}{1.224173in}}%
\pgfpathlineto{\pgfqpoint{4.683989in}{1.276714in}}%
\pgfpathlineto{\pgfqpoint{4.684410in}{1.237308in}}%
\pgfpathlineto{\pgfqpoint{4.684832in}{1.237308in}}%
\pgfpathlineto{\pgfqpoint{4.686095in}{1.171633in}}%
\pgfpathlineto{\pgfqpoint{4.687359in}{1.224173in}}%
\pgfpathlineto{\pgfqpoint{4.688202in}{1.211038in}}%
\pgfpathlineto{\pgfqpoint{4.689044in}{1.263578in}}%
\pgfpathlineto{\pgfqpoint{4.689466in}{1.211038in}}%
\pgfpathlineto{\pgfqpoint{4.690308in}{1.224173in}}%
\pgfpathlineto{\pgfqpoint{4.691572in}{1.224173in}}%
\pgfpathlineto{\pgfqpoint{4.691993in}{1.211038in}}%
\pgfpathlineto{\pgfqpoint{4.692415in}{1.224173in}}%
\pgfpathlineto{\pgfqpoint{4.693257in}{1.224173in}}%
\pgfpathlineto{\pgfqpoint{4.694521in}{1.237308in}}%
\pgfpathlineto{\pgfqpoint{4.694942in}{1.237308in}}%
\pgfpathlineto{\pgfqpoint{4.695363in}{1.211038in}}%
\pgfpathlineto{\pgfqpoint{4.696206in}{1.224173in}}%
\pgfpathlineto{\pgfqpoint{4.696627in}{1.224173in}}%
\pgfpathlineto{\pgfqpoint{4.697049in}{1.237308in}}%
\pgfpathlineto{\pgfqpoint{4.697470in}{1.224173in}}%
\pgfpathlineto{\pgfqpoint{4.697891in}{1.211038in}}%
\pgfpathlineto{\pgfqpoint{4.698312in}{1.224173in}}%
\pgfpathlineto{\pgfqpoint{4.699576in}{1.224173in}}%
\pgfpathlineto{\pgfqpoint{4.699998in}{1.211038in}}%
\pgfpathlineto{\pgfqpoint{4.700419in}{1.224173in}}%
\pgfpathlineto{\pgfqpoint{4.700840in}{1.224173in}}%
\pgfpathlineto{\pgfqpoint{4.701261in}{1.237308in}}%
\pgfpathlineto{\pgfqpoint{4.701683in}{1.224173in}}%
\pgfpathlineto{\pgfqpoint{4.702525in}{1.197903in}}%
\pgfpathlineto{\pgfqpoint{4.702946in}{1.211038in}}%
\pgfpathlineto{\pgfqpoint{4.704210in}{1.224173in}}%
\pgfpathlineto{\pgfqpoint{4.704632in}{1.224173in}}%
\pgfpathlineto{\pgfqpoint{4.705474in}{1.197903in}}%
\pgfpathlineto{\pgfqpoint{4.705895in}{1.224173in}}%
\pgfpathlineto{\pgfqpoint{4.706738in}{1.211038in}}%
\pgfpathlineto{\pgfqpoint{4.707159in}{1.224173in}}%
\pgfpathlineto{\pgfqpoint{4.707581in}{1.211038in}}%
\pgfpathlineto{\pgfqpoint{4.708423in}{1.211038in}}%
\pgfpathlineto{\pgfqpoint{4.708844in}{1.237308in}}%
\pgfpathlineto{\pgfqpoint{4.709687in}{1.224173in}}%
\pgfpathlineto{\pgfqpoint{4.710108in}{1.224173in}}%
\pgfpathlineto{\pgfqpoint{4.711372in}{1.211038in}}%
\pgfpathlineto{\pgfqpoint{4.711793in}{1.211038in}}%
\pgfpathlineto{\pgfqpoint{4.713478in}{1.250443in}}%
\pgfpathlineto{\pgfqpoint{4.715164in}{1.211038in}}%
\pgfpathlineto{\pgfqpoint{4.716427in}{1.237308in}}%
\pgfpathlineto{\pgfqpoint{4.717691in}{1.224173in}}%
\pgfpathlineto{\pgfqpoint{4.718113in}{1.224173in}}%
\pgfpathlineto{\pgfqpoint{4.718955in}{1.237308in}}%
\pgfpathlineto{\pgfqpoint{4.720640in}{1.211038in}}%
\pgfpathlineto{\pgfqpoint{4.721904in}{1.224173in}}%
\pgfpathlineto{\pgfqpoint{4.724432in}{1.224173in}}%
\pgfpathlineto{\pgfqpoint{4.724853in}{1.237308in}}%
\pgfpathlineto{\pgfqpoint{4.725274in}{1.224173in}}%
\pgfpathlineto{\pgfqpoint{4.725696in}{1.224173in}}%
\pgfpathlineto{\pgfqpoint{4.726117in}{1.237308in}}%
\pgfpathlineto{\pgfqpoint{4.726538in}{1.224173in}}%
\pgfpathlineto{\pgfqpoint{4.726959in}{1.211038in}}%
\pgfpathlineto{\pgfqpoint{4.727381in}{1.237308in}}%
\pgfpathlineto{\pgfqpoint{4.728223in}{1.224173in}}%
\pgfpathlineto{\pgfqpoint{4.729487in}{1.237308in}}%
\pgfpathlineto{\pgfqpoint{4.730751in}{1.211038in}}%
\pgfpathlineto{\pgfqpoint{4.732015in}{1.224173in}}%
\pgfpathlineto{\pgfqpoint{4.733700in}{1.224173in}}%
\pgfpathlineto{\pgfqpoint{4.734964in}{1.237308in}}%
\pgfpathlineto{\pgfqpoint{4.736227in}{1.224173in}}%
\pgfpathlineto{\pgfqpoint{4.736649in}{1.224173in}}%
\pgfpathlineto{\pgfqpoint{4.737070in}{1.211038in}}%
\pgfpathlineto{\pgfqpoint{4.737491in}{1.224173in}}%
\pgfpathlineto{\pgfqpoint{4.738334in}{1.224173in}}%
\pgfpathlineto{\pgfqpoint{4.738755in}{1.237308in}}%
\pgfpathlineto{\pgfqpoint{4.739176in}{1.224173in}}%
\pgfpathlineto{\pgfqpoint{4.740019in}{1.224173in}}%
\pgfpathlineto{\pgfqpoint{4.740862in}{1.211038in}}%
\pgfpathlineto{\pgfqpoint{4.741283in}{1.237308in}}%
\pgfpathlineto{\pgfqpoint{4.742125in}{1.224173in}}%
\pgfpathlineto{\pgfqpoint{4.743810in}{1.224173in}}%
\pgfpathlineto{\pgfqpoint{4.744232in}{1.250443in}}%
\pgfpathlineto{\pgfqpoint{4.744653in}{1.224173in}}%
\pgfpathlineto{\pgfqpoint{4.745496in}{1.224173in}}%
\pgfpathlineto{\pgfqpoint{4.745917in}{1.211038in}}%
\pgfpathlineto{\pgfqpoint{4.746338in}{1.224173in}}%
\pgfpathlineto{\pgfqpoint{4.750130in}{1.224173in}}%
\pgfpathlineto{\pgfqpoint{4.750551in}{1.237308in}}%
\pgfpathlineto{\pgfqpoint{4.750972in}{1.224173in}}%
\pgfpathlineto{\pgfqpoint{4.751815in}{1.224173in}}%
\pgfpathlineto{\pgfqpoint{4.752236in}{1.211038in}}%
\pgfpathlineto{\pgfqpoint{4.752657in}{1.224173in}}%
\pgfpathlineto{\pgfqpoint{4.755606in}{1.224173in}}%
\pgfpathlineto{\pgfqpoint{4.756028in}{1.211038in}}%
\pgfpathlineto{\pgfqpoint{4.756449in}{1.224173in}}%
\pgfpathlineto{\pgfqpoint{4.756870in}{1.224173in}}%
\pgfpathlineto{\pgfqpoint{4.757713in}{1.211038in}}%
\pgfpathlineto{\pgfqpoint{4.759398in}{1.237308in}}%
\pgfpathlineto{\pgfqpoint{4.760662in}{1.224173in}}%
\pgfpathlineto{\pgfqpoint{4.761504in}{1.224173in}}%
\pgfpathlineto{\pgfqpoint{4.762347in}{1.211038in}}%
\pgfpathlineto{\pgfqpoint{4.763611in}{1.224173in}}%
\pgfpathlineto{\pgfqpoint{4.764032in}{1.211038in}}%
\pgfpathlineto{\pgfqpoint{4.764453in}{1.224173in}}%
\pgfpathlineto{\pgfqpoint{4.768666in}{1.224173in}}%
\pgfpathlineto{\pgfqpoint{4.769087in}{1.237308in}}%
\pgfpathlineto{\pgfqpoint{4.769508in}{1.224173in}}%
\pgfpathlineto{\pgfqpoint{4.770351in}{1.211038in}}%
\pgfpathlineto{\pgfqpoint{4.770772in}{1.237308in}}%
\pgfpathlineto{\pgfqpoint{4.771615in}{1.224173in}}%
\pgfpathlineto{\pgfqpoint{4.772036in}{1.211038in}}%
\pgfpathlineto{\pgfqpoint{4.772457in}{1.224173in}}%
\pgfpathlineto{\pgfqpoint{4.772879in}{1.224173in}}%
\pgfpathlineto{\pgfqpoint{4.773300in}{1.237308in}}%
\pgfpathlineto{\pgfqpoint{4.773721in}{1.224173in}}%
\pgfpathlineto{\pgfqpoint{4.774985in}{1.224173in}}%
\pgfpathlineto{\pgfqpoint{4.775828in}{1.237308in}}%
\pgfpathlineto{\pgfqpoint{4.777091in}{1.224173in}}%
\pgfpathlineto{\pgfqpoint{4.777513in}{1.250443in}}%
\pgfpathlineto{\pgfqpoint{4.777934in}{1.237308in}}%
\pgfpathlineto{\pgfqpoint{4.779198in}{1.224173in}}%
\pgfpathlineto{\pgfqpoint{4.780883in}{1.224173in}}%
\pgfpathlineto{\pgfqpoint{4.782568in}{1.197903in}}%
\pgfpathlineto{\pgfqpoint{4.783832in}{1.224173in}}%
\pgfpathlineto{\pgfqpoint{4.784253in}{1.211038in}}%
\pgfpathlineto{\pgfqpoint{4.784674in}{1.224173in}}%
\pgfpathlineto{\pgfqpoint{4.785096in}{1.237308in}}%
\pgfpathlineto{\pgfqpoint{4.786781in}{1.197903in}}%
\pgfpathlineto{\pgfqpoint{4.788045in}{1.211038in}}%
\pgfpathlineto{\pgfqpoint{4.788887in}{1.211038in}}%
\pgfpathlineto{\pgfqpoint{4.790151in}{1.237308in}}%
\pgfpathlineto{\pgfqpoint{4.791415in}{1.197903in}}%
\pgfpathlineto{\pgfqpoint{4.792679in}{1.224173in}}%
\pgfpathlineto{\pgfqpoint{4.793100in}{1.224173in}}%
\pgfpathlineto{\pgfqpoint{4.793521in}{1.211038in}}%
\pgfpathlineto{\pgfqpoint{4.793943in}{1.224173in}}%
\pgfpathlineto{\pgfqpoint{4.795206in}{1.224173in}}%
\pgfpathlineto{\pgfqpoint{4.795628in}{1.237308in}}%
\pgfpathlineto{\pgfqpoint{4.796049in}{1.224173in}}%
\pgfpathlineto{\pgfqpoint{4.798155in}{1.224173in}}%
\pgfpathlineto{\pgfqpoint{4.798577in}{1.211038in}}%
\pgfpathlineto{\pgfqpoint{4.798998in}{1.224173in}}%
\pgfpathlineto{\pgfqpoint{4.799840in}{1.224173in}}%
\pgfpathlineto{\pgfqpoint{4.800262in}{1.237308in}}%
\pgfpathlineto{\pgfqpoint{4.800683in}{1.224173in}}%
\pgfpathlineto{\pgfqpoint{4.801104in}{1.224173in}}%
\pgfpathlineto{\pgfqpoint{4.801526in}{1.237308in}}%
\pgfpathlineto{\pgfqpoint{4.801947in}{1.224173in}}%
\pgfpathlineto{\pgfqpoint{4.803211in}{1.224173in}}%
\pgfpathlineto{\pgfqpoint{4.803632in}{1.197903in}}%
\pgfpathlineto{\pgfqpoint{4.804053in}{1.211038in}}%
\pgfpathlineto{\pgfqpoint{4.804475in}{1.237308in}}%
\pgfpathlineto{\pgfqpoint{4.804896in}{1.224173in}}%
\pgfpathlineto{\pgfqpoint{4.806160in}{1.211038in}}%
\pgfpathlineto{\pgfqpoint{4.806581in}{1.211038in}}%
\pgfpathlineto{\pgfqpoint{4.807002in}{1.224173in}}%
\pgfpathlineto{\pgfqpoint{4.807423in}{1.211038in}}%
\pgfpathlineto{\pgfqpoint{4.807845in}{1.211038in}}%
\pgfpathlineto{\pgfqpoint{4.809109in}{1.224173in}}%
\pgfpathlineto{\pgfqpoint{4.810372in}{1.224173in}}%
\pgfpathlineto{\pgfqpoint{4.810794in}{1.237308in}}%
\pgfpathlineto{\pgfqpoint{4.811215in}{1.224173in}}%
\pgfpathlineto{\pgfqpoint{4.811636in}{1.224173in}}%
\pgfpathlineto{\pgfqpoint{4.812900in}{1.237308in}}%
\pgfpathlineto{\pgfqpoint{4.813321in}{1.237308in}}%
\pgfpathlineto{\pgfqpoint{4.813743in}{1.211038in}}%
\pgfpathlineto{\pgfqpoint{4.814164in}{1.224173in}}%
\pgfpathlineto{\pgfqpoint{4.814585in}{1.237308in}}%
\pgfpathlineto{\pgfqpoint{4.815006in}{1.211038in}}%
\pgfpathlineto{\pgfqpoint{4.815849in}{1.224173in}}%
\pgfpathlineto{\pgfqpoint{4.816270in}{1.211038in}}%
\pgfpathlineto{\pgfqpoint{4.817534in}{1.237308in}}%
\pgfpathlineto{\pgfqpoint{4.817955in}{1.211038in}}%
\pgfpathlineto{\pgfqpoint{4.818377in}{1.224173in}}%
\pgfpathlineto{\pgfqpoint{4.819641in}{1.237308in}}%
\pgfpathlineto{\pgfqpoint{4.820062in}{1.224173in}}%
\pgfpathlineto{\pgfqpoint{4.820483in}{1.237308in}}%
\pgfpathlineto{\pgfqpoint{4.821326in}{1.237308in}}%
\pgfpathlineto{\pgfqpoint{4.822589in}{1.211038in}}%
\pgfpathlineto{\pgfqpoint{4.823853in}{1.224173in}}%
\pgfpathlineto{\pgfqpoint{4.825117in}{1.224173in}}%
\pgfpathlineto{\pgfqpoint{4.826381in}{1.211038in}}%
\pgfpathlineto{\pgfqpoint{4.827224in}{1.237308in}}%
\pgfpathlineto{\pgfqpoint{4.827645in}{1.224173in}}%
\pgfpathlineto{\pgfqpoint{4.828066in}{1.211038in}}%
\pgfpathlineto{\pgfqpoint{4.828487in}{1.237308in}}%
\pgfpathlineto{\pgfqpoint{4.829330in}{1.224173in}}%
\pgfpathlineto{\pgfqpoint{4.829751in}{1.224173in}}%
\pgfpathlineto{\pgfqpoint{4.830172in}{1.211038in}}%
\pgfpathlineto{\pgfqpoint{4.831436in}{1.237308in}}%
\pgfpathlineto{\pgfqpoint{4.833121in}{1.211038in}}%
\pgfpathlineto{\pgfqpoint{4.833543in}{1.224173in}}%
\pgfpathlineto{\pgfqpoint{4.833964in}{1.211038in}}%
\pgfpathlineto{\pgfqpoint{4.834807in}{1.211038in}}%
\pgfpathlineto{\pgfqpoint{4.836492in}{1.237308in}}%
\pgfpathlineto{\pgfqpoint{4.837755in}{1.211038in}}%
\pgfpathlineto{\pgfqpoint{4.839441in}{1.237308in}}%
\pgfpathlineto{\pgfqpoint{4.839862in}{1.237308in}}%
\pgfpathlineto{\pgfqpoint{4.840283in}{1.211038in}}%
\pgfpathlineto{\pgfqpoint{4.841126in}{1.224173in}}%
\pgfpathlineto{\pgfqpoint{4.841547in}{1.224173in}}%
\pgfpathlineto{\pgfqpoint{4.841968in}{1.211038in}}%
\pgfpathlineto{\pgfqpoint{4.842390in}{1.224173in}}%
\pgfpathlineto{\pgfqpoint{4.842811in}{1.250443in}}%
\pgfpathlineto{\pgfqpoint{4.843653in}{1.237308in}}%
\pgfpathlineto{\pgfqpoint{4.844075in}{1.211038in}}%
\pgfpathlineto{\pgfqpoint{4.844917in}{1.224173in}}%
\pgfpathlineto{\pgfqpoint{4.845339in}{1.224173in}}%
\pgfpathlineto{\pgfqpoint{4.845760in}{1.237308in}}%
\pgfpathlineto{\pgfqpoint{4.846181in}{1.224173in}}%
\pgfpathlineto{\pgfqpoint{4.847024in}{1.224173in}}%
\pgfpathlineto{\pgfqpoint{4.847445in}{1.250443in}}%
\pgfpathlineto{\pgfqpoint{4.848287in}{1.237308in}}%
\pgfpathlineto{\pgfqpoint{4.848709in}{1.237308in}}%
\pgfpathlineto{\pgfqpoint{4.849973in}{1.211038in}}%
\pgfpathlineto{\pgfqpoint{4.851236in}{1.237308in}}%
\pgfpathlineto{\pgfqpoint{4.852500in}{1.224173in}}%
\pgfpathlineto{\pgfqpoint{4.854607in}{1.224173in}}%
\pgfpathlineto{\pgfqpoint{4.855870in}{1.237308in}}%
\pgfpathlineto{\pgfqpoint{4.856292in}{1.197903in}}%
\pgfpathlineto{\pgfqpoint{4.857134in}{1.211038in}}%
\pgfpathlineto{\pgfqpoint{4.857556in}{1.237308in}}%
\pgfpathlineto{\pgfqpoint{4.857977in}{1.211038in}}%
\pgfpathlineto{\pgfqpoint{4.858398in}{1.211038in}}%
\pgfpathlineto{\pgfqpoint{4.859662in}{1.237308in}}%
\pgfpathlineto{\pgfqpoint{4.860926in}{1.211038in}}%
\pgfpathlineto{\pgfqpoint{4.862611in}{1.237308in}}%
\pgfpathlineto{\pgfqpoint{4.863032in}{1.197903in}}%
\pgfpathlineto{\pgfqpoint{4.863875in}{1.211038in}}%
\pgfpathlineto{\pgfqpoint{4.865139in}{1.237308in}}%
\pgfpathlineto{\pgfqpoint{4.866824in}{1.211038in}}%
\pgfpathlineto{\pgfqpoint{4.867245in}{1.237308in}}%
\pgfpathlineto{\pgfqpoint{4.868088in}{1.224173in}}%
\pgfpathlineto{\pgfqpoint{4.868509in}{1.237308in}}%
\pgfpathlineto{\pgfqpoint{4.868930in}{1.224173in}}%
\pgfpathlineto{\pgfqpoint{4.869773in}{1.224173in}}%
\pgfpathlineto{\pgfqpoint{4.870194in}{1.237308in}}%
\pgfpathlineto{\pgfqpoint{4.870615in}{1.224173in}}%
\pgfpathlineto{\pgfqpoint{4.871036in}{1.224173in}}%
\pgfpathlineto{\pgfqpoint{4.871458in}{1.237308in}}%
\pgfpathlineto{\pgfqpoint{4.871879in}{1.224173in}}%
\pgfpathlineto{\pgfqpoint{4.872722in}{1.197903in}}%
\pgfpathlineto{\pgfqpoint{4.873143in}{1.211038in}}%
\pgfpathlineto{\pgfqpoint{4.874407in}{1.237308in}}%
\pgfpathlineto{\pgfqpoint{4.875671in}{1.211038in}}%
\pgfpathlineto{\pgfqpoint{4.876513in}{1.250443in}}%
\pgfpathlineto{\pgfqpoint{4.876934in}{1.224173in}}%
\pgfpathlineto{\pgfqpoint{4.877356in}{1.224173in}}%
\pgfpathlineto{\pgfqpoint{4.877777in}{1.211038in}}%
\pgfpathlineto{\pgfqpoint{4.878198in}{1.224173in}}%
\pgfpathlineto{\pgfqpoint{4.878619in}{1.224173in}}%
\pgfpathlineto{\pgfqpoint{4.879041in}{1.250443in}}%
\pgfpathlineto{\pgfqpoint{4.879462in}{1.224173in}}%
\pgfpathlineto{\pgfqpoint{4.879883in}{1.224173in}}%
\pgfpathlineto{\pgfqpoint{4.880305in}{1.237308in}}%
\pgfpathlineto{\pgfqpoint{4.881990in}{1.197903in}}%
\pgfpathlineto{\pgfqpoint{4.883675in}{1.224173in}}%
\pgfpathlineto{\pgfqpoint{4.884096in}{1.224173in}}%
\pgfpathlineto{\pgfqpoint{4.884517in}{1.211038in}}%
\pgfpathlineto{\pgfqpoint{4.884939in}{1.237308in}}%
\pgfpathlineto{\pgfqpoint{4.885781in}{1.224173in}}%
\pgfpathlineto{\pgfqpoint{4.887466in}{1.224173in}}%
\pgfpathlineto{\pgfqpoint{4.887888in}{1.197903in}}%
\pgfpathlineto{\pgfqpoint{4.888730in}{1.211038in}}%
\pgfpathlineto{\pgfqpoint{4.890415in}{1.237308in}}%
\pgfpathlineto{\pgfqpoint{4.892100in}{1.211038in}}%
\pgfpathlineto{\pgfqpoint{4.892522in}{1.211038in}}%
\pgfpathlineto{\pgfqpoint{4.893785in}{1.224173in}}%
\pgfpathlineto{\pgfqpoint{4.894628in}{1.224173in}}%
\pgfpathlineto{\pgfqpoint{4.895892in}{1.237308in}}%
\pgfpathlineto{\pgfqpoint{4.896313in}{1.237308in}}%
\pgfpathlineto{\pgfqpoint{4.897156in}{1.211038in}}%
\pgfpathlineto{\pgfqpoint{4.897577in}{1.224173in}}%
\pgfpathlineto{\pgfqpoint{4.897998in}{1.237308in}}%
\pgfpathlineto{\pgfqpoint{4.898420in}{1.197903in}}%
\pgfpathlineto{\pgfqpoint{4.898841in}{1.237308in}}%
\pgfpathlineto{\pgfqpoint{4.900105in}{1.237308in}}%
\pgfpathlineto{\pgfqpoint{4.900947in}{1.211038in}}%
\pgfpathlineto{\pgfqpoint{4.901368in}{1.224173in}}%
\pgfpathlineto{\pgfqpoint{4.901790in}{1.237308in}}%
\pgfpathlineto{\pgfqpoint{4.902211in}{1.224173in}}%
\pgfpathlineto{\pgfqpoint{4.903054in}{1.224173in}}%
\pgfpathlineto{\pgfqpoint{4.903475in}{1.211038in}}%
\pgfpathlineto{\pgfqpoint{4.903896in}{1.224173in}}%
\pgfpathlineto{\pgfqpoint{4.904317in}{1.224173in}}%
\pgfpathlineto{\pgfqpoint{4.904739in}{1.211038in}}%
\pgfpathlineto{\pgfqpoint{4.905160in}{1.224173in}}%
\pgfpathlineto{\pgfqpoint{4.905581in}{1.237308in}}%
\pgfpathlineto{\pgfqpoint{4.906003in}{1.211038in}}%
\pgfpathlineto{\pgfqpoint{4.906424in}{1.224173in}}%
\pgfpathlineto{\pgfqpoint{4.906845in}{1.237308in}}%
\pgfpathlineto{\pgfqpoint{4.907266in}{1.211038in}}%
\pgfpathlineto{\pgfqpoint{4.907688in}{1.224173in}}%
\pgfpathlineto{\pgfqpoint{4.908109in}{1.237308in}}%
\pgfpathlineto{\pgfqpoint{4.908530in}{1.211038in}}%
\pgfpathlineto{\pgfqpoint{4.909373in}{1.224173in}}%
\pgfpathlineto{\pgfqpoint{4.909794in}{1.224173in}}%
\pgfpathlineto{\pgfqpoint{4.910215in}{1.237308in}}%
\pgfpathlineto{\pgfqpoint{4.910637in}{1.224173in}}%
\pgfpathlineto{\pgfqpoint{4.911900in}{1.224173in}}%
\pgfpathlineto{\pgfqpoint{4.913164in}{1.211038in}}%
\pgfpathlineto{\pgfqpoint{4.913586in}{1.211038in}}%
\pgfpathlineto{\pgfqpoint{4.914428in}{1.237308in}}%
\pgfpathlineto{\pgfqpoint{4.914849in}{1.224173in}}%
\pgfpathlineto{\pgfqpoint{4.915271in}{1.224173in}}%
\pgfpathlineto{\pgfqpoint{4.916535in}{1.237308in}}%
\pgfpathlineto{\pgfqpoint{4.917377in}{1.197903in}}%
\pgfpathlineto{\pgfqpoint{4.917798in}{1.224173in}}%
\pgfpathlineto{\pgfqpoint{4.918220in}{1.224173in}}%
\pgfpathlineto{\pgfqpoint{4.918641in}{1.237308in}}%
\pgfpathlineto{\pgfqpoint{4.919062in}{1.211038in}}%
\pgfpathlineto{\pgfqpoint{4.919483in}{1.224173in}}%
\pgfpathlineto{\pgfqpoint{4.919905in}{1.237308in}}%
\pgfpathlineto{\pgfqpoint{4.920326in}{1.211038in}}%
\pgfpathlineto{\pgfqpoint{4.920747in}{1.237308in}}%
\pgfpathlineto{\pgfqpoint{4.921169in}{1.237308in}}%
\pgfpathlineto{\pgfqpoint{4.922432in}{1.211038in}}%
\pgfpathlineto{\pgfqpoint{4.922854in}{1.224173in}}%
\pgfpathlineto{\pgfqpoint{4.923275in}{1.211038in}}%
\pgfpathlineto{\pgfqpoint{4.923696in}{1.211038in}}%
\pgfpathlineto{\pgfqpoint{4.924539in}{1.237308in}}%
\pgfpathlineto{\pgfqpoint{4.924960in}{1.211038in}}%
\pgfpathlineto{\pgfqpoint{4.925803in}{1.224173in}}%
\pgfpathlineto{\pgfqpoint{4.926645in}{1.224173in}}%
\pgfpathlineto{\pgfqpoint{4.927066in}{1.237308in}}%
\pgfpathlineto{\pgfqpoint{4.927488in}{1.224173in}}%
\pgfpathlineto{\pgfqpoint{4.927909in}{1.211038in}}%
\pgfpathlineto{\pgfqpoint{4.928330in}{1.237308in}}%
\pgfpathlineto{\pgfqpoint{4.929173in}{1.224173in}}%
\pgfpathlineto{\pgfqpoint{4.929594in}{1.224173in}}%
\pgfpathlineto{\pgfqpoint{4.930015in}{1.211038in}}%
\pgfpathlineto{\pgfqpoint{4.930437in}{1.224173in}}%
\pgfpathlineto{\pgfqpoint{4.930858in}{1.250443in}}%
\pgfpathlineto{\pgfqpoint{4.931279in}{1.237308in}}%
\pgfpathlineto{\pgfqpoint{4.932543in}{1.224173in}}%
\pgfpathlineto{\pgfqpoint{4.933807in}{1.224173in}}%
\pgfpathlineto{\pgfqpoint{4.935071in}{1.211038in}}%
\pgfpathlineto{\pgfqpoint{4.936756in}{1.237308in}}%
\pgfpathlineto{\pgfqpoint{4.938020in}{1.211038in}}%
\pgfpathlineto{\pgfqpoint{4.939284in}{1.237308in}}%
\pgfpathlineto{\pgfqpoint{4.940547in}{1.211038in}}%
\pgfpathlineto{\pgfqpoint{4.941390in}{1.237308in}}%
\pgfpathlineto{\pgfqpoint{4.941811in}{1.224173in}}%
\pgfpathlineto{\pgfqpoint{4.942232in}{1.224173in}}%
\pgfpathlineto{\pgfqpoint{4.942654in}{1.211038in}}%
\pgfpathlineto{\pgfqpoint{4.943075in}{1.224173in}}%
\pgfpathlineto{\pgfqpoint{4.943496in}{1.250443in}}%
\pgfpathlineto{\pgfqpoint{4.943918in}{1.224173in}}%
\pgfpathlineto{\pgfqpoint{4.944339in}{1.211038in}}%
\pgfpathlineto{\pgfqpoint{4.944760in}{1.224173in}}%
\pgfpathlineto{\pgfqpoint{4.945181in}{1.224173in}}%
\pgfpathlineto{\pgfqpoint{4.945603in}{1.211038in}}%
\pgfpathlineto{\pgfqpoint{4.946024in}{1.224173in}}%
\pgfpathlineto{\pgfqpoint{4.946445in}{1.224173in}}%
\pgfpathlineto{\pgfqpoint{4.946867in}{1.211038in}}%
\pgfpathlineto{\pgfqpoint{4.947288in}{1.237308in}}%
\pgfpathlineto{\pgfqpoint{4.948130in}{1.224173in}}%
\pgfpathlineto{\pgfqpoint{4.948552in}{1.237308in}}%
\pgfpathlineto{\pgfqpoint{4.949394in}{1.211038in}}%
\pgfpathlineto{\pgfqpoint{4.949815in}{1.224173in}}%
\pgfpathlineto{\pgfqpoint{4.950237in}{1.211038in}}%
\pgfpathlineto{\pgfqpoint{4.951079in}{1.237308in}}%
\pgfpathlineto{\pgfqpoint{4.951501in}{1.224173in}}%
\pgfpathlineto{\pgfqpoint{4.951922in}{1.224173in}}%
\pgfpathlineto{\pgfqpoint{4.952764in}{1.211038in}}%
\pgfpathlineto{\pgfqpoint{4.954028in}{1.237308in}}%
\pgfpathlineto{\pgfqpoint{4.955292in}{1.211038in}}%
\pgfpathlineto{\pgfqpoint{4.956135in}{1.250443in}}%
\pgfpathlineto{\pgfqpoint{4.956556in}{1.237308in}}%
\pgfpathlineto{\pgfqpoint{4.956977in}{1.211038in}}%
\pgfpathlineto{\pgfqpoint{4.957820in}{1.224173in}}%
\pgfpathlineto{\pgfqpoint{4.958241in}{1.211038in}}%
\pgfpathlineto{\pgfqpoint{4.959505in}{1.237308in}}%
\pgfpathlineto{\pgfqpoint{4.960769in}{1.211038in}}%
\pgfpathlineto{\pgfqpoint{4.962033in}{1.224173in}}%
\pgfpathlineto{\pgfqpoint{4.962875in}{1.224173in}}%
\pgfpathlineto{\pgfqpoint{4.963296in}{1.237308in}}%
\pgfpathlineto{\pgfqpoint{4.963718in}{1.224173in}}%
\pgfpathlineto{\pgfqpoint{4.964560in}{1.197903in}}%
\pgfpathlineto{\pgfqpoint{4.964981in}{1.237308in}}%
\pgfpathlineto{\pgfqpoint{4.965824in}{1.224173in}}%
\pgfpathlineto{\pgfqpoint{4.966245in}{1.237308in}}%
\pgfpathlineto{\pgfqpoint{4.967509in}{1.211038in}}%
\pgfpathlineto{\pgfqpoint{4.967930in}{1.211038in}}%
\pgfpathlineto{\pgfqpoint{4.969194in}{1.224173in}}%
\pgfpathlineto{\pgfqpoint{4.970458in}{1.211038in}}%
\pgfpathlineto{\pgfqpoint{4.971301in}{1.237308in}}%
\pgfpathlineto{\pgfqpoint{4.972143in}{1.211038in}}%
\pgfpathlineto{\pgfqpoint{4.972564in}{1.237308in}}%
\pgfpathlineto{\pgfqpoint{4.973407in}{1.224173in}}%
\pgfpathlineto{\pgfqpoint{4.975092in}{1.224173in}}%
\pgfpathlineto{\pgfqpoint{4.975513in}{1.211038in}}%
\pgfpathlineto{\pgfqpoint{4.976356in}{1.237308in}}%
\pgfpathlineto{\pgfqpoint{4.976777in}{1.211038in}}%
\pgfpathlineto{\pgfqpoint{4.977620in}{1.224173in}}%
\pgfpathlineto{\pgfqpoint{4.978041in}{1.237308in}}%
\pgfpathlineto{\pgfqpoint{4.978462in}{1.224173in}}%
\pgfpathlineto{\pgfqpoint{4.978884in}{1.224173in}}%
\pgfpathlineto{\pgfqpoint{4.979305in}{1.250443in}}%
\pgfpathlineto{\pgfqpoint{4.979726in}{1.224173in}}%
\pgfpathlineto{\pgfqpoint{4.980148in}{1.224173in}}%
\pgfpathlineto{\pgfqpoint{4.980569in}{1.211038in}}%
\pgfpathlineto{\pgfqpoint{4.980990in}{1.224173in}}%
\pgfpathlineto{\pgfqpoint{4.981411in}{1.237308in}}%
\pgfpathlineto{\pgfqpoint{4.981833in}{1.211038in}}%
\pgfpathlineto{\pgfqpoint{4.982675in}{1.224173in}}%
\pgfpathlineto{\pgfqpoint{4.983096in}{1.211038in}}%
\pgfpathlineto{\pgfqpoint{4.983518in}{1.224173in}}%
\pgfpathlineto{\pgfqpoint{4.983939in}{1.237308in}}%
\pgfpathlineto{\pgfqpoint{4.984360in}{1.224173in}}%
\pgfpathlineto{\pgfqpoint{4.984782in}{1.224173in}}%
\pgfpathlineto{\pgfqpoint{4.985624in}{1.237308in}}%
\pgfpathlineto{\pgfqpoint{4.986467in}{1.211038in}}%
\pgfpathlineto{\pgfqpoint{4.986888in}{1.224173in}}%
\pgfpathlineto{\pgfqpoint{4.987309in}{1.224173in}}%
\pgfpathlineto{\pgfqpoint{4.988152in}{1.237308in}}%
\pgfpathlineto{\pgfqpoint{4.989416in}{1.211038in}}%
\pgfpathlineto{\pgfqpoint{4.991101in}{1.237308in}}%
\pgfpathlineto{\pgfqpoint{4.991943in}{1.211038in}}%
\pgfpathlineto{\pgfqpoint{4.992365in}{1.237308in}}%
\pgfpathlineto{\pgfqpoint{4.992786in}{1.224173in}}%
\pgfpathlineto{\pgfqpoint{4.993207in}{1.211038in}}%
\pgfpathlineto{\pgfqpoint{4.993628in}{1.224173in}}%
\pgfpathlineto{\pgfqpoint{4.994471in}{1.237308in}}%
\pgfpathlineto{\pgfqpoint{4.996156in}{1.211038in}}%
\pgfpathlineto{\pgfqpoint{4.997420in}{1.237308in}}%
\pgfpathlineto{\pgfqpoint{4.997841in}{1.237308in}}%
\pgfpathlineto{\pgfqpoint{4.998262in}{1.211038in}}%
\pgfpathlineto{\pgfqpoint{4.999105in}{1.224173in}}%
\pgfpathlineto{\pgfqpoint{4.999526in}{1.224173in}}%
\pgfpathlineto{\pgfqpoint{5.000369in}{1.237308in}}%
\pgfpathlineto{\pgfqpoint{5.002054in}{1.211038in}}%
\pgfpathlineto{\pgfqpoint{5.002475in}{1.211038in}}%
\pgfpathlineto{\pgfqpoint{5.004160in}{1.237308in}}%
\pgfpathlineto{\pgfqpoint{5.004582in}{1.211038in}}%
\pgfpathlineto{\pgfqpoint{5.005003in}{1.250443in}}%
\pgfpathlineto{\pgfqpoint{5.006267in}{1.197903in}}%
\pgfpathlineto{\pgfqpoint{5.006688in}{1.237308in}}%
\pgfpathlineto{\pgfqpoint{5.007109in}{1.224173in}}%
\pgfpathlineto{\pgfqpoint{5.008373in}{1.211038in}}%
\pgfpathlineto{\pgfqpoint{5.008794in}{1.211038in}}%
\pgfpathlineto{\pgfqpoint{5.009637in}{1.224173in}}%
\pgfpathlineto{\pgfqpoint{5.010058in}{1.211038in}}%
\pgfpathlineto{\pgfqpoint{5.010480in}{1.237308in}}%
\pgfpathlineto{\pgfqpoint{5.010901in}{1.197903in}}%
\pgfpathlineto{\pgfqpoint{5.011322in}{1.224173in}}%
\pgfpathlineto{\pgfqpoint{5.011743in}{1.237308in}}%
\pgfpathlineto{\pgfqpoint{5.012165in}{1.224173in}}%
\pgfpathlineto{\pgfqpoint{5.013007in}{1.224173in}}%
\pgfpathlineto{\pgfqpoint{5.013428in}{1.211038in}}%
\pgfpathlineto{\pgfqpoint{5.013850in}{1.224173in}}%
\pgfpathlineto{\pgfqpoint{5.014271in}{1.224173in}}%
\pgfpathlineto{\pgfqpoint{5.014692in}{1.211038in}}%
\pgfpathlineto{\pgfqpoint{5.015535in}{1.263578in}}%
\pgfpathlineto{\pgfqpoint{5.015956in}{1.224173in}}%
\pgfpathlineto{\pgfqpoint{5.016377in}{1.224173in}}%
\pgfpathlineto{\pgfqpoint{5.016799in}{1.237308in}}%
\pgfpathlineto{\pgfqpoint{5.018484in}{1.197903in}}%
\pgfpathlineto{\pgfqpoint{5.019748in}{1.237308in}}%
\pgfpathlineto{\pgfqpoint{5.020169in}{1.224173in}}%
\pgfpathlineto{\pgfqpoint{5.021854in}{1.224173in}}%
\pgfpathlineto{\pgfqpoint{5.022697in}{1.237308in}}%
\pgfpathlineto{\pgfqpoint{5.023960in}{1.211038in}}%
\pgfpathlineto{\pgfqpoint{5.024382in}{1.237308in}}%
\pgfpathlineto{\pgfqpoint{5.024803in}{1.211038in}}%
\pgfpathlineto{\pgfqpoint{5.025224in}{1.211038in}}%
\pgfpathlineto{\pgfqpoint{5.026488in}{1.237308in}}%
\pgfpathlineto{\pgfqpoint{5.027331in}{1.211038in}}%
\pgfpathlineto{\pgfqpoint{5.028173in}{1.237308in}}%
\pgfpathlineto{\pgfqpoint{5.028594in}{1.224173in}}%
\pgfpathlineto{\pgfqpoint{5.029016in}{1.197903in}}%
\pgfpathlineto{\pgfqpoint{5.029437in}{1.237308in}}%
\pgfpathlineto{\pgfqpoint{5.031122in}{1.211038in}}%
\pgfpathlineto{\pgfqpoint{5.031965in}{1.237308in}}%
\pgfpathlineto{\pgfqpoint{5.032386in}{1.224173in}}%
\pgfpathlineto{\pgfqpoint{5.032807in}{1.224173in}}%
\pgfpathlineto{\pgfqpoint{5.033650in}{1.211038in}}%
\pgfpathlineto{\pgfqpoint{5.034492in}{1.237308in}}%
\pgfpathlineto{\pgfqpoint{5.034914in}{1.224173in}}%
\pgfpathlineto{\pgfqpoint{5.035756in}{1.237308in}}%
\pgfpathlineto{\pgfqpoint{5.037020in}{1.211038in}}%
\pgfpathlineto{\pgfqpoint{5.038284in}{1.224173in}}%
\pgfpathlineto{\pgfqpoint{5.038705in}{1.224173in}}%
\pgfpathlineto{\pgfqpoint{5.039969in}{1.211038in}}%
\pgfpathlineto{\pgfqpoint{5.040812in}{1.237308in}}%
\pgfpathlineto{\pgfqpoint{5.042075in}{1.211038in}}%
\pgfpathlineto{\pgfqpoint{5.043339in}{1.224173in}}%
\pgfpathlineto{\pgfqpoint{5.043761in}{1.211038in}}%
\pgfpathlineto{\pgfqpoint{5.044603in}{1.237308in}}%
\pgfpathlineto{\pgfqpoint{5.045024in}{1.224173in}}%
\pgfpathlineto{\pgfqpoint{5.046288in}{1.211038in}}%
\pgfpathlineto{\pgfqpoint{5.047131in}{1.237308in}}%
\pgfpathlineto{\pgfqpoint{5.047552in}{1.211038in}}%
\pgfpathlineto{\pgfqpoint{5.047973in}{1.224173in}}%
\pgfpathlineto{\pgfqpoint{5.048395in}{1.237308in}}%
\pgfpathlineto{\pgfqpoint{5.048816in}{1.224173in}}%
\pgfpathlineto{\pgfqpoint{5.050080in}{1.211038in}}%
\pgfpathlineto{\pgfqpoint{5.050922in}{1.237308in}}%
\pgfpathlineto{\pgfqpoint{5.051344in}{1.211038in}}%
\pgfpathlineto{\pgfqpoint{5.052186in}{1.224173in}}%
\pgfpathlineto{\pgfqpoint{5.055978in}{1.224173in}}%
\pgfpathlineto{\pgfqpoint{5.056399in}{1.211038in}}%
\pgfpathlineto{\pgfqpoint{5.056820in}{1.224173in}}%
\pgfpathlineto{\pgfqpoint{5.057241in}{1.237308in}}%
\pgfpathlineto{\pgfqpoint{5.058084in}{1.211038in}}%
\pgfpathlineto{\pgfqpoint{5.058505in}{1.237308in}}%
\pgfpathlineto{\pgfqpoint{5.059348in}{1.224173in}}%
\pgfpathlineto{\pgfqpoint{5.059769in}{1.224173in}}%
\pgfpathlineto{\pgfqpoint{5.060190in}{1.237308in}}%
\pgfpathlineto{\pgfqpoint{5.060612in}{1.224173in}}%
\pgfpathlineto{\pgfqpoint{5.061033in}{1.224173in}}%
\pgfpathlineto{\pgfqpoint{5.061454in}{1.211038in}}%
\pgfpathlineto{\pgfqpoint{5.061875in}{1.224173in}}%
\pgfpathlineto{\pgfqpoint{5.062718in}{1.224173in}}%
\pgfpathlineto{\pgfqpoint{5.063982in}{1.237308in}}%
\pgfpathlineto{\pgfqpoint{5.064403in}{1.211038in}}%
\pgfpathlineto{\pgfqpoint{5.065246in}{1.224173in}}%
\pgfpathlineto{\pgfqpoint{5.066088in}{1.237308in}}%
\pgfpathlineto{\pgfqpoint{5.066510in}{1.224173in}}%
\pgfpathlineto{\pgfqpoint{5.066931in}{1.237308in}}%
\pgfpathlineto{\pgfqpoint{5.067352in}{1.237308in}}%
\pgfpathlineto{\pgfqpoint{5.067773in}{1.211038in}}%
\pgfpathlineto{\pgfqpoint{5.068616in}{1.224173in}}%
\pgfpathlineto{\pgfqpoint{5.069458in}{1.224173in}}%
\pgfpathlineto{\pgfqpoint{5.069880in}{1.237308in}}%
\pgfpathlineto{\pgfqpoint{5.070301in}{1.224173in}}%
\pgfpathlineto{\pgfqpoint{5.070722in}{1.211038in}}%
\pgfpathlineto{\pgfqpoint{5.071144in}{1.224173in}}%
\pgfpathlineto{\pgfqpoint{5.072407in}{1.237308in}}%
\pgfpathlineto{\pgfqpoint{5.073671in}{1.224173in}}%
\pgfpathlineto{\pgfqpoint{5.074093in}{1.224173in}}%
\pgfpathlineto{\pgfqpoint{5.074514in}{0.830120in}}%
\pgfpathlineto{\pgfqpoint{5.074935in}{1.237308in}}%
\pgfpathlineto{\pgfqpoint{5.075356in}{1.211038in}}%
\pgfpathlineto{\pgfqpoint{5.075778in}{1.224173in}}%
\pgfpathlineto{\pgfqpoint{5.076199in}{1.237308in}}%
\pgfpathlineto{\pgfqpoint{5.077463in}{1.211038in}}%
\pgfpathlineto{\pgfqpoint{5.077884in}{1.250443in}}%
\pgfpathlineto{\pgfqpoint{5.078305in}{1.237308in}}%
\pgfpathlineto{\pgfqpoint{5.079569in}{1.211038in}}%
\pgfpathlineto{\pgfqpoint{5.080833in}{1.224173in}}%
\pgfpathlineto{\pgfqpoint{5.081676in}{1.224173in}}%
\pgfpathlineto{\pgfqpoint{5.082939in}{1.237308in}}%
\pgfpathlineto{\pgfqpoint{5.083782in}{1.224173in}}%
\pgfpathlineto{\pgfqpoint{5.084624in}{1.237308in}}%
\pgfpathlineto{\pgfqpoint{5.085888in}{1.211038in}}%
\pgfpathlineto{\pgfqpoint{5.086310in}{1.237308in}}%
\pgfpathlineto{\pgfqpoint{5.086731in}{1.211038in}}%
\pgfpathlineto{\pgfqpoint{5.087573in}{1.211038in}}%
\pgfpathlineto{\pgfqpoint{5.087995in}{1.618226in}}%
\pgfpathlineto{\pgfqpoint{5.088416in}{1.211038in}}%
\pgfpathlineto{\pgfqpoint{5.090101in}{1.276714in}}%
\pgfpathlineto{\pgfqpoint{5.091365in}{1.211038in}}%
\pgfpathlineto{\pgfqpoint{5.091786in}{1.211038in}}%
\pgfpathlineto{\pgfqpoint{5.093471in}{1.237308in}}%
\pgfpathlineto{\pgfqpoint{5.094735in}{1.224173in}}%
\pgfpathlineto{\pgfqpoint{5.095578in}{1.237308in}}%
\pgfpathlineto{\pgfqpoint{5.095999in}{1.211038in}}%
\pgfpathlineto{\pgfqpoint{5.096842in}{1.224173in}}%
\pgfpathlineto{\pgfqpoint{5.097684in}{1.237308in}}%
\pgfpathlineto{\pgfqpoint{5.098527in}{1.224173in}}%
\pgfpathlineto{\pgfqpoint{5.099369in}{1.237308in}}%
\pgfpathlineto{\pgfqpoint{5.099790in}{1.329254in}}%
\pgfpathlineto{\pgfqpoint{5.100212in}{1.237308in}}%
\pgfpathlineto{\pgfqpoint{5.101476in}{1.237308in}}%
\pgfpathlineto{\pgfqpoint{5.101897in}{1.250443in}}%
\pgfpathlineto{\pgfqpoint{5.103582in}{1.184768in}}%
\pgfpathlineto{\pgfqpoint{5.104003in}{1.237308in}}%
\pgfpathlineto{\pgfqpoint{5.104846in}{1.211038in}}%
\pgfpathlineto{\pgfqpoint{5.105688in}{1.237308in}}%
\pgfpathlineto{\pgfqpoint{5.106110in}{1.211038in}}%
\pgfpathlineto{\pgfqpoint{5.106952in}{1.224173in}}%
\pgfpathlineto{\pgfqpoint{5.107373in}{1.211038in}}%
\pgfpathlineto{\pgfqpoint{5.108216in}{1.237308in}}%
\pgfpathlineto{\pgfqpoint{5.109059in}{1.211038in}}%
\pgfpathlineto{\pgfqpoint{5.109480in}{1.237308in}}%
\pgfpathlineto{\pgfqpoint{5.109901in}{1.211038in}}%
\pgfpathlineto{\pgfqpoint{5.110744in}{1.211038in}}%
\pgfpathlineto{\pgfqpoint{5.111165in}{1.197903in}}%
\pgfpathlineto{\pgfqpoint{5.112008in}{1.250443in}}%
\pgfpathlineto{\pgfqpoint{5.112429in}{1.224173in}}%
\pgfpathlineto{\pgfqpoint{5.112850in}{1.224173in}}%
\pgfpathlineto{\pgfqpoint{5.113271in}{1.105957in}}%
\pgfpathlineto{\pgfqpoint{5.113693in}{1.211038in}}%
\pgfpathlineto{\pgfqpoint{5.114114in}{1.237308in}}%
\pgfpathlineto{\pgfqpoint{5.114535in}{1.224173in}}%
\pgfpathlineto{\pgfqpoint{5.115378in}{1.211038in}}%
\pgfpathlineto{\pgfqpoint{5.116642in}{1.237308in}}%
\pgfpathlineto{\pgfqpoint{5.117484in}{1.197903in}}%
\pgfpathlineto{\pgfqpoint{5.117905in}{1.224173in}}%
\pgfpathlineto{\pgfqpoint{5.118327in}{1.237308in}}%
\pgfpathlineto{\pgfqpoint{5.118748in}{1.224173in}}%
\pgfpathlineto{\pgfqpoint{5.120433in}{1.224173in}}%
\pgfpathlineto{\pgfqpoint{5.120854in}{1.237308in}}%
\pgfpathlineto{\pgfqpoint{5.121276in}{1.197903in}}%
\pgfpathlineto{\pgfqpoint{5.121697in}{1.211038in}}%
\pgfpathlineto{\pgfqpoint{5.122961in}{1.224173in}}%
\pgfpathlineto{\pgfqpoint{5.123803in}{1.224173in}}%
\pgfpathlineto{\pgfqpoint{5.124646in}{1.250443in}}%
\pgfpathlineto{\pgfqpoint{5.126331in}{1.211038in}}%
\pgfpathlineto{\pgfqpoint{5.126752in}{1.237308in}}%
\pgfpathlineto{\pgfqpoint{5.127174in}{1.224173in}}%
\pgfpathlineto{\pgfqpoint{5.127595in}{1.197903in}}%
\pgfpathlineto{\pgfqpoint{5.128016in}{1.211038in}}%
\pgfpathlineto{\pgfqpoint{5.128437in}{1.237308in}}%
\pgfpathlineto{\pgfqpoint{5.129280in}{1.224173in}}%
\pgfpathlineto{\pgfqpoint{5.130123in}{1.211038in}}%
\pgfpathlineto{\pgfqpoint{5.130965in}{1.250443in}}%
\pgfpathlineto{\pgfqpoint{5.131386in}{1.224173in}}%
\pgfpathlineto{\pgfqpoint{5.131808in}{1.211038in}}%
\pgfpathlineto{\pgfqpoint{5.133071in}{1.237308in}}%
\pgfpathlineto{\pgfqpoint{5.133914in}{1.211038in}}%
\pgfpathlineto{\pgfqpoint{5.134335in}{1.224173in}}%
\pgfpathlineto{\pgfqpoint{5.135599in}{1.250443in}}%
\pgfpathlineto{\pgfqpoint{5.136442in}{1.211038in}}%
\pgfpathlineto{\pgfqpoint{5.136863in}{1.237308in}}%
\pgfpathlineto{\pgfqpoint{5.137284in}{1.237308in}}%
\pgfpathlineto{\pgfqpoint{5.138127in}{1.197903in}}%
\pgfpathlineto{\pgfqpoint{5.138548in}{1.211038in}}%
\pgfpathlineto{\pgfqpoint{5.139812in}{1.237308in}}%
\pgfpathlineto{\pgfqpoint{5.140233in}{1.211038in}}%
\pgfpathlineto{\pgfqpoint{5.140654in}{1.237308in}}%
\pgfpathlineto{\pgfqpoint{5.141076in}{1.237308in}}%
\pgfpathlineto{\pgfqpoint{5.141918in}{1.211038in}}%
\pgfpathlineto{\pgfqpoint{5.143182in}{1.237308in}}%
\pgfpathlineto{\pgfqpoint{5.144025in}{1.237308in}}%
\pgfpathlineto{\pgfqpoint{5.144446in}{1.211038in}}%
\pgfpathlineto{\pgfqpoint{5.145289in}{1.224173in}}%
\pgfpathlineto{\pgfqpoint{5.145710in}{1.224173in}}%
\pgfpathlineto{\pgfqpoint{5.146552in}{1.211038in}}%
\pgfpathlineto{\pgfqpoint{5.147395in}{1.237308in}}%
\pgfpathlineto{\pgfqpoint{5.148237in}{1.211038in}}%
\pgfpathlineto{\pgfqpoint{5.149080in}{1.197903in}}%
\pgfpathlineto{\pgfqpoint{5.149923in}{1.250443in}}%
\pgfpathlineto{\pgfqpoint{5.150344in}{1.211038in}}%
\pgfpathlineto{\pgfqpoint{5.151186in}{1.224173in}}%
\pgfpathlineto{\pgfqpoint{5.152029in}{1.237308in}}%
\pgfpathlineto{\pgfqpoint{5.153293in}{1.211038in}}%
\pgfpathlineto{\pgfqpoint{5.153714in}{1.237308in}}%
\pgfpathlineto{\pgfqpoint{5.154557in}{1.224173in}}%
\pgfpathlineto{\pgfqpoint{5.154978in}{1.237308in}}%
\pgfpathlineto{\pgfqpoint{5.155399in}{1.224173in}}%
\pgfpathlineto{\pgfqpoint{5.155820in}{1.211038in}}%
\pgfpathlineto{\pgfqpoint{5.156242in}{1.237308in}}%
\pgfpathlineto{\pgfqpoint{5.157084in}{1.224173in}}%
\pgfpathlineto{\pgfqpoint{5.157506in}{1.211038in}}%
\pgfpathlineto{\pgfqpoint{5.157927in}{1.224173in}}%
\pgfpathlineto{\pgfqpoint{5.158348in}{1.224173in}}%
\pgfpathlineto{\pgfqpoint{5.158769in}{1.237308in}}%
\pgfpathlineto{\pgfqpoint{5.159191in}{1.211038in}}%
\pgfpathlineto{\pgfqpoint{5.159612in}{1.250443in}}%
\pgfpathlineto{\pgfqpoint{5.160455in}{1.211038in}}%
\pgfpathlineto{\pgfqpoint{5.160876in}{1.224173in}}%
\pgfpathlineto{\pgfqpoint{5.161718in}{1.237308in}}%
\pgfpathlineto{\pgfqpoint{5.163403in}{1.197903in}}%
\pgfpathlineto{\pgfqpoint{5.164667in}{1.237308in}}%
\pgfpathlineto{\pgfqpoint{5.165510in}{1.211038in}}%
\pgfpathlineto{\pgfqpoint{5.166774in}{1.237308in}}%
\pgfpathlineto{\pgfqpoint{5.168459in}{1.184768in}}%
\pgfpathlineto{\pgfqpoint{5.168880in}{1.237308in}}%
\pgfpathlineto{\pgfqpoint{5.169723in}{1.211038in}}%
\pgfpathlineto{\pgfqpoint{5.170986in}{1.237308in}}%
\pgfpathlineto{\pgfqpoint{5.172250in}{1.224173in}}%
\pgfpathlineto{\pgfqpoint{5.172672in}{1.237308in}}%
\pgfpathlineto{\pgfqpoint{5.173093in}{1.211038in}}%
\pgfpathlineto{\pgfqpoint{5.173935in}{1.224173in}}%
\pgfpathlineto{\pgfqpoint{5.174357in}{1.211038in}}%
\pgfpathlineto{\pgfqpoint{5.174778in}{1.224173in}}%
\pgfpathlineto{\pgfqpoint{5.175621in}{1.224173in}}%
\pgfpathlineto{\pgfqpoint{5.176042in}{1.237308in}}%
\pgfpathlineto{\pgfqpoint{5.176463in}{1.224173in}}%
\pgfpathlineto{\pgfqpoint{5.176884in}{1.224173in}}%
\pgfpathlineto{\pgfqpoint{5.177306in}{1.237308in}}%
\pgfpathlineto{\pgfqpoint{5.177727in}{1.224173in}}%
\pgfpathlineto{\pgfqpoint{5.178570in}{1.197903in}}%
\pgfpathlineto{\pgfqpoint{5.178991in}{1.237308in}}%
\pgfpathlineto{\pgfqpoint{5.179833in}{1.224173in}}%
\pgfpathlineto{\pgfqpoint{5.180255in}{1.211038in}}%
\pgfpathlineto{\pgfqpoint{5.180676in}{1.224173in}}%
\pgfpathlineto{\pgfqpoint{5.181940in}{1.250443in}}%
\pgfpathlineto{\pgfqpoint{5.182782in}{1.224173in}}%
\pgfpathlineto{\pgfqpoint{5.183204in}{1.237308in}}%
\pgfpathlineto{\pgfqpoint{5.183625in}{1.237308in}}%
\pgfpathlineto{\pgfqpoint{5.184889in}{1.224173in}}%
\pgfpathlineto{\pgfqpoint{5.185310in}{1.224173in}}%
\pgfpathlineto{\pgfqpoint{5.186153in}{1.211038in}}%
\pgfpathlineto{\pgfqpoint{5.187838in}{1.250443in}}%
\pgfpathlineto{\pgfqpoint{5.188680in}{1.197903in}}%
\pgfpathlineto{\pgfqpoint{5.189523in}{1.211038in}}%
\pgfpathlineto{\pgfqpoint{5.190365in}{1.224173in}}%
\pgfpathlineto{\pgfqpoint{5.190787in}{1.211038in}}%
\pgfpathlineto{\pgfqpoint{5.192050in}{1.237308in}}%
\pgfpathlineto{\pgfqpoint{5.193314in}{1.211038in}}%
\pgfpathlineto{\pgfqpoint{5.194157in}{1.237308in}}%
\pgfpathlineto{\pgfqpoint{5.194999in}{1.211038in}}%
\pgfpathlineto{\pgfqpoint{5.195421in}{1.224173in}}%
\pgfpathlineto{\pgfqpoint{5.196684in}{1.224173in}}%
\pgfpathlineto{\pgfqpoint{5.197106in}{1.211038in}}%
\pgfpathlineto{\pgfqpoint{5.197527in}{1.237308in}}%
\pgfpathlineto{\pgfqpoint{5.198370in}{1.224173in}}%
\pgfpathlineto{\pgfqpoint{5.198791in}{1.211038in}}%
\pgfpathlineto{\pgfqpoint{5.199633in}{1.237308in}}%
\pgfpathlineto{\pgfqpoint{5.200055in}{1.224173in}}%
\pgfpathlineto{\pgfqpoint{5.200897in}{1.224173in}}%
\pgfpathlineto{\pgfqpoint{5.201319in}{1.197903in}}%
\pgfpathlineto{\pgfqpoint{5.201740in}{1.224173in}}%
\pgfpathlineto{\pgfqpoint{5.202161in}{1.250443in}}%
\pgfpathlineto{\pgfqpoint{5.203004in}{1.237308in}}%
\pgfpathlineto{\pgfqpoint{5.203846in}{1.237308in}}%
\pgfpathlineto{\pgfqpoint{5.204689in}{1.224173in}}%
\pgfpathlineto{\pgfqpoint{5.205531in}{1.237308in}}%
\pgfpathlineto{\pgfqpoint{5.205953in}{1.211038in}}%
\pgfpathlineto{\pgfqpoint{5.206374in}{1.237308in}}%
\pgfpathlineto{\pgfqpoint{5.206795in}{1.237308in}}%
\pgfpathlineto{\pgfqpoint{5.207638in}{1.211038in}}%
\pgfpathlineto{\pgfqpoint{5.208059in}{1.237308in}}%
\pgfpathlineto{\pgfqpoint{5.208902in}{1.224173in}}%
\pgfpathlineto{\pgfqpoint{5.209323in}{1.224173in}}%
\pgfpathlineto{\pgfqpoint{5.210165in}{1.211038in}}%
\pgfpathlineto{\pgfqpoint{5.211429in}{1.237308in}}%
\pgfpathlineto{\pgfqpoint{5.211850in}{1.211038in}}%
\pgfpathlineto{\pgfqpoint{5.212693in}{1.224173in}}%
\pgfpathlineto{\pgfqpoint{5.213114in}{1.237308in}}%
\pgfpathlineto{\pgfqpoint{5.213957in}{1.211038in}}%
\pgfpathlineto{\pgfqpoint{5.214378in}{1.224173in}}%
\pgfpathlineto{\pgfqpoint{5.214799in}{1.237308in}}%
\pgfpathlineto{\pgfqpoint{5.215221in}{1.224173in}}%
\pgfpathlineto{\pgfqpoint{5.216063in}{1.211038in}}%
\pgfpathlineto{\pgfqpoint{5.216906in}{1.237308in}}%
\pgfpathlineto{\pgfqpoint{5.217327in}{1.224173in}}%
\pgfpathlineto{\pgfqpoint{5.217748in}{1.224173in}}%
\pgfpathlineto{\pgfqpoint{5.219012in}{1.197903in}}%
\pgfpathlineto{\pgfqpoint{5.219433in}{1.237308in}}%
\pgfpathlineto{\pgfqpoint{5.220276in}{1.224173in}}%
\pgfpathlineto{\pgfqpoint{5.220697in}{1.237308in}}%
\pgfpathlineto{\pgfqpoint{5.221119in}{1.211038in}}%
\pgfpathlineto{\pgfqpoint{5.221961in}{1.224173in}}%
\pgfpathlineto{\pgfqpoint{5.222382in}{1.237308in}}%
\pgfpathlineto{\pgfqpoint{5.222804in}{1.211038in}}%
\pgfpathlineto{\pgfqpoint{5.223225in}{1.237308in}}%
\pgfpathlineto{\pgfqpoint{5.223646in}{1.250443in}}%
\pgfpathlineto{\pgfqpoint{5.224910in}{1.211038in}}%
\pgfpathlineto{\pgfqpoint{5.225331in}{1.237308in}}%
\pgfpathlineto{\pgfqpoint{5.225753in}{1.224173in}}%
\pgfpathlineto{\pgfqpoint{5.226595in}{1.211038in}}%
\pgfpathlineto{\pgfqpoint{5.227859in}{1.237308in}}%
\pgfpathlineto{\pgfqpoint{5.229123in}{1.211038in}}%
\pgfpathlineto{\pgfqpoint{5.229965in}{1.237308in}}%
\pgfpathlineto{\pgfqpoint{5.230387in}{1.197903in}}%
\pgfpathlineto{\pgfqpoint{5.230808in}{1.211038in}}%
\pgfpathlineto{\pgfqpoint{5.231229in}{1.237308in}}%
\pgfpathlineto{\pgfqpoint{5.231651in}{1.224173in}}%
\pgfpathlineto{\pgfqpoint{5.232072in}{1.211038in}}%
\pgfpathlineto{\pgfqpoint{5.232493in}{1.237308in}}%
\pgfpathlineto{\pgfqpoint{5.233336in}{1.224173in}}%
\pgfpathlineto{\pgfqpoint{5.233757in}{1.211038in}}%
\pgfpathlineto{\pgfqpoint{5.234178in}{1.224173in}}%
\pgfpathlineto{\pgfqpoint{5.234599in}{1.224173in}}%
\pgfpathlineto{\pgfqpoint{5.235442in}{1.237308in}}%
\pgfpathlineto{\pgfqpoint{5.237127in}{1.197903in}}%
\pgfpathlineto{\pgfqpoint{5.237970in}{1.250443in}}%
\pgfpathlineto{\pgfqpoint{5.238391in}{1.224173in}}%
\pgfpathlineto{\pgfqpoint{5.238812in}{1.224173in}}%
\pgfpathlineto{\pgfqpoint{5.240076in}{1.211038in}}%
\pgfpathlineto{\pgfqpoint{5.240919in}{1.224173in}}%
\pgfpathlineto{\pgfqpoint{5.241761in}{1.211038in}}%
\pgfpathlineto{\pgfqpoint{5.242604in}{1.237308in}}%
\pgfpathlineto{\pgfqpoint{5.243025in}{1.224173in}}%
\pgfpathlineto{\pgfqpoint{5.243446in}{1.197903in}}%
\pgfpathlineto{\pgfqpoint{5.243868in}{1.224173in}}%
\pgfpathlineto{\pgfqpoint{5.244289in}{1.237308in}}%
\pgfpathlineto{\pgfqpoint{5.244710in}{1.211038in}}%
\pgfpathlineto{\pgfqpoint{5.245131in}{1.224173in}}%
\pgfpathlineto{\pgfqpoint{5.245974in}{1.237308in}}%
\pgfpathlineto{\pgfqpoint{5.247659in}{1.211038in}}%
\pgfpathlineto{\pgfqpoint{5.248080in}{1.237308in}}%
\pgfpathlineto{\pgfqpoint{5.248502in}{1.197903in}}%
\pgfpathlineto{\pgfqpoint{5.249766in}{1.224173in}}%
\pgfpathlineto{\pgfqpoint{5.250187in}{1.197903in}}%
\pgfpathlineto{\pgfqpoint{5.250608in}{1.224173in}}%
\pgfpathlineto{\pgfqpoint{5.251029in}{1.237308in}}%
\pgfpathlineto{\pgfqpoint{5.251451in}{1.197903in}}%
\pgfpathlineto{\pgfqpoint{5.252293in}{1.211038in}}%
\pgfpathlineto{\pgfqpoint{5.252714in}{1.250443in}}%
\pgfpathlineto{\pgfqpoint{5.253136in}{1.237308in}}%
\pgfpathlineto{\pgfqpoint{5.254400in}{1.224173in}}%
\pgfpathlineto{\pgfqpoint{5.254821in}{1.237308in}}%
\pgfpathlineto{\pgfqpoint{5.255242in}{1.224173in}}%
\pgfpathlineto{\pgfqpoint{5.255663in}{1.224173in}}%
\pgfpathlineto{\pgfqpoint{5.256085in}{1.237308in}}%
\pgfpathlineto{\pgfqpoint{5.256506in}{1.211038in}}%
\pgfpathlineto{\pgfqpoint{5.256927in}{1.237308in}}%
\pgfpathlineto{\pgfqpoint{5.257349in}{1.237308in}}%
\pgfpathlineto{\pgfqpoint{5.257770in}{1.211038in}}%
\pgfpathlineto{\pgfqpoint{5.258191in}{1.224173in}}%
\pgfpathlineto{\pgfqpoint{5.258612in}{1.250443in}}%
\pgfpathlineto{\pgfqpoint{5.259034in}{1.224173in}}%
\pgfpathlineto{\pgfqpoint{5.259455in}{1.224173in}}%
\pgfpathlineto{\pgfqpoint{5.260719in}{1.250443in}}%
\pgfpathlineto{\pgfqpoint{5.261561in}{1.211038in}}%
\pgfpathlineto{\pgfqpoint{5.261983in}{1.224173in}}%
\pgfpathlineto{\pgfqpoint{5.262825in}{1.237308in}}%
\pgfpathlineto{\pgfqpoint{5.263246in}{1.224173in}}%
\pgfpathlineto{\pgfqpoint{5.263668in}{1.250443in}}%
\pgfpathlineto{\pgfqpoint{5.264089in}{1.224173in}}%
\pgfpathlineto{\pgfqpoint{5.265353in}{1.211038in}}%
\pgfpathlineto{\pgfqpoint{5.265774in}{1.237308in}}%
\pgfpathlineto{\pgfqpoint{5.266195in}{1.224173in}}%
\pgfpathlineto{\pgfqpoint{5.266617in}{1.211038in}}%
\pgfpathlineto{\pgfqpoint{5.267038in}{1.224173in}}%
\pgfpathlineto{\pgfqpoint{5.267459in}{1.237308in}}%
\pgfpathlineto{\pgfqpoint{5.267880in}{1.211038in}}%
\pgfpathlineto{\pgfqpoint{5.268723in}{1.224173in}}%
\pgfpathlineto{\pgfqpoint{5.269566in}{1.211038in}}%
\pgfpathlineto{\pgfqpoint{5.269987in}{1.237308in}}%
\pgfpathlineto{\pgfqpoint{5.270829in}{1.224173in}}%
\pgfpathlineto{\pgfqpoint{5.271251in}{1.237308in}}%
\pgfpathlineto{\pgfqpoint{5.271672in}{1.224173in}}%
\pgfpathlineto{\pgfqpoint{5.272093in}{1.224173in}}%
\pgfpathlineto{\pgfqpoint{5.273357in}{1.211038in}}%
\pgfpathlineto{\pgfqpoint{5.273778in}{1.237308in}}%
\pgfpathlineto{\pgfqpoint{5.274200in}{1.197903in}}%
\pgfpathlineto{\pgfqpoint{5.275885in}{1.224173in}}%
\pgfpathlineto{\pgfqpoint{5.276306in}{1.224173in}}%
\pgfpathlineto{\pgfqpoint{5.276727in}{1.211038in}}%
\pgfpathlineto{\pgfqpoint{5.277149in}{1.224173in}}%
\pgfpathlineto{\pgfqpoint{5.278412in}{1.237308in}}%
\pgfpathlineto{\pgfqpoint{5.279255in}{1.211038in}}%
\pgfpathlineto{\pgfqpoint{5.279676in}{1.224173in}}%
\pgfpathlineto{\pgfqpoint{5.280519in}{1.237308in}}%
\pgfpathlineto{\pgfqpoint{5.281783in}{1.197903in}}%
\pgfpathlineto{\pgfqpoint{5.283046in}{1.224173in}}%
\pgfpathlineto{\pgfqpoint{5.283468in}{1.224173in}}%
\pgfpathlineto{\pgfqpoint{5.283889in}{1.263578in}}%
\pgfpathlineto{\pgfqpoint{5.284310in}{1.211038in}}%
\pgfpathlineto{\pgfqpoint{5.285153in}{1.224173in}}%
\pgfpathlineto{\pgfqpoint{5.285574in}{1.197903in}}%
\pgfpathlineto{\pgfqpoint{5.285995in}{1.224173in}}%
\pgfpathlineto{\pgfqpoint{5.286417in}{1.250443in}}%
\pgfpathlineto{\pgfqpoint{5.286838in}{1.224173in}}%
\pgfpathlineto{\pgfqpoint{5.287259in}{1.224173in}}%
\pgfpathlineto{\pgfqpoint{5.287681in}{1.250443in}}%
\pgfpathlineto{\pgfqpoint{5.288523in}{1.237308in}}%
\pgfpathlineto{\pgfqpoint{5.288944in}{1.237308in}}%
\pgfpathlineto{\pgfqpoint{5.290629in}{1.211038in}}%
\pgfpathlineto{\pgfqpoint{5.291472in}{1.237308in}}%
\pgfpathlineto{\pgfqpoint{5.291893in}{1.211038in}}%
\pgfpathlineto{\pgfqpoint{5.292315in}{1.224173in}}%
\pgfpathlineto{\pgfqpoint{5.292736in}{1.237308in}}%
\pgfpathlineto{\pgfqpoint{5.294000in}{1.197903in}}%
\pgfpathlineto{\pgfqpoint{5.296106in}{1.237308in}}%
\pgfpathlineto{\pgfqpoint{5.297370in}{1.211038in}}%
\pgfpathlineto{\pgfqpoint{5.297791in}{1.237308in}}%
\pgfpathlineto{\pgfqpoint{5.298634in}{1.224173in}}%
\pgfpathlineto{\pgfqpoint{5.299055in}{1.237308in}}%
\pgfpathlineto{\pgfqpoint{5.299476in}{1.224173in}}%
\pgfpathlineto{\pgfqpoint{5.299898in}{1.211038in}}%
\pgfpathlineto{\pgfqpoint{5.300319in}{1.237308in}}%
\pgfpathlineto{\pgfqpoint{5.300740in}{1.211038in}}%
\pgfpathlineto{\pgfqpoint{5.301161in}{1.211038in}}%
\pgfpathlineto{\pgfqpoint{5.301583in}{1.237308in}}%
\pgfpathlineto{\pgfqpoint{5.302425in}{1.224173in}}%
\pgfpathlineto{\pgfqpoint{5.302847in}{1.211038in}}%
\pgfpathlineto{\pgfqpoint{5.303268in}{1.224173in}}%
\pgfpathlineto{\pgfqpoint{5.303689in}{1.237308in}}%
\pgfpathlineto{\pgfqpoint{5.304110in}{1.224173in}}%
\pgfpathlineto{\pgfqpoint{5.304532in}{1.211038in}}%
\pgfpathlineto{\pgfqpoint{5.304953in}{1.224173in}}%
\pgfpathlineto{\pgfqpoint{5.305374in}{1.237308in}}%
\pgfpathlineto{\pgfqpoint{5.305795in}{1.224173in}}%
\pgfpathlineto{\pgfqpoint{5.306638in}{1.224173in}}%
\pgfpathlineto{\pgfqpoint{5.307902in}{1.250443in}}%
\pgfpathlineto{\pgfqpoint{5.308323in}{1.211038in}}%
\pgfpathlineto{\pgfqpoint{5.309166in}{1.224173in}}%
\pgfpathlineto{\pgfqpoint{5.309587in}{1.211038in}}%
\pgfpathlineto{\pgfqpoint{5.310008in}{1.237308in}}%
\pgfpathlineto{\pgfqpoint{5.310430in}{1.224173in}}%
\pgfpathlineto{\pgfqpoint{5.310851in}{1.197903in}}%
\pgfpathlineto{\pgfqpoint{5.311272in}{1.224173in}}%
\pgfpathlineto{\pgfqpoint{5.312536in}{1.224173in}}%
\pgfpathlineto{\pgfqpoint{5.312957in}{1.237308in}}%
\pgfpathlineto{\pgfqpoint{5.313379in}{1.211038in}}%
\pgfpathlineto{\pgfqpoint{5.313800in}{1.224173in}}%
\pgfpathlineto{\pgfqpoint{5.314221in}{1.250443in}}%
\pgfpathlineto{\pgfqpoint{5.314642in}{1.211038in}}%
\pgfpathlineto{\pgfqpoint{5.315064in}{1.211038in}}%
\pgfpathlineto{\pgfqpoint{5.316327in}{1.237308in}}%
\pgfpathlineto{\pgfqpoint{5.316749in}{1.237308in}}%
\pgfpathlineto{\pgfqpoint{5.317170in}{1.211038in}}%
\pgfpathlineto{\pgfqpoint{5.317591in}{1.224173in}}%
\pgfpathlineto{\pgfqpoint{5.318013in}{1.237308in}}%
\pgfpathlineto{\pgfqpoint{5.318434in}{1.211038in}}%
\pgfpathlineto{\pgfqpoint{5.319276in}{1.224173in}}%
\pgfpathlineto{\pgfqpoint{5.319698in}{1.211038in}}%
\pgfpathlineto{\pgfqpoint{5.320540in}{1.237308in}}%
\pgfpathlineto{\pgfqpoint{5.321383in}{1.197903in}}%
\pgfpathlineto{\pgfqpoint{5.321804in}{1.237308in}}%
\pgfpathlineto{\pgfqpoint{5.322225in}{1.211038in}}%
\pgfpathlineto{\pgfqpoint{5.322647in}{1.197903in}}%
\pgfpathlineto{\pgfqpoint{5.323910in}{1.237308in}}%
\pgfpathlineto{\pgfqpoint{5.324332in}{1.237308in}}%
\pgfpathlineto{\pgfqpoint{5.324753in}{1.211038in}}%
\pgfpathlineto{\pgfqpoint{5.325596in}{1.224173in}}%
\pgfpathlineto{\pgfqpoint{5.326438in}{1.224173in}}%
\pgfpathlineto{\pgfqpoint{5.326859in}{1.237308in}}%
\pgfpathlineto{\pgfqpoint{5.327702in}{1.211038in}}%
\pgfpathlineto{\pgfqpoint{5.328123in}{1.237308in}}%
\pgfpathlineto{\pgfqpoint{5.328966in}{1.224173in}}%
\pgfpathlineto{\pgfqpoint{5.329387in}{1.224173in}}%
\pgfpathlineto{\pgfqpoint{5.329808in}{1.211038in}}%
\pgfpathlineto{\pgfqpoint{5.330230in}{1.224173in}}%
\pgfpathlineto{\pgfqpoint{5.330651in}{1.237308in}}%
\pgfpathlineto{\pgfqpoint{5.331072in}{1.224173in}}%
\pgfpathlineto{\pgfqpoint{5.331915in}{1.224173in}}%
\pgfpathlineto{\pgfqpoint{5.332336in}{1.211038in}}%
\pgfpathlineto{\pgfqpoint{5.333179in}{1.237308in}}%
\pgfpathlineto{\pgfqpoint{5.334021in}{1.211038in}}%
\pgfpathlineto{\pgfqpoint{5.334864in}{1.237308in}}%
\pgfpathlineto{\pgfqpoint{5.335285in}{1.224173in}}%
\pgfpathlineto{\pgfqpoint{5.336128in}{1.237308in}}%
\pgfpathlineto{\pgfqpoint{5.337391in}{1.211038in}}%
\pgfpathlineto{\pgfqpoint{5.338234in}{1.224173in}}%
\pgfpathlineto{\pgfqpoint{5.339076in}{1.211038in}}%
\pgfpathlineto{\pgfqpoint{5.340762in}{1.237308in}}%
\pgfpathlineto{\pgfqpoint{5.341604in}{1.224173in}}%
\pgfpathlineto{\pgfqpoint{5.342025in}{1.237308in}}%
\pgfpathlineto{\pgfqpoint{5.342447in}{1.211038in}}%
\pgfpathlineto{\pgfqpoint{5.342868in}{1.237308in}}%
\pgfpathlineto{\pgfqpoint{5.343289in}{1.237308in}}%
\pgfpathlineto{\pgfqpoint{5.343711in}{1.211038in}}%
\pgfpathlineto{\pgfqpoint{5.344553in}{1.224173in}}%
\pgfpathlineto{\pgfqpoint{5.344974in}{1.211038in}}%
\pgfpathlineto{\pgfqpoint{5.345817in}{1.250443in}}%
\pgfpathlineto{\pgfqpoint{5.346659in}{1.211038in}}%
\pgfpathlineto{\pgfqpoint{5.347081in}{1.250443in}}%
\pgfpathlineto{\pgfqpoint{5.347502in}{1.224173in}}%
\pgfpathlineto{\pgfqpoint{5.347923in}{1.224173in}}%
\pgfpathlineto{\pgfqpoint{5.348766in}{1.211038in}}%
\pgfpathlineto{\pgfqpoint{5.349608in}{1.237308in}}%
\pgfpathlineto{\pgfqpoint{5.350030in}{1.224173in}}%
\pgfpathlineto{\pgfqpoint{5.350872in}{1.224173in}}%
\pgfpathlineto{\pgfqpoint{5.351294in}{1.211038in}}%
\pgfpathlineto{\pgfqpoint{5.351715in}{1.224173in}}%
\pgfpathlineto{\pgfqpoint{5.352136in}{1.237308in}}%
\pgfpathlineto{\pgfqpoint{5.352557in}{1.224173in}}%
\pgfpathlineto{\pgfqpoint{5.352979in}{1.224173in}}%
\pgfpathlineto{\pgfqpoint{5.353400in}{1.237308in}}%
\pgfpathlineto{\pgfqpoint{5.353821in}{1.224173in}}%
\pgfpathlineto{\pgfqpoint{5.354242in}{1.224173in}}%
\pgfpathlineto{\pgfqpoint{5.355506in}{1.211038in}}%
\pgfpathlineto{\pgfqpoint{5.355928in}{1.237308in}}%
\pgfpathlineto{\pgfqpoint{5.356770in}{1.224173in}}%
\pgfpathlineto{\pgfqpoint{5.358034in}{1.237308in}}%
\pgfpathlineto{\pgfqpoint{5.358455in}{1.237308in}}%
\pgfpathlineto{\pgfqpoint{5.359298in}{1.211038in}}%
\pgfpathlineto{\pgfqpoint{5.359719in}{1.237308in}}%
\pgfpathlineto{\pgfqpoint{5.360140in}{1.224173in}}%
\pgfpathlineto{\pgfqpoint{5.361404in}{1.211038in}}%
\pgfpathlineto{\pgfqpoint{5.362247in}{1.237308in}}%
\pgfpathlineto{\pgfqpoint{5.363511in}{1.211038in}}%
\pgfpathlineto{\pgfqpoint{5.363932in}{1.211038in}}%
\pgfpathlineto{\pgfqpoint{5.364774in}{1.250443in}}%
\pgfpathlineto{\pgfqpoint{5.365196in}{1.237308in}}%
\pgfpathlineto{\pgfqpoint{5.366460in}{1.211038in}}%
\pgfpathlineto{\pgfqpoint{5.367302in}{1.237308in}}%
\pgfpathlineto{\pgfqpoint{5.367723in}{1.211038in}}%
\pgfpathlineto{\pgfqpoint{5.368145in}{1.237308in}}%
\pgfpathlineto{\pgfqpoint{5.368566in}{1.237308in}}%
\pgfpathlineto{\pgfqpoint{5.370251in}{1.211038in}}%
\pgfpathlineto{\pgfqpoint{5.371094in}{1.224173in}}%
\pgfpathlineto{\pgfqpoint{5.371936in}{1.211038in}}%
\pgfpathlineto{\pgfqpoint{5.372357in}{1.237308in}}%
\pgfpathlineto{\pgfqpoint{5.373200in}{1.224173in}}%
\pgfpathlineto{\pgfqpoint{5.373621in}{1.237308in}}%
\pgfpathlineto{\pgfqpoint{5.374043in}{1.211038in}}%
\pgfpathlineto{\pgfqpoint{5.374464in}{1.224173in}}%
\pgfpathlineto{\pgfqpoint{5.374885in}{1.237308in}}%
\pgfpathlineto{\pgfqpoint{5.375728in}{1.197903in}}%
\pgfpathlineto{\pgfqpoint{5.376149in}{1.224173in}}%
\pgfpathlineto{\pgfqpoint{5.376570in}{1.211038in}}%
\pgfpathlineto{\pgfqpoint{5.377413in}{1.237308in}}%
\pgfpathlineto{\pgfqpoint{5.377834in}{1.224173in}}%
\pgfpathlineto{\pgfqpoint{5.378677in}{1.211038in}}%
\pgfpathlineto{\pgfqpoint{5.379519in}{1.237308in}}%
\pgfpathlineto{\pgfqpoint{5.379940in}{1.224173in}}%
\pgfpathlineto{\pgfqpoint{5.380783in}{1.211038in}}%
\pgfpathlineto{\pgfqpoint{5.381204in}{1.237308in}}%
\pgfpathlineto{\pgfqpoint{5.382047in}{1.224173in}}%
\pgfpathlineto{\pgfqpoint{5.382468in}{1.224173in}}%
\pgfpathlineto{\pgfqpoint{5.382889in}{1.211038in}}%
\pgfpathlineto{\pgfqpoint{5.383732in}{1.237308in}}%
\pgfpathlineto{\pgfqpoint{5.384153in}{1.211038in}}%
\pgfpathlineto{\pgfqpoint{5.384575in}{1.224173in}}%
\pgfpathlineto{\pgfqpoint{5.385417in}{1.237308in}}%
\pgfpathlineto{\pgfqpoint{5.385838in}{1.197903in}}%
\pgfpathlineto{\pgfqpoint{5.386681in}{1.211038in}}%
\pgfpathlineto{\pgfqpoint{5.387523in}{1.237308in}}%
\pgfpathlineto{\pgfqpoint{5.387945in}{1.211038in}}%
\pgfpathlineto{\pgfqpoint{5.388787in}{1.224173in}}%
\pgfpathlineto{\pgfqpoint{5.389209in}{1.250443in}}%
\pgfpathlineto{\pgfqpoint{5.390051in}{1.237308in}}%
\pgfpathlineto{\pgfqpoint{5.390894in}{1.211038in}}%
\pgfpathlineto{\pgfqpoint{5.391315in}{1.224173in}}%
\pgfpathlineto{\pgfqpoint{5.391736in}{1.211038in}}%
\pgfpathlineto{\pgfqpoint{5.392158in}{1.250443in}}%
\pgfpathlineto{\pgfqpoint{5.392579in}{1.211038in}}%
\pgfpathlineto{\pgfqpoint{5.393000in}{1.211038in}}%
\pgfpathlineto{\pgfqpoint{5.393843in}{1.237308in}}%
\pgfpathlineto{\pgfqpoint{5.394685in}{1.211038in}}%
\pgfpathlineto{\pgfqpoint{5.395106in}{1.224173in}}%
\pgfpathlineto{\pgfqpoint{5.395949in}{1.211038in}}%
\pgfpathlineto{\pgfqpoint{5.396370in}{1.237308in}}%
\pgfpathlineto{\pgfqpoint{5.397213in}{1.224173in}}%
\pgfpathlineto{\pgfqpoint{5.397634in}{1.211038in}}%
\pgfpathlineto{\pgfqpoint{5.398055in}{1.224173in}}%
\pgfpathlineto{\pgfqpoint{5.398898in}{1.224173in}}%
\pgfpathlineto{\pgfqpoint{5.400162in}{1.237308in}}%
\pgfpathlineto{\pgfqpoint{5.400583in}{1.211038in}}%
\pgfpathlineto{\pgfqpoint{5.401426in}{1.224173in}}%
\pgfpathlineto{\pgfqpoint{5.401847in}{1.211038in}}%
\pgfpathlineto{\pgfqpoint{5.402268in}{1.237308in}}%
\pgfpathlineto{\pgfqpoint{5.403111in}{1.224173in}}%
\pgfpathlineto{\pgfqpoint{5.404375in}{1.224173in}}%
\pgfpathlineto{\pgfqpoint{5.404796in}{1.237308in}}%
\pgfpathlineto{\pgfqpoint{5.405217in}{1.224173in}}%
\pgfpathlineto{\pgfqpoint{5.405638in}{1.197903in}}%
\pgfpathlineto{\pgfqpoint{5.406060in}{1.211038in}}%
\pgfpathlineto{\pgfqpoint{5.407324in}{1.224173in}}%
\pgfpathlineto{\pgfqpoint{5.408166in}{1.224173in}}%
\pgfpathlineto{\pgfqpoint{5.409430in}{1.250443in}}%
\pgfpathlineto{\pgfqpoint{5.410694in}{1.211038in}}%
\pgfpathlineto{\pgfqpoint{5.411115in}{1.250443in}}%
\pgfpathlineto{\pgfqpoint{5.411536in}{1.224173in}}%
\pgfpathlineto{\pgfqpoint{5.411958in}{1.211038in}}%
\pgfpathlineto{\pgfqpoint{5.412379in}{1.224173in}}%
\pgfpathlineto{\pgfqpoint{5.412800in}{1.237308in}}%
\pgfpathlineto{\pgfqpoint{5.413221in}{1.211038in}}%
\pgfpathlineto{\pgfqpoint{5.414064in}{1.224173in}}%
\pgfpathlineto{\pgfqpoint{5.414485in}{1.211038in}}%
\pgfpathlineto{\pgfqpoint{5.415328in}{1.237308in}}%
\pgfpathlineto{\pgfqpoint{5.416170in}{1.197903in}}%
\pgfpathlineto{\pgfqpoint{5.416592in}{1.224173in}}%
\pgfpathlineto{\pgfqpoint{5.417013in}{1.211038in}}%
\pgfpathlineto{\pgfqpoint{5.417855in}{1.237308in}}%
\pgfpathlineto{\pgfqpoint{5.418277in}{1.211038in}}%
\pgfpathlineto{\pgfqpoint{5.418698in}{1.237308in}}%
\pgfpathlineto{\pgfqpoint{5.419962in}{1.237308in}}%
\pgfpathlineto{\pgfqpoint{5.420804in}{1.224173in}}%
\pgfpathlineto{\pgfqpoint{5.421647in}{1.250443in}}%
\pgfpathlineto{\pgfqpoint{5.422911in}{1.211038in}}%
\pgfpathlineto{\pgfqpoint{5.423332in}{1.211038in}}%
\pgfpathlineto{\pgfqpoint{5.424175in}{1.237308in}}%
\pgfpathlineto{\pgfqpoint{5.424596in}{1.211038in}}%
\pgfpathlineto{\pgfqpoint{5.425017in}{1.224173in}}%
\pgfpathlineto{\pgfqpoint{5.425438in}{1.237308in}}%
\pgfpathlineto{\pgfqpoint{5.426281in}{1.211038in}}%
\pgfpathlineto{\pgfqpoint{5.426702in}{1.224173in}}%
\pgfpathlineto{\pgfqpoint{5.427124in}{1.211038in}}%
\pgfpathlineto{\pgfqpoint{5.427966in}{1.237308in}}%
\pgfpathlineto{\pgfqpoint{5.428387in}{1.211038in}}%
\pgfpathlineto{\pgfqpoint{5.429230in}{1.224173in}}%
\pgfpathlineto{\pgfqpoint{5.429651in}{1.250443in}}%
\pgfpathlineto{\pgfqpoint{5.430494in}{1.237308in}}%
\pgfpathlineto{\pgfqpoint{5.431336in}{1.211038in}}%
\pgfpathlineto{\pgfqpoint{5.431758in}{1.250443in}}%
\pgfpathlineto{\pgfqpoint{5.432179in}{1.224173in}}%
\pgfpathlineto{\pgfqpoint{5.433021in}{1.224173in}}%
\pgfpathlineto{\pgfqpoint{5.433443in}{1.211038in}}%
\pgfpathlineto{\pgfqpoint{5.433864in}{1.237308in}}%
\pgfpathlineto{\pgfqpoint{5.434285in}{1.224173in}}%
\pgfpathlineto{\pgfqpoint{5.435128in}{1.211038in}}%
\pgfpathlineto{\pgfqpoint{5.436813in}{1.237308in}}%
\pgfpathlineto{\pgfqpoint{5.437234in}{1.211038in}}%
\pgfpathlineto{\pgfqpoint{5.437656in}{1.224173in}}%
\pgfpathlineto{\pgfqpoint{5.438077in}{1.250443in}}%
\pgfpathlineto{\pgfqpoint{5.438498in}{1.211038in}}%
\pgfpathlineto{\pgfqpoint{5.440183in}{1.237308in}}%
\pgfpathlineto{\pgfqpoint{5.441026in}{1.237308in}}%
\pgfpathlineto{\pgfqpoint{5.442290in}{1.224173in}}%
\pgfpathlineto{\pgfqpoint{5.442711in}{1.237308in}}%
\pgfpathlineto{\pgfqpoint{5.443553in}{1.211038in}}%
\pgfpathlineto{\pgfqpoint{5.443975in}{1.224173in}}%
\pgfpathlineto{\pgfqpoint{5.444396in}{1.237308in}}%
\pgfpathlineto{\pgfqpoint{5.444817in}{1.224173in}}%
\pgfpathlineto{\pgfqpoint{5.445239in}{1.211038in}}%
\pgfpathlineto{\pgfqpoint{5.446924in}{1.250443in}}%
\pgfpathlineto{\pgfqpoint{5.447345in}{1.211038in}}%
\pgfpathlineto{\pgfqpoint{5.448188in}{1.224173in}}%
\pgfpathlineto{\pgfqpoint{5.448609in}{1.224173in}}%
\pgfpathlineto{\pgfqpoint{5.449451in}{1.250443in}}%
\pgfpathlineto{\pgfqpoint{5.451136in}{1.211038in}}%
\pgfpathlineto{\pgfqpoint{5.451558in}{1.211038in}}%
\pgfpathlineto{\pgfqpoint{5.453243in}{1.237308in}}%
\pgfpathlineto{\pgfqpoint{5.454507in}{1.197903in}}%
\pgfpathlineto{\pgfqpoint{5.455349in}{1.237308in}}%
\pgfpathlineto{\pgfqpoint{5.456613in}{1.211038in}}%
\pgfpathlineto{\pgfqpoint{5.457034in}{1.237308in}}%
\pgfpathlineto{\pgfqpoint{5.457877in}{1.224173in}}%
\pgfpathlineto{\pgfqpoint{5.458298in}{1.224173in}}%
\pgfpathlineto{\pgfqpoint{5.458719in}{1.211038in}}%
\pgfpathlineto{\pgfqpoint{5.459141in}{1.224173in}}%
\pgfpathlineto{\pgfqpoint{5.459562in}{1.237308in}}%
\pgfpathlineto{\pgfqpoint{5.459983in}{1.224173in}}%
\pgfpathlineto{\pgfqpoint{5.461247in}{1.211038in}}%
\pgfpathlineto{\pgfqpoint{5.461668in}{1.237308in}}%
\pgfpathlineto{\pgfqpoint{5.462090in}{1.224173in}}%
\pgfpathlineto{\pgfqpoint{5.462932in}{1.211038in}}%
\pgfpathlineto{\pgfqpoint{5.463775in}{1.237308in}}%
\pgfpathlineto{\pgfqpoint{5.464196in}{1.224173in}}%
\pgfpathlineto{\pgfqpoint{5.464617in}{1.237308in}}%
\pgfpathlineto{\pgfqpoint{5.465881in}{1.197903in}}%
\pgfpathlineto{\pgfqpoint{5.467566in}{1.224173in}}%
\pgfpathlineto{\pgfqpoint{5.467988in}{1.224173in}}%
\pgfpathlineto{\pgfqpoint{5.468409in}{1.197903in}}%
\pgfpathlineto{\pgfqpoint{5.468830in}{1.211038in}}%
\pgfpathlineto{\pgfqpoint{5.470094in}{1.237308in}}%
\pgfpathlineto{\pgfqpoint{5.471358in}{1.224173in}}%
\pgfpathlineto{\pgfqpoint{5.472200in}{1.237308in}}%
\pgfpathlineto{\pgfqpoint{5.472622in}{1.197903in}}%
\pgfpathlineto{\pgfqpoint{5.473043in}{1.224173in}}%
\pgfpathlineto{\pgfqpoint{5.473464in}{1.224173in}}%
\pgfpathlineto{\pgfqpoint{5.473885in}{1.211038in}}%
\pgfpathlineto{\pgfqpoint{5.474307in}{1.237308in}}%
\pgfpathlineto{\pgfqpoint{5.475149in}{1.224173in}}%
\pgfpathlineto{\pgfqpoint{5.475992in}{1.237308in}}%
\pgfpathlineto{\pgfqpoint{5.477256in}{1.197903in}}%
\pgfpathlineto{\pgfqpoint{5.478520in}{1.237308in}}%
\pgfpathlineto{\pgfqpoint{5.478941in}{1.211038in}}%
\pgfpathlineto{\pgfqpoint{5.479362in}{1.237308in}}%
\pgfpathlineto{\pgfqpoint{5.479783in}{1.237308in}}%
\pgfpathlineto{\pgfqpoint{5.480205in}{1.211038in}}%
\pgfpathlineto{\pgfqpoint{5.481047in}{1.224173in}}%
\pgfpathlineto{\pgfqpoint{5.483154in}{1.224173in}}%
\pgfpathlineto{\pgfqpoint{5.483575in}{1.237308in}}%
\pgfpathlineto{\pgfqpoint{5.483996in}{1.211038in}}%
\pgfpathlineto{\pgfqpoint{5.484839in}{1.224173in}}%
\pgfpathlineto{\pgfqpoint{5.485260in}{1.211038in}}%
\pgfpathlineto{\pgfqpoint{5.485681in}{1.224173in}}%
\pgfpathlineto{\pgfqpoint{5.486103in}{1.224173in}}%
\pgfpathlineto{\pgfqpoint{5.486524in}{1.211038in}}%
\pgfpathlineto{\pgfqpoint{5.486945in}{1.224173in}}%
\pgfpathlineto{\pgfqpoint{5.487366in}{1.237308in}}%
\pgfpathlineto{\pgfqpoint{5.489051in}{1.184768in}}%
\pgfpathlineto{\pgfqpoint{5.490315in}{1.224173in}}%
\pgfpathlineto{\pgfqpoint{5.491158in}{1.250443in}}%
\pgfpathlineto{\pgfqpoint{5.492000in}{1.211038in}}%
\pgfpathlineto{\pgfqpoint{5.492422in}{1.237308in}}%
\pgfpathlineto{\pgfqpoint{5.493264in}{1.224173in}}%
\pgfpathlineto{\pgfqpoint{5.493686in}{1.237308in}}%
\pgfpathlineto{\pgfqpoint{5.494107in}{1.211038in}}%
\pgfpathlineto{\pgfqpoint{5.494528in}{1.224173in}}%
\pgfpathlineto{\pgfqpoint{5.494949in}{1.237308in}}%
\pgfpathlineto{\pgfqpoint{5.495371in}{0.738175in}}%
\pgfpathlineto{\pgfqpoint{5.495792in}{1.237308in}}%
\pgfpathlineto{\pgfqpoint{5.496634in}{1.211038in}}%
\pgfpathlineto{\pgfqpoint{5.497056in}{1.224173in}}%
\pgfpathlineto{\pgfqpoint{5.497477in}{1.237308in}}%
\pgfpathlineto{\pgfqpoint{5.497898in}{1.211038in}}%
\pgfpathlineto{\pgfqpoint{5.498741in}{1.224173in}}%
\pgfpathlineto{\pgfqpoint{5.499162in}{1.211038in}}%
\pgfpathlineto{\pgfqpoint{5.499583in}{1.250443in}}%
\pgfpathlineto{\pgfqpoint{5.500005in}{1.237308in}}%
\pgfpathlineto{\pgfqpoint{5.500426in}{1.211038in}}%
\pgfpathlineto{\pgfqpoint{5.500847in}{1.224173in}}%
\pgfpathlineto{\pgfqpoint{5.502111in}{1.237308in}}%
\pgfpathlineto{\pgfqpoint{5.502532in}{1.237308in}}%
\pgfpathlineto{\pgfqpoint{5.502954in}{1.211038in}}%
\pgfpathlineto{\pgfqpoint{5.503375in}{1.224173in}}%
\pgfpathlineto{\pgfqpoint{5.503796in}{1.237308in}}%
\pgfpathlineto{\pgfqpoint{5.504639in}{1.211038in}}%
\pgfpathlineto{\pgfqpoint{5.505481in}{1.250443in}}%
\pgfpathlineto{\pgfqpoint{5.505903in}{1.224173in}}%
\pgfpathlineto{\pgfqpoint{5.507166in}{1.211038in}}%
\pgfpathlineto{\pgfqpoint{5.507588in}{1.237308in}}%
\pgfpathlineto{\pgfqpoint{5.508009in}{1.224173in}}%
\pgfpathlineto{\pgfqpoint{5.508430in}{1.211038in}}%
\pgfpathlineto{\pgfqpoint{5.508852in}{1.710172in}}%
\pgfpathlineto{\pgfqpoint{5.509273in}{1.211038in}}%
\pgfpathlineto{\pgfqpoint{5.510115in}{1.237308in}}%
\pgfpathlineto{\pgfqpoint{5.510537in}{1.211038in}}%
\pgfpathlineto{\pgfqpoint{5.511379in}{1.224173in}}%
\pgfpathlineto{\pgfqpoint{5.511801in}{1.211038in}}%
\pgfpathlineto{\pgfqpoint{5.512222in}{1.237308in}}%
\pgfpathlineto{\pgfqpoint{5.512643in}{1.224173in}}%
\pgfpathlineto{\pgfqpoint{5.513064in}{1.197903in}}%
\pgfpathlineto{\pgfqpoint{5.513486in}{1.224173in}}%
\pgfpathlineto{\pgfqpoint{5.515171in}{1.224173in}}%
\pgfpathlineto{\pgfqpoint{5.515592in}{1.211038in}}%
\pgfpathlineto{\pgfqpoint{5.516013in}{1.224173in}}%
\pgfpathlineto{\pgfqpoint{5.516435in}{1.237308in}}%
\pgfpathlineto{\pgfqpoint{5.516856in}{1.224173in}}%
\pgfpathlineto{\pgfqpoint{5.517277in}{1.211038in}}%
\pgfpathlineto{\pgfqpoint{5.517698in}{1.224173in}}%
\pgfpathlineto{\pgfqpoint{5.518120in}{1.224173in}}%
\pgfpathlineto{\pgfqpoint{5.518962in}{1.197903in}}%
\pgfpathlineto{\pgfqpoint{5.519384in}{1.211038in}}%
\pgfpathlineto{\pgfqpoint{5.519805in}{1.237308in}}%
\pgfpathlineto{\pgfqpoint{5.520647in}{1.224173in}}%
\pgfpathlineto{\pgfqpoint{5.521911in}{1.224173in}}%
\pgfpathlineto{\pgfqpoint{5.522754in}{1.237308in}}%
\pgfpathlineto{\pgfqpoint{5.524018in}{1.211038in}}%
\pgfpathlineto{\pgfqpoint{5.524439in}{1.211038in}}%
\pgfpathlineto{\pgfqpoint{5.525281in}{1.237308in}}%
\pgfpathlineto{\pgfqpoint{5.525703in}{1.211038in}}%
\pgfpathlineto{\pgfqpoint{5.526124in}{1.224173in}}%
\pgfpathlineto{\pgfqpoint{5.527388in}{1.237308in}}%
\pgfpathlineto{\pgfqpoint{5.527809in}{1.211038in}}%
\pgfpathlineto{\pgfqpoint{5.528652in}{1.224173in}}%
\pgfpathlineto{\pgfqpoint{5.529073in}{1.224173in}}%
\pgfpathlineto{\pgfqpoint{5.529494in}{1.237308in}}%
\pgfpathlineto{\pgfqpoint{5.530758in}{1.211038in}}%
\pgfpathlineto{\pgfqpoint{5.531179in}{1.237308in}}%
\pgfpathlineto{\pgfqpoint{5.531601in}{1.197903in}}%
\pgfpathlineto{\pgfqpoint{5.532022in}{1.224173in}}%
\pgfpathlineto{\pgfqpoint{5.532864in}{1.224173in}}%
\pgfpathlineto{\pgfqpoint{5.533286in}{1.211038in}}%
\pgfpathlineto{\pgfqpoint{5.533707in}{1.224173in}}%
\pgfpathlineto{\pgfqpoint{5.534128in}{1.224173in}}%
\pgfpathlineto{\pgfqpoint{5.534550in}{1.211038in}}%
\pgfpathlineto{\pgfqpoint{5.534971in}{1.224173in}}%
\pgfpathlineto{\pgfqpoint{5.535392in}{1.237308in}}%
\pgfpathlineto{\pgfqpoint{5.535813in}{1.224173in}}%
\pgfpathlineto{\pgfqpoint{5.536656in}{1.224173in}}%
\pgfpathlineto{\pgfqpoint{5.537077in}{1.211038in}}%
\pgfpathlineto{\pgfqpoint{5.537920in}{1.237308in}}%
\pgfpathlineto{\pgfqpoint{5.538341in}{1.211038in}}%
\pgfpathlineto{\pgfqpoint{5.538762in}{1.224173in}}%
\pgfpathlineto{\pgfqpoint{5.539184in}{1.237308in}}%
\pgfpathlineto{\pgfqpoint{5.539605in}{1.224173in}}%
\pgfpathlineto{\pgfqpoint{5.540026in}{1.224173in}}%
\pgfpathlineto{\pgfqpoint{5.540869in}{1.211038in}}%
\pgfpathlineto{\pgfqpoint{5.542554in}{1.237308in}}%
\pgfpathlineto{\pgfqpoint{5.543818in}{1.224173in}}%
\pgfpathlineto{\pgfqpoint{5.545081in}{1.250443in}}%
\pgfpathlineto{\pgfqpoint{5.547188in}{1.211038in}}%
\pgfpathlineto{\pgfqpoint{5.548030in}{1.237308in}}%
\pgfpathlineto{\pgfqpoint{5.549294in}{1.211038in}}%
\pgfpathlineto{\pgfqpoint{5.549716in}{1.211038in}}%
\pgfpathlineto{\pgfqpoint{5.550137in}{1.250443in}}%
\pgfpathlineto{\pgfqpoint{5.550558in}{1.224173in}}%
\pgfpathlineto{\pgfqpoint{5.551822in}{1.224173in}}%
\pgfpathlineto{\pgfqpoint{5.552664in}{1.211038in}}%
\pgfpathlineto{\pgfqpoint{5.553928in}{1.237308in}}%
\pgfpathlineto{\pgfqpoint{5.554771in}{1.224173in}}%
\pgfpathlineto{\pgfqpoint{5.555613in}{1.237308in}}%
\pgfpathlineto{\pgfqpoint{5.556035in}{1.211038in}}%
\pgfpathlineto{\pgfqpoint{5.556877in}{1.224173in}}%
\pgfpathlineto{\pgfqpoint{5.558141in}{1.211038in}}%
\pgfpathlineto{\pgfqpoint{5.558562in}{1.211038in}}%
\pgfpathlineto{\pgfqpoint{5.560247in}{1.237308in}}%
\pgfpathlineto{\pgfqpoint{5.560669in}{1.237308in}}%
\pgfpathlineto{\pgfqpoint{5.561090in}{1.211038in}}%
\pgfpathlineto{\pgfqpoint{5.561511in}{1.224173in}}%
\pgfpathlineto{\pgfqpoint{5.562354in}{1.263578in}}%
\pgfpathlineto{\pgfqpoint{5.562775in}{1.237308in}}%
\pgfpathlineto{\pgfqpoint{5.563196in}{1.237308in}}%
\pgfpathlineto{\pgfqpoint{5.563618in}{1.197903in}}%
\pgfpathlineto{\pgfqpoint{5.564039in}{1.237308in}}%
\pgfpathlineto{\pgfqpoint{5.564460in}{1.250443in}}%
\pgfpathlineto{\pgfqpoint{5.565724in}{1.224173in}}%
\pgfpathlineto{\pgfqpoint{5.567830in}{1.224173in}}%
\pgfpathlineto{\pgfqpoint{5.568252in}{1.237308in}}%
\pgfpathlineto{\pgfqpoint{5.569094in}{1.211038in}}%
\pgfpathlineto{\pgfqpoint{5.570358in}{1.237308in}}%
\pgfpathlineto{\pgfqpoint{5.571622in}{1.237308in}}%
\pgfpathlineto{\pgfqpoint{5.572043in}{1.224173in}}%
\pgfpathlineto{\pgfqpoint{5.572465in}{1.237308in}}%
\pgfpathlineto{\pgfqpoint{5.572886in}{1.237308in}}%
\pgfpathlineto{\pgfqpoint{5.574150in}{1.211038in}}%
\pgfpathlineto{\pgfqpoint{5.575414in}{1.237308in}}%
\pgfpathlineto{\pgfqpoint{5.575835in}{1.184768in}}%
\pgfpathlineto{\pgfqpoint{5.576256in}{1.211038in}}%
\pgfpathlineto{\pgfqpoint{5.577099in}{1.237308in}}%
\pgfpathlineto{\pgfqpoint{5.577520in}{1.224173in}}%
\pgfpathlineto{\pgfqpoint{5.578784in}{1.211038in}}%
\pgfpathlineto{\pgfqpoint{5.580469in}{1.237308in}}%
\pgfpathlineto{\pgfqpoint{5.580890in}{1.237308in}}%
\pgfpathlineto{\pgfqpoint{5.582154in}{1.224173in}}%
\pgfpathlineto{\pgfqpoint{5.582997in}{1.224173in}}%
\pgfpathlineto{\pgfqpoint{5.583418in}{1.237308in}}%
\pgfpathlineto{\pgfqpoint{5.583839in}{1.224173in}}%
\pgfpathlineto{\pgfqpoint{5.585103in}{1.224173in}}%
\pgfpathlineto{\pgfqpoint{5.585524in}{1.237308in}}%
\pgfpathlineto{\pgfqpoint{5.586367in}{1.211038in}}%
\pgfpathlineto{\pgfqpoint{5.586788in}{1.224173in}}%
\pgfpathlineto{\pgfqpoint{5.588473in}{1.224173in}}%
\pgfpathlineto{\pgfqpoint{5.588894in}{1.211038in}}%
\pgfpathlineto{\pgfqpoint{5.589737in}{1.237308in}}%
\pgfpathlineto{\pgfqpoint{5.591001in}{1.211038in}}%
\pgfpathlineto{\pgfqpoint{5.591422in}{1.211038in}}%
\pgfpathlineto{\pgfqpoint{5.592686in}{1.237308in}}%
\pgfpathlineto{\pgfqpoint{5.594371in}{1.211038in}}%
\pgfpathlineto{\pgfqpoint{5.596056in}{1.237308in}}%
\pgfpathlineto{\pgfqpoint{5.597320in}{1.211038in}}%
\pgfpathlineto{\pgfqpoint{5.598584in}{1.224173in}}%
\pgfpathlineto{\pgfqpoint{5.599426in}{1.224173in}}%
\pgfpathlineto{\pgfqpoint{5.599848in}{1.237308in}}%
\pgfpathlineto{\pgfqpoint{5.600269in}{1.211038in}}%
\pgfpathlineto{\pgfqpoint{5.601111in}{1.224173in}}%
\pgfpathlineto{\pgfqpoint{5.602375in}{1.224173in}}%
\pgfpathlineto{\pgfqpoint{5.602797in}{1.211038in}}%
\pgfpathlineto{\pgfqpoint{5.603218in}{1.224173in}}%
\pgfpathlineto{\pgfqpoint{5.604482in}{1.237308in}}%
\pgfpathlineto{\pgfqpoint{5.604903in}{1.237308in}}%
\pgfpathlineto{\pgfqpoint{5.606588in}{1.211038in}}%
\pgfpathlineto{\pgfqpoint{5.607009in}{1.237308in}}%
\pgfpathlineto{\pgfqpoint{5.607431in}{1.211038in}}%
\pgfpathlineto{\pgfqpoint{5.607852in}{1.211038in}}%
\pgfpathlineto{\pgfqpoint{5.608694in}{1.224173in}}%
\pgfpathlineto{\pgfqpoint{5.609116in}{1.211038in}}%
\pgfpathlineto{\pgfqpoint{5.609537in}{1.224173in}}%
\pgfpathlineto{\pgfqpoint{5.609958in}{1.224173in}}%
\pgfpathlineto{\pgfqpoint{5.610380in}{1.211038in}}%
\pgfpathlineto{\pgfqpoint{5.611643in}{1.237308in}}%
\pgfpathlineto{\pgfqpoint{5.612907in}{1.211038in}}%
\pgfpathlineto{\pgfqpoint{5.613329in}{1.237308in}}%
\pgfpathlineto{\pgfqpoint{5.613750in}{1.224173in}}%
\pgfusepath{stroke}%
\end{pgfscope}%
\begin{pgfscope}%
\pgfpathrectangle{\pgfqpoint{0.885050in}{0.587778in}}{\pgfqpoint{4.955200in}{1.285926in}}%
\pgfusepath{clip}%
\pgfsetrectcap%
\pgfsetroundjoin%
\pgfsetlinewidth{1.505625pt}%
\definecolor{currentstroke}{rgb}{1.000000,0.145098,0.145098}%
\pgfsetstrokecolor{currentstroke}%
\pgfsetdash{}{0pt}%
\pgfpathmoveto{\pgfqpoint{1.110287in}{1.243876in}}%
\pgfpathlineto{\pgfqpoint{1.112393in}{1.211038in}}%
\pgfpathlineto{\pgfqpoint{1.112814in}{1.237308in}}%
\pgfpathlineto{\pgfqpoint{1.113657in}{1.230741in}}%
\pgfpathlineto{\pgfqpoint{1.114078in}{1.237308in}}%
\pgfpathlineto{\pgfqpoint{1.114499in}{1.230741in}}%
\pgfpathlineto{\pgfqpoint{1.114921in}{1.211038in}}%
\pgfpathlineto{\pgfqpoint{1.115342in}{1.237308in}}%
\pgfpathlineto{\pgfqpoint{1.115763in}{1.237308in}}%
\pgfpathlineto{\pgfqpoint{1.117027in}{1.217606in}}%
\pgfpathlineto{\pgfqpoint{1.117448in}{1.224173in}}%
\pgfpathlineto{\pgfqpoint{1.117870in}{1.217606in}}%
\pgfpathlineto{\pgfqpoint{1.118291in}{1.211038in}}%
\pgfpathlineto{\pgfqpoint{1.119133in}{1.243876in}}%
\pgfpathlineto{\pgfqpoint{1.119976in}{1.230741in}}%
\pgfpathlineto{\pgfqpoint{1.120819in}{1.217606in}}%
\pgfpathlineto{\pgfqpoint{1.121240in}{1.243876in}}%
\pgfpathlineto{\pgfqpoint{1.121661in}{1.211038in}}%
\pgfpathlineto{\pgfqpoint{1.123346in}{1.243876in}}%
\pgfpathlineto{\pgfqpoint{1.125031in}{1.204471in}}%
\pgfpathlineto{\pgfqpoint{1.126295in}{1.224173in}}%
\pgfpathlineto{\pgfqpoint{1.127138in}{1.204471in}}%
\pgfpathlineto{\pgfqpoint{1.128402in}{1.217606in}}%
\pgfpathlineto{\pgfqpoint{1.128823in}{1.178200in}}%
\pgfpathlineto{\pgfqpoint{1.129244in}{1.224173in}}%
\pgfpathlineto{\pgfqpoint{1.130087in}{1.204471in}}%
\pgfpathlineto{\pgfqpoint{1.131350in}{1.217606in}}%
\pgfpathlineto{\pgfqpoint{1.132614in}{1.191335in}}%
\pgfpathlineto{\pgfqpoint{1.133036in}{1.217606in}}%
\pgfpathlineto{\pgfqpoint{1.133457in}{1.211038in}}%
\pgfpathlineto{\pgfqpoint{1.134299in}{1.217606in}}%
\pgfpathlineto{\pgfqpoint{1.134721in}{1.197903in}}%
\pgfpathlineto{\pgfqpoint{1.135985in}{1.230741in}}%
\pgfpathlineto{\pgfqpoint{1.136406in}{1.191335in}}%
\pgfpathlineto{\pgfqpoint{1.137248in}{1.197903in}}%
\pgfpathlineto{\pgfqpoint{1.137670in}{1.191335in}}%
\pgfpathlineto{\pgfqpoint{1.139355in}{1.230741in}}%
\pgfpathlineto{\pgfqpoint{1.139776in}{1.217606in}}%
\pgfpathlineto{\pgfqpoint{1.140197in}{1.224173in}}%
\pgfpathlineto{\pgfqpoint{1.140619in}{1.237308in}}%
\pgfpathlineto{\pgfqpoint{1.141040in}{1.224173in}}%
\pgfpathlineto{\pgfqpoint{1.141461in}{1.217606in}}%
\pgfpathlineto{\pgfqpoint{1.141882in}{1.224173in}}%
\pgfpathlineto{\pgfqpoint{1.142304in}{1.237308in}}%
\pgfpathlineto{\pgfqpoint{1.142725in}{1.197903in}}%
\pgfpathlineto{\pgfqpoint{1.143146in}{1.217606in}}%
\pgfpathlineto{\pgfqpoint{1.143568in}{1.230741in}}%
\pgfpathlineto{\pgfqpoint{1.144410in}{1.224173in}}%
\pgfpathlineto{\pgfqpoint{1.144831in}{1.217606in}}%
\pgfpathlineto{\pgfqpoint{1.145674in}{1.237308in}}%
\pgfpathlineto{\pgfqpoint{1.146095in}{1.230741in}}%
\pgfpathlineto{\pgfqpoint{1.146938in}{1.217606in}}%
\pgfpathlineto{\pgfqpoint{1.147359in}{1.243876in}}%
\pgfpathlineto{\pgfqpoint{1.147780in}{1.211038in}}%
\pgfpathlineto{\pgfqpoint{1.148202in}{1.211038in}}%
\pgfpathlineto{\pgfqpoint{1.148623in}{1.217606in}}%
\pgfpathlineto{\pgfqpoint{1.149044in}{1.250443in}}%
\pgfpathlineto{\pgfqpoint{1.149887in}{1.237308in}}%
\pgfpathlineto{\pgfqpoint{1.150308in}{1.237308in}}%
\pgfpathlineto{\pgfqpoint{1.150729in}{1.224173in}}%
\pgfpathlineto{\pgfqpoint{1.151151in}{1.335822in}}%
\pgfpathlineto{\pgfqpoint{1.151572in}{1.224173in}}%
\pgfpathlineto{\pgfqpoint{1.151993in}{1.250443in}}%
\pgfpathlineto{\pgfqpoint{1.152414in}{1.237308in}}%
\pgfpathlineto{\pgfqpoint{1.152836in}{1.230741in}}%
\pgfpathlineto{\pgfqpoint{1.153257in}{1.243876in}}%
\pgfpathlineto{\pgfqpoint{1.153678in}{1.237308in}}%
\pgfpathlineto{\pgfqpoint{1.154100in}{1.224173in}}%
\pgfpathlineto{\pgfqpoint{1.154521in}{1.230741in}}%
\pgfpathlineto{\pgfqpoint{1.155785in}{1.243876in}}%
\pgfpathlineto{\pgfqpoint{1.156206in}{1.237308in}}%
\pgfpathlineto{\pgfqpoint{1.156627in}{1.257011in}}%
\pgfpathlineto{\pgfqpoint{1.157470in}{1.224173in}}%
\pgfpathlineto{\pgfqpoint{1.157891in}{1.243876in}}%
\pgfpathlineto{\pgfqpoint{1.159155in}{1.217606in}}%
\pgfpathlineto{\pgfqpoint{1.159576in}{1.224173in}}%
\pgfpathlineto{\pgfqpoint{1.161261in}{1.250443in}}%
\pgfpathlineto{\pgfqpoint{1.161683in}{1.243876in}}%
\pgfpathlineto{\pgfqpoint{1.162104in}{1.243876in}}%
\pgfpathlineto{\pgfqpoint{1.162525in}{1.197903in}}%
\pgfpathlineto{\pgfqpoint{1.162946in}{1.224173in}}%
\pgfpathlineto{\pgfqpoint{1.163368in}{1.243876in}}%
\pgfpathlineto{\pgfqpoint{1.164631in}{1.119092in}}%
\pgfpathlineto{\pgfqpoint{1.165053in}{1.230741in}}%
\pgfpathlineto{\pgfqpoint{1.165895in}{1.217606in}}%
\pgfpathlineto{\pgfqpoint{1.166317in}{1.230741in}}%
\pgfpathlineto{\pgfqpoint{1.166738in}{1.197903in}}%
\pgfpathlineto{\pgfqpoint{1.167159in}{1.224173in}}%
\pgfpathlineto{\pgfqpoint{1.167580in}{1.230741in}}%
\pgfpathlineto{\pgfqpoint{1.168002in}{1.217606in}}%
\pgfpathlineto{\pgfqpoint{1.168423in}{1.230741in}}%
\pgfpathlineto{\pgfqpoint{1.168844in}{1.230741in}}%
\pgfpathlineto{\pgfqpoint{1.169266in}{1.217606in}}%
\pgfpathlineto{\pgfqpoint{1.170529in}{1.250443in}}%
\pgfpathlineto{\pgfqpoint{1.170951in}{1.211038in}}%
\pgfpathlineto{\pgfqpoint{1.171793in}{1.224173in}}%
\pgfpathlineto{\pgfqpoint{1.173057in}{1.230741in}}%
\pgfpathlineto{\pgfqpoint{1.174742in}{1.217606in}}%
\pgfpathlineto{\pgfqpoint{1.175585in}{1.211038in}}%
\pgfpathlineto{\pgfqpoint{1.176427in}{1.237308in}}%
\pgfpathlineto{\pgfqpoint{1.177270in}{1.224173in}}%
\pgfpathlineto{\pgfqpoint{1.177691in}{1.237308in}}%
\pgfpathlineto{\pgfqpoint{1.178112in}{1.230741in}}%
\pgfpathlineto{\pgfqpoint{1.178955in}{1.217606in}}%
\pgfpathlineto{\pgfqpoint{1.180219in}{1.243876in}}%
\pgfpathlineto{\pgfqpoint{1.181483in}{1.211038in}}%
\pgfpathlineto{\pgfqpoint{1.181904in}{1.224173in}}%
\pgfpathlineto{\pgfqpoint{1.182746in}{1.217606in}}%
\pgfpathlineto{\pgfqpoint{1.183168in}{1.197903in}}%
\pgfpathlineto{\pgfqpoint{1.183589in}{1.217606in}}%
\pgfpathlineto{\pgfqpoint{1.184010in}{1.211038in}}%
\pgfpathlineto{\pgfqpoint{1.184432in}{1.230741in}}%
\pgfpathlineto{\pgfqpoint{1.185274in}{1.224173in}}%
\pgfpathlineto{\pgfqpoint{1.185695in}{1.237308in}}%
\pgfpathlineto{\pgfqpoint{1.186117in}{1.230741in}}%
\pgfpathlineto{\pgfqpoint{1.186959in}{1.217606in}}%
\pgfpathlineto{\pgfqpoint{1.187380in}{1.237308in}}%
\pgfpathlineto{\pgfqpoint{1.187802in}{1.217606in}}%
\pgfpathlineto{\pgfqpoint{1.189066in}{1.230741in}}%
\pgfpathlineto{\pgfqpoint{1.189908in}{1.204471in}}%
\pgfpathlineto{\pgfqpoint{1.190751in}{1.270146in}}%
\pgfpathlineto{\pgfqpoint{1.191172in}{1.243876in}}%
\pgfpathlineto{\pgfqpoint{1.191593in}{1.243876in}}%
\pgfpathlineto{\pgfqpoint{1.192015in}{1.217606in}}%
\pgfpathlineto{\pgfqpoint{1.192436in}{1.224173in}}%
\pgfpathlineto{\pgfqpoint{1.192857in}{1.309551in}}%
\pgfpathlineto{\pgfqpoint{1.193278in}{1.250443in}}%
\pgfpathlineto{\pgfqpoint{1.193700in}{1.230741in}}%
\pgfpathlineto{\pgfqpoint{1.194121in}{1.263578in}}%
\pgfpathlineto{\pgfqpoint{1.194542in}{1.224173in}}%
\pgfpathlineto{\pgfqpoint{1.194963in}{1.257011in}}%
\pgfpathlineto{\pgfqpoint{1.195385in}{1.257011in}}%
\pgfpathlineto{\pgfqpoint{1.197491in}{1.211038in}}%
\pgfpathlineto{\pgfqpoint{1.197912in}{1.243876in}}%
\pgfpathlineto{\pgfqpoint{1.198755in}{1.224173in}}%
\pgfpathlineto{\pgfqpoint{1.199176in}{1.204471in}}%
\pgfpathlineto{\pgfqpoint{1.199598in}{1.217606in}}%
\pgfpathlineto{\pgfqpoint{1.200019in}{1.224173in}}%
\pgfpathlineto{\pgfqpoint{1.200861in}{1.211038in}}%
\pgfpathlineto{\pgfqpoint{1.201704in}{1.224173in}}%
\pgfpathlineto{\pgfqpoint{1.202125in}{1.211038in}}%
\pgfpathlineto{\pgfqpoint{1.202546in}{1.217606in}}%
\pgfpathlineto{\pgfqpoint{1.202968in}{1.250443in}}%
\pgfpathlineto{\pgfqpoint{1.203389in}{1.224173in}}%
\pgfpathlineto{\pgfqpoint{1.203810in}{1.171633in}}%
\pgfpathlineto{\pgfqpoint{1.204653in}{1.204471in}}%
\pgfpathlineto{\pgfqpoint{1.205074in}{1.211038in}}%
\pgfpathlineto{\pgfqpoint{1.205495in}{1.237308in}}%
\pgfpathlineto{\pgfqpoint{1.205917in}{1.224173in}}%
\pgfpathlineto{\pgfqpoint{1.206338in}{1.138795in}}%
\pgfpathlineto{\pgfqpoint{1.206759in}{1.197903in}}%
\pgfpathlineto{\pgfqpoint{1.208444in}{1.224173in}}%
\pgfpathlineto{\pgfqpoint{1.208866in}{1.204471in}}%
\pgfpathlineto{\pgfqpoint{1.209708in}{1.211038in}}%
\pgfpathlineto{\pgfqpoint{1.210130in}{1.204471in}}%
\pgfpathlineto{\pgfqpoint{1.210972in}{1.230741in}}%
\pgfpathlineto{\pgfqpoint{1.211393in}{1.224173in}}%
\pgfpathlineto{\pgfqpoint{1.211815in}{1.217606in}}%
\pgfpathlineto{\pgfqpoint{1.212236in}{1.224173in}}%
\pgfpathlineto{\pgfqpoint{1.212657in}{1.224173in}}%
\pgfpathlineto{\pgfqpoint{1.213078in}{1.211038in}}%
\pgfpathlineto{\pgfqpoint{1.213500in}{1.224173in}}%
\pgfpathlineto{\pgfqpoint{1.213921in}{1.237308in}}%
\pgfpathlineto{\pgfqpoint{1.214342in}{1.230741in}}%
\pgfpathlineto{\pgfqpoint{1.215185in}{1.217606in}}%
\pgfpathlineto{\pgfqpoint{1.215606in}{1.224173in}}%
\pgfpathlineto{\pgfqpoint{1.216027in}{1.230741in}}%
\pgfpathlineto{\pgfqpoint{1.216449in}{1.211038in}}%
\pgfpathlineto{\pgfqpoint{1.217291in}{1.217606in}}%
\pgfpathlineto{\pgfqpoint{1.217713in}{1.217606in}}%
\pgfpathlineto{\pgfqpoint{1.218976in}{1.204471in}}%
\pgfpathlineto{\pgfqpoint{1.220661in}{1.230741in}}%
\pgfpathlineto{\pgfqpoint{1.222768in}{1.191335in}}%
\pgfpathlineto{\pgfqpoint{1.223189in}{1.224173in}}%
\pgfpathlineto{\pgfqpoint{1.224032in}{1.217606in}}%
\pgfpathlineto{\pgfqpoint{1.224453in}{1.224173in}}%
\pgfpathlineto{\pgfqpoint{1.224874in}{1.204471in}}%
\pgfpathlineto{\pgfqpoint{1.225296in}{1.230741in}}%
\pgfpathlineto{\pgfqpoint{1.225717in}{1.211038in}}%
\pgfpathlineto{\pgfqpoint{1.226981in}{1.230741in}}%
\pgfpathlineto{\pgfqpoint{1.227402in}{1.224173in}}%
\pgfpathlineto{\pgfqpoint{1.227823in}{1.197903in}}%
\pgfpathlineto{\pgfqpoint{1.228244in}{1.230741in}}%
\pgfpathlineto{\pgfqpoint{1.228666in}{1.230741in}}%
\pgfpathlineto{\pgfqpoint{1.229930in}{1.211038in}}%
\pgfpathlineto{\pgfqpoint{1.230772in}{1.230741in}}%
\pgfpathlineto{\pgfqpoint{1.231193in}{1.217606in}}%
\pgfpathlineto{\pgfqpoint{1.231615in}{1.230741in}}%
\pgfpathlineto{\pgfqpoint{1.232036in}{1.230741in}}%
\pgfpathlineto{\pgfqpoint{1.233300in}{1.224173in}}%
\pgfpathlineto{\pgfqpoint{1.233721in}{1.224173in}}%
\pgfpathlineto{\pgfqpoint{1.234142in}{1.211038in}}%
\pgfpathlineto{\pgfqpoint{1.234564in}{1.224173in}}%
\pgfpathlineto{\pgfqpoint{1.234985in}{1.237308in}}%
\pgfpathlineto{\pgfqpoint{1.235406in}{1.230741in}}%
\pgfpathlineto{\pgfqpoint{1.236249in}{1.211038in}}%
\pgfpathlineto{\pgfqpoint{1.236670in}{1.224173in}}%
\pgfpathlineto{\pgfqpoint{1.237091in}{1.237308in}}%
\pgfpathlineto{\pgfqpoint{1.237513in}{1.230741in}}%
\pgfpathlineto{\pgfqpoint{1.238776in}{1.224173in}}%
\pgfpathlineto{\pgfqpoint{1.239198in}{1.237308in}}%
\pgfpathlineto{\pgfqpoint{1.239619in}{1.217606in}}%
\pgfpathlineto{\pgfqpoint{1.240040in}{1.230741in}}%
\pgfpathlineto{\pgfqpoint{1.240462in}{1.204471in}}%
\pgfpathlineto{\pgfqpoint{1.240883in}{1.230741in}}%
\pgfpathlineto{\pgfqpoint{1.241304in}{1.237308in}}%
\pgfpathlineto{\pgfqpoint{1.242989in}{1.217606in}}%
\pgfpathlineto{\pgfqpoint{1.244253in}{1.224173in}}%
\pgfpathlineto{\pgfqpoint{1.245096in}{1.217606in}}%
\pgfpathlineto{\pgfqpoint{1.246781in}{1.230741in}}%
\pgfpathlineto{\pgfqpoint{1.247202in}{1.217606in}}%
\pgfpathlineto{\pgfqpoint{1.248045in}{1.224173in}}%
\pgfpathlineto{\pgfqpoint{1.248466in}{1.224173in}}%
\pgfpathlineto{\pgfqpoint{1.248887in}{1.217606in}}%
\pgfpathlineto{\pgfqpoint{1.249730in}{1.237308in}}%
\pgfpathlineto{\pgfqpoint{1.250151in}{1.224173in}}%
\pgfpathlineto{\pgfqpoint{1.250993in}{1.217606in}}%
\pgfpathlineto{\pgfqpoint{1.252257in}{1.230741in}}%
\pgfpathlineto{\pgfqpoint{1.253521in}{1.217606in}}%
\pgfpathlineto{\pgfqpoint{1.253942in}{1.243876in}}%
\pgfpathlineto{\pgfqpoint{1.254785in}{1.230741in}}%
\pgfpathlineto{\pgfqpoint{1.255206in}{1.217606in}}%
\pgfpathlineto{\pgfqpoint{1.255628in}{1.230741in}}%
\pgfpathlineto{\pgfqpoint{1.256470in}{1.230741in}}%
\pgfpathlineto{\pgfqpoint{1.257313in}{1.217606in}}%
\pgfpathlineto{\pgfqpoint{1.257734in}{1.230741in}}%
\pgfpathlineto{\pgfqpoint{1.258576in}{1.224173in}}%
\pgfpathlineto{\pgfqpoint{1.259419in}{1.224173in}}%
\pgfpathlineto{\pgfqpoint{1.260262in}{1.217606in}}%
\pgfpathlineto{\pgfqpoint{1.260683in}{1.237308in}}%
\pgfpathlineto{\pgfqpoint{1.261104in}{1.230741in}}%
\pgfpathlineto{\pgfqpoint{1.261525in}{1.217606in}}%
\pgfpathlineto{\pgfqpoint{1.262368in}{1.224173in}}%
\pgfpathlineto{\pgfqpoint{1.262789in}{1.230741in}}%
\pgfpathlineto{\pgfqpoint{1.263211in}{1.224173in}}%
\pgfpathlineto{\pgfqpoint{1.263632in}{1.211038in}}%
\pgfpathlineto{\pgfqpoint{1.264053in}{1.224173in}}%
\pgfpathlineto{\pgfqpoint{1.264474in}{1.230741in}}%
\pgfpathlineto{\pgfqpoint{1.264896in}{1.217606in}}%
\pgfpathlineto{\pgfqpoint{1.265738in}{1.224173in}}%
\pgfpathlineto{\pgfqpoint{1.266581in}{1.224173in}}%
\pgfpathlineto{\pgfqpoint{1.267002in}{1.230741in}}%
\pgfpathlineto{\pgfqpoint{1.268266in}{1.211038in}}%
\pgfpathlineto{\pgfqpoint{1.269530in}{1.230741in}}%
\pgfpathlineto{\pgfqpoint{1.270372in}{1.217606in}}%
\pgfpathlineto{\pgfqpoint{1.271636in}{1.230741in}}%
\pgfpathlineto{\pgfqpoint{1.272479in}{1.217606in}}%
\pgfpathlineto{\pgfqpoint{1.272900in}{1.224173in}}%
\pgfpathlineto{\pgfqpoint{1.274164in}{1.211038in}}%
\pgfpathlineto{\pgfqpoint{1.275006in}{1.230741in}}%
\pgfpathlineto{\pgfqpoint{1.275428in}{1.224173in}}%
\pgfpathlineto{\pgfqpoint{1.275849in}{1.230741in}}%
\pgfpathlineto{\pgfqpoint{1.276270in}{1.204471in}}%
\pgfpathlineto{\pgfqpoint{1.276691in}{1.230741in}}%
\pgfpathlineto{\pgfqpoint{1.277113in}{1.237308in}}%
\pgfpathlineto{\pgfqpoint{1.277534in}{1.230741in}}%
\pgfpathlineto{\pgfqpoint{1.278377in}{1.217606in}}%
\pgfpathlineto{\pgfqpoint{1.278798in}{1.224173in}}%
\pgfpathlineto{\pgfqpoint{1.279219in}{1.224173in}}%
\pgfpathlineto{\pgfqpoint{1.280483in}{1.204471in}}%
\pgfpathlineto{\pgfqpoint{1.282168in}{1.230741in}}%
\pgfpathlineto{\pgfqpoint{1.283432in}{1.224173in}}%
\pgfpathlineto{\pgfqpoint{1.283853in}{1.237308in}}%
\pgfpathlineto{\pgfqpoint{1.284274in}{1.217606in}}%
\pgfpathlineto{\pgfqpoint{1.284696in}{1.230741in}}%
\pgfpathlineto{\pgfqpoint{1.285117in}{1.224173in}}%
\pgfpathlineto{\pgfqpoint{1.285538in}{1.237308in}}%
\pgfpathlineto{\pgfqpoint{1.286381in}{1.230741in}}%
\pgfpathlineto{\pgfqpoint{1.286802in}{1.237308in}}%
\pgfpathlineto{\pgfqpoint{1.287223in}{1.230741in}}%
\pgfpathlineto{\pgfqpoint{1.288066in}{1.211038in}}%
\pgfpathlineto{\pgfqpoint{1.288487in}{1.230741in}}%
\pgfpathlineto{\pgfqpoint{1.288909in}{1.217606in}}%
\pgfpathlineto{\pgfqpoint{1.289330in}{1.217606in}}%
\pgfpathlineto{\pgfqpoint{1.289751in}{1.230741in}}%
\pgfpathlineto{\pgfqpoint{1.290594in}{1.224173in}}%
\pgfpathlineto{\pgfqpoint{1.291857in}{1.237308in}}%
\pgfpathlineto{\pgfqpoint{1.292700in}{1.211038in}}%
\pgfpathlineto{\pgfqpoint{1.293964in}{1.237308in}}%
\pgfpathlineto{\pgfqpoint{1.296070in}{1.217606in}}%
\pgfpathlineto{\pgfqpoint{1.296492in}{1.230741in}}%
\pgfpathlineto{\pgfqpoint{1.296913in}{1.224173in}}%
\pgfpathlineto{\pgfqpoint{1.297334in}{1.217606in}}%
\pgfpathlineto{\pgfqpoint{1.297755in}{1.230741in}}%
\pgfpathlineto{\pgfqpoint{1.298177in}{1.211038in}}%
\pgfpathlineto{\pgfqpoint{1.299862in}{1.224173in}}%
\pgfpathlineto{\pgfqpoint{1.300283in}{1.224173in}}%
\pgfpathlineto{\pgfqpoint{1.300704in}{1.217606in}}%
\pgfpathlineto{\pgfqpoint{1.301126in}{1.230741in}}%
\pgfpathlineto{\pgfqpoint{1.301547in}{1.211038in}}%
\pgfpathlineto{\pgfqpoint{1.303232in}{1.237308in}}%
\pgfpathlineto{\pgfqpoint{1.304496in}{1.211038in}}%
\pgfpathlineto{\pgfqpoint{1.304917in}{1.230741in}}%
\pgfpathlineto{\pgfqpoint{1.305338in}{1.217606in}}%
\pgfpathlineto{\pgfqpoint{1.306181in}{1.217606in}}%
\pgfpathlineto{\pgfqpoint{1.307445in}{1.230741in}}%
\pgfpathlineto{\pgfqpoint{1.308709in}{1.211038in}}%
\pgfpathlineto{\pgfqpoint{1.309130in}{1.237308in}}%
\pgfpathlineto{\pgfqpoint{1.309551in}{1.224173in}}%
\pgfpathlineto{\pgfqpoint{1.309972in}{1.204471in}}%
\pgfpathlineto{\pgfqpoint{1.310394in}{1.224173in}}%
\pgfpathlineto{\pgfqpoint{1.310815in}{1.230741in}}%
\pgfpathlineto{\pgfqpoint{1.311236in}{1.224173in}}%
\pgfpathlineto{\pgfqpoint{1.312079in}{1.224173in}}%
\pgfpathlineto{\pgfqpoint{1.312921in}{1.237308in}}%
\pgfpathlineto{\pgfqpoint{1.313764in}{1.217606in}}%
\pgfpathlineto{\pgfqpoint{1.315028in}{1.250443in}}%
\pgfpathlineto{\pgfqpoint{1.316292in}{1.211038in}}%
\pgfpathlineto{\pgfqpoint{1.317134in}{1.230741in}}%
\pgfpathlineto{\pgfqpoint{1.317555in}{1.217606in}}%
\pgfpathlineto{\pgfqpoint{1.317977in}{1.224173in}}%
\pgfpathlineto{\pgfqpoint{1.318398in}{1.211038in}}%
\pgfpathlineto{\pgfqpoint{1.318819in}{1.217606in}}%
\pgfpathlineto{\pgfqpoint{1.320083in}{1.230741in}}%
\pgfpathlineto{\pgfqpoint{1.320926in}{1.217606in}}%
\pgfpathlineto{\pgfqpoint{1.322189in}{1.237308in}}%
\pgfpathlineto{\pgfqpoint{1.322611in}{1.211038in}}%
\pgfpathlineto{\pgfqpoint{1.323032in}{1.230741in}}%
\pgfpathlineto{\pgfqpoint{1.323453in}{1.230741in}}%
\pgfpathlineto{\pgfqpoint{1.324717in}{1.217606in}}%
\pgfpathlineto{\pgfqpoint{1.325560in}{1.230741in}}%
\pgfpathlineto{\pgfqpoint{1.325981in}{1.211038in}}%
\pgfpathlineto{\pgfqpoint{1.326402in}{1.217606in}}%
\pgfpathlineto{\pgfqpoint{1.327245in}{1.230741in}}%
\pgfpathlineto{\pgfqpoint{1.328509in}{1.211038in}}%
\pgfpathlineto{\pgfqpoint{1.329772in}{1.237308in}}%
\pgfpathlineto{\pgfqpoint{1.330615in}{1.217606in}}%
\pgfpathlineto{\pgfqpoint{1.331036in}{1.224173in}}%
\pgfpathlineto{\pgfqpoint{1.331458in}{1.224173in}}%
\pgfpathlineto{\pgfqpoint{1.331879in}{1.237308in}}%
\pgfpathlineto{\pgfqpoint{1.332300in}{1.230741in}}%
\pgfpathlineto{\pgfqpoint{1.332721in}{1.224173in}}%
\pgfpathlineto{\pgfqpoint{1.333143in}{1.237308in}}%
\pgfpathlineto{\pgfqpoint{1.333564in}{1.217606in}}%
\pgfpathlineto{\pgfqpoint{1.333985in}{1.217606in}}%
\pgfpathlineto{\pgfqpoint{1.334828in}{1.224173in}}%
\pgfpathlineto{\pgfqpoint{1.335249in}{1.211038in}}%
\pgfpathlineto{\pgfqpoint{1.335670in}{1.217606in}}%
\pgfpathlineto{\pgfqpoint{1.336934in}{1.224173in}}%
\pgfpathlineto{\pgfqpoint{1.337355in}{1.224173in}}%
\pgfpathlineto{\pgfqpoint{1.338619in}{1.230741in}}%
\pgfpathlineto{\pgfqpoint{1.339462in}{1.217606in}}%
\pgfpathlineto{\pgfqpoint{1.339883in}{1.224173in}}%
\pgfpathlineto{\pgfqpoint{1.340304in}{1.224173in}}%
\pgfpathlineto{\pgfqpoint{1.341568in}{1.211038in}}%
\pgfpathlineto{\pgfqpoint{1.341990in}{1.230741in}}%
\pgfpathlineto{\pgfqpoint{1.342832in}{1.224173in}}%
\pgfpathlineto{\pgfqpoint{1.344517in}{1.224173in}}%
\pgfpathlineto{\pgfqpoint{1.345360in}{1.217606in}}%
\pgfpathlineto{\pgfqpoint{1.345781in}{1.230741in}}%
\pgfpathlineto{\pgfqpoint{1.346624in}{1.224173in}}%
\pgfpathlineto{\pgfqpoint{1.347045in}{1.224173in}}%
\pgfpathlineto{\pgfqpoint{1.347466in}{1.230741in}}%
\pgfpathlineto{\pgfqpoint{1.347887in}{1.217606in}}%
\pgfpathlineto{\pgfqpoint{1.348730in}{1.224173in}}%
\pgfpathlineto{\pgfqpoint{1.349151in}{1.224173in}}%
\pgfpathlineto{\pgfqpoint{1.349573in}{1.230741in}}%
\pgfpathlineto{\pgfqpoint{1.349994in}{1.204471in}}%
\pgfpathlineto{\pgfqpoint{1.350415in}{1.224173in}}%
\pgfpathlineto{\pgfqpoint{1.350836in}{1.224173in}}%
\pgfpathlineto{\pgfqpoint{1.351258in}{1.217606in}}%
\pgfpathlineto{\pgfqpoint{1.352522in}{1.230741in}}%
\pgfpathlineto{\pgfqpoint{1.353364in}{1.230741in}}%
\pgfpathlineto{\pgfqpoint{1.354207in}{1.224173in}}%
\pgfpathlineto{\pgfqpoint{1.355049in}{1.230741in}}%
\pgfpathlineto{\pgfqpoint{1.355470in}{1.217606in}}%
\pgfpathlineto{\pgfqpoint{1.356313in}{1.224173in}}%
\pgfpathlineto{\pgfqpoint{1.356734in}{1.224173in}}%
\pgfpathlineto{\pgfqpoint{1.357156in}{1.217606in}}%
\pgfpathlineto{\pgfqpoint{1.357998in}{1.230741in}}%
\pgfpathlineto{\pgfqpoint{1.359262in}{1.217606in}}%
\pgfpathlineto{\pgfqpoint{1.359683in}{1.230741in}}%
\pgfpathlineto{\pgfqpoint{1.360105in}{1.217606in}}%
\pgfpathlineto{\pgfqpoint{1.360526in}{1.204471in}}%
\pgfpathlineto{\pgfqpoint{1.360947in}{1.224173in}}%
\pgfpathlineto{\pgfqpoint{1.362211in}{1.230741in}}%
\pgfpathlineto{\pgfqpoint{1.363053in}{1.224173in}}%
\pgfpathlineto{\pgfqpoint{1.363475in}{1.237308in}}%
\pgfpathlineto{\pgfqpoint{1.364317in}{1.230741in}}%
\pgfpathlineto{\pgfqpoint{1.364739in}{1.230741in}}%
\pgfpathlineto{\pgfqpoint{1.366002in}{1.217606in}}%
\pgfpathlineto{\pgfqpoint{1.366424in}{1.243876in}}%
\pgfpathlineto{\pgfqpoint{1.366845in}{1.204471in}}%
\pgfpathlineto{\pgfqpoint{1.367688in}{1.230741in}}%
\pgfpathlineto{\pgfqpoint{1.368109in}{1.224173in}}%
\pgfpathlineto{\pgfqpoint{1.368530in}{1.230741in}}%
\pgfpathlineto{\pgfqpoint{1.368951in}{1.224173in}}%
\pgfpathlineto{\pgfqpoint{1.370215in}{1.211038in}}%
\pgfpathlineto{\pgfqpoint{1.371900in}{1.237308in}}%
\pgfpathlineto{\pgfqpoint{1.372322in}{1.211038in}}%
\pgfpathlineto{\pgfqpoint{1.373164in}{1.224173in}}%
\pgfpathlineto{\pgfqpoint{1.373585in}{1.230741in}}%
\pgfpathlineto{\pgfqpoint{1.374007in}{1.257011in}}%
\pgfpathlineto{\pgfqpoint{1.374428in}{1.217606in}}%
\pgfpathlineto{\pgfqpoint{1.374849in}{1.237308in}}%
\pgfpathlineto{\pgfqpoint{1.375271in}{1.204471in}}%
\pgfpathlineto{\pgfqpoint{1.376113in}{1.224173in}}%
\pgfpathlineto{\pgfqpoint{1.376534in}{1.217606in}}%
\pgfpathlineto{\pgfqpoint{1.376956in}{1.224173in}}%
\pgfpathlineto{\pgfqpoint{1.377377in}{1.243876in}}%
\pgfpathlineto{\pgfqpoint{1.377798in}{1.230741in}}%
\pgfpathlineto{\pgfqpoint{1.379483in}{1.211038in}}%
\pgfpathlineto{\pgfqpoint{1.379905in}{1.204471in}}%
\pgfpathlineto{\pgfqpoint{1.380326in}{1.243876in}}%
\pgfpathlineto{\pgfqpoint{1.380747in}{1.230741in}}%
\pgfpathlineto{\pgfqpoint{1.381168in}{1.204471in}}%
\pgfpathlineto{\pgfqpoint{1.382011in}{1.217606in}}%
\pgfpathlineto{\pgfqpoint{1.383696in}{1.230741in}}%
\pgfpathlineto{\pgfqpoint{1.384539in}{1.211038in}}%
\pgfpathlineto{\pgfqpoint{1.384960in}{1.217606in}}%
\pgfpathlineto{\pgfqpoint{1.386224in}{1.230741in}}%
\pgfpathlineto{\pgfqpoint{1.387066in}{1.211038in}}%
\pgfpathlineto{\pgfqpoint{1.387488in}{1.217606in}}%
\pgfpathlineto{\pgfqpoint{1.388751in}{1.263578in}}%
\pgfpathlineto{\pgfqpoint{1.390437in}{1.224173in}}%
\pgfpathlineto{\pgfqpoint{1.390858in}{1.204471in}}%
\pgfpathlineto{\pgfqpoint{1.391279in}{1.211038in}}%
\pgfpathlineto{\pgfqpoint{1.392964in}{1.257011in}}%
\pgfpathlineto{\pgfqpoint{1.394228in}{1.211038in}}%
\pgfpathlineto{\pgfqpoint{1.394649in}{1.237308in}}%
\pgfpathlineto{\pgfqpoint{1.395071in}{1.224173in}}%
\pgfpathlineto{\pgfqpoint{1.395492in}{1.217606in}}%
\pgfpathlineto{\pgfqpoint{1.396334in}{1.250443in}}%
\pgfpathlineto{\pgfqpoint{1.396756in}{1.230741in}}%
\pgfpathlineto{\pgfqpoint{1.397177in}{1.230741in}}%
\pgfpathlineto{\pgfqpoint{1.397598in}{1.224173in}}%
\pgfpathlineto{\pgfqpoint{1.398020in}{1.237308in}}%
\pgfpathlineto{\pgfqpoint{1.398441in}{1.204471in}}%
\pgfpathlineto{\pgfqpoint{1.398862in}{1.224173in}}%
\pgfpathlineto{\pgfqpoint{1.399283in}{1.230741in}}%
\pgfpathlineto{\pgfqpoint{1.400968in}{1.211038in}}%
\pgfpathlineto{\pgfqpoint{1.401390in}{1.211038in}}%
\pgfpathlineto{\pgfqpoint{1.401811in}{1.191335in}}%
\pgfpathlineto{\pgfqpoint{1.402654in}{1.197903in}}%
\pgfpathlineto{\pgfqpoint{1.404339in}{1.224173in}}%
\pgfpathlineto{\pgfqpoint{1.405603in}{1.204471in}}%
\pgfpathlineto{\pgfqpoint{1.406024in}{1.250443in}}%
\pgfpathlineto{\pgfqpoint{1.406445in}{1.211038in}}%
\pgfpathlineto{\pgfqpoint{1.406866in}{1.211038in}}%
\pgfpathlineto{\pgfqpoint{1.407288in}{1.204471in}}%
\pgfpathlineto{\pgfqpoint{1.408552in}{1.230741in}}%
\pgfpathlineto{\pgfqpoint{1.408973in}{1.230741in}}%
\pgfpathlineto{\pgfqpoint{1.409815in}{1.191335in}}%
\pgfpathlineto{\pgfqpoint{1.410237in}{1.211038in}}%
\pgfpathlineto{\pgfqpoint{1.411079in}{1.276714in}}%
\pgfpathlineto{\pgfqpoint{1.412343in}{1.211038in}}%
\pgfpathlineto{\pgfqpoint{1.414028in}{1.250443in}}%
\pgfpathlineto{\pgfqpoint{1.415292in}{1.211038in}}%
\pgfpathlineto{\pgfqpoint{1.415713in}{1.217606in}}%
\pgfpathlineto{\pgfqpoint{1.416135in}{1.243876in}}%
\pgfpathlineto{\pgfqpoint{1.416556in}{1.237308in}}%
\pgfpathlineto{\pgfqpoint{1.417820in}{1.224173in}}%
\pgfpathlineto{\pgfqpoint{1.418241in}{1.250443in}}%
\pgfpathlineto{\pgfqpoint{1.418662in}{1.224173in}}%
\pgfpathlineto{\pgfqpoint{1.419083in}{1.224173in}}%
\pgfpathlineto{\pgfqpoint{1.419505in}{1.197903in}}%
\pgfpathlineto{\pgfqpoint{1.419926in}{1.204471in}}%
\pgfpathlineto{\pgfqpoint{1.420347in}{1.230741in}}%
\pgfpathlineto{\pgfqpoint{1.421190in}{1.224173in}}%
\pgfpathlineto{\pgfqpoint{1.422032in}{1.211038in}}%
\pgfpathlineto{\pgfqpoint{1.423296in}{1.230741in}}%
\pgfpathlineto{\pgfqpoint{1.423718in}{1.224173in}}%
\pgfpathlineto{\pgfqpoint{1.424560in}{1.171633in}}%
\pgfpathlineto{\pgfqpoint{1.425824in}{1.237308in}}%
\pgfpathlineto{\pgfqpoint{1.427509in}{1.204471in}}%
\pgfpathlineto{\pgfqpoint{1.428773in}{1.237308in}}%
\pgfpathlineto{\pgfqpoint{1.430037in}{1.217606in}}%
\pgfpathlineto{\pgfqpoint{1.430458in}{1.237308in}}%
\pgfpathlineto{\pgfqpoint{1.431301in}{1.230741in}}%
\pgfpathlineto{\pgfqpoint{1.432143in}{1.224173in}}%
\pgfpathlineto{\pgfqpoint{1.433407in}{1.250443in}}%
\pgfpathlineto{\pgfqpoint{1.434249in}{1.224173in}}%
\pgfpathlineto{\pgfqpoint{1.434671in}{1.230741in}}%
\pgfpathlineto{\pgfqpoint{1.435513in}{1.237308in}}%
\pgfpathlineto{\pgfqpoint{1.436356in}{1.230741in}}%
\pgfpathlineto{\pgfqpoint{1.436777in}{1.237308in}}%
\pgfpathlineto{\pgfqpoint{1.437198in}{1.230741in}}%
\pgfpathlineto{\pgfqpoint{1.437620in}{1.230741in}}%
\pgfpathlineto{\pgfqpoint{1.438041in}{1.237308in}}%
\pgfpathlineto{\pgfqpoint{1.439305in}{1.224173in}}%
\pgfpathlineto{\pgfqpoint{1.440147in}{1.237308in}}%
\pgfpathlineto{\pgfqpoint{1.441411in}{1.217606in}}%
\pgfpathlineto{\pgfqpoint{1.442254in}{1.230741in}}%
\pgfpathlineto{\pgfqpoint{1.442675in}{1.224173in}}%
\pgfpathlineto{\pgfqpoint{1.443518in}{1.230741in}}%
\pgfpathlineto{\pgfqpoint{1.443939in}{1.217606in}}%
\pgfpathlineto{\pgfqpoint{1.444360in}{1.237308in}}%
\pgfpathlineto{\pgfqpoint{1.444781in}{1.224173in}}%
\pgfpathlineto{\pgfqpoint{1.445203in}{1.217606in}}%
\pgfpathlineto{\pgfqpoint{1.445624in}{1.243876in}}%
\pgfpathlineto{\pgfqpoint{1.446045in}{1.224173in}}%
\pgfpathlineto{\pgfqpoint{1.446467in}{1.211038in}}%
\pgfpathlineto{\pgfqpoint{1.446888in}{1.217606in}}%
\pgfpathlineto{\pgfqpoint{1.447730in}{1.243876in}}%
\pgfpathlineto{\pgfqpoint{1.448994in}{1.217606in}}%
\pgfpathlineto{\pgfqpoint{1.449837in}{1.237308in}}%
\pgfpathlineto{\pgfqpoint{1.450258in}{1.204471in}}%
\pgfpathlineto{\pgfqpoint{1.451101in}{1.217606in}}%
\pgfpathlineto{\pgfqpoint{1.451522in}{1.211038in}}%
\pgfpathlineto{\pgfqpoint{1.452786in}{1.237308in}}%
\pgfpathlineto{\pgfqpoint{1.453207in}{1.211038in}}%
\pgfpathlineto{\pgfqpoint{1.453628in}{1.230741in}}%
\pgfpathlineto{\pgfqpoint{1.454050in}{1.243876in}}%
\pgfpathlineto{\pgfqpoint{1.454471in}{1.230741in}}%
\pgfpathlineto{\pgfqpoint{1.454892in}{1.230741in}}%
\pgfpathlineto{\pgfqpoint{1.455313in}{1.211038in}}%
\pgfpathlineto{\pgfqpoint{1.455735in}{1.224173in}}%
\pgfpathlineto{\pgfqpoint{1.456156in}{1.230741in}}%
\pgfpathlineto{\pgfqpoint{1.456577in}{1.217606in}}%
\pgfpathlineto{\pgfqpoint{1.456998in}{1.224173in}}%
\pgfpathlineto{\pgfqpoint{1.457420in}{1.237308in}}%
\pgfpathlineto{\pgfqpoint{1.457841in}{1.217606in}}%
\pgfpathlineto{\pgfqpoint{1.458684in}{1.230741in}}%
\pgfpathlineto{\pgfqpoint{1.459526in}{1.211038in}}%
\pgfpathlineto{\pgfqpoint{1.459947in}{1.217606in}}%
\pgfpathlineto{\pgfqpoint{1.460369in}{1.217606in}}%
\pgfpathlineto{\pgfqpoint{1.460790in}{1.237308in}}%
\pgfpathlineto{\pgfqpoint{1.461633in}{1.230741in}}%
\pgfpathlineto{\pgfqpoint{1.462054in}{1.230741in}}%
\pgfpathlineto{\pgfqpoint{1.462475in}{1.237308in}}%
\pgfpathlineto{\pgfqpoint{1.462896in}{1.217606in}}%
\pgfpathlineto{\pgfqpoint{1.463318in}{1.224173in}}%
\pgfpathlineto{\pgfqpoint{1.464581in}{1.237308in}}%
\pgfpathlineto{\pgfqpoint{1.466267in}{1.211038in}}%
\pgfpathlineto{\pgfqpoint{1.466688in}{1.237308in}}%
\pgfpathlineto{\pgfqpoint{1.467530in}{1.224173in}}%
\pgfpathlineto{\pgfqpoint{1.467952in}{1.224173in}}%
\pgfpathlineto{\pgfqpoint{1.468794in}{1.237308in}}%
\pgfpathlineto{\pgfqpoint{1.470058in}{1.211038in}}%
\pgfpathlineto{\pgfqpoint{1.470479in}{1.217606in}}%
\pgfpathlineto{\pgfqpoint{1.471743in}{1.224173in}}%
\pgfpathlineto{\pgfqpoint{1.472164in}{1.211038in}}%
\pgfpathlineto{\pgfqpoint{1.472586in}{1.230741in}}%
\pgfpathlineto{\pgfqpoint{1.473007in}{1.243876in}}%
\pgfpathlineto{\pgfqpoint{1.473428in}{1.237308in}}%
\pgfpathlineto{\pgfqpoint{1.473850in}{1.230741in}}%
\pgfpathlineto{\pgfqpoint{1.474271in}{1.204471in}}%
\pgfpathlineto{\pgfqpoint{1.474692in}{1.224173in}}%
\pgfpathlineto{\pgfqpoint{1.475113in}{1.230741in}}%
\pgfpathlineto{\pgfqpoint{1.475535in}{1.211038in}}%
\pgfpathlineto{\pgfqpoint{1.476377in}{1.217606in}}%
\pgfpathlineto{\pgfqpoint{1.477641in}{1.224173in}}%
\pgfpathlineto{\pgfqpoint{1.478062in}{1.217606in}}%
\pgfpathlineto{\pgfqpoint{1.479326in}{1.237308in}}%
\pgfpathlineto{\pgfqpoint{1.480590in}{1.211038in}}%
\pgfpathlineto{\pgfqpoint{1.481011in}{1.230741in}}%
\pgfpathlineto{\pgfqpoint{1.481433in}{1.217606in}}%
\pgfpathlineto{\pgfqpoint{1.481854in}{1.211038in}}%
\pgfpathlineto{\pgfqpoint{1.482275in}{1.230741in}}%
\pgfpathlineto{\pgfqpoint{1.482696in}{1.217606in}}%
\pgfpathlineto{\pgfqpoint{1.483118in}{1.217606in}}%
\pgfpathlineto{\pgfqpoint{1.483539in}{1.237308in}}%
\pgfpathlineto{\pgfqpoint{1.483960in}{1.230741in}}%
\pgfpathlineto{\pgfqpoint{1.484382in}{1.211038in}}%
\pgfpathlineto{\pgfqpoint{1.484803in}{1.237308in}}%
\pgfpathlineto{\pgfqpoint{1.486067in}{1.217606in}}%
\pgfpathlineto{\pgfqpoint{1.487752in}{1.237308in}}%
\pgfpathlineto{\pgfqpoint{1.488594in}{1.211038in}}%
\pgfpathlineto{\pgfqpoint{1.489437in}{1.217606in}}%
\pgfpathlineto{\pgfqpoint{1.489858in}{1.230741in}}%
\pgfpathlineto{\pgfqpoint{1.490701in}{1.224173in}}%
\pgfpathlineto{\pgfqpoint{1.491965in}{1.217606in}}%
\pgfpathlineto{\pgfqpoint{1.492386in}{1.224173in}}%
\pgfpathlineto{\pgfqpoint{1.492807in}{1.217606in}}%
\pgfpathlineto{\pgfqpoint{1.493228in}{1.211038in}}%
\pgfpathlineto{\pgfqpoint{1.494071in}{1.237308in}}%
\pgfpathlineto{\pgfqpoint{1.494492in}{1.224173in}}%
\pgfpathlineto{\pgfqpoint{1.494914in}{1.224173in}}%
\pgfpathlineto{\pgfqpoint{1.495756in}{1.217606in}}%
\pgfpathlineto{\pgfqpoint{1.497020in}{1.224173in}}%
\pgfpathlineto{\pgfqpoint{1.498284in}{1.224173in}}%
\pgfpathlineto{\pgfqpoint{1.499126in}{1.230741in}}%
\pgfpathlineto{\pgfqpoint{1.499548in}{1.211038in}}%
\pgfpathlineto{\pgfqpoint{1.499969in}{1.224173in}}%
\pgfpathlineto{\pgfqpoint{1.500390in}{1.224173in}}%
\pgfpathlineto{\pgfqpoint{1.500811in}{1.217606in}}%
\pgfpathlineto{\pgfqpoint{1.501233in}{1.224173in}}%
\pgfpathlineto{\pgfqpoint{1.502497in}{1.237308in}}%
\pgfpathlineto{\pgfqpoint{1.502918in}{1.211038in}}%
\pgfpathlineto{\pgfqpoint{1.503339in}{1.224173in}}%
\pgfpathlineto{\pgfqpoint{1.504603in}{1.230741in}}%
\pgfpathlineto{\pgfqpoint{1.505445in}{1.230741in}}%
\pgfpathlineto{\pgfqpoint{1.505867in}{1.197903in}}%
\pgfpathlineto{\pgfqpoint{1.506288in}{1.230741in}}%
\pgfpathlineto{\pgfqpoint{1.506709in}{1.230741in}}%
\pgfpathlineto{\pgfqpoint{1.507131in}{1.237308in}}%
\pgfpathlineto{\pgfqpoint{1.507973in}{1.217606in}}%
\pgfpathlineto{\pgfqpoint{1.508394in}{1.224173in}}%
\pgfpathlineto{\pgfqpoint{1.508816in}{1.237308in}}%
\pgfpathlineto{\pgfqpoint{1.509237in}{1.224173in}}%
\pgfpathlineto{\pgfqpoint{1.509658in}{1.224173in}}%
\pgfpathlineto{\pgfqpoint{1.510080in}{1.217606in}}%
\pgfpathlineto{\pgfqpoint{1.510501in}{1.224173in}}%
\pgfpathlineto{\pgfqpoint{1.511343in}{1.224173in}}%
\pgfpathlineto{\pgfqpoint{1.512186in}{1.204471in}}%
\pgfpathlineto{\pgfqpoint{1.512607in}{1.388362in}}%
\pgfpathlineto{\pgfqpoint{1.513028in}{1.243876in}}%
\pgfpathlineto{\pgfqpoint{1.514292in}{1.211038in}}%
\pgfpathlineto{\pgfqpoint{1.515135in}{1.204471in}}%
\pgfpathlineto{\pgfqpoint{1.515556in}{1.243876in}}%
\pgfpathlineto{\pgfqpoint{1.516820in}{1.224173in}}%
\pgfpathlineto{\pgfqpoint{1.517241in}{1.230741in}}%
\pgfpathlineto{\pgfqpoint{1.517663in}{1.224173in}}%
\pgfpathlineto{\pgfqpoint{1.518084in}{1.224173in}}%
\pgfpathlineto{\pgfqpoint{1.518505in}{1.217606in}}%
\pgfpathlineto{\pgfqpoint{1.518926in}{1.224173in}}%
\pgfpathlineto{\pgfqpoint{1.519348in}{1.243876in}}%
\pgfpathlineto{\pgfqpoint{1.519769in}{1.230741in}}%
\pgfpathlineto{\pgfqpoint{1.520611in}{1.204471in}}%
\pgfpathlineto{\pgfqpoint{1.521033in}{1.211038in}}%
\pgfpathlineto{\pgfqpoint{1.522297in}{1.224173in}}%
\pgfpathlineto{\pgfqpoint{1.522718in}{1.211038in}}%
\pgfpathlineto{\pgfqpoint{1.523139in}{1.230741in}}%
\pgfpathlineto{\pgfqpoint{1.523560in}{1.243876in}}%
\pgfpathlineto{\pgfqpoint{1.523982in}{1.230741in}}%
\pgfpathlineto{\pgfqpoint{1.524824in}{1.224173in}}%
\pgfpathlineto{\pgfqpoint{1.525667in}{1.237308in}}%
\pgfpathlineto{\pgfqpoint{1.526088in}{1.059984in}}%
\pgfpathlineto{\pgfqpoint{1.526509in}{1.217606in}}%
\pgfpathlineto{\pgfqpoint{1.527352in}{1.217606in}}%
\pgfpathlineto{\pgfqpoint{1.528616in}{1.230741in}}%
\pgfpathlineto{\pgfqpoint{1.529037in}{1.197903in}}%
\pgfpathlineto{\pgfqpoint{1.529458in}{1.224173in}}%
\pgfpathlineto{\pgfqpoint{1.530301in}{1.230741in}}%
\pgfpathlineto{\pgfqpoint{1.530722in}{1.211038in}}%
\pgfpathlineto{\pgfqpoint{1.531143in}{1.217606in}}%
\pgfpathlineto{\pgfqpoint{1.532407in}{1.230741in}}%
\pgfpathlineto{\pgfqpoint{1.533250in}{1.197903in}}%
\pgfpathlineto{\pgfqpoint{1.534514in}{1.243876in}}%
\pgfpathlineto{\pgfqpoint{1.535777in}{1.211038in}}%
\pgfpathlineto{\pgfqpoint{1.536199in}{1.237308in}}%
\pgfpathlineto{\pgfqpoint{1.536620in}{1.217606in}}%
\pgfpathlineto{\pgfqpoint{1.537041in}{1.211038in}}%
\pgfpathlineto{\pgfqpoint{1.537884in}{1.230741in}}%
\pgfpathlineto{\pgfqpoint{1.538305in}{1.224173in}}%
\pgfpathlineto{\pgfqpoint{1.538726in}{1.217606in}}%
\pgfpathlineto{\pgfqpoint{1.539148in}{1.237308in}}%
\pgfpathlineto{\pgfqpoint{1.539990in}{1.230741in}}%
\pgfpathlineto{\pgfqpoint{1.540412in}{1.230741in}}%
\pgfpathlineto{\pgfqpoint{1.540833in}{1.224173in}}%
\pgfpathlineto{\pgfqpoint{1.541254in}{1.230741in}}%
\pgfpathlineto{\pgfqpoint{1.541675in}{1.230741in}}%
\pgfpathlineto{\pgfqpoint{1.542097in}{1.224173in}}%
\pgfpathlineto{\pgfqpoint{1.542518in}{1.237308in}}%
\pgfpathlineto{\pgfqpoint{1.542939in}{1.224173in}}%
\pgfpathlineto{\pgfqpoint{1.543361in}{1.217606in}}%
\pgfpathlineto{\pgfqpoint{1.544624in}{1.237308in}}%
\pgfpathlineto{\pgfqpoint{1.545888in}{1.224173in}}%
\pgfpathlineto{\pgfqpoint{1.546309in}{1.243876in}}%
\pgfpathlineto{\pgfqpoint{1.546731in}{1.230741in}}%
\pgfpathlineto{\pgfqpoint{1.547995in}{1.211038in}}%
\pgfpathlineto{\pgfqpoint{1.548416in}{1.217606in}}%
\pgfpathlineto{\pgfqpoint{1.549258in}{1.237308in}}%
\pgfpathlineto{\pgfqpoint{1.549680in}{1.224173in}}%
\pgfpathlineto{\pgfqpoint{1.550101in}{1.211038in}}%
\pgfpathlineto{\pgfqpoint{1.550522in}{1.230741in}}%
\pgfpathlineto{\pgfqpoint{1.550944in}{1.230741in}}%
\pgfpathlineto{\pgfqpoint{1.552207in}{1.204471in}}%
\pgfpathlineto{\pgfqpoint{1.552629in}{1.224173in}}%
\pgfpathlineto{\pgfqpoint{1.553471in}{1.217606in}}%
\pgfpathlineto{\pgfqpoint{1.553892in}{1.230741in}}%
\pgfpathlineto{\pgfqpoint{1.554735in}{1.224173in}}%
\pgfpathlineto{\pgfqpoint{1.555156in}{1.211038in}}%
\pgfpathlineto{\pgfqpoint{1.555578in}{1.224173in}}%
\pgfpathlineto{\pgfqpoint{1.555999in}{1.230741in}}%
\pgfpathlineto{\pgfqpoint{1.557263in}{1.211038in}}%
\pgfpathlineto{\pgfqpoint{1.557684in}{1.230741in}}%
\pgfpathlineto{\pgfqpoint{1.558105in}{1.211038in}}%
\pgfpathlineto{\pgfqpoint{1.558527in}{1.204471in}}%
\pgfpathlineto{\pgfqpoint{1.558948in}{1.237308in}}%
\pgfpathlineto{\pgfqpoint{1.559369in}{1.204471in}}%
\pgfpathlineto{\pgfqpoint{1.559790in}{1.204471in}}%
\pgfpathlineto{\pgfqpoint{1.560212in}{1.243876in}}%
\pgfpathlineto{\pgfqpoint{1.561054in}{1.224173in}}%
\pgfpathlineto{\pgfqpoint{1.561475in}{1.237308in}}%
\pgfpathlineto{\pgfqpoint{1.561897in}{1.230741in}}%
\pgfpathlineto{\pgfqpoint{1.563161in}{1.217606in}}%
\pgfpathlineto{\pgfqpoint{1.563582in}{1.243876in}}%
\pgfpathlineto{\pgfqpoint{1.564003in}{1.224173in}}%
\pgfpathlineto{\pgfqpoint{1.564424in}{1.211038in}}%
\pgfpathlineto{\pgfqpoint{1.564846in}{1.217606in}}%
\pgfpathlineto{\pgfqpoint{1.565688in}{1.243876in}}%
\pgfpathlineto{\pgfqpoint{1.566110in}{1.211038in}}%
\pgfpathlineto{\pgfqpoint{1.566952in}{1.224173in}}%
\pgfpathlineto{\pgfqpoint{1.567373in}{1.217606in}}%
\pgfpathlineto{\pgfqpoint{1.567795in}{1.224173in}}%
\pgfpathlineto{\pgfqpoint{1.568216in}{1.224173in}}%
\pgfpathlineto{\pgfqpoint{1.568637in}{1.237308in}}%
\pgfpathlineto{\pgfqpoint{1.569058in}{1.224173in}}%
\pgfpathlineto{\pgfqpoint{1.569480in}{1.211038in}}%
\pgfpathlineto{\pgfqpoint{1.570322in}{1.217606in}}%
\pgfpathlineto{\pgfqpoint{1.572007in}{1.237308in}}%
\pgfpathlineto{\pgfqpoint{1.572429in}{1.204471in}}%
\pgfpathlineto{\pgfqpoint{1.572850in}{1.243876in}}%
\pgfpathlineto{\pgfqpoint{1.573271in}{1.224173in}}%
\pgfpathlineto{\pgfqpoint{1.573693in}{1.211038in}}%
\pgfpathlineto{\pgfqpoint{1.574114in}{1.217606in}}%
\pgfpathlineto{\pgfqpoint{1.574535in}{1.224173in}}%
\pgfpathlineto{\pgfqpoint{1.574956in}{1.217606in}}%
\pgfpathlineto{\pgfqpoint{1.575799in}{1.217606in}}%
\pgfpathlineto{\pgfqpoint{1.576641in}{1.230741in}}%
\pgfpathlineto{\pgfqpoint{1.577484in}{1.204471in}}%
\pgfpathlineto{\pgfqpoint{1.577905in}{1.224173in}}%
\pgfpathlineto{\pgfqpoint{1.579169in}{1.224173in}}%
\pgfpathlineto{\pgfqpoint{1.579590in}{1.230741in}}%
\pgfpathlineto{\pgfqpoint{1.580012in}{1.224173in}}%
\pgfpathlineto{\pgfqpoint{1.580854in}{1.224173in}}%
\pgfpathlineto{\pgfqpoint{1.581276in}{1.230741in}}%
\pgfpathlineto{\pgfqpoint{1.581697in}{1.224173in}}%
\pgfpathlineto{\pgfqpoint{1.582118in}{1.211038in}}%
\pgfpathlineto{\pgfqpoint{1.582539in}{1.224173in}}%
\pgfpathlineto{\pgfqpoint{1.583803in}{1.230741in}}%
\pgfpathlineto{\pgfqpoint{1.584224in}{1.230741in}}%
\pgfpathlineto{\pgfqpoint{1.585067in}{1.211038in}}%
\pgfpathlineto{\pgfqpoint{1.585488in}{1.224173in}}%
\pgfpathlineto{\pgfqpoint{1.585910in}{1.224173in}}%
\pgfpathlineto{\pgfqpoint{1.586331in}{1.217606in}}%
\pgfpathlineto{\pgfqpoint{1.587595in}{1.230741in}}%
\pgfpathlineto{\pgfqpoint{1.588016in}{1.211038in}}%
\pgfpathlineto{\pgfqpoint{1.588437in}{1.217606in}}%
\pgfpathlineto{\pgfqpoint{1.589280in}{1.237308in}}%
\pgfpathlineto{\pgfqpoint{1.589701in}{1.230741in}}%
\pgfpathlineto{\pgfqpoint{1.590122in}{1.217606in}}%
\pgfpathlineto{\pgfqpoint{1.590544in}{1.224173in}}%
\pgfpathlineto{\pgfqpoint{1.590965in}{1.243876in}}%
\pgfpathlineto{\pgfqpoint{1.591386in}{1.224173in}}%
\pgfpathlineto{\pgfqpoint{1.591807in}{1.224173in}}%
\pgfpathlineto{\pgfqpoint{1.592229in}{1.204471in}}%
\pgfpathlineto{\pgfqpoint{1.593071in}{1.211038in}}%
\pgfpathlineto{\pgfqpoint{1.593914in}{1.230741in}}%
\pgfpathlineto{\pgfqpoint{1.594335in}{1.211038in}}%
\pgfpathlineto{\pgfqpoint{1.594756in}{1.224173in}}%
\pgfpathlineto{\pgfqpoint{1.596020in}{1.230741in}}%
\pgfpathlineto{\pgfqpoint{1.596863in}{1.217606in}}%
\pgfpathlineto{\pgfqpoint{1.598548in}{1.237308in}}%
\pgfpathlineto{\pgfqpoint{1.599812in}{1.211038in}}%
\pgfpathlineto{\pgfqpoint{1.601497in}{1.243876in}}%
\pgfpathlineto{\pgfqpoint{1.602761in}{1.211038in}}%
\pgfpathlineto{\pgfqpoint{1.603182in}{1.217606in}}%
\pgfpathlineto{\pgfqpoint{1.603603in}{1.224173in}}%
\pgfpathlineto{\pgfqpoint{1.604025in}{1.217606in}}%
\pgfpathlineto{\pgfqpoint{1.604446in}{1.211038in}}%
\pgfpathlineto{\pgfqpoint{1.605710in}{1.243876in}}%
\pgfpathlineto{\pgfqpoint{1.606974in}{1.204471in}}%
\pgfpathlineto{\pgfqpoint{1.607395in}{1.211038in}}%
\pgfpathlineto{\pgfqpoint{1.607816in}{1.237308in}}%
\pgfpathlineto{\pgfqpoint{1.608659in}{1.224173in}}%
\pgfpathlineto{\pgfqpoint{1.609080in}{1.211038in}}%
\pgfpathlineto{\pgfqpoint{1.609501in}{1.217606in}}%
\pgfpathlineto{\pgfqpoint{1.610344in}{1.237308in}}%
\pgfpathlineto{\pgfqpoint{1.611186in}{1.217606in}}%
\pgfpathlineto{\pgfqpoint{1.611608in}{1.224173in}}%
\pgfpathlineto{\pgfqpoint{1.612029in}{1.224173in}}%
\pgfpathlineto{\pgfqpoint{1.613293in}{1.217606in}}%
\pgfpathlineto{\pgfqpoint{1.613714in}{1.217606in}}%
\pgfpathlineto{\pgfqpoint{1.614978in}{1.224173in}}%
\pgfpathlineto{\pgfqpoint{1.615820in}{1.224173in}}%
\pgfpathlineto{\pgfqpoint{1.616242in}{1.237308in}}%
\pgfpathlineto{\pgfqpoint{1.616663in}{1.230741in}}%
\pgfpathlineto{\pgfqpoint{1.617084in}{1.217606in}}%
\pgfpathlineto{\pgfqpoint{1.617505in}{1.230741in}}%
\pgfpathlineto{\pgfqpoint{1.618348in}{1.230741in}}%
\pgfpathlineto{\pgfqpoint{1.619191in}{1.217606in}}%
\pgfpathlineto{\pgfqpoint{1.619612in}{1.224173in}}%
\pgfpathlineto{\pgfqpoint{1.620033in}{1.224173in}}%
\pgfpathlineto{\pgfqpoint{1.620454in}{1.257011in}}%
\pgfpathlineto{\pgfqpoint{1.620876in}{1.237308in}}%
\pgfpathlineto{\pgfqpoint{1.622140in}{1.217606in}}%
\pgfpathlineto{\pgfqpoint{1.622561in}{1.237308in}}%
\pgfpathlineto{\pgfqpoint{1.622982in}{1.224173in}}%
\pgfpathlineto{\pgfqpoint{1.623825in}{1.204471in}}%
\pgfpathlineto{\pgfqpoint{1.624667in}{1.230741in}}%
\pgfpathlineto{\pgfqpoint{1.625088in}{1.217606in}}%
\pgfpathlineto{\pgfqpoint{1.626774in}{1.243876in}}%
\pgfpathlineto{\pgfqpoint{1.628037in}{1.211038in}}%
\pgfpathlineto{\pgfqpoint{1.629723in}{1.230741in}}%
\pgfpathlineto{\pgfqpoint{1.630144in}{1.211038in}}%
\pgfpathlineto{\pgfqpoint{1.630565in}{1.257011in}}%
\pgfpathlineto{\pgfqpoint{1.631408in}{1.230741in}}%
\pgfpathlineto{\pgfqpoint{1.632250in}{1.217606in}}%
\pgfpathlineto{\pgfqpoint{1.632671in}{1.224173in}}%
\pgfpathlineto{\pgfqpoint{1.633514in}{1.224173in}}%
\pgfpathlineto{\pgfqpoint{1.634357in}{1.211038in}}%
\pgfpathlineto{\pgfqpoint{1.635199in}{1.230741in}}%
\pgfpathlineto{\pgfqpoint{1.635620in}{1.211038in}}%
\pgfpathlineto{\pgfqpoint{1.636463in}{1.217606in}}%
\pgfpathlineto{\pgfqpoint{1.637306in}{1.237308in}}%
\pgfpathlineto{\pgfqpoint{1.637727in}{1.224173in}}%
\pgfpathlineto{\pgfqpoint{1.638148in}{1.217606in}}%
\pgfpathlineto{\pgfqpoint{1.639412in}{1.230741in}}%
\pgfpathlineto{\pgfqpoint{1.641097in}{1.217606in}}%
\pgfpathlineto{\pgfqpoint{1.641518in}{1.237308in}}%
\pgfpathlineto{\pgfqpoint{1.641940in}{1.224173in}}%
\pgfpathlineto{\pgfqpoint{1.643203in}{1.204471in}}%
\pgfpathlineto{\pgfqpoint{1.643625in}{1.230741in}}%
\pgfpathlineto{\pgfqpoint{1.644467in}{1.224173in}}%
\pgfpathlineto{\pgfqpoint{1.644889in}{1.211038in}}%
\pgfpathlineto{\pgfqpoint{1.645310in}{1.224173in}}%
\pgfpathlineto{\pgfqpoint{1.645731in}{1.237308in}}%
\pgfpathlineto{\pgfqpoint{1.646574in}{1.230741in}}%
\pgfpathlineto{\pgfqpoint{1.646995in}{1.211038in}}%
\pgfpathlineto{\pgfqpoint{1.647416in}{1.224173in}}%
\pgfpathlineto{\pgfqpoint{1.647837in}{1.243876in}}%
\pgfpathlineto{\pgfqpoint{1.648259in}{1.211038in}}%
\pgfpathlineto{\pgfqpoint{1.649101in}{1.224173in}}%
\pgfpathlineto{\pgfqpoint{1.649944in}{1.237308in}}%
\pgfpathlineto{\pgfqpoint{1.650365in}{1.230741in}}%
\pgfpathlineto{\pgfqpoint{1.650786in}{1.230741in}}%
\pgfpathlineto{\pgfqpoint{1.651208in}{1.204471in}}%
\pgfpathlineto{\pgfqpoint{1.652050in}{1.217606in}}%
\pgfpathlineto{\pgfqpoint{1.652472in}{1.230741in}}%
\pgfpathlineto{\pgfqpoint{1.652893in}{1.211038in}}%
\pgfpathlineto{\pgfqpoint{1.654578in}{1.230741in}}%
\pgfpathlineto{\pgfqpoint{1.655420in}{1.211038in}}%
\pgfpathlineto{\pgfqpoint{1.656684in}{1.243876in}}%
\pgfpathlineto{\pgfqpoint{1.657527in}{1.204471in}}%
\pgfpathlineto{\pgfqpoint{1.657948in}{1.224173in}}%
\pgfpathlineto{\pgfqpoint{1.658369in}{1.230741in}}%
\pgfpathlineto{\pgfqpoint{1.658791in}{1.217606in}}%
\pgfpathlineto{\pgfqpoint{1.659633in}{1.224173in}}%
\pgfpathlineto{\pgfqpoint{1.660055in}{1.217606in}}%
\pgfpathlineto{\pgfqpoint{1.660476in}{1.224173in}}%
\pgfpathlineto{\pgfqpoint{1.660897in}{1.230741in}}%
\pgfpathlineto{\pgfqpoint{1.661318in}{1.217606in}}%
\pgfpathlineto{\pgfqpoint{1.661740in}{1.230741in}}%
\pgfpathlineto{\pgfqpoint{1.662161in}{1.230741in}}%
\pgfpathlineto{\pgfqpoint{1.662582in}{1.243876in}}%
\pgfpathlineto{\pgfqpoint{1.663846in}{1.211038in}}%
\pgfpathlineto{\pgfqpoint{1.664267in}{1.217606in}}%
\pgfpathlineto{\pgfqpoint{1.664689in}{1.243876in}}%
\pgfpathlineto{\pgfqpoint{1.665531in}{1.230741in}}%
\pgfpathlineto{\pgfqpoint{1.666374in}{1.237308in}}%
\pgfpathlineto{\pgfqpoint{1.666795in}{1.217606in}}%
\pgfpathlineto{\pgfqpoint{1.667216in}{1.224173in}}%
\pgfpathlineto{\pgfqpoint{1.668059in}{1.204471in}}%
\pgfpathlineto{\pgfqpoint{1.668901in}{1.230741in}}%
\pgfpathlineto{\pgfqpoint{1.669323in}{1.224173in}}%
\pgfpathlineto{\pgfqpoint{1.669744in}{1.217606in}}%
\pgfpathlineto{\pgfqpoint{1.670165in}{1.191335in}}%
\pgfpathlineto{\pgfqpoint{1.670586in}{1.230741in}}%
\pgfpathlineto{\pgfqpoint{1.671008in}{1.237308in}}%
\pgfpathlineto{\pgfqpoint{1.672272in}{1.204471in}}%
\pgfpathlineto{\pgfqpoint{1.672693in}{1.211038in}}%
\pgfpathlineto{\pgfqpoint{1.673114in}{1.211038in}}%
\pgfpathlineto{\pgfqpoint{1.673957in}{1.243876in}}%
\pgfpathlineto{\pgfqpoint{1.674378in}{1.211038in}}%
\pgfpathlineto{\pgfqpoint{1.675221in}{1.224173in}}%
\pgfpathlineto{\pgfqpoint{1.676063in}{1.217606in}}%
\pgfpathlineto{\pgfqpoint{1.677327in}{1.237308in}}%
\pgfpathlineto{\pgfqpoint{1.678591in}{1.217606in}}%
\pgfpathlineto{\pgfqpoint{1.680276in}{1.237308in}}%
\pgfpathlineto{\pgfqpoint{1.680697in}{1.211038in}}%
\pgfpathlineto{\pgfqpoint{1.681118in}{1.230741in}}%
\pgfpathlineto{\pgfqpoint{1.681540in}{1.243876in}}%
\pgfpathlineto{\pgfqpoint{1.682382in}{1.237308in}}%
\pgfpathlineto{\pgfqpoint{1.682804in}{1.217606in}}%
\pgfpathlineto{\pgfqpoint{1.683225in}{1.230741in}}%
\pgfpathlineto{\pgfqpoint{1.683646in}{1.243876in}}%
\pgfpathlineto{\pgfqpoint{1.684067in}{1.224173in}}%
\pgfpathlineto{\pgfqpoint{1.684489in}{1.224173in}}%
\pgfpathlineto{\pgfqpoint{1.684910in}{1.217606in}}%
\pgfpathlineto{\pgfqpoint{1.685331in}{1.224173in}}%
\pgfpathlineto{\pgfqpoint{1.685753in}{1.250443in}}%
\pgfpathlineto{\pgfqpoint{1.686174in}{1.237308in}}%
\pgfpathlineto{\pgfqpoint{1.687438in}{1.211038in}}%
\pgfpathlineto{\pgfqpoint{1.687859in}{1.230741in}}%
\pgfpathlineto{\pgfqpoint{1.688280in}{1.224173in}}%
\pgfpathlineto{\pgfqpoint{1.688701in}{1.217606in}}%
\pgfpathlineto{\pgfqpoint{1.689544in}{1.230741in}}%
\pgfpathlineto{\pgfqpoint{1.689965in}{1.224173in}}%
\pgfpathlineto{\pgfqpoint{1.690808in}{1.217606in}}%
\pgfpathlineto{\pgfqpoint{1.692072in}{1.224173in}}%
\pgfpathlineto{\pgfqpoint{1.692493in}{1.224173in}}%
\pgfpathlineto{\pgfqpoint{1.692914in}{1.230741in}}%
\pgfpathlineto{\pgfqpoint{1.693336in}{1.204471in}}%
\pgfpathlineto{\pgfqpoint{1.693757in}{1.217606in}}%
\pgfpathlineto{\pgfqpoint{1.694178in}{1.237308in}}%
\pgfpathlineto{\pgfqpoint{1.694599in}{1.224173in}}%
\pgfpathlineto{\pgfqpoint{1.695442in}{1.224173in}}%
\pgfpathlineto{\pgfqpoint{1.695863in}{1.217606in}}%
\pgfpathlineto{\pgfqpoint{1.697127in}{1.230741in}}%
\pgfpathlineto{\pgfqpoint{1.697970in}{1.224173in}}%
\pgfpathlineto{\pgfqpoint{1.698391in}{1.230741in}}%
\pgfpathlineto{\pgfqpoint{1.698812in}{1.211038in}}%
\pgfpathlineto{\pgfqpoint{1.699233in}{1.230741in}}%
\pgfpathlineto{\pgfqpoint{1.700076in}{1.224173in}}%
\pgfpathlineto{\pgfqpoint{1.701340in}{1.237308in}}%
\pgfpathlineto{\pgfqpoint{1.701761in}{1.217606in}}%
\pgfpathlineto{\pgfqpoint{1.702182in}{1.237308in}}%
\pgfpathlineto{\pgfqpoint{1.703025in}{1.217606in}}%
\pgfpathlineto{\pgfqpoint{1.703867in}{1.224173in}}%
\pgfpathlineto{\pgfqpoint{1.704710in}{1.237308in}}%
\pgfpathlineto{\pgfqpoint{1.705131in}{1.230741in}}%
\pgfpathlineto{\pgfqpoint{1.705553in}{1.230741in}}%
\pgfpathlineto{\pgfqpoint{1.705974in}{1.217606in}}%
\pgfpathlineto{\pgfqpoint{1.706395in}{1.224173in}}%
\pgfpathlineto{\pgfqpoint{1.706816in}{1.243876in}}%
\pgfpathlineto{\pgfqpoint{1.707238in}{1.230741in}}%
\pgfpathlineto{\pgfqpoint{1.708923in}{1.217606in}}%
\pgfpathlineto{\pgfqpoint{1.709765in}{1.217606in}}%
\pgfpathlineto{\pgfqpoint{1.710187in}{1.211038in}}%
\pgfpathlineto{\pgfqpoint{1.710608in}{1.217606in}}%
\pgfpathlineto{\pgfqpoint{1.711450in}{1.230741in}}%
\pgfpathlineto{\pgfqpoint{1.711872in}{1.224173in}}%
\pgfpathlineto{\pgfqpoint{1.712714in}{1.211038in}}%
\pgfpathlineto{\pgfqpoint{1.713136in}{1.217606in}}%
\pgfpathlineto{\pgfqpoint{1.713557in}{1.230741in}}%
\pgfpathlineto{\pgfqpoint{1.713978in}{1.217606in}}%
\pgfpathlineto{\pgfqpoint{1.714399in}{1.204471in}}%
\pgfpathlineto{\pgfqpoint{1.714821in}{1.211038in}}%
\pgfpathlineto{\pgfqpoint{1.715242in}{1.237308in}}%
\pgfpathlineto{\pgfqpoint{1.715663in}{1.224173in}}%
\pgfpathlineto{\pgfqpoint{1.716085in}{1.211038in}}%
\pgfpathlineto{\pgfqpoint{1.716506in}{1.224173in}}%
\pgfpathlineto{\pgfqpoint{1.716927in}{1.224173in}}%
\pgfpathlineto{\pgfqpoint{1.718191in}{1.211038in}}%
\pgfpathlineto{\pgfqpoint{1.719455in}{1.230741in}}%
\pgfpathlineto{\pgfqpoint{1.720719in}{1.211038in}}%
\pgfpathlineto{\pgfqpoint{1.721982in}{1.230741in}}%
\pgfpathlineto{\pgfqpoint{1.722825in}{1.217606in}}%
\pgfpathlineto{\pgfqpoint{1.723246in}{1.224173in}}%
\pgfpathlineto{\pgfqpoint{1.723668in}{1.237308in}}%
\pgfpathlineto{\pgfqpoint{1.724089in}{1.224173in}}%
\pgfpathlineto{\pgfqpoint{1.724510in}{1.224173in}}%
\pgfpathlineto{\pgfqpoint{1.724931in}{1.211038in}}%
\pgfpathlineto{\pgfqpoint{1.725353in}{1.217606in}}%
\pgfpathlineto{\pgfqpoint{1.725774in}{1.237308in}}%
\pgfpathlineto{\pgfqpoint{1.726616in}{1.230741in}}%
\pgfpathlineto{\pgfqpoint{1.727038in}{1.204471in}}%
\pgfpathlineto{\pgfqpoint{1.727459in}{1.224173in}}%
\pgfpathlineto{\pgfqpoint{1.727880in}{1.243876in}}%
\pgfpathlineto{\pgfqpoint{1.728302in}{1.224173in}}%
\pgfpathlineto{\pgfqpoint{1.728723in}{1.224173in}}%
\pgfpathlineto{\pgfqpoint{1.729144in}{1.211038in}}%
\pgfpathlineto{\pgfqpoint{1.729565in}{1.224173in}}%
\pgfpathlineto{\pgfqpoint{1.729987in}{1.224173in}}%
\pgfpathlineto{\pgfqpoint{1.730408in}{1.217606in}}%
\pgfpathlineto{\pgfqpoint{1.730829in}{1.230741in}}%
\pgfpathlineto{\pgfqpoint{1.731251in}{1.197903in}}%
\pgfpathlineto{\pgfqpoint{1.731672in}{1.237308in}}%
\pgfpathlineto{\pgfqpoint{1.732093in}{1.211038in}}%
\pgfpathlineto{\pgfqpoint{1.732514in}{1.230741in}}%
\pgfpathlineto{\pgfqpoint{1.732936in}{1.224173in}}%
\pgfpathlineto{\pgfqpoint{1.733778in}{1.211038in}}%
\pgfpathlineto{\pgfqpoint{1.735042in}{1.230741in}}%
\pgfpathlineto{\pgfqpoint{1.735463in}{1.217606in}}%
\pgfpathlineto{\pgfqpoint{1.735885in}{1.224173in}}%
\pgfpathlineto{\pgfqpoint{1.736306in}{1.237308in}}%
\pgfpathlineto{\pgfqpoint{1.736727in}{1.224173in}}%
\pgfpathlineto{\pgfqpoint{1.737570in}{1.224173in}}%
\pgfpathlineto{\pgfqpoint{1.737991in}{1.217606in}}%
\pgfpathlineto{\pgfqpoint{1.739255in}{1.230741in}}%
\pgfpathlineto{\pgfqpoint{1.740097in}{1.211038in}}%
\pgfpathlineto{\pgfqpoint{1.740519in}{1.237308in}}%
\pgfpathlineto{\pgfqpoint{1.741361in}{1.230741in}}%
\pgfpathlineto{\pgfqpoint{1.742204in}{1.217606in}}%
\pgfpathlineto{\pgfqpoint{1.742625in}{1.237308in}}%
\pgfpathlineto{\pgfqpoint{1.743046in}{1.224173in}}%
\pgfpathlineto{\pgfqpoint{1.743468in}{1.217606in}}%
\pgfpathlineto{\pgfqpoint{1.743889in}{1.224173in}}%
\pgfpathlineto{\pgfqpoint{1.744731in}{1.243876in}}%
\pgfpathlineto{\pgfqpoint{1.745995in}{1.211038in}}%
\pgfpathlineto{\pgfqpoint{1.747680in}{1.230741in}}%
\pgfpathlineto{\pgfqpoint{1.748523in}{1.217606in}}%
\pgfpathlineto{\pgfqpoint{1.748944in}{1.224173in}}%
\pgfpathlineto{\pgfqpoint{1.749787in}{1.224173in}}%
\pgfpathlineto{\pgfqpoint{1.750208in}{1.211038in}}%
\pgfpathlineto{\pgfqpoint{1.750629in}{1.230741in}}%
\pgfpathlineto{\pgfqpoint{1.751051in}{1.237308in}}%
\pgfpathlineto{\pgfqpoint{1.752314in}{1.204471in}}%
\pgfpathlineto{\pgfqpoint{1.753578in}{1.230741in}}%
\pgfpathlineto{\pgfqpoint{1.754000in}{1.230741in}}%
\pgfpathlineto{\pgfqpoint{1.754842in}{1.211038in}}%
\pgfpathlineto{\pgfqpoint{1.755685in}{1.230741in}}%
\pgfpathlineto{\pgfqpoint{1.756527in}{1.211038in}}%
\pgfpathlineto{\pgfqpoint{1.756949in}{1.243876in}}%
\pgfpathlineto{\pgfqpoint{1.757370in}{1.224173in}}%
\pgfpathlineto{\pgfqpoint{1.757791in}{1.211038in}}%
\pgfpathlineto{\pgfqpoint{1.758212in}{1.230741in}}%
\pgfpathlineto{\pgfqpoint{1.758634in}{1.217606in}}%
\pgfpathlineto{\pgfqpoint{1.759055in}{1.211038in}}%
\pgfpathlineto{\pgfqpoint{1.759476in}{1.237308in}}%
\pgfpathlineto{\pgfqpoint{1.760319in}{1.230741in}}%
\pgfpathlineto{\pgfqpoint{1.760740in}{1.211038in}}%
\pgfpathlineto{\pgfqpoint{1.761161in}{1.217606in}}%
\pgfpathlineto{\pgfqpoint{1.761583in}{1.224173in}}%
\pgfpathlineto{\pgfqpoint{1.762004in}{1.217606in}}%
\pgfpathlineto{\pgfqpoint{1.762425in}{1.211038in}}%
\pgfpathlineto{\pgfqpoint{1.763689in}{1.230741in}}%
\pgfpathlineto{\pgfqpoint{1.764532in}{1.204471in}}%
\pgfpathlineto{\pgfqpoint{1.764953in}{1.217606in}}%
\pgfpathlineto{\pgfqpoint{1.765374in}{1.217606in}}%
\pgfpathlineto{\pgfqpoint{1.765795in}{1.237308in}}%
\pgfpathlineto{\pgfqpoint{1.766217in}{1.211038in}}%
\pgfpathlineto{\pgfqpoint{1.766638in}{1.204471in}}%
\pgfpathlineto{\pgfqpoint{1.767059in}{1.211038in}}%
\pgfpathlineto{\pgfqpoint{1.767480in}{1.276714in}}%
\pgfpathlineto{\pgfqpoint{1.767902in}{1.224173in}}%
\pgfpathlineto{\pgfqpoint{1.768323in}{1.224173in}}%
\pgfpathlineto{\pgfqpoint{1.768744in}{0.981174in}}%
\pgfpathlineto{\pgfqpoint{1.769166in}{1.204471in}}%
\pgfpathlineto{\pgfqpoint{1.770008in}{1.230741in}}%
\pgfpathlineto{\pgfqpoint{1.771272in}{1.204471in}}%
\pgfpathlineto{\pgfqpoint{1.772115in}{1.230741in}}%
\pgfpathlineto{\pgfqpoint{1.772957in}{1.224173in}}%
\pgfpathlineto{\pgfqpoint{1.773800in}{1.204471in}}%
\pgfpathlineto{\pgfqpoint{1.774642in}{1.224173in}}%
\pgfpathlineto{\pgfqpoint{1.775063in}{1.217606in}}%
\pgfpathlineto{\pgfqpoint{1.775906in}{1.230741in}}%
\pgfpathlineto{\pgfqpoint{1.776327in}{1.204471in}}%
\pgfpathlineto{\pgfqpoint{1.776749in}{1.211038in}}%
\pgfpathlineto{\pgfqpoint{1.778012in}{1.224173in}}%
\pgfpathlineto{\pgfqpoint{1.778855in}{1.204471in}}%
\pgfpathlineto{\pgfqpoint{1.779276in}{1.053417in}}%
\pgfpathlineto{\pgfqpoint{1.779698in}{1.230741in}}%
\pgfpathlineto{\pgfqpoint{1.780119in}{1.224173in}}%
\pgfpathlineto{\pgfqpoint{1.780540in}{1.237308in}}%
\pgfpathlineto{\pgfqpoint{1.780961in}{1.165065in}}%
\pgfpathlineto{\pgfqpoint{1.781383in}{1.224173in}}%
\pgfpathlineto{\pgfqpoint{1.781804in}{1.217606in}}%
\pgfpathlineto{\pgfqpoint{1.782225in}{1.467172in}}%
\pgfpathlineto{\pgfqpoint{1.782646in}{1.237308in}}%
\pgfpathlineto{\pgfqpoint{1.783910in}{1.217606in}}%
\pgfpathlineto{\pgfqpoint{1.784332in}{1.217606in}}%
\pgfpathlineto{\pgfqpoint{1.784753in}{1.237308in}}%
\pgfpathlineto{\pgfqpoint{1.785174in}{1.217606in}}%
\pgfpathlineto{\pgfqpoint{1.786017in}{1.211038in}}%
\pgfpathlineto{\pgfqpoint{1.787702in}{1.230741in}}%
\pgfpathlineto{\pgfqpoint{1.788123in}{1.217606in}}%
\pgfpathlineto{\pgfqpoint{1.788966in}{1.224173in}}%
\pgfpathlineto{\pgfqpoint{1.789808in}{1.230741in}}%
\pgfpathlineto{\pgfqpoint{1.790229in}{1.211038in}}%
\pgfpathlineto{\pgfqpoint{1.790651in}{1.224173in}}%
\pgfpathlineto{\pgfqpoint{1.791915in}{1.237308in}}%
\pgfpathlineto{\pgfqpoint{1.792336in}{1.230741in}}%
\pgfpathlineto{\pgfqpoint{1.792757in}{1.394929in}}%
\pgfpathlineto{\pgfqpoint{1.793178in}{1.224173in}}%
\pgfpathlineto{\pgfqpoint{1.793600in}{1.237308in}}%
\pgfpathlineto{\pgfqpoint{1.794021in}{1.230741in}}%
\pgfpathlineto{\pgfqpoint{1.795706in}{1.211038in}}%
\pgfpathlineto{\pgfqpoint{1.797391in}{1.237308in}}%
\pgfpathlineto{\pgfqpoint{1.797812in}{1.230741in}}%
\pgfpathlineto{\pgfqpoint{1.799076in}{1.230741in}}%
\pgfpathlineto{\pgfqpoint{1.799498in}{1.243876in}}%
\pgfpathlineto{\pgfqpoint{1.799919in}{1.217606in}}%
\pgfpathlineto{\pgfqpoint{1.800761in}{1.230741in}}%
\pgfpathlineto{\pgfqpoint{1.801183in}{1.217606in}}%
\pgfpathlineto{\pgfqpoint{1.802025in}{1.224173in}}%
\pgfpathlineto{\pgfqpoint{1.802447in}{1.230741in}}%
\pgfpathlineto{\pgfqpoint{1.802868in}{1.204471in}}%
\pgfpathlineto{\pgfqpoint{1.803289in}{1.224173in}}%
\pgfpathlineto{\pgfqpoint{1.803710in}{1.250443in}}%
\pgfpathlineto{\pgfqpoint{1.804132in}{1.237308in}}%
\pgfpathlineto{\pgfqpoint{1.804974in}{1.217606in}}%
\pgfpathlineto{\pgfqpoint{1.805396in}{1.224173in}}%
\pgfpathlineto{\pgfqpoint{1.805817in}{1.224173in}}%
\pgfpathlineto{\pgfqpoint{1.806238in}{1.230741in}}%
\pgfpathlineto{\pgfqpoint{1.807502in}{1.217606in}}%
\pgfpathlineto{\pgfqpoint{1.808766in}{1.230741in}}%
\pgfpathlineto{\pgfqpoint{1.809187in}{1.211038in}}%
\pgfpathlineto{\pgfqpoint{1.809608in}{1.237308in}}%
\pgfpathlineto{\pgfqpoint{1.810872in}{1.211038in}}%
\pgfpathlineto{\pgfqpoint{1.812557in}{1.230741in}}%
\pgfpathlineto{\pgfqpoint{1.812979in}{1.224173in}}%
\pgfpathlineto{\pgfqpoint{1.813821in}{1.217606in}}%
\pgfpathlineto{\pgfqpoint{1.814664in}{1.243876in}}%
\pgfpathlineto{\pgfqpoint{1.815927in}{1.224173in}}%
\pgfpathlineto{\pgfqpoint{1.816349in}{1.243876in}}%
\pgfpathlineto{\pgfqpoint{1.816770in}{1.237308in}}%
\pgfpathlineto{\pgfqpoint{1.817613in}{1.211038in}}%
\pgfpathlineto{\pgfqpoint{1.818034in}{1.230741in}}%
\pgfpathlineto{\pgfqpoint{1.819298in}{1.237308in}}%
\pgfpathlineto{\pgfqpoint{1.819719in}{1.224173in}}%
\pgfpathlineto{\pgfqpoint{1.820562in}{1.230741in}}%
\pgfpathlineto{\pgfqpoint{1.821404in}{1.230741in}}%
\pgfpathlineto{\pgfqpoint{1.821825in}{1.217606in}}%
\pgfpathlineto{\pgfqpoint{1.822247in}{1.224173in}}%
\pgfpathlineto{\pgfqpoint{1.822668in}{1.250443in}}%
\pgfpathlineto{\pgfqpoint{1.823089in}{1.224173in}}%
\pgfpathlineto{\pgfqpoint{1.823510in}{1.230741in}}%
\pgfpathlineto{\pgfqpoint{1.823932in}{1.217606in}}%
\pgfpathlineto{\pgfqpoint{1.824353in}{1.237308in}}%
\pgfpathlineto{\pgfqpoint{1.824774in}{1.224173in}}%
\pgfpathlineto{\pgfqpoint{1.825196in}{1.230741in}}%
\pgfpathlineto{\pgfqpoint{1.826038in}{1.217606in}}%
\pgfpathlineto{\pgfqpoint{1.826459in}{1.224173in}}%
\pgfpathlineto{\pgfqpoint{1.826881in}{1.217606in}}%
\pgfpathlineto{\pgfqpoint{1.827302in}{1.230741in}}%
\pgfpathlineto{\pgfqpoint{1.827723in}{1.224173in}}%
\pgfpathlineto{\pgfqpoint{1.828145in}{1.211038in}}%
\pgfpathlineto{\pgfqpoint{1.828566in}{1.224173in}}%
\pgfpathlineto{\pgfqpoint{1.828987in}{1.237308in}}%
\pgfpathlineto{\pgfqpoint{1.829408in}{1.224173in}}%
\pgfpathlineto{\pgfqpoint{1.829830in}{1.224173in}}%
\pgfpathlineto{\pgfqpoint{1.830251in}{1.211038in}}%
\pgfpathlineto{\pgfqpoint{1.830672in}{1.224173in}}%
\pgfpathlineto{\pgfqpoint{1.831093in}{1.230741in}}%
\pgfpathlineto{\pgfqpoint{1.831515in}{1.204471in}}%
\pgfpathlineto{\pgfqpoint{1.832357in}{1.211038in}}%
\pgfpathlineto{\pgfqpoint{1.833200in}{1.230741in}}%
\pgfpathlineto{\pgfqpoint{1.834464in}{1.197903in}}%
\pgfpathlineto{\pgfqpoint{1.835306in}{1.237308in}}%
\pgfpathlineto{\pgfqpoint{1.835728in}{1.230741in}}%
\pgfpathlineto{\pgfqpoint{1.836149in}{1.230741in}}%
\pgfpathlineto{\pgfqpoint{1.836570in}{1.217606in}}%
\pgfpathlineto{\pgfqpoint{1.837413in}{1.224173in}}%
\pgfpathlineto{\pgfqpoint{1.838255in}{1.230741in}}%
\pgfpathlineto{\pgfqpoint{1.838676in}{1.211038in}}%
\pgfpathlineto{\pgfqpoint{1.839098in}{1.224173in}}%
\pgfpathlineto{\pgfqpoint{1.839519in}{1.230741in}}%
\pgfpathlineto{\pgfqpoint{1.839940in}{1.224173in}}%
\pgfpathlineto{\pgfqpoint{1.840362in}{1.224173in}}%
\pgfpathlineto{\pgfqpoint{1.840783in}{1.204471in}}%
\pgfpathlineto{\pgfqpoint{1.841204in}{1.224173in}}%
\pgfpathlineto{\pgfqpoint{1.842468in}{1.224173in}}%
\pgfpathlineto{\pgfqpoint{1.842889in}{1.217606in}}%
\pgfpathlineto{\pgfqpoint{1.843311in}{1.224173in}}%
\pgfpathlineto{\pgfqpoint{1.843732in}{1.230741in}}%
\pgfpathlineto{\pgfqpoint{1.844153in}{1.217606in}}%
\pgfpathlineto{\pgfqpoint{1.844574in}{1.230741in}}%
\pgfpathlineto{\pgfqpoint{1.844996in}{1.237308in}}%
\pgfpathlineto{\pgfqpoint{1.846681in}{1.217606in}}%
\pgfpathlineto{\pgfqpoint{1.847945in}{1.250443in}}%
\pgfpathlineto{\pgfqpoint{1.849208in}{1.204471in}}%
\pgfpathlineto{\pgfqpoint{1.849630in}{1.217606in}}%
\pgfpathlineto{\pgfqpoint{1.850051in}{1.243876in}}%
\pgfpathlineto{\pgfqpoint{1.850894in}{1.230741in}}%
\pgfpathlineto{\pgfqpoint{1.851315in}{1.211038in}}%
\pgfpathlineto{\pgfqpoint{1.851736in}{1.224173in}}%
\pgfpathlineto{\pgfqpoint{1.852157in}{1.237308in}}%
\pgfpathlineto{\pgfqpoint{1.852579in}{1.230741in}}%
\pgfpathlineto{\pgfqpoint{1.853421in}{1.211038in}}%
\pgfpathlineto{\pgfqpoint{1.853842in}{1.224173in}}%
\pgfpathlineto{\pgfqpoint{1.854264in}{1.243876in}}%
\pgfpathlineto{\pgfqpoint{1.854685in}{1.224173in}}%
\pgfpathlineto{\pgfqpoint{1.855106in}{1.230741in}}%
\pgfpathlineto{\pgfqpoint{1.855949in}{1.211038in}}%
\pgfpathlineto{\pgfqpoint{1.857213in}{1.230741in}}%
\pgfpathlineto{\pgfqpoint{1.857634in}{1.224173in}}%
\pgfpathlineto{\pgfqpoint{1.858055in}{1.230741in}}%
\pgfpathlineto{\pgfqpoint{1.858477in}{1.237308in}}%
\pgfpathlineto{\pgfqpoint{1.859740in}{1.204471in}}%
\pgfpathlineto{\pgfqpoint{1.861004in}{1.230741in}}%
\pgfpathlineto{\pgfqpoint{1.861847in}{1.204471in}}%
\pgfpathlineto{\pgfqpoint{1.862689in}{1.250443in}}%
\pgfpathlineto{\pgfqpoint{1.863111in}{1.237308in}}%
\pgfpathlineto{\pgfqpoint{1.863532in}{1.204471in}}%
\pgfpathlineto{\pgfqpoint{1.864374in}{1.224173in}}%
\pgfpathlineto{\pgfqpoint{1.865217in}{1.230741in}}%
\pgfpathlineto{\pgfqpoint{1.866060in}{1.211038in}}%
\pgfpathlineto{\pgfqpoint{1.866481in}{1.224173in}}%
\pgfpathlineto{\pgfqpoint{1.866902in}{1.237308in}}%
\pgfpathlineto{\pgfqpoint{1.867323in}{1.230741in}}%
\pgfpathlineto{\pgfqpoint{1.868587in}{1.224173in}}%
\pgfpathlineto{\pgfqpoint{1.869430in}{1.237308in}}%
\pgfpathlineto{\pgfqpoint{1.869851in}{1.230741in}}%
\pgfpathlineto{\pgfqpoint{1.870272in}{1.197903in}}%
\pgfpathlineto{\pgfqpoint{1.870694in}{1.217606in}}%
\pgfpathlineto{\pgfqpoint{1.871115in}{1.237308in}}%
\pgfpathlineto{\pgfqpoint{1.871536in}{1.211038in}}%
\pgfpathlineto{\pgfqpoint{1.873221in}{1.237308in}}%
\pgfpathlineto{\pgfqpoint{1.874485in}{1.217606in}}%
\pgfpathlineto{\pgfqpoint{1.874906in}{1.224173in}}%
\pgfpathlineto{\pgfqpoint{1.875328in}{1.237308in}}%
\pgfpathlineto{\pgfqpoint{1.876592in}{1.211038in}}%
\pgfpathlineto{\pgfqpoint{1.877434in}{1.230741in}}%
\pgfpathlineto{\pgfqpoint{1.878277in}{1.224173in}}%
\pgfpathlineto{\pgfqpoint{1.878698in}{1.211038in}}%
\pgfpathlineto{\pgfqpoint{1.879119in}{1.224173in}}%
\pgfpathlineto{\pgfqpoint{1.879540in}{1.230741in}}%
\pgfpathlineto{\pgfqpoint{1.880804in}{1.211038in}}%
\pgfpathlineto{\pgfqpoint{1.882489in}{1.230741in}}%
\pgfpathlineto{\pgfqpoint{1.883332in}{1.224173in}}%
\pgfpathlineto{\pgfqpoint{1.883753in}{1.237308in}}%
\pgfpathlineto{\pgfqpoint{1.884175in}{1.230741in}}%
\pgfpathlineto{\pgfqpoint{1.885017in}{1.204471in}}%
\pgfpathlineto{\pgfqpoint{1.885438in}{1.217606in}}%
\pgfpathlineto{\pgfqpoint{1.885860in}{1.217606in}}%
\pgfpathlineto{\pgfqpoint{1.886702in}{1.230741in}}%
\pgfpathlineto{\pgfqpoint{1.887123in}{1.224173in}}%
\pgfpathlineto{\pgfqpoint{1.887545in}{1.224173in}}%
\pgfpathlineto{\pgfqpoint{1.888387in}{1.230741in}}%
\pgfpathlineto{\pgfqpoint{1.889230in}{1.211038in}}%
\pgfpathlineto{\pgfqpoint{1.889651in}{1.224173in}}%
\pgfpathlineto{\pgfqpoint{1.890072in}{1.217606in}}%
\pgfpathlineto{\pgfqpoint{1.890494in}{1.237308in}}%
\pgfpathlineto{\pgfqpoint{1.890915in}{1.217606in}}%
\pgfpathlineto{\pgfqpoint{1.891336in}{1.217606in}}%
\pgfpathlineto{\pgfqpoint{1.893021in}{1.237308in}}%
\pgfpathlineto{\pgfqpoint{1.894706in}{1.217606in}}%
\pgfpathlineto{\pgfqpoint{1.895128in}{1.230741in}}%
\pgfpathlineto{\pgfqpoint{1.895549in}{1.224173in}}%
\pgfpathlineto{\pgfqpoint{1.896392in}{1.217606in}}%
\pgfpathlineto{\pgfqpoint{1.897234in}{1.224173in}}%
\pgfpathlineto{\pgfqpoint{1.897655in}{1.217606in}}%
\pgfpathlineto{\pgfqpoint{1.898077in}{1.224173in}}%
\pgfpathlineto{\pgfqpoint{1.898498in}{1.250443in}}%
\pgfpathlineto{\pgfqpoint{1.898919in}{1.230741in}}%
\pgfpathlineto{\pgfqpoint{1.899341in}{1.230741in}}%
\pgfpathlineto{\pgfqpoint{1.900604in}{1.217606in}}%
\pgfpathlineto{\pgfqpoint{1.901026in}{1.217606in}}%
\pgfpathlineto{\pgfqpoint{1.902711in}{1.243876in}}%
\pgfpathlineto{\pgfqpoint{1.903975in}{1.204471in}}%
\pgfpathlineto{\pgfqpoint{1.904817in}{1.237308in}}%
\pgfpathlineto{\pgfqpoint{1.905238in}{1.211038in}}%
\pgfpathlineto{\pgfqpoint{1.906081in}{1.217606in}}%
\pgfpathlineto{\pgfqpoint{1.906924in}{1.217606in}}%
\pgfpathlineto{\pgfqpoint{1.907345in}{1.224173in}}%
\pgfpathlineto{\pgfqpoint{1.907766in}{1.217606in}}%
\pgfpathlineto{\pgfqpoint{1.908187in}{1.211038in}}%
\pgfpathlineto{\pgfqpoint{1.908609in}{1.217606in}}%
\pgfpathlineto{\pgfqpoint{1.909030in}{1.217606in}}%
\pgfpathlineto{\pgfqpoint{1.909451in}{1.237308in}}%
\pgfpathlineto{\pgfqpoint{1.909872in}{1.224173in}}%
\pgfpathlineto{\pgfqpoint{1.910294in}{1.204471in}}%
\pgfpathlineto{\pgfqpoint{1.910715in}{1.224173in}}%
\pgfpathlineto{\pgfqpoint{1.911558in}{1.224173in}}%
\pgfpathlineto{\pgfqpoint{1.912400in}{1.211038in}}%
\pgfpathlineto{\pgfqpoint{1.913243in}{1.230741in}}%
\pgfpathlineto{\pgfqpoint{1.913664in}{1.224173in}}%
\pgfpathlineto{\pgfqpoint{1.914507in}{1.224173in}}%
\pgfpathlineto{\pgfqpoint{1.914928in}{1.217606in}}%
\pgfpathlineto{\pgfqpoint{1.915349in}{1.224173in}}%
\pgfpathlineto{\pgfqpoint{1.915770in}{1.224173in}}%
\pgfpathlineto{\pgfqpoint{1.916192in}{1.217606in}}%
\pgfpathlineto{\pgfqpoint{1.916613in}{1.224173in}}%
\pgfpathlineto{\pgfqpoint{1.917034in}{1.224173in}}%
\pgfpathlineto{\pgfqpoint{1.917455in}{1.243876in}}%
\pgfpathlineto{\pgfqpoint{1.917877in}{1.211038in}}%
\pgfpathlineto{\pgfqpoint{1.918719in}{1.224173in}}%
\pgfpathlineto{\pgfqpoint{1.919141in}{1.224173in}}%
\pgfpathlineto{\pgfqpoint{1.919562in}{1.230741in}}%
\pgfpathlineto{\pgfqpoint{1.919983in}{1.224173in}}%
\pgfpathlineto{\pgfqpoint{1.920404in}{1.224173in}}%
\pgfpathlineto{\pgfqpoint{1.920826in}{1.204471in}}%
\pgfpathlineto{\pgfqpoint{1.921247in}{1.230741in}}%
\pgfpathlineto{\pgfqpoint{1.921668in}{1.243876in}}%
\pgfpathlineto{\pgfqpoint{1.922090in}{1.230741in}}%
\pgfpathlineto{\pgfqpoint{1.922932in}{1.211038in}}%
\pgfpathlineto{\pgfqpoint{1.923775in}{1.237308in}}%
\pgfpathlineto{\pgfqpoint{1.924196in}{1.230741in}}%
\pgfpathlineto{\pgfqpoint{1.925038in}{1.211038in}}%
\pgfpathlineto{\pgfqpoint{1.925460in}{1.224173in}}%
\pgfpathlineto{\pgfqpoint{1.925881in}{1.237308in}}%
\pgfpathlineto{\pgfqpoint{1.926302in}{1.217606in}}%
\pgfpathlineto{\pgfqpoint{1.928409in}{1.237308in}}%
\pgfpathlineto{\pgfqpoint{1.929251in}{1.211038in}}%
\pgfpathlineto{\pgfqpoint{1.929673in}{1.217606in}}%
\pgfpathlineto{\pgfqpoint{1.930094in}{1.230741in}}%
\pgfpathlineto{\pgfqpoint{1.930515in}{1.217606in}}%
\pgfpathlineto{\pgfqpoint{1.930936in}{1.217606in}}%
\pgfpathlineto{\pgfqpoint{1.932200in}{1.230741in}}%
\pgfpathlineto{\pgfqpoint{1.932621in}{1.217606in}}%
\pgfpathlineto{\pgfqpoint{1.933043in}{1.224173in}}%
\pgfpathlineto{\pgfqpoint{1.934307in}{1.243876in}}%
\pgfpathlineto{\pgfqpoint{1.935149in}{1.211038in}}%
\pgfpathlineto{\pgfqpoint{1.935992in}{1.224173in}}%
\pgfpathlineto{\pgfqpoint{1.936834in}{1.224173in}}%
\pgfpathlineto{\pgfqpoint{1.937256in}{1.230741in}}%
\pgfpathlineto{\pgfqpoint{1.937677in}{1.217606in}}%
\pgfpathlineto{\pgfqpoint{1.938098in}{1.230741in}}%
\pgfpathlineto{\pgfqpoint{1.938519in}{1.243876in}}%
\pgfpathlineto{\pgfqpoint{1.938941in}{1.230741in}}%
\pgfpathlineto{\pgfqpoint{1.939783in}{1.211038in}}%
\pgfpathlineto{\pgfqpoint{1.940626in}{1.217606in}}%
\pgfpathlineto{\pgfqpoint{1.941047in}{1.224173in}}%
\pgfpathlineto{\pgfqpoint{1.941468in}{1.217606in}}%
\pgfpathlineto{\pgfqpoint{1.941890in}{1.191335in}}%
\pgfpathlineto{\pgfqpoint{1.942311in}{1.217606in}}%
\pgfpathlineto{\pgfqpoint{1.942732in}{1.243876in}}%
\pgfpathlineto{\pgfqpoint{1.943153in}{1.237308in}}%
\pgfpathlineto{\pgfqpoint{1.943996in}{1.211038in}}%
\pgfpathlineto{\pgfqpoint{1.944417in}{1.224173in}}%
\pgfpathlineto{\pgfqpoint{1.944839in}{1.230741in}}%
\pgfpathlineto{\pgfqpoint{1.946524in}{1.211038in}}%
\pgfpathlineto{\pgfqpoint{1.947366in}{1.230741in}}%
\pgfpathlineto{\pgfqpoint{1.947788in}{1.217606in}}%
\pgfpathlineto{\pgfqpoint{1.948209in}{1.211038in}}%
\pgfpathlineto{\pgfqpoint{1.949894in}{1.230741in}}%
\pgfpathlineto{\pgfqpoint{1.950315in}{1.204471in}}%
\pgfpathlineto{\pgfqpoint{1.950736in}{1.230741in}}%
\pgfpathlineto{\pgfqpoint{1.952422in}{1.211038in}}%
\pgfpathlineto{\pgfqpoint{1.953264in}{1.243876in}}%
\pgfpathlineto{\pgfqpoint{1.954107in}{1.230741in}}%
\pgfpathlineto{\pgfqpoint{1.954528in}{1.211038in}}%
\pgfpathlineto{\pgfqpoint{1.954949in}{1.224173in}}%
\pgfpathlineto{\pgfqpoint{1.955371in}{1.250443in}}%
\pgfpathlineto{\pgfqpoint{1.955792in}{1.224173in}}%
\pgfpathlineto{\pgfqpoint{1.957056in}{1.217606in}}%
\pgfpathlineto{\pgfqpoint{1.957477in}{1.237308in}}%
\pgfpathlineto{\pgfqpoint{1.958319in}{1.230741in}}%
\pgfpathlineto{\pgfqpoint{1.959583in}{1.230741in}}%
\pgfpathlineto{\pgfqpoint{1.960005in}{1.237308in}}%
\pgfpathlineto{\pgfqpoint{1.960847in}{1.224173in}}%
\pgfpathlineto{\pgfqpoint{1.961268in}{1.230741in}}%
\pgfpathlineto{\pgfqpoint{1.961690in}{1.230741in}}%
\pgfpathlineto{\pgfqpoint{1.962111in}{1.243876in}}%
\pgfpathlineto{\pgfqpoint{1.962532in}{1.224173in}}%
\pgfpathlineto{\pgfqpoint{1.963796in}{1.250443in}}%
\pgfpathlineto{\pgfqpoint{1.964217in}{1.217606in}}%
\pgfpathlineto{\pgfqpoint{1.965060in}{1.230741in}}%
\pgfpathlineto{\pgfqpoint{1.965481in}{1.230741in}}%
\pgfpathlineto{\pgfqpoint{1.965902in}{1.243876in}}%
\pgfpathlineto{\pgfqpoint{1.966324in}{1.224173in}}%
\pgfpathlineto{\pgfqpoint{1.966745in}{1.224173in}}%
\pgfpathlineto{\pgfqpoint{1.967166in}{1.204471in}}%
\pgfpathlineto{\pgfqpoint{1.968009in}{1.237308in}}%
\pgfpathlineto{\pgfqpoint{1.968430in}{1.230741in}}%
\pgfpathlineto{\pgfqpoint{1.968851in}{1.178200in}}%
\pgfpathlineto{\pgfqpoint{1.969694in}{1.211038in}}%
\pgfpathlineto{\pgfqpoint{1.970115in}{1.211038in}}%
\pgfpathlineto{\pgfqpoint{1.970958in}{1.230741in}}%
\pgfpathlineto{\pgfqpoint{1.972222in}{1.211038in}}%
\pgfpathlineto{\pgfqpoint{1.973485in}{1.204471in}}%
\pgfpathlineto{\pgfqpoint{1.975171in}{1.224173in}}%
\pgfpathlineto{\pgfqpoint{1.976013in}{1.230741in}}%
\pgfpathlineto{\pgfqpoint{1.976434in}{1.204471in}}%
\pgfpathlineto{\pgfqpoint{1.976856in}{1.204471in}}%
\pgfpathlineto{\pgfqpoint{1.977277in}{1.197903in}}%
\pgfpathlineto{\pgfqpoint{1.977698in}{1.243876in}}%
\pgfpathlineto{\pgfqpoint{1.978120in}{1.211038in}}%
\pgfpathlineto{\pgfqpoint{1.978541in}{1.211038in}}%
\pgfpathlineto{\pgfqpoint{1.979383in}{1.204471in}}%
\pgfpathlineto{\pgfqpoint{1.981068in}{1.224173in}}%
\pgfpathlineto{\pgfqpoint{1.981911in}{1.211038in}}%
\pgfpathlineto{\pgfqpoint{1.982332in}{1.257011in}}%
\pgfpathlineto{\pgfqpoint{1.983175in}{1.230741in}}%
\pgfpathlineto{\pgfqpoint{1.983596in}{1.230741in}}%
\pgfpathlineto{\pgfqpoint{1.984439in}{1.217606in}}%
\pgfpathlineto{\pgfqpoint{1.985703in}{1.230741in}}%
\pgfpathlineto{\pgfqpoint{1.986545in}{1.217606in}}%
\pgfpathlineto{\pgfqpoint{1.986966in}{1.243876in}}%
\pgfpathlineto{\pgfqpoint{1.987809in}{1.237308in}}%
\pgfpathlineto{\pgfqpoint{1.988230in}{1.237308in}}%
\pgfpathlineto{\pgfqpoint{1.989494in}{1.204471in}}%
\pgfpathlineto{\pgfqpoint{1.990758in}{1.237308in}}%
\pgfpathlineto{\pgfqpoint{1.991179in}{1.184768in}}%
\pgfpathlineto{\pgfqpoint{1.991600in}{1.230741in}}%
\pgfpathlineto{\pgfqpoint{1.993286in}{1.217606in}}%
\pgfpathlineto{\pgfqpoint{1.994128in}{1.230741in}}%
\pgfpathlineto{\pgfqpoint{1.994549in}{1.191335in}}%
\pgfpathlineto{\pgfqpoint{1.994971in}{1.217606in}}%
\pgfpathlineto{\pgfqpoint{1.995813in}{1.237308in}}%
\pgfpathlineto{\pgfqpoint{1.996234in}{1.224173in}}%
\pgfpathlineto{\pgfqpoint{1.996656in}{1.217606in}}%
\pgfpathlineto{\pgfqpoint{1.997077in}{1.224173in}}%
\pgfpathlineto{\pgfqpoint{1.997498in}{1.224173in}}%
\pgfpathlineto{\pgfqpoint{1.997920in}{1.230741in}}%
\pgfpathlineto{\pgfqpoint{1.998341in}{1.204471in}}%
\pgfpathlineto{\pgfqpoint{1.999183in}{1.217606in}}%
\pgfpathlineto{\pgfqpoint{1.999605in}{1.237308in}}%
\pgfpathlineto{\pgfqpoint{2.000026in}{1.230741in}}%
\pgfpathlineto{\pgfqpoint{2.001290in}{1.211038in}}%
\pgfpathlineto{\pgfqpoint{2.003396in}{1.230741in}}%
\pgfpathlineto{\pgfqpoint{2.003818in}{1.230741in}}%
\pgfpathlineto{\pgfqpoint{2.004239in}{1.224173in}}%
\pgfpathlineto{\pgfqpoint{2.004660in}{1.237308in}}%
\pgfpathlineto{\pgfqpoint{2.005081in}{1.204471in}}%
\pgfpathlineto{\pgfqpoint{2.005924in}{1.224173in}}%
\pgfpathlineto{\pgfqpoint{2.006345in}{1.217606in}}%
\pgfpathlineto{\pgfqpoint{2.007188in}{1.237308in}}%
\pgfpathlineto{\pgfqpoint{2.007609in}{1.224173in}}%
\pgfpathlineto{\pgfqpoint{2.008030in}{1.243876in}}%
\pgfpathlineto{\pgfqpoint{2.008873in}{1.217606in}}%
\pgfpathlineto{\pgfqpoint{2.009294in}{1.204471in}}%
\pgfpathlineto{\pgfqpoint{2.010558in}{1.237308in}}%
\pgfpathlineto{\pgfqpoint{2.011401in}{1.211038in}}%
\pgfpathlineto{\pgfqpoint{2.011822in}{1.243876in}}%
\pgfpathlineto{\pgfqpoint{2.012664in}{1.230741in}}%
\pgfpathlineto{\pgfqpoint{2.013086in}{1.230741in}}%
\pgfpathlineto{\pgfqpoint{2.013507in}{1.211038in}}%
\pgfpathlineto{\pgfqpoint{2.013928in}{1.217606in}}%
\pgfpathlineto{\pgfqpoint{2.014771in}{1.230741in}}%
\pgfpathlineto{\pgfqpoint{2.015192in}{1.224173in}}%
\pgfpathlineto{\pgfqpoint{2.016035in}{1.230741in}}%
\pgfpathlineto{\pgfqpoint{2.017298in}{1.211038in}}%
\pgfpathlineto{\pgfqpoint{2.018562in}{1.230741in}}%
\pgfpathlineto{\pgfqpoint{2.018984in}{1.211038in}}%
\pgfpathlineto{\pgfqpoint{2.019826in}{1.217606in}}%
\pgfpathlineto{\pgfqpoint{2.020247in}{1.204471in}}%
\pgfpathlineto{\pgfqpoint{2.021090in}{1.211038in}}%
\pgfpathlineto{\pgfqpoint{2.021511in}{1.211038in}}%
\pgfpathlineto{\pgfqpoint{2.021932in}{1.197903in}}%
\pgfpathlineto{\pgfqpoint{2.022354in}{1.217606in}}%
\pgfpathlineto{\pgfqpoint{2.022775in}{1.224173in}}%
\pgfpathlineto{\pgfqpoint{2.024460in}{1.204471in}}%
\pgfpathlineto{\pgfqpoint{2.025724in}{1.217606in}}%
\pgfpathlineto{\pgfqpoint{2.026145in}{1.211038in}}%
\pgfpathlineto{\pgfqpoint{2.026567in}{1.217606in}}%
\pgfpathlineto{\pgfqpoint{2.026988in}{1.230741in}}%
\pgfpathlineto{\pgfqpoint{2.027409in}{1.224173in}}%
\pgfpathlineto{\pgfqpoint{2.027830in}{1.204471in}}%
\pgfpathlineto{\pgfqpoint{2.028673in}{1.211038in}}%
\pgfpathlineto{\pgfqpoint{2.029094in}{1.204471in}}%
\pgfpathlineto{\pgfqpoint{2.029515in}{1.211038in}}%
\pgfpathlineto{\pgfqpoint{2.029937in}{1.217606in}}%
\pgfpathlineto{\pgfqpoint{2.030358in}{1.204471in}}%
\pgfpathlineto{\pgfqpoint{2.030779in}{1.217606in}}%
\pgfpathlineto{\pgfqpoint{2.031201in}{1.230741in}}%
\pgfpathlineto{\pgfqpoint{2.032043in}{1.224173in}}%
\pgfpathlineto{\pgfqpoint{2.032464in}{1.224173in}}%
\pgfpathlineto{\pgfqpoint{2.033307in}{1.230741in}}%
\pgfpathlineto{\pgfqpoint{2.034571in}{1.211038in}}%
\pgfpathlineto{\pgfqpoint{2.035413in}{1.237308in}}%
\pgfpathlineto{\pgfqpoint{2.035835in}{1.230741in}}%
\pgfpathlineto{\pgfqpoint{2.036677in}{1.211038in}}%
\pgfpathlineto{\pgfqpoint{2.037098in}{1.217606in}}%
\pgfpathlineto{\pgfqpoint{2.037941in}{1.230741in}}%
\pgfpathlineto{\pgfqpoint{2.038362in}{1.224173in}}%
\pgfpathlineto{\pgfqpoint{2.038784in}{1.224173in}}%
\pgfpathlineto{\pgfqpoint{2.039205in}{1.230741in}}%
\pgfpathlineto{\pgfqpoint{2.039626in}{1.217606in}}%
\pgfpathlineto{\pgfqpoint{2.040469in}{1.224173in}}%
\pgfpathlineto{\pgfqpoint{2.040890in}{1.224173in}}%
\pgfpathlineto{\pgfqpoint{2.041311in}{1.243876in}}%
\pgfpathlineto{\pgfqpoint{2.041733in}{1.217606in}}%
\pgfpathlineto{\pgfqpoint{2.042154in}{1.237308in}}%
\pgfpathlineto{\pgfqpoint{2.042575in}{1.243876in}}%
\pgfpathlineto{\pgfqpoint{2.043418in}{1.224173in}}%
\pgfpathlineto{\pgfqpoint{2.043839in}{1.230741in}}%
\pgfpathlineto{\pgfqpoint{2.044260in}{1.230741in}}%
\pgfpathlineto{\pgfqpoint{2.045103in}{1.237308in}}%
\pgfpathlineto{\pgfqpoint{2.045524in}{1.217606in}}%
\pgfpathlineto{\pgfqpoint{2.045945in}{1.224173in}}%
\pgfpathlineto{\pgfqpoint{2.046367in}{1.217606in}}%
\pgfpathlineto{\pgfqpoint{2.046788in}{1.217606in}}%
\pgfpathlineto{\pgfqpoint{2.048052in}{1.237308in}}%
\pgfpathlineto{\pgfqpoint{2.048473in}{1.224173in}}%
\pgfpathlineto{\pgfqpoint{2.048894in}{1.243876in}}%
\pgfpathlineto{\pgfqpoint{2.049316in}{1.230741in}}%
\pgfpathlineto{\pgfqpoint{2.050158in}{1.257011in}}%
\pgfpathlineto{\pgfqpoint{2.051001in}{1.217606in}}%
\pgfpathlineto{\pgfqpoint{2.051422in}{1.224173in}}%
\pgfpathlineto{\pgfqpoint{2.051843in}{1.250443in}}%
\pgfpathlineto{\pgfqpoint{2.052264in}{1.217606in}}%
\pgfpathlineto{\pgfqpoint{2.052686in}{1.217606in}}%
\pgfpathlineto{\pgfqpoint{2.053950in}{1.243876in}}%
\pgfpathlineto{\pgfqpoint{2.055635in}{1.197903in}}%
\pgfpathlineto{\pgfqpoint{2.057320in}{1.224173in}}%
\pgfpathlineto{\pgfqpoint{2.057741in}{1.224173in}}%
\pgfpathlineto{\pgfqpoint{2.058162in}{1.217606in}}%
\pgfpathlineto{\pgfqpoint{2.058584in}{1.224173in}}%
\pgfpathlineto{\pgfqpoint{2.059005in}{1.230741in}}%
\pgfpathlineto{\pgfqpoint{2.059426in}{1.217606in}}%
\pgfpathlineto{\pgfqpoint{2.059847in}{1.230741in}}%
\pgfpathlineto{\pgfqpoint{2.060269in}{1.237308in}}%
\pgfpathlineto{\pgfqpoint{2.062375in}{1.204471in}}%
\pgfpathlineto{\pgfqpoint{2.062796in}{1.211038in}}%
\pgfpathlineto{\pgfqpoint{2.063218in}{1.204471in}}%
\pgfpathlineto{\pgfqpoint{2.063639in}{1.204471in}}%
\pgfpathlineto{\pgfqpoint{2.064060in}{1.224173in}}%
\pgfpathlineto{\pgfqpoint{2.064903in}{1.217606in}}%
\pgfpathlineto{\pgfqpoint{2.065324in}{1.197903in}}%
\pgfpathlineto{\pgfqpoint{2.065745in}{1.217606in}}%
\pgfpathlineto{\pgfqpoint{2.066588in}{1.224173in}}%
\pgfpathlineto{\pgfqpoint{2.067430in}{1.204471in}}%
\pgfpathlineto{\pgfqpoint{2.069116in}{1.237308in}}%
\pgfpathlineto{\pgfqpoint{2.069537in}{1.224173in}}%
\pgfpathlineto{\pgfqpoint{2.069958in}{1.230741in}}%
\pgfpathlineto{\pgfqpoint{2.070379in}{1.243876in}}%
\pgfpathlineto{\pgfqpoint{2.071222in}{1.237308in}}%
\pgfpathlineto{\pgfqpoint{2.072065in}{1.224173in}}%
\pgfpathlineto{\pgfqpoint{2.072486in}{1.230741in}}%
\pgfpathlineto{\pgfqpoint{2.072907in}{1.243876in}}%
\pgfpathlineto{\pgfqpoint{2.073750in}{1.217606in}}%
\pgfpathlineto{\pgfqpoint{2.074171in}{1.224173in}}%
\pgfpathlineto{\pgfqpoint{2.075435in}{1.224173in}}%
\pgfpathlineto{\pgfqpoint{2.076277in}{1.217606in}}%
\pgfpathlineto{\pgfqpoint{2.077962in}{1.230741in}}%
\pgfpathlineto{\pgfqpoint{2.078384in}{1.224173in}}%
\pgfpathlineto{\pgfqpoint{2.078805in}{1.237308in}}%
\pgfpathlineto{\pgfqpoint{2.079226in}{1.224173in}}%
\pgfpathlineto{\pgfqpoint{2.079648in}{1.217606in}}%
\pgfpathlineto{\pgfqpoint{2.080069in}{1.224173in}}%
\pgfpathlineto{\pgfqpoint{2.080490in}{1.237308in}}%
\pgfpathlineto{\pgfqpoint{2.080911in}{1.217606in}}%
\pgfpathlineto{\pgfqpoint{2.081333in}{1.224173in}}%
\pgfpathlineto{\pgfqpoint{2.081754in}{1.217606in}}%
\pgfpathlineto{\pgfqpoint{2.082597in}{1.217606in}}%
\pgfpathlineto{\pgfqpoint{2.083018in}{1.224173in}}%
\pgfpathlineto{\pgfqpoint{2.083439in}{1.217606in}}%
\pgfpathlineto{\pgfqpoint{2.083860in}{1.211038in}}%
\pgfpathlineto{\pgfqpoint{2.084282in}{1.217606in}}%
\pgfpathlineto{\pgfqpoint{2.084703in}{1.217606in}}%
\pgfpathlineto{\pgfqpoint{2.085967in}{1.211038in}}%
\pgfpathlineto{\pgfqpoint{2.086388in}{1.211038in}}%
\pgfpathlineto{\pgfqpoint{2.087652in}{1.224173in}}%
\pgfpathlineto{\pgfqpoint{2.088494in}{1.224173in}}%
\pgfpathlineto{\pgfqpoint{2.089758in}{1.230741in}}%
\pgfpathlineto{\pgfqpoint{2.090601in}{1.230741in}}%
\pgfpathlineto{\pgfqpoint{2.091022in}{1.211038in}}%
\pgfpathlineto{\pgfqpoint{2.091443in}{1.224173in}}%
\pgfpathlineto{\pgfqpoint{2.091865in}{1.230741in}}%
\pgfpathlineto{\pgfqpoint{2.092286in}{1.211038in}}%
\pgfpathlineto{\pgfqpoint{2.092707in}{1.230741in}}%
\pgfpathlineto{\pgfqpoint{2.093128in}{1.230741in}}%
\pgfpathlineto{\pgfqpoint{2.093971in}{1.217606in}}%
\pgfpathlineto{\pgfqpoint{2.095656in}{1.237308in}}%
\pgfpathlineto{\pgfqpoint{2.096920in}{1.217606in}}%
\pgfpathlineto{\pgfqpoint{2.097341in}{1.224173in}}%
\pgfpathlineto{\pgfqpoint{2.097763in}{1.224173in}}%
\pgfpathlineto{\pgfqpoint{2.098184in}{1.217606in}}%
\pgfpathlineto{\pgfqpoint{2.099448in}{1.230741in}}%
\pgfpathlineto{\pgfqpoint{2.100290in}{1.217606in}}%
\pgfpathlineto{\pgfqpoint{2.100711in}{1.230741in}}%
\pgfpathlineto{\pgfqpoint{2.101133in}{1.224173in}}%
\pgfpathlineto{\pgfqpoint{2.101975in}{1.211038in}}%
\pgfpathlineto{\pgfqpoint{2.102397in}{1.217606in}}%
\pgfpathlineto{\pgfqpoint{2.102818in}{1.230741in}}%
\pgfpathlineto{\pgfqpoint{2.103239in}{1.224173in}}%
\pgfpathlineto{\pgfqpoint{2.103660in}{1.211038in}}%
\pgfpathlineto{\pgfqpoint{2.104082in}{1.224173in}}%
\pgfpathlineto{\pgfqpoint{2.104503in}{1.237308in}}%
\pgfpathlineto{\pgfqpoint{2.104924in}{1.217606in}}%
\pgfpathlineto{\pgfqpoint{2.105767in}{1.230741in}}%
\pgfpathlineto{\pgfqpoint{2.106188in}{1.224173in}}%
\pgfpathlineto{\pgfqpoint{2.106609in}{1.224173in}}%
\pgfpathlineto{\pgfqpoint{2.107452in}{1.217606in}}%
\pgfpathlineto{\pgfqpoint{2.108294in}{1.224173in}}%
\pgfpathlineto{\pgfqpoint{2.109137in}{1.217606in}}%
\pgfpathlineto{\pgfqpoint{2.109558in}{1.224173in}}%
\pgfpathlineto{\pgfqpoint{2.109980in}{1.211038in}}%
\pgfpathlineto{\pgfqpoint{2.110401in}{1.224173in}}%
\pgfpathlineto{\pgfqpoint{2.112086in}{1.224173in}}%
\pgfpathlineto{\pgfqpoint{2.112507in}{1.204471in}}%
\pgfpathlineto{\pgfqpoint{2.112929in}{1.224173in}}%
\pgfpathlineto{\pgfqpoint{2.113771in}{1.230741in}}%
\pgfpathlineto{\pgfqpoint{2.114614in}{1.217606in}}%
\pgfpathlineto{\pgfqpoint{2.115456in}{1.250443in}}%
\pgfpathlineto{\pgfqpoint{2.115877in}{1.237308in}}%
\pgfpathlineto{\pgfqpoint{2.116720in}{1.230741in}}%
\pgfpathlineto{\pgfqpoint{2.117141in}{1.243876in}}%
\pgfpathlineto{\pgfqpoint{2.117563in}{1.230741in}}%
\pgfpathlineto{\pgfqpoint{2.117984in}{1.230741in}}%
\pgfpathlineto{\pgfqpoint{2.118405in}{1.243876in}}%
\pgfpathlineto{\pgfqpoint{2.118826in}{1.224173in}}%
\pgfpathlineto{\pgfqpoint{2.120090in}{1.224173in}}%
\pgfpathlineto{\pgfqpoint{2.120933in}{1.230741in}}%
\pgfpathlineto{\pgfqpoint{2.121775in}{1.217606in}}%
\pgfpathlineto{\pgfqpoint{2.123460in}{1.237308in}}%
\pgfpathlineto{\pgfqpoint{2.125146in}{1.217606in}}%
\pgfpathlineto{\pgfqpoint{2.125988in}{1.237308in}}%
\pgfpathlineto{\pgfqpoint{2.126409in}{1.217606in}}%
\pgfpathlineto{\pgfqpoint{2.127252in}{1.224173in}}%
\pgfpathlineto{\pgfqpoint{2.127673in}{1.217606in}}%
\pgfpathlineto{\pgfqpoint{2.128095in}{1.230741in}}%
\pgfpathlineto{\pgfqpoint{2.128516in}{1.217606in}}%
\pgfpathlineto{\pgfqpoint{2.128937in}{1.204471in}}%
\pgfpathlineto{\pgfqpoint{2.129358in}{1.217606in}}%
\pgfpathlineto{\pgfqpoint{2.131043in}{1.237308in}}%
\pgfpathlineto{\pgfqpoint{2.131886in}{1.211038in}}%
\pgfpathlineto{\pgfqpoint{2.132307in}{1.230741in}}%
\pgfpathlineto{\pgfqpoint{2.133571in}{1.224173in}}%
\pgfpathlineto{\pgfqpoint{2.134835in}{1.237308in}}%
\pgfpathlineto{\pgfqpoint{2.135256in}{1.230741in}}%
\pgfpathlineto{\pgfqpoint{2.135678in}{1.237308in}}%
\pgfpathlineto{\pgfqpoint{2.136099in}{1.243876in}}%
\pgfpathlineto{\pgfqpoint{2.137784in}{1.217606in}}%
\pgfpathlineto{\pgfqpoint{2.139048in}{1.237308in}}%
\pgfpathlineto{\pgfqpoint{2.139469in}{1.237308in}}%
\pgfpathlineto{\pgfqpoint{2.139890in}{1.243876in}}%
\pgfpathlineto{\pgfqpoint{2.140733in}{1.230741in}}%
\pgfpathlineto{\pgfqpoint{2.141154in}{1.243876in}}%
\pgfpathlineto{\pgfqpoint{2.141575in}{1.230741in}}%
\pgfpathlineto{\pgfqpoint{2.141997in}{1.230741in}}%
\pgfpathlineto{\pgfqpoint{2.142418in}{1.237308in}}%
\pgfpathlineto{\pgfqpoint{2.144524in}{1.204471in}}%
\pgfpathlineto{\pgfqpoint{2.145788in}{1.230741in}}%
\pgfpathlineto{\pgfqpoint{2.146631in}{1.224173in}}%
\pgfpathlineto{\pgfqpoint{2.147052in}{1.243876in}}%
\pgfpathlineto{\pgfqpoint{2.147473in}{1.224173in}}%
\pgfpathlineto{\pgfqpoint{2.148316in}{1.217606in}}%
\pgfpathlineto{\pgfqpoint{2.149580in}{1.224173in}}%
\pgfpathlineto{\pgfqpoint{2.150001in}{1.224173in}}%
\pgfpathlineto{\pgfqpoint{2.150422in}{1.204471in}}%
\pgfpathlineto{\pgfqpoint{2.150844in}{1.224173in}}%
\pgfpathlineto{\pgfqpoint{2.151265in}{1.243876in}}%
\pgfpathlineto{\pgfqpoint{2.151686in}{1.217606in}}%
\pgfpathlineto{\pgfqpoint{2.152529in}{1.211038in}}%
\pgfpathlineto{\pgfqpoint{2.152950in}{1.237308in}}%
\pgfpathlineto{\pgfqpoint{2.154635in}{1.224173in}}%
\pgfpathlineto{\pgfqpoint{2.155056in}{1.230741in}}%
\pgfpathlineto{\pgfqpoint{2.155478in}{1.224173in}}%
\pgfpathlineto{\pgfqpoint{2.155899in}{1.224173in}}%
\pgfpathlineto{\pgfqpoint{2.156320in}{1.237308in}}%
\pgfpathlineto{\pgfqpoint{2.156741in}{1.230741in}}%
\pgfpathlineto{\pgfqpoint{2.157163in}{1.224173in}}%
\pgfpathlineto{\pgfqpoint{2.157584in}{1.237308in}}%
\pgfpathlineto{\pgfqpoint{2.158005in}{1.224173in}}%
\pgfpathlineto{\pgfqpoint{2.158848in}{1.224173in}}%
\pgfpathlineto{\pgfqpoint{2.159690in}{1.237308in}}%
\pgfpathlineto{\pgfqpoint{2.160112in}{1.230741in}}%
\pgfpathlineto{\pgfqpoint{2.160533in}{1.211038in}}%
\pgfpathlineto{\pgfqpoint{2.160954in}{1.217606in}}%
\pgfpathlineto{\pgfqpoint{2.162639in}{1.230741in}}%
\pgfpathlineto{\pgfqpoint{2.163061in}{1.204471in}}%
\pgfpathlineto{\pgfqpoint{2.163482in}{1.230741in}}%
\pgfpathlineto{\pgfqpoint{2.163903in}{1.243876in}}%
\pgfpathlineto{\pgfqpoint{2.165167in}{1.211038in}}%
\pgfpathlineto{\pgfqpoint{2.165588in}{1.211038in}}%
\pgfpathlineto{\pgfqpoint{2.166010in}{1.237308in}}%
\pgfpathlineto{\pgfqpoint{2.166431in}{1.211038in}}%
\pgfpathlineto{\pgfqpoint{2.167695in}{1.224173in}}%
\pgfpathlineto{\pgfqpoint{2.168537in}{1.217606in}}%
\pgfpathlineto{\pgfqpoint{2.169380in}{1.224173in}}%
\pgfpathlineto{\pgfqpoint{2.170222in}{1.211038in}}%
\pgfpathlineto{\pgfqpoint{2.170644in}{1.217606in}}%
\pgfpathlineto{\pgfqpoint{2.171486in}{1.243876in}}%
\pgfpathlineto{\pgfqpoint{2.171907in}{1.230741in}}%
\pgfpathlineto{\pgfqpoint{2.172329in}{1.230741in}}%
\pgfpathlineto{\pgfqpoint{2.173171in}{1.217606in}}%
\pgfpathlineto{\pgfqpoint{2.173593in}{1.224173in}}%
\pgfpathlineto{\pgfqpoint{2.174014in}{1.217606in}}%
\pgfpathlineto{\pgfqpoint{2.174856in}{1.237308in}}%
\pgfpathlineto{\pgfqpoint{2.175278in}{1.230741in}}%
\pgfpathlineto{\pgfqpoint{2.176120in}{1.211038in}}%
\pgfpathlineto{\pgfqpoint{2.176542in}{1.230741in}}%
\pgfpathlineto{\pgfqpoint{2.176963in}{1.211038in}}%
\pgfpathlineto{\pgfqpoint{2.177384in}{1.211038in}}%
\pgfpathlineto{\pgfqpoint{2.179069in}{1.230741in}}%
\pgfpathlineto{\pgfqpoint{2.180333in}{1.211038in}}%
\pgfpathlineto{\pgfqpoint{2.180754in}{1.230741in}}%
\pgfpathlineto{\pgfqpoint{2.181176in}{1.217606in}}%
\pgfpathlineto{\pgfqpoint{2.181597in}{1.217606in}}%
\pgfpathlineto{\pgfqpoint{2.182018in}{1.211038in}}%
\pgfpathlineto{\pgfqpoint{2.182439in}{1.217606in}}%
\pgfpathlineto{\pgfqpoint{2.182861in}{1.217606in}}%
\pgfpathlineto{\pgfqpoint{2.183282in}{1.211038in}}%
\pgfpathlineto{\pgfqpoint{2.183703in}{1.217606in}}%
\pgfpathlineto{\pgfqpoint{2.184125in}{1.217606in}}%
\pgfpathlineto{\pgfqpoint{2.185388in}{1.204471in}}%
\pgfpathlineto{\pgfqpoint{2.186652in}{1.217606in}}%
\pgfpathlineto{\pgfqpoint{2.187073in}{1.211038in}}%
\pgfpathlineto{\pgfqpoint{2.187495in}{1.224173in}}%
\pgfpathlineto{\pgfqpoint{2.187916in}{1.217606in}}%
\pgfpathlineto{\pgfqpoint{2.188337in}{1.197903in}}%
\pgfpathlineto{\pgfqpoint{2.188759in}{1.217606in}}%
\pgfpathlineto{\pgfqpoint{2.190022in}{1.224173in}}%
\pgfpathlineto{\pgfqpoint{2.190444in}{1.217606in}}%
\pgfpathlineto{\pgfqpoint{2.190865in}{1.224173in}}%
\pgfpathlineto{\pgfqpoint{2.191286in}{1.224173in}}%
\pgfpathlineto{\pgfqpoint{2.191708in}{1.230741in}}%
\pgfpathlineto{\pgfqpoint{2.192129in}{1.217606in}}%
\pgfpathlineto{\pgfqpoint{2.192550in}{0.915498in}}%
\pgfpathlineto{\pgfqpoint{2.192971in}{1.224173in}}%
\pgfpathlineto{\pgfqpoint{2.193393in}{1.217606in}}%
\pgfpathlineto{\pgfqpoint{2.193814in}{1.230741in}}%
\pgfpathlineto{\pgfqpoint{2.194235in}{1.217606in}}%
\pgfpathlineto{\pgfqpoint{2.194656in}{0.941769in}}%
\pgfpathlineto{\pgfqpoint{2.195078in}{1.224173in}}%
\pgfpathlineto{\pgfqpoint{2.195499in}{1.230741in}}%
\pgfpathlineto{\pgfqpoint{2.195920in}{1.224173in}}%
\pgfpathlineto{\pgfqpoint{2.196342in}{1.224173in}}%
\pgfpathlineto{\pgfqpoint{2.196763in}{0.981174in}}%
\pgfpathlineto{\pgfqpoint{2.197184in}{1.230741in}}%
\pgfpathlineto{\pgfqpoint{2.198027in}{1.230741in}}%
\pgfpathlineto{\pgfqpoint{2.198448in}{1.224173in}}%
\pgfpathlineto{\pgfqpoint{2.198869in}{0.994309in}}%
\pgfpathlineto{\pgfqpoint{2.199291in}{1.224173in}}%
\pgfpathlineto{\pgfqpoint{2.199712in}{1.230741in}}%
\pgfpathlineto{\pgfqpoint{2.200133in}{1.224173in}}%
\pgfpathlineto{\pgfqpoint{2.200554in}{1.224173in}}%
\pgfpathlineto{\pgfqpoint{2.200976in}{1.014012in}}%
\pgfpathlineto{\pgfqpoint{2.201397in}{1.224173in}}%
\pgfpathlineto{\pgfqpoint{2.201818in}{1.230741in}}%
\pgfpathlineto{\pgfqpoint{2.202240in}{1.224173in}}%
\pgfpathlineto{\pgfqpoint{2.202661in}{1.224173in}}%
\pgfpathlineto{\pgfqpoint{2.203082in}{1.046849in}}%
\pgfpathlineto{\pgfqpoint{2.203503in}{1.217606in}}%
\pgfpathlineto{\pgfqpoint{2.204767in}{1.243876in}}%
\pgfpathlineto{\pgfqpoint{2.205188in}{1.237308in}}%
\pgfpathlineto{\pgfqpoint{2.205610in}{1.237308in}}%
\pgfpathlineto{\pgfqpoint{2.206031in}{1.532848in}}%
\pgfpathlineto{\pgfqpoint{2.206452in}{1.237308in}}%
\pgfpathlineto{\pgfqpoint{2.206874in}{1.243876in}}%
\pgfpathlineto{\pgfqpoint{2.207295in}{1.191335in}}%
\pgfpathlineto{\pgfqpoint{2.207716in}{1.243876in}}%
\pgfpathlineto{\pgfqpoint{2.208137in}{1.513145in}}%
\pgfpathlineto{\pgfqpoint{2.208559in}{1.151930in}}%
\pgfpathlineto{\pgfqpoint{2.209401in}{1.099390in}}%
\pgfpathlineto{\pgfqpoint{2.210244in}{1.480308in}}%
\pgfpathlineto{\pgfqpoint{2.211508in}{1.119092in}}%
\pgfpathlineto{\pgfqpoint{2.212350in}{1.467172in}}%
\pgfpathlineto{\pgfqpoint{2.212771in}{1.230741in}}%
\pgfpathlineto{\pgfqpoint{2.213193in}{1.230741in}}%
\pgfpathlineto{\pgfqpoint{2.213614in}{1.125660in}}%
\pgfpathlineto{\pgfqpoint{2.214035in}{1.237308in}}%
\pgfpathlineto{\pgfqpoint{2.214457in}{1.440902in}}%
\pgfpathlineto{\pgfqpoint{2.214878in}{1.230741in}}%
\pgfpathlineto{\pgfqpoint{2.215299in}{1.224173in}}%
\pgfpathlineto{\pgfqpoint{2.215720in}{1.178200in}}%
\pgfpathlineto{\pgfqpoint{2.216142in}{1.243876in}}%
\pgfpathlineto{\pgfqpoint{2.216563in}{1.408065in}}%
\pgfpathlineto{\pgfqpoint{2.216984in}{1.230741in}}%
\pgfpathlineto{\pgfqpoint{2.217406in}{1.224173in}}%
\pgfpathlineto{\pgfqpoint{2.217827in}{1.178200in}}%
\pgfpathlineto{\pgfqpoint{2.218248in}{1.204471in}}%
\pgfpathlineto{\pgfqpoint{2.218669in}{1.224173in}}%
\pgfpathlineto{\pgfqpoint{2.219512in}{1.217606in}}%
\pgfpathlineto{\pgfqpoint{2.220776in}{1.243876in}}%
\pgfpathlineto{\pgfqpoint{2.221197in}{1.211038in}}%
\pgfpathlineto{\pgfqpoint{2.221618in}{1.224173in}}%
\pgfpathlineto{\pgfqpoint{2.222882in}{1.348957in}}%
\pgfpathlineto{\pgfqpoint{2.223725in}{1.217606in}}%
\pgfpathlineto{\pgfqpoint{2.224146in}{1.237308in}}%
\pgfpathlineto{\pgfqpoint{2.224567in}{1.230741in}}%
\pgfpathlineto{\pgfqpoint{2.224989in}{1.329254in}}%
\pgfpathlineto{\pgfqpoint{2.225410in}{1.217606in}}%
\pgfpathlineto{\pgfqpoint{2.225831in}{1.224173in}}%
\pgfpathlineto{\pgfqpoint{2.226674in}{1.217606in}}%
\pgfpathlineto{\pgfqpoint{2.227095in}{1.329254in}}%
\pgfpathlineto{\pgfqpoint{2.227516in}{1.211038in}}%
\pgfpathlineto{\pgfqpoint{2.228359in}{1.217606in}}%
\pgfpathlineto{\pgfqpoint{2.228780in}{1.224173in}}%
\pgfpathlineto{\pgfqpoint{2.229201in}{1.270146in}}%
\pgfpathlineto{\pgfqpoint{2.229623in}{1.211038in}}%
\pgfpathlineto{\pgfqpoint{2.231308in}{1.263578in}}%
\pgfpathlineto{\pgfqpoint{2.232150in}{1.217606in}}%
\pgfpathlineto{\pgfqpoint{2.232572in}{1.224173in}}%
\pgfpathlineto{\pgfqpoint{2.232993in}{1.224173in}}%
\pgfpathlineto{\pgfqpoint{2.233835in}{1.217606in}}%
\pgfpathlineto{\pgfqpoint{2.234257in}{1.230741in}}%
\pgfpathlineto{\pgfqpoint{2.235099in}{1.224173in}}%
\pgfpathlineto{\pgfqpoint{2.235520in}{1.191335in}}%
\pgfpathlineto{\pgfqpoint{2.235942in}{1.217606in}}%
\pgfpathlineto{\pgfqpoint{2.237206in}{1.230741in}}%
\pgfpathlineto{\pgfqpoint{2.237627in}{1.211038in}}%
\pgfpathlineto{\pgfqpoint{2.238469in}{1.217606in}}%
\pgfpathlineto{\pgfqpoint{2.238891in}{1.217606in}}%
\pgfpathlineto{\pgfqpoint{2.239312in}{1.224173in}}%
\pgfpathlineto{\pgfqpoint{2.239733in}{1.151930in}}%
\pgfpathlineto{\pgfqpoint{2.240155in}{1.224173in}}%
\pgfpathlineto{\pgfqpoint{2.240576in}{1.217606in}}%
\pgfpathlineto{\pgfqpoint{2.240997in}{1.224173in}}%
\pgfpathlineto{\pgfqpoint{2.242261in}{1.230741in}}%
\pgfpathlineto{\pgfqpoint{2.242682in}{1.224173in}}%
\pgfpathlineto{\pgfqpoint{2.243103in}{1.230741in}}%
\pgfpathlineto{\pgfqpoint{2.243525in}{1.230741in}}%
\pgfpathlineto{\pgfqpoint{2.244367in}{1.211038in}}%
\pgfpathlineto{\pgfqpoint{2.244789in}{1.224173in}}%
\pgfpathlineto{\pgfqpoint{2.245210in}{1.237308in}}%
\pgfpathlineto{\pgfqpoint{2.246052in}{1.230741in}}%
\pgfpathlineto{\pgfqpoint{2.247316in}{1.224173in}}%
\pgfpathlineto{\pgfqpoint{2.247738in}{1.224173in}}%
\pgfpathlineto{\pgfqpoint{2.248159in}{1.217606in}}%
\pgfpathlineto{\pgfqpoint{2.248580in}{1.224173in}}%
\pgfpathlineto{\pgfqpoint{2.249001in}{1.224173in}}%
\pgfpathlineto{\pgfqpoint{2.249423in}{1.217606in}}%
\pgfpathlineto{\pgfqpoint{2.249844in}{1.237308in}}%
\pgfpathlineto{\pgfqpoint{2.250265in}{1.230741in}}%
\pgfpathlineto{\pgfqpoint{2.251108in}{1.217606in}}%
\pgfpathlineto{\pgfqpoint{2.251529in}{1.224173in}}%
\pgfpathlineto{\pgfqpoint{2.251950in}{1.237308in}}%
\pgfpathlineto{\pgfqpoint{2.252793in}{1.230741in}}%
\pgfpathlineto{\pgfqpoint{2.253635in}{1.230741in}}%
\pgfpathlineto{\pgfqpoint{2.254899in}{1.211038in}}%
\pgfpathlineto{\pgfqpoint{2.256163in}{1.224173in}}%
\pgfpathlineto{\pgfqpoint{2.256584in}{1.224173in}}%
\pgfpathlineto{\pgfqpoint{2.257006in}{1.217606in}}%
\pgfpathlineto{\pgfqpoint{2.257427in}{1.237308in}}%
\pgfpathlineto{\pgfqpoint{2.257848in}{1.230741in}}%
\pgfpathlineto{\pgfqpoint{2.258691in}{1.211038in}}%
\pgfpathlineto{\pgfqpoint{2.259533in}{1.217606in}}%
\pgfpathlineto{\pgfqpoint{2.260376in}{1.224173in}}%
\pgfpathlineto{\pgfqpoint{2.260797in}{1.217606in}}%
\pgfpathlineto{\pgfqpoint{2.261218in}{1.224173in}}%
\pgfpathlineto{\pgfqpoint{2.262061in}{1.224173in}}%
\pgfpathlineto{\pgfqpoint{2.262482in}{1.237308in}}%
\pgfpathlineto{\pgfqpoint{2.262904in}{1.224173in}}%
\pgfpathlineto{\pgfqpoint{2.264167in}{1.217606in}}%
\pgfpathlineto{\pgfqpoint{2.264589in}{1.224173in}}%
\pgfpathlineto{\pgfqpoint{2.265010in}{1.217606in}}%
\pgfpathlineto{\pgfqpoint{2.265431in}{1.211038in}}%
\pgfpathlineto{\pgfqpoint{2.266695in}{1.230741in}}%
\pgfpathlineto{\pgfqpoint{2.267116in}{1.224173in}}%
\pgfpathlineto{\pgfqpoint{2.267538in}{1.230741in}}%
\pgfpathlineto{\pgfqpoint{2.268801in}{1.230741in}}%
\pgfpathlineto{\pgfqpoint{2.269223in}{1.211038in}}%
\pgfpathlineto{\pgfqpoint{2.270065in}{1.217606in}}%
\pgfpathlineto{\pgfqpoint{2.270908in}{1.230741in}}%
\pgfpathlineto{\pgfqpoint{2.271750in}{1.217606in}}%
\pgfpathlineto{\pgfqpoint{2.272172in}{1.224173in}}%
\pgfpathlineto{\pgfqpoint{2.272593in}{1.237308in}}%
\pgfpathlineto{\pgfqpoint{2.273014in}{1.224173in}}%
\pgfpathlineto{\pgfqpoint{2.273436in}{1.224173in}}%
\pgfpathlineto{\pgfqpoint{2.273857in}{1.211038in}}%
\pgfpathlineto{\pgfqpoint{2.274278in}{1.224173in}}%
\pgfpathlineto{\pgfqpoint{2.274699in}{1.237308in}}%
\pgfpathlineto{\pgfqpoint{2.275121in}{1.230741in}}%
\pgfpathlineto{\pgfqpoint{2.275963in}{1.211038in}}%
\pgfpathlineto{\pgfqpoint{2.277227in}{1.237308in}}%
\pgfpathlineto{\pgfqpoint{2.278070in}{1.224173in}}%
\pgfpathlineto{\pgfqpoint{2.278491in}{1.230741in}}%
\pgfpathlineto{\pgfqpoint{2.278912in}{1.230741in}}%
\pgfpathlineto{\pgfqpoint{2.280176in}{1.217606in}}%
\pgfpathlineto{\pgfqpoint{2.280597in}{1.217606in}}%
\pgfpathlineto{\pgfqpoint{2.281019in}{1.230741in}}%
\pgfpathlineto{\pgfqpoint{2.281861in}{1.224173in}}%
\pgfpathlineto{\pgfqpoint{2.282282in}{1.217606in}}%
\pgfpathlineto{\pgfqpoint{2.282704in}{1.237308in}}%
\pgfpathlineto{\pgfqpoint{2.283546in}{1.230741in}}%
\pgfpathlineto{\pgfqpoint{2.284389in}{1.211038in}}%
\pgfpathlineto{\pgfqpoint{2.284810in}{1.224173in}}%
\pgfpathlineto{\pgfqpoint{2.285653in}{1.230741in}}%
\pgfpathlineto{\pgfqpoint{2.286074in}{1.211038in}}%
\pgfpathlineto{\pgfqpoint{2.286495in}{1.230741in}}%
\pgfpathlineto{\pgfqpoint{2.286916in}{1.230741in}}%
\pgfpathlineto{\pgfqpoint{2.287338in}{1.237308in}}%
\pgfpathlineto{\pgfqpoint{2.287759in}{1.230741in}}%
\pgfpathlineto{\pgfqpoint{2.288180in}{1.217606in}}%
\pgfpathlineto{\pgfqpoint{2.289023in}{1.224173in}}%
\pgfpathlineto{\pgfqpoint{2.289444in}{1.224173in}}%
\pgfpathlineto{\pgfqpoint{2.289865in}{1.237308in}}%
\pgfpathlineto{\pgfqpoint{2.290287in}{1.230741in}}%
\pgfpathlineto{\pgfqpoint{2.290708in}{1.217606in}}%
\pgfpathlineto{\pgfqpoint{2.291129in}{1.230741in}}%
\pgfpathlineto{\pgfqpoint{2.291972in}{1.237308in}}%
\pgfpathlineto{\pgfqpoint{2.293657in}{1.224173in}}%
\pgfpathlineto{\pgfqpoint{2.294078in}{1.237308in}}%
\pgfpathlineto{\pgfqpoint{2.294499in}{1.217606in}}%
\pgfpathlineto{\pgfqpoint{2.294921in}{1.230741in}}%
\pgfpathlineto{\pgfqpoint{2.295342in}{1.224173in}}%
\pgfpathlineto{\pgfqpoint{2.295763in}{1.230741in}}%
\pgfpathlineto{\pgfqpoint{2.296185in}{1.230741in}}%
\pgfpathlineto{\pgfqpoint{2.297448in}{1.224173in}}%
\pgfpathlineto{\pgfqpoint{2.298291in}{1.237308in}}%
\pgfpathlineto{\pgfqpoint{2.299555in}{1.224173in}}%
\pgfpathlineto{\pgfqpoint{2.300397in}{1.230741in}}%
\pgfpathlineto{\pgfqpoint{2.301661in}{1.224173in}}%
\pgfpathlineto{\pgfqpoint{2.302925in}{1.230741in}}%
\pgfpathlineto{\pgfqpoint{2.303346in}{1.217606in}}%
\pgfpathlineto{\pgfqpoint{2.303768in}{1.224173in}}%
\pgfpathlineto{\pgfqpoint{2.304189in}{1.230741in}}%
\pgfpathlineto{\pgfqpoint{2.304610in}{1.211038in}}%
\pgfpathlineto{\pgfqpoint{2.305453in}{1.217606in}}%
\pgfpathlineto{\pgfqpoint{2.306716in}{1.230741in}}%
\pgfpathlineto{\pgfqpoint{2.307559in}{1.217606in}}%
\pgfpathlineto{\pgfqpoint{2.308823in}{1.237308in}}%
\pgfpathlineto{\pgfqpoint{2.309665in}{1.211038in}}%
\pgfpathlineto{\pgfqpoint{2.310087in}{1.224173in}}%
\pgfpathlineto{\pgfqpoint{2.310508in}{1.224173in}}%
\pgfpathlineto{\pgfqpoint{2.310929in}{1.230741in}}%
\pgfpathlineto{\pgfqpoint{2.311351in}{1.211038in}}%
\pgfpathlineto{\pgfqpoint{2.312193in}{1.217606in}}%
\pgfpathlineto{\pgfqpoint{2.312614in}{1.217606in}}%
\pgfpathlineto{\pgfqpoint{2.313036in}{1.230741in}}%
\pgfpathlineto{\pgfqpoint{2.313457in}{1.217606in}}%
\pgfpathlineto{\pgfqpoint{2.314721in}{1.211038in}}%
\pgfpathlineto{\pgfqpoint{2.315142in}{1.224173in}}%
\pgfpathlineto{\pgfqpoint{2.315563in}{1.217606in}}%
\pgfpathlineto{\pgfqpoint{2.315985in}{1.211038in}}%
\pgfpathlineto{\pgfqpoint{2.316406in}{1.217606in}}%
\pgfpathlineto{\pgfqpoint{2.316827in}{1.217606in}}%
\pgfpathlineto{\pgfqpoint{2.318091in}{1.224173in}}%
\pgfpathlineto{\pgfqpoint{2.318512in}{1.224173in}}%
\pgfpathlineto{\pgfqpoint{2.318934in}{1.250443in}}%
\pgfpathlineto{\pgfqpoint{2.319355in}{1.230741in}}%
\pgfpathlineto{\pgfqpoint{2.319776in}{1.230741in}}%
\pgfpathlineto{\pgfqpoint{2.320197in}{1.217606in}}%
\pgfpathlineto{\pgfqpoint{2.320619in}{1.230741in}}%
\pgfpathlineto{\pgfqpoint{2.321040in}{1.237308in}}%
\pgfpathlineto{\pgfqpoint{2.322725in}{1.211038in}}%
\pgfpathlineto{\pgfqpoint{2.323568in}{1.237308in}}%
\pgfpathlineto{\pgfqpoint{2.323989in}{1.230741in}}%
\pgfpathlineto{\pgfqpoint{2.324410in}{1.217606in}}%
\pgfpathlineto{\pgfqpoint{2.324831in}{1.257011in}}%
\pgfpathlineto{\pgfqpoint{2.325674in}{1.237308in}}%
\pgfpathlineto{\pgfqpoint{2.326095in}{1.224173in}}%
\pgfpathlineto{\pgfqpoint{2.326517in}{1.230741in}}%
\pgfpathlineto{\pgfqpoint{2.326938in}{1.243876in}}%
\pgfpathlineto{\pgfqpoint{2.327359in}{1.237308in}}%
\pgfpathlineto{\pgfqpoint{2.327780in}{1.230741in}}%
\pgfpathlineto{\pgfqpoint{2.328202in}{1.250443in}}%
\pgfpathlineto{\pgfqpoint{2.328623in}{1.237308in}}%
\pgfpathlineto{\pgfqpoint{2.329887in}{1.237308in}}%
\pgfpathlineto{\pgfqpoint{2.330308in}{1.224173in}}%
\pgfpathlineto{\pgfqpoint{2.330729in}{1.230741in}}%
\pgfpathlineto{\pgfqpoint{2.331572in}{1.237308in}}%
\pgfpathlineto{\pgfqpoint{2.331993in}{1.230741in}}%
\pgfpathlineto{\pgfqpoint{2.332414in}{1.204471in}}%
\pgfpathlineto{\pgfqpoint{2.333257in}{1.217606in}}%
\pgfpathlineto{\pgfqpoint{2.334521in}{1.224173in}}%
\pgfpathlineto{\pgfqpoint{2.334942in}{1.224173in}}%
\pgfpathlineto{\pgfqpoint{2.336206in}{1.243876in}}%
\pgfpathlineto{\pgfqpoint{2.337470in}{1.224173in}}%
\pgfpathlineto{\pgfqpoint{2.337891in}{1.230741in}}%
\pgfpathlineto{\pgfqpoint{2.338312in}{1.204471in}}%
\pgfpathlineto{\pgfqpoint{2.338734in}{1.217606in}}%
\pgfpathlineto{\pgfqpoint{2.339576in}{1.237308in}}%
\pgfpathlineto{\pgfqpoint{2.339997in}{1.204471in}}%
\pgfpathlineto{\pgfqpoint{2.340840in}{1.217606in}}%
\pgfpathlineto{\pgfqpoint{2.341261in}{1.230741in}}%
\pgfpathlineto{\pgfqpoint{2.341683in}{1.211038in}}%
\pgfpathlineto{\pgfqpoint{2.343368in}{1.224173in}}%
\pgfpathlineto{\pgfqpoint{2.344632in}{1.224173in}}%
\pgfpathlineto{\pgfqpoint{2.345053in}{1.211038in}}%
\pgfpathlineto{\pgfqpoint{2.345474in}{1.224173in}}%
\pgfpathlineto{\pgfqpoint{2.346317in}{1.224173in}}%
\pgfpathlineto{\pgfqpoint{2.347159in}{1.230741in}}%
\pgfpathlineto{\pgfqpoint{2.348844in}{1.211038in}}%
\pgfpathlineto{\pgfqpoint{2.350108in}{1.230741in}}%
\pgfpathlineto{\pgfqpoint{2.350951in}{1.224173in}}%
\pgfpathlineto{\pgfqpoint{2.352215in}{1.230741in}}%
\pgfpathlineto{\pgfqpoint{2.353057in}{1.211038in}}%
\pgfpathlineto{\pgfqpoint{2.353478in}{1.224173in}}%
\pgfpathlineto{\pgfqpoint{2.353900in}{1.224173in}}%
\pgfpathlineto{\pgfqpoint{2.354321in}{1.243876in}}%
\pgfpathlineto{\pgfqpoint{2.354742in}{1.217606in}}%
\pgfpathlineto{\pgfqpoint{2.356006in}{1.224173in}}%
\pgfpathlineto{\pgfqpoint{2.356427in}{1.224173in}}%
\pgfpathlineto{\pgfqpoint{2.356849in}{1.217606in}}%
\pgfpathlineto{\pgfqpoint{2.357691in}{1.230741in}}%
\pgfpathlineto{\pgfqpoint{2.358112in}{1.224173in}}%
\pgfpathlineto{\pgfqpoint{2.358534in}{1.230741in}}%
\pgfpathlineto{\pgfqpoint{2.358955in}{1.224173in}}%
\pgfpathlineto{\pgfqpoint{2.359376in}{1.224173in}}%
\pgfpathlineto{\pgfqpoint{2.359798in}{1.237308in}}%
\pgfpathlineto{\pgfqpoint{2.360640in}{1.230741in}}%
\pgfpathlineto{\pgfqpoint{2.361904in}{1.224173in}}%
\pgfpathlineto{\pgfqpoint{2.363168in}{1.230741in}}%
\pgfpathlineto{\pgfqpoint{2.363589in}{1.217606in}}%
\pgfpathlineto{\pgfqpoint{2.364432in}{1.224173in}}%
\pgfpathlineto{\pgfqpoint{2.365695in}{1.217606in}}%
\pgfpathlineto{\pgfqpoint{2.366959in}{1.237308in}}%
\pgfpathlineto{\pgfqpoint{2.367802in}{1.217606in}}%
\pgfpathlineto{\pgfqpoint{2.368223in}{1.224173in}}%
\pgfpathlineto{\pgfqpoint{2.368644in}{1.237308in}}%
\pgfpathlineto{\pgfqpoint{2.369066in}{1.224173in}}%
\pgfpathlineto{\pgfqpoint{2.369487in}{1.217606in}}%
\pgfpathlineto{\pgfqpoint{2.369908in}{1.230741in}}%
\pgfpathlineto{\pgfqpoint{2.370751in}{1.224173in}}%
\pgfpathlineto{\pgfqpoint{2.371172in}{1.224173in}}%
\pgfpathlineto{\pgfqpoint{2.371593in}{1.211038in}}%
\pgfpathlineto{\pgfqpoint{2.372015in}{1.224173in}}%
\pgfpathlineto{\pgfqpoint{2.372436in}{1.224173in}}%
\pgfpathlineto{\pgfqpoint{2.373278in}{1.204471in}}%
\pgfpathlineto{\pgfqpoint{2.374121in}{1.211038in}}%
\pgfpathlineto{\pgfqpoint{2.374542in}{1.211038in}}%
\pgfpathlineto{\pgfqpoint{2.374964in}{1.204471in}}%
\pgfpathlineto{\pgfqpoint{2.375806in}{1.217606in}}%
\pgfpathlineto{\pgfqpoint{2.376227in}{1.211038in}}%
\pgfpathlineto{\pgfqpoint{2.376649in}{1.211038in}}%
\pgfpathlineto{\pgfqpoint{2.377070in}{1.217606in}}%
\pgfpathlineto{\pgfqpoint{2.377491in}{1.204471in}}%
\pgfpathlineto{\pgfqpoint{2.378334in}{1.211038in}}%
\pgfpathlineto{\pgfqpoint{2.378755in}{1.211038in}}%
\pgfpathlineto{\pgfqpoint{2.379176in}{1.217606in}}%
\pgfpathlineto{\pgfqpoint{2.379598in}{1.204471in}}%
\pgfpathlineto{\pgfqpoint{2.380019in}{1.237308in}}%
\pgfpathlineto{\pgfqpoint{2.380861in}{1.217606in}}%
\pgfpathlineto{\pgfqpoint{2.381283in}{1.204471in}}%
\pgfpathlineto{\pgfqpoint{2.381704in}{1.217606in}}%
\pgfpathlineto{\pgfqpoint{2.382968in}{1.217606in}}%
\pgfpathlineto{\pgfqpoint{2.383389in}{1.204471in}}%
\pgfpathlineto{\pgfqpoint{2.383810in}{1.230741in}}%
\pgfpathlineto{\pgfqpoint{2.384232in}{1.217606in}}%
\pgfpathlineto{\pgfqpoint{2.384653in}{1.204471in}}%
\pgfpathlineto{\pgfqpoint{2.385074in}{1.224173in}}%
\pgfpathlineto{\pgfqpoint{2.385917in}{1.211038in}}%
\pgfpathlineto{\pgfqpoint{2.386338in}{1.217606in}}%
\pgfpathlineto{\pgfqpoint{2.386759in}{1.237308in}}%
\pgfpathlineto{\pgfqpoint{2.387181in}{1.217606in}}%
\pgfpathlineto{\pgfqpoint{2.387602in}{1.217606in}}%
\pgfpathlineto{\pgfqpoint{2.388444in}{1.237308in}}%
\pgfpathlineto{\pgfqpoint{2.388866in}{1.224173in}}%
\pgfpathlineto{\pgfqpoint{2.389287in}{1.217606in}}%
\pgfpathlineto{\pgfqpoint{2.389708in}{1.243876in}}%
\pgfpathlineto{\pgfqpoint{2.390551in}{1.230741in}}%
\pgfpathlineto{\pgfqpoint{2.390972in}{1.237308in}}%
\pgfpathlineto{\pgfqpoint{2.391815in}{1.224173in}}%
\pgfpathlineto{\pgfqpoint{2.392236in}{1.237308in}}%
\pgfpathlineto{\pgfqpoint{2.392657in}{1.217606in}}%
\pgfpathlineto{\pgfqpoint{2.393078in}{1.230741in}}%
\pgfpathlineto{\pgfqpoint{2.394342in}{1.224173in}}%
\pgfpathlineto{\pgfqpoint{2.395185in}{1.237308in}}%
\pgfpathlineto{\pgfqpoint{2.395606in}{1.217606in}}%
\pgfpathlineto{\pgfqpoint{2.396449in}{1.224173in}}%
\pgfpathlineto{\pgfqpoint{2.397713in}{1.237308in}}%
\pgfpathlineto{\pgfqpoint{2.398555in}{1.224173in}}%
\pgfpathlineto{\pgfqpoint{2.399819in}{1.243876in}}%
\pgfpathlineto{\pgfqpoint{2.400240in}{1.224173in}}%
\pgfpathlineto{\pgfqpoint{2.400662in}{1.237308in}}%
\pgfpathlineto{\pgfqpoint{2.401504in}{1.237308in}}%
\pgfpathlineto{\pgfqpoint{2.401925in}{1.217606in}}%
\pgfpathlineto{\pgfqpoint{2.402347in}{1.237308in}}%
\pgfpathlineto{\pgfqpoint{2.404453in}{1.211038in}}%
\pgfpathlineto{\pgfqpoint{2.405296in}{1.224173in}}%
\pgfpathlineto{\pgfqpoint{2.405717in}{1.211038in}}%
\pgfpathlineto{\pgfqpoint{2.406138in}{1.224173in}}%
\pgfpathlineto{\pgfqpoint{2.406559in}{1.237308in}}%
\pgfpathlineto{\pgfqpoint{2.406981in}{1.211038in}}%
\pgfpathlineto{\pgfqpoint{2.407402in}{1.217606in}}%
\pgfpathlineto{\pgfqpoint{2.407823in}{1.237308in}}%
\pgfpathlineto{\pgfqpoint{2.408245in}{1.211038in}}%
\pgfpathlineto{\pgfqpoint{2.409930in}{1.224173in}}%
\pgfpathlineto{\pgfqpoint{2.411193in}{1.204471in}}%
\pgfpathlineto{\pgfqpoint{2.412036in}{1.224173in}}%
\pgfpathlineto{\pgfqpoint{2.412457in}{1.217606in}}%
\pgfpathlineto{\pgfqpoint{2.412879in}{1.204471in}}%
\pgfpathlineto{\pgfqpoint{2.413721in}{1.211038in}}%
\pgfpathlineto{\pgfqpoint{2.415406in}{1.211038in}}%
\pgfpathlineto{\pgfqpoint{2.415828in}{1.204471in}}%
\pgfpathlineto{\pgfqpoint{2.416249in}{1.217606in}}%
\pgfpathlineto{\pgfqpoint{2.417091in}{1.211038in}}%
\pgfpathlineto{\pgfqpoint{2.417513in}{1.211038in}}%
\pgfpathlineto{\pgfqpoint{2.418776in}{1.224173in}}%
\pgfpathlineto{\pgfqpoint{2.420040in}{1.204471in}}%
\pgfpathlineto{\pgfqpoint{2.420462in}{1.217606in}}%
\pgfpathlineto{\pgfqpoint{2.420883in}{1.204471in}}%
\pgfpathlineto{\pgfqpoint{2.421304in}{1.204471in}}%
\pgfpathlineto{\pgfqpoint{2.422568in}{1.217606in}}%
\pgfpathlineto{\pgfqpoint{2.422989in}{1.211038in}}%
\pgfpathlineto{\pgfqpoint{2.423411in}{1.217606in}}%
\pgfpathlineto{\pgfqpoint{2.425096in}{1.243876in}}%
\pgfpathlineto{\pgfqpoint{2.425517in}{1.217606in}}%
\pgfpathlineto{\pgfqpoint{2.425938in}{1.224173in}}%
\pgfpathlineto{\pgfqpoint{2.426359in}{1.237308in}}%
\pgfpathlineto{\pgfqpoint{2.426781in}{1.211038in}}%
\pgfpathlineto{\pgfqpoint{2.427202in}{1.230741in}}%
\pgfpathlineto{\pgfqpoint{2.427623in}{1.230741in}}%
\pgfpathlineto{\pgfqpoint{2.428045in}{1.224173in}}%
\pgfpathlineto{\pgfqpoint{2.428466in}{1.230741in}}%
\pgfpathlineto{\pgfqpoint{2.429308in}{1.243876in}}%
\pgfpathlineto{\pgfqpoint{2.430994in}{1.224173in}}%
\pgfpathlineto{\pgfqpoint{2.432679in}{1.237308in}}%
\pgfpathlineto{\pgfqpoint{2.433100in}{1.224173in}}%
\pgfpathlineto{\pgfqpoint{2.433521in}{1.237308in}}%
\pgfpathlineto{\pgfqpoint{2.433942in}{1.237308in}}%
\pgfpathlineto{\pgfqpoint{2.434364in}{1.243876in}}%
\pgfpathlineto{\pgfqpoint{2.435628in}{1.224173in}}%
\pgfpathlineto{\pgfqpoint{2.436470in}{1.243876in}}%
\pgfpathlineto{\pgfqpoint{2.437313in}{1.224173in}}%
\pgfpathlineto{\pgfqpoint{2.437734in}{1.237308in}}%
\pgfpathlineto{\pgfqpoint{2.438155in}{1.211038in}}%
\pgfpathlineto{\pgfqpoint{2.438577in}{1.224173in}}%
\pgfpathlineto{\pgfqpoint{2.438998in}{1.230741in}}%
\pgfpathlineto{\pgfqpoint{2.439840in}{1.217606in}}%
\pgfpathlineto{\pgfqpoint{2.440262in}{1.243876in}}%
\pgfpathlineto{\pgfqpoint{2.440683in}{1.217606in}}%
\pgfpathlineto{\pgfqpoint{2.441525in}{1.224173in}}%
\pgfpathlineto{\pgfqpoint{2.441947in}{1.217606in}}%
\pgfpathlineto{\pgfqpoint{2.442368in}{1.230741in}}%
\pgfpathlineto{\pgfqpoint{2.442789in}{1.224173in}}%
\pgfpathlineto{\pgfqpoint{2.443211in}{1.211038in}}%
\pgfpathlineto{\pgfqpoint{2.443632in}{1.230741in}}%
\pgfpathlineto{\pgfqpoint{2.444053in}{1.237308in}}%
\pgfpathlineto{\pgfqpoint{2.445317in}{1.224173in}}%
\pgfpathlineto{\pgfqpoint{2.445738in}{1.224173in}}%
\pgfpathlineto{\pgfqpoint{2.446160in}{1.217606in}}%
\pgfpathlineto{\pgfqpoint{2.446581in}{1.230741in}}%
\pgfpathlineto{\pgfqpoint{2.447002in}{1.211038in}}%
\pgfpathlineto{\pgfqpoint{2.447423in}{1.224173in}}%
\pgfpathlineto{\pgfqpoint{2.447845in}{1.224173in}}%
\pgfpathlineto{\pgfqpoint{2.448266in}{1.217606in}}%
\pgfpathlineto{\pgfqpoint{2.448687in}{1.224173in}}%
\pgfpathlineto{\pgfqpoint{2.449108in}{1.243876in}}%
\pgfpathlineto{\pgfqpoint{2.449530in}{1.211038in}}%
\pgfpathlineto{\pgfqpoint{2.450372in}{1.230741in}}%
\pgfpathlineto{\pgfqpoint{2.451215in}{1.224173in}}%
\pgfpathlineto{\pgfqpoint{2.451636in}{1.230741in}}%
\pgfpathlineto{\pgfqpoint{2.452057in}{1.211038in}}%
\pgfpathlineto{\pgfqpoint{2.452479in}{1.230741in}}%
\pgfpathlineto{\pgfqpoint{2.452900in}{1.230741in}}%
\pgfpathlineto{\pgfqpoint{2.453743in}{1.204471in}}%
\pgfpathlineto{\pgfqpoint{2.454164in}{1.230741in}}%
\pgfpathlineto{\pgfqpoint{2.455006in}{1.224173in}}%
\pgfpathlineto{\pgfqpoint{2.455428in}{1.224173in}}%
\pgfpathlineto{\pgfqpoint{2.455849in}{1.204471in}}%
\pgfpathlineto{\pgfqpoint{2.456270in}{1.217606in}}%
\pgfpathlineto{\pgfqpoint{2.456691in}{1.230741in}}%
\pgfpathlineto{\pgfqpoint{2.457534in}{1.204471in}}%
\pgfpathlineto{\pgfqpoint{2.457955in}{1.230741in}}%
\pgfpathlineto{\pgfqpoint{2.458798in}{1.217606in}}%
\pgfpathlineto{\pgfqpoint{2.459219in}{1.204471in}}%
\pgfpathlineto{\pgfqpoint{2.459640in}{1.224173in}}%
\pgfpathlineto{\pgfqpoint{2.460904in}{1.217606in}}%
\pgfpathlineto{\pgfqpoint{2.461326in}{1.217606in}}%
\pgfpathlineto{\pgfqpoint{2.462168in}{1.224173in}}%
\pgfpathlineto{\pgfqpoint{2.463432in}{1.211038in}}%
\pgfpathlineto{\pgfqpoint{2.464274in}{1.224173in}}%
\pgfpathlineto{\pgfqpoint{2.465117in}{1.211038in}}%
\pgfpathlineto{\pgfqpoint{2.465538in}{1.230741in}}%
\pgfpathlineto{\pgfqpoint{2.465960in}{1.204471in}}%
\pgfpathlineto{\pgfqpoint{2.466381in}{1.204471in}}%
\pgfpathlineto{\pgfqpoint{2.468066in}{1.237308in}}%
\pgfpathlineto{\pgfqpoint{2.469751in}{1.211038in}}%
\pgfpathlineto{\pgfqpoint{2.470594in}{1.230741in}}%
\pgfpathlineto{\pgfqpoint{2.471015in}{1.224173in}}%
\pgfpathlineto{\pgfqpoint{2.471436in}{1.217606in}}%
\pgfpathlineto{\pgfqpoint{2.472700in}{1.237308in}}%
\pgfpathlineto{\pgfqpoint{2.473964in}{1.224173in}}%
\pgfpathlineto{\pgfqpoint{2.474385in}{1.224173in}}%
\pgfpathlineto{\pgfqpoint{2.474806in}{1.217606in}}%
\pgfpathlineto{\pgfqpoint{2.475228in}{1.224173in}}%
\pgfpathlineto{\pgfqpoint{2.475649in}{1.224173in}}%
\pgfpathlineto{\pgfqpoint{2.476070in}{1.211038in}}%
\pgfpathlineto{\pgfqpoint{2.476492in}{1.224173in}}%
\pgfpathlineto{\pgfqpoint{2.476913in}{1.237308in}}%
\pgfpathlineto{\pgfqpoint{2.477334in}{1.217606in}}%
\pgfpathlineto{\pgfqpoint{2.478177in}{1.230741in}}%
\pgfpathlineto{\pgfqpoint{2.478598in}{1.224173in}}%
\pgfpathlineto{\pgfqpoint{2.479019in}{1.217606in}}%
\pgfpathlineto{\pgfqpoint{2.479441in}{1.243876in}}%
\pgfpathlineto{\pgfqpoint{2.479862in}{1.224173in}}%
\pgfpathlineto{\pgfqpoint{2.480283in}{1.224173in}}%
\pgfpathlineto{\pgfqpoint{2.480704in}{1.250443in}}%
\pgfpathlineto{\pgfqpoint{2.481126in}{1.224173in}}%
\pgfpathlineto{\pgfqpoint{2.481547in}{1.217606in}}%
\pgfpathlineto{\pgfqpoint{2.481968in}{1.237308in}}%
\pgfpathlineto{\pgfqpoint{2.482389in}{1.224173in}}%
\pgfpathlineto{\pgfqpoint{2.482811in}{1.224173in}}%
\pgfpathlineto{\pgfqpoint{2.483232in}{1.237308in}}%
\pgfpathlineto{\pgfqpoint{2.483653in}{1.230741in}}%
\pgfpathlineto{\pgfqpoint{2.484075in}{1.224173in}}%
\pgfpathlineto{\pgfqpoint{2.484496in}{1.243876in}}%
\pgfpathlineto{\pgfqpoint{2.484917in}{1.230741in}}%
\pgfpathlineto{\pgfqpoint{2.485338in}{1.230741in}}%
\pgfpathlineto{\pgfqpoint{2.485760in}{1.237308in}}%
\pgfpathlineto{\pgfqpoint{2.486181in}{1.224173in}}%
\pgfpathlineto{\pgfqpoint{2.486602in}{1.230741in}}%
\pgfpathlineto{\pgfqpoint{2.487024in}{1.237308in}}%
\pgfpathlineto{\pgfqpoint{2.487445in}{1.224173in}}%
\pgfpathlineto{\pgfqpoint{2.488287in}{1.230741in}}%
\pgfpathlineto{\pgfqpoint{2.489130in}{1.224173in}}%
\pgfpathlineto{\pgfqpoint{2.489551in}{1.230741in}}%
\pgfpathlineto{\pgfqpoint{2.489972in}{1.224173in}}%
\pgfpathlineto{\pgfqpoint{2.490394in}{1.224173in}}%
\pgfpathlineto{\pgfqpoint{2.490815in}{1.237308in}}%
\pgfpathlineto{\pgfqpoint{2.491236in}{1.230741in}}%
\pgfpathlineto{\pgfqpoint{2.492500in}{1.224173in}}%
\pgfpathlineto{\pgfqpoint{2.492921in}{1.224173in}}%
\pgfpathlineto{\pgfqpoint{2.493343in}{1.237308in}}%
\pgfpathlineto{\pgfqpoint{2.493764in}{1.224173in}}%
\pgfpathlineto{\pgfqpoint{2.494185in}{1.217606in}}%
\pgfpathlineto{\pgfqpoint{2.494607in}{1.224173in}}%
\pgfpathlineto{\pgfqpoint{2.496292in}{1.224173in}}%
\pgfpathlineto{\pgfqpoint{2.497134in}{1.217606in}}%
\pgfpathlineto{\pgfqpoint{2.497555in}{1.237308in}}%
\pgfpathlineto{\pgfqpoint{2.497977in}{1.217606in}}%
\pgfpathlineto{\pgfqpoint{2.498819in}{1.217606in}}%
\pgfpathlineto{\pgfqpoint{2.499662in}{1.224173in}}%
\pgfpathlineto{\pgfqpoint{2.500083in}{1.217606in}}%
\pgfpathlineto{\pgfqpoint{2.500504in}{1.224173in}}%
\pgfpathlineto{\pgfqpoint{2.500926in}{1.224173in}}%
\pgfpathlineto{\pgfqpoint{2.501347in}{1.217606in}}%
\pgfpathlineto{\pgfqpoint{2.501768in}{1.230741in}}%
\pgfpathlineto{\pgfqpoint{2.502190in}{1.224173in}}%
\pgfpathlineto{\pgfqpoint{2.502611in}{1.217606in}}%
\pgfpathlineto{\pgfqpoint{2.503453in}{1.237308in}}%
\pgfpathlineto{\pgfqpoint{2.504717in}{1.217606in}}%
\pgfpathlineto{\pgfqpoint{2.505560in}{1.230741in}}%
\pgfpathlineto{\pgfqpoint{2.505981in}{1.224173in}}%
\pgfpathlineto{\pgfqpoint{2.506824in}{1.217606in}}%
\pgfpathlineto{\pgfqpoint{2.507245in}{1.224173in}}%
\pgfpathlineto{\pgfqpoint{2.507666in}{1.211038in}}%
\pgfpathlineto{\pgfqpoint{2.508087in}{1.230741in}}%
\pgfpathlineto{\pgfqpoint{2.508930in}{1.204471in}}%
\pgfpathlineto{\pgfqpoint{2.509351in}{1.237308in}}%
\pgfpathlineto{\pgfqpoint{2.510194in}{1.230741in}}%
\pgfpathlineto{\pgfqpoint{2.510615in}{1.230741in}}%
\pgfpathlineto{\pgfqpoint{2.511036in}{1.211038in}}%
\pgfpathlineto{\pgfqpoint{2.511879in}{1.217606in}}%
\pgfpathlineto{\pgfqpoint{2.512300in}{1.224173in}}%
\pgfpathlineto{\pgfqpoint{2.512721in}{1.211038in}}%
\pgfpathlineto{\pgfqpoint{2.513143in}{1.230741in}}%
\pgfpathlineto{\pgfqpoint{2.513564in}{1.237308in}}%
\pgfpathlineto{\pgfqpoint{2.513985in}{1.217606in}}%
\pgfpathlineto{\pgfqpoint{2.514828in}{1.224173in}}%
\pgfpathlineto{\pgfqpoint{2.515249in}{1.217606in}}%
\pgfpathlineto{\pgfqpoint{2.516092in}{1.230741in}}%
\pgfpathlineto{\pgfqpoint{2.516934in}{1.211038in}}%
\pgfpathlineto{\pgfqpoint{2.518198in}{1.230741in}}%
\pgfpathlineto{\pgfqpoint{2.519041in}{1.217606in}}%
\pgfpathlineto{\pgfqpoint{2.520726in}{1.237308in}}%
\pgfpathlineto{\pgfqpoint{2.521568in}{1.217606in}}%
\pgfpathlineto{\pgfqpoint{2.521990in}{1.224173in}}%
\pgfpathlineto{\pgfqpoint{2.522411in}{1.230741in}}%
\pgfpathlineto{\pgfqpoint{2.522832in}{1.316119in}}%
\pgfpathlineto{\pgfqpoint{2.523253in}{1.230741in}}%
\pgfpathlineto{\pgfqpoint{2.523675in}{1.230741in}}%
\pgfpathlineto{\pgfqpoint{2.524096in}{1.217606in}}%
\pgfpathlineto{\pgfqpoint{2.524517in}{1.230741in}}%
\pgfpathlineto{\pgfqpoint{2.525781in}{1.237308in}}%
\pgfpathlineto{\pgfqpoint{2.527466in}{1.217606in}}%
\pgfpathlineto{\pgfqpoint{2.527887in}{1.217606in}}%
\pgfpathlineto{\pgfqpoint{2.528309in}{1.237308in}}%
\pgfpathlineto{\pgfqpoint{2.528730in}{1.230741in}}%
\pgfpathlineto{\pgfqpoint{2.529573in}{1.224173in}}%
\pgfpathlineto{\pgfqpoint{2.529994in}{1.243876in}}%
\pgfpathlineto{\pgfqpoint{2.530415in}{1.224173in}}%
\pgfpathlineto{\pgfqpoint{2.531258in}{1.230741in}}%
\pgfpathlineto{\pgfqpoint{2.531679in}{1.217606in}}%
\pgfpathlineto{\pgfqpoint{2.532100in}{1.224173in}}%
\pgfpathlineto{\pgfqpoint{2.533364in}{1.237308in}}%
\pgfpathlineto{\pgfqpoint{2.534207in}{1.224173in}}%
\pgfpathlineto{\pgfqpoint{2.534628in}{1.230741in}}%
\pgfpathlineto{\pgfqpoint{2.535049in}{1.224173in}}%
\pgfpathlineto{\pgfqpoint{2.535471in}{1.230741in}}%
\pgfpathlineto{\pgfqpoint{2.535892in}{1.230741in}}%
\pgfpathlineto{\pgfqpoint{2.536313in}{1.125660in}}%
\pgfpathlineto{\pgfqpoint{2.536734in}{1.217606in}}%
\pgfpathlineto{\pgfqpoint{2.537577in}{1.230741in}}%
\pgfpathlineto{\pgfqpoint{2.539262in}{1.211038in}}%
\pgfpathlineto{\pgfqpoint{2.539683in}{1.237308in}}%
\pgfpathlineto{\pgfqpoint{2.540105in}{1.224173in}}%
\pgfpathlineto{\pgfqpoint{2.540526in}{1.217606in}}%
\pgfpathlineto{\pgfqpoint{2.540947in}{1.237308in}}%
\pgfpathlineto{\pgfqpoint{2.541368in}{1.224173in}}%
\pgfpathlineto{\pgfqpoint{2.541790in}{1.211038in}}%
\pgfpathlineto{\pgfqpoint{2.542632in}{1.217606in}}%
\pgfpathlineto{\pgfqpoint{2.544317in}{1.230741in}}%
\pgfpathlineto{\pgfqpoint{2.545581in}{1.224173in}}%
\pgfpathlineto{\pgfqpoint{2.546424in}{1.224173in}}%
\pgfpathlineto{\pgfqpoint{2.546845in}{1.211038in}}%
\pgfpathlineto{\pgfqpoint{2.547688in}{1.217606in}}%
\pgfpathlineto{\pgfqpoint{2.548109in}{1.217606in}}%
\pgfpathlineto{\pgfqpoint{2.548530in}{1.230741in}}%
\pgfpathlineto{\pgfqpoint{2.548951in}{1.224173in}}%
\pgfpathlineto{\pgfqpoint{2.549373in}{1.217606in}}%
\pgfpathlineto{\pgfqpoint{2.550215in}{1.230741in}}%
\pgfpathlineto{\pgfqpoint{2.550637in}{1.204471in}}%
\pgfpathlineto{\pgfqpoint{2.551058in}{1.230741in}}%
\pgfpathlineto{\pgfqpoint{2.551900in}{1.217606in}}%
\pgfpathlineto{\pgfqpoint{2.552322in}{1.243876in}}%
\pgfpathlineto{\pgfqpoint{2.552743in}{1.224173in}}%
\pgfpathlineto{\pgfqpoint{2.553164in}{1.211038in}}%
\pgfpathlineto{\pgfqpoint{2.553585in}{1.237308in}}%
\pgfpathlineto{\pgfqpoint{2.554007in}{1.224173in}}%
\pgfpathlineto{\pgfqpoint{2.554428in}{1.217606in}}%
\pgfpathlineto{\pgfqpoint{2.554849in}{1.224173in}}%
\pgfpathlineto{\pgfqpoint{2.556113in}{1.230741in}}%
\pgfpathlineto{\pgfqpoint{2.556956in}{1.217606in}}%
\pgfpathlineto{\pgfqpoint{2.557377in}{1.230741in}}%
\pgfpathlineto{\pgfqpoint{2.557798in}{1.217606in}}%
\pgfpathlineto{\pgfqpoint{2.558220in}{1.217606in}}%
\pgfpathlineto{\pgfqpoint{2.558641in}{1.243876in}}%
\pgfpathlineto{\pgfqpoint{2.559062in}{1.224173in}}%
\pgfpathlineto{\pgfqpoint{2.559483in}{1.211038in}}%
\pgfpathlineto{\pgfqpoint{2.560747in}{1.237308in}}%
\pgfpathlineto{\pgfqpoint{2.561168in}{1.217606in}}%
\pgfpathlineto{\pgfqpoint{2.562011in}{1.224173in}}%
\pgfpathlineto{\pgfqpoint{2.562854in}{1.243876in}}%
\pgfpathlineto{\pgfqpoint{2.563275in}{1.211038in}}%
\pgfpathlineto{\pgfqpoint{2.563696in}{1.243876in}}%
\pgfpathlineto{\pgfqpoint{2.565381in}{1.211038in}}%
\pgfpathlineto{\pgfqpoint{2.565803in}{1.224173in}}%
\pgfpathlineto{\pgfqpoint{2.566645in}{1.230741in}}%
\pgfpathlineto{\pgfqpoint{2.567488in}{1.211038in}}%
\pgfpathlineto{\pgfqpoint{2.567909in}{1.224173in}}%
\pgfpathlineto{\pgfqpoint{2.568330in}{1.224173in}}%
\pgfpathlineto{\pgfqpoint{2.569594in}{1.217606in}}%
\pgfpathlineto{\pgfqpoint{2.570015in}{1.224173in}}%
\pgfpathlineto{\pgfqpoint{2.570437in}{1.217606in}}%
\pgfpathlineto{\pgfqpoint{2.570858in}{1.211038in}}%
\pgfpathlineto{\pgfqpoint{2.571279in}{1.230741in}}%
\pgfpathlineto{\pgfqpoint{2.571700in}{1.224173in}}%
\pgfpathlineto{\pgfqpoint{2.572122in}{1.204471in}}%
\pgfpathlineto{\pgfqpoint{2.572543in}{1.217606in}}%
\pgfpathlineto{\pgfqpoint{2.572964in}{1.243876in}}%
\pgfpathlineto{\pgfqpoint{2.573386in}{1.204471in}}%
\pgfpathlineto{\pgfqpoint{2.574649in}{1.230741in}}%
\pgfpathlineto{\pgfqpoint{2.575071in}{1.211038in}}%
\pgfpathlineto{\pgfqpoint{2.575913in}{1.217606in}}%
\pgfpathlineto{\pgfqpoint{2.576334in}{1.204471in}}%
\pgfpathlineto{\pgfqpoint{2.576756in}{1.230741in}}%
\pgfpathlineto{\pgfqpoint{2.577598in}{1.224173in}}%
\pgfpathlineto{\pgfqpoint{2.578441in}{1.217606in}}%
\pgfpathlineto{\pgfqpoint{2.579283in}{1.211038in}}%
\pgfpathlineto{\pgfqpoint{2.579705in}{1.237308in}}%
\pgfpathlineto{\pgfqpoint{2.580126in}{1.217606in}}%
\pgfpathlineto{\pgfqpoint{2.580547in}{1.316119in}}%
\pgfpathlineto{\pgfqpoint{2.580969in}{1.224173in}}%
\pgfpathlineto{\pgfqpoint{2.581811in}{1.224173in}}%
\pgfpathlineto{\pgfqpoint{2.583075in}{1.230741in}}%
\pgfpathlineto{\pgfqpoint{2.583496in}{1.217606in}}%
\pgfpathlineto{\pgfqpoint{2.583917in}{1.230741in}}%
\pgfpathlineto{\pgfqpoint{2.584339in}{1.230741in}}%
\pgfpathlineto{\pgfqpoint{2.585603in}{1.224173in}}%
\pgfpathlineto{\pgfqpoint{2.586024in}{1.230741in}}%
\pgfpathlineto{\pgfqpoint{2.586445in}{1.217606in}}%
\pgfpathlineto{\pgfqpoint{2.587288in}{1.224173in}}%
\pgfpathlineto{\pgfqpoint{2.588130in}{1.217606in}}%
\pgfpathlineto{\pgfqpoint{2.588552in}{1.237308in}}%
\pgfpathlineto{\pgfqpoint{2.588973in}{1.224173in}}%
\pgfpathlineto{\pgfqpoint{2.589815in}{1.224173in}}%
\pgfpathlineto{\pgfqpoint{2.590237in}{1.217606in}}%
\pgfpathlineto{\pgfqpoint{2.590658in}{1.224173in}}%
\pgfpathlineto{\pgfqpoint{2.591500in}{1.224173in}}%
\pgfpathlineto{\pgfqpoint{2.591922in}{1.230741in}}%
\pgfpathlineto{\pgfqpoint{2.592343in}{1.224173in}}%
\pgfpathlineto{\pgfqpoint{2.593186in}{1.211038in}}%
\pgfpathlineto{\pgfqpoint{2.593607in}{1.230741in}}%
\pgfpathlineto{\pgfqpoint{2.594028in}{1.119092in}}%
\pgfpathlineto{\pgfqpoint{2.594449in}{1.224173in}}%
\pgfpathlineto{\pgfqpoint{2.595713in}{1.230741in}}%
\pgfpathlineto{\pgfqpoint{2.596556in}{1.217606in}}%
\pgfpathlineto{\pgfqpoint{2.596977in}{1.230741in}}%
\pgfpathlineto{\pgfqpoint{2.597398in}{1.224173in}}%
\pgfpathlineto{\pgfqpoint{2.598241in}{1.211038in}}%
\pgfpathlineto{\pgfqpoint{2.599505in}{1.230741in}}%
\pgfpathlineto{\pgfqpoint{2.600769in}{1.204471in}}%
\pgfpathlineto{\pgfqpoint{2.601611in}{1.237308in}}%
\pgfpathlineto{\pgfqpoint{2.602032in}{1.224173in}}%
\pgfpathlineto{\pgfqpoint{2.602875in}{1.217606in}}%
\pgfpathlineto{\pgfqpoint{2.603718in}{1.237308in}}%
\pgfpathlineto{\pgfqpoint{2.604981in}{1.217606in}}%
\pgfpathlineto{\pgfqpoint{2.605824in}{1.237308in}}%
\pgfpathlineto{\pgfqpoint{2.606245in}{1.230741in}}%
\pgfpathlineto{\pgfqpoint{2.606667in}{1.211038in}}%
\pgfpathlineto{\pgfqpoint{2.607088in}{1.217606in}}%
\pgfpathlineto{\pgfqpoint{2.607509in}{1.230741in}}%
\pgfpathlineto{\pgfqpoint{2.608352in}{1.224173in}}%
\pgfpathlineto{\pgfqpoint{2.608773in}{1.224173in}}%
\pgfpathlineto{\pgfqpoint{2.609194in}{1.211038in}}%
\pgfpathlineto{\pgfqpoint{2.609615in}{1.230741in}}%
\pgfpathlineto{\pgfqpoint{2.610879in}{1.224173in}}%
\pgfpathlineto{\pgfqpoint{2.611301in}{1.224173in}}%
\pgfpathlineto{\pgfqpoint{2.611722in}{1.230741in}}%
\pgfpathlineto{\pgfqpoint{2.612143in}{1.217606in}}%
\pgfpathlineto{\pgfqpoint{2.612986in}{1.224173in}}%
\pgfpathlineto{\pgfqpoint{2.613407in}{1.217606in}}%
\pgfpathlineto{\pgfqpoint{2.613828in}{1.224173in}}%
\pgfpathlineto{\pgfqpoint{2.614250in}{1.243876in}}%
\pgfpathlineto{\pgfqpoint{2.614671in}{1.224173in}}%
\pgfpathlineto{\pgfqpoint{2.615092in}{1.217606in}}%
\pgfpathlineto{\pgfqpoint{2.615513in}{0.915498in}}%
\pgfpathlineto{\pgfqpoint{2.615935in}{1.237308in}}%
\pgfpathlineto{\pgfqpoint{2.617198in}{1.217606in}}%
\pgfpathlineto{\pgfqpoint{2.617620in}{0.941769in}}%
\pgfpathlineto{\pgfqpoint{2.618041in}{1.224173in}}%
\pgfpathlineto{\pgfqpoint{2.618462in}{1.230741in}}%
\pgfpathlineto{\pgfqpoint{2.618884in}{1.224173in}}%
\pgfpathlineto{\pgfqpoint{2.619305in}{1.211038in}}%
\pgfpathlineto{\pgfqpoint{2.619726in}{0.961471in}}%
\pgfpathlineto{\pgfqpoint{2.620147in}{1.237308in}}%
\pgfpathlineto{\pgfqpoint{2.620569in}{1.237308in}}%
\pgfpathlineto{\pgfqpoint{2.620990in}{1.224173in}}%
\pgfpathlineto{\pgfqpoint{2.621411in}{1.230741in}}%
\pgfpathlineto{\pgfqpoint{2.621833in}{0.974606in}}%
\pgfpathlineto{\pgfqpoint{2.622254in}{1.224173in}}%
\pgfpathlineto{\pgfqpoint{2.622675in}{1.237308in}}%
\pgfpathlineto{\pgfqpoint{2.623939in}{0.994309in}}%
\pgfpathlineto{\pgfqpoint{2.624781in}{1.230741in}}%
\pgfpathlineto{\pgfqpoint{2.625203in}{1.217606in}}%
\pgfpathlineto{\pgfqpoint{2.625624in}{1.224173in}}%
\pgfpathlineto{\pgfqpoint{2.626045in}{1.027147in}}%
\pgfpathlineto{\pgfqpoint{2.626467in}{1.224173in}}%
\pgfpathlineto{\pgfqpoint{2.626888in}{1.237308in}}%
\pgfpathlineto{\pgfqpoint{2.627309in}{1.224173in}}%
\pgfpathlineto{\pgfqpoint{2.627730in}{1.224173in}}%
\pgfpathlineto{\pgfqpoint{2.628152in}{1.046849in}}%
\pgfpathlineto{\pgfqpoint{2.628573in}{1.224173in}}%
\pgfpathlineto{\pgfqpoint{2.628994in}{1.539416in}}%
\pgfpathlineto{\pgfqpoint{2.629416in}{1.224173in}}%
\pgfpathlineto{\pgfqpoint{2.629837in}{1.224173in}}%
\pgfpathlineto{\pgfqpoint{2.630258in}{1.073120in}}%
\pgfpathlineto{\pgfqpoint{2.630679in}{1.230741in}}%
\pgfpathlineto{\pgfqpoint{2.631101in}{1.506578in}}%
\pgfpathlineto{\pgfqpoint{2.631522in}{1.237308in}}%
\pgfpathlineto{\pgfqpoint{2.631943in}{1.224173in}}%
\pgfpathlineto{\pgfqpoint{2.632364in}{1.079687in}}%
\pgfpathlineto{\pgfqpoint{2.632786in}{1.237308in}}%
\pgfpathlineto{\pgfqpoint{2.633207in}{1.473740in}}%
\pgfpathlineto{\pgfqpoint{2.633628in}{1.224173in}}%
\pgfpathlineto{\pgfqpoint{2.634050in}{1.230741in}}%
\pgfpathlineto{\pgfqpoint{2.634471in}{1.105957in}}%
\pgfpathlineto{\pgfqpoint{2.634892in}{1.224173in}}%
\pgfpathlineto{\pgfqpoint{2.635313in}{1.467172in}}%
\pgfpathlineto{\pgfqpoint{2.635735in}{1.224173in}}%
\pgfpathlineto{\pgfqpoint{2.636156in}{1.217606in}}%
\pgfpathlineto{\pgfqpoint{2.637420in}{1.447470in}}%
\pgfpathlineto{\pgfqpoint{2.638684in}{1.165065in}}%
\pgfpathlineto{\pgfqpoint{2.639526in}{1.421200in}}%
\pgfpathlineto{\pgfqpoint{2.639947in}{1.230741in}}%
\pgfpathlineto{\pgfqpoint{2.640790in}{1.178200in}}%
\pgfpathlineto{\pgfqpoint{2.641211in}{1.204471in}}%
\pgfpathlineto{\pgfqpoint{2.641633in}{1.408065in}}%
\pgfpathlineto{\pgfqpoint{2.642054in}{1.230741in}}%
\pgfpathlineto{\pgfqpoint{2.642896in}{1.197903in}}%
\pgfpathlineto{\pgfqpoint{2.643318in}{1.224173in}}%
\pgfpathlineto{\pgfqpoint{2.643739in}{1.368659in}}%
\pgfpathlineto{\pgfqpoint{2.644160in}{1.224173in}}%
\pgfpathlineto{\pgfqpoint{2.645003in}{1.217606in}}%
\pgfpathlineto{\pgfqpoint{2.645424in}{1.230741in}}%
\pgfpathlineto{\pgfqpoint{2.645845in}{1.368659in}}%
\pgfpathlineto{\pgfqpoint{2.646267in}{1.217606in}}%
\pgfpathlineto{\pgfqpoint{2.647109in}{1.237308in}}%
\pgfpathlineto{\pgfqpoint{2.647530in}{1.211038in}}%
\pgfpathlineto{\pgfqpoint{2.647952in}{1.355524in}}%
\pgfpathlineto{\pgfqpoint{2.648373in}{1.230741in}}%
\pgfpathlineto{\pgfqpoint{2.648794in}{1.224173in}}%
\pgfpathlineto{\pgfqpoint{2.649216in}{1.237308in}}%
\pgfpathlineto{\pgfqpoint{2.649637in}{1.224173in}}%
\pgfpathlineto{\pgfqpoint{2.650058in}{1.302984in}}%
\pgfpathlineto{\pgfqpoint{2.650479in}{1.230741in}}%
\pgfpathlineto{\pgfqpoint{2.651322in}{1.224173in}}%
\pgfpathlineto{\pgfqpoint{2.651743in}{1.230741in}}%
\pgfpathlineto{\pgfqpoint{2.652165in}{1.289849in}}%
\pgfpathlineto{\pgfqpoint{2.652586in}{1.217606in}}%
\pgfpathlineto{\pgfqpoint{2.654271in}{1.276714in}}%
\pgfpathlineto{\pgfqpoint{2.655113in}{1.211038in}}%
\pgfpathlineto{\pgfqpoint{2.655956in}{1.230741in}}%
\pgfpathlineto{\pgfqpoint{2.656377in}{1.230741in}}%
\pgfpathlineto{\pgfqpoint{2.657641in}{1.217606in}}%
\pgfpathlineto{\pgfqpoint{2.658062in}{1.237308in}}%
\pgfpathlineto{\pgfqpoint{2.658484in}{1.217606in}}%
\pgfpathlineto{\pgfqpoint{2.660590in}{1.217606in}}%
\pgfpathlineto{\pgfqpoint{2.661011in}{1.224173in}}%
\pgfpathlineto{\pgfqpoint{2.661433in}{1.204471in}}%
\pgfpathlineto{\pgfqpoint{2.661854in}{1.224173in}}%
\pgfpathlineto{\pgfqpoint{2.662275in}{1.224173in}}%
\pgfpathlineto{\pgfqpoint{2.662696in}{1.211038in}}%
\pgfpathlineto{\pgfqpoint{2.663118in}{1.237308in}}%
\pgfpathlineto{\pgfqpoint{2.663539in}{1.224173in}}%
\pgfpathlineto{\pgfqpoint{2.663960in}{1.217606in}}%
\pgfpathlineto{\pgfqpoint{2.664382in}{1.224173in}}%
\pgfpathlineto{\pgfqpoint{2.664803in}{1.224173in}}%
\pgfpathlineto{\pgfqpoint{2.665224in}{1.217606in}}%
\pgfpathlineto{\pgfqpoint{2.665645in}{1.224173in}}%
\pgfpathlineto{\pgfqpoint{2.666488in}{1.224173in}}%
\pgfpathlineto{\pgfqpoint{2.666909in}{1.237308in}}%
\pgfpathlineto{\pgfqpoint{2.667331in}{1.224173in}}%
\pgfpathlineto{\pgfqpoint{2.667752in}{1.217606in}}%
\pgfpathlineto{\pgfqpoint{2.668594in}{1.230741in}}%
\pgfpathlineto{\pgfqpoint{2.669016in}{1.211038in}}%
\pgfpathlineto{\pgfqpoint{2.669437in}{1.230741in}}%
\pgfpathlineto{\pgfqpoint{2.670280in}{1.217606in}}%
\pgfpathlineto{\pgfqpoint{2.670701in}{1.237308in}}%
\pgfpathlineto{\pgfqpoint{2.671122in}{1.224173in}}%
\pgfpathlineto{\pgfqpoint{2.671543in}{1.217606in}}%
\pgfpathlineto{\pgfqpoint{2.673228in}{1.237308in}}%
\pgfpathlineto{\pgfqpoint{2.674071in}{1.211038in}}%
\pgfpathlineto{\pgfqpoint{2.674492in}{1.230741in}}%
\pgfpathlineto{\pgfqpoint{2.674914in}{1.230741in}}%
\pgfpathlineto{\pgfqpoint{2.676599in}{1.211038in}}%
\pgfpathlineto{\pgfqpoint{2.677020in}{1.237308in}}%
\pgfpathlineto{\pgfqpoint{2.677863in}{1.224173in}}%
\pgfpathlineto{\pgfqpoint{2.678284in}{1.217606in}}%
\pgfpathlineto{\pgfqpoint{2.678705in}{1.224173in}}%
\pgfpathlineto{\pgfqpoint{2.679126in}{1.224173in}}%
\pgfpathlineto{\pgfqpoint{2.679548in}{1.230741in}}%
\pgfpathlineto{\pgfqpoint{2.680390in}{1.197903in}}%
\pgfpathlineto{\pgfqpoint{2.680811in}{1.230741in}}%
\pgfpathlineto{\pgfqpoint{2.681654in}{1.224173in}}%
\pgfpathlineto{\pgfqpoint{2.682075in}{1.237308in}}%
\pgfpathlineto{\pgfqpoint{2.682497in}{1.224173in}}%
\pgfpathlineto{\pgfqpoint{2.682918in}{1.217606in}}%
\pgfpathlineto{\pgfqpoint{2.683339in}{1.243876in}}%
\pgfpathlineto{\pgfqpoint{2.683760in}{1.224173in}}%
\pgfpathlineto{\pgfqpoint{2.684182in}{1.217606in}}%
\pgfpathlineto{\pgfqpoint{2.684603in}{1.224173in}}%
\pgfpathlineto{\pgfqpoint{2.685867in}{1.230741in}}%
\pgfpathlineto{\pgfqpoint{2.686709in}{1.211038in}}%
\pgfpathlineto{\pgfqpoint{2.687131in}{1.237308in}}%
\pgfpathlineto{\pgfqpoint{2.687552in}{1.230741in}}%
\pgfpathlineto{\pgfqpoint{2.687973in}{1.211038in}}%
\pgfpathlineto{\pgfqpoint{2.688394in}{1.224173in}}%
\pgfpathlineto{\pgfqpoint{2.688816in}{1.224173in}}%
\pgfpathlineto{\pgfqpoint{2.689237in}{1.217606in}}%
\pgfpathlineto{\pgfqpoint{2.689658in}{1.237308in}}%
\pgfpathlineto{\pgfqpoint{2.690080in}{1.230741in}}%
\pgfpathlineto{\pgfqpoint{2.690501in}{1.217606in}}%
\pgfpathlineto{\pgfqpoint{2.691343in}{1.224173in}}%
\pgfpathlineto{\pgfqpoint{2.692186in}{1.230741in}}%
\pgfpathlineto{\pgfqpoint{2.693029in}{1.217606in}}%
\pgfpathlineto{\pgfqpoint{2.693871in}{1.237308in}}%
\pgfpathlineto{\pgfqpoint{2.695135in}{1.211038in}}%
\pgfpathlineto{\pgfqpoint{2.695556in}{1.211038in}}%
\pgfpathlineto{\pgfqpoint{2.695977in}{1.230741in}}%
\pgfpathlineto{\pgfqpoint{2.696399in}{1.224173in}}%
\pgfpathlineto{\pgfqpoint{2.696820in}{1.217606in}}%
\pgfpathlineto{\pgfqpoint{2.697241in}{1.230741in}}%
\pgfpathlineto{\pgfqpoint{2.698084in}{1.224173in}}%
\pgfpathlineto{\pgfqpoint{2.698505in}{1.230741in}}%
\pgfpathlineto{\pgfqpoint{2.698926in}{1.224173in}}%
\pgfpathlineto{\pgfqpoint{2.699348in}{1.217606in}}%
\pgfpathlineto{\pgfqpoint{2.699769in}{1.230741in}}%
\pgfpathlineto{\pgfqpoint{2.700190in}{1.224173in}}%
\pgfpathlineto{\pgfqpoint{2.700612in}{1.211038in}}%
\pgfpathlineto{\pgfqpoint{2.701033in}{1.217606in}}%
\pgfpathlineto{\pgfqpoint{2.702718in}{1.230741in}}%
\pgfpathlineto{\pgfqpoint{2.703560in}{1.217606in}}%
\pgfpathlineto{\pgfqpoint{2.703982in}{1.224173in}}%
\pgfpathlineto{\pgfqpoint{2.704403in}{1.224173in}}%
\pgfpathlineto{\pgfqpoint{2.704824in}{1.230741in}}%
\pgfpathlineto{\pgfqpoint{2.705667in}{1.217606in}}%
\pgfpathlineto{\pgfqpoint{2.706088in}{1.230741in}}%
\pgfpathlineto{\pgfqpoint{2.706509in}{1.224173in}}%
\pgfpathlineto{\pgfqpoint{2.706931in}{1.217606in}}%
\pgfpathlineto{\pgfqpoint{2.707352in}{1.237308in}}%
\pgfpathlineto{\pgfqpoint{2.707773in}{1.217606in}}%
\pgfpathlineto{\pgfqpoint{2.708616in}{1.237308in}}%
\pgfpathlineto{\pgfqpoint{2.709037in}{1.217606in}}%
\pgfpathlineto{\pgfqpoint{2.709880in}{1.224173in}}%
\pgfpathlineto{\pgfqpoint{2.710301in}{1.237308in}}%
\pgfpathlineto{\pgfqpoint{2.711143in}{1.230741in}}%
\pgfpathlineto{\pgfqpoint{2.711986in}{1.211038in}}%
\pgfpathlineto{\pgfqpoint{2.712407in}{1.230741in}}%
\pgfpathlineto{\pgfqpoint{2.713250in}{1.224173in}}%
\pgfpathlineto{\pgfqpoint{2.713671in}{1.237308in}}%
\pgfpathlineto{\pgfqpoint{2.714092in}{1.211038in}}%
\pgfpathlineto{\pgfqpoint{2.714514in}{1.237308in}}%
\pgfpathlineto{\pgfqpoint{2.714935in}{1.237308in}}%
\pgfpathlineto{\pgfqpoint{2.716620in}{1.217606in}}%
\pgfpathlineto{\pgfqpoint{2.717463in}{1.230741in}}%
\pgfpathlineto{\pgfqpoint{2.718305in}{1.217606in}}%
\pgfpathlineto{\pgfqpoint{2.719148in}{1.230741in}}%
\pgfpathlineto{\pgfqpoint{2.719569in}{1.224173in}}%
\pgfpathlineto{\pgfqpoint{2.719990in}{1.224173in}}%
\pgfpathlineto{\pgfqpoint{2.720412in}{1.211038in}}%
\pgfpathlineto{\pgfqpoint{2.720833in}{1.217606in}}%
\pgfpathlineto{\pgfqpoint{2.721254in}{1.237308in}}%
\pgfpathlineto{\pgfqpoint{2.721675in}{1.224173in}}%
\pgfpathlineto{\pgfqpoint{2.722097in}{1.224173in}}%
\pgfpathlineto{\pgfqpoint{2.723361in}{1.230741in}}%
\pgfpathlineto{\pgfqpoint{2.724624in}{1.217606in}}%
\pgfpathlineto{\pgfqpoint{2.725046in}{1.230741in}}%
\pgfpathlineto{\pgfqpoint{2.725888in}{1.224173in}}%
\pgfpathlineto{\pgfqpoint{2.726309in}{1.230741in}}%
\pgfpathlineto{\pgfqpoint{2.726731in}{1.211038in}}%
\pgfpathlineto{\pgfqpoint{2.727152in}{1.224173in}}%
\pgfpathlineto{\pgfqpoint{2.727573in}{1.243876in}}%
\pgfpathlineto{\pgfqpoint{2.727995in}{1.224173in}}%
\pgfpathlineto{\pgfqpoint{2.728416in}{1.224173in}}%
\pgfpathlineto{\pgfqpoint{2.729258in}{1.211038in}}%
\pgfpathlineto{\pgfqpoint{2.730101in}{1.230741in}}%
\pgfpathlineto{\pgfqpoint{2.730944in}{1.217606in}}%
\pgfpathlineto{\pgfqpoint{2.731365in}{1.230741in}}%
\pgfpathlineto{\pgfqpoint{2.732207in}{1.224173in}}%
\pgfpathlineto{\pgfqpoint{2.733050in}{1.211038in}}%
\pgfpathlineto{\pgfqpoint{2.733893in}{1.237308in}}%
\pgfpathlineto{\pgfqpoint{2.734314in}{1.211038in}}%
\pgfpathlineto{\pgfqpoint{2.735156in}{1.224173in}}%
\pgfpathlineto{\pgfqpoint{2.735578in}{1.217606in}}%
\pgfpathlineto{\pgfqpoint{2.735999in}{1.237308in}}%
\pgfpathlineto{\pgfqpoint{2.736420in}{1.224173in}}%
\pgfpathlineto{\pgfqpoint{2.737263in}{1.211038in}}%
\pgfpathlineto{\pgfqpoint{2.738105in}{1.230741in}}%
\pgfpathlineto{\pgfqpoint{2.738527in}{1.217606in}}%
\pgfpathlineto{\pgfqpoint{2.739369in}{1.217606in}}%
\pgfpathlineto{\pgfqpoint{2.740212in}{1.230741in}}%
\pgfpathlineto{\pgfqpoint{2.740633in}{1.224173in}}%
\pgfpathlineto{\pgfqpoint{2.741054in}{1.224173in}}%
\pgfpathlineto{\pgfqpoint{2.741476in}{1.204471in}}%
\pgfpathlineto{\pgfqpoint{2.741897in}{1.217606in}}%
\pgfpathlineto{\pgfqpoint{2.742739in}{1.224173in}}%
\pgfpathlineto{\pgfqpoint{2.743582in}{1.217606in}}%
\pgfpathlineto{\pgfqpoint{2.744846in}{1.224173in}}%
\pgfpathlineto{\pgfqpoint{2.745267in}{1.224173in}}%
\pgfpathlineto{\pgfqpoint{2.745688in}{1.217606in}}%
\pgfpathlineto{\pgfqpoint{2.746531in}{1.237308in}}%
\pgfpathlineto{\pgfqpoint{2.746952in}{1.217606in}}%
\pgfpathlineto{\pgfqpoint{2.747373in}{1.224173in}}%
\pgfpathlineto{\pgfqpoint{2.747795in}{1.230741in}}%
\pgfpathlineto{\pgfqpoint{2.749480in}{1.211038in}}%
\pgfpathlineto{\pgfqpoint{2.750744in}{1.230741in}}%
\pgfpathlineto{\pgfqpoint{2.751165in}{1.224173in}}%
\pgfpathlineto{\pgfqpoint{2.751586in}{1.211038in}}%
\pgfpathlineto{\pgfqpoint{2.752007in}{1.224173in}}%
\pgfpathlineto{\pgfqpoint{2.752429in}{1.237308in}}%
\pgfpathlineto{\pgfqpoint{2.752850in}{1.224173in}}%
\pgfpathlineto{\pgfqpoint{2.753271in}{1.217606in}}%
\pgfpathlineto{\pgfqpoint{2.753693in}{1.230741in}}%
\pgfpathlineto{\pgfqpoint{2.754114in}{1.224173in}}%
\pgfpathlineto{\pgfqpoint{2.754535in}{1.211038in}}%
\pgfpathlineto{\pgfqpoint{2.754956in}{1.243876in}}%
\pgfpathlineto{\pgfqpoint{2.755799in}{1.224173in}}%
\pgfpathlineto{\pgfqpoint{2.756642in}{1.224173in}}%
\pgfpathlineto{\pgfqpoint{2.757063in}{1.217606in}}%
\pgfpathlineto{\pgfqpoint{2.757905in}{1.230741in}}%
\pgfpathlineto{\pgfqpoint{2.758327in}{1.211038in}}%
\pgfpathlineto{\pgfqpoint{2.758748in}{1.217606in}}%
\pgfpathlineto{\pgfqpoint{2.759169in}{1.224173in}}%
\pgfpathlineto{\pgfqpoint{2.759590in}{1.211038in}}%
\pgfpathlineto{\pgfqpoint{2.760012in}{1.230741in}}%
\pgfpathlineto{\pgfqpoint{2.760433in}{1.217606in}}%
\pgfpathlineto{\pgfqpoint{2.760854in}{1.211038in}}%
\pgfpathlineto{\pgfqpoint{2.761697in}{1.230741in}}%
\pgfpathlineto{\pgfqpoint{2.762118in}{1.211038in}}%
\pgfpathlineto{\pgfqpoint{2.762539in}{1.230741in}}%
\pgfpathlineto{\pgfqpoint{2.763382in}{1.224173in}}%
\pgfpathlineto{\pgfqpoint{2.763803in}{1.243876in}}%
\pgfpathlineto{\pgfqpoint{2.764225in}{1.217606in}}%
\pgfpathlineto{\pgfqpoint{2.764646in}{1.217606in}}%
\pgfpathlineto{\pgfqpoint{2.766331in}{1.230741in}}%
\pgfpathlineto{\pgfqpoint{2.767173in}{1.217606in}}%
\pgfpathlineto{\pgfqpoint{2.767595in}{1.230741in}}%
\pgfpathlineto{\pgfqpoint{2.768016in}{1.224173in}}%
\pgfpathlineto{\pgfqpoint{2.768437in}{1.204471in}}%
\pgfpathlineto{\pgfqpoint{2.769280in}{1.237308in}}%
\pgfpathlineto{\pgfqpoint{2.769701in}{1.230741in}}%
\pgfpathlineto{\pgfqpoint{2.770122in}{1.237308in}}%
\pgfpathlineto{\pgfqpoint{2.770544in}{1.230741in}}%
\pgfpathlineto{\pgfqpoint{2.771386in}{1.217606in}}%
\pgfpathlineto{\pgfqpoint{2.771808in}{1.224173in}}%
\pgfpathlineto{\pgfqpoint{2.773071in}{1.237308in}}%
\pgfpathlineto{\pgfqpoint{2.773493in}{1.197903in}}%
\pgfpathlineto{\pgfqpoint{2.773914in}{1.237308in}}%
\pgfpathlineto{\pgfqpoint{2.774335in}{1.237308in}}%
\pgfpathlineto{\pgfqpoint{2.774756in}{1.217606in}}%
\pgfpathlineto{\pgfqpoint{2.775178in}{1.230741in}}%
\pgfpathlineto{\pgfqpoint{2.775599in}{1.230741in}}%
\pgfpathlineto{\pgfqpoint{2.777284in}{1.217606in}}%
\pgfpathlineto{\pgfqpoint{2.778548in}{1.243876in}}%
\pgfpathlineto{\pgfqpoint{2.778969in}{1.237308in}}%
\pgfpathlineto{\pgfqpoint{2.779391in}{1.217606in}}%
\pgfpathlineto{\pgfqpoint{2.780233in}{1.224173in}}%
\pgfpathlineto{\pgfqpoint{2.780654in}{1.230741in}}%
\pgfpathlineto{\pgfqpoint{2.781076in}{1.211038in}}%
\pgfpathlineto{\pgfqpoint{2.781497in}{1.250443in}}%
\pgfpathlineto{\pgfqpoint{2.781918in}{1.237308in}}%
\pgfpathlineto{\pgfqpoint{2.782339in}{1.211038in}}%
\pgfpathlineto{\pgfqpoint{2.783182in}{1.217606in}}%
\pgfpathlineto{\pgfqpoint{2.783603in}{1.217606in}}%
\pgfpathlineto{\pgfqpoint{2.784446in}{1.224173in}}%
\pgfpathlineto{\pgfqpoint{2.784867in}{1.217606in}}%
\pgfpathlineto{\pgfqpoint{2.785710in}{1.230741in}}%
\pgfpathlineto{\pgfqpoint{2.786131in}{1.224173in}}%
\pgfpathlineto{\pgfqpoint{2.786552in}{1.217606in}}%
\pgfpathlineto{\pgfqpoint{2.786974in}{1.237308in}}%
\pgfpathlineto{\pgfqpoint{2.787816in}{1.230741in}}%
\pgfpathlineto{\pgfqpoint{2.788237in}{1.243876in}}%
\pgfpathlineto{\pgfqpoint{2.789501in}{1.211038in}}%
\pgfpathlineto{\pgfqpoint{2.790344in}{1.237308in}}%
\pgfpathlineto{\pgfqpoint{2.790765in}{1.211038in}}%
\pgfpathlineto{\pgfqpoint{2.791608in}{1.224173in}}%
\pgfpathlineto{\pgfqpoint{2.792450in}{1.224173in}}%
\pgfpathlineto{\pgfqpoint{2.792871in}{1.230741in}}%
\pgfpathlineto{\pgfqpoint{2.793293in}{1.211038in}}%
\pgfpathlineto{\pgfqpoint{2.793714in}{1.230741in}}%
\pgfpathlineto{\pgfqpoint{2.794978in}{1.211038in}}%
\pgfpathlineto{\pgfqpoint{2.796242in}{1.230741in}}%
\pgfpathlineto{\pgfqpoint{2.797084in}{1.224173in}}%
\pgfpathlineto{\pgfqpoint{2.798348in}{1.230741in}}%
\pgfpathlineto{\pgfqpoint{2.798769in}{1.230741in}}%
\pgfpathlineto{\pgfqpoint{2.799191in}{1.224173in}}%
\pgfpathlineto{\pgfqpoint{2.799612in}{1.237308in}}%
\pgfpathlineto{\pgfqpoint{2.800454in}{1.230741in}}%
\pgfpathlineto{\pgfqpoint{2.801718in}{1.217606in}}%
\pgfpathlineto{\pgfqpoint{2.802140in}{1.217606in}}%
\pgfpathlineto{\pgfqpoint{2.802982in}{1.243876in}}%
\pgfpathlineto{\pgfqpoint{2.803403in}{1.224173in}}%
\pgfpathlineto{\pgfqpoint{2.803825in}{1.224173in}}%
\pgfpathlineto{\pgfqpoint{2.804246in}{1.237308in}}%
\pgfpathlineto{\pgfqpoint{2.804667in}{1.217606in}}%
\pgfpathlineto{\pgfqpoint{2.805510in}{1.224173in}}%
\pgfpathlineto{\pgfqpoint{2.805931in}{1.217606in}}%
\pgfpathlineto{\pgfqpoint{2.806352in}{1.224173in}}%
\pgfpathlineto{\pgfqpoint{2.806774in}{1.230741in}}%
\pgfpathlineto{\pgfqpoint{2.807616in}{1.211038in}}%
\pgfpathlineto{\pgfqpoint{2.808037in}{1.230741in}}%
\pgfpathlineto{\pgfqpoint{2.808880in}{1.224173in}}%
\pgfpathlineto{\pgfqpoint{2.809301in}{1.224173in}}%
\pgfpathlineto{\pgfqpoint{2.809723in}{1.211038in}}%
\pgfpathlineto{\pgfqpoint{2.810144in}{1.217606in}}%
\pgfpathlineto{\pgfqpoint{2.810565in}{1.224173in}}%
\pgfpathlineto{\pgfqpoint{2.810986in}{1.211038in}}%
\pgfpathlineto{\pgfqpoint{2.811408in}{1.224173in}}%
\pgfpathlineto{\pgfqpoint{2.811829in}{1.224173in}}%
\pgfpathlineto{\pgfqpoint{2.813093in}{1.217606in}}%
\pgfpathlineto{\pgfqpoint{2.814357in}{1.243876in}}%
\pgfpathlineto{\pgfqpoint{2.814778in}{1.217606in}}%
\pgfpathlineto{\pgfqpoint{2.815620in}{1.230741in}}%
\pgfpathlineto{\pgfqpoint{2.817306in}{1.211038in}}%
\pgfpathlineto{\pgfqpoint{2.818148in}{1.230741in}}%
\pgfpathlineto{\pgfqpoint{2.818569in}{1.224173in}}%
\pgfpathlineto{\pgfqpoint{2.818991in}{1.211038in}}%
\pgfpathlineto{\pgfqpoint{2.819412in}{1.237308in}}%
\pgfpathlineto{\pgfqpoint{2.820255in}{1.230741in}}%
\pgfpathlineto{\pgfqpoint{2.821940in}{1.230741in}}%
\pgfpathlineto{\pgfqpoint{2.823203in}{1.224173in}}%
\pgfpathlineto{\pgfqpoint{2.823625in}{1.237308in}}%
\pgfpathlineto{\pgfqpoint{2.824467in}{1.230741in}}%
\pgfpathlineto{\pgfqpoint{2.825310in}{1.224173in}}%
\pgfpathlineto{\pgfqpoint{2.825731in}{1.230741in}}%
\pgfpathlineto{\pgfqpoint{2.826152in}{1.224173in}}%
\pgfpathlineto{\pgfqpoint{2.826574in}{1.224173in}}%
\pgfpathlineto{\pgfqpoint{2.826995in}{1.230741in}}%
\pgfpathlineto{\pgfqpoint{2.827838in}{1.217606in}}%
\pgfpathlineto{\pgfqpoint{2.828259in}{1.237308in}}%
\pgfpathlineto{\pgfqpoint{2.829101in}{1.230741in}}%
\pgfpathlineto{\pgfqpoint{2.829944in}{1.211038in}}%
\pgfpathlineto{\pgfqpoint{2.830786in}{1.230741in}}%
\pgfpathlineto{\pgfqpoint{2.831208in}{1.224173in}}%
\pgfpathlineto{\pgfqpoint{2.831629in}{1.217606in}}%
\pgfpathlineto{\pgfqpoint{2.832050in}{1.230741in}}%
\pgfpathlineto{\pgfqpoint{2.832472in}{1.224173in}}%
\pgfpathlineto{\pgfqpoint{2.832893in}{1.217606in}}%
\pgfpathlineto{\pgfqpoint{2.833314in}{1.230741in}}%
\pgfpathlineto{\pgfqpoint{2.834157in}{1.224173in}}%
\pgfpathlineto{\pgfqpoint{2.834578in}{1.224173in}}%
\pgfpathlineto{\pgfqpoint{2.835421in}{1.204471in}}%
\pgfpathlineto{\pgfqpoint{2.835842in}{1.237308in}}%
\pgfpathlineto{\pgfqpoint{2.836684in}{1.224173in}}%
\pgfpathlineto{\pgfqpoint{2.837527in}{1.217606in}}%
\pgfpathlineto{\pgfqpoint{2.838369in}{1.243876in}}%
\pgfpathlineto{\pgfqpoint{2.838791in}{1.217606in}}%
\pgfpathlineto{\pgfqpoint{2.839633in}{1.230741in}}%
\pgfpathlineto{\pgfqpoint{2.840055in}{1.211038in}}%
\pgfpathlineto{\pgfqpoint{2.840476in}{1.224173in}}%
\pgfpathlineto{\pgfqpoint{2.840897in}{1.237308in}}%
\pgfpathlineto{\pgfqpoint{2.841740in}{1.211038in}}%
\pgfpathlineto{\pgfqpoint{2.843004in}{1.230741in}}%
\pgfpathlineto{\pgfqpoint{2.843425in}{1.237308in}}%
\pgfpathlineto{\pgfqpoint{2.844267in}{1.224173in}}%
\pgfpathlineto{\pgfqpoint{2.844689in}{1.237308in}}%
\pgfpathlineto{\pgfqpoint{2.845110in}{1.230741in}}%
\pgfpathlineto{\pgfqpoint{2.846374in}{1.217606in}}%
\pgfpathlineto{\pgfqpoint{2.847216in}{1.230741in}}%
\pgfpathlineto{\pgfqpoint{2.847638in}{1.211038in}}%
\pgfpathlineto{\pgfqpoint{2.848059in}{1.230741in}}%
\pgfpathlineto{\pgfqpoint{2.849323in}{1.211038in}}%
\pgfpathlineto{\pgfqpoint{2.850165in}{1.230741in}}%
\pgfpathlineto{\pgfqpoint{2.850587in}{1.224173in}}%
\pgfpathlineto{\pgfqpoint{2.851008in}{1.230741in}}%
\pgfpathlineto{\pgfqpoint{2.851429in}{1.224173in}}%
\pgfpathlineto{\pgfqpoint{2.851850in}{1.217606in}}%
\pgfpathlineto{\pgfqpoint{2.852272in}{1.237308in}}%
\pgfpathlineto{\pgfqpoint{2.852693in}{1.204471in}}%
\pgfpathlineto{\pgfqpoint{2.853114in}{1.224173in}}%
\pgfpathlineto{\pgfqpoint{2.853535in}{1.230741in}}%
\pgfpathlineto{\pgfqpoint{2.853957in}{1.211038in}}%
\pgfpathlineto{\pgfqpoint{2.854378in}{1.230741in}}%
\pgfpathlineto{\pgfqpoint{2.854799in}{1.237308in}}%
\pgfpathlineto{\pgfqpoint{2.855221in}{1.230741in}}%
\pgfpathlineto{\pgfqpoint{2.855642in}{1.217606in}}%
\pgfpathlineto{\pgfqpoint{2.856063in}{1.224173in}}%
\pgfpathlineto{\pgfqpoint{2.857327in}{1.237308in}}%
\pgfpathlineto{\pgfqpoint{2.858170in}{1.224173in}}%
\pgfpathlineto{\pgfqpoint{2.858591in}{1.230741in}}%
\pgfpathlineto{\pgfqpoint{2.859012in}{1.237308in}}%
\pgfpathlineto{\pgfqpoint{2.860276in}{1.211038in}}%
\pgfpathlineto{\pgfqpoint{2.860697in}{1.211038in}}%
\pgfpathlineto{\pgfqpoint{2.861961in}{1.230741in}}%
\pgfpathlineto{\pgfqpoint{2.862804in}{1.230741in}}%
\pgfpathlineto{\pgfqpoint{2.863646in}{1.211038in}}%
\pgfpathlineto{\pgfqpoint{2.864067in}{1.237308in}}%
\pgfpathlineto{\pgfqpoint{2.864489in}{1.217606in}}%
\pgfpathlineto{\pgfqpoint{2.864910in}{1.217606in}}%
\pgfpathlineto{\pgfqpoint{2.865331in}{1.211038in}}%
\pgfpathlineto{\pgfqpoint{2.865753in}{1.217606in}}%
\pgfpathlineto{\pgfqpoint{2.867438in}{1.230741in}}%
\pgfpathlineto{\pgfqpoint{2.868280in}{1.211038in}}%
\pgfpathlineto{\pgfqpoint{2.868702in}{1.217606in}}%
\pgfpathlineto{\pgfqpoint{2.869123in}{1.224173in}}%
\pgfpathlineto{\pgfqpoint{2.869965in}{1.204471in}}%
\pgfpathlineto{\pgfqpoint{2.870387in}{1.217606in}}%
\pgfpathlineto{\pgfqpoint{2.872072in}{1.204471in}}%
\pgfpathlineto{\pgfqpoint{2.872914in}{1.230741in}}%
\pgfpathlineto{\pgfqpoint{2.874178in}{1.224173in}}%
\pgfpathlineto{\pgfqpoint{2.874599in}{1.224173in}}%
\pgfpathlineto{\pgfqpoint{2.875863in}{1.217606in}}%
\pgfpathlineto{\pgfqpoint{2.876285in}{1.217606in}}%
\pgfpathlineto{\pgfqpoint{2.877127in}{1.230741in}}%
\pgfpathlineto{\pgfqpoint{2.877548in}{1.211038in}}%
\pgfpathlineto{\pgfqpoint{2.877970in}{1.217606in}}%
\pgfpathlineto{\pgfqpoint{2.879655in}{1.230741in}}%
\pgfpathlineto{\pgfqpoint{2.880076in}{1.204471in}}%
\pgfpathlineto{\pgfqpoint{2.880919in}{1.217606in}}%
\pgfpathlineto{\pgfqpoint{2.881340in}{1.217606in}}%
\pgfpathlineto{\pgfqpoint{2.881761in}{1.237308in}}%
\pgfpathlineto{\pgfqpoint{2.882182in}{1.224173in}}%
\pgfpathlineto{\pgfqpoint{2.882604in}{1.217606in}}%
\pgfpathlineto{\pgfqpoint{2.883025in}{1.224173in}}%
\pgfpathlineto{\pgfqpoint{2.883446in}{1.230741in}}%
\pgfpathlineto{\pgfqpoint{2.883868in}{1.224173in}}%
\pgfpathlineto{\pgfqpoint{2.884289in}{1.224173in}}%
\pgfpathlineto{\pgfqpoint{2.885131in}{1.217606in}}%
\pgfpathlineto{\pgfqpoint{2.885553in}{1.230741in}}%
\pgfpathlineto{\pgfqpoint{2.885974in}{1.211038in}}%
\pgfpathlineto{\pgfqpoint{2.886395in}{1.211038in}}%
\pgfpathlineto{\pgfqpoint{2.887238in}{1.230741in}}%
\pgfpathlineto{\pgfqpoint{2.887659in}{1.217606in}}%
\pgfpathlineto{\pgfqpoint{2.888923in}{1.217606in}}%
\pgfpathlineto{\pgfqpoint{2.889344in}{1.230741in}}%
\pgfpathlineto{\pgfqpoint{2.890187in}{1.224173in}}%
\pgfpathlineto{\pgfqpoint{2.890608in}{1.224173in}}%
\pgfpathlineto{\pgfqpoint{2.891029in}{1.217606in}}%
\pgfpathlineto{\pgfqpoint{2.891451in}{1.230741in}}%
\pgfpathlineto{\pgfqpoint{2.892293in}{1.224173in}}%
\pgfpathlineto{\pgfqpoint{2.893136in}{1.224173in}}%
\pgfpathlineto{\pgfqpoint{2.894399in}{1.230741in}}%
\pgfpathlineto{\pgfqpoint{2.895242in}{1.217606in}}%
\pgfpathlineto{\pgfqpoint{2.895663in}{1.224173in}}%
\pgfpathlineto{\pgfqpoint{2.896085in}{1.237308in}}%
\pgfpathlineto{\pgfqpoint{2.896506in}{1.224173in}}%
\pgfpathlineto{\pgfqpoint{2.898612in}{1.224173in}}%
\pgfpathlineto{\pgfqpoint{2.899034in}{1.217606in}}%
\pgfpathlineto{\pgfqpoint{2.900297in}{1.237308in}}%
\pgfpathlineto{\pgfqpoint{2.901561in}{1.224173in}}%
\pgfpathlineto{\pgfqpoint{2.902404in}{1.230741in}}%
\pgfpathlineto{\pgfqpoint{2.903668in}{1.217606in}}%
\pgfpathlineto{\pgfqpoint{2.904510in}{1.237308in}}%
\pgfpathlineto{\pgfqpoint{2.904931in}{1.230741in}}%
\pgfpathlineto{\pgfqpoint{2.905353in}{1.230741in}}%
\pgfpathlineto{\pgfqpoint{2.906195in}{1.217606in}}%
\pgfpathlineto{\pgfqpoint{2.906617in}{1.224173in}}%
\pgfpathlineto{\pgfqpoint{2.907038in}{1.230741in}}%
\pgfpathlineto{\pgfqpoint{2.907880in}{1.217606in}}%
\pgfpathlineto{\pgfqpoint{2.908302in}{1.224173in}}%
\pgfpathlineto{\pgfqpoint{2.908723in}{1.217606in}}%
\pgfpathlineto{\pgfqpoint{2.909144in}{1.230741in}}%
\pgfpathlineto{\pgfqpoint{2.909987in}{1.224173in}}%
\pgfpathlineto{\pgfqpoint{2.911251in}{1.217606in}}%
\pgfpathlineto{\pgfqpoint{2.912514in}{1.230741in}}%
\pgfpathlineto{\pgfqpoint{2.914200in}{1.217606in}}%
\pgfpathlineto{\pgfqpoint{2.915042in}{1.230741in}}%
\pgfpathlineto{\pgfqpoint{2.916306in}{1.217606in}}%
\pgfpathlineto{\pgfqpoint{2.916727in}{1.217606in}}%
\pgfpathlineto{\pgfqpoint{2.917148in}{1.230741in}}%
\pgfpathlineto{\pgfqpoint{2.917991in}{1.224173in}}%
\pgfpathlineto{\pgfqpoint{2.918834in}{1.217606in}}%
\pgfpathlineto{\pgfqpoint{2.919676in}{1.230741in}}%
\pgfpathlineto{\pgfqpoint{2.920097in}{1.224173in}}%
\pgfpathlineto{\pgfqpoint{2.920519in}{1.211038in}}%
\pgfpathlineto{\pgfqpoint{2.920940in}{1.224173in}}%
\pgfpathlineto{\pgfqpoint{2.921361in}{1.224173in}}%
\pgfpathlineto{\pgfqpoint{2.921783in}{1.237308in}}%
\pgfpathlineto{\pgfqpoint{2.922204in}{1.230741in}}%
\pgfpathlineto{\pgfqpoint{2.922625in}{1.211038in}}%
\pgfpathlineto{\pgfqpoint{2.923046in}{1.230741in}}%
\pgfpathlineto{\pgfqpoint{2.923889in}{1.230741in}}%
\pgfpathlineto{\pgfqpoint{2.924310in}{1.217606in}}%
\pgfpathlineto{\pgfqpoint{2.924731in}{1.224173in}}%
\pgfpathlineto{\pgfqpoint{2.925153in}{1.230741in}}%
\pgfpathlineto{\pgfqpoint{2.925574in}{1.224173in}}%
\pgfpathlineto{\pgfqpoint{2.925995in}{1.224173in}}%
\pgfpathlineto{\pgfqpoint{2.926417in}{1.217606in}}%
\pgfpathlineto{\pgfqpoint{2.926838in}{1.224173in}}%
\pgfpathlineto{\pgfqpoint{2.927259in}{1.224173in}}%
\pgfpathlineto{\pgfqpoint{2.927680in}{1.237308in}}%
\pgfpathlineto{\pgfqpoint{2.928102in}{1.230741in}}%
\pgfpathlineto{\pgfqpoint{2.928523in}{1.211038in}}%
\pgfpathlineto{\pgfqpoint{2.929366in}{1.217606in}}%
\pgfpathlineto{\pgfqpoint{2.930208in}{1.237308in}}%
\pgfpathlineto{\pgfqpoint{2.930629in}{1.230741in}}%
\pgfpathlineto{\pgfqpoint{2.931893in}{1.224173in}}%
\pgfpathlineto{\pgfqpoint{2.932315in}{1.230741in}}%
\pgfpathlineto{\pgfqpoint{2.932736in}{1.224173in}}%
\pgfpathlineto{\pgfqpoint{2.933157in}{1.217606in}}%
\pgfpathlineto{\pgfqpoint{2.933578in}{1.224173in}}%
\pgfpathlineto{\pgfqpoint{2.934000in}{1.237308in}}%
\pgfpathlineto{\pgfqpoint{2.934842in}{1.230741in}}%
\pgfpathlineto{\pgfqpoint{2.935263in}{1.211038in}}%
\pgfpathlineto{\pgfqpoint{2.935685in}{1.224173in}}%
\pgfpathlineto{\pgfqpoint{2.936106in}{1.230741in}}%
\pgfpathlineto{\pgfqpoint{2.936527in}{1.211038in}}%
\pgfpathlineto{\pgfqpoint{2.936949in}{1.230741in}}%
\pgfpathlineto{\pgfqpoint{2.938212in}{1.224173in}}%
\pgfpathlineto{\pgfqpoint{2.939055in}{1.224173in}}%
\pgfpathlineto{\pgfqpoint{2.939476in}{1.217606in}}%
\pgfpathlineto{\pgfqpoint{2.940319in}{1.230741in}}%
\pgfpathlineto{\pgfqpoint{2.940740in}{1.224173in}}%
\pgfpathlineto{\pgfqpoint{2.941161in}{1.211038in}}%
\pgfpathlineto{\pgfqpoint{2.941583in}{1.217606in}}%
\pgfpathlineto{\pgfqpoint{2.942004in}{1.243876in}}%
\pgfpathlineto{\pgfqpoint{2.942425in}{1.224173in}}%
\pgfpathlineto{\pgfqpoint{2.942846in}{1.224173in}}%
\pgfpathlineto{\pgfqpoint{2.943268in}{1.230741in}}%
\pgfpathlineto{\pgfqpoint{2.944110in}{1.217606in}}%
\pgfpathlineto{\pgfqpoint{2.944953in}{1.230741in}}%
\pgfpathlineto{\pgfqpoint{2.945374in}{1.224173in}}%
\pgfpathlineto{\pgfqpoint{2.945795in}{1.217606in}}%
\pgfpathlineto{\pgfqpoint{2.946217in}{1.224173in}}%
\pgfpathlineto{\pgfqpoint{2.947059in}{1.230741in}}%
\pgfpathlineto{\pgfqpoint{2.947902in}{1.211038in}}%
\pgfpathlineto{\pgfqpoint{2.949166in}{1.237308in}}%
\pgfpathlineto{\pgfqpoint{2.950429in}{1.217606in}}%
\pgfpathlineto{\pgfqpoint{2.950851in}{1.237308in}}%
\pgfpathlineto{\pgfqpoint{2.951693in}{1.230741in}}%
\pgfpathlineto{\pgfqpoint{2.952536in}{1.224173in}}%
\pgfpathlineto{\pgfqpoint{2.952957in}{1.243876in}}%
\pgfpathlineto{\pgfqpoint{2.953378in}{1.230741in}}%
\pgfpathlineto{\pgfqpoint{2.954221in}{1.217606in}}%
\pgfpathlineto{\pgfqpoint{2.954642in}{1.224173in}}%
\pgfpathlineto{\pgfqpoint{2.955064in}{1.224173in}}%
\pgfpathlineto{\pgfqpoint{2.955485in}{1.211038in}}%
\pgfpathlineto{\pgfqpoint{2.955906in}{1.237308in}}%
\pgfpathlineto{\pgfqpoint{2.956327in}{1.217606in}}%
\pgfpathlineto{\pgfqpoint{2.956749in}{1.211038in}}%
\pgfpathlineto{\pgfqpoint{2.957170in}{1.217606in}}%
\pgfpathlineto{\pgfqpoint{2.957591in}{1.230741in}}%
\pgfpathlineto{\pgfqpoint{2.958012in}{1.217606in}}%
\pgfpathlineto{\pgfqpoint{2.958434in}{1.211038in}}%
\pgfpathlineto{\pgfqpoint{2.959698in}{1.230741in}}%
\pgfpathlineto{\pgfqpoint{2.960961in}{1.217606in}}%
\pgfpathlineto{\pgfqpoint{2.961383in}{1.237308in}}%
\pgfpathlineto{\pgfqpoint{2.961804in}{1.211038in}}%
\pgfpathlineto{\pgfqpoint{2.962225in}{1.230741in}}%
\pgfpathlineto{\pgfqpoint{2.962647in}{1.204471in}}%
\pgfpathlineto{\pgfqpoint{2.963068in}{1.217606in}}%
\pgfpathlineto{\pgfqpoint{2.963489in}{1.237308in}}%
\pgfpathlineto{\pgfqpoint{2.963910in}{1.224173in}}%
\pgfpathlineto{\pgfqpoint{2.964753in}{1.211038in}}%
\pgfpathlineto{\pgfqpoint{2.965174in}{1.217606in}}%
\pgfpathlineto{\pgfqpoint{2.965595in}{1.230741in}}%
\pgfpathlineto{\pgfqpoint{2.966017in}{1.224173in}}%
\pgfpathlineto{\pgfqpoint{2.966438in}{1.204471in}}%
\pgfpathlineto{\pgfqpoint{2.966859in}{1.211038in}}%
\pgfpathlineto{\pgfqpoint{2.968123in}{1.230741in}}%
\pgfpathlineto{\pgfqpoint{2.968544in}{1.237308in}}%
\pgfpathlineto{\pgfqpoint{2.968966in}{1.230741in}}%
\pgfpathlineto{\pgfqpoint{2.969808in}{1.224173in}}%
\pgfpathlineto{\pgfqpoint{2.971072in}{1.230741in}}%
\pgfpathlineto{\pgfqpoint{2.972336in}{1.224173in}}%
\pgfpathlineto{\pgfqpoint{2.972757in}{1.230741in}}%
\pgfpathlineto{\pgfqpoint{2.973178in}{1.224173in}}%
\pgfpathlineto{\pgfqpoint{2.973600in}{1.224173in}}%
\pgfpathlineto{\pgfqpoint{2.974021in}{1.217606in}}%
\pgfpathlineto{\pgfqpoint{2.975285in}{1.237308in}}%
\pgfpathlineto{\pgfqpoint{2.975706in}{1.217606in}}%
\pgfpathlineto{\pgfqpoint{2.976127in}{1.230741in}}%
\pgfpathlineto{\pgfqpoint{2.976549in}{1.237308in}}%
\pgfpathlineto{\pgfqpoint{2.976970in}{1.211038in}}%
\pgfpathlineto{\pgfqpoint{2.977391in}{1.224173in}}%
\pgfpathlineto{\pgfqpoint{2.978655in}{1.237308in}}%
\pgfpathlineto{\pgfqpoint{2.979076in}{1.211038in}}%
\pgfpathlineto{\pgfqpoint{2.979498in}{1.224173in}}%
\pgfpathlineto{\pgfqpoint{2.980340in}{1.230741in}}%
\pgfpathlineto{\pgfqpoint{2.980761in}{1.217606in}}%
\pgfpathlineto{\pgfqpoint{2.981604in}{1.224173in}}%
\pgfpathlineto{\pgfqpoint{2.982025in}{1.217606in}}%
\pgfpathlineto{\pgfqpoint{2.983289in}{1.237308in}}%
\pgfpathlineto{\pgfqpoint{2.984553in}{1.211038in}}%
\pgfpathlineto{\pgfqpoint{2.985396in}{1.230741in}}%
\pgfpathlineto{\pgfqpoint{2.987081in}{1.211038in}}%
\pgfpathlineto{\pgfqpoint{2.988344in}{1.224173in}}%
\pgfpathlineto{\pgfqpoint{2.988766in}{1.211038in}}%
\pgfpathlineto{\pgfqpoint{2.989187in}{1.237308in}}%
\pgfpathlineto{\pgfqpoint{2.989608in}{1.217606in}}%
\pgfpathlineto{\pgfqpoint{2.990030in}{1.217606in}}%
\pgfpathlineto{\pgfqpoint{2.990451in}{1.224173in}}%
\pgfpathlineto{\pgfqpoint{2.990872in}{1.211038in}}%
\pgfpathlineto{\pgfqpoint{2.991293in}{1.217606in}}%
\pgfpathlineto{\pgfqpoint{2.992557in}{1.230741in}}%
\pgfpathlineto{\pgfqpoint{2.993400in}{1.217606in}}%
\pgfpathlineto{\pgfqpoint{2.994242in}{1.230741in}}%
\pgfpathlineto{\pgfqpoint{2.994664in}{1.211038in}}%
\pgfpathlineto{\pgfqpoint{2.995085in}{1.217606in}}%
\pgfpathlineto{\pgfqpoint{2.995506in}{1.224173in}}%
\pgfpathlineto{\pgfqpoint{2.995928in}{1.217606in}}%
\pgfpathlineto{\pgfqpoint{2.996770in}{1.211038in}}%
\pgfpathlineto{\pgfqpoint{2.998034in}{1.224173in}}%
\pgfpathlineto{\pgfqpoint{2.998876in}{1.211038in}}%
\pgfpathlineto{\pgfqpoint{2.999298in}{1.217606in}}%
\pgfpathlineto{\pgfqpoint{3.000562in}{1.230741in}}%
\pgfpathlineto{\pgfqpoint{3.000983in}{1.217606in}}%
\pgfpathlineto{\pgfqpoint{3.001825in}{1.224173in}}%
\pgfpathlineto{\pgfqpoint{3.002247in}{1.224173in}}%
\pgfpathlineto{\pgfqpoint{3.002668in}{1.211038in}}%
\pgfpathlineto{\pgfqpoint{3.003089in}{1.230741in}}%
\pgfpathlineto{\pgfqpoint{3.003511in}{1.230741in}}%
\pgfpathlineto{\pgfqpoint{3.003932in}{1.243876in}}%
\pgfpathlineto{\pgfqpoint{3.004353in}{1.381794in}}%
\pgfpathlineto{\pgfqpoint{3.004774in}{1.230741in}}%
\pgfpathlineto{\pgfqpoint{3.005196in}{1.237308in}}%
\pgfpathlineto{\pgfqpoint{3.005617in}{1.224173in}}%
\pgfpathlineto{\pgfqpoint{3.006459in}{1.230741in}}%
\pgfpathlineto{\pgfqpoint{3.007302in}{1.237308in}}%
\pgfpathlineto{\pgfqpoint{3.007723in}{1.230741in}}%
\pgfpathlineto{\pgfqpoint{3.008145in}{1.243876in}}%
\pgfpathlineto{\pgfqpoint{3.008987in}{1.237308in}}%
\pgfpathlineto{\pgfqpoint{3.009408in}{1.224173in}}%
\pgfpathlineto{\pgfqpoint{3.009830in}{1.243876in}}%
\pgfpathlineto{\pgfqpoint{3.010672in}{1.224173in}}%
\pgfpathlineto{\pgfqpoint{3.011515in}{1.230741in}}%
\pgfpathlineto{\pgfqpoint{3.012779in}{1.250443in}}%
\pgfpathlineto{\pgfqpoint{3.013200in}{1.230741in}}%
\pgfpathlineto{\pgfqpoint{3.013621in}{1.243876in}}%
\pgfpathlineto{\pgfqpoint{3.014042in}{1.243876in}}%
\pgfpathlineto{\pgfqpoint{3.015306in}{1.230741in}}%
\pgfpathlineto{\pgfqpoint{3.016149in}{1.243876in}}%
\pgfpathlineto{\pgfqpoint{3.016570in}{1.237308in}}%
\pgfpathlineto{\pgfqpoint{3.017413in}{1.211038in}}%
\pgfpathlineto{\pgfqpoint{3.017834in}{1.086255in}}%
\pgfpathlineto{\pgfqpoint{3.018255in}{1.217606in}}%
\pgfpathlineto{\pgfqpoint{3.018677in}{1.217606in}}%
\pgfpathlineto{\pgfqpoint{3.019940in}{1.230741in}}%
\pgfpathlineto{\pgfqpoint{3.021204in}{1.217606in}}%
\pgfpathlineto{\pgfqpoint{3.022047in}{1.230741in}}%
\pgfpathlineto{\pgfqpoint{3.022468in}{1.224173in}}%
\pgfpathlineto{\pgfqpoint{3.022889in}{1.224173in}}%
\pgfpathlineto{\pgfqpoint{3.023311in}{1.217606in}}%
\pgfpathlineto{\pgfqpoint{3.023732in}{1.230741in}}%
\pgfpathlineto{\pgfqpoint{3.024153in}{1.224173in}}%
\pgfpathlineto{\pgfqpoint{3.024574in}{1.211038in}}%
\pgfpathlineto{\pgfqpoint{3.024996in}{1.217606in}}%
\pgfpathlineto{\pgfqpoint{3.025838in}{1.250443in}}%
\pgfpathlineto{\pgfqpoint{3.026260in}{1.224173in}}%
\pgfpathlineto{\pgfqpoint{3.026681in}{1.224173in}}%
\pgfpathlineto{\pgfqpoint{3.027945in}{1.230741in}}%
\pgfpathlineto{\pgfqpoint{3.028366in}{1.217606in}}%
\pgfpathlineto{\pgfqpoint{3.028787in}{1.243876in}}%
\pgfpathlineto{\pgfqpoint{3.029208in}{1.237308in}}%
\pgfpathlineto{\pgfqpoint{3.029630in}{1.211038in}}%
\pgfpathlineto{\pgfqpoint{3.030472in}{1.224173in}}%
\pgfpathlineto{\pgfqpoint{3.031736in}{1.237308in}}%
\pgfpathlineto{\pgfqpoint{3.033000in}{1.224173in}}%
\pgfpathlineto{\pgfqpoint{3.033843in}{1.224173in}}%
\pgfpathlineto{\pgfqpoint{3.034264in}{1.217606in}}%
\pgfpathlineto{\pgfqpoint{3.034685in}{1.230741in}}%
\pgfpathlineto{\pgfqpoint{3.035528in}{1.224173in}}%
\pgfpathlineto{\pgfqpoint{3.035949in}{1.217606in}}%
\pgfpathlineto{\pgfqpoint{3.036791in}{1.230741in}}%
\pgfpathlineto{\pgfqpoint{3.037213in}{1.217606in}}%
\pgfpathlineto{\pgfqpoint{3.037634in}{1.224173in}}%
\pgfpathlineto{\pgfqpoint{3.038477in}{1.237308in}}%
\pgfpathlineto{\pgfqpoint{3.039319in}{1.211038in}}%
\pgfpathlineto{\pgfqpoint{3.039740in}{1.224173in}}%
\pgfpathlineto{\pgfqpoint{3.040162in}{1.224173in}}%
\pgfpathlineto{\pgfqpoint{3.040583in}{1.211038in}}%
\pgfpathlineto{\pgfqpoint{3.041004in}{1.217606in}}%
\pgfpathlineto{\pgfqpoint{3.041847in}{1.230741in}}%
\pgfpathlineto{\pgfqpoint{3.042268in}{1.211038in}}%
\pgfpathlineto{\pgfqpoint{3.042689in}{1.217606in}}%
\pgfpathlineto{\pgfqpoint{3.043111in}{1.237308in}}%
\pgfpathlineto{\pgfqpoint{3.043532in}{1.211038in}}%
\pgfpathlineto{\pgfqpoint{3.044796in}{1.230741in}}%
\pgfpathlineto{\pgfqpoint{3.046060in}{1.224173in}}%
\pgfpathlineto{\pgfqpoint{3.046902in}{1.224173in}}%
\pgfpathlineto{\pgfqpoint{3.047323in}{1.257011in}}%
\pgfpathlineto{\pgfqpoint{3.047745in}{1.230741in}}%
\pgfpathlineto{\pgfqpoint{3.048587in}{1.217606in}}%
\pgfpathlineto{\pgfqpoint{3.049009in}{1.230741in}}%
\pgfpathlineto{\pgfqpoint{3.049430in}{1.224173in}}%
\pgfpathlineto{\pgfqpoint{3.049851in}{1.217606in}}%
\pgfpathlineto{\pgfqpoint{3.051115in}{1.230741in}}%
\pgfpathlineto{\pgfqpoint{3.051957in}{1.217606in}}%
\pgfpathlineto{\pgfqpoint{3.052379in}{1.224173in}}%
\pgfpathlineto{\pgfqpoint{3.054064in}{1.224173in}}%
\pgfpathlineto{\pgfqpoint{3.054906in}{1.204471in}}%
\pgfpathlineto{\pgfqpoint{3.055328in}{1.119092in}}%
\pgfpathlineto{\pgfqpoint{3.055749in}{1.217606in}}%
\pgfpathlineto{\pgfqpoint{3.057434in}{1.230741in}}%
\pgfpathlineto{\pgfqpoint{3.058698in}{1.217606in}}%
\pgfpathlineto{\pgfqpoint{3.059540in}{1.224173in}}%
\pgfpathlineto{\pgfqpoint{3.059962in}{1.217606in}}%
\pgfpathlineto{\pgfqpoint{3.060383in}{1.224173in}}%
\pgfpathlineto{\pgfqpoint{3.060804in}{1.158498in}}%
\pgfpathlineto{\pgfqpoint{3.061226in}{1.224173in}}%
\pgfpathlineto{\pgfqpoint{3.062911in}{1.211038in}}%
\pgfpathlineto{\pgfqpoint{3.064175in}{1.224173in}}%
\pgfpathlineto{\pgfqpoint{3.064596in}{1.224173in}}%
\pgfpathlineto{\pgfqpoint{3.065017in}{1.217606in}}%
\pgfpathlineto{\pgfqpoint{3.065438in}{1.224173in}}%
\pgfpathlineto{\pgfqpoint{3.067124in}{1.224173in}}%
\pgfpathlineto{\pgfqpoint{3.068387in}{1.237308in}}%
\pgfpathlineto{\pgfqpoint{3.068809in}{1.322686in}}%
\pgfpathlineto{\pgfqpoint{3.069230in}{1.230741in}}%
\pgfpathlineto{\pgfqpoint{3.069651in}{1.217606in}}%
\pgfpathlineto{\pgfqpoint{3.070494in}{1.224173in}}%
\pgfpathlineto{\pgfqpoint{3.071336in}{1.224173in}}%
\pgfpathlineto{\pgfqpoint{3.072600in}{1.217606in}}%
\pgfpathlineto{\pgfqpoint{3.073443in}{1.224173in}}%
\pgfpathlineto{\pgfqpoint{3.073864in}{1.217606in}}%
\pgfpathlineto{\pgfqpoint{3.074285in}{1.270146in}}%
\pgfpathlineto{\pgfqpoint{3.074707in}{1.211038in}}%
\pgfpathlineto{\pgfqpoint{3.076392in}{1.243876in}}%
\pgfpathlineto{\pgfqpoint{3.077234in}{1.224173in}}%
\pgfpathlineto{\pgfqpoint{3.077655in}{1.243876in}}%
\pgfpathlineto{\pgfqpoint{3.078077in}{1.224173in}}%
\pgfpathlineto{\pgfqpoint{3.078919in}{1.217606in}}%
\pgfpathlineto{\pgfqpoint{3.079762in}{1.230741in}}%
\pgfpathlineto{\pgfqpoint{3.080183in}{1.217606in}}%
\pgfpathlineto{\pgfqpoint{3.081026in}{1.224173in}}%
\pgfpathlineto{\pgfqpoint{3.081447in}{1.224173in}}%
\pgfpathlineto{\pgfqpoint{3.081868in}{1.230741in}}%
\pgfpathlineto{\pgfqpoint{3.082290in}{1.224173in}}%
\pgfpathlineto{\pgfqpoint{3.082711in}{1.224173in}}%
\pgfpathlineto{\pgfqpoint{3.083132in}{1.230741in}}%
\pgfpathlineto{\pgfqpoint{3.084396in}{1.217606in}}%
\pgfpathlineto{\pgfqpoint{3.084817in}{1.217606in}}%
\pgfpathlineto{\pgfqpoint{3.085660in}{1.237308in}}%
\pgfpathlineto{\pgfqpoint{3.086081in}{1.224173in}}%
\pgfpathlineto{\pgfqpoint{3.086502in}{1.224173in}}%
\pgfpathlineto{\pgfqpoint{3.086924in}{1.230741in}}%
\pgfpathlineto{\pgfqpoint{3.087345in}{1.224173in}}%
\pgfpathlineto{\pgfqpoint{3.087766in}{1.217606in}}%
\pgfpathlineto{\pgfqpoint{3.088187in}{1.237308in}}%
\pgfpathlineto{\pgfqpoint{3.088609in}{1.224173in}}%
\pgfpathlineto{\pgfqpoint{3.089030in}{1.217606in}}%
\pgfpathlineto{\pgfqpoint{3.089451in}{1.224173in}}%
\pgfpathlineto{\pgfqpoint{3.090715in}{1.224173in}}%
\pgfpathlineto{\pgfqpoint{3.091136in}{1.211038in}}%
\pgfpathlineto{\pgfqpoint{3.091558in}{1.224173in}}%
\pgfpathlineto{\pgfqpoint{3.091979in}{1.237308in}}%
\pgfpathlineto{\pgfqpoint{3.092400in}{1.230741in}}%
\pgfpathlineto{\pgfqpoint{3.093243in}{1.217606in}}%
\pgfpathlineto{\pgfqpoint{3.094085in}{1.230741in}}%
\pgfpathlineto{\pgfqpoint{3.094507in}{1.224173in}}%
\pgfpathlineto{\pgfqpoint{3.095349in}{1.217606in}}%
\pgfpathlineto{\pgfqpoint{3.095770in}{1.230741in}}%
\pgfpathlineto{\pgfqpoint{3.096192in}{1.211038in}}%
\pgfpathlineto{\pgfqpoint{3.096613in}{1.224173in}}%
\pgfpathlineto{\pgfqpoint{3.097456in}{1.230741in}}%
\pgfpathlineto{\pgfqpoint{3.098298in}{1.224173in}}%
\pgfpathlineto{\pgfqpoint{3.098719in}{1.230741in}}%
\pgfpathlineto{\pgfqpoint{3.099983in}{1.211038in}}%
\pgfpathlineto{\pgfqpoint{3.102511in}{1.250443in}}%
\pgfpathlineto{\pgfqpoint{3.102932in}{1.217606in}}%
\pgfpathlineto{\pgfqpoint{3.103775in}{1.224173in}}%
\pgfpathlineto{\pgfqpoint{3.104196in}{1.217606in}}%
\pgfpathlineto{\pgfqpoint{3.104617in}{1.230741in}}%
\pgfpathlineto{\pgfqpoint{3.105460in}{1.224173in}}%
\pgfpathlineto{\pgfqpoint{3.105881in}{1.224173in}}%
\pgfpathlineto{\pgfqpoint{3.106724in}{1.230741in}}%
\pgfpathlineto{\pgfqpoint{3.107987in}{1.224173in}}%
\pgfpathlineto{\pgfqpoint{3.108830in}{1.237308in}}%
\pgfpathlineto{\pgfqpoint{3.110094in}{1.217606in}}%
\pgfpathlineto{\pgfqpoint{3.110936in}{1.224173in}}%
\pgfpathlineto{\pgfqpoint{3.112200in}{1.217606in}}%
\pgfpathlineto{\pgfqpoint{3.113464in}{1.237308in}}%
\pgfpathlineto{\pgfqpoint{3.113885in}{1.230741in}}%
\pgfpathlineto{\pgfqpoint{3.114307in}{1.217606in}}%
\pgfpathlineto{\pgfqpoint{3.115149in}{1.224173in}}%
\pgfpathlineto{\pgfqpoint{3.115570in}{1.224173in}}%
\pgfpathlineto{\pgfqpoint{3.115992in}{1.211038in}}%
\pgfpathlineto{\pgfqpoint{3.116413in}{1.230741in}}%
\pgfpathlineto{\pgfqpoint{3.117256in}{1.217606in}}%
\pgfpathlineto{\pgfqpoint{3.117677in}{1.237308in}}%
\pgfpathlineto{\pgfqpoint{3.118519in}{1.230741in}}%
\pgfpathlineto{\pgfqpoint{3.120205in}{1.211038in}}%
\pgfpathlineto{\pgfqpoint{3.121047in}{1.230741in}}%
\pgfpathlineto{\pgfqpoint{3.121890in}{1.224173in}}%
\pgfpathlineto{\pgfqpoint{3.122311in}{1.224173in}}%
\pgfpathlineto{\pgfqpoint{3.123575in}{1.230741in}}%
\pgfpathlineto{\pgfqpoint{3.124417in}{1.217606in}}%
\pgfpathlineto{\pgfqpoint{3.124839in}{1.237308in}}%
\pgfpathlineto{\pgfqpoint{3.125681in}{1.230741in}}%
\pgfpathlineto{\pgfqpoint{3.126945in}{1.217606in}}%
\pgfpathlineto{\pgfqpoint{3.128209in}{1.224173in}}%
\pgfpathlineto{\pgfqpoint{3.128630in}{1.211038in}}%
\pgfpathlineto{\pgfqpoint{3.129051in}{1.237308in}}%
\pgfpathlineto{\pgfqpoint{3.129473in}{1.217606in}}%
\pgfpathlineto{\pgfqpoint{3.129894in}{1.217606in}}%
\pgfpathlineto{\pgfqpoint{3.130737in}{1.224173in}}%
\pgfpathlineto{\pgfqpoint{3.131158in}{1.211038in}}%
\pgfpathlineto{\pgfqpoint{3.131579in}{1.224173in}}%
\pgfpathlineto{\pgfqpoint{3.132422in}{1.224173in}}%
\pgfpathlineto{\pgfqpoint{3.133264in}{1.230741in}}%
\pgfpathlineto{\pgfqpoint{3.134528in}{1.211038in}}%
\pgfpathlineto{\pgfqpoint{3.134949in}{1.230741in}}%
\pgfpathlineto{\pgfqpoint{3.135371in}{1.224173in}}%
\pgfpathlineto{\pgfqpoint{3.135792in}{1.211038in}}%
\pgfpathlineto{\pgfqpoint{3.136213in}{1.217606in}}%
\pgfpathlineto{\pgfqpoint{3.137056in}{1.230741in}}%
\pgfpathlineto{\pgfqpoint{3.137477in}{1.224173in}}%
\pgfpathlineto{\pgfqpoint{3.137898in}{1.224173in}}%
\pgfpathlineto{\pgfqpoint{3.138320in}{1.230741in}}%
\pgfpathlineto{\pgfqpoint{3.138741in}{1.224173in}}%
\pgfpathlineto{\pgfqpoint{3.139162in}{1.224173in}}%
\pgfpathlineto{\pgfqpoint{3.139583in}{1.217606in}}%
\pgfpathlineto{\pgfqpoint{3.140005in}{1.224173in}}%
\pgfpathlineto{\pgfqpoint{3.141268in}{1.230741in}}%
\pgfpathlineto{\pgfqpoint{3.142111in}{1.237308in}}%
\pgfpathlineto{\pgfqpoint{3.142532in}{1.204471in}}%
\pgfpathlineto{\pgfqpoint{3.143375in}{1.237308in}}%
\pgfpathlineto{\pgfqpoint{3.143796in}{1.224173in}}%
\pgfpathlineto{\pgfqpoint{3.144639in}{1.230741in}}%
\pgfpathlineto{\pgfqpoint{3.145481in}{1.217606in}}%
\pgfpathlineto{\pgfqpoint{3.145903in}{1.224173in}}%
\pgfpathlineto{\pgfqpoint{3.146324in}{1.224173in}}%
\pgfpathlineto{\pgfqpoint{3.146745in}{1.217606in}}%
\pgfpathlineto{\pgfqpoint{3.148009in}{1.230741in}}%
\pgfpathlineto{\pgfqpoint{3.148851in}{1.217606in}}%
\pgfpathlineto{\pgfqpoint{3.149273in}{1.230741in}}%
\pgfpathlineto{\pgfqpoint{3.150115in}{1.224173in}}%
\pgfpathlineto{\pgfqpoint{3.150537in}{1.204471in}}%
\pgfpathlineto{\pgfqpoint{3.150958in}{1.224173in}}%
\pgfpathlineto{\pgfqpoint{3.151379in}{1.224173in}}%
\pgfpathlineto{\pgfqpoint{3.151800in}{1.211038in}}%
\pgfpathlineto{\pgfqpoint{3.152643in}{1.217606in}}%
\pgfpathlineto{\pgfqpoint{3.153486in}{1.217606in}}%
\pgfpathlineto{\pgfqpoint{3.154749in}{1.204471in}}%
\pgfpathlineto{\pgfqpoint{3.156013in}{1.217606in}}%
\pgfpathlineto{\pgfqpoint{3.156856in}{1.204471in}}%
\pgfpathlineto{\pgfqpoint{3.158120in}{1.224173in}}%
\pgfpathlineto{\pgfqpoint{3.158541in}{1.211038in}}%
\pgfpathlineto{\pgfqpoint{3.158962in}{1.224173in}}%
\pgfpathlineto{\pgfqpoint{3.159383in}{1.224173in}}%
\pgfpathlineto{\pgfqpoint{3.161069in}{1.204471in}}%
\pgfpathlineto{\pgfqpoint{3.163175in}{1.224173in}}%
\pgfpathlineto{\pgfqpoint{3.163596in}{1.217606in}}%
\pgfpathlineto{\pgfqpoint{3.164017in}{1.230741in}}%
\pgfpathlineto{\pgfqpoint{3.164439in}{1.217606in}}%
\pgfpathlineto{\pgfqpoint{3.164860in}{1.211038in}}%
\pgfpathlineto{\pgfqpoint{3.165281in}{1.230741in}}%
\pgfpathlineto{\pgfqpoint{3.166124in}{1.224173in}}%
\pgfpathlineto{\pgfqpoint{3.166545in}{1.224173in}}%
\pgfpathlineto{\pgfqpoint{3.167809in}{1.217606in}}%
\pgfpathlineto{\pgfqpoint{3.168230in}{1.237308in}}%
\pgfpathlineto{\pgfqpoint{3.169073in}{1.230741in}}%
\pgfpathlineto{\pgfqpoint{3.169494in}{1.237308in}}%
\pgfpathlineto{\pgfqpoint{3.169915in}{1.224173in}}%
\pgfpathlineto{\pgfqpoint{3.170758in}{1.230741in}}%
\pgfpathlineto{\pgfqpoint{3.171600in}{1.224173in}}%
\pgfpathlineto{\pgfqpoint{3.172022in}{1.237308in}}%
\pgfpathlineto{\pgfqpoint{3.172443in}{1.224173in}}%
\pgfpathlineto{\pgfqpoint{3.172864in}{1.211038in}}%
\pgfpathlineto{\pgfqpoint{3.173286in}{1.224173in}}%
\pgfpathlineto{\pgfqpoint{3.173707in}{1.237308in}}%
\pgfpathlineto{\pgfqpoint{3.174128in}{1.224173in}}%
\pgfpathlineto{\pgfqpoint{3.174549in}{1.224173in}}%
\pgfpathlineto{\pgfqpoint{3.175392in}{1.230741in}}%
\pgfpathlineto{\pgfqpoint{3.176235in}{1.217606in}}%
\pgfpathlineto{\pgfqpoint{3.176656in}{1.230741in}}%
\pgfpathlineto{\pgfqpoint{3.177498in}{1.224173in}}%
\pgfpathlineto{\pgfqpoint{3.177920in}{1.224173in}}%
\pgfpathlineto{\pgfqpoint{3.178762in}{1.217606in}}%
\pgfpathlineto{\pgfqpoint{3.179183in}{1.237308in}}%
\pgfpathlineto{\pgfqpoint{3.180447in}{1.224173in}}%
\pgfpathlineto{\pgfqpoint{3.180869in}{1.224173in}}%
\pgfpathlineto{\pgfqpoint{3.181290in}{1.211038in}}%
\pgfpathlineto{\pgfqpoint{3.181711in}{1.224173in}}%
\pgfpathlineto{\pgfqpoint{3.182132in}{1.237308in}}%
\pgfpathlineto{\pgfqpoint{3.182554in}{1.230741in}}%
\pgfpathlineto{\pgfqpoint{3.182975in}{1.211038in}}%
\pgfpathlineto{\pgfqpoint{3.183396in}{1.230741in}}%
\pgfpathlineto{\pgfqpoint{3.184239in}{1.224173in}}%
\pgfpathlineto{\pgfqpoint{3.184660in}{1.230741in}}%
\pgfpathlineto{\pgfqpoint{3.185081in}{1.211038in}}%
\pgfpathlineto{\pgfqpoint{3.185503in}{1.224173in}}%
\pgfpathlineto{\pgfqpoint{3.185924in}{1.224173in}}%
\pgfpathlineto{\pgfqpoint{3.186766in}{1.230741in}}%
\pgfpathlineto{\pgfqpoint{3.187609in}{1.224173in}}%
\pgfpathlineto{\pgfqpoint{3.188030in}{1.230741in}}%
\pgfpathlineto{\pgfqpoint{3.188452in}{1.224173in}}%
\pgfpathlineto{\pgfqpoint{3.188873in}{1.224173in}}%
\pgfpathlineto{\pgfqpoint{3.189294in}{1.230741in}}%
\pgfpathlineto{\pgfqpoint{3.189715in}{1.224173in}}%
\pgfpathlineto{\pgfqpoint{3.190137in}{1.211038in}}%
\pgfpathlineto{\pgfqpoint{3.190558in}{1.230741in}}%
\pgfpathlineto{\pgfqpoint{3.191401in}{1.217606in}}%
\pgfpathlineto{\pgfqpoint{3.191822in}{1.224173in}}%
\pgfpathlineto{\pgfqpoint{3.192243in}{1.224173in}}%
\pgfpathlineto{\pgfqpoint{3.192664in}{1.217606in}}%
\pgfpathlineto{\pgfqpoint{3.193928in}{1.230741in}}%
\pgfpathlineto{\pgfqpoint{3.194350in}{1.224173in}}%
\pgfpathlineto{\pgfqpoint{3.195192in}{1.243876in}}%
\pgfpathlineto{\pgfqpoint{3.196035in}{1.217606in}}%
\pgfpathlineto{\pgfqpoint{3.196456in}{1.224173in}}%
\pgfpathlineto{\pgfqpoint{3.196877in}{1.224173in}}%
\pgfpathlineto{\pgfqpoint{3.198141in}{1.217606in}}%
\pgfpathlineto{\pgfqpoint{3.198562in}{1.237308in}}%
\pgfpathlineto{\pgfqpoint{3.198984in}{1.211038in}}%
\pgfpathlineto{\pgfqpoint{3.200669in}{1.224173in}}%
\pgfpathlineto{\pgfqpoint{3.201511in}{1.224173in}}%
\pgfpathlineto{\pgfqpoint{3.202354in}{1.217606in}}%
\pgfpathlineto{\pgfqpoint{3.203618in}{1.230741in}}%
\pgfpathlineto{\pgfqpoint{3.205303in}{1.211038in}}%
\pgfpathlineto{\pgfqpoint{3.206145in}{1.224173in}}%
\pgfpathlineto{\pgfqpoint{3.206567in}{1.217606in}}%
\pgfpathlineto{\pgfqpoint{3.206988in}{1.217606in}}%
\pgfpathlineto{\pgfqpoint{3.207830in}{1.230741in}}%
\pgfpathlineto{\pgfqpoint{3.208252in}{1.224173in}}%
\pgfpathlineto{\pgfqpoint{3.208673in}{1.204471in}}%
\pgfpathlineto{\pgfqpoint{3.209094in}{1.217606in}}%
\pgfpathlineto{\pgfqpoint{3.209937in}{1.230741in}}%
\pgfpathlineto{\pgfqpoint{3.210358in}{1.211038in}}%
\pgfpathlineto{\pgfqpoint{3.211201in}{1.217606in}}%
\pgfpathlineto{\pgfqpoint{3.212043in}{1.217606in}}%
\pgfpathlineto{\pgfqpoint{3.212886in}{1.230741in}}%
\pgfpathlineto{\pgfqpoint{3.213307in}{1.217606in}}%
\pgfpathlineto{\pgfqpoint{3.214150in}{1.224173in}}%
\pgfpathlineto{\pgfqpoint{3.214571in}{1.230741in}}%
\pgfpathlineto{\pgfqpoint{3.214992in}{1.224173in}}%
\pgfpathlineto{\pgfqpoint{3.217099in}{1.224173in}}%
\pgfpathlineto{\pgfqpoint{3.217520in}{1.217606in}}%
\pgfpathlineto{\pgfqpoint{3.218784in}{1.243876in}}%
\pgfpathlineto{\pgfqpoint{3.220047in}{1.217606in}}%
\pgfpathlineto{\pgfqpoint{3.220469in}{1.224173in}}%
\pgfpathlineto{\pgfqpoint{3.221733in}{1.224173in}}%
\pgfpathlineto{\pgfqpoint{3.222154in}{1.230741in}}%
\pgfpathlineto{\pgfqpoint{3.223418in}{1.217606in}}%
\pgfpathlineto{\pgfqpoint{3.223839in}{1.230741in}}%
\pgfpathlineto{\pgfqpoint{3.224682in}{1.224173in}}%
\pgfpathlineto{\pgfqpoint{3.225524in}{1.230741in}}%
\pgfpathlineto{\pgfqpoint{3.226367in}{1.224173in}}%
\pgfpathlineto{\pgfqpoint{3.226788in}{1.230741in}}%
\pgfpathlineto{\pgfqpoint{3.228052in}{1.211038in}}%
\pgfpathlineto{\pgfqpoint{3.230158in}{1.230741in}}%
\pgfpathlineto{\pgfqpoint{3.230579in}{1.211038in}}%
\pgfpathlineto{\pgfqpoint{3.231422in}{1.217606in}}%
\pgfpathlineto{\pgfqpoint{3.231843in}{1.224173in}}%
\pgfpathlineto{\pgfqpoint{3.232265in}{1.217606in}}%
\pgfpathlineto{\pgfqpoint{3.232686in}{1.204471in}}%
\pgfpathlineto{\pgfqpoint{3.233107in}{1.230741in}}%
\pgfpathlineto{\pgfqpoint{3.233950in}{1.224173in}}%
\pgfpathlineto{\pgfqpoint{3.234371in}{1.224173in}}%
\pgfpathlineto{\pgfqpoint{3.235213in}{1.217606in}}%
\pgfpathlineto{\pgfqpoint{3.235635in}{1.237308in}}%
\pgfpathlineto{\pgfqpoint{3.236477in}{1.230741in}}%
\pgfpathlineto{\pgfqpoint{3.237741in}{1.224173in}}%
\pgfpathlineto{\pgfqpoint{3.238162in}{1.224173in}}%
\pgfpathlineto{\pgfqpoint{3.239005in}{1.211038in}}%
\pgfpathlineto{\pgfqpoint{3.239426in}{1.230741in}}%
\pgfpathlineto{\pgfqpoint{3.240269in}{1.224173in}}%
\pgfpathlineto{\pgfqpoint{3.241111in}{1.230741in}}%
\pgfpathlineto{\pgfqpoint{3.241533in}{1.224173in}}%
\pgfpathlineto{\pgfqpoint{3.241954in}{1.237308in}}%
\pgfpathlineto{\pgfqpoint{3.242375in}{1.217606in}}%
\pgfpathlineto{\pgfqpoint{3.242796in}{1.217606in}}%
\pgfpathlineto{\pgfqpoint{3.244060in}{1.237308in}}%
\pgfpathlineto{\pgfqpoint{3.244482in}{1.230741in}}%
\pgfpathlineto{\pgfqpoint{3.244903in}{1.230741in}}%
\pgfpathlineto{\pgfqpoint{3.245745in}{1.211038in}}%
\pgfpathlineto{\pgfqpoint{3.246167in}{1.243876in}}%
\pgfpathlineto{\pgfqpoint{3.247009in}{1.230741in}}%
\pgfpathlineto{\pgfqpoint{3.247852in}{1.224173in}}%
\pgfpathlineto{\pgfqpoint{3.248694in}{1.230741in}}%
\pgfpathlineto{\pgfqpoint{3.249116in}{1.197903in}}%
\pgfpathlineto{\pgfqpoint{3.249537in}{1.224173in}}%
\pgfpathlineto{\pgfqpoint{3.249958in}{1.224173in}}%
\pgfpathlineto{\pgfqpoint{3.250379in}{1.230741in}}%
\pgfpathlineto{\pgfqpoint{3.250801in}{1.224173in}}%
\pgfpathlineto{\pgfqpoint{3.251222in}{1.217606in}}%
\pgfpathlineto{\pgfqpoint{3.251643in}{1.224173in}}%
\pgfpathlineto{\pgfqpoint{3.252065in}{1.224173in}}%
\pgfpathlineto{\pgfqpoint{3.252486in}{1.243876in}}%
\pgfpathlineto{\pgfqpoint{3.252907in}{1.217606in}}%
\pgfpathlineto{\pgfqpoint{3.253328in}{1.224173in}}%
\pgfpathlineto{\pgfqpoint{3.253750in}{1.217606in}}%
\pgfpathlineto{\pgfqpoint{3.254171in}{1.211038in}}%
\pgfpathlineto{\pgfqpoint{3.255014in}{1.224173in}}%
\pgfpathlineto{\pgfqpoint{3.255435in}{1.204471in}}%
\pgfpathlineto{\pgfqpoint{3.255856in}{1.224173in}}%
\pgfpathlineto{\pgfqpoint{3.256277in}{1.230741in}}%
\pgfpathlineto{\pgfqpoint{3.256699in}{1.224173in}}%
\pgfpathlineto{\pgfqpoint{3.257120in}{1.224173in}}%
\pgfpathlineto{\pgfqpoint{3.257541in}{1.217606in}}%
\pgfpathlineto{\pgfqpoint{3.257963in}{1.224173in}}%
\pgfpathlineto{\pgfqpoint{3.258384in}{1.224173in}}%
\pgfpathlineto{\pgfqpoint{3.258805in}{1.237308in}}%
\pgfpathlineto{\pgfqpoint{3.259226in}{1.230741in}}%
\pgfpathlineto{\pgfqpoint{3.259648in}{1.217606in}}%
\pgfpathlineto{\pgfqpoint{3.260490in}{1.224173in}}%
\pgfpathlineto{\pgfqpoint{3.260911in}{1.224173in}}%
\pgfpathlineto{\pgfqpoint{3.261754in}{1.211038in}}%
\pgfpathlineto{\pgfqpoint{3.262175in}{1.217606in}}%
\pgfpathlineto{\pgfqpoint{3.263439in}{1.230741in}}%
\pgfpathlineto{\pgfqpoint{3.264282in}{1.224173in}}%
\pgfpathlineto{\pgfqpoint{3.265546in}{1.230741in}}%
\pgfpathlineto{\pgfqpoint{3.265967in}{1.211038in}}%
\pgfpathlineto{\pgfqpoint{3.266388in}{1.230741in}}%
\pgfpathlineto{\pgfqpoint{3.266809in}{1.230741in}}%
\pgfpathlineto{\pgfqpoint{3.267231in}{1.224173in}}%
\pgfpathlineto{\pgfqpoint{3.267652in}{1.237308in}}%
\pgfpathlineto{\pgfqpoint{3.268073in}{1.217606in}}%
\pgfpathlineto{\pgfqpoint{3.268494in}{1.230741in}}%
\pgfpathlineto{\pgfqpoint{3.269758in}{1.217606in}}%
\pgfpathlineto{\pgfqpoint{3.270180in}{1.217606in}}%
\pgfpathlineto{\pgfqpoint{3.270601in}{1.230741in}}%
\pgfpathlineto{\pgfqpoint{3.271443in}{1.224173in}}%
\pgfpathlineto{\pgfqpoint{3.272286in}{1.217606in}}%
\pgfpathlineto{\pgfqpoint{3.273550in}{1.224173in}}%
\pgfpathlineto{\pgfqpoint{3.273971in}{1.217606in}}%
\pgfpathlineto{\pgfqpoint{3.274392in}{1.224173in}}%
\pgfpathlineto{\pgfqpoint{3.274814in}{1.224173in}}%
\pgfpathlineto{\pgfqpoint{3.275656in}{1.237308in}}%
\pgfpathlineto{\pgfqpoint{3.276077in}{1.211038in}}%
\pgfpathlineto{\pgfqpoint{3.276920in}{1.217606in}}%
\pgfpathlineto{\pgfqpoint{3.277341in}{1.217606in}}%
\pgfpathlineto{\pgfqpoint{3.277763in}{1.224173in}}%
\pgfpathlineto{\pgfqpoint{3.278184in}{1.217606in}}%
\pgfpathlineto{\pgfqpoint{3.278605in}{1.217606in}}%
\pgfpathlineto{\pgfqpoint{3.279869in}{1.224173in}}%
\pgfpathlineto{\pgfqpoint{3.280290in}{1.217606in}}%
\pgfpathlineto{\pgfqpoint{3.280712in}{1.230741in}}%
\pgfpathlineto{\pgfqpoint{3.281554in}{1.224173in}}%
\pgfpathlineto{\pgfqpoint{3.281975in}{1.224173in}}%
\pgfpathlineto{\pgfqpoint{3.282397in}{1.237308in}}%
\pgfpathlineto{\pgfqpoint{3.283239in}{1.230741in}}%
\pgfpathlineto{\pgfqpoint{3.283660in}{1.230741in}}%
\pgfpathlineto{\pgfqpoint{3.284924in}{1.224173in}}%
\pgfpathlineto{\pgfqpoint{3.285346in}{1.224173in}}%
\pgfpathlineto{\pgfqpoint{3.286609in}{1.217606in}}%
\pgfpathlineto{\pgfqpoint{3.287873in}{1.224173in}}%
\pgfpathlineto{\pgfqpoint{3.288295in}{1.224173in}}%
\pgfpathlineto{\pgfqpoint{3.288716in}{1.230741in}}%
\pgfpathlineto{\pgfqpoint{3.289137in}{1.211038in}}%
\pgfpathlineto{\pgfqpoint{3.289558in}{1.243876in}}%
\pgfpathlineto{\pgfqpoint{3.289980in}{1.230741in}}%
\pgfpathlineto{\pgfqpoint{3.291243in}{1.217606in}}%
\pgfpathlineto{\pgfqpoint{3.291665in}{1.237308in}}%
\pgfpathlineto{\pgfqpoint{3.292507in}{1.230741in}}%
\pgfpathlineto{\pgfqpoint{3.293350in}{1.217606in}}%
\pgfpathlineto{\pgfqpoint{3.294614in}{1.237308in}}%
\pgfpathlineto{\pgfqpoint{3.295456in}{1.217606in}}%
\pgfpathlineto{\pgfqpoint{3.296299in}{1.224173in}}%
\pgfpathlineto{\pgfqpoint{3.296720in}{1.230741in}}%
\pgfpathlineto{\pgfqpoint{3.297141in}{1.211038in}}%
\pgfpathlineto{\pgfqpoint{3.297984in}{1.217606in}}%
\pgfpathlineto{\pgfqpoint{3.298405in}{1.217606in}}%
\pgfpathlineto{\pgfqpoint{3.299248in}{1.230741in}}%
\pgfpathlineto{\pgfqpoint{3.299669in}{1.224173in}}%
\pgfpathlineto{\pgfqpoint{3.300512in}{1.224173in}}%
\pgfpathlineto{\pgfqpoint{3.301775in}{1.237308in}}%
\pgfpathlineto{\pgfqpoint{3.303039in}{1.211038in}}%
\pgfpathlineto{\pgfqpoint{3.303461in}{1.217606in}}%
\pgfpathlineto{\pgfqpoint{3.304724in}{1.237308in}}%
\pgfpathlineto{\pgfqpoint{3.305988in}{1.204471in}}%
\pgfpathlineto{\pgfqpoint{3.307252in}{1.224173in}}%
\pgfpathlineto{\pgfqpoint{3.308095in}{1.211038in}}%
\pgfpathlineto{\pgfqpoint{3.308937in}{1.230741in}}%
\pgfpathlineto{\pgfqpoint{3.310201in}{1.204471in}}%
\pgfpathlineto{\pgfqpoint{3.310622in}{1.237308in}}%
\pgfpathlineto{\pgfqpoint{3.311465in}{1.224173in}}%
\pgfpathlineto{\pgfqpoint{3.312307in}{1.224173in}}%
\pgfpathlineto{\pgfqpoint{3.312729in}{1.217606in}}%
\pgfpathlineto{\pgfqpoint{3.313571in}{1.230741in}}%
\pgfpathlineto{\pgfqpoint{3.314414in}{1.217606in}}%
\pgfpathlineto{\pgfqpoint{3.314835in}{1.230741in}}%
\pgfpathlineto{\pgfqpoint{3.315256in}{1.217606in}}%
\pgfpathlineto{\pgfqpoint{3.315678in}{1.211038in}}%
\pgfpathlineto{\pgfqpoint{3.317363in}{1.230741in}}%
\pgfpathlineto{\pgfqpoint{3.318205in}{1.211038in}}%
\pgfpathlineto{\pgfqpoint{3.318627in}{1.224173in}}%
\pgfpathlineto{\pgfqpoint{3.319048in}{1.243876in}}%
\pgfpathlineto{\pgfqpoint{3.319469in}{1.237308in}}%
\pgfpathlineto{\pgfqpoint{3.320312in}{1.217606in}}%
\pgfpathlineto{\pgfqpoint{3.320733in}{1.224173in}}%
\pgfpathlineto{\pgfqpoint{3.321154in}{1.230741in}}%
\pgfpathlineto{\pgfqpoint{3.322418in}{1.217606in}}%
\pgfpathlineto{\pgfqpoint{3.323261in}{1.217606in}}%
\pgfpathlineto{\pgfqpoint{3.323682in}{1.237308in}}%
\pgfpathlineto{\pgfqpoint{3.324103in}{1.224173in}}%
\pgfpathlineto{\pgfqpoint{3.324524in}{1.217606in}}%
\pgfpathlineto{\pgfqpoint{3.324946in}{1.230741in}}%
\pgfpathlineto{\pgfqpoint{3.325788in}{1.224173in}}%
\pgfpathlineto{\pgfqpoint{3.326210in}{1.230741in}}%
\pgfpathlineto{\pgfqpoint{3.326631in}{1.224173in}}%
\pgfpathlineto{\pgfqpoint{3.327052in}{1.217606in}}%
\pgfpathlineto{\pgfqpoint{3.327473in}{1.230741in}}%
\pgfpathlineto{\pgfqpoint{3.327895in}{1.224173in}}%
\pgfpathlineto{\pgfqpoint{3.328737in}{1.217606in}}%
\pgfpathlineto{\pgfqpoint{3.330422in}{1.230741in}}%
\pgfpathlineto{\pgfqpoint{3.330844in}{1.217606in}}%
\pgfpathlineto{\pgfqpoint{3.331686in}{1.224173in}}%
\pgfpathlineto{\pgfqpoint{3.332950in}{1.204471in}}%
\pgfpathlineto{\pgfqpoint{3.333793in}{1.237308in}}%
\pgfpathlineto{\pgfqpoint{3.334214in}{1.224173in}}%
\pgfpathlineto{\pgfqpoint{3.334635in}{1.224173in}}%
\pgfpathlineto{\pgfqpoint{3.335056in}{1.230741in}}%
\pgfpathlineto{\pgfqpoint{3.335478in}{1.224173in}}%
\pgfpathlineto{\pgfqpoint{3.335899in}{1.224173in}}%
\pgfpathlineto{\pgfqpoint{3.336320in}{1.237308in}}%
\pgfpathlineto{\pgfqpoint{3.336742in}{1.224173in}}%
\pgfpathlineto{\pgfqpoint{3.337584in}{1.224173in}}%
\pgfpathlineto{\pgfqpoint{3.338005in}{1.237308in}}%
\pgfpathlineto{\pgfqpoint{3.338427in}{1.224173in}}%
\pgfpathlineto{\pgfqpoint{3.339269in}{1.224173in}}%
\pgfpathlineto{\pgfqpoint{3.339690in}{1.217606in}}%
\pgfpathlineto{\pgfqpoint{3.340112in}{1.224173in}}%
\pgfpathlineto{\pgfqpoint{3.341376in}{1.230741in}}%
\pgfpathlineto{\pgfqpoint{3.342218in}{1.224173in}}%
\pgfpathlineto{\pgfqpoint{3.342639in}{1.230741in}}%
\pgfpathlineto{\pgfqpoint{3.343903in}{1.217606in}}%
\pgfpathlineto{\pgfqpoint{3.345588in}{1.230741in}}%
\pgfpathlineto{\pgfqpoint{3.346010in}{1.224173in}}%
\pgfpathlineto{\pgfqpoint{3.346431in}{1.243876in}}%
\pgfpathlineto{\pgfqpoint{3.346852in}{1.224173in}}%
\pgfpathlineto{\pgfqpoint{3.348116in}{1.217606in}}%
\pgfpathlineto{\pgfqpoint{3.349380in}{1.230741in}}%
\pgfpathlineto{\pgfqpoint{3.349801in}{1.217606in}}%
\pgfpathlineto{\pgfqpoint{3.350644in}{1.224173in}}%
\pgfpathlineto{\pgfqpoint{3.351486in}{1.211038in}}%
\pgfpathlineto{\pgfqpoint{3.351908in}{1.217606in}}%
\pgfpathlineto{\pgfqpoint{3.353171in}{1.230741in}}%
\pgfpathlineto{\pgfqpoint{3.354435in}{1.217606in}}%
\pgfpathlineto{\pgfqpoint{3.355278in}{1.237308in}}%
\pgfpathlineto{\pgfqpoint{3.356120in}{1.211038in}}%
\pgfpathlineto{\pgfqpoint{3.356542in}{1.217606in}}%
\pgfpathlineto{\pgfqpoint{3.356963in}{1.217606in}}%
\pgfpathlineto{\pgfqpoint{3.357384in}{1.237308in}}%
\pgfpathlineto{\pgfqpoint{3.357805in}{1.230741in}}%
\pgfpathlineto{\pgfqpoint{3.358648in}{1.211038in}}%
\pgfpathlineto{\pgfqpoint{3.359069in}{1.224173in}}%
\pgfpathlineto{\pgfqpoint{3.359491in}{1.230741in}}%
\pgfpathlineto{\pgfqpoint{3.359912in}{1.217606in}}%
\pgfpathlineto{\pgfqpoint{3.360333in}{1.224173in}}%
\pgfpathlineto{\pgfqpoint{3.360754in}{1.237308in}}%
\pgfpathlineto{\pgfqpoint{3.361176in}{1.217606in}}%
\pgfpathlineto{\pgfqpoint{3.361597in}{1.224173in}}%
\pgfpathlineto{\pgfqpoint{3.362018in}{1.217606in}}%
\pgfpathlineto{\pgfqpoint{3.362861in}{1.217606in}}%
\pgfpathlineto{\pgfqpoint{3.364546in}{1.230741in}}%
\pgfpathlineto{\pgfqpoint{3.365810in}{1.230741in}}%
\pgfpathlineto{\pgfqpoint{3.367074in}{1.217606in}}%
\pgfpathlineto{\pgfqpoint{3.367495in}{1.230741in}}%
\pgfpathlineto{\pgfqpoint{3.367916in}{1.224173in}}%
\pgfpathlineto{\pgfqpoint{3.368759in}{1.211038in}}%
\pgfpathlineto{\pgfqpoint{3.370022in}{1.237308in}}%
\pgfpathlineto{\pgfqpoint{3.371286in}{1.217606in}}%
\pgfpathlineto{\pgfqpoint{3.372129in}{1.230741in}}%
\pgfpathlineto{\pgfqpoint{3.372550in}{1.224173in}}%
\pgfpathlineto{\pgfqpoint{3.373814in}{1.224173in}}%
\pgfpathlineto{\pgfqpoint{3.374235in}{1.217606in}}%
\pgfpathlineto{\pgfqpoint{3.374657in}{1.230741in}}%
\pgfpathlineto{\pgfqpoint{3.375499in}{1.224173in}}%
\pgfpathlineto{\pgfqpoint{3.376763in}{1.230741in}}%
\pgfpathlineto{\pgfqpoint{3.378448in}{1.217606in}}%
\pgfpathlineto{\pgfqpoint{3.379712in}{1.224173in}}%
\pgfpathlineto{\pgfqpoint{3.380133in}{1.224173in}}%
\pgfpathlineto{\pgfqpoint{3.380554in}{1.230741in}}%
\pgfpathlineto{\pgfqpoint{3.380976in}{1.224173in}}%
\pgfpathlineto{\pgfqpoint{3.381397in}{1.224173in}}%
\pgfpathlineto{\pgfqpoint{3.382661in}{1.230741in}}%
\pgfpathlineto{\pgfqpoint{3.383925in}{1.217606in}}%
\pgfpathlineto{\pgfqpoint{3.384346in}{1.217606in}}%
\pgfpathlineto{\pgfqpoint{3.385610in}{1.230741in}}%
\pgfpathlineto{\pgfqpoint{3.386452in}{1.217606in}}%
\pgfpathlineto{\pgfqpoint{3.386874in}{1.224173in}}%
\pgfpathlineto{\pgfqpoint{3.387295in}{1.224173in}}%
\pgfpathlineto{\pgfqpoint{3.387716in}{1.217606in}}%
\pgfpathlineto{\pgfqpoint{3.388137in}{1.224173in}}%
\pgfpathlineto{\pgfqpoint{3.388559in}{1.224173in}}%
\pgfpathlineto{\pgfqpoint{3.388980in}{1.230741in}}%
\pgfpathlineto{\pgfqpoint{3.390244in}{1.217606in}}%
\pgfpathlineto{\pgfqpoint{3.391929in}{1.230741in}}%
\pgfpathlineto{\pgfqpoint{3.392350in}{1.217606in}}%
\pgfpathlineto{\pgfqpoint{3.392772in}{1.224173in}}%
\pgfpathlineto{\pgfqpoint{3.394035in}{1.230741in}}%
\pgfpathlineto{\pgfqpoint{3.394878in}{1.237308in}}%
\pgfpathlineto{\pgfqpoint{3.395299in}{1.217606in}}%
\pgfpathlineto{\pgfqpoint{3.396142in}{1.230741in}}%
\pgfpathlineto{\pgfqpoint{3.396563in}{1.224173in}}%
\pgfpathlineto{\pgfqpoint{3.397406in}{1.224173in}}%
\pgfpathlineto{\pgfqpoint{3.397827in}{1.237308in}}%
\pgfpathlineto{\pgfqpoint{3.398248in}{1.224173in}}%
\pgfpathlineto{\pgfqpoint{3.398669in}{1.224173in}}%
\pgfpathlineto{\pgfqpoint{3.399512in}{1.217606in}}%
\pgfpathlineto{\pgfqpoint{3.400776in}{1.230741in}}%
\pgfpathlineto{\pgfqpoint{3.402040in}{1.230741in}}%
\pgfpathlineto{\pgfqpoint{3.403303in}{1.217606in}}%
\pgfpathlineto{\pgfqpoint{3.403725in}{1.230741in}}%
\pgfpathlineto{\pgfqpoint{3.404146in}{1.224173in}}%
\pgfpathlineto{\pgfqpoint{3.404567in}{1.217606in}}%
\pgfpathlineto{\pgfqpoint{3.404989in}{1.224173in}}%
\pgfpathlineto{\pgfqpoint{3.405831in}{1.224173in}}%
\pgfpathlineto{\pgfqpoint{3.406252in}{1.237308in}}%
\pgfpathlineto{\pgfqpoint{3.406674in}{1.224173in}}%
\pgfpathlineto{\pgfqpoint{3.407516in}{1.211038in}}%
\pgfpathlineto{\pgfqpoint{3.407938in}{1.230741in}}%
\pgfpathlineto{\pgfqpoint{3.408780in}{1.224173in}}%
\pgfpathlineto{\pgfqpoint{3.409623in}{1.217606in}}%
\pgfpathlineto{\pgfqpoint{3.410886in}{1.230741in}}%
\pgfpathlineto{\pgfqpoint{3.411729in}{1.217606in}}%
\pgfpathlineto{\pgfqpoint{3.412993in}{1.243876in}}%
\pgfpathlineto{\pgfqpoint{3.413835in}{1.217606in}}%
\pgfpathlineto{\pgfqpoint{3.414257in}{1.237308in}}%
\pgfpathlineto{\pgfqpoint{3.414678in}{1.217606in}}%
\pgfpathlineto{\pgfqpoint{3.415099in}{1.217606in}}%
\pgfpathlineto{\pgfqpoint{3.416784in}{1.237308in}}%
\pgfpathlineto{\pgfqpoint{3.417627in}{1.217606in}}%
\pgfpathlineto{\pgfqpoint{3.418048in}{1.237308in}}%
\pgfpathlineto{\pgfqpoint{3.418469in}{1.224173in}}%
\pgfpathlineto{\pgfqpoint{3.418891in}{1.217606in}}%
\pgfpathlineto{\pgfqpoint{3.419312in}{1.230741in}}%
\pgfpathlineto{\pgfqpoint{3.419733in}{1.211038in}}%
\pgfpathlineto{\pgfqpoint{3.420576in}{1.237308in}}%
\pgfpathlineto{\pgfqpoint{3.420997in}{1.230741in}}%
\pgfpathlineto{\pgfqpoint{3.421840in}{1.217606in}}%
\pgfpathlineto{\pgfqpoint{3.423525in}{1.237308in}}%
\pgfpathlineto{\pgfqpoint{3.424367in}{1.217606in}}%
\pgfpathlineto{\pgfqpoint{3.424789in}{1.224173in}}%
\pgfpathlineto{\pgfqpoint{3.426052in}{1.230741in}}%
\pgfpathlineto{\pgfqpoint{3.426474in}{1.211038in}}%
\pgfpathlineto{\pgfqpoint{3.426895in}{1.230741in}}%
\pgfpathlineto{\pgfqpoint{3.427316in}{1.230741in}}%
\pgfpathlineto{\pgfqpoint{3.427738in}{1.204471in}}%
\pgfpathlineto{\pgfqpoint{3.428580in}{1.217606in}}%
\pgfpathlineto{\pgfqpoint{3.429001in}{1.217606in}}%
\pgfpathlineto{\pgfqpoint{3.429423in}{1.224173in}}%
\pgfpathlineto{\pgfqpoint{3.429844in}{1.217606in}}%
\pgfpathlineto{\pgfqpoint{3.430265in}{1.204471in}}%
\pgfpathlineto{\pgfqpoint{3.431108in}{1.230741in}}%
\pgfpathlineto{\pgfqpoint{3.431529in}{1.211038in}}%
\pgfpathlineto{\pgfqpoint{3.431950in}{1.217606in}}%
\pgfpathlineto{\pgfqpoint{3.432372in}{1.230741in}}%
\pgfpathlineto{\pgfqpoint{3.433214in}{1.224173in}}%
\pgfpathlineto{\pgfqpoint{3.433635in}{1.230741in}}%
\pgfpathlineto{\pgfqpoint{3.434057in}{1.211038in}}%
\pgfpathlineto{\pgfqpoint{3.434478in}{1.224173in}}%
\pgfpathlineto{\pgfqpoint{3.435321in}{1.230741in}}%
\pgfpathlineto{\pgfqpoint{3.437006in}{1.217606in}}%
\pgfpathlineto{\pgfqpoint{3.437427in}{1.230741in}}%
\pgfpathlineto{\pgfqpoint{3.437848in}{1.224173in}}%
\pgfpathlineto{\pgfqpoint{3.439112in}{1.217606in}}%
\pgfpathlineto{\pgfqpoint{3.440376in}{1.224173in}}%
\pgfpathlineto{\pgfqpoint{3.440797in}{1.224173in}}%
\pgfpathlineto{\pgfqpoint{3.442061in}{1.230741in}}%
\pgfpathlineto{\pgfqpoint{3.442904in}{1.217606in}}%
\pgfpathlineto{\pgfqpoint{3.443746in}{1.230741in}}%
\pgfpathlineto{\pgfqpoint{3.444589in}{1.217606in}}%
\pgfpathlineto{\pgfqpoint{3.445853in}{1.230741in}}%
\pgfpathlineto{\pgfqpoint{3.446274in}{1.224173in}}%
\pgfpathlineto{\pgfqpoint{3.446695in}{1.237308in}}%
\pgfpathlineto{\pgfqpoint{3.447116in}{1.217606in}}%
\pgfpathlineto{\pgfqpoint{3.447538in}{1.230741in}}%
\pgfpathlineto{\pgfqpoint{3.448380in}{1.224173in}}%
\pgfpathlineto{\pgfqpoint{3.448801in}{1.230741in}}%
\pgfpathlineto{\pgfqpoint{3.449644in}{1.217606in}}%
\pgfpathlineto{\pgfqpoint{3.450065in}{1.224173in}}%
\pgfpathlineto{\pgfqpoint{3.450487in}{1.224173in}}%
\pgfpathlineto{\pgfqpoint{3.450908in}{1.217606in}}%
\pgfpathlineto{\pgfqpoint{3.452172in}{1.230741in}}%
\pgfpathlineto{\pgfqpoint{3.452593in}{1.224173in}}%
\pgfpathlineto{\pgfqpoint{3.453014in}{1.230741in}}%
\pgfpathlineto{\pgfqpoint{3.453436in}{1.230741in}}%
\pgfpathlineto{\pgfqpoint{3.454699in}{1.211038in}}%
\pgfpathlineto{\pgfqpoint{3.455121in}{1.204471in}}%
\pgfpathlineto{\pgfqpoint{3.455542in}{1.211038in}}%
\pgfpathlineto{\pgfqpoint{3.456385in}{1.230741in}}%
\pgfpathlineto{\pgfqpoint{3.456806in}{1.224173in}}%
\pgfpathlineto{\pgfqpoint{3.457227in}{1.217606in}}%
\pgfpathlineto{\pgfqpoint{3.457648in}{1.230741in}}%
\pgfpathlineto{\pgfqpoint{3.458491in}{1.224173in}}%
\pgfpathlineto{\pgfqpoint{3.458912in}{1.224173in}}%
\pgfpathlineto{\pgfqpoint{3.459333in}{1.211038in}}%
\pgfpathlineto{\pgfqpoint{3.459755in}{1.230741in}}%
\pgfpathlineto{\pgfqpoint{3.460176in}{1.217606in}}%
\pgfpathlineto{\pgfqpoint{3.461440in}{1.230741in}}%
\pgfpathlineto{\pgfqpoint{3.462282in}{1.224173in}}%
\pgfpathlineto{\pgfqpoint{3.463125in}{1.243876in}}%
\pgfpathlineto{\pgfqpoint{3.463546in}{1.230741in}}%
\pgfpathlineto{\pgfqpoint{3.464810in}{1.224173in}}%
\pgfpathlineto{\pgfqpoint{3.466916in}{1.224173in}}%
\pgfpathlineto{\pgfqpoint{3.467338in}{1.243876in}}%
\pgfpathlineto{\pgfqpoint{3.467759in}{1.230741in}}%
\pgfpathlineto{\pgfqpoint{3.468180in}{1.230741in}}%
\pgfpathlineto{\pgfqpoint{3.469023in}{1.237308in}}%
\pgfpathlineto{\pgfqpoint{3.469865in}{1.211038in}}%
\pgfpathlineto{\pgfqpoint{3.470287in}{1.217606in}}%
\pgfpathlineto{\pgfqpoint{3.471129in}{1.237308in}}%
\pgfpathlineto{\pgfqpoint{3.471551in}{1.230741in}}%
\pgfpathlineto{\pgfqpoint{3.472393in}{1.224173in}}%
\pgfpathlineto{\pgfqpoint{3.473236in}{1.230741in}}%
\pgfpathlineto{\pgfqpoint{3.473657in}{1.224173in}}%
\pgfpathlineto{\pgfqpoint{3.474078in}{1.237308in}}%
\pgfpathlineto{\pgfqpoint{3.474499in}{1.217606in}}%
\pgfpathlineto{\pgfqpoint{3.474921in}{1.230741in}}%
\pgfpathlineto{\pgfqpoint{3.476606in}{1.211038in}}%
\pgfpathlineto{\pgfqpoint{3.477870in}{1.224173in}}%
\pgfpathlineto{\pgfqpoint{3.478291in}{1.211038in}}%
\pgfpathlineto{\pgfqpoint{3.478712in}{1.217606in}}%
\pgfpathlineto{\pgfqpoint{3.479976in}{1.224173in}}%
\pgfpathlineto{\pgfqpoint{3.480819in}{1.217606in}}%
\pgfpathlineto{\pgfqpoint{3.482082in}{1.237308in}}%
\pgfpathlineto{\pgfqpoint{3.482504in}{1.217606in}}%
\pgfpathlineto{\pgfqpoint{3.482925in}{1.224173in}}%
\pgfpathlineto{\pgfqpoint{3.483768in}{1.237308in}}%
\pgfpathlineto{\pgfqpoint{3.484610in}{1.204471in}}%
\pgfpathlineto{\pgfqpoint{3.485031in}{1.217606in}}%
\pgfpathlineto{\pgfqpoint{3.485874in}{1.217606in}}%
\pgfpathlineto{\pgfqpoint{3.486295in}{1.224173in}}%
\pgfpathlineto{\pgfqpoint{3.487138in}{1.230741in}}%
\pgfpathlineto{\pgfqpoint{3.487559in}{1.211038in}}%
\pgfpathlineto{\pgfqpoint{3.489244in}{1.230741in}}%
\pgfpathlineto{\pgfqpoint{3.489665in}{1.237308in}}%
\pgfpathlineto{\pgfqpoint{3.490508in}{1.217606in}}%
\pgfpathlineto{\pgfqpoint{3.491772in}{1.230741in}}%
\pgfpathlineto{\pgfqpoint{3.492193in}{1.230741in}}%
\pgfpathlineto{\pgfqpoint{3.493036in}{1.197903in}}%
\pgfpathlineto{\pgfqpoint{3.493457in}{1.224173in}}%
\pgfpathlineto{\pgfqpoint{3.493878in}{1.243876in}}%
\pgfpathlineto{\pgfqpoint{3.494300in}{1.224173in}}%
\pgfpathlineto{\pgfqpoint{3.494721in}{1.230741in}}%
\pgfpathlineto{\pgfqpoint{3.495142in}{1.224173in}}%
\pgfpathlineto{\pgfqpoint{3.495563in}{1.211038in}}%
\pgfpathlineto{\pgfqpoint{3.495985in}{1.237308in}}%
\pgfpathlineto{\pgfqpoint{3.496827in}{1.224173in}}%
\pgfpathlineto{\pgfqpoint{3.497248in}{1.217606in}}%
\pgfpathlineto{\pgfqpoint{3.497670in}{1.224173in}}%
\pgfpathlineto{\pgfqpoint{3.498512in}{1.224173in}}%
\pgfpathlineto{\pgfqpoint{3.499355in}{1.230741in}}%
\pgfpathlineto{\pgfqpoint{3.500619in}{1.211038in}}%
\pgfpathlineto{\pgfqpoint{3.501040in}{1.230741in}}%
\pgfpathlineto{\pgfqpoint{3.501461in}{1.224173in}}%
\pgfpathlineto{\pgfqpoint{3.501883in}{1.211038in}}%
\pgfpathlineto{\pgfqpoint{3.502304in}{1.230741in}}%
\pgfpathlineto{\pgfqpoint{3.502725in}{1.217606in}}%
\pgfpathlineto{\pgfqpoint{3.503146in}{1.217606in}}%
\pgfpathlineto{\pgfqpoint{3.503989in}{1.230741in}}%
\pgfpathlineto{\pgfqpoint{3.504410in}{1.224173in}}%
\pgfpathlineto{\pgfqpoint{3.504831in}{1.211038in}}%
\pgfpathlineto{\pgfqpoint{3.505253in}{1.230741in}}%
\pgfpathlineto{\pgfqpoint{3.505674in}{1.224173in}}%
\pgfpathlineto{\pgfqpoint{3.506095in}{1.230741in}}%
\pgfpathlineto{\pgfqpoint{3.506517in}{1.250443in}}%
\pgfpathlineto{\pgfqpoint{3.506938in}{1.224173in}}%
\pgfpathlineto{\pgfqpoint{3.507359in}{1.197903in}}%
\pgfpathlineto{\pgfqpoint{3.508202in}{1.204471in}}%
\pgfpathlineto{\pgfqpoint{3.509044in}{1.230741in}}%
\pgfpathlineto{\pgfqpoint{3.509466in}{1.224173in}}%
\pgfpathlineto{\pgfqpoint{3.509887in}{1.217606in}}%
\pgfpathlineto{\pgfqpoint{3.511151in}{1.230741in}}%
\pgfpathlineto{\pgfqpoint{3.511993in}{1.230741in}}%
\pgfpathlineto{\pgfqpoint{3.512414in}{1.217606in}}%
\pgfpathlineto{\pgfqpoint{3.512836in}{1.224173in}}%
\pgfpathlineto{\pgfqpoint{3.513257in}{1.230741in}}%
\pgfpathlineto{\pgfqpoint{3.514521in}{1.211038in}}%
\pgfpathlineto{\pgfqpoint{3.516206in}{1.230741in}}%
\pgfpathlineto{\pgfqpoint{3.516627in}{1.217606in}}%
\pgfpathlineto{\pgfqpoint{3.517470in}{1.224173in}}%
\pgfpathlineto{\pgfqpoint{3.517891in}{1.224173in}}%
\pgfpathlineto{\pgfqpoint{3.518312in}{1.230741in}}%
\pgfpathlineto{\pgfqpoint{3.518734in}{1.217606in}}%
\pgfpathlineto{\pgfqpoint{3.519576in}{1.224173in}}%
\pgfpathlineto{\pgfqpoint{3.520840in}{1.224173in}}%
\pgfpathlineto{\pgfqpoint{3.522104in}{1.237308in}}%
\pgfpathlineto{\pgfqpoint{3.523789in}{1.211038in}}%
\pgfpathlineto{\pgfqpoint{3.524210in}{1.237308in}}%
\pgfpathlineto{\pgfqpoint{3.524632in}{1.217606in}}%
\pgfpathlineto{\pgfqpoint{3.525053in}{1.211038in}}%
\pgfpathlineto{\pgfqpoint{3.525895in}{1.230741in}}%
\pgfpathlineto{\pgfqpoint{3.526317in}{1.224173in}}%
\pgfpathlineto{\pgfqpoint{3.526738in}{1.224173in}}%
\pgfpathlineto{\pgfqpoint{3.528002in}{1.237308in}}%
\pgfpathlineto{\pgfqpoint{3.529687in}{1.224173in}}%
\pgfpathlineto{\pgfqpoint{3.530529in}{1.224173in}}%
\pgfpathlineto{\pgfqpoint{3.530951in}{1.217606in}}%
\pgfpathlineto{\pgfqpoint{3.531372in}{1.224173in}}%
\pgfpathlineto{\pgfqpoint{3.531793in}{1.224173in}}%
\pgfpathlineto{\pgfqpoint{3.532215in}{1.230741in}}%
\pgfpathlineto{\pgfqpoint{3.532636in}{1.224173in}}%
\pgfpathlineto{\pgfqpoint{3.533057in}{1.224173in}}%
\pgfpathlineto{\pgfqpoint{3.533478in}{1.230741in}}%
\pgfpathlineto{\pgfqpoint{3.534742in}{1.217606in}}%
\pgfpathlineto{\pgfqpoint{3.535164in}{1.217606in}}%
\pgfpathlineto{\pgfqpoint{3.536006in}{1.211038in}}%
\pgfpathlineto{\pgfqpoint{3.537270in}{1.237308in}}%
\pgfpathlineto{\pgfqpoint{3.537691in}{1.217606in}}%
\pgfpathlineto{\pgfqpoint{3.538112in}{1.230741in}}%
\pgfpathlineto{\pgfqpoint{3.538534in}{1.237308in}}%
\pgfpathlineto{\pgfqpoint{3.538955in}{1.211038in}}%
\pgfpathlineto{\pgfqpoint{3.539798in}{1.224173in}}%
\pgfpathlineto{\pgfqpoint{3.540640in}{1.217606in}}%
\pgfpathlineto{\pgfqpoint{3.541061in}{1.230741in}}%
\pgfpathlineto{\pgfqpoint{3.541483in}{1.224173in}}%
\pgfpathlineto{\pgfqpoint{3.542325in}{1.217606in}}%
\pgfpathlineto{\pgfqpoint{3.543589in}{1.237308in}}%
\pgfpathlineto{\pgfqpoint{3.544010in}{1.217606in}}%
\pgfpathlineto{\pgfqpoint{3.544853in}{1.224173in}}%
\pgfpathlineto{\pgfqpoint{3.545695in}{1.217606in}}%
\pgfpathlineto{\pgfqpoint{3.547381in}{1.230741in}}%
\pgfpathlineto{\pgfqpoint{3.548644in}{1.230741in}}%
\pgfpathlineto{\pgfqpoint{3.549066in}{1.224173in}}%
\pgfpathlineto{\pgfqpoint{3.549487in}{1.237308in}}%
\pgfpathlineto{\pgfqpoint{3.549908in}{1.224173in}}%
\pgfpathlineto{\pgfqpoint{3.550330in}{1.217606in}}%
\pgfpathlineto{\pgfqpoint{3.550751in}{1.230741in}}%
\pgfpathlineto{\pgfqpoint{3.551593in}{1.224173in}}%
\pgfpathlineto{\pgfqpoint{3.552015in}{1.224173in}}%
\pgfpathlineto{\pgfqpoint{3.552436in}{1.243876in}}%
\pgfpathlineto{\pgfqpoint{3.552857in}{1.230741in}}%
\pgfpathlineto{\pgfqpoint{3.554542in}{1.217606in}}%
\pgfpathlineto{\pgfqpoint{3.555385in}{1.224173in}}%
\pgfpathlineto{\pgfqpoint{3.556649in}{1.211038in}}%
\pgfpathlineto{\pgfqpoint{3.557491in}{1.230741in}}%
\pgfpathlineto{\pgfqpoint{3.557913in}{1.217606in}}%
\pgfpathlineto{\pgfqpoint{3.559598in}{1.230741in}}%
\pgfpathlineto{\pgfqpoint{3.560440in}{1.217606in}}%
\pgfpathlineto{\pgfqpoint{3.560861in}{1.224173in}}%
\pgfpathlineto{\pgfqpoint{3.561283in}{1.224173in}}%
\pgfpathlineto{\pgfqpoint{3.561704in}{1.217606in}}%
\pgfpathlineto{\pgfqpoint{3.562125in}{1.230741in}}%
\pgfpathlineto{\pgfqpoint{3.562968in}{1.224173in}}%
\pgfpathlineto{\pgfqpoint{3.563810in}{1.224173in}}%
\pgfpathlineto{\pgfqpoint{3.564232in}{1.211038in}}%
\pgfpathlineto{\pgfqpoint{3.564653in}{1.217606in}}%
\pgfpathlineto{\pgfqpoint{3.565074in}{1.224173in}}%
\pgfpathlineto{\pgfqpoint{3.565496in}{1.217606in}}%
\pgfpathlineto{\pgfqpoint{3.566338in}{1.217606in}}%
\pgfpathlineto{\pgfqpoint{3.567602in}{1.237308in}}%
\pgfpathlineto{\pgfqpoint{3.568866in}{1.224173in}}%
\pgfpathlineto{\pgfqpoint{3.570130in}{1.243876in}}%
\pgfpathlineto{\pgfqpoint{3.570551in}{1.211038in}}%
\pgfpathlineto{\pgfqpoint{3.571393in}{1.224173in}}%
\pgfpathlineto{\pgfqpoint{3.571815in}{1.224173in}}%
\pgfpathlineto{\pgfqpoint{3.572236in}{1.217606in}}%
\pgfpathlineto{\pgfqpoint{3.572657in}{1.230741in}}%
\pgfpathlineto{\pgfqpoint{3.573500in}{1.224173in}}%
\pgfpathlineto{\pgfqpoint{3.573921in}{1.230741in}}%
\pgfpathlineto{\pgfqpoint{3.574342in}{1.224173in}}%
\pgfpathlineto{\pgfqpoint{3.575185in}{1.224173in}}%
\pgfpathlineto{\pgfqpoint{3.575606in}{1.217606in}}%
\pgfpathlineto{\pgfqpoint{3.576027in}{1.230741in}}%
\pgfpathlineto{\pgfqpoint{3.576870in}{1.224173in}}%
\pgfpathlineto{\pgfqpoint{3.577713in}{1.230741in}}%
\pgfpathlineto{\pgfqpoint{3.578555in}{1.217606in}}%
\pgfpathlineto{\pgfqpoint{3.578976in}{1.237308in}}%
\pgfpathlineto{\pgfqpoint{3.579398in}{1.211038in}}%
\pgfpathlineto{\pgfqpoint{3.579819in}{1.230741in}}%
\pgfpathlineto{\pgfqpoint{3.581083in}{1.217606in}}%
\pgfpathlineto{\pgfqpoint{3.582347in}{1.237308in}}%
\pgfpathlineto{\pgfqpoint{3.582768in}{1.230741in}}%
\pgfpathlineto{\pgfqpoint{3.583610in}{1.217606in}}%
\pgfpathlineto{\pgfqpoint{3.584032in}{1.224173in}}%
\pgfpathlineto{\pgfqpoint{3.584453in}{1.230741in}}%
\pgfpathlineto{\pgfqpoint{3.584874in}{1.224173in}}%
\pgfpathlineto{\pgfqpoint{3.585717in}{1.224173in}}%
\pgfpathlineto{\pgfqpoint{3.586559in}{1.217606in}}%
\pgfpathlineto{\pgfqpoint{3.587823in}{1.224173in}}%
\pgfpathlineto{\pgfqpoint{3.588245in}{1.224173in}}%
\pgfpathlineto{\pgfqpoint{3.589087in}{1.230741in}}%
\pgfpathlineto{\pgfqpoint{3.589508in}{1.217606in}}%
\pgfpathlineto{\pgfqpoint{3.590351in}{1.224173in}}%
\pgfpathlineto{\pgfqpoint{3.591194in}{1.217606in}}%
\pgfpathlineto{\pgfqpoint{3.591615in}{1.237308in}}%
\pgfpathlineto{\pgfqpoint{3.592036in}{1.230741in}}%
\pgfpathlineto{\pgfqpoint{3.592457in}{1.217606in}}%
\pgfpathlineto{\pgfqpoint{3.592879in}{1.237308in}}%
\pgfpathlineto{\pgfqpoint{3.593300in}{1.224173in}}%
\pgfpathlineto{\pgfqpoint{3.593721in}{1.224173in}}%
\pgfpathlineto{\pgfqpoint{3.594142in}{1.230741in}}%
\pgfpathlineto{\pgfqpoint{3.594564in}{1.211038in}}%
\pgfpathlineto{\pgfqpoint{3.594985in}{1.224173in}}%
\pgfpathlineto{\pgfqpoint{3.595406in}{1.224173in}}%
\pgfpathlineto{\pgfqpoint{3.595828in}{1.211038in}}%
\pgfpathlineto{\pgfqpoint{3.596670in}{1.217606in}}%
\pgfpathlineto{\pgfqpoint{3.597513in}{1.224173in}}%
\pgfpathlineto{\pgfqpoint{3.597934in}{1.217606in}}%
\pgfpathlineto{\pgfqpoint{3.598355in}{1.230741in}}%
\pgfpathlineto{\pgfqpoint{3.598777in}{1.211038in}}%
\pgfpathlineto{\pgfqpoint{3.599198in}{1.224173in}}%
\pgfpathlineto{\pgfqpoint{3.600462in}{1.224173in}}%
\pgfpathlineto{\pgfqpoint{3.601725in}{1.217606in}}%
\pgfpathlineto{\pgfqpoint{3.602989in}{1.230741in}}%
\pgfpathlineto{\pgfqpoint{3.603832in}{1.217606in}}%
\pgfpathlineto{\pgfqpoint{3.604253in}{1.230741in}}%
\pgfpathlineto{\pgfqpoint{3.605096in}{1.224173in}}%
\pgfpathlineto{\pgfqpoint{3.605517in}{1.224173in}}%
\pgfpathlineto{\pgfqpoint{3.606781in}{1.217606in}}%
\pgfpathlineto{\pgfqpoint{3.608045in}{1.237308in}}%
\pgfpathlineto{\pgfqpoint{3.609730in}{1.217606in}}%
\pgfpathlineto{\pgfqpoint{3.610151in}{1.230741in}}%
\pgfpathlineto{\pgfqpoint{3.610994in}{1.224173in}}%
\pgfpathlineto{\pgfqpoint{3.611415in}{1.230741in}}%
\pgfpathlineto{\pgfqpoint{3.611836in}{1.211038in}}%
\pgfpathlineto{\pgfqpoint{3.612257in}{1.243876in}}%
\pgfpathlineto{\pgfqpoint{3.613100in}{1.230741in}}%
\pgfpathlineto{\pgfqpoint{3.614364in}{1.211038in}}%
\pgfpathlineto{\pgfqpoint{3.614785in}{1.217606in}}%
\pgfpathlineto{\pgfqpoint{3.615206in}{1.217606in}}%
\pgfpathlineto{\pgfqpoint{3.616470in}{1.224173in}}%
\pgfpathlineto{\pgfqpoint{3.616891in}{1.224173in}}%
\pgfpathlineto{\pgfqpoint{3.618155in}{1.230741in}}%
\pgfpathlineto{\pgfqpoint{3.618577in}{1.217606in}}%
\pgfpathlineto{\pgfqpoint{3.618998in}{1.224173in}}%
\pgfpathlineto{\pgfqpoint{3.619419in}{1.230741in}}%
\pgfpathlineto{\pgfqpoint{3.621104in}{1.211038in}}%
\pgfpathlineto{\pgfqpoint{3.622368in}{1.224173in}}%
\pgfpathlineto{\pgfqpoint{3.623632in}{1.217606in}}%
\pgfpathlineto{\pgfqpoint{3.624053in}{1.217606in}}%
\pgfpathlineto{\pgfqpoint{3.624474in}{1.230741in}}%
\pgfpathlineto{\pgfqpoint{3.625317in}{1.224173in}}%
\pgfpathlineto{\pgfqpoint{3.625738in}{1.217606in}}%
\pgfpathlineto{\pgfqpoint{3.626160in}{1.224173in}}%
\pgfpathlineto{\pgfqpoint{3.626581in}{1.224173in}}%
\pgfpathlineto{\pgfqpoint{3.627423in}{1.217606in}}%
\pgfpathlineto{\pgfqpoint{3.628687in}{1.237308in}}%
\pgfpathlineto{\pgfqpoint{3.629951in}{1.217606in}}%
\pgfpathlineto{\pgfqpoint{3.630372in}{1.217606in}}%
\pgfpathlineto{\pgfqpoint{3.630794in}{1.224173in}}%
\pgfpathlineto{\pgfqpoint{3.631215in}{1.204471in}}%
\pgfpathlineto{\pgfqpoint{3.631636in}{1.224173in}}%
\pgfpathlineto{\pgfqpoint{3.632479in}{1.224173in}}%
\pgfpathlineto{\pgfqpoint{3.632900in}{1.211038in}}%
\pgfpathlineto{\pgfqpoint{3.633321in}{1.224173in}}%
\pgfpathlineto{\pgfqpoint{3.634585in}{1.224173in}}%
\pgfpathlineto{\pgfqpoint{3.635849in}{1.217606in}}%
\pgfpathlineto{\pgfqpoint{3.637534in}{1.230741in}}%
\pgfpathlineto{\pgfqpoint{3.637955in}{1.230741in}}%
\pgfpathlineto{\pgfqpoint{3.639219in}{1.211038in}}%
\pgfpathlineto{\pgfqpoint{3.640062in}{1.237308in}}%
\pgfpathlineto{\pgfqpoint{3.641326in}{1.230741in}}%
\pgfpathlineto{\pgfqpoint{3.641747in}{1.230741in}}%
\pgfpathlineto{\pgfqpoint{3.642168in}{1.217606in}}%
\pgfpathlineto{\pgfqpoint{3.642589in}{1.230741in}}%
\pgfpathlineto{\pgfqpoint{3.643011in}{1.230741in}}%
\pgfpathlineto{\pgfqpoint{3.644275in}{1.217606in}}%
\pgfpathlineto{\pgfqpoint{3.644696in}{1.237308in}}%
\pgfpathlineto{\pgfqpoint{3.645117in}{1.211038in}}%
\pgfpathlineto{\pgfqpoint{3.646802in}{1.230741in}}%
\pgfpathlineto{\pgfqpoint{3.647223in}{1.217606in}}%
\pgfpathlineto{\pgfqpoint{3.647645in}{1.230741in}}%
\pgfpathlineto{\pgfqpoint{3.648066in}{1.230741in}}%
\pgfpathlineto{\pgfqpoint{3.648487in}{1.237308in}}%
\pgfpathlineto{\pgfqpoint{3.648909in}{1.224173in}}%
\pgfpathlineto{\pgfqpoint{3.649751in}{1.230741in}}%
\pgfpathlineto{\pgfqpoint{3.651436in}{1.211038in}}%
\pgfpathlineto{\pgfqpoint{3.652700in}{1.237308in}}%
\pgfpathlineto{\pgfqpoint{3.653121in}{1.230741in}}%
\pgfpathlineto{\pgfqpoint{3.653543in}{1.204471in}}%
\pgfpathlineto{\pgfqpoint{3.654385in}{1.211038in}}%
\pgfpathlineto{\pgfqpoint{3.655649in}{1.230741in}}%
\pgfpathlineto{\pgfqpoint{3.657334in}{1.211038in}}%
\pgfpathlineto{\pgfqpoint{3.658598in}{1.230741in}}%
\pgfpathlineto{\pgfqpoint{3.659019in}{1.224173in}}%
\pgfpathlineto{\pgfqpoint{3.659441in}{1.224173in}}%
\pgfpathlineto{\pgfqpoint{3.659862in}{1.230741in}}%
\pgfpathlineto{\pgfqpoint{3.661547in}{1.211038in}}%
\pgfpathlineto{\pgfqpoint{3.662390in}{1.224173in}}%
\pgfpathlineto{\pgfqpoint{3.662811in}{1.211038in}}%
\pgfpathlineto{\pgfqpoint{3.663232in}{1.217606in}}%
\pgfpathlineto{\pgfqpoint{3.663653in}{1.224173in}}%
\pgfpathlineto{\pgfqpoint{3.664075in}{1.217606in}}%
\pgfpathlineto{\pgfqpoint{3.664496in}{1.217606in}}%
\pgfpathlineto{\pgfqpoint{3.665760in}{1.230741in}}%
\pgfpathlineto{\pgfqpoint{3.666181in}{1.217606in}}%
\pgfpathlineto{\pgfqpoint{3.666602in}{1.224173in}}%
\pgfpathlineto{\pgfqpoint{3.667024in}{1.230741in}}%
\pgfpathlineto{\pgfqpoint{3.667445in}{1.217606in}}%
\pgfpathlineto{\pgfqpoint{3.668287in}{1.224173in}}%
\pgfpathlineto{\pgfqpoint{3.668709in}{1.230741in}}%
\pgfpathlineto{\pgfqpoint{3.669130in}{1.217606in}}%
\pgfpathlineto{\pgfqpoint{3.669551in}{1.237308in}}%
\pgfpathlineto{\pgfqpoint{3.669973in}{1.224173in}}%
\pgfpathlineto{\pgfqpoint{3.671236in}{1.230741in}}%
\pgfpathlineto{\pgfqpoint{3.672500in}{1.217606in}}%
\pgfpathlineto{\pgfqpoint{3.673764in}{1.230741in}}%
\pgfpathlineto{\pgfqpoint{3.674607in}{1.217606in}}%
\pgfpathlineto{\pgfqpoint{3.675870in}{1.230741in}}%
\pgfpathlineto{\pgfqpoint{3.677134in}{1.224173in}}%
\pgfpathlineto{\pgfqpoint{3.677556in}{1.224173in}}%
\pgfpathlineto{\pgfqpoint{3.679241in}{1.237308in}}%
\pgfpathlineto{\pgfqpoint{3.679662in}{1.237308in}}%
\pgfpathlineto{\pgfqpoint{3.680504in}{1.211038in}}%
\pgfpathlineto{\pgfqpoint{3.680926in}{1.230741in}}%
\pgfpathlineto{\pgfqpoint{3.682190in}{1.224173in}}%
\pgfpathlineto{\pgfqpoint{3.682611in}{1.224173in}}%
\pgfpathlineto{\pgfqpoint{3.683032in}{1.217606in}}%
\pgfpathlineto{\pgfqpoint{3.684296in}{1.237308in}}%
\pgfpathlineto{\pgfqpoint{3.685560in}{1.237308in}}%
\pgfpathlineto{\pgfqpoint{3.685981in}{1.243876in}}%
\pgfpathlineto{\pgfqpoint{3.686824in}{1.224173in}}%
\pgfpathlineto{\pgfqpoint{3.687245in}{1.230741in}}%
\pgfpathlineto{\pgfqpoint{3.687666in}{1.230741in}}%
\pgfpathlineto{\pgfqpoint{3.688087in}{1.250443in}}%
\pgfpathlineto{\pgfqpoint{3.688509in}{1.243876in}}%
\pgfpathlineto{\pgfqpoint{3.689351in}{1.230741in}}%
\pgfpathlineto{\pgfqpoint{3.689773in}{1.237308in}}%
\pgfpathlineto{\pgfqpoint{3.690194in}{1.237308in}}%
\pgfpathlineto{\pgfqpoint{3.691458in}{1.217606in}}%
\pgfpathlineto{\pgfqpoint{3.691879in}{1.224173in}}%
\pgfpathlineto{\pgfqpoint{3.692300in}{1.217606in}}%
\pgfpathlineto{\pgfqpoint{3.693143in}{1.204471in}}%
\pgfpathlineto{\pgfqpoint{3.694828in}{1.230741in}}%
\pgfpathlineto{\pgfqpoint{3.696934in}{1.211038in}}%
\pgfpathlineto{\pgfqpoint{3.697356in}{1.224173in}}%
\pgfpathlineto{\pgfqpoint{3.698198in}{1.217606in}}%
\pgfpathlineto{\pgfqpoint{3.698619in}{1.217606in}}%
\pgfpathlineto{\pgfqpoint{3.699883in}{1.211038in}}%
\pgfpathlineto{\pgfqpoint{3.701147in}{1.224173in}}%
\pgfpathlineto{\pgfqpoint{3.701990in}{1.197903in}}%
\pgfpathlineto{\pgfqpoint{3.703253in}{1.217606in}}%
\pgfpathlineto{\pgfqpoint{3.703675in}{1.217606in}}%
\pgfpathlineto{\pgfqpoint{3.704939in}{1.237308in}}%
\pgfpathlineto{\pgfqpoint{3.705781in}{1.224173in}}%
\pgfpathlineto{\pgfqpoint{3.706202in}{1.230741in}}%
\pgfpathlineto{\pgfqpoint{3.707466in}{1.224173in}}%
\pgfpathlineto{\pgfqpoint{3.707888in}{1.224173in}}%
\pgfpathlineto{\pgfqpoint{3.708309in}{1.217606in}}%
\pgfpathlineto{\pgfqpoint{3.709151in}{1.237308in}}%
\pgfpathlineto{\pgfqpoint{3.709573in}{1.230741in}}%
\pgfpathlineto{\pgfqpoint{3.710415in}{1.230741in}}%
\pgfpathlineto{\pgfqpoint{3.711679in}{1.224173in}}%
\pgfpathlineto{\pgfqpoint{3.712100in}{1.224173in}}%
\pgfpathlineto{\pgfqpoint{3.713364in}{1.237308in}}%
\pgfpathlineto{\pgfqpoint{3.714207in}{1.237308in}}%
\pgfpathlineto{\pgfqpoint{3.714628in}{1.224173in}}%
\pgfpathlineto{\pgfqpoint{3.715049in}{1.237308in}}%
\pgfpathlineto{\pgfqpoint{3.715471in}{1.243876in}}%
\pgfpathlineto{\pgfqpoint{3.716734in}{1.224173in}}%
\pgfpathlineto{\pgfqpoint{3.717577in}{1.230741in}}%
\pgfpathlineto{\pgfqpoint{3.717998in}{1.217606in}}%
\pgfpathlineto{\pgfqpoint{3.718419in}{1.224173in}}%
\pgfpathlineto{\pgfqpoint{3.720105in}{1.257011in}}%
\pgfpathlineto{\pgfqpoint{3.720526in}{1.224173in}}%
\pgfpathlineto{\pgfqpoint{3.721368in}{1.237308in}}%
\pgfpathlineto{\pgfqpoint{3.723054in}{1.224173in}}%
\pgfpathlineto{\pgfqpoint{3.723475in}{1.237308in}}%
\pgfpathlineto{\pgfqpoint{3.724317in}{1.230741in}}%
\pgfpathlineto{\pgfqpoint{3.724739in}{1.237308in}}%
\pgfpathlineto{\pgfqpoint{3.725160in}{1.230741in}}%
\pgfpathlineto{\pgfqpoint{3.725581in}{1.230741in}}%
\pgfpathlineto{\pgfqpoint{3.727266in}{1.211038in}}%
\pgfpathlineto{\pgfqpoint{3.728530in}{1.230741in}}%
\pgfpathlineto{\pgfqpoint{3.728951in}{1.211038in}}%
\pgfpathlineto{\pgfqpoint{3.729373in}{1.230741in}}%
\pgfpathlineto{\pgfqpoint{3.729794in}{1.230741in}}%
\pgfpathlineto{\pgfqpoint{3.731058in}{1.211038in}}%
\pgfpathlineto{\pgfqpoint{3.731479in}{1.230741in}}%
\pgfpathlineto{\pgfqpoint{3.731900in}{1.217606in}}%
\pgfpathlineto{\pgfqpoint{3.732322in}{1.211038in}}%
\pgfpathlineto{\pgfqpoint{3.732743in}{1.217606in}}%
\pgfpathlineto{\pgfqpoint{3.733164in}{1.217606in}}%
\pgfpathlineto{\pgfqpoint{3.734007in}{1.224173in}}%
\pgfpathlineto{\pgfqpoint{3.734428in}{1.204471in}}%
\pgfpathlineto{\pgfqpoint{3.735271in}{1.230741in}}%
\pgfpathlineto{\pgfqpoint{3.735692in}{1.211038in}}%
\pgfpathlineto{\pgfqpoint{3.736113in}{1.217606in}}%
\pgfpathlineto{\pgfqpoint{3.736534in}{1.224173in}}%
\pgfpathlineto{\pgfqpoint{3.737377in}{1.211038in}}%
\pgfpathlineto{\pgfqpoint{3.737798in}{1.224173in}}%
\pgfpathlineto{\pgfqpoint{3.738641in}{1.217606in}}%
\pgfpathlineto{\pgfqpoint{3.739062in}{1.217606in}}%
\pgfpathlineto{\pgfqpoint{3.739483in}{1.224173in}}%
\pgfpathlineto{\pgfqpoint{3.739905in}{1.217606in}}%
\pgfpathlineto{\pgfqpoint{3.740326in}{1.217606in}}%
\pgfpathlineto{\pgfqpoint{3.740747in}{1.230741in}}%
\pgfpathlineto{\pgfqpoint{3.741590in}{1.224173in}}%
\pgfpathlineto{\pgfqpoint{3.742011in}{1.217606in}}%
\pgfpathlineto{\pgfqpoint{3.742432in}{1.230741in}}%
\pgfpathlineto{\pgfqpoint{3.742854in}{1.224173in}}%
\pgfpathlineto{\pgfqpoint{3.743275in}{1.211038in}}%
\pgfpathlineto{\pgfqpoint{3.743696in}{1.224173in}}%
\pgfpathlineto{\pgfqpoint{3.744539in}{1.230741in}}%
\pgfpathlineto{\pgfqpoint{3.744960in}{1.217606in}}%
\pgfpathlineto{\pgfqpoint{3.745381in}{1.224173in}}%
\pgfpathlineto{\pgfqpoint{3.745803in}{1.237308in}}%
\pgfpathlineto{\pgfqpoint{3.746224in}{1.224173in}}%
\pgfpathlineto{\pgfqpoint{3.747066in}{1.224173in}}%
\pgfpathlineto{\pgfqpoint{3.747488in}{1.217606in}}%
\pgfpathlineto{\pgfqpoint{3.747909in}{1.230741in}}%
\pgfpathlineto{\pgfqpoint{3.748330in}{1.217606in}}%
\pgfpathlineto{\pgfqpoint{3.748752in}{1.217606in}}%
\pgfpathlineto{\pgfqpoint{3.749594in}{1.237308in}}%
\pgfpathlineto{\pgfqpoint{3.750437in}{1.230741in}}%
\pgfpathlineto{\pgfqpoint{3.750858in}{1.230741in}}%
\pgfpathlineto{\pgfqpoint{3.751700in}{1.211038in}}%
\pgfpathlineto{\pgfqpoint{3.752122in}{1.224173in}}%
\pgfpathlineto{\pgfqpoint{3.752543in}{1.217606in}}%
\pgfpathlineto{\pgfqpoint{3.752964in}{1.224173in}}%
\pgfpathlineto{\pgfqpoint{3.753386in}{1.224173in}}%
\pgfpathlineto{\pgfqpoint{3.754228in}{1.230741in}}%
\pgfpathlineto{\pgfqpoint{3.755913in}{1.217606in}}%
\pgfpathlineto{\pgfqpoint{3.756335in}{1.217606in}}%
\pgfpathlineto{\pgfqpoint{3.756756in}{1.230741in}}%
\pgfpathlineto{\pgfqpoint{3.757177in}{1.217606in}}%
\pgfpathlineto{\pgfqpoint{3.757598in}{1.211038in}}%
\pgfpathlineto{\pgfqpoint{3.758020in}{1.230741in}}%
\pgfpathlineto{\pgfqpoint{3.758862in}{1.224173in}}%
\pgfpathlineto{\pgfqpoint{3.759283in}{1.211038in}}%
\pgfpathlineto{\pgfqpoint{3.759705in}{1.224173in}}%
\pgfpathlineto{\pgfqpoint{3.760547in}{1.224173in}}%
\pgfpathlineto{\pgfqpoint{3.761390in}{1.237308in}}%
\pgfpathlineto{\pgfqpoint{3.763075in}{1.217606in}}%
\pgfpathlineto{\pgfqpoint{3.763496in}{1.224173in}}%
\pgfpathlineto{\pgfqpoint{3.763918in}{1.211038in}}%
\pgfpathlineto{\pgfqpoint{3.764339in}{1.217606in}}%
\pgfpathlineto{\pgfqpoint{3.765603in}{1.230741in}}%
\pgfpathlineto{\pgfqpoint{3.766445in}{1.217606in}}%
\pgfpathlineto{\pgfqpoint{3.766866in}{1.224173in}}%
\pgfpathlineto{\pgfqpoint{3.767288in}{1.224173in}}%
\pgfpathlineto{\pgfqpoint{3.767709in}{1.211038in}}%
\pgfpathlineto{\pgfqpoint{3.768130in}{1.230741in}}%
\pgfpathlineto{\pgfqpoint{3.768973in}{1.224173in}}%
\pgfpathlineto{\pgfqpoint{3.769394in}{1.230741in}}%
\pgfpathlineto{\pgfqpoint{3.769815in}{1.224173in}}%
\pgfpathlineto{\pgfqpoint{3.770237in}{1.224173in}}%
\pgfpathlineto{\pgfqpoint{3.771079in}{1.230741in}}%
\pgfpathlineto{\pgfqpoint{3.771501in}{1.211038in}}%
\pgfpathlineto{\pgfqpoint{3.772343in}{1.217606in}}%
\pgfpathlineto{\pgfqpoint{3.773186in}{1.224173in}}%
\pgfpathlineto{\pgfqpoint{3.773607in}{1.217606in}}%
\pgfpathlineto{\pgfqpoint{3.774028in}{1.230741in}}%
\pgfpathlineto{\pgfqpoint{3.774449in}{1.224173in}}%
\pgfpathlineto{\pgfqpoint{3.774871in}{1.204471in}}%
\pgfpathlineto{\pgfqpoint{3.775292in}{1.224173in}}%
\pgfpathlineto{\pgfqpoint{3.775713in}{1.217606in}}%
\pgfpathlineto{\pgfqpoint{3.776135in}{1.237308in}}%
\pgfpathlineto{\pgfqpoint{3.776556in}{1.217606in}}%
\pgfpathlineto{\pgfqpoint{3.777820in}{1.224173in}}%
\pgfpathlineto{\pgfqpoint{3.779084in}{1.224173in}}%
\pgfpathlineto{\pgfqpoint{3.779926in}{1.230741in}}%
\pgfpathlineto{\pgfqpoint{3.780769in}{1.217606in}}%
\pgfpathlineto{\pgfqpoint{3.781190in}{1.224173in}}%
\pgfpathlineto{\pgfqpoint{3.781611in}{1.217606in}}%
\pgfpathlineto{\pgfqpoint{3.782032in}{1.230741in}}%
\pgfpathlineto{\pgfqpoint{3.782454in}{1.224173in}}%
\pgfpathlineto{\pgfqpoint{3.782875in}{1.217606in}}%
\pgfpathlineto{\pgfqpoint{3.783296in}{1.230741in}}%
\pgfpathlineto{\pgfqpoint{3.784139in}{1.224173in}}%
\pgfpathlineto{\pgfqpoint{3.784981in}{1.224173in}}%
\pgfpathlineto{\pgfqpoint{3.785403in}{1.217606in}}%
\pgfpathlineto{\pgfqpoint{3.786245in}{1.237308in}}%
\pgfpathlineto{\pgfqpoint{3.787509in}{1.217606in}}%
\pgfpathlineto{\pgfqpoint{3.788352in}{1.230741in}}%
\pgfpathlineto{\pgfqpoint{3.788773in}{1.217606in}}%
\pgfpathlineto{\pgfqpoint{3.789616in}{1.224173in}}%
\pgfpathlineto{\pgfqpoint{3.790458in}{1.230741in}}%
\pgfpathlineto{\pgfqpoint{3.791722in}{1.211038in}}%
\pgfpathlineto{\pgfqpoint{3.792143in}{1.217606in}}%
\pgfpathlineto{\pgfqpoint{3.792986in}{1.217606in}}%
\pgfpathlineto{\pgfqpoint{3.794250in}{1.224173in}}%
\pgfpathlineto{\pgfqpoint{3.794671in}{1.224173in}}%
\pgfpathlineto{\pgfqpoint{3.795092in}{1.243876in}}%
\pgfpathlineto{\pgfqpoint{3.795513in}{1.217606in}}%
\pgfpathlineto{\pgfqpoint{3.796356in}{1.230741in}}%
\pgfpathlineto{\pgfqpoint{3.796777in}{1.224173in}}%
\pgfpathlineto{\pgfqpoint{3.797199in}{1.230741in}}%
\pgfpathlineto{\pgfqpoint{3.797620in}{1.217606in}}%
\pgfpathlineto{\pgfqpoint{3.798462in}{1.224173in}}%
\pgfpathlineto{\pgfqpoint{3.798884in}{1.237308in}}%
\pgfpathlineto{\pgfqpoint{3.799305in}{1.224173in}}%
\pgfpathlineto{\pgfqpoint{3.799726in}{1.217606in}}%
\pgfpathlineto{\pgfqpoint{3.800990in}{1.230741in}}%
\pgfpathlineto{\pgfqpoint{3.801833in}{1.224173in}}%
\pgfpathlineto{\pgfqpoint{3.802675in}{1.230741in}}%
\pgfpathlineto{\pgfqpoint{3.803096in}{1.217606in}}%
\pgfpathlineto{\pgfqpoint{3.803939in}{1.224173in}}%
\pgfpathlineto{\pgfqpoint{3.805203in}{1.230741in}}%
\pgfpathlineto{\pgfqpoint{3.805624in}{1.230741in}}%
\pgfpathlineto{\pgfqpoint{3.806888in}{1.217606in}}%
\pgfpathlineto{\pgfqpoint{3.807309in}{1.230741in}}%
\pgfpathlineto{\pgfqpoint{3.808152in}{1.224173in}}%
\pgfpathlineto{\pgfqpoint{3.809416in}{1.217606in}}%
\pgfpathlineto{\pgfqpoint{3.809837in}{1.217606in}}%
\pgfpathlineto{\pgfqpoint{3.811101in}{1.224173in}}%
\pgfpathlineto{\pgfqpoint{3.811943in}{1.224173in}}%
\pgfpathlineto{\pgfqpoint{3.813207in}{1.217606in}}%
\pgfpathlineto{\pgfqpoint{3.814050in}{1.224173in}}%
\pgfpathlineto{\pgfqpoint{3.814471in}{1.217606in}}%
\pgfpathlineto{\pgfqpoint{3.814892in}{1.224173in}}%
\pgfpathlineto{\pgfqpoint{3.816156in}{1.224173in}}%
\pgfpathlineto{\pgfqpoint{3.816999in}{1.211038in}}%
\pgfpathlineto{\pgfqpoint{3.817420in}{1.217606in}}%
\pgfpathlineto{\pgfqpoint{3.817841in}{1.224173in}}%
\pgfpathlineto{\pgfqpoint{3.818262in}{1.211038in}}%
\pgfpathlineto{\pgfqpoint{3.818684in}{1.224173in}}%
\pgfpathlineto{\pgfqpoint{3.819948in}{1.230741in}}%
\pgfpathlineto{\pgfqpoint{3.820790in}{1.230741in}}%
\pgfpathlineto{\pgfqpoint{3.821211in}{1.224173in}}%
\pgfpathlineto{\pgfqpoint{3.821633in}{1.237308in}}%
\pgfpathlineto{\pgfqpoint{3.822054in}{1.211038in}}%
\pgfpathlineto{\pgfqpoint{3.822896in}{1.224173in}}%
\pgfpathlineto{\pgfqpoint{3.823739in}{1.224173in}}%
\pgfpathlineto{\pgfqpoint{3.824160in}{1.211038in}}%
\pgfpathlineto{\pgfqpoint{3.824582in}{1.230741in}}%
\pgfpathlineto{\pgfqpoint{3.825424in}{1.230741in}}%
\pgfpathlineto{\pgfqpoint{3.825845in}{1.224173in}}%
\pgfpathlineto{\pgfqpoint{3.826267in}{1.230741in}}%
\pgfpathlineto{\pgfqpoint{3.826688in}{1.230741in}}%
\pgfpathlineto{\pgfqpoint{3.827109in}{1.217606in}}%
\pgfpathlineto{\pgfqpoint{3.827952in}{1.224173in}}%
\pgfpathlineto{\pgfqpoint{3.828373in}{1.217606in}}%
\pgfpathlineto{\pgfqpoint{3.828794in}{1.224173in}}%
\pgfpathlineto{\pgfqpoint{3.829637in}{1.224173in}}%
\pgfpathlineto{\pgfqpoint{3.830058in}{1.237308in}}%
\pgfpathlineto{\pgfqpoint{3.830479in}{1.230741in}}%
\pgfpathlineto{\pgfqpoint{3.831743in}{1.224173in}}%
\pgfpathlineto{\pgfqpoint{3.832165in}{1.224173in}}%
\pgfpathlineto{\pgfqpoint{3.833428in}{1.197903in}}%
\pgfpathlineto{\pgfqpoint{3.835535in}{1.230741in}}%
\pgfpathlineto{\pgfqpoint{3.835956in}{1.224173in}}%
\pgfpathlineto{\pgfqpoint{3.836377in}{1.230741in}}%
\pgfpathlineto{\pgfqpoint{3.836799in}{1.230741in}}%
\pgfpathlineto{\pgfqpoint{3.837641in}{1.237308in}}%
\pgfpathlineto{\pgfqpoint{3.838484in}{1.211038in}}%
\pgfpathlineto{\pgfqpoint{3.839326in}{1.224173in}}%
\pgfpathlineto{\pgfqpoint{3.839748in}{1.211038in}}%
\pgfpathlineto{\pgfqpoint{3.840169in}{1.230741in}}%
\pgfpathlineto{\pgfqpoint{3.841433in}{1.224173in}}%
\pgfpathlineto{\pgfqpoint{3.841854in}{1.224173in}}%
\pgfpathlineto{\pgfqpoint{3.842275in}{1.217606in}}%
\pgfpathlineto{\pgfqpoint{3.842697in}{1.224173in}}%
\pgfpathlineto{\pgfqpoint{3.843118in}{1.224173in}}%
\pgfpathlineto{\pgfqpoint{3.843539in}{1.211038in}}%
\pgfpathlineto{\pgfqpoint{3.844382in}{1.217606in}}%
\pgfpathlineto{\pgfqpoint{3.844803in}{1.217606in}}%
\pgfpathlineto{\pgfqpoint{3.846488in}{1.237308in}}%
\pgfpathlineto{\pgfqpoint{3.846909in}{1.237308in}}%
\pgfpathlineto{\pgfqpoint{3.848594in}{1.224173in}}%
\pgfpathlineto{\pgfqpoint{3.849016in}{1.224173in}}%
\pgfpathlineto{\pgfqpoint{3.849858in}{1.211038in}}%
\pgfpathlineto{\pgfqpoint{3.850701in}{1.230741in}}%
\pgfpathlineto{\pgfqpoint{3.851122in}{1.204471in}}%
\pgfpathlineto{\pgfqpoint{3.851543in}{1.224173in}}%
\pgfpathlineto{\pgfqpoint{3.851965in}{1.230741in}}%
\pgfpathlineto{\pgfqpoint{3.852386in}{1.217606in}}%
\pgfpathlineto{\pgfqpoint{3.852807in}{1.230741in}}%
\pgfpathlineto{\pgfqpoint{3.853229in}{1.230741in}}%
\pgfpathlineto{\pgfqpoint{3.853650in}{1.211038in}}%
\pgfpathlineto{\pgfqpoint{3.854071in}{1.230741in}}%
\pgfpathlineto{\pgfqpoint{3.854914in}{1.224173in}}%
\pgfpathlineto{\pgfqpoint{3.855335in}{1.237308in}}%
\pgfpathlineto{\pgfqpoint{3.856177in}{1.230741in}}%
\pgfpathlineto{\pgfqpoint{3.857441in}{1.217606in}}%
\pgfpathlineto{\pgfqpoint{3.857863in}{1.237308in}}%
\pgfpathlineto{\pgfqpoint{3.858284in}{1.224173in}}%
\pgfpathlineto{\pgfqpoint{3.858705in}{1.211038in}}%
\pgfpathlineto{\pgfqpoint{3.859548in}{1.217606in}}%
\pgfpathlineto{\pgfqpoint{3.859969in}{1.211038in}}%
\pgfpathlineto{\pgfqpoint{3.861654in}{1.230741in}}%
\pgfpathlineto{\pgfqpoint{3.862497in}{1.217606in}}%
\pgfpathlineto{\pgfqpoint{3.862918in}{1.237308in}}%
\pgfpathlineto{\pgfqpoint{3.863339in}{1.230741in}}%
\pgfpathlineto{\pgfqpoint{3.863760in}{1.217606in}}%
\pgfpathlineto{\pgfqpoint{3.864182in}{1.224173in}}%
\pgfpathlineto{\pgfqpoint{3.864603in}{1.237308in}}%
\pgfpathlineto{\pgfqpoint{3.865446in}{1.230741in}}%
\pgfpathlineto{\pgfqpoint{3.865867in}{1.230741in}}%
\pgfpathlineto{\pgfqpoint{3.866288in}{1.217606in}}%
\pgfpathlineto{\pgfqpoint{3.866709in}{1.224173in}}%
\pgfpathlineto{\pgfqpoint{3.867131in}{1.230741in}}%
\pgfpathlineto{\pgfqpoint{3.867552in}{1.224173in}}%
\pgfpathlineto{\pgfqpoint{3.867973in}{1.224173in}}%
\pgfpathlineto{\pgfqpoint{3.868816in}{1.211038in}}%
\pgfpathlineto{\pgfqpoint{3.870501in}{1.230741in}}%
\pgfpathlineto{\pgfqpoint{3.870922in}{1.237308in}}%
\pgfpathlineto{\pgfqpoint{3.871343in}{1.217606in}}%
\pgfpathlineto{\pgfqpoint{3.871765in}{1.224173in}}%
\pgfpathlineto{\pgfqpoint{3.872607in}{1.217606in}}%
\pgfpathlineto{\pgfqpoint{3.873450in}{1.243876in}}%
\pgfpathlineto{\pgfqpoint{3.875135in}{1.224173in}}%
\pgfpathlineto{\pgfqpoint{3.875556in}{1.224173in}}%
\pgfpathlineto{\pgfqpoint{3.875978in}{1.230741in}}%
\pgfpathlineto{\pgfqpoint{3.876399in}{1.217606in}}%
\pgfpathlineto{\pgfqpoint{3.876820in}{1.230741in}}%
\pgfpathlineto{\pgfqpoint{3.877241in}{1.230741in}}%
\pgfpathlineto{\pgfqpoint{3.878084in}{1.237308in}}%
\pgfpathlineto{\pgfqpoint{3.878926in}{1.211038in}}%
\pgfpathlineto{\pgfqpoint{3.880612in}{1.230741in}}%
\pgfpathlineto{\pgfqpoint{3.881454in}{1.224173in}}%
\pgfpathlineto{\pgfqpoint{3.882297in}{1.230741in}}%
\pgfpathlineto{\pgfqpoint{3.882718in}{1.217606in}}%
\pgfpathlineto{\pgfqpoint{3.883561in}{1.224173in}}%
\pgfpathlineto{\pgfqpoint{3.883982in}{1.217606in}}%
\pgfpathlineto{\pgfqpoint{3.884403in}{1.224173in}}%
\pgfpathlineto{\pgfqpoint{3.885246in}{1.230741in}}%
\pgfpathlineto{\pgfqpoint{3.886931in}{1.204471in}}%
\pgfpathlineto{\pgfqpoint{3.887773in}{1.230741in}}%
\pgfpathlineto{\pgfqpoint{3.888195in}{1.224173in}}%
\pgfpathlineto{\pgfqpoint{3.888616in}{1.230741in}}%
\pgfpathlineto{\pgfqpoint{3.889037in}{1.217606in}}%
\pgfpathlineto{\pgfqpoint{3.889880in}{1.224173in}}%
\pgfpathlineto{\pgfqpoint{3.890301in}{1.217606in}}%
\pgfpathlineto{\pgfqpoint{3.890722in}{1.224173in}}%
\pgfpathlineto{\pgfqpoint{3.891144in}{1.230741in}}%
\pgfpathlineto{\pgfqpoint{3.891565in}{1.217606in}}%
\pgfpathlineto{\pgfqpoint{3.891986in}{1.230741in}}%
\pgfpathlineto{\pgfqpoint{3.892407in}{1.230741in}}%
\pgfpathlineto{\pgfqpoint{3.892829in}{1.020579in}}%
\pgfpathlineto{\pgfqpoint{3.893250in}{1.230741in}}%
\pgfpathlineto{\pgfqpoint{3.893671in}{1.230741in}}%
\pgfpathlineto{\pgfqpoint{3.894092in}{1.211038in}}%
\pgfpathlineto{\pgfqpoint{3.894935in}{1.217606in}}%
\pgfpathlineto{\pgfqpoint{3.896620in}{1.230741in}}%
\pgfpathlineto{\pgfqpoint{3.898727in}{1.204471in}}%
\pgfpathlineto{\pgfqpoint{3.900412in}{1.243876in}}%
\pgfpathlineto{\pgfqpoint{3.901254in}{1.217606in}}%
\pgfpathlineto{\pgfqpoint{3.901675in}{1.224173in}}%
\pgfpathlineto{\pgfqpoint{3.902518in}{1.224173in}}%
\pgfpathlineto{\pgfqpoint{3.902939in}{1.211038in}}%
\pgfpathlineto{\pgfqpoint{3.903361in}{1.224173in}}%
\pgfpathlineto{\pgfqpoint{3.903782in}{1.224173in}}%
\pgfpathlineto{\pgfqpoint{3.904203in}{1.217606in}}%
\pgfpathlineto{\pgfqpoint{3.904624in}{1.224173in}}%
\pgfpathlineto{\pgfqpoint{3.905046in}{1.230741in}}%
\pgfpathlineto{\pgfqpoint{3.905888in}{1.217606in}}%
\pgfpathlineto{\pgfqpoint{3.906310in}{1.427767in}}%
\pgfpathlineto{\pgfqpoint{3.906731in}{1.230741in}}%
\pgfpathlineto{\pgfqpoint{3.907152in}{1.217606in}}%
\pgfpathlineto{\pgfqpoint{3.907995in}{1.224173in}}%
\pgfpathlineto{\pgfqpoint{3.908837in}{1.224173in}}%
\pgfpathlineto{\pgfqpoint{3.909258in}{1.217606in}}%
\pgfpathlineto{\pgfqpoint{3.909680in}{1.230741in}}%
\pgfpathlineto{\pgfqpoint{3.910101in}{1.211038in}}%
\pgfpathlineto{\pgfqpoint{3.910522in}{1.224173in}}%
\pgfpathlineto{\pgfqpoint{3.910944in}{1.224173in}}%
\pgfpathlineto{\pgfqpoint{3.912207in}{1.243876in}}%
\pgfpathlineto{\pgfqpoint{3.913050in}{1.211038in}}%
\pgfpathlineto{\pgfqpoint{3.913471in}{1.230741in}}%
\pgfpathlineto{\pgfqpoint{3.913893in}{1.230741in}}%
\pgfpathlineto{\pgfqpoint{3.914314in}{1.211038in}}%
\pgfpathlineto{\pgfqpoint{3.915156in}{1.217606in}}%
\pgfpathlineto{\pgfqpoint{3.915578in}{1.217606in}}%
\pgfpathlineto{\pgfqpoint{3.916420in}{1.237308in}}%
\pgfpathlineto{\pgfqpoint{3.917263in}{1.224173in}}%
\pgfpathlineto{\pgfqpoint{3.917684in}{1.230741in}}%
\pgfpathlineto{\pgfqpoint{3.918105in}{1.283281in}}%
\pgfpathlineto{\pgfqpoint{3.918527in}{1.224173in}}%
\pgfpathlineto{\pgfqpoint{3.921897in}{1.224173in}}%
\pgfpathlineto{\pgfqpoint{3.922739in}{1.230741in}}%
\pgfpathlineto{\pgfqpoint{3.923161in}{1.211038in}}%
\pgfpathlineto{\pgfqpoint{3.923582in}{1.237308in}}%
\pgfpathlineto{\pgfqpoint{3.925267in}{1.211038in}}%
\pgfpathlineto{\pgfqpoint{3.925688in}{1.211038in}}%
\pgfpathlineto{\pgfqpoint{3.926531in}{1.243876in}}%
\pgfpathlineto{\pgfqpoint{3.926952in}{1.224173in}}%
\pgfpathlineto{\pgfqpoint{3.927373in}{1.211038in}}%
\pgfpathlineto{\pgfqpoint{3.927795in}{1.230741in}}%
\pgfpathlineto{\pgfqpoint{3.928637in}{1.230741in}}%
\pgfpathlineto{\pgfqpoint{3.929480in}{1.204471in}}%
\pgfpathlineto{\pgfqpoint{3.930322in}{1.217606in}}%
\pgfpathlineto{\pgfqpoint{3.930744in}{1.224173in}}%
\pgfpathlineto{\pgfqpoint{3.931165in}{1.217606in}}%
\pgfpathlineto{\pgfqpoint{3.931586in}{1.158498in}}%
\pgfpathlineto{\pgfqpoint{3.932008in}{1.217606in}}%
\pgfpathlineto{\pgfqpoint{3.932429in}{1.217606in}}%
\pgfpathlineto{\pgfqpoint{3.932850in}{1.230741in}}%
\pgfpathlineto{\pgfqpoint{3.933271in}{1.204471in}}%
\pgfpathlineto{\pgfqpoint{3.934114in}{1.217606in}}%
\pgfpathlineto{\pgfqpoint{3.934956in}{1.230741in}}%
\pgfpathlineto{\pgfqpoint{3.935378in}{1.224173in}}%
\pgfpathlineto{\pgfqpoint{3.935799in}{1.211038in}}%
\pgfpathlineto{\pgfqpoint{3.936220in}{1.224173in}}%
\pgfpathlineto{\pgfqpoint{3.936642in}{1.224173in}}%
\pgfpathlineto{\pgfqpoint{3.937063in}{1.211038in}}%
\pgfpathlineto{\pgfqpoint{3.937484in}{1.230741in}}%
\pgfpathlineto{\pgfqpoint{3.937905in}{1.217606in}}%
\pgfpathlineto{\pgfqpoint{3.938327in}{1.211038in}}%
\pgfpathlineto{\pgfqpoint{3.938748in}{1.243876in}}%
\pgfpathlineto{\pgfqpoint{3.939169in}{1.224173in}}%
\pgfpathlineto{\pgfqpoint{3.940433in}{1.211038in}}%
\pgfpathlineto{\pgfqpoint{3.941276in}{1.217606in}}%
\pgfpathlineto{\pgfqpoint{3.941697in}{1.211038in}}%
\pgfpathlineto{\pgfqpoint{3.942961in}{1.230741in}}%
\pgfpathlineto{\pgfqpoint{3.944225in}{1.217606in}}%
\pgfpathlineto{\pgfqpoint{3.945910in}{1.230741in}}%
\pgfpathlineto{\pgfqpoint{3.946331in}{1.224173in}}%
\pgfpathlineto{\pgfqpoint{3.946752in}{1.237308in}}%
\pgfpathlineto{\pgfqpoint{3.947595in}{1.230741in}}%
\pgfpathlineto{\pgfqpoint{3.948437in}{1.217606in}}%
\pgfpathlineto{\pgfqpoint{3.948859in}{1.237308in}}%
\pgfpathlineto{\pgfqpoint{3.949280in}{1.224173in}}%
\pgfpathlineto{\pgfqpoint{3.949701in}{1.217606in}}%
\pgfpathlineto{\pgfqpoint{3.950122in}{1.230741in}}%
\pgfpathlineto{\pgfqpoint{3.950544in}{1.224173in}}%
\pgfpathlineto{\pgfqpoint{3.950965in}{1.217606in}}%
\pgfpathlineto{\pgfqpoint{3.951386in}{1.224173in}}%
\pgfpathlineto{\pgfqpoint{3.951808in}{1.230741in}}%
\pgfpathlineto{\pgfqpoint{3.952229in}{1.217606in}}%
\pgfpathlineto{\pgfqpoint{3.953071in}{1.224173in}}%
\pgfpathlineto{\pgfqpoint{3.954335in}{1.230741in}}%
\pgfpathlineto{\pgfqpoint{3.955599in}{1.217606in}}%
\pgfpathlineto{\pgfqpoint{3.956863in}{1.230741in}}%
\pgfpathlineto{\pgfqpoint{3.958548in}{1.217606in}}%
\pgfpathlineto{\pgfqpoint{3.958969in}{1.224173in}}%
\pgfpathlineto{\pgfqpoint{3.959391in}{1.217606in}}%
\pgfpathlineto{\pgfqpoint{3.959812in}{1.217606in}}%
\pgfpathlineto{\pgfqpoint{3.960233in}{1.224173in}}%
\pgfpathlineto{\pgfqpoint{3.960654in}{1.217606in}}%
\pgfpathlineto{\pgfqpoint{3.961076in}{1.217606in}}%
\pgfpathlineto{\pgfqpoint{3.961497in}{1.237308in}}%
\pgfpathlineto{\pgfqpoint{3.961918in}{1.224173in}}%
\pgfpathlineto{\pgfqpoint{3.963182in}{1.224173in}}%
\pgfpathlineto{\pgfqpoint{3.963603in}{1.230741in}}%
\pgfpathlineto{\pgfqpoint{3.964025in}{1.224173in}}%
\pgfpathlineto{\pgfqpoint{3.964446in}{1.224173in}}%
\pgfpathlineto{\pgfqpoint{3.966131in}{1.204471in}}%
\pgfpathlineto{\pgfqpoint{3.967395in}{1.224173in}}%
\pgfpathlineto{\pgfqpoint{3.968659in}{1.217606in}}%
\pgfpathlineto{\pgfqpoint{3.969080in}{1.224173in}}%
\pgfpathlineto{\pgfqpoint{3.970344in}{1.211038in}}%
\pgfpathlineto{\pgfqpoint{3.971608in}{1.224173in}}%
\pgfpathlineto{\pgfqpoint{3.972029in}{1.224173in}}%
\pgfpathlineto{\pgfqpoint{3.972450in}{1.217606in}}%
\pgfpathlineto{\pgfqpoint{3.972871in}{1.224173in}}%
\pgfpathlineto{\pgfqpoint{3.974135in}{1.230741in}}%
\pgfpathlineto{\pgfqpoint{3.975820in}{1.204471in}}%
\pgfpathlineto{\pgfqpoint{3.976663in}{1.224173in}}%
\pgfpathlineto{\pgfqpoint{3.977084in}{1.211038in}}%
\pgfpathlineto{\pgfqpoint{3.977506in}{1.230741in}}%
\pgfpathlineto{\pgfqpoint{3.977927in}{1.224173in}}%
\pgfpathlineto{\pgfqpoint{3.978348in}{1.230741in}}%
\pgfpathlineto{\pgfqpoint{3.978769in}{1.237308in}}%
\pgfpathlineto{\pgfqpoint{3.979191in}{1.230741in}}%
\pgfpathlineto{\pgfqpoint{3.979612in}{1.230741in}}%
\pgfpathlineto{\pgfqpoint{3.980033in}{1.217606in}}%
\pgfpathlineto{\pgfqpoint{3.980454in}{1.224173in}}%
\pgfpathlineto{\pgfqpoint{3.980876in}{1.230741in}}%
\pgfpathlineto{\pgfqpoint{3.981297in}{1.224173in}}%
\pgfpathlineto{\pgfqpoint{3.981718in}{1.211038in}}%
\pgfpathlineto{\pgfqpoint{3.982140in}{1.224173in}}%
\pgfpathlineto{\pgfqpoint{3.982561in}{1.224173in}}%
\pgfpathlineto{\pgfqpoint{3.983825in}{1.243876in}}%
\pgfpathlineto{\pgfqpoint{3.985089in}{1.224173in}}%
\pgfpathlineto{\pgfqpoint{3.985931in}{1.224173in}}%
\pgfpathlineto{\pgfqpoint{3.986352in}{1.230741in}}%
\pgfpathlineto{\pgfqpoint{3.986774in}{1.217606in}}%
\pgfpathlineto{\pgfqpoint{3.987616in}{1.224173in}}%
\pgfpathlineto{\pgfqpoint{3.988038in}{1.230741in}}%
\pgfpathlineto{\pgfqpoint{3.988459in}{1.224173in}}%
\pgfpathlineto{\pgfqpoint{3.988880in}{1.224173in}}%
\pgfpathlineto{\pgfqpoint{3.989301in}{1.230741in}}%
\pgfpathlineto{\pgfqpoint{3.989723in}{1.224173in}}%
\pgfpathlineto{\pgfqpoint{3.990565in}{1.224173in}}%
\pgfpathlineto{\pgfqpoint{3.990986in}{1.211038in}}%
\pgfpathlineto{\pgfqpoint{3.991408in}{1.230741in}}%
\pgfpathlineto{\pgfqpoint{3.992250in}{1.211038in}}%
\pgfpathlineto{\pgfqpoint{3.992672in}{1.224173in}}%
\pgfpathlineto{\pgfqpoint{3.993093in}{1.237308in}}%
\pgfpathlineto{\pgfqpoint{3.993935in}{1.230741in}}%
\pgfpathlineto{\pgfqpoint{3.994357in}{1.217606in}}%
\pgfpathlineto{\pgfqpoint{3.994778in}{1.224173in}}%
\pgfpathlineto{\pgfqpoint{3.995199in}{1.237308in}}%
\pgfpathlineto{\pgfqpoint{3.995621in}{1.224173in}}%
\pgfpathlineto{\pgfqpoint{3.996463in}{1.224173in}}%
\pgfpathlineto{\pgfqpoint{3.996884in}{1.230741in}}%
\pgfpathlineto{\pgfqpoint{3.998148in}{1.217606in}}%
\pgfpathlineto{\pgfqpoint{3.999412in}{1.224173in}}%
\pgfpathlineto{\pgfqpoint{3.999833in}{1.217606in}}%
\pgfpathlineto{\pgfqpoint{4.000255in}{1.224173in}}%
\pgfpathlineto{\pgfqpoint{4.000676in}{1.230741in}}%
\pgfpathlineto{\pgfqpoint{4.001097in}{1.217606in}}%
\pgfpathlineto{\pgfqpoint{4.001518in}{1.224173in}}%
\pgfpathlineto{\pgfqpoint{4.003204in}{1.237308in}}%
\pgfpathlineto{\pgfqpoint{4.004889in}{1.211038in}}%
\pgfpathlineto{\pgfqpoint{4.006152in}{1.230741in}}%
\pgfpathlineto{\pgfqpoint{4.007416in}{1.211038in}}%
\pgfpathlineto{\pgfqpoint{4.008259in}{1.230741in}}%
\pgfpathlineto{\pgfqpoint{4.008680in}{1.224173in}}%
\pgfpathlineto{\pgfqpoint{4.009944in}{1.224173in}}%
\pgfpathlineto{\pgfqpoint{4.010365in}{1.217606in}}%
\pgfpathlineto{\pgfqpoint{4.010787in}{1.224173in}}%
\pgfpathlineto{\pgfqpoint{4.011208in}{1.224173in}}%
\pgfpathlineto{\pgfqpoint{4.011629in}{1.230741in}}%
\pgfpathlineto{\pgfqpoint{4.012050in}{1.224173in}}%
\pgfpathlineto{\pgfqpoint{4.012472in}{1.224173in}}%
\pgfpathlineto{\pgfqpoint{4.012893in}{1.230741in}}%
\pgfpathlineto{\pgfqpoint{4.013314in}{1.224173in}}%
\pgfpathlineto{\pgfqpoint{4.013735in}{1.224173in}}%
\pgfpathlineto{\pgfqpoint{4.014999in}{1.217606in}}%
\pgfpathlineto{\pgfqpoint{4.015421in}{1.217606in}}%
\pgfpathlineto{\pgfqpoint{4.015842in}{1.224173in}}%
\pgfpathlineto{\pgfqpoint{4.016263in}{1.217606in}}%
\pgfpathlineto{\pgfqpoint{4.016684in}{1.217606in}}%
\pgfpathlineto{\pgfqpoint{4.017948in}{1.230741in}}%
\pgfpathlineto{\pgfqpoint{4.018791in}{1.217606in}}%
\pgfpathlineto{\pgfqpoint{4.019212in}{1.230741in}}%
\pgfpathlineto{\pgfqpoint{4.020055in}{1.224173in}}%
\pgfpathlineto{\pgfqpoint{4.020476in}{1.224173in}}%
\pgfpathlineto{\pgfqpoint{4.020897in}{1.230741in}}%
\pgfpathlineto{\pgfqpoint{4.021318in}{1.224173in}}%
\pgfpathlineto{\pgfqpoint{4.022582in}{1.217606in}}%
\pgfpathlineto{\pgfqpoint{4.023425in}{1.230741in}}%
\pgfpathlineto{\pgfqpoint{4.023846in}{1.224173in}}%
\pgfpathlineto{\pgfqpoint{4.024689in}{1.224173in}}%
\pgfpathlineto{\pgfqpoint{4.025110in}{1.217606in}}%
\pgfpathlineto{\pgfqpoint{4.025531in}{1.224173in}}%
\pgfpathlineto{\pgfqpoint{4.025953in}{1.230741in}}%
\pgfpathlineto{\pgfqpoint{4.026374in}{1.217606in}}%
\pgfpathlineto{\pgfqpoint{4.026795in}{1.230741in}}%
\pgfpathlineto{\pgfqpoint{4.027216in}{1.230741in}}%
\pgfpathlineto{\pgfqpoint{4.028059in}{1.224173in}}%
\pgfpathlineto{\pgfqpoint{4.028480in}{1.230741in}}%
\pgfpathlineto{\pgfqpoint{4.028901in}{1.224173in}}%
\pgfpathlineto{\pgfqpoint{4.029323in}{1.217606in}}%
\pgfpathlineto{\pgfqpoint{4.029744in}{1.237308in}}%
\pgfpathlineto{\pgfqpoint{4.030165in}{1.224173in}}%
\pgfpathlineto{\pgfqpoint{4.030587in}{1.211038in}}%
\pgfpathlineto{\pgfqpoint{4.031008in}{1.224173in}}%
\pgfpathlineto{\pgfqpoint{4.031429in}{1.224173in}}%
\pgfpathlineto{\pgfqpoint{4.031850in}{1.237308in}}%
\pgfpathlineto{\pgfqpoint{4.032272in}{1.224173in}}%
\pgfpathlineto{\pgfqpoint{4.032693in}{1.224173in}}%
\pgfpathlineto{\pgfqpoint{4.033114in}{1.237308in}}%
\pgfpathlineto{\pgfqpoint{4.033536in}{1.224173in}}%
\pgfpathlineto{\pgfqpoint{4.033957in}{1.224173in}}%
\pgfpathlineto{\pgfqpoint{4.034378in}{1.217606in}}%
\pgfpathlineto{\pgfqpoint{4.036063in}{1.237308in}}%
\pgfpathlineto{\pgfqpoint{4.036484in}{1.217606in}}%
\pgfpathlineto{\pgfqpoint{4.037327in}{1.224173in}}%
\pgfpathlineto{\pgfqpoint{4.037748in}{1.224173in}}%
\pgfpathlineto{\pgfqpoint{4.038170in}{1.230741in}}%
\pgfpathlineto{\pgfqpoint{4.039012in}{1.217606in}}%
\pgfpathlineto{\pgfqpoint{4.039433in}{1.224173in}}%
\pgfpathlineto{\pgfqpoint{4.039855in}{1.224173in}}%
\pgfpathlineto{\pgfqpoint{4.040697in}{1.211038in}}%
\pgfpathlineto{\pgfqpoint{4.041540in}{1.230741in}}%
\pgfpathlineto{\pgfqpoint{4.041961in}{1.224173in}}%
\pgfpathlineto{\pgfqpoint{4.042382in}{1.224173in}}%
\pgfpathlineto{\pgfqpoint{4.042804in}{1.230741in}}%
\pgfpathlineto{\pgfqpoint{4.043225in}{1.211038in}}%
\pgfpathlineto{\pgfqpoint{4.043646in}{1.224173in}}%
\pgfpathlineto{\pgfqpoint{4.044067in}{1.224173in}}%
\pgfpathlineto{\pgfqpoint{4.044489in}{1.237308in}}%
\pgfpathlineto{\pgfqpoint{4.044910in}{1.217606in}}%
\pgfpathlineto{\pgfqpoint{4.045331in}{1.217606in}}%
\pgfpathlineto{\pgfqpoint{4.045753in}{1.224173in}}%
\pgfpathlineto{\pgfqpoint{4.046174in}{1.217606in}}%
\pgfpathlineto{\pgfqpoint{4.046595in}{1.211038in}}%
\pgfpathlineto{\pgfqpoint{4.047438in}{1.237308in}}%
\pgfpathlineto{\pgfqpoint{4.047859in}{1.224173in}}%
\pgfpathlineto{\pgfqpoint{4.048280in}{1.224173in}}%
\pgfpathlineto{\pgfqpoint{4.049544in}{1.217606in}}%
\pgfpathlineto{\pgfqpoint{4.050387in}{1.217606in}}%
\pgfpathlineto{\pgfqpoint{4.052493in}{1.237308in}}%
\pgfpathlineto{\pgfqpoint{4.052914in}{1.217606in}}%
\pgfpathlineto{\pgfqpoint{4.053757in}{1.224173in}}%
\pgfpathlineto{\pgfqpoint{4.054178in}{1.230741in}}%
\pgfpathlineto{\pgfqpoint{4.054599in}{1.211038in}}%
\pgfpathlineto{\pgfqpoint{4.055442in}{1.217606in}}%
\pgfpathlineto{\pgfqpoint{4.056706in}{1.230741in}}%
\pgfpathlineto{\pgfqpoint{4.057970in}{1.211038in}}%
\pgfpathlineto{\pgfqpoint{4.059234in}{1.224173in}}%
\pgfpathlineto{\pgfqpoint{4.059655in}{1.224173in}}%
\pgfpathlineto{\pgfqpoint{4.060497in}{1.230741in}}%
\pgfpathlineto{\pgfqpoint{4.060919in}{1.211038in}}%
\pgfpathlineto{\pgfqpoint{4.061761in}{1.217606in}}%
\pgfpathlineto{\pgfqpoint{4.063446in}{1.237308in}}%
\pgfpathlineto{\pgfqpoint{4.064710in}{1.217606in}}%
\pgfpathlineto{\pgfqpoint{4.065131in}{1.224173in}}%
\pgfpathlineto{\pgfqpoint{4.065553in}{1.211038in}}%
\pgfpathlineto{\pgfqpoint{4.065974in}{1.217606in}}%
\pgfpathlineto{\pgfqpoint{4.066395in}{1.230741in}}%
\pgfpathlineto{\pgfqpoint{4.067238in}{1.224173in}}%
\pgfpathlineto{\pgfqpoint{4.067659in}{1.224173in}}%
\pgfpathlineto{\pgfqpoint{4.068080in}{1.237308in}}%
\pgfpathlineto{\pgfqpoint{4.068502in}{1.224173in}}%
\pgfpathlineto{\pgfqpoint{4.068923in}{1.224173in}}%
\pgfpathlineto{\pgfqpoint{4.070608in}{1.211038in}}%
\pgfpathlineto{\pgfqpoint{4.071029in}{1.237308in}}%
\pgfpathlineto{\pgfqpoint{4.071872in}{1.230741in}}%
\pgfpathlineto{\pgfqpoint{4.072293in}{1.237308in}}%
\pgfpathlineto{\pgfqpoint{4.073978in}{1.217606in}}%
\pgfpathlineto{\pgfqpoint{4.075663in}{1.230741in}}%
\pgfpathlineto{\pgfqpoint{4.076927in}{1.217606in}}%
\pgfpathlineto{\pgfqpoint{4.078191in}{1.230741in}}%
\pgfpathlineto{\pgfqpoint{4.079876in}{1.211038in}}%
\pgfpathlineto{\pgfqpoint{4.081140in}{1.224173in}}%
\pgfpathlineto{\pgfqpoint{4.081561in}{1.217606in}}%
\pgfpathlineto{\pgfqpoint{4.081983in}{1.224173in}}%
\pgfpathlineto{\pgfqpoint{4.083668in}{1.243876in}}%
\pgfpathlineto{\pgfqpoint{4.085353in}{1.217606in}}%
\pgfpathlineto{\pgfqpoint{4.085774in}{1.217606in}}%
\pgfpathlineto{\pgfqpoint{4.086617in}{1.230741in}}%
\pgfpathlineto{\pgfqpoint{4.087038in}{1.224173in}}%
\pgfpathlineto{\pgfqpoint{4.087459in}{1.217606in}}%
\pgfpathlineto{\pgfqpoint{4.087880in}{1.224173in}}%
\pgfpathlineto{\pgfqpoint{4.088302in}{1.230741in}}%
\pgfpathlineto{\pgfqpoint{4.088723in}{1.224173in}}%
\pgfpathlineto{\pgfqpoint{4.089144in}{1.217606in}}%
\pgfpathlineto{\pgfqpoint{4.089566in}{1.224173in}}%
\pgfpathlineto{\pgfqpoint{4.089987in}{1.230741in}}%
\pgfpathlineto{\pgfqpoint{4.090408in}{1.224173in}}%
\pgfpathlineto{\pgfqpoint{4.091251in}{1.224173in}}%
\pgfpathlineto{\pgfqpoint{4.091672in}{1.237308in}}%
\pgfpathlineto{\pgfqpoint{4.092093in}{1.217606in}}%
\pgfpathlineto{\pgfqpoint{4.092514in}{1.217606in}}%
\pgfpathlineto{\pgfqpoint{4.093357in}{1.224173in}}%
\pgfpathlineto{\pgfqpoint{4.093778in}{1.217606in}}%
\pgfpathlineto{\pgfqpoint{4.094200in}{1.224173in}}%
\pgfpathlineto{\pgfqpoint{4.094621in}{1.224173in}}%
\pgfpathlineto{\pgfqpoint{4.095463in}{1.230741in}}%
\pgfpathlineto{\pgfqpoint{4.095885in}{1.211038in}}%
\pgfpathlineto{\pgfqpoint{4.096306in}{1.224173in}}%
\pgfpathlineto{\pgfqpoint{4.096727in}{1.224173in}}%
\pgfpathlineto{\pgfqpoint{4.097149in}{1.211038in}}%
\pgfpathlineto{\pgfqpoint{4.097570in}{1.224173in}}%
\pgfpathlineto{\pgfqpoint{4.098834in}{1.224173in}}%
\pgfpathlineto{\pgfqpoint{4.099255in}{1.230741in}}%
\pgfpathlineto{\pgfqpoint{4.099676in}{1.224173in}}%
\pgfpathlineto{\pgfqpoint{4.100097in}{1.211038in}}%
\pgfpathlineto{\pgfqpoint{4.100519in}{1.224173in}}%
\pgfpathlineto{\pgfqpoint{4.100940in}{1.230741in}}%
\pgfpathlineto{\pgfqpoint{4.102204in}{1.217606in}}%
\pgfpathlineto{\pgfqpoint{4.103046in}{1.224173in}}%
\pgfpathlineto{\pgfqpoint{4.103468in}{1.217606in}}%
\pgfpathlineto{\pgfqpoint{4.104732in}{1.230741in}}%
\pgfpathlineto{\pgfqpoint{4.105153in}{1.211038in}}%
\pgfpathlineto{\pgfqpoint{4.105574in}{1.243876in}}%
\pgfpathlineto{\pgfqpoint{4.106417in}{1.224173in}}%
\pgfpathlineto{\pgfqpoint{4.107259in}{1.243876in}}%
\pgfpathlineto{\pgfqpoint{4.107680in}{1.230741in}}%
\pgfpathlineto{\pgfqpoint{4.108102in}{1.230741in}}%
\pgfpathlineto{\pgfqpoint{4.108523in}{1.217606in}}%
\pgfpathlineto{\pgfqpoint{4.108944in}{1.224173in}}%
\pgfpathlineto{\pgfqpoint{4.110208in}{1.230741in}}%
\pgfpathlineto{\pgfqpoint{4.111051in}{1.230741in}}%
\pgfpathlineto{\pgfqpoint{4.112736in}{1.217606in}}%
\pgfpathlineto{\pgfqpoint{4.114421in}{1.237308in}}%
\pgfpathlineto{\pgfqpoint{4.115263in}{1.211038in}}%
\pgfpathlineto{\pgfqpoint{4.116106in}{1.224173in}}%
\pgfpathlineto{\pgfqpoint{4.116527in}{1.224173in}}%
\pgfpathlineto{\pgfqpoint{4.116949in}{1.230741in}}%
\pgfpathlineto{\pgfqpoint{4.117370in}{1.224173in}}%
\pgfpathlineto{\pgfqpoint{4.118634in}{1.224173in}}%
\pgfpathlineto{\pgfqpoint{4.119055in}{1.217606in}}%
\pgfpathlineto{\pgfqpoint{4.119476in}{1.230741in}}%
\pgfpathlineto{\pgfqpoint{4.119898in}{1.224173in}}%
\pgfpathlineto{\pgfqpoint{4.120740in}{1.204471in}}%
\pgfpathlineto{\pgfqpoint{4.121161in}{1.211038in}}%
\pgfpathlineto{\pgfqpoint{4.122004in}{1.224173in}}%
\pgfpathlineto{\pgfqpoint{4.122425in}{1.217606in}}%
\pgfpathlineto{\pgfqpoint{4.122847in}{1.217606in}}%
\pgfpathlineto{\pgfqpoint{4.124110in}{1.224173in}}%
\pgfpathlineto{\pgfqpoint{4.124953in}{1.224173in}}%
\pgfpathlineto{\pgfqpoint{4.126217in}{1.211038in}}%
\pgfpathlineto{\pgfqpoint{4.127481in}{1.224173in}}%
\pgfpathlineto{\pgfqpoint{4.128323in}{1.224173in}}%
\pgfpathlineto{\pgfqpoint{4.129166in}{1.230741in}}%
\pgfpathlineto{\pgfqpoint{4.130851in}{1.211038in}}%
\pgfpathlineto{\pgfqpoint{4.132115in}{1.230741in}}%
\pgfpathlineto{\pgfqpoint{4.132536in}{1.217606in}}%
\pgfpathlineto{\pgfqpoint{4.133378in}{1.224173in}}%
\pgfpathlineto{\pgfqpoint{4.134221in}{1.237308in}}%
\pgfpathlineto{\pgfqpoint{4.134642in}{1.217606in}}%
\pgfpathlineto{\pgfqpoint{4.135485in}{1.224173in}}%
\pgfpathlineto{\pgfqpoint{4.135906in}{1.230741in}}%
\pgfpathlineto{\pgfqpoint{4.136327in}{1.224173in}}%
\pgfpathlineto{\pgfqpoint{4.137591in}{1.217606in}}%
\pgfpathlineto{\pgfqpoint{4.138013in}{1.217606in}}%
\pgfpathlineto{\pgfqpoint{4.139698in}{1.243876in}}%
\pgfpathlineto{\pgfqpoint{4.140540in}{1.204471in}}%
\pgfpathlineto{\pgfqpoint{4.140961in}{1.224173in}}%
\pgfpathlineto{\pgfqpoint{4.141383in}{1.217606in}}%
\pgfpathlineto{\pgfqpoint{4.141804in}{1.230741in}}%
\pgfpathlineto{\pgfqpoint{4.142647in}{1.224173in}}%
\pgfpathlineto{\pgfqpoint{4.143068in}{1.224173in}}%
\pgfpathlineto{\pgfqpoint{4.144332in}{1.237308in}}%
\pgfpathlineto{\pgfqpoint{4.146438in}{1.217606in}}%
\pgfpathlineto{\pgfqpoint{4.147281in}{1.237308in}}%
\pgfpathlineto{\pgfqpoint{4.148544in}{1.217606in}}%
\pgfpathlineto{\pgfqpoint{4.148966in}{1.230741in}}%
\pgfpathlineto{\pgfqpoint{4.149808in}{1.224173in}}%
\pgfpathlineto{\pgfqpoint{4.150230in}{1.230741in}}%
\pgfpathlineto{\pgfqpoint{4.150651in}{1.224173in}}%
\pgfpathlineto{\pgfqpoint{4.152336in}{1.224173in}}%
\pgfpathlineto{\pgfqpoint{4.152757in}{1.217606in}}%
\pgfpathlineto{\pgfqpoint{4.153179in}{1.224173in}}%
\pgfpathlineto{\pgfqpoint{4.153600in}{1.224173in}}%
\pgfpathlineto{\pgfqpoint{4.154021in}{1.237308in}}%
\pgfpathlineto{\pgfqpoint{4.154864in}{1.230741in}}%
\pgfpathlineto{\pgfqpoint{4.155706in}{1.217606in}}%
\pgfpathlineto{\pgfqpoint{4.156127in}{1.224173in}}%
\pgfpathlineto{\pgfqpoint{4.156549in}{1.224173in}}%
\pgfpathlineto{\pgfqpoint{4.157391in}{1.217606in}}%
\pgfpathlineto{\pgfqpoint{4.158655in}{1.230741in}}%
\pgfpathlineto{\pgfqpoint{4.159076in}{1.217606in}}%
\pgfpathlineto{\pgfqpoint{4.159498in}{1.224173in}}%
\pgfpathlineto{\pgfqpoint{4.160340in}{1.230741in}}%
\pgfpathlineto{\pgfqpoint{4.160762in}{1.217606in}}%
\pgfpathlineto{\pgfqpoint{4.161604in}{1.224173in}}%
\pgfpathlineto{\pgfqpoint{4.162025in}{1.224173in}}%
\pgfpathlineto{\pgfqpoint{4.162868in}{1.230741in}}%
\pgfpathlineto{\pgfqpoint{4.163710in}{1.217606in}}%
\pgfpathlineto{\pgfqpoint{4.164132in}{1.224173in}}%
\pgfpathlineto{\pgfqpoint{4.164553in}{1.224173in}}%
\pgfpathlineto{\pgfqpoint{4.164974in}{1.230741in}}%
\pgfpathlineto{\pgfqpoint{4.165396in}{1.224173in}}%
\pgfpathlineto{\pgfqpoint{4.165817in}{1.217606in}}%
\pgfpathlineto{\pgfqpoint{4.166238in}{1.224173in}}%
\pgfpathlineto{\pgfqpoint{4.166659in}{1.224173in}}%
\pgfpathlineto{\pgfqpoint{4.167081in}{1.217606in}}%
\pgfpathlineto{\pgfqpoint{4.167502in}{1.224173in}}%
\pgfpathlineto{\pgfqpoint{4.167923in}{1.224173in}}%
\pgfpathlineto{\pgfqpoint{4.168766in}{1.217606in}}%
\pgfpathlineto{\pgfqpoint{4.169608in}{1.237308in}}%
\pgfpathlineto{\pgfqpoint{4.171293in}{1.217606in}}%
\pgfpathlineto{\pgfqpoint{4.172136in}{1.217606in}}%
\pgfpathlineto{\pgfqpoint{4.172979in}{1.224173in}}%
\pgfpathlineto{\pgfqpoint{4.173400in}{1.217606in}}%
\pgfpathlineto{\pgfqpoint{4.173821in}{1.230741in}}%
\pgfpathlineto{\pgfqpoint{4.174242in}{1.217606in}}%
\pgfpathlineto{\pgfqpoint{4.174664in}{1.217606in}}%
\pgfpathlineto{\pgfqpoint{4.175085in}{1.230741in}}%
\pgfpathlineto{\pgfqpoint{4.175506in}{1.224173in}}%
\pgfpathlineto{\pgfqpoint{4.175928in}{1.204471in}}%
\pgfpathlineto{\pgfqpoint{4.176349in}{1.217606in}}%
\pgfpathlineto{\pgfqpoint{4.177191in}{1.230741in}}%
\pgfpathlineto{\pgfqpoint{4.177613in}{1.224173in}}%
\pgfpathlineto{\pgfqpoint{4.178034in}{1.224173in}}%
\pgfpathlineto{\pgfqpoint{4.178455in}{1.217606in}}%
\pgfpathlineto{\pgfqpoint{4.179719in}{1.230741in}}%
\pgfpathlineto{\pgfqpoint{4.180983in}{1.224173in}}%
\pgfpathlineto{\pgfqpoint{4.181404in}{1.243876in}}%
\pgfpathlineto{\pgfqpoint{4.181825in}{1.217606in}}%
\pgfpathlineto{\pgfqpoint{4.182668in}{1.230741in}}%
\pgfpathlineto{\pgfqpoint{4.183089in}{1.211038in}}%
\pgfpathlineto{\pgfqpoint{4.183511in}{1.230741in}}%
\pgfpathlineto{\pgfqpoint{4.183932in}{1.230741in}}%
\pgfpathlineto{\pgfqpoint{4.184353in}{1.217606in}}%
\pgfpathlineto{\pgfqpoint{4.184774in}{1.237308in}}%
\pgfpathlineto{\pgfqpoint{4.185196in}{1.224173in}}%
\pgfpathlineto{\pgfqpoint{4.186460in}{1.237308in}}%
\pgfpathlineto{\pgfqpoint{4.187302in}{1.211038in}}%
\pgfpathlineto{\pgfqpoint{4.188145in}{1.217606in}}%
\pgfpathlineto{\pgfqpoint{4.189408in}{1.230741in}}%
\pgfpathlineto{\pgfqpoint{4.190672in}{1.217606in}}%
\pgfpathlineto{\pgfqpoint{4.191936in}{1.230741in}}%
\pgfpathlineto{\pgfqpoint{4.193200in}{1.217606in}}%
\pgfpathlineto{\pgfqpoint{4.193621in}{1.217606in}}%
\pgfpathlineto{\pgfqpoint{4.194043in}{1.230741in}}%
\pgfpathlineto{\pgfqpoint{4.194464in}{1.224173in}}%
\pgfpathlineto{\pgfqpoint{4.194885in}{1.211038in}}%
\pgfpathlineto{\pgfqpoint{4.195306in}{1.230741in}}%
\pgfpathlineto{\pgfqpoint{4.196149in}{1.217606in}}%
\pgfpathlineto{\pgfqpoint{4.196570in}{1.230741in}}%
\pgfpathlineto{\pgfqpoint{4.196991in}{1.217606in}}%
\pgfpathlineto{\pgfqpoint{4.197413in}{1.217606in}}%
\pgfpathlineto{\pgfqpoint{4.197834in}{1.224173in}}%
\pgfpathlineto{\pgfqpoint{4.198255in}{1.211038in}}%
\pgfpathlineto{\pgfqpoint{4.198677in}{1.224173in}}%
\pgfpathlineto{\pgfqpoint{4.199098in}{1.224173in}}%
\pgfpathlineto{\pgfqpoint{4.199940in}{1.211038in}}%
\pgfpathlineto{\pgfqpoint{4.200362in}{1.217606in}}%
\pgfpathlineto{\pgfqpoint{4.201626in}{1.237308in}}%
\pgfpathlineto{\pgfqpoint{4.202468in}{1.217606in}}%
\pgfpathlineto{\pgfqpoint{4.202889in}{1.224173in}}%
\pgfpathlineto{\pgfqpoint{4.203311in}{1.217606in}}%
\pgfpathlineto{\pgfqpoint{4.203732in}{1.224173in}}%
\pgfpathlineto{\pgfqpoint{4.204574in}{1.224173in}}%
\pgfpathlineto{\pgfqpoint{4.204996in}{1.237308in}}%
\pgfpathlineto{\pgfqpoint{4.205417in}{1.217606in}}%
\pgfpathlineto{\pgfqpoint{4.207102in}{1.230741in}}%
\pgfpathlineto{\pgfqpoint{4.208787in}{1.211038in}}%
\pgfpathlineto{\pgfqpoint{4.210472in}{1.237308in}}%
\pgfpathlineto{\pgfqpoint{4.211736in}{1.224173in}}%
\pgfpathlineto{\pgfqpoint{4.212157in}{1.224173in}}%
\pgfpathlineto{\pgfqpoint{4.212579in}{1.217606in}}%
\pgfpathlineto{\pgfqpoint{4.213000in}{1.224173in}}%
\pgfpathlineto{\pgfqpoint{4.214264in}{1.230741in}}%
\pgfpathlineto{\pgfqpoint{4.215106in}{1.224173in}}%
\pgfpathlineto{\pgfqpoint{4.216370in}{1.243876in}}%
\pgfpathlineto{\pgfqpoint{4.217634in}{1.224173in}}%
\pgfpathlineto{\pgfqpoint{4.218055in}{1.230741in}}%
\pgfpathlineto{\pgfqpoint{4.218477in}{1.204471in}}%
\pgfpathlineto{\pgfqpoint{4.218898in}{1.224173in}}%
\pgfpathlineto{\pgfqpoint{4.219319in}{1.230741in}}%
\pgfpathlineto{\pgfqpoint{4.219740in}{1.224173in}}%
\pgfpathlineto{\pgfqpoint{4.220162in}{1.224173in}}%
\pgfpathlineto{\pgfqpoint{4.220583in}{1.217606in}}%
\pgfpathlineto{\pgfqpoint{4.221004in}{1.230741in}}%
\pgfpathlineto{\pgfqpoint{4.221847in}{1.224173in}}%
\pgfpathlineto{\pgfqpoint{4.222268in}{1.230741in}}%
\pgfpathlineto{\pgfqpoint{4.222689in}{1.224173in}}%
\pgfpathlineto{\pgfqpoint{4.223953in}{1.224173in}}%
\pgfpathlineto{\pgfqpoint{4.225217in}{1.230741in}}%
\pgfpathlineto{\pgfqpoint{4.226481in}{1.224173in}}%
\pgfpathlineto{\pgfqpoint{4.226902in}{1.224173in}}%
\pgfpathlineto{\pgfqpoint{4.227323in}{1.230741in}}%
\pgfpathlineto{\pgfqpoint{4.228587in}{1.211038in}}%
\pgfpathlineto{\pgfqpoint{4.229430in}{1.224173in}}%
\pgfpathlineto{\pgfqpoint{4.229851in}{1.204471in}}%
\pgfpathlineto{\pgfqpoint{4.230272in}{1.230741in}}%
\pgfpathlineto{\pgfqpoint{4.231536in}{1.217606in}}%
\pgfpathlineto{\pgfqpoint{4.232379in}{1.230741in}}%
\pgfpathlineto{\pgfqpoint{4.232800in}{1.217606in}}%
\pgfpathlineto{\pgfqpoint{4.233643in}{1.224173in}}%
\pgfpathlineto{\pgfqpoint{4.234064in}{1.230741in}}%
\pgfpathlineto{\pgfqpoint{4.234485in}{1.224173in}}%
\pgfpathlineto{\pgfqpoint{4.234906in}{1.217606in}}%
\pgfpathlineto{\pgfqpoint{4.235328in}{1.224173in}}%
\pgfpathlineto{\pgfqpoint{4.235749in}{1.237308in}}%
\pgfpathlineto{\pgfqpoint{4.236170in}{1.224173in}}%
\pgfpathlineto{\pgfqpoint{4.236592in}{1.224173in}}%
\pgfpathlineto{\pgfqpoint{4.237434in}{1.211038in}}%
\pgfpathlineto{\pgfqpoint{4.237855in}{1.217606in}}%
\pgfpathlineto{\pgfqpoint{4.239541in}{1.237308in}}%
\pgfpathlineto{\pgfqpoint{4.241226in}{1.217606in}}%
\pgfpathlineto{\pgfqpoint{4.241647in}{1.217606in}}%
\pgfpathlineto{\pgfqpoint{4.242489in}{1.230741in}}%
\pgfpathlineto{\pgfqpoint{4.242911in}{1.204471in}}%
\pgfpathlineto{\pgfqpoint{4.243332in}{1.230741in}}%
\pgfpathlineto{\pgfqpoint{4.244596in}{1.217606in}}%
\pgfpathlineto{\pgfqpoint{4.245438in}{1.230741in}}%
\pgfpathlineto{\pgfqpoint{4.245860in}{1.224173in}}%
\pgfpathlineto{\pgfqpoint{4.246702in}{1.224173in}}%
\pgfpathlineto{\pgfqpoint{4.247124in}{1.230741in}}%
\pgfpathlineto{\pgfqpoint{4.247966in}{1.217606in}}%
\pgfpathlineto{\pgfqpoint{4.248387in}{1.224173in}}%
\pgfpathlineto{\pgfqpoint{4.248809in}{1.237308in}}%
\pgfpathlineto{\pgfqpoint{4.249230in}{1.211038in}}%
\pgfpathlineto{\pgfqpoint{4.249651in}{1.230741in}}%
\pgfpathlineto{\pgfqpoint{4.250494in}{1.230741in}}%
\pgfpathlineto{\pgfqpoint{4.251758in}{1.224173in}}%
\pgfpathlineto{\pgfqpoint{4.252179in}{1.224173in}}%
\pgfpathlineto{\pgfqpoint{4.253021in}{1.211038in}}%
\pgfpathlineto{\pgfqpoint{4.254707in}{1.243876in}}%
\pgfpathlineto{\pgfqpoint{4.255970in}{1.224173in}}%
\pgfpathlineto{\pgfqpoint{4.256392in}{1.237308in}}%
\pgfpathlineto{\pgfqpoint{4.256813in}{1.217606in}}%
\pgfpathlineto{\pgfqpoint{4.257234in}{1.224173in}}%
\pgfpathlineto{\pgfqpoint{4.257656in}{1.217606in}}%
\pgfpathlineto{\pgfqpoint{4.258919in}{1.217606in}}%
\pgfpathlineto{\pgfqpoint{4.259341in}{1.211038in}}%
\pgfpathlineto{\pgfqpoint{4.259762in}{1.230741in}}%
\pgfpathlineto{\pgfqpoint{4.260183in}{1.217606in}}%
\pgfpathlineto{\pgfqpoint{4.260604in}{1.211038in}}%
\pgfpathlineto{\pgfqpoint{4.261447in}{1.237308in}}%
\pgfpathlineto{\pgfqpoint{4.261868in}{1.230741in}}%
\pgfpathlineto{\pgfqpoint{4.263553in}{1.217606in}}%
\pgfpathlineto{\pgfqpoint{4.264817in}{1.230741in}}%
\pgfpathlineto{\pgfqpoint{4.266081in}{1.224173in}}%
\pgfpathlineto{\pgfqpoint{4.266502in}{1.224173in}}%
\pgfpathlineto{\pgfqpoint{4.267345in}{1.217606in}}%
\pgfpathlineto{\pgfqpoint{4.267766in}{1.224173in}}%
\pgfpathlineto{\pgfqpoint{4.268187in}{1.211038in}}%
\pgfpathlineto{\pgfqpoint{4.268609in}{1.230741in}}%
\pgfpathlineto{\pgfqpoint{4.269030in}{1.217606in}}%
\pgfpathlineto{\pgfqpoint{4.269451in}{1.217606in}}%
\pgfpathlineto{\pgfqpoint{4.269873in}{1.230741in}}%
\pgfpathlineto{\pgfqpoint{4.270715in}{1.224173in}}%
\pgfpathlineto{\pgfqpoint{4.271136in}{1.224173in}}%
\pgfpathlineto{\pgfqpoint{4.271979in}{1.237308in}}%
\pgfpathlineto{\pgfqpoint{4.272400in}{1.230741in}}%
\pgfpathlineto{\pgfqpoint{4.272822in}{1.230741in}}%
\pgfpathlineto{\pgfqpoint{4.273243in}{1.217606in}}%
\pgfpathlineto{\pgfqpoint{4.273664in}{1.243876in}}%
\pgfpathlineto{\pgfqpoint{4.274507in}{1.230741in}}%
\pgfpathlineto{\pgfqpoint{4.275770in}{1.204471in}}%
\pgfpathlineto{\pgfqpoint{4.277456in}{1.250443in}}%
\pgfpathlineto{\pgfqpoint{4.277877in}{1.243876in}}%
\pgfpathlineto{\pgfqpoint{4.278298in}{1.217606in}}%
\pgfpathlineto{\pgfqpoint{4.279141in}{1.230741in}}%
\pgfpathlineto{\pgfqpoint{4.279562in}{1.230741in}}%
\pgfpathlineto{\pgfqpoint{4.279983in}{1.250443in}}%
\pgfpathlineto{\pgfqpoint{4.280405in}{1.230741in}}%
\pgfpathlineto{\pgfqpoint{4.280826in}{1.230741in}}%
\pgfpathlineto{\pgfqpoint{4.282090in}{1.211038in}}%
\pgfpathlineto{\pgfqpoint{4.282511in}{1.230741in}}%
\pgfpathlineto{\pgfqpoint{4.282932in}{1.224173in}}%
\pgfpathlineto{\pgfqpoint{4.284196in}{1.217606in}}%
\pgfpathlineto{\pgfqpoint{4.284617in}{1.230741in}}%
\pgfpathlineto{\pgfqpoint{4.285039in}{1.217606in}}%
\pgfpathlineto{\pgfqpoint{4.285460in}{1.211038in}}%
\pgfpathlineto{\pgfqpoint{4.286724in}{1.230741in}}%
\pgfpathlineto{\pgfqpoint{4.287145in}{1.211038in}}%
\pgfpathlineto{\pgfqpoint{4.287988in}{1.217606in}}%
\pgfpathlineto{\pgfqpoint{4.289251in}{1.237308in}}%
\pgfpathlineto{\pgfqpoint{4.290936in}{1.191335in}}%
\pgfpathlineto{\pgfqpoint{4.291779in}{1.230741in}}%
\pgfpathlineto{\pgfqpoint{4.292200in}{1.217606in}}%
\pgfpathlineto{\pgfqpoint{4.292622in}{1.217606in}}%
\pgfpathlineto{\pgfqpoint{4.293043in}{1.211038in}}%
\pgfpathlineto{\pgfqpoint{4.293464in}{1.184768in}}%
\pgfpathlineto{\pgfqpoint{4.293885in}{1.211038in}}%
\pgfpathlineto{\pgfqpoint{4.295149in}{1.224173in}}%
\pgfpathlineto{\pgfqpoint{4.295992in}{1.224173in}}%
\pgfpathlineto{\pgfqpoint{4.297256in}{1.230741in}}%
\pgfpathlineto{\pgfqpoint{4.298098in}{1.217606in}}%
\pgfpathlineto{\pgfqpoint{4.298519in}{1.224173in}}%
\pgfpathlineto{\pgfqpoint{4.298941in}{1.237308in}}%
\pgfpathlineto{\pgfqpoint{4.299362in}{1.224173in}}%
\pgfpathlineto{\pgfqpoint{4.300626in}{1.224173in}}%
\pgfpathlineto{\pgfqpoint{4.301890in}{1.230741in}}%
\pgfpathlineto{\pgfqpoint{4.303575in}{1.217606in}}%
\pgfpathlineto{\pgfqpoint{4.304417in}{1.230741in}}%
\pgfpathlineto{\pgfqpoint{4.304839in}{1.224173in}}%
\pgfpathlineto{\pgfqpoint{4.305260in}{1.217606in}}%
\pgfpathlineto{\pgfqpoint{4.305681in}{1.224173in}}%
\pgfpathlineto{\pgfqpoint{4.306945in}{1.237308in}}%
\pgfpathlineto{\pgfqpoint{4.308209in}{1.224173in}}%
\pgfpathlineto{\pgfqpoint{4.308630in}{1.224173in}}%
\pgfpathlineto{\pgfqpoint{4.309051in}{1.230741in}}%
\pgfpathlineto{\pgfqpoint{4.309473in}{1.217606in}}%
\pgfpathlineto{\pgfqpoint{4.310315in}{1.224173in}}%
\pgfpathlineto{\pgfqpoint{4.310737in}{1.217606in}}%
\pgfpathlineto{\pgfqpoint{4.311579in}{1.230741in}}%
\pgfpathlineto{\pgfqpoint{4.312000in}{1.224173in}}%
\pgfpathlineto{\pgfqpoint{4.313264in}{1.217606in}}%
\pgfpathlineto{\pgfqpoint{4.315371in}{1.217606in}}%
\pgfpathlineto{\pgfqpoint{4.316634in}{1.230741in}}%
\pgfpathlineto{\pgfqpoint{4.318320in}{1.217606in}}%
\pgfpathlineto{\pgfqpoint{4.318741in}{1.217606in}}%
\pgfpathlineto{\pgfqpoint{4.320005in}{1.230741in}}%
\pgfpathlineto{\pgfqpoint{4.320426in}{1.230741in}}%
\pgfpathlineto{\pgfqpoint{4.321269in}{1.211038in}}%
\pgfpathlineto{\pgfqpoint{4.321690in}{1.224173in}}%
\pgfpathlineto{\pgfqpoint{4.322532in}{1.224173in}}%
\pgfpathlineto{\pgfqpoint{4.322954in}{1.230741in}}%
\pgfpathlineto{\pgfqpoint{4.323375in}{1.224173in}}%
\pgfpathlineto{\pgfqpoint{4.323796in}{1.211038in}}%
\pgfpathlineto{\pgfqpoint{4.324639in}{1.217606in}}%
\pgfpathlineto{\pgfqpoint{4.325481in}{1.230741in}}%
\pgfpathlineto{\pgfqpoint{4.325903in}{1.217606in}}%
\pgfpathlineto{\pgfqpoint{4.326324in}{1.230741in}}%
\pgfpathlineto{\pgfqpoint{4.327166in}{1.230741in}}%
\pgfpathlineto{\pgfqpoint{4.327588in}{1.217606in}}%
\pgfpathlineto{\pgfqpoint{4.328009in}{1.224173in}}%
\pgfpathlineto{\pgfqpoint{4.328430in}{1.230741in}}%
\pgfpathlineto{\pgfqpoint{4.328852in}{1.224173in}}%
\pgfpathlineto{\pgfqpoint{4.329273in}{1.224173in}}%
\pgfpathlineto{\pgfqpoint{4.329694in}{1.217606in}}%
\pgfpathlineto{\pgfqpoint{4.330115in}{1.224173in}}%
\pgfpathlineto{\pgfqpoint{4.330537in}{1.230741in}}%
\pgfpathlineto{\pgfqpoint{4.330958in}{1.224173in}}%
\pgfpathlineto{\pgfqpoint{4.331379in}{1.224173in}}%
\pgfpathlineto{\pgfqpoint{4.332222in}{1.230741in}}%
\pgfpathlineto{\pgfqpoint{4.332643in}{1.184768in}}%
\pgfpathlineto{\pgfqpoint{4.333486in}{1.211038in}}%
\pgfpathlineto{\pgfqpoint{4.333907in}{1.211038in}}%
\pgfpathlineto{\pgfqpoint{4.335171in}{1.237308in}}%
\pgfpathlineto{\pgfqpoint{4.336856in}{1.217606in}}%
\pgfpathlineto{\pgfqpoint{4.338120in}{1.237308in}}%
\pgfpathlineto{\pgfqpoint{4.338962in}{1.217606in}}%
\pgfpathlineto{\pgfqpoint{4.339383in}{1.224173in}}%
\pgfpathlineto{\pgfqpoint{4.340647in}{1.217606in}}%
\pgfpathlineto{\pgfqpoint{4.341911in}{1.230741in}}%
\pgfpathlineto{\pgfqpoint{4.342754in}{1.217606in}}%
\pgfpathlineto{\pgfqpoint{4.343175in}{1.237308in}}%
\pgfpathlineto{\pgfqpoint{4.343596in}{1.217606in}}%
\pgfpathlineto{\pgfqpoint{4.344018in}{1.217606in}}%
\pgfpathlineto{\pgfqpoint{4.345281in}{1.224173in}}%
\pgfpathlineto{\pgfqpoint{4.345703in}{1.224173in}}%
\pgfpathlineto{\pgfqpoint{4.346124in}{1.263578in}}%
\pgfpathlineto{\pgfqpoint{4.346545in}{1.224173in}}%
\pgfpathlineto{\pgfqpoint{4.347388in}{1.230741in}}%
\pgfpathlineto{\pgfqpoint{4.348652in}{1.211038in}}%
\pgfpathlineto{\pgfqpoint{4.350337in}{1.230741in}}%
\pgfpathlineto{\pgfqpoint{4.351601in}{1.217606in}}%
\pgfpathlineto{\pgfqpoint{4.352443in}{1.230741in}}%
\pgfpathlineto{\pgfqpoint{4.352864in}{1.224173in}}%
\pgfpathlineto{\pgfqpoint{4.353286in}{1.230741in}}%
\pgfpathlineto{\pgfqpoint{4.353707in}{1.224173in}}%
\pgfpathlineto{\pgfqpoint{4.354128in}{1.217606in}}%
\pgfpathlineto{\pgfqpoint{4.354549in}{1.230741in}}%
\pgfpathlineto{\pgfqpoint{4.354971in}{1.217606in}}%
\pgfpathlineto{\pgfqpoint{4.355392in}{1.211038in}}%
\pgfpathlineto{\pgfqpoint{4.355813in}{1.217606in}}%
\pgfpathlineto{\pgfqpoint{4.356235in}{1.217606in}}%
\pgfpathlineto{\pgfqpoint{4.356656in}{1.204471in}}%
\pgfpathlineto{\pgfqpoint{4.357498in}{1.230741in}}%
\pgfpathlineto{\pgfqpoint{4.357920in}{1.224173in}}%
\pgfpathlineto{\pgfqpoint{4.358341in}{1.230741in}}%
\pgfpathlineto{\pgfqpoint{4.358762in}{1.224173in}}%
\pgfpathlineto{\pgfqpoint{4.360026in}{1.224173in}}%
\pgfpathlineto{\pgfqpoint{4.360447in}{1.230741in}}%
\pgfpathlineto{\pgfqpoint{4.360869in}{1.224173in}}%
\pgfpathlineto{\pgfqpoint{4.361290in}{1.217606in}}%
\pgfpathlineto{\pgfqpoint{4.361711in}{1.224173in}}%
\pgfpathlineto{\pgfqpoint{4.362554in}{1.224173in}}%
\pgfpathlineto{\pgfqpoint{4.362975in}{1.230741in}}%
\pgfpathlineto{\pgfqpoint{4.363818in}{1.217606in}}%
\pgfpathlineto{\pgfqpoint{4.364239in}{1.224173in}}%
\pgfpathlineto{\pgfqpoint{4.364660in}{1.237308in}}%
\pgfpathlineto{\pgfqpoint{4.365081in}{1.224173in}}%
\pgfpathlineto{\pgfqpoint{4.365924in}{1.224173in}}%
\pgfpathlineto{\pgfqpoint{4.366345in}{1.211038in}}%
\pgfpathlineto{\pgfqpoint{4.366767in}{1.224173in}}%
\pgfpathlineto{\pgfqpoint{4.367188in}{1.224173in}}%
\pgfpathlineto{\pgfqpoint{4.367609in}{1.230741in}}%
\pgfpathlineto{\pgfqpoint{4.368030in}{1.217606in}}%
\pgfpathlineto{\pgfqpoint{4.368873in}{1.224173in}}%
\pgfpathlineto{\pgfqpoint{4.370137in}{1.237308in}}%
\pgfpathlineto{\pgfqpoint{4.371822in}{1.217606in}}%
\pgfpathlineto{\pgfqpoint{4.373086in}{1.230741in}}%
\pgfpathlineto{\pgfqpoint{4.373928in}{1.217606in}}%
\pgfpathlineto{\pgfqpoint{4.374350in}{1.224173in}}%
\pgfpathlineto{\pgfqpoint{4.374771in}{1.224173in}}%
\pgfpathlineto{\pgfqpoint{4.376035in}{1.230741in}}%
\pgfpathlineto{\pgfqpoint{4.376456in}{1.217606in}}%
\pgfpathlineto{\pgfqpoint{4.377298in}{1.224173in}}%
\pgfpathlineto{\pgfqpoint{4.377720in}{1.230741in}}%
\pgfpathlineto{\pgfqpoint{4.378141in}{1.204471in}}%
\pgfpathlineto{\pgfqpoint{4.378562in}{1.217606in}}%
\pgfpathlineto{\pgfqpoint{4.379826in}{1.250443in}}%
\pgfpathlineto{\pgfqpoint{4.380247in}{1.217606in}}%
\pgfpathlineto{\pgfqpoint{4.381090in}{1.230741in}}%
\pgfpathlineto{\pgfqpoint{4.381933in}{1.230741in}}%
\pgfpathlineto{\pgfqpoint{4.382354in}{1.224173in}}%
\pgfpathlineto{\pgfqpoint{4.382775in}{1.230741in}}%
\pgfpathlineto{\pgfqpoint{4.383196in}{1.230741in}}%
\pgfpathlineto{\pgfqpoint{4.384039in}{1.211038in}}%
\pgfpathlineto{\pgfqpoint{4.384460in}{1.217606in}}%
\pgfpathlineto{\pgfqpoint{4.384882in}{1.230741in}}%
\pgfpathlineto{\pgfqpoint{4.385303in}{1.224173in}}%
\pgfpathlineto{\pgfqpoint{4.386567in}{1.211038in}}%
\pgfpathlineto{\pgfqpoint{4.388673in}{1.230741in}}%
\pgfpathlineto{\pgfqpoint{4.389094in}{1.217606in}}%
\pgfpathlineto{\pgfqpoint{4.389937in}{1.224173in}}%
\pgfpathlineto{\pgfqpoint{4.390358in}{1.224173in}}%
\pgfpathlineto{\pgfqpoint{4.391622in}{1.243876in}}%
\pgfpathlineto{\pgfqpoint{4.393307in}{1.204471in}}%
\pgfpathlineto{\pgfqpoint{4.393728in}{1.230741in}}%
\pgfpathlineto{\pgfqpoint{4.394571in}{1.217606in}}%
\pgfpathlineto{\pgfqpoint{4.395835in}{1.230741in}}%
\pgfpathlineto{\pgfqpoint{4.396677in}{1.211038in}}%
\pgfpathlineto{\pgfqpoint{4.397099in}{1.224173in}}%
\pgfpathlineto{\pgfqpoint{4.397520in}{1.224173in}}%
\pgfpathlineto{\pgfqpoint{4.398784in}{1.243876in}}%
\pgfpathlineto{\pgfqpoint{4.400890in}{1.224173in}}%
\pgfpathlineto{\pgfqpoint{4.401311in}{1.230741in}}%
\pgfpathlineto{\pgfqpoint{4.401733in}{1.224173in}}%
\pgfpathlineto{\pgfqpoint{4.402154in}{1.217606in}}%
\pgfpathlineto{\pgfqpoint{4.402575in}{1.224173in}}%
\pgfpathlineto{\pgfqpoint{4.402996in}{1.224173in}}%
\pgfpathlineto{\pgfqpoint{4.404682in}{1.211038in}}%
\pgfpathlineto{\pgfqpoint{4.405945in}{1.224173in}}%
\pgfpathlineto{\pgfqpoint{4.406367in}{1.217606in}}%
\pgfpathlineto{\pgfqpoint{4.406788in}{1.230741in}}%
\pgfpathlineto{\pgfqpoint{4.407209in}{1.217606in}}%
\pgfpathlineto{\pgfqpoint{4.407631in}{1.217606in}}%
\pgfpathlineto{\pgfqpoint{4.408894in}{1.237308in}}%
\pgfpathlineto{\pgfqpoint{4.409737in}{1.211038in}}%
\pgfpathlineto{\pgfqpoint{4.410158in}{1.224173in}}%
\pgfpathlineto{\pgfqpoint{4.410579in}{1.224173in}}%
\pgfpathlineto{\pgfqpoint{4.411001in}{1.230741in}}%
\pgfpathlineto{\pgfqpoint{4.411422in}{1.224173in}}%
\pgfpathlineto{\pgfqpoint{4.411843in}{1.224173in}}%
\pgfpathlineto{\pgfqpoint{4.412686in}{1.211038in}}%
\pgfpathlineto{\pgfqpoint{4.413950in}{1.230741in}}%
\pgfpathlineto{\pgfqpoint{4.414371in}{1.230741in}}%
\pgfpathlineto{\pgfqpoint{4.415635in}{1.217606in}}%
\pgfpathlineto{\pgfqpoint{4.416056in}{1.237308in}}%
\pgfpathlineto{\pgfqpoint{4.416899in}{1.230741in}}%
\pgfpathlineto{\pgfqpoint{4.417320in}{1.211038in}}%
\pgfpathlineto{\pgfqpoint{4.417741in}{1.230741in}}%
\pgfpathlineto{\pgfqpoint{4.418162in}{1.237308in}}%
\pgfpathlineto{\pgfqpoint{4.418584in}{1.217606in}}%
\pgfpathlineto{\pgfqpoint{4.419426in}{1.224173in}}%
\pgfpathlineto{\pgfqpoint{4.419848in}{1.217606in}}%
\pgfpathlineto{\pgfqpoint{4.420269in}{1.230741in}}%
\pgfpathlineto{\pgfqpoint{4.421111in}{1.224173in}}%
\pgfpathlineto{\pgfqpoint{4.421533in}{1.230741in}}%
\pgfpathlineto{\pgfqpoint{4.421954in}{1.211038in}}%
\pgfpathlineto{\pgfqpoint{4.422375in}{1.217606in}}%
\pgfpathlineto{\pgfqpoint{4.424060in}{1.230741in}}%
\pgfpathlineto{\pgfqpoint{4.425324in}{1.217606in}}%
\pgfpathlineto{\pgfqpoint{4.425745in}{1.217606in}}%
\pgfpathlineto{\pgfqpoint{4.426167in}{1.230741in}}%
\pgfpathlineto{\pgfqpoint{4.426588in}{1.224173in}}%
\pgfpathlineto{\pgfqpoint{4.427009in}{1.197903in}}%
\pgfpathlineto{\pgfqpoint{4.427431in}{1.224173in}}%
\pgfpathlineto{\pgfqpoint{4.427852in}{1.217606in}}%
\pgfpathlineto{\pgfqpoint{4.428273in}{1.230741in}}%
\pgfpathlineto{\pgfqpoint{4.428694in}{1.204471in}}%
\pgfpathlineto{\pgfqpoint{4.429537in}{1.211038in}}%
\pgfpathlineto{\pgfqpoint{4.430801in}{1.237308in}}%
\pgfpathlineto{\pgfqpoint{4.431222in}{1.217606in}}%
\pgfpathlineto{\pgfqpoint{4.432065in}{1.224173in}}%
\pgfpathlineto{\pgfqpoint{4.432486in}{1.211038in}}%
\pgfpathlineto{\pgfqpoint{4.432907in}{1.224173in}}%
\pgfpathlineto{\pgfqpoint{4.433328in}{1.237308in}}%
\pgfpathlineto{\pgfqpoint{4.433750in}{1.217606in}}%
\pgfpathlineto{\pgfqpoint{4.434171in}{1.230741in}}%
\pgfpathlineto{\pgfqpoint{4.435014in}{1.211038in}}%
\pgfpathlineto{\pgfqpoint{4.435856in}{1.217606in}}%
\pgfpathlineto{\pgfqpoint{4.436277in}{1.217606in}}%
\pgfpathlineto{\pgfqpoint{4.436699in}{1.237308in}}%
\pgfpathlineto{\pgfqpoint{4.437120in}{1.224173in}}%
\pgfpathlineto{\pgfqpoint{4.437541in}{1.211038in}}%
\pgfpathlineto{\pgfqpoint{4.437963in}{1.224173in}}%
\pgfpathlineto{\pgfqpoint{4.438805in}{1.230741in}}%
\pgfpathlineto{\pgfqpoint{4.440069in}{1.217606in}}%
\pgfpathlineto{\pgfqpoint{4.440490in}{1.243876in}}%
\pgfpathlineto{\pgfqpoint{4.441333in}{1.230741in}}%
\pgfpathlineto{\pgfqpoint{4.442597in}{1.217606in}}%
\pgfpathlineto{\pgfqpoint{4.443018in}{1.230741in}}%
\pgfpathlineto{\pgfqpoint{4.443439in}{1.224173in}}%
\pgfpathlineto{\pgfqpoint{4.443860in}{1.211038in}}%
\pgfpathlineto{\pgfqpoint{4.444282in}{1.224173in}}%
\pgfpathlineto{\pgfqpoint{4.444703in}{1.224173in}}%
\pgfpathlineto{\pgfqpoint{4.445124in}{1.217606in}}%
\pgfpathlineto{\pgfqpoint{4.445546in}{1.230741in}}%
\pgfpathlineto{\pgfqpoint{4.445967in}{1.217606in}}%
\pgfpathlineto{\pgfqpoint{4.446809in}{1.217606in}}%
\pgfpathlineto{\pgfqpoint{4.448073in}{1.230741in}}%
\pgfpathlineto{\pgfqpoint{4.448495in}{1.230741in}}%
\pgfpathlineto{\pgfqpoint{4.448916in}{1.237308in}}%
\pgfpathlineto{\pgfqpoint{4.450180in}{1.211038in}}%
\pgfpathlineto{\pgfqpoint{4.450601in}{1.217606in}}%
\pgfpathlineto{\pgfqpoint{4.451865in}{1.230741in}}%
\pgfpathlineto{\pgfqpoint{4.452286in}{1.217606in}}%
\pgfpathlineto{\pgfqpoint{4.452707in}{1.224173in}}%
\pgfpathlineto{\pgfqpoint{4.453550in}{1.230741in}}%
\pgfpathlineto{\pgfqpoint{4.454814in}{1.211038in}}%
\pgfpathlineto{\pgfqpoint{4.456078in}{1.237308in}}%
\pgfpathlineto{\pgfqpoint{4.456499in}{1.217606in}}%
\pgfpathlineto{\pgfqpoint{4.456920in}{1.224173in}}%
\pgfpathlineto{\pgfqpoint{4.458184in}{1.230741in}}%
\pgfpathlineto{\pgfqpoint{4.459026in}{1.217606in}}%
\pgfpathlineto{\pgfqpoint{4.459448in}{1.237308in}}%
\pgfpathlineto{\pgfqpoint{4.459869in}{1.230741in}}%
\pgfpathlineto{\pgfqpoint{4.460290in}{1.217606in}}%
\pgfpathlineto{\pgfqpoint{4.461133in}{1.224173in}}%
\pgfpathlineto{\pgfqpoint{4.461975in}{1.217606in}}%
\pgfpathlineto{\pgfqpoint{4.462818in}{1.230741in}}%
\pgfpathlineto{\pgfqpoint{4.463239in}{1.217606in}}%
\pgfpathlineto{\pgfqpoint{4.463661in}{1.230741in}}%
\pgfpathlineto{\pgfqpoint{4.464503in}{1.230741in}}%
\pgfpathlineto{\pgfqpoint{4.465346in}{1.211038in}}%
\pgfpathlineto{\pgfqpoint{4.466188in}{1.217606in}}%
\pgfpathlineto{\pgfqpoint{4.467031in}{1.217606in}}%
\pgfpathlineto{\pgfqpoint{4.468295in}{1.230741in}}%
\pgfpathlineto{\pgfqpoint{4.469137in}{1.211038in}}%
\pgfpathlineto{\pgfqpoint{4.469558in}{1.224173in}}%
\pgfpathlineto{\pgfqpoint{4.470822in}{1.237308in}}%
\pgfpathlineto{\pgfqpoint{4.471665in}{1.217606in}}%
\pgfpathlineto{\pgfqpoint{4.472086in}{1.224173in}}%
\pgfpathlineto{\pgfqpoint{4.472507in}{1.230741in}}%
\pgfpathlineto{\pgfqpoint{4.473350in}{1.217606in}}%
\pgfpathlineto{\pgfqpoint{4.473771in}{1.224173in}}%
\pgfpathlineto{\pgfqpoint{4.474192in}{1.217606in}}%
\pgfpathlineto{\pgfqpoint{4.474614in}{1.237308in}}%
\pgfpathlineto{\pgfqpoint{4.475035in}{1.217606in}}%
\pgfpathlineto{\pgfqpoint{4.475456in}{1.224173in}}%
\pgfpathlineto{\pgfqpoint{4.475878in}{1.217606in}}%
\pgfpathlineto{\pgfqpoint{4.476299in}{1.217606in}}%
\pgfpathlineto{\pgfqpoint{4.476720in}{1.230741in}}%
\pgfpathlineto{\pgfqpoint{4.477141in}{1.224173in}}%
\pgfpathlineto{\pgfqpoint{4.477563in}{1.217606in}}%
\pgfpathlineto{\pgfqpoint{4.477984in}{1.224173in}}%
\pgfpathlineto{\pgfqpoint{4.478827in}{1.230741in}}%
\pgfpathlineto{\pgfqpoint{4.479669in}{1.224173in}}%
\pgfpathlineto{\pgfqpoint{4.480090in}{1.230741in}}%
\pgfpathlineto{\pgfqpoint{4.480512in}{1.224173in}}%
\pgfpathlineto{\pgfqpoint{4.481354in}{1.217606in}}%
\pgfpathlineto{\pgfqpoint{4.482618in}{1.237308in}}%
\pgfpathlineto{\pgfqpoint{4.483039in}{1.217606in}}%
\pgfpathlineto{\pgfqpoint{4.483882in}{1.224173in}}%
\pgfpathlineto{\pgfqpoint{4.484303in}{1.211038in}}%
\pgfpathlineto{\pgfqpoint{4.484724in}{1.230741in}}%
\pgfpathlineto{\pgfqpoint{4.485567in}{1.217606in}}%
\pgfpathlineto{\pgfqpoint{4.485988in}{1.224173in}}%
\pgfpathlineto{\pgfqpoint{4.486410in}{1.224173in}}%
\pgfpathlineto{\pgfqpoint{4.487252in}{1.237308in}}%
\pgfpathlineto{\pgfqpoint{4.488095in}{1.204471in}}%
\pgfpathlineto{\pgfqpoint{4.488516in}{1.230741in}}%
\pgfpathlineto{\pgfqpoint{4.489780in}{1.237308in}}%
\pgfpathlineto{\pgfqpoint{4.490201in}{1.217606in}}%
\pgfpathlineto{\pgfqpoint{4.490622in}{1.224173in}}%
\pgfpathlineto{\pgfqpoint{4.491886in}{1.230741in}}%
\pgfpathlineto{\pgfqpoint{4.492307in}{1.217606in}}%
\pgfpathlineto{\pgfqpoint{4.493150in}{1.224173in}}%
\pgfpathlineto{\pgfqpoint{4.493571in}{1.217606in}}%
\pgfpathlineto{\pgfqpoint{4.493993in}{1.224173in}}%
\pgfpathlineto{\pgfqpoint{4.494414in}{1.230741in}}%
\pgfpathlineto{\pgfqpoint{4.494835in}{1.224173in}}%
\pgfpathlineto{\pgfqpoint{4.495256in}{1.224173in}}%
\pgfpathlineto{\pgfqpoint{4.495678in}{1.237308in}}%
\pgfpathlineto{\pgfqpoint{4.496099in}{1.217606in}}%
\pgfpathlineto{\pgfqpoint{4.496941in}{1.211038in}}%
\pgfpathlineto{\pgfqpoint{4.497784in}{1.230741in}}%
\pgfpathlineto{\pgfqpoint{4.498205in}{1.217606in}}%
\pgfpathlineto{\pgfqpoint{4.498627in}{1.230741in}}%
\pgfpathlineto{\pgfqpoint{4.499048in}{1.237308in}}%
\pgfpathlineto{\pgfqpoint{4.500733in}{1.211038in}}%
\pgfpathlineto{\pgfqpoint{4.501576in}{1.224173in}}%
\pgfpathlineto{\pgfqpoint{4.501997in}{1.217606in}}%
\pgfpathlineto{\pgfqpoint{4.502839in}{1.224173in}}%
\pgfpathlineto{\pgfqpoint{4.503261in}{1.217606in}}%
\pgfpathlineto{\pgfqpoint{4.503682in}{1.224173in}}%
\pgfpathlineto{\pgfqpoint{4.504103in}{1.224173in}}%
\pgfpathlineto{\pgfqpoint{4.505367in}{1.204471in}}%
\pgfpathlineto{\pgfqpoint{4.507052in}{1.230741in}}%
\pgfpathlineto{\pgfqpoint{4.507473in}{1.217606in}}%
\pgfpathlineto{\pgfqpoint{4.508316in}{1.224173in}}%
\pgfpathlineto{\pgfqpoint{4.508737in}{1.217606in}}%
\pgfpathlineto{\pgfqpoint{4.510001in}{1.243876in}}%
\pgfpathlineto{\pgfqpoint{4.510844in}{1.211038in}}%
\pgfpathlineto{\pgfqpoint{4.511265in}{1.224173in}}%
\pgfpathlineto{\pgfqpoint{4.512107in}{1.224173in}}%
\pgfpathlineto{\pgfqpoint{4.512529in}{1.211038in}}%
\pgfpathlineto{\pgfqpoint{4.512950in}{1.217606in}}%
\pgfpathlineto{\pgfqpoint{4.514214in}{1.237308in}}%
\pgfpathlineto{\pgfqpoint{4.514635in}{1.230741in}}%
\pgfpathlineto{\pgfqpoint{4.515056in}{1.230741in}}%
\pgfpathlineto{\pgfqpoint{4.516320in}{1.211038in}}%
\pgfpathlineto{\pgfqpoint{4.517163in}{1.237308in}}%
\pgfpathlineto{\pgfqpoint{4.518005in}{1.230741in}}%
\pgfpathlineto{\pgfqpoint{4.518427in}{1.217606in}}%
\pgfpathlineto{\pgfqpoint{4.518848in}{1.224173in}}%
\pgfpathlineto{\pgfqpoint{4.519269in}{1.237308in}}%
\pgfpathlineto{\pgfqpoint{4.519691in}{1.230741in}}%
\pgfpathlineto{\pgfqpoint{4.520533in}{1.217606in}}%
\pgfpathlineto{\pgfqpoint{4.521797in}{1.230741in}}%
\pgfpathlineto{\pgfqpoint{4.523061in}{1.224173in}}%
\pgfpathlineto{\pgfqpoint{4.523482in}{1.224173in}}%
\pgfpathlineto{\pgfqpoint{4.524325in}{1.237308in}}%
\pgfpathlineto{\pgfqpoint{4.524746in}{1.217606in}}%
\pgfpathlineto{\pgfqpoint{4.525167in}{1.230741in}}%
\pgfpathlineto{\pgfqpoint{4.525588in}{1.230741in}}%
\pgfpathlineto{\pgfqpoint{4.526010in}{1.224173in}}%
\pgfpathlineto{\pgfqpoint{4.526431in}{1.237308in}}%
\pgfpathlineto{\pgfqpoint{4.526852in}{1.217606in}}%
\pgfpathlineto{\pgfqpoint{4.527274in}{1.230741in}}%
\pgfpathlineto{\pgfqpoint{4.528116in}{1.217606in}}%
\pgfpathlineto{\pgfqpoint{4.528537in}{1.224173in}}%
\pgfpathlineto{\pgfqpoint{4.529380in}{1.224173in}}%
\pgfpathlineto{\pgfqpoint{4.529801in}{1.237308in}}%
\pgfpathlineto{\pgfqpoint{4.530222in}{1.230741in}}%
\pgfpathlineto{\pgfqpoint{4.530644in}{1.211038in}}%
\pgfpathlineto{\pgfqpoint{4.531065in}{1.224173in}}%
\pgfpathlineto{\pgfqpoint{4.532329in}{1.224173in}}%
\pgfpathlineto{\pgfqpoint{4.532750in}{1.217606in}}%
\pgfpathlineto{\pgfqpoint{4.533593in}{1.237308in}}%
\pgfpathlineto{\pgfqpoint{4.534014in}{1.230741in}}%
\pgfpathlineto{\pgfqpoint{4.534435in}{1.230741in}}%
\pgfpathlineto{\pgfqpoint{4.536120in}{1.211038in}}%
\pgfpathlineto{\pgfqpoint{4.536963in}{1.224173in}}%
\pgfpathlineto{\pgfqpoint{4.537384in}{1.217606in}}%
\pgfpathlineto{\pgfqpoint{4.538227in}{1.230741in}}%
\pgfpathlineto{\pgfqpoint{4.538648in}{1.224173in}}%
\pgfpathlineto{\pgfqpoint{4.539069in}{1.217606in}}%
\pgfpathlineto{\pgfqpoint{4.539491in}{1.237308in}}%
\pgfpathlineto{\pgfqpoint{4.540333in}{1.230741in}}%
\pgfpathlineto{\pgfqpoint{4.541176in}{1.217606in}}%
\pgfpathlineto{\pgfqpoint{4.542440in}{1.230741in}}%
\pgfpathlineto{\pgfqpoint{4.542861in}{1.230741in}}%
\pgfpathlineto{\pgfqpoint{4.543282in}{1.211038in}}%
\pgfpathlineto{\pgfqpoint{4.543703in}{1.217606in}}%
\pgfpathlineto{\pgfqpoint{4.544125in}{1.237308in}}%
\pgfpathlineto{\pgfqpoint{4.544546in}{1.224173in}}%
\pgfpathlineto{\pgfqpoint{4.544967in}{1.217606in}}%
\pgfpathlineto{\pgfqpoint{4.545810in}{1.237308in}}%
\pgfpathlineto{\pgfqpoint{4.546231in}{1.230741in}}%
\pgfpathlineto{\pgfqpoint{4.547916in}{1.217606in}}%
\pgfpathlineto{\pgfqpoint{4.548337in}{1.224173in}}%
\pgfpathlineto{\pgfqpoint{4.548759in}{1.211038in}}%
\pgfpathlineto{\pgfqpoint{4.549180in}{1.230741in}}%
\pgfpathlineto{\pgfqpoint{4.550444in}{1.224173in}}%
\pgfpathlineto{\pgfqpoint{4.550865in}{1.224173in}}%
\pgfpathlineto{\pgfqpoint{4.551286in}{1.217606in}}%
\pgfpathlineto{\pgfqpoint{4.551708in}{1.230741in}}%
\pgfpathlineto{\pgfqpoint{4.552550in}{1.224173in}}%
\pgfpathlineto{\pgfqpoint{4.552971in}{1.224173in}}%
\pgfpathlineto{\pgfqpoint{4.553393in}{1.217606in}}%
\pgfpathlineto{\pgfqpoint{4.553814in}{1.224173in}}%
\pgfpathlineto{\pgfqpoint{4.554235in}{1.224173in}}%
\pgfpathlineto{\pgfqpoint{4.554657in}{1.230741in}}%
\pgfpathlineto{\pgfqpoint{4.555078in}{1.211038in}}%
\pgfpathlineto{\pgfqpoint{4.555499in}{1.237308in}}%
\pgfpathlineto{\pgfqpoint{4.555920in}{1.217606in}}%
\pgfpathlineto{\pgfqpoint{4.556342in}{1.217606in}}%
\pgfpathlineto{\pgfqpoint{4.557184in}{1.224173in}}%
\pgfpathlineto{\pgfqpoint{4.557606in}{1.217606in}}%
\pgfpathlineto{\pgfqpoint{4.558448in}{1.230741in}}%
\pgfpathlineto{\pgfqpoint{4.558869in}{1.224173in}}%
\pgfpathlineto{\pgfqpoint{4.560554in}{1.224173in}}%
\pgfpathlineto{\pgfqpoint{4.560976in}{1.211038in}}%
\pgfpathlineto{\pgfqpoint{4.561397in}{1.224173in}}%
\pgfpathlineto{\pgfqpoint{4.561818in}{1.230741in}}%
\pgfpathlineto{\pgfqpoint{4.562240in}{1.217606in}}%
\pgfpathlineto{\pgfqpoint{4.562661in}{1.224173in}}%
\pgfpathlineto{\pgfqpoint{4.563082in}{1.230741in}}%
\pgfpathlineto{\pgfqpoint{4.563925in}{1.217606in}}%
\pgfpathlineto{\pgfqpoint{4.564346in}{1.224173in}}%
\pgfpathlineto{\pgfqpoint{4.564767in}{1.230741in}}%
\pgfpathlineto{\pgfqpoint{4.565189in}{1.204471in}}%
\pgfpathlineto{\pgfqpoint{4.565610in}{1.224173in}}%
\pgfpathlineto{\pgfqpoint{4.566452in}{1.224173in}}%
\pgfpathlineto{\pgfqpoint{4.567716in}{1.230741in}}%
\pgfpathlineto{\pgfqpoint{4.568559in}{1.237308in}}%
\pgfpathlineto{\pgfqpoint{4.568980in}{1.217606in}}%
\pgfpathlineto{\pgfqpoint{4.570244in}{1.230741in}}%
\pgfpathlineto{\pgfqpoint{4.571508in}{1.217606in}}%
\pgfpathlineto{\pgfqpoint{4.573193in}{1.237308in}}%
\pgfpathlineto{\pgfqpoint{4.574035in}{1.224173in}}%
\pgfpathlineto{\pgfqpoint{4.574457in}{1.250443in}}%
\pgfpathlineto{\pgfqpoint{4.574878in}{1.224173in}}%
\pgfpathlineto{\pgfqpoint{4.575299in}{1.204471in}}%
\pgfpathlineto{\pgfqpoint{4.575720in}{1.237308in}}%
\pgfpathlineto{\pgfqpoint{4.576142in}{1.217606in}}%
\pgfpathlineto{\pgfqpoint{4.576563in}{1.211038in}}%
\pgfpathlineto{\pgfqpoint{4.576984in}{1.230741in}}%
\pgfpathlineto{\pgfqpoint{4.577406in}{1.224173in}}%
\pgfpathlineto{\pgfqpoint{4.577827in}{1.217606in}}%
\pgfpathlineto{\pgfqpoint{4.578669in}{1.237308in}}%
\pgfpathlineto{\pgfqpoint{4.580355in}{1.217606in}}%
\pgfpathlineto{\pgfqpoint{4.581618in}{1.224173in}}%
\pgfpathlineto{\pgfqpoint{4.582461in}{1.211038in}}%
\pgfpathlineto{\pgfqpoint{4.582882in}{1.230741in}}%
\pgfpathlineto{\pgfqpoint{4.583725in}{1.224173in}}%
\pgfpathlineto{\pgfqpoint{4.584146in}{1.224173in}}%
\pgfpathlineto{\pgfqpoint{4.584567in}{1.237308in}}%
\pgfpathlineto{\pgfqpoint{4.584989in}{1.224173in}}%
\pgfpathlineto{\pgfqpoint{4.585410in}{1.217606in}}%
\pgfpathlineto{\pgfqpoint{4.585831in}{1.224173in}}%
\pgfpathlineto{\pgfqpoint{4.586252in}{1.224173in}}%
\pgfpathlineto{\pgfqpoint{4.586674in}{1.197903in}}%
\pgfpathlineto{\pgfqpoint{4.587095in}{1.224173in}}%
\pgfpathlineto{\pgfqpoint{4.587938in}{1.211038in}}%
\pgfpathlineto{\pgfqpoint{4.588780in}{1.237308in}}%
\pgfpathlineto{\pgfqpoint{4.589201in}{1.224173in}}%
\pgfpathlineto{\pgfqpoint{4.589623in}{1.237308in}}%
\pgfpathlineto{\pgfqpoint{4.590044in}{1.230741in}}%
\pgfpathlineto{\pgfqpoint{4.591729in}{1.204471in}}%
\pgfpathlineto{\pgfqpoint{4.592993in}{1.224173in}}%
\pgfpathlineto{\pgfqpoint{4.594257in}{1.230741in}}%
\pgfpathlineto{\pgfqpoint{4.595099in}{1.224173in}}%
\pgfpathlineto{\pgfqpoint{4.595942in}{1.237308in}}%
\pgfpathlineto{\pgfqpoint{4.596784in}{1.217606in}}%
\pgfpathlineto{\pgfqpoint{4.597206in}{1.237308in}}%
\pgfpathlineto{\pgfqpoint{4.597627in}{1.224173in}}%
\pgfpathlineto{\pgfqpoint{4.598470in}{1.230741in}}%
\pgfpathlineto{\pgfqpoint{4.598891in}{1.211038in}}%
\pgfpathlineto{\pgfqpoint{4.600155in}{1.237308in}}%
\pgfpathlineto{\pgfqpoint{4.600576in}{1.224173in}}%
\pgfpathlineto{\pgfqpoint{4.600997in}{1.224173in}}%
\pgfpathlineto{\pgfqpoint{4.602261in}{1.217606in}}%
\pgfpathlineto{\pgfqpoint{4.603525in}{1.237308in}}%
\pgfpathlineto{\pgfqpoint{4.604789in}{1.217606in}}%
\pgfpathlineto{\pgfqpoint{4.606053in}{1.224173in}}%
\pgfpathlineto{\pgfqpoint{4.606474in}{1.224173in}}%
\pgfpathlineto{\pgfqpoint{4.607738in}{1.211038in}}%
\pgfpathlineto{\pgfqpoint{4.609001in}{1.230741in}}%
\pgfpathlineto{\pgfqpoint{4.610687in}{1.217606in}}%
\pgfpathlineto{\pgfqpoint{4.611108in}{1.237308in}}%
\pgfpathlineto{\pgfqpoint{4.611529in}{1.230741in}}%
\pgfpathlineto{\pgfqpoint{4.611950in}{1.211038in}}%
\pgfpathlineto{\pgfqpoint{4.612372in}{1.237308in}}%
\pgfpathlineto{\pgfqpoint{4.613636in}{1.217606in}}%
\pgfpathlineto{\pgfqpoint{4.614899in}{1.230741in}}%
\pgfpathlineto{\pgfqpoint{4.616163in}{1.224173in}}%
\pgfpathlineto{\pgfqpoint{4.616584in}{1.224173in}}%
\pgfpathlineto{\pgfqpoint{4.617006in}{1.211038in}}%
\pgfpathlineto{\pgfqpoint{4.617427in}{1.224173in}}%
\pgfpathlineto{\pgfqpoint{4.618270in}{1.230741in}}%
\pgfpathlineto{\pgfqpoint{4.618691in}{1.224173in}}%
\pgfpathlineto{\pgfqpoint{4.619112in}{1.230741in}}%
\pgfpathlineto{\pgfqpoint{4.619533in}{1.230741in}}%
\pgfpathlineto{\pgfqpoint{4.619955in}{1.224173in}}%
\pgfpathlineto{\pgfqpoint{4.620376in}{1.237308in}}%
\pgfpathlineto{\pgfqpoint{4.620797in}{1.211038in}}%
\pgfpathlineto{\pgfqpoint{4.621640in}{1.224173in}}%
\pgfpathlineto{\pgfqpoint{4.622904in}{1.211038in}}%
\pgfpathlineto{\pgfqpoint{4.623746in}{1.230741in}}%
\pgfpathlineto{\pgfqpoint{4.624167in}{1.217606in}}%
\pgfpathlineto{\pgfqpoint{4.624589in}{1.204471in}}%
\pgfpathlineto{\pgfqpoint{4.625010in}{1.217606in}}%
\pgfpathlineto{\pgfqpoint{4.625431in}{1.237308in}}%
\pgfpathlineto{\pgfqpoint{4.625853in}{1.217606in}}%
\pgfpathlineto{\pgfqpoint{4.626274in}{1.224173in}}%
\pgfpathlineto{\pgfqpoint{4.626695in}{1.217606in}}%
\pgfpathlineto{\pgfqpoint{4.627538in}{1.211038in}}%
\pgfpathlineto{\pgfqpoint{4.628802in}{1.230741in}}%
\pgfpathlineto{\pgfqpoint{4.630065in}{1.217606in}}%
\pgfpathlineto{\pgfqpoint{4.630487in}{1.230741in}}%
\pgfpathlineto{\pgfqpoint{4.631329in}{1.224173in}}%
\pgfpathlineto{\pgfqpoint{4.632593in}{1.224173in}}%
\pgfpathlineto{\pgfqpoint{4.633014in}{1.217606in}}%
\pgfpathlineto{\pgfqpoint{4.633436in}{1.224173in}}%
\pgfpathlineto{\pgfqpoint{4.633857in}{1.224173in}}%
\pgfpathlineto{\pgfqpoint{4.634699in}{1.237308in}}%
\pgfpathlineto{\pgfqpoint{4.635121in}{1.217606in}}%
\pgfpathlineto{\pgfqpoint{4.635963in}{1.224173in}}%
\pgfpathlineto{\pgfqpoint{4.636385in}{1.237308in}}%
\pgfpathlineto{\pgfqpoint{4.636806in}{1.217606in}}%
\pgfpathlineto{\pgfqpoint{4.637227in}{1.204471in}}%
\pgfpathlineto{\pgfqpoint{4.638491in}{1.230741in}}%
\pgfpathlineto{\pgfqpoint{4.639755in}{1.217606in}}%
\pgfpathlineto{\pgfqpoint{4.640597in}{1.243876in}}%
\pgfpathlineto{\pgfqpoint{4.641019in}{1.230741in}}%
\pgfpathlineto{\pgfqpoint{4.642282in}{1.217606in}}%
\pgfpathlineto{\pgfqpoint{4.643546in}{1.230741in}}%
\pgfpathlineto{\pgfqpoint{4.644810in}{1.217606in}}%
\pgfpathlineto{\pgfqpoint{4.645653in}{1.224173in}}%
\pgfpathlineto{\pgfqpoint{4.646074in}{1.217606in}}%
\pgfpathlineto{\pgfqpoint{4.646495in}{1.224173in}}%
\pgfpathlineto{\pgfqpoint{4.646917in}{1.224173in}}%
\pgfpathlineto{\pgfqpoint{4.648180in}{1.204471in}}%
\pgfpathlineto{\pgfqpoint{4.649023in}{1.224173in}}%
\pgfpathlineto{\pgfqpoint{4.649444in}{1.217606in}}%
\pgfpathlineto{\pgfqpoint{4.651129in}{1.230741in}}%
\pgfpathlineto{\pgfqpoint{4.651551in}{1.230741in}}%
\pgfpathlineto{\pgfqpoint{4.652393in}{1.204471in}}%
\pgfpathlineto{\pgfqpoint{4.652814in}{1.224173in}}%
\pgfpathlineto{\pgfqpoint{4.653236in}{1.217606in}}%
\pgfpathlineto{\pgfqpoint{4.653657in}{1.230741in}}%
\pgfpathlineto{\pgfqpoint{4.654078in}{1.217606in}}%
\pgfpathlineto{\pgfqpoint{4.654500in}{1.211038in}}%
\pgfpathlineto{\pgfqpoint{4.654921in}{1.217606in}}%
\pgfpathlineto{\pgfqpoint{4.655342in}{1.237308in}}%
\pgfpathlineto{\pgfqpoint{4.655763in}{1.230741in}}%
\pgfpathlineto{\pgfqpoint{4.657448in}{1.211038in}}%
\pgfpathlineto{\pgfqpoint{4.659134in}{1.230741in}}%
\pgfpathlineto{\pgfqpoint{4.660397in}{1.217606in}}%
\pgfpathlineto{\pgfqpoint{4.661661in}{1.237308in}}%
\pgfpathlineto{\pgfqpoint{4.662504in}{1.211038in}}%
\pgfpathlineto{\pgfqpoint{4.662925in}{1.224173in}}%
\pgfpathlineto{\pgfqpoint{4.663346in}{1.224173in}}%
\pgfpathlineto{\pgfqpoint{4.664189in}{1.237308in}}%
\pgfpathlineto{\pgfqpoint{4.664610in}{1.211038in}}%
\pgfpathlineto{\pgfqpoint{4.665031in}{1.230741in}}%
\pgfpathlineto{\pgfqpoint{4.665453in}{1.230741in}}%
\pgfpathlineto{\pgfqpoint{4.665874in}{1.237308in}}%
\pgfpathlineto{\pgfqpoint{4.667138in}{1.217606in}}%
\pgfpathlineto{\pgfqpoint{4.667980in}{1.237308in}}%
\pgfpathlineto{\pgfqpoint{4.668402in}{1.224173in}}%
\pgfpathlineto{\pgfqpoint{4.668823in}{1.204471in}}%
\pgfpathlineto{\pgfqpoint{4.669244in}{1.211038in}}%
\pgfpathlineto{\pgfqpoint{4.670087in}{1.237308in}}%
\pgfpathlineto{\pgfqpoint{4.670929in}{1.230741in}}%
\pgfpathlineto{\pgfqpoint{4.671351in}{1.230741in}}%
\pgfpathlineto{\pgfqpoint{4.672614in}{1.217606in}}%
\pgfpathlineto{\pgfqpoint{4.673036in}{1.237308in}}%
\pgfpathlineto{\pgfqpoint{4.673457in}{1.224173in}}%
\pgfpathlineto{\pgfqpoint{4.673878in}{1.217606in}}%
\pgfpathlineto{\pgfqpoint{4.674300in}{1.224173in}}%
\pgfpathlineto{\pgfqpoint{4.674721in}{1.224173in}}%
\pgfpathlineto{\pgfqpoint{4.675142in}{1.211038in}}%
\pgfpathlineto{\pgfqpoint{4.675985in}{1.237308in}}%
\pgfpathlineto{\pgfqpoint{4.676406in}{1.230741in}}%
\pgfpathlineto{\pgfqpoint{4.677670in}{1.211038in}}%
\pgfpathlineto{\pgfqpoint{4.678091in}{1.230741in}}%
\pgfpathlineto{\pgfqpoint{4.678934in}{1.224173in}}%
\pgfpathlineto{\pgfqpoint{4.679776in}{1.224173in}}%
\pgfpathlineto{\pgfqpoint{4.680619in}{1.230741in}}%
\pgfpathlineto{\pgfqpoint{4.681883in}{1.224173in}}%
\pgfpathlineto{\pgfqpoint{4.682725in}{1.230741in}}%
\pgfpathlineto{\pgfqpoint{4.683568in}{1.204471in}}%
\pgfpathlineto{\pgfqpoint{4.683989in}{1.217606in}}%
\pgfpathlineto{\pgfqpoint{4.684410in}{1.224173in}}%
\pgfpathlineto{\pgfqpoint{4.684832in}{1.217606in}}%
\pgfpathlineto{\pgfqpoint{4.685253in}{1.217606in}}%
\pgfpathlineto{\pgfqpoint{4.685674in}{1.224173in}}%
\pgfpathlineto{\pgfqpoint{4.686095in}{1.217606in}}%
\pgfpathlineto{\pgfqpoint{4.686517in}{1.217606in}}%
\pgfpathlineto{\pgfqpoint{4.687359in}{1.224173in}}%
\pgfpathlineto{\pgfqpoint{4.687780in}{1.217606in}}%
\pgfpathlineto{\pgfqpoint{4.688202in}{1.224173in}}%
\pgfpathlineto{\pgfqpoint{4.688623in}{1.224173in}}%
\pgfpathlineto{\pgfqpoint{4.689887in}{1.211038in}}%
\pgfpathlineto{\pgfqpoint{4.691151in}{1.230741in}}%
\pgfpathlineto{\pgfqpoint{4.692415in}{1.217606in}}%
\pgfpathlineto{\pgfqpoint{4.693257in}{1.224173in}}%
\pgfpathlineto{\pgfqpoint{4.693678in}{1.217606in}}%
\pgfpathlineto{\pgfqpoint{4.694100in}{1.224173in}}%
\pgfpathlineto{\pgfqpoint{4.694521in}{1.224173in}}%
\pgfpathlineto{\pgfqpoint{4.695785in}{1.230741in}}%
\pgfpathlineto{\pgfqpoint{4.696627in}{1.224173in}}%
\pgfpathlineto{\pgfqpoint{4.697470in}{1.237308in}}%
\pgfpathlineto{\pgfqpoint{4.697891in}{1.224173in}}%
\pgfpathlineto{\pgfqpoint{4.698734in}{1.230741in}}%
\pgfpathlineto{\pgfqpoint{4.699155in}{1.230741in}}%
\pgfpathlineto{\pgfqpoint{4.699576in}{1.237308in}}%
\pgfpathlineto{\pgfqpoint{4.700419in}{1.224173in}}%
\pgfpathlineto{\pgfqpoint{4.700840in}{1.230741in}}%
\pgfpathlineto{\pgfqpoint{4.701261in}{1.237308in}}%
\pgfpathlineto{\pgfqpoint{4.701683in}{1.230741in}}%
\pgfpathlineto{\pgfqpoint{4.702946in}{1.224173in}}%
\pgfpathlineto{\pgfqpoint{4.703368in}{1.230741in}}%
\pgfpathlineto{\pgfqpoint{4.703789in}{1.217606in}}%
\pgfpathlineto{\pgfqpoint{4.704632in}{1.224173in}}%
\pgfpathlineto{\pgfqpoint{4.705053in}{1.217606in}}%
\pgfpathlineto{\pgfqpoint{4.705474in}{1.230741in}}%
\pgfpathlineto{\pgfqpoint{4.706317in}{1.224173in}}%
\pgfpathlineto{\pgfqpoint{4.707159in}{1.224173in}}%
\pgfpathlineto{\pgfqpoint{4.708423in}{1.217606in}}%
\pgfpathlineto{\pgfqpoint{4.709687in}{1.230741in}}%
\pgfpathlineto{\pgfqpoint{4.710951in}{1.217606in}}%
\pgfpathlineto{\pgfqpoint{4.712215in}{1.224173in}}%
\pgfpathlineto{\pgfqpoint{4.712636in}{1.224173in}}%
\pgfpathlineto{\pgfqpoint{4.713057in}{1.217606in}}%
\pgfpathlineto{\pgfqpoint{4.714321in}{1.230741in}}%
\pgfpathlineto{\pgfqpoint{4.715164in}{1.217606in}}%
\pgfpathlineto{\pgfqpoint{4.716427in}{1.237308in}}%
\pgfpathlineto{\pgfqpoint{4.717691in}{1.217606in}}%
\pgfpathlineto{\pgfqpoint{4.718113in}{1.224173in}}%
\pgfpathlineto{\pgfqpoint{4.718534in}{1.224173in}}%
\pgfpathlineto{\pgfqpoint{4.718955in}{1.217606in}}%
\pgfpathlineto{\pgfqpoint{4.719376in}{1.224173in}}%
\pgfpathlineto{\pgfqpoint{4.719798in}{1.230741in}}%
\pgfpathlineto{\pgfqpoint{4.720219in}{1.217606in}}%
\pgfpathlineto{\pgfqpoint{4.720640in}{1.224173in}}%
\pgfpathlineto{\pgfqpoint{4.721061in}{1.230741in}}%
\pgfpathlineto{\pgfqpoint{4.721483in}{1.224173in}}%
\pgfpathlineto{\pgfqpoint{4.721904in}{1.224173in}}%
\pgfpathlineto{\pgfqpoint{4.722325in}{1.211038in}}%
\pgfpathlineto{\pgfqpoint{4.722747in}{1.230741in}}%
\pgfpathlineto{\pgfqpoint{4.724010in}{1.217606in}}%
\pgfpathlineto{\pgfqpoint{4.725274in}{1.230741in}}%
\pgfpathlineto{\pgfqpoint{4.725696in}{1.230741in}}%
\pgfpathlineto{\pgfqpoint{4.726959in}{1.224173in}}%
\pgfpathlineto{\pgfqpoint{4.727381in}{1.230741in}}%
\pgfpathlineto{\pgfqpoint{4.728223in}{1.217606in}}%
\pgfpathlineto{\pgfqpoint{4.728644in}{1.224173in}}%
\pgfpathlineto{\pgfqpoint{4.729487in}{1.224173in}}%
\pgfpathlineto{\pgfqpoint{4.730330in}{1.217606in}}%
\pgfpathlineto{\pgfqpoint{4.731172in}{1.224173in}}%
\pgfpathlineto{\pgfqpoint{4.731593in}{1.217606in}}%
\pgfpathlineto{\pgfqpoint{4.732015in}{1.224173in}}%
\pgfpathlineto{\pgfqpoint{4.732436in}{1.230741in}}%
\pgfpathlineto{\pgfqpoint{4.732857in}{1.224173in}}%
\pgfpathlineto{\pgfqpoint{4.734121in}{1.217606in}}%
\pgfpathlineto{\pgfqpoint{4.735385in}{1.224173in}}%
\pgfpathlineto{\pgfqpoint{4.735806in}{1.224173in}}%
\pgfpathlineto{\pgfqpoint{4.736227in}{1.211038in}}%
\pgfpathlineto{\pgfqpoint{4.736649in}{1.217606in}}%
\pgfpathlineto{\pgfqpoint{4.737491in}{1.230741in}}%
\pgfpathlineto{\pgfqpoint{4.738334in}{1.217606in}}%
\pgfpathlineto{\pgfqpoint{4.738755in}{1.224173in}}%
\pgfpathlineto{\pgfqpoint{4.739176in}{1.217606in}}%
\pgfpathlineto{\pgfqpoint{4.740019in}{1.230741in}}%
\pgfpathlineto{\pgfqpoint{4.741283in}{1.211038in}}%
\pgfpathlineto{\pgfqpoint{4.742547in}{1.237308in}}%
\pgfpathlineto{\pgfqpoint{4.742968in}{1.217606in}}%
\pgfpathlineto{\pgfqpoint{4.743810in}{1.224173in}}%
\pgfpathlineto{\pgfqpoint{4.744653in}{1.224173in}}%
\pgfpathlineto{\pgfqpoint{4.745496in}{1.230741in}}%
\pgfpathlineto{\pgfqpoint{4.745917in}{1.224173in}}%
\pgfpathlineto{\pgfqpoint{4.746338in}{1.237308in}}%
\pgfpathlineto{\pgfqpoint{4.746759in}{1.224173in}}%
\pgfpathlineto{\pgfqpoint{4.747181in}{1.112525in}}%
\pgfpathlineto{\pgfqpoint{4.747602in}{1.237308in}}%
\pgfpathlineto{\pgfqpoint{4.748023in}{1.224173in}}%
\pgfpathlineto{\pgfqpoint{4.748445in}{1.243876in}}%
\pgfpathlineto{\pgfqpoint{4.749708in}{1.230741in}}%
\pgfpathlineto{\pgfqpoint{4.750130in}{1.243876in}}%
\pgfpathlineto{\pgfqpoint{4.750551in}{1.224173in}}%
\pgfpathlineto{\pgfqpoint{4.750972in}{1.224173in}}%
\pgfpathlineto{\pgfqpoint{4.752236in}{1.237308in}}%
\pgfpathlineto{\pgfqpoint{4.752657in}{1.237308in}}%
\pgfpathlineto{\pgfqpoint{4.753500in}{1.211038in}}%
\pgfpathlineto{\pgfqpoint{4.754764in}{1.250443in}}%
\pgfpathlineto{\pgfqpoint{4.756028in}{1.224173in}}%
\pgfpathlineto{\pgfqpoint{4.757713in}{1.243876in}}%
\pgfpathlineto{\pgfqpoint{4.758976in}{1.211038in}}%
\pgfpathlineto{\pgfqpoint{4.759819in}{1.224173in}}%
\pgfpathlineto{\pgfqpoint{4.760240in}{1.224173in}}%
\pgfpathlineto{\pgfqpoint{4.760662in}{1.362092in}}%
\pgfpathlineto{\pgfqpoint{4.761083in}{1.224173in}}%
\pgfpathlineto{\pgfqpoint{4.761504in}{1.224173in}}%
\pgfpathlineto{\pgfqpoint{4.762347in}{1.211038in}}%
\pgfpathlineto{\pgfqpoint{4.762768in}{1.237308in}}%
\pgfpathlineto{\pgfqpoint{4.763189in}{1.224173in}}%
\pgfpathlineto{\pgfqpoint{4.763611in}{1.204471in}}%
\pgfpathlineto{\pgfqpoint{4.764032in}{1.224173in}}%
\pgfpathlineto{\pgfqpoint{4.764874in}{1.224173in}}%
\pgfpathlineto{\pgfqpoint{4.765296in}{1.237308in}}%
\pgfpathlineto{\pgfqpoint{4.765717in}{1.211038in}}%
\pgfpathlineto{\pgfqpoint{4.766559in}{1.224173in}}%
\pgfpathlineto{\pgfqpoint{4.767402in}{1.217606in}}%
\pgfpathlineto{\pgfqpoint{4.768245in}{1.224173in}}%
\pgfpathlineto{\pgfqpoint{4.769508in}{1.217606in}}%
\pgfpathlineto{\pgfqpoint{4.769930in}{1.217606in}}%
\pgfpathlineto{\pgfqpoint{4.770351in}{1.230741in}}%
\pgfpathlineto{\pgfqpoint{4.770772in}{1.224173in}}%
\pgfpathlineto{\pgfqpoint{4.771194in}{1.217606in}}%
\pgfpathlineto{\pgfqpoint{4.771615in}{1.224173in}}%
\pgfpathlineto{\pgfqpoint{4.772036in}{1.224173in}}%
\pgfpathlineto{\pgfqpoint{4.772457in}{1.230741in}}%
\pgfpathlineto{\pgfqpoint{4.772879in}{1.224173in}}%
\pgfpathlineto{\pgfqpoint{4.774564in}{1.211038in}}%
\pgfpathlineto{\pgfqpoint{4.775828in}{1.230741in}}%
\pgfpathlineto{\pgfqpoint{4.776249in}{1.211038in}}%
\pgfpathlineto{\pgfqpoint{4.776670in}{1.224173in}}%
\pgfpathlineto{\pgfqpoint{4.777091in}{1.230741in}}%
\pgfpathlineto{\pgfqpoint{4.778777in}{1.211038in}}%
\pgfpathlineto{\pgfqpoint{4.779198in}{1.230741in}}%
\pgfpathlineto{\pgfqpoint{4.779619in}{1.224173in}}%
\pgfpathlineto{\pgfqpoint{4.780040in}{1.217606in}}%
\pgfpathlineto{\pgfqpoint{4.780462in}{1.243876in}}%
\pgfpathlineto{\pgfqpoint{4.780883in}{1.224173in}}%
\pgfpathlineto{\pgfqpoint{4.782147in}{1.224173in}}%
\pgfpathlineto{\pgfqpoint{4.782568in}{1.217606in}}%
\pgfpathlineto{\pgfqpoint{4.782989in}{1.237308in}}%
\pgfpathlineto{\pgfqpoint{4.783411in}{1.224173in}}%
\pgfpathlineto{\pgfqpoint{4.783832in}{1.217606in}}%
\pgfpathlineto{\pgfqpoint{4.784253in}{1.230741in}}%
\pgfpathlineto{\pgfqpoint{4.784674in}{1.217606in}}%
\pgfpathlineto{\pgfqpoint{4.785096in}{1.217606in}}%
\pgfpathlineto{\pgfqpoint{4.785938in}{1.237308in}}%
\pgfpathlineto{\pgfqpoint{4.786781in}{1.230741in}}%
\pgfpathlineto{\pgfqpoint{4.788045in}{1.230741in}}%
\pgfpathlineto{\pgfqpoint{4.788466in}{1.217606in}}%
\pgfpathlineto{\pgfqpoint{4.788887in}{1.230741in}}%
\pgfpathlineto{\pgfqpoint{4.789309in}{1.230741in}}%
\pgfpathlineto{\pgfqpoint{4.789730in}{1.217606in}}%
\pgfpathlineto{\pgfqpoint{4.790151in}{1.230741in}}%
\pgfpathlineto{\pgfqpoint{4.790572in}{1.230741in}}%
\pgfpathlineto{\pgfqpoint{4.791415in}{1.224173in}}%
\pgfpathlineto{\pgfqpoint{4.792257in}{1.230741in}}%
\pgfpathlineto{\pgfqpoint{4.793943in}{1.217606in}}%
\pgfpathlineto{\pgfqpoint{4.794364in}{1.230741in}}%
\pgfpathlineto{\pgfqpoint{4.795206in}{1.224173in}}%
\pgfpathlineto{\pgfqpoint{4.796049in}{1.230741in}}%
\pgfpathlineto{\pgfqpoint{4.796470in}{1.211038in}}%
\pgfpathlineto{\pgfqpoint{4.796892in}{1.224173in}}%
\pgfpathlineto{\pgfqpoint{4.797734in}{1.224173in}}%
\pgfpathlineto{\pgfqpoint{4.798155in}{1.230741in}}%
\pgfpathlineto{\pgfqpoint{4.798577in}{1.224173in}}%
\pgfpathlineto{\pgfqpoint{4.798998in}{1.224173in}}%
\pgfpathlineto{\pgfqpoint{4.799419in}{1.237308in}}%
\pgfpathlineto{\pgfqpoint{4.799840in}{1.224173in}}%
\pgfpathlineto{\pgfqpoint{4.800262in}{1.224173in}}%
\pgfpathlineto{\pgfqpoint{4.800683in}{1.230741in}}%
\pgfpathlineto{\pgfqpoint{4.801104in}{1.217606in}}%
\pgfpathlineto{\pgfqpoint{4.801947in}{1.224173in}}%
\pgfpathlineto{\pgfqpoint{4.802789in}{1.224173in}}%
\pgfpathlineto{\pgfqpoint{4.803211in}{1.243876in}}%
\pgfpathlineto{\pgfqpoint{4.803632in}{1.217606in}}%
\pgfpathlineto{\pgfqpoint{4.804053in}{1.230741in}}%
\pgfpathlineto{\pgfqpoint{4.804475in}{1.224173in}}%
\pgfpathlineto{\pgfqpoint{4.804896in}{1.217606in}}%
\pgfpathlineto{\pgfqpoint{4.805317in}{1.224173in}}%
\pgfpathlineto{\pgfqpoint{4.805738in}{1.237308in}}%
\pgfpathlineto{\pgfqpoint{4.806160in}{1.217606in}}%
\pgfpathlineto{\pgfqpoint{4.806581in}{1.217606in}}%
\pgfpathlineto{\pgfqpoint{4.807002in}{1.230741in}}%
\pgfpathlineto{\pgfqpoint{4.807423in}{1.217606in}}%
\pgfpathlineto{\pgfqpoint{4.807845in}{1.211038in}}%
\pgfpathlineto{\pgfqpoint{4.808266in}{1.230741in}}%
\pgfpathlineto{\pgfqpoint{4.808687in}{1.217606in}}%
\pgfpathlineto{\pgfqpoint{4.809109in}{1.217606in}}%
\pgfpathlineto{\pgfqpoint{4.809530in}{1.230741in}}%
\pgfpathlineto{\pgfqpoint{4.810372in}{1.224173in}}%
\pgfpathlineto{\pgfqpoint{4.810794in}{1.224173in}}%
\pgfpathlineto{\pgfqpoint{4.811636in}{1.211038in}}%
\pgfpathlineto{\pgfqpoint{4.812058in}{1.217606in}}%
\pgfpathlineto{\pgfqpoint{4.812479in}{1.217606in}}%
\pgfpathlineto{\pgfqpoint{4.812900in}{1.204471in}}%
\pgfpathlineto{\pgfqpoint{4.813321in}{1.237308in}}%
\pgfpathlineto{\pgfqpoint{4.814164in}{1.217606in}}%
\pgfpathlineto{\pgfqpoint{4.814585in}{1.230741in}}%
\pgfpathlineto{\pgfqpoint{4.815006in}{1.211038in}}%
\pgfpathlineto{\pgfqpoint{4.815428in}{1.224173in}}%
\pgfpathlineto{\pgfqpoint{4.815849in}{1.224173in}}%
\pgfpathlineto{\pgfqpoint{4.816692in}{1.211038in}}%
\pgfpathlineto{\pgfqpoint{4.817113in}{1.237308in}}%
\pgfpathlineto{\pgfqpoint{4.817955in}{1.224173in}}%
\pgfpathlineto{\pgfqpoint{4.818377in}{1.230741in}}%
\pgfpathlineto{\pgfqpoint{4.819219in}{1.211038in}}%
\pgfpathlineto{\pgfqpoint{4.819641in}{1.224173in}}%
\pgfpathlineto{\pgfqpoint{4.820062in}{1.224173in}}%
\pgfpathlineto{\pgfqpoint{4.820483in}{1.217606in}}%
\pgfpathlineto{\pgfqpoint{4.821326in}{1.230741in}}%
\pgfpathlineto{\pgfqpoint{4.823011in}{1.211038in}}%
\pgfpathlineto{\pgfqpoint{4.823432in}{1.230741in}}%
\pgfpathlineto{\pgfqpoint{4.823853in}{1.224173in}}%
\pgfpathlineto{\pgfqpoint{4.824275in}{1.217606in}}%
\pgfpathlineto{\pgfqpoint{4.824696in}{1.237308in}}%
\pgfpathlineto{\pgfqpoint{4.825538in}{1.230741in}}%
\pgfpathlineto{\pgfqpoint{4.825960in}{1.230741in}}%
\pgfpathlineto{\pgfqpoint{4.826802in}{1.204471in}}%
\pgfpathlineto{\pgfqpoint{4.827224in}{1.224173in}}%
\pgfpathlineto{\pgfqpoint{4.827645in}{1.224173in}}%
\pgfpathlineto{\pgfqpoint{4.828066in}{1.217606in}}%
\pgfpathlineto{\pgfqpoint{4.828487in}{1.230741in}}%
\pgfpathlineto{\pgfqpoint{4.828909in}{1.224173in}}%
\pgfpathlineto{\pgfqpoint{4.829330in}{1.204471in}}%
\pgfpathlineto{\pgfqpoint{4.829751in}{1.230741in}}%
\pgfpathlineto{\pgfqpoint{4.830594in}{1.217606in}}%
\pgfpathlineto{\pgfqpoint{4.831015in}{1.230741in}}%
\pgfpathlineto{\pgfqpoint{4.831436in}{1.224173in}}%
\pgfpathlineto{\pgfqpoint{4.831858in}{1.217606in}}%
\pgfpathlineto{\pgfqpoint{4.832279in}{1.224173in}}%
\pgfpathlineto{\pgfqpoint{4.832700in}{1.224173in}}%
\pgfpathlineto{\pgfqpoint{4.833121in}{1.217606in}}%
\pgfpathlineto{\pgfqpoint{4.833543in}{1.230741in}}%
\pgfpathlineto{\pgfqpoint{4.834385in}{1.224173in}}%
\pgfpathlineto{\pgfqpoint{4.834807in}{1.230741in}}%
\pgfpathlineto{\pgfqpoint{4.835228in}{1.224173in}}%
\pgfpathlineto{\pgfqpoint{4.836070in}{1.224173in}}%
\pgfpathlineto{\pgfqpoint{4.837334in}{1.230741in}}%
\pgfpathlineto{\pgfqpoint{4.837755in}{1.230741in}}%
\pgfpathlineto{\pgfqpoint{4.838177in}{1.217606in}}%
\pgfpathlineto{\pgfqpoint{4.839019in}{1.224173in}}%
\pgfpathlineto{\pgfqpoint{4.839441in}{1.217606in}}%
\pgfpathlineto{\pgfqpoint{4.839862in}{1.224173in}}%
\pgfpathlineto{\pgfqpoint{4.841126in}{1.230741in}}%
\pgfpathlineto{\pgfqpoint{4.842390in}{1.217606in}}%
\pgfpathlineto{\pgfqpoint{4.842811in}{1.237308in}}%
\pgfpathlineto{\pgfqpoint{4.843653in}{1.230741in}}%
\pgfpathlineto{\pgfqpoint{4.844496in}{1.217606in}}%
\pgfpathlineto{\pgfqpoint{4.844917in}{1.224173in}}%
\pgfpathlineto{\pgfqpoint{4.845760in}{1.217606in}}%
\pgfpathlineto{\pgfqpoint{4.846602in}{1.230741in}}%
\pgfpathlineto{\pgfqpoint{4.847024in}{1.217606in}}%
\pgfpathlineto{\pgfqpoint{4.847866in}{1.224173in}}%
\pgfpathlineto{\pgfqpoint{4.848287in}{1.224173in}}%
\pgfpathlineto{\pgfqpoint{4.849551in}{1.217606in}}%
\pgfpathlineto{\pgfqpoint{4.849973in}{1.224173in}}%
\pgfpathlineto{\pgfqpoint{4.850394in}{1.211038in}}%
\pgfpathlineto{\pgfqpoint{4.850815in}{1.224173in}}%
\pgfpathlineto{\pgfqpoint{4.851236in}{1.230741in}}%
\pgfpathlineto{\pgfqpoint{4.851658in}{1.224173in}}%
\pgfpathlineto{\pgfqpoint{4.852500in}{1.224173in}}%
\pgfpathlineto{\pgfqpoint{4.853764in}{1.230741in}}%
\pgfpathlineto{\pgfqpoint{4.854607in}{1.211038in}}%
\pgfpathlineto{\pgfqpoint{4.855028in}{1.230741in}}%
\pgfpathlineto{\pgfqpoint{4.855870in}{1.224173in}}%
\pgfpathlineto{\pgfqpoint{4.856292in}{1.230741in}}%
\pgfpathlineto{\pgfqpoint{4.856713in}{1.224173in}}%
\pgfpathlineto{\pgfqpoint{4.857134in}{1.217606in}}%
\pgfpathlineto{\pgfqpoint{4.857977in}{1.230741in}}%
\pgfpathlineto{\pgfqpoint{4.858398in}{1.217606in}}%
\pgfpathlineto{\pgfqpoint{4.858819in}{1.237308in}}%
\pgfpathlineto{\pgfqpoint{4.859662in}{1.224173in}}%
\pgfpathlineto{\pgfqpoint{4.860083in}{1.230741in}}%
\pgfpathlineto{\pgfqpoint{4.860505in}{1.230741in}}%
\pgfpathlineto{\pgfqpoint{4.860926in}{1.237308in}}%
\pgfpathlineto{\pgfqpoint{4.861768in}{1.217606in}}%
\pgfpathlineto{\pgfqpoint{4.862190in}{1.230741in}}%
\pgfpathlineto{\pgfqpoint{4.863875in}{1.230741in}}%
\pgfpathlineto{\pgfqpoint{4.864717in}{1.217606in}}%
\pgfpathlineto{\pgfqpoint{4.865139in}{1.230741in}}%
\pgfpathlineto{\pgfqpoint{4.865981in}{1.224173in}}%
\pgfpathlineto{\pgfqpoint{4.866402in}{1.230741in}}%
\pgfpathlineto{\pgfqpoint{4.866824in}{1.217606in}}%
\pgfpathlineto{\pgfqpoint{4.867666in}{1.224173in}}%
\pgfpathlineto{\pgfqpoint{4.868930in}{1.230741in}}%
\pgfpathlineto{\pgfqpoint{4.869351in}{1.230741in}}%
\pgfpathlineto{\pgfqpoint{4.869773in}{1.211038in}}%
\pgfpathlineto{\pgfqpoint{4.870615in}{1.217606in}}%
\pgfpathlineto{\pgfqpoint{4.871036in}{1.224173in}}%
\pgfpathlineto{\pgfqpoint{4.871458in}{1.211038in}}%
\pgfpathlineto{\pgfqpoint{4.871879in}{1.230741in}}%
\pgfpathlineto{\pgfqpoint{4.872300in}{1.217606in}}%
\pgfpathlineto{\pgfqpoint{4.872722in}{1.230741in}}%
\pgfpathlineto{\pgfqpoint{4.873143in}{1.224173in}}%
\pgfpathlineto{\pgfqpoint{4.874407in}{1.211038in}}%
\pgfpathlineto{\pgfqpoint{4.875249in}{1.230741in}}%
\pgfpathlineto{\pgfqpoint{4.875671in}{1.224173in}}%
\pgfpathlineto{\pgfqpoint{4.876513in}{1.230741in}}%
\pgfpathlineto{\pgfqpoint{4.876934in}{1.211038in}}%
\pgfpathlineto{\pgfqpoint{4.877777in}{1.230741in}}%
\pgfpathlineto{\pgfqpoint{4.878619in}{1.217606in}}%
\pgfpathlineto{\pgfqpoint{4.879041in}{1.230741in}}%
\pgfpathlineto{\pgfqpoint{4.879462in}{1.217606in}}%
\pgfpathlineto{\pgfqpoint{4.879883in}{1.217606in}}%
\pgfpathlineto{\pgfqpoint{4.880726in}{1.224173in}}%
\pgfpathlineto{\pgfqpoint{4.882411in}{1.211038in}}%
\pgfpathlineto{\pgfqpoint{4.883254in}{1.224173in}}%
\pgfpathlineto{\pgfqpoint{4.884096in}{1.230741in}}%
\pgfpathlineto{\pgfqpoint{4.884517in}{1.211038in}}%
\pgfpathlineto{\pgfqpoint{4.885360in}{1.224173in}}%
\pgfpathlineto{\pgfqpoint{4.886624in}{1.204471in}}%
\pgfpathlineto{\pgfqpoint{4.887888in}{1.217606in}}%
\pgfpathlineto{\pgfqpoint{4.889151in}{1.217606in}}%
\pgfpathlineto{\pgfqpoint{4.889573in}{1.224173in}}%
\pgfpathlineto{\pgfqpoint{4.889994in}{1.204471in}}%
\pgfpathlineto{\pgfqpoint{4.890415in}{1.230741in}}%
\pgfpathlineto{\pgfqpoint{4.890837in}{1.211038in}}%
\pgfpathlineto{\pgfqpoint{4.891258in}{1.204471in}}%
\pgfpathlineto{\pgfqpoint{4.891679in}{1.224173in}}%
\pgfpathlineto{\pgfqpoint{4.892100in}{1.217606in}}%
\pgfpathlineto{\pgfqpoint{4.892522in}{1.211038in}}%
\pgfpathlineto{\pgfqpoint{4.893364in}{1.224173in}}%
\pgfpathlineto{\pgfqpoint{4.893785in}{1.217606in}}%
\pgfpathlineto{\pgfqpoint{4.894207in}{1.217606in}}%
\pgfpathlineto{\pgfqpoint{4.895471in}{1.230741in}}%
\pgfpathlineto{\pgfqpoint{4.895892in}{1.217606in}}%
\pgfpathlineto{\pgfqpoint{4.896734in}{1.224173in}}%
\pgfpathlineto{\pgfqpoint{4.897156in}{1.224173in}}%
\pgfpathlineto{\pgfqpoint{4.898420in}{1.230741in}}%
\pgfpathlineto{\pgfqpoint{4.900105in}{1.217606in}}%
\pgfpathlineto{\pgfqpoint{4.900526in}{1.230741in}}%
\pgfpathlineto{\pgfqpoint{4.901368in}{1.224173in}}%
\pgfpathlineto{\pgfqpoint{4.901790in}{1.230741in}}%
\pgfpathlineto{\pgfqpoint{4.902211in}{1.211038in}}%
\pgfpathlineto{\pgfqpoint{4.902632in}{1.230741in}}%
\pgfpathlineto{\pgfqpoint{4.903054in}{1.237308in}}%
\pgfpathlineto{\pgfqpoint{4.903475in}{1.230741in}}%
\pgfpathlineto{\pgfqpoint{4.904317in}{1.224173in}}%
\pgfpathlineto{\pgfqpoint{4.904739in}{1.230741in}}%
\pgfpathlineto{\pgfqpoint{4.905160in}{1.217606in}}%
\pgfpathlineto{\pgfqpoint{4.906003in}{1.224173in}}%
\pgfpathlineto{\pgfqpoint{4.906424in}{1.230741in}}%
\pgfpathlineto{\pgfqpoint{4.906845in}{1.224173in}}%
\pgfpathlineto{\pgfqpoint{4.907688in}{1.224173in}}%
\pgfpathlineto{\pgfqpoint{4.908109in}{1.237308in}}%
\pgfpathlineto{\pgfqpoint{4.908530in}{1.230741in}}%
\pgfpathlineto{\pgfqpoint{4.908951in}{1.217606in}}%
\pgfpathlineto{\pgfqpoint{4.909373in}{1.237308in}}%
\pgfpathlineto{\pgfqpoint{4.909794in}{1.224173in}}%
\pgfpathlineto{\pgfqpoint{4.910215in}{1.224173in}}%
\pgfpathlineto{\pgfqpoint{4.910637in}{1.230741in}}%
\pgfpathlineto{\pgfqpoint{4.911058in}{1.204471in}}%
\pgfpathlineto{\pgfqpoint{4.911479in}{1.230741in}}%
\pgfpathlineto{\pgfqpoint{4.913164in}{1.211038in}}%
\pgfpathlineto{\pgfqpoint{4.914428in}{1.243876in}}%
\pgfpathlineto{\pgfqpoint{4.915271in}{1.217606in}}%
\pgfpathlineto{\pgfqpoint{4.915692in}{1.243876in}}%
\pgfpathlineto{\pgfqpoint{4.916113in}{1.224173in}}%
\pgfpathlineto{\pgfqpoint{4.916535in}{1.224173in}}%
\pgfpathlineto{\pgfqpoint{4.918220in}{1.237308in}}%
\pgfpathlineto{\pgfqpoint{4.918641in}{1.237308in}}%
\pgfpathlineto{\pgfqpoint{4.919905in}{1.217606in}}%
\pgfpathlineto{\pgfqpoint{4.921169in}{1.237308in}}%
\pgfpathlineto{\pgfqpoint{4.922011in}{1.224173in}}%
\pgfpathlineto{\pgfqpoint{4.922432in}{1.230741in}}%
\pgfpathlineto{\pgfqpoint{4.923696in}{1.230741in}}%
\pgfpathlineto{\pgfqpoint{4.924118in}{1.224173in}}%
\pgfpathlineto{\pgfqpoint{4.924539in}{1.237308in}}%
\pgfpathlineto{\pgfqpoint{4.924960in}{1.230741in}}%
\pgfpathlineto{\pgfqpoint{4.925381in}{1.217606in}}%
\pgfpathlineto{\pgfqpoint{4.925803in}{1.237308in}}%
\pgfpathlineto{\pgfqpoint{4.926645in}{1.237308in}}%
\pgfpathlineto{\pgfqpoint{4.927066in}{1.230741in}}%
\pgfpathlineto{\pgfqpoint{4.927488in}{1.204471in}}%
\pgfpathlineto{\pgfqpoint{4.927909in}{1.211038in}}%
\pgfpathlineto{\pgfqpoint{4.928752in}{1.230741in}}%
\pgfpathlineto{\pgfqpoint{4.929173in}{1.224173in}}%
\pgfpathlineto{\pgfqpoint{4.929594in}{1.224173in}}%
\pgfpathlineto{\pgfqpoint{4.930015in}{1.230741in}}%
\pgfpathlineto{\pgfqpoint{4.931279in}{1.217606in}}%
\pgfpathlineto{\pgfqpoint{4.931701in}{1.217606in}}%
\pgfpathlineto{\pgfqpoint{4.932122in}{1.230741in}}%
\pgfpathlineto{\pgfqpoint{4.932543in}{1.224173in}}%
\pgfpathlineto{\pgfqpoint{4.932964in}{1.217606in}}%
\pgfpathlineto{\pgfqpoint{4.933386in}{1.243876in}}%
\pgfpathlineto{\pgfqpoint{4.933807in}{1.217606in}}%
\pgfpathlineto{\pgfqpoint{4.934228in}{1.217606in}}%
\pgfpathlineto{\pgfqpoint{4.935071in}{1.211038in}}%
\pgfpathlineto{\pgfqpoint{4.935913in}{1.217606in}}%
\pgfpathlineto{\pgfqpoint{4.936335in}{1.204471in}}%
\pgfpathlineto{\pgfqpoint{4.936756in}{1.224173in}}%
\pgfpathlineto{\pgfqpoint{4.938020in}{1.224173in}}%
\pgfpathlineto{\pgfqpoint{4.938862in}{1.230741in}}%
\pgfpathlineto{\pgfqpoint{4.939284in}{1.211038in}}%
\pgfpathlineto{\pgfqpoint{4.939705in}{1.211038in}}%
\pgfpathlineto{\pgfqpoint{4.940126in}{1.224173in}}%
\pgfpathlineto{\pgfqpoint{4.940547in}{1.204471in}}%
\pgfpathlineto{\pgfqpoint{4.940969in}{1.217606in}}%
\pgfpathlineto{\pgfqpoint{4.942232in}{1.224173in}}%
\pgfpathlineto{\pgfqpoint{4.943496in}{1.211038in}}%
\pgfpathlineto{\pgfqpoint{4.943918in}{1.230741in}}%
\pgfpathlineto{\pgfqpoint{4.944339in}{1.224173in}}%
\pgfpathlineto{\pgfqpoint{4.945603in}{1.211038in}}%
\pgfpathlineto{\pgfqpoint{4.946024in}{1.224173in}}%
\pgfpathlineto{\pgfqpoint{4.946445in}{1.217606in}}%
\pgfpathlineto{\pgfqpoint{4.946867in}{1.204471in}}%
\pgfpathlineto{\pgfqpoint{4.947288in}{1.224173in}}%
\pgfpathlineto{\pgfqpoint{4.947709in}{1.224173in}}%
\pgfpathlineto{\pgfqpoint{4.948130in}{1.217606in}}%
\pgfpathlineto{\pgfqpoint{4.948552in}{1.224173in}}%
\pgfpathlineto{\pgfqpoint{4.949394in}{1.224173in}}%
\pgfpathlineto{\pgfqpoint{4.950237in}{1.230741in}}%
\pgfpathlineto{\pgfqpoint{4.951079in}{1.217606in}}%
\pgfpathlineto{\pgfqpoint{4.951501in}{1.224173in}}%
\pgfpathlineto{\pgfqpoint{4.951922in}{1.230741in}}%
\pgfpathlineto{\pgfqpoint{4.952764in}{1.217606in}}%
\pgfpathlineto{\pgfqpoint{4.954028in}{1.237308in}}%
\pgfpathlineto{\pgfqpoint{4.954450in}{1.237308in}}%
\pgfpathlineto{\pgfqpoint{4.955713in}{1.211038in}}%
\pgfpathlineto{\pgfqpoint{4.956135in}{1.224173in}}%
\pgfpathlineto{\pgfqpoint{4.956556in}{1.224173in}}%
\pgfpathlineto{\pgfqpoint{4.956977in}{1.211038in}}%
\pgfpathlineto{\pgfqpoint{4.957820in}{1.217606in}}%
\pgfpathlineto{\pgfqpoint{4.958662in}{1.237308in}}%
\pgfpathlineto{\pgfqpoint{4.959084in}{1.224173in}}%
\pgfpathlineto{\pgfqpoint{4.959505in}{1.211038in}}%
\pgfpathlineto{\pgfqpoint{4.959926in}{1.217606in}}%
\pgfpathlineto{\pgfqpoint{4.960347in}{1.237308in}}%
\pgfpathlineto{\pgfqpoint{4.960769in}{1.217606in}}%
\pgfpathlineto{\pgfqpoint{4.962033in}{1.237308in}}%
\pgfpathlineto{\pgfqpoint{4.963718in}{1.217606in}}%
\pgfpathlineto{\pgfqpoint{4.964560in}{1.230741in}}%
\pgfpathlineto{\pgfqpoint{4.964981in}{1.224173in}}%
\pgfpathlineto{\pgfqpoint{4.965403in}{1.211038in}}%
\pgfpathlineto{\pgfqpoint{4.965824in}{1.217606in}}%
\pgfpathlineto{\pgfqpoint{4.966245in}{1.230741in}}%
\pgfpathlineto{\pgfqpoint{4.967088in}{1.224173in}}%
\pgfpathlineto{\pgfqpoint{4.967509in}{1.224173in}}%
\pgfpathlineto{\pgfqpoint{4.968352in}{1.211038in}}%
\pgfpathlineto{\pgfqpoint{4.968773in}{1.217606in}}%
\pgfpathlineto{\pgfqpoint{4.969616in}{1.230741in}}%
\pgfpathlineto{\pgfqpoint{4.970879in}{1.217606in}}%
\pgfpathlineto{\pgfqpoint{4.972564in}{1.237308in}}%
\pgfpathlineto{\pgfqpoint{4.973828in}{1.217606in}}%
\pgfpathlineto{\pgfqpoint{4.974671in}{1.237308in}}%
\pgfpathlineto{\pgfqpoint{4.975092in}{1.224173in}}%
\pgfpathlineto{\pgfqpoint{4.975935in}{1.217606in}}%
\pgfpathlineto{\pgfqpoint{4.977620in}{1.230741in}}%
\pgfpathlineto{\pgfqpoint{4.978041in}{1.217606in}}%
\pgfpathlineto{\pgfqpoint{4.978462in}{1.224173in}}%
\pgfpathlineto{\pgfqpoint{4.978884in}{1.237308in}}%
\pgfpathlineto{\pgfqpoint{4.979726in}{1.230741in}}%
\pgfpathlineto{\pgfqpoint{4.980990in}{1.217606in}}%
\pgfpathlineto{\pgfqpoint{4.982675in}{1.237308in}}%
\pgfpathlineto{\pgfqpoint{4.983518in}{1.224173in}}%
\pgfpathlineto{\pgfqpoint{4.983939in}{1.250443in}}%
\pgfpathlineto{\pgfqpoint{4.984360in}{1.230741in}}%
\pgfpathlineto{\pgfqpoint{4.985624in}{1.211038in}}%
\pgfpathlineto{\pgfqpoint{4.986467in}{1.230741in}}%
\pgfpathlineto{\pgfqpoint{4.986888in}{1.217606in}}%
\pgfpathlineto{\pgfqpoint{4.987731in}{1.230741in}}%
\pgfpathlineto{\pgfqpoint{4.988152in}{1.211038in}}%
\pgfpathlineto{\pgfqpoint{4.988994in}{1.217606in}}%
\pgfpathlineto{\pgfqpoint{4.990258in}{1.243876in}}%
\pgfpathlineto{\pgfqpoint{4.991943in}{1.211038in}}%
\pgfpathlineto{\pgfqpoint{4.992786in}{1.224173in}}%
\pgfpathlineto{\pgfqpoint{4.993207in}{1.211038in}}%
\pgfpathlineto{\pgfqpoint{4.993628in}{1.224173in}}%
\pgfpathlineto{\pgfqpoint{4.994050in}{1.230741in}}%
\pgfpathlineto{\pgfqpoint{4.994471in}{1.224173in}}%
\pgfpathlineto{\pgfqpoint{4.995314in}{1.217606in}}%
\pgfpathlineto{\pgfqpoint{4.996577in}{1.224173in}}%
\pgfpathlineto{\pgfqpoint{4.997841in}{1.217606in}}%
\pgfpathlineto{\pgfqpoint{4.999105in}{1.237308in}}%
\pgfpathlineto{\pgfqpoint{4.999526in}{1.211038in}}%
\pgfpathlineto{\pgfqpoint{5.000369in}{1.224173in}}%
\pgfpathlineto{\pgfqpoint{5.001633in}{1.224173in}}%
\pgfpathlineto{\pgfqpoint{5.002475in}{1.217606in}}%
\pgfpathlineto{\pgfqpoint{5.003318in}{1.224173in}}%
\pgfpathlineto{\pgfqpoint{5.003739in}{1.211038in}}%
\pgfpathlineto{\pgfqpoint{5.004582in}{1.217606in}}%
\pgfpathlineto{\pgfqpoint{5.005424in}{1.230741in}}%
\pgfpathlineto{\pgfqpoint{5.005845in}{1.224173in}}%
\pgfpathlineto{\pgfqpoint{5.006267in}{1.217606in}}%
\pgfpathlineto{\pgfqpoint{5.006688in}{1.230741in}}%
\pgfpathlineto{\pgfqpoint{5.007109in}{1.217606in}}%
\pgfpathlineto{\pgfqpoint{5.007531in}{1.217606in}}%
\pgfpathlineto{\pgfqpoint{5.009216in}{1.230741in}}%
\pgfpathlineto{\pgfqpoint{5.010058in}{1.211038in}}%
\pgfpathlineto{\pgfqpoint{5.010480in}{1.243876in}}%
\pgfpathlineto{\pgfqpoint{5.011322in}{1.224173in}}%
\pgfpathlineto{\pgfqpoint{5.012165in}{1.224173in}}%
\pgfpathlineto{\pgfqpoint{5.012586in}{1.204471in}}%
\pgfpathlineto{\pgfqpoint{5.013007in}{1.230741in}}%
\pgfpathlineto{\pgfqpoint{5.014271in}{1.230741in}}%
\pgfpathlineto{\pgfqpoint{5.015535in}{1.224173in}}%
\pgfpathlineto{\pgfqpoint{5.015956in}{1.237308in}}%
\pgfpathlineto{\pgfqpoint{5.016377in}{1.217606in}}%
\pgfpathlineto{\pgfqpoint{5.016799in}{1.230741in}}%
\pgfpathlineto{\pgfqpoint{5.017220in}{1.217606in}}%
\pgfpathlineto{\pgfqpoint{5.017641in}{1.230741in}}%
\pgfpathlineto{\pgfqpoint{5.018063in}{1.237308in}}%
\pgfpathlineto{\pgfqpoint{5.018484in}{1.217606in}}%
\pgfpathlineto{\pgfqpoint{5.019326in}{1.224173in}}%
\pgfpathlineto{\pgfqpoint{5.019748in}{1.224173in}}%
\pgfpathlineto{\pgfqpoint{5.020169in}{1.217606in}}%
\pgfpathlineto{\pgfqpoint{5.021433in}{1.230741in}}%
\pgfpathlineto{\pgfqpoint{5.021854in}{1.230741in}}%
\pgfpathlineto{\pgfqpoint{5.022275in}{1.217606in}}%
\pgfpathlineto{\pgfqpoint{5.022697in}{1.224173in}}%
\pgfpathlineto{\pgfqpoint{5.023539in}{1.230741in}}%
\pgfpathlineto{\pgfqpoint{5.025224in}{1.217606in}}%
\pgfpathlineto{\pgfqpoint{5.026067in}{1.230741in}}%
\pgfpathlineto{\pgfqpoint{5.027331in}{1.217606in}}%
\pgfpathlineto{\pgfqpoint{5.027752in}{1.217606in}}%
\pgfpathlineto{\pgfqpoint{5.029016in}{1.230741in}}%
\pgfpathlineto{\pgfqpoint{5.029437in}{1.217606in}}%
\pgfpathlineto{\pgfqpoint{5.030280in}{1.224173in}}%
\pgfpathlineto{\pgfqpoint{5.030701in}{1.237308in}}%
\pgfpathlineto{\pgfqpoint{5.031122in}{1.217606in}}%
\pgfpathlineto{\pgfqpoint{5.031543in}{1.211038in}}%
\pgfpathlineto{\pgfqpoint{5.031965in}{1.230741in}}%
\pgfpathlineto{\pgfqpoint{5.032807in}{1.224173in}}%
\pgfpathlineto{\pgfqpoint{5.033229in}{1.237308in}}%
\pgfpathlineto{\pgfqpoint{5.033650in}{1.230741in}}%
\pgfpathlineto{\pgfqpoint{5.034071in}{1.224173in}}%
\pgfpathlineto{\pgfqpoint{5.034492in}{1.237308in}}%
\pgfpathlineto{\pgfqpoint{5.034914in}{1.224173in}}%
\pgfpathlineto{\pgfqpoint{5.035335in}{1.217606in}}%
\pgfpathlineto{\pgfqpoint{5.035756in}{1.230741in}}%
\pgfpathlineto{\pgfqpoint{5.036177in}{1.224173in}}%
\pgfpathlineto{\pgfqpoint{5.036599in}{1.217606in}}%
\pgfpathlineto{\pgfqpoint{5.037441in}{1.211038in}}%
\pgfpathlineto{\pgfqpoint{5.037863in}{1.230741in}}%
\pgfpathlineto{\pgfqpoint{5.039126in}{1.217606in}}%
\pgfpathlineto{\pgfqpoint{5.040812in}{1.237308in}}%
\pgfpathlineto{\pgfqpoint{5.041654in}{1.211038in}}%
\pgfpathlineto{\pgfqpoint{5.042075in}{1.217606in}}%
\pgfpathlineto{\pgfqpoint{5.042497in}{1.211038in}}%
\pgfpathlineto{\pgfqpoint{5.042918in}{1.230741in}}%
\pgfpathlineto{\pgfqpoint{5.043761in}{1.224173in}}%
\pgfpathlineto{\pgfqpoint{5.044603in}{1.224173in}}%
\pgfpathlineto{\pgfqpoint{5.045024in}{1.230741in}}%
\pgfpathlineto{\pgfqpoint{5.046288in}{1.217606in}}%
\pgfpathlineto{\pgfqpoint{5.047131in}{1.217606in}}%
\pgfpathlineto{\pgfqpoint{5.047552in}{1.211038in}}%
\pgfpathlineto{\pgfqpoint{5.047973in}{1.217606in}}%
\pgfpathlineto{\pgfqpoint{5.048816in}{1.224173in}}%
\pgfpathlineto{\pgfqpoint{5.049658in}{1.211038in}}%
\pgfpathlineto{\pgfqpoint{5.050080in}{1.230741in}}%
\pgfpathlineto{\pgfqpoint{5.050922in}{1.224173in}}%
\pgfpathlineto{\pgfqpoint{5.051765in}{1.224173in}}%
\pgfpathlineto{\pgfqpoint{5.052607in}{1.217606in}}%
\pgfpathlineto{\pgfqpoint{5.053871in}{1.224173in}}%
\pgfpathlineto{\pgfqpoint{5.054292in}{1.211038in}}%
\pgfpathlineto{\pgfqpoint{5.054714in}{1.224173in}}%
\pgfpathlineto{\pgfqpoint{5.055978in}{1.237308in}}%
\pgfpathlineto{\pgfqpoint{5.056820in}{1.211038in}}%
\pgfpathlineto{\pgfqpoint{5.057241in}{1.224173in}}%
\pgfpathlineto{\pgfqpoint{5.058084in}{1.224173in}}%
\pgfpathlineto{\pgfqpoint{5.058927in}{1.230741in}}%
\pgfpathlineto{\pgfqpoint{5.059348in}{1.217606in}}%
\pgfpathlineto{\pgfqpoint{5.060190in}{1.224173in}}%
\pgfpathlineto{\pgfqpoint{5.060612in}{1.224173in}}%
\pgfpathlineto{\pgfqpoint{5.061033in}{1.243876in}}%
\pgfpathlineto{\pgfqpoint{5.061454in}{1.217606in}}%
\pgfpathlineto{\pgfqpoint{5.062297in}{1.217606in}}%
\pgfpathlineto{\pgfqpoint{5.063139in}{1.237308in}}%
\pgfpathlineto{\pgfqpoint{5.063982in}{1.217606in}}%
\pgfpathlineto{\pgfqpoint{5.065246in}{1.230741in}}%
\pgfpathlineto{\pgfqpoint{5.065667in}{1.224173in}}%
\pgfpathlineto{\pgfqpoint{5.066088in}{1.237308in}}%
\pgfpathlineto{\pgfqpoint{5.066510in}{1.217606in}}%
\pgfpathlineto{\pgfqpoint{5.066931in}{1.217606in}}%
\pgfpathlineto{\pgfqpoint{5.068616in}{1.230741in}}%
\pgfpathlineto{\pgfqpoint{5.069880in}{1.224173in}}%
\pgfpathlineto{\pgfqpoint{5.070301in}{1.237308in}}%
\pgfpathlineto{\pgfqpoint{5.071144in}{1.230741in}}%
\pgfpathlineto{\pgfqpoint{5.071986in}{1.217606in}}%
\pgfpathlineto{\pgfqpoint{5.072407in}{1.224173in}}%
\pgfpathlineto{\pgfqpoint{5.072829in}{1.230741in}}%
\pgfpathlineto{\pgfqpoint{5.073250in}{1.224173in}}%
\pgfpathlineto{\pgfqpoint{5.073671in}{1.217606in}}%
\pgfpathlineto{\pgfqpoint{5.074093in}{1.224173in}}%
\pgfpathlineto{\pgfqpoint{5.074514in}{1.224173in}}%
\pgfpathlineto{\pgfqpoint{5.075778in}{1.237308in}}%
\pgfpathlineto{\pgfqpoint{5.077041in}{1.211038in}}%
\pgfpathlineto{\pgfqpoint{5.077884in}{1.204471in}}%
\pgfpathlineto{\pgfqpoint{5.078305in}{1.224173in}}%
\pgfpathlineto{\pgfqpoint{5.079148in}{1.224173in}}%
\pgfpathlineto{\pgfqpoint{5.079569in}{1.211038in}}%
\pgfpathlineto{\pgfqpoint{5.079990in}{1.224173in}}%
\pgfpathlineto{\pgfqpoint{5.080412in}{1.224173in}}%
\pgfpathlineto{\pgfqpoint{5.080833in}{1.217606in}}%
\pgfpathlineto{\pgfqpoint{5.081254in}{1.230741in}}%
\pgfpathlineto{\pgfqpoint{5.081676in}{1.211038in}}%
\pgfpathlineto{\pgfqpoint{5.082518in}{1.211038in}}%
\pgfpathlineto{\pgfqpoint{5.083782in}{1.224173in}}%
\pgfpathlineto{\pgfqpoint{5.084624in}{1.211038in}}%
\pgfpathlineto{\pgfqpoint{5.085046in}{1.230741in}}%
\pgfpathlineto{\pgfqpoint{5.085467in}{1.224173in}}%
\pgfpathlineto{\pgfqpoint{5.086731in}{1.217606in}}%
\pgfpathlineto{\pgfqpoint{5.087573in}{1.224173in}}%
\pgfpathlineto{\pgfqpoint{5.087995in}{1.204471in}}%
\pgfpathlineto{\pgfqpoint{5.088416in}{1.224173in}}%
\pgfpathlineto{\pgfqpoint{5.088837in}{1.217606in}}%
\pgfpathlineto{\pgfqpoint{5.089259in}{1.224173in}}%
\pgfpathlineto{\pgfqpoint{5.090522in}{1.237308in}}%
\pgfpathlineto{\pgfqpoint{5.090944in}{1.230741in}}%
\pgfpathlineto{\pgfqpoint{5.091365in}{1.243876in}}%
\pgfpathlineto{\pgfqpoint{5.092207in}{1.217606in}}%
\pgfpathlineto{\pgfqpoint{5.092629in}{1.230741in}}%
\pgfpathlineto{\pgfqpoint{5.093050in}{1.224173in}}%
\pgfpathlineto{\pgfqpoint{5.093893in}{1.237308in}}%
\pgfpathlineto{\pgfqpoint{5.094314in}{1.230741in}}%
\pgfpathlineto{\pgfqpoint{5.094735in}{1.224173in}}%
\pgfpathlineto{\pgfqpoint{5.095156in}{1.243876in}}%
\pgfpathlineto{\pgfqpoint{5.095999in}{1.237308in}}%
\pgfpathlineto{\pgfqpoint{5.097263in}{1.224173in}}%
\pgfpathlineto{\pgfqpoint{5.097684in}{1.237308in}}%
\pgfpathlineto{\pgfqpoint{5.098105in}{1.224173in}}%
\pgfpathlineto{\pgfqpoint{5.098527in}{1.224173in}}%
\pgfpathlineto{\pgfqpoint{5.099369in}{1.230741in}}%
\pgfpathlineto{\pgfqpoint{5.099790in}{1.224173in}}%
\pgfpathlineto{\pgfqpoint{5.100212in}{1.230741in}}%
\pgfpathlineto{\pgfqpoint{5.101476in}{1.237308in}}%
\pgfpathlineto{\pgfqpoint{5.101897in}{1.224173in}}%
\pgfpathlineto{\pgfqpoint{5.102739in}{1.230741in}}%
\pgfpathlineto{\pgfqpoint{5.103582in}{1.224173in}}%
\pgfpathlineto{\pgfqpoint{5.104003in}{1.243876in}}%
\pgfpathlineto{\pgfqpoint{5.104425in}{1.224173in}}%
\pgfpathlineto{\pgfqpoint{5.105688in}{1.230741in}}%
\pgfpathlineto{\pgfqpoint{5.106110in}{1.230741in}}%
\pgfpathlineto{\pgfqpoint{5.106531in}{1.250443in}}%
\pgfpathlineto{\pgfqpoint{5.106952in}{1.217606in}}%
\pgfpathlineto{\pgfqpoint{5.107795in}{1.237308in}}%
\pgfpathlineto{\pgfqpoint{5.108216in}{1.217606in}}%
\pgfpathlineto{\pgfqpoint{5.108637in}{1.224173in}}%
\pgfpathlineto{\pgfqpoint{5.109059in}{1.237308in}}%
\pgfpathlineto{\pgfqpoint{5.109480in}{1.224173in}}%
\pgfpathlineto{\pgfqpoint{5.111165in}{1.224173in}}%
\pgfpathlineto{\pgfqpoint{5.112429in}{1.237308in}}%
\pgfpathlineto{\pgfqpoint{5.113271in}{1.217606in}}%
\pgfpathlineto{\pgfqpoint{5.113693in}{1.224173in}}%
\pgfpathlineto{\pgfqpoint{5.114114in}{1.224173in}}%
\pgfpathlineto{\pgfqpoint{5.114535in}{1.217606in}}%
\pgfpathlineto{\pgfqpoint{5.114957in}{1.224173in}}%
\pgfpathlineto{\pgfqpoint{5.115799in}{1.224173in}}%
\pgfpathlineto{\pgfqpoint{5.116642in}{1.217606in}}%
\pgfpathlineto{\pgfqpoint{5.117063in}{1.224173in}}%
\pgfpathlineto{\pgfqpoint{5.117484in}{1.204471in}}%
\pgfpathlineto{\pgfqpoint{5.118327in}{1.211038in}}%
\pgfpathlineto{\pgfqpoint{5.119169in}{1.230741in}}%
\pgfpathlineto{\pgfqpoint{5.119591in}{1.217606in}}%
\pgfpathlineto{\pgfqpoint{5.120012in}{1.211038in}}%
\pgfpathlineto{\pgfqpoint{5.120854in}{1.224173in}}%
\pgfpathlineto{\pgfqpoint{5.121276in}{1.204471in}}%
\pgfpathlineto{\pgfqpoint{5.121697in}{1.243876in}}%
\pgfpathlineto{\pgfqpoint{5.122118in}{1.217606in}}%
\pgfpathlineto{\pgfqpoint{5.122540in}{1.204471in}}%
\pgfpathlineto{\pgfqpoint{5.122961in}{1.224173in}}%
\pgfpathlineto{\pgfqpoint{5.124646in}{1.224173in}}%
\pgfpathlineto{\pgfqpoint{5.125488in}{1.230741in}}%
\pgfpathlineto{\pgfqpoint{5.125910in}{1.211038in}}%
\pgfpathlineto{\pgfqpoint{5.126752in}{1.230741in}}%
\pgfpathlineto{\pgfqpoint{5.127174in}{1.224173in}}%
\pgfpathlineto{\pgfqpoint{5.128437in}{1.204471in}}%
\pgfpathlineto{\pgfqpoint{5.128859in}{1.230741in}}%
\pgfpathlineto{\pgfqpoint{5.129701in}{1.224173in}}%
\pgfpathlineto{\pgfqpoint{5.130123in}{1.224173in}}%
\pgfpathlineto{\pgfqpoint{5.130544in}{1.217606in}}%
\pgfpathlineto{\pgfqpoint{5.131386in}{1.230741in}}%
\pgfpathlineto{\pgfqpoint{5.131808in}{1.224173in}}%
\pgfpathlineto{\pgfqpoint{5.132650in}{1.204471in}}%
\pgfpathlineto{\pgfqpoint{5.133914in}{1.224173in}}%
\pgfpathlineto{\pgfqpoint{5.134335in}{1.224173in}}%
\pgfpathlineto{\pgfqpoint{5.134757in}{1.237308in}}%
\pgfpathlineto{\pgfqpoint{5.135178in}{1.211038in}}%
\pgfpathlineto{\pgfqpoint{5.136020in}{1.224173in}}%
\pgfpathlineto{\pgfqpoint{5.136442in}{1.217606in}}%
\pgfpathlineto{\pgfqpoint{5.136863in}{1.224173in}}%
\pgfpathlineto{\pgfqpoint{5.137284in}{1.224173in}}%
\pgfpathlineto{\pgfqpoint{5.137706in}{1.230741in}}%
\pgfpathlineto{\pgfqpoint{5.138969in}{1.217606in}}%
\pgfpathlineto{\pgfqpoint{5.140233in}{1.217606in}}%
\pgfpathlineto{\pgfqpoint{5.141918in}{1.243876in}}%
\pgfpathlineto{\pgfqpoint{5.142340in}{1.204471in}}%
\pgfpathlineto{\pgfqpoint{5.143182in}{1.217606in}}%
\pgfpathlineto{\pgfqpoint{5.144025in}{1.230741in}}%
\pgfpathlineto{\pgfqpoint{5.144446in}{1.224173in}}%
\pgfpathlineto{\pgfqpoint{5.144867in}{1.217606in}}%
\pgfpathlineto{\pgfqpoint{5.145289in}{1.224173in}}%
\pgfpathlineto{\pgfqpoint{5.145710in}{1.224173in}}%
\pgfpathlineto{\pgfqpoint{5.146131in}{1.230741in}}%
\pgfpathlineto{\pgfqpoint{5.146552in}{1.217606in}}%
\pgfpathlineto{\pgfqpoint{5.147395in}{1.224173in}}%
\pgfpathlineto{\pgfqpoint{5.147816in}{1.224173in}}%
\pgfpathlineto{\pgfqpoint{5.148659in}{1.211038in}}%
\pgfpathlineto{\pgfqpoint{5.149501in}{1.224173in}}%
\pgfpathlineto{\pgfqpoint{5.149923in}{1.217606in}}%
\pgfpathlineto{\pgfqpoint{5.151608in}{1.230741in}}%
\pgfpathlineto{\pgfqpoint{5.152872in}{1.217606in}}%
\pgfpathlineto{\pgfqpoint{5.153293in}{1.230741in}}%
\pgfpathlineto{\pgfqpoint{5.153714in}{1.224173in}}%
\pgfpathlineto{\pgfqpoint{5.154135in}{1.217606in}}%
\pgfpathlineto{\pgfqpoint{5.155820in}{1.243876in}}%
\pgfpathlineto{\pgfqpoint{5.157084in}{1.224173in}}%
\pgfpathlineto{\pgfqpoint{5.157506in}{1.217606in}}%
\pgfpathlineto{\pgfqpoint{5.157927in}{1.224173in}}%
\pgfpathlineto{\pgfqpoint{5.160876in}{1.224173in}}%
\pgfpathlineto{\pgfqpoint{5.161297in}{1.217606in}}%
\pgfpathlineto{\pgfqpoint{5.162140in}{1.230741in}}%
\pgfpathlineto{\pgfqpoint{5.162561in}{1.211038in}}%
\pgfpathlineto{\pgfqpoint{5.162982in}{1.224173in}}%
\pgfpathlineto{\pgfqpoint{5.163403in}{1.230741in}}%
\pgfpathlineto{\pgfqpoint{5.163825in}{1.224173in}}%
\pgfpathlineto{\pgfqpoint{5.164246in}{1.217606in}}%
\pgfpathlineto{\pgfqpoint{5.164667in}{1.224173in}}%
\pgfpathlineto{\pgfqpoint{5.165089in}{1.224173in}}%
\pgfpathlineto{\pgfqpoint{5.165510in}{1.230741in}}%
\pgfpathlineto{\pgfqpoint{5.166352in}{1.237308in}}%
\pgfpathlineto{\pgfqpoint{5.166774in}{1.217606in}}%
\pgfpathlineto{\pgfqpoint{5.167616in}{1.224173in}}%
\pgfpathlineto{\pgfqpoint{5.168038in}{1.211038in}}%
\pgfpathlineto{\pgfqpoint{5.168880in}{1.217606in}}%
\pgfpathlineto{\pgfqpoint{5.169301in}{1.217606in}}%
\pgfpathlineto{\pgfqpoint{5.170144in}{1.211038in}}%
\pgfpathlineto{\pgfqpoint{5.170986in}{1.237308in}}%
\pgfpathlineto{\pgfqpoint{5.172250in}{1.224173in}}%
\pgfpathlineto{\pgfqpoint{5.173514in}{1.224173in}}%
\pgfpathlineto{\pgfqpoint{5.173935in}{1.217606in}}%
\pgfpathlineto{\pgfqpoint{5.174357in}{1.224173in}}%
\pgfpathlineto{\pgfqpoint{5.174778in}{1.230741in}}%
\pgfpathlineto{\pgfqpoint{5.175199in}{1.211038in}}%
\pgfpathlineto{\pgfqpoint{5.175621in}{1.230741in}}%
\pgfpathlineto{\pgfqpoint{5.176042in}{1.243876in}}%
\pgfpathlineto{\pgfqpoint{5.176463in}{1.230741in}}%
\pgfpathlineto{\pgfqpoint{5.177306in}{1.224173in}}%
\pgfpathlineto{\pgfqpoint{5.177727in}{1.230741in}}%
\pgfpathlineto{\pgfqpoint{5.178148in}{1.217606in}}%
\pgfpathlineto{\pgfqpoint{5.178991in}{1.224173in}}%
\pgfpathlineto{\pgfqpoint{5.179833in}{1.224173in}}%
\pgfpathlineto{\pgfqpoint{5.180676in}{1.237308in}}%
\pgfpathlineto{\pgfqpoint{5.181097in}{1.230741in}}%
\pgfpathlineto{\pgfqpoint{5.181518in}{1.230741in}}%
\pgfpathlineto{\pgfqpoint{5.183204in}{1.211038in}}%
\pgfpathlineto{\pgfqpoint{5.183625in}{1.224173in}}%
\pgfpathlineto{\pgfqpoint{5.184467in}{1.217606in}}%
\pgfpathlineto{\pgfqpoint{5.185731in}{1.230741in}}%
\pgfpathlineto{\pgfqpoint{5.186153in}{1.224173in}}%
\pgfpathlineto{\pgfqpoint{5.186574in}{1.237308in}}%
\pgfpathlineto{\pgfqpoint{5.187416in}{1.230741in}}%
\pgfpathlineto{\pgfqpoint{5.189101in}{1.217606in}}%
\pgfpathlineto{\pgfqpoint{5.189523in}{1.224173in}}%
\pgfpathlineto{\pgfqpoint{5.189944in}{1.211038in}}%
\pgfpathlineto{\pgfqpoint{5.190365in}{1.230741in}}%
\pgfpathlineto{\pgfqpoint{5.190787in}{1.224173in}}%
\pgfpathlineto{\pgfqpoint{5.191208in}{1.237308in}}%
\pgfpathlineto{\pgfqpoint{5.191629in}{1.230741in}}%
\pgfpathlineto{\pgfqpoint{5.192050in}{1.211038in}}%
\pgfpathlineto{\pgfqpoint{5.192472in}{1.217606in}}%
\pgfpathlineto{\pgfqpoint{5.192893in}{1.224173in}}%
\pgfpathlineto{\pgfqpoint{5.194157in}{1.211038in}}%
\pgfpathlineto{\pgfqpoint{5.195842in}{1.237308in}}%
\pgfpathlineto{\pgfqpoint{5.196263in}{1.224173in}}%
\pgfpathlineto{\pgfqpoint{5.197106in}{1.230741in}}%
\pgfpathlineto{\pgfqpoint{5.197948in}{1.217606in}}%
\pgfpathlineto{\pgfqpoint{5.198370in}{1.224173in}}%
\pgfpathlineto{\pgfqpoint{5.198791in}{1.230741in}}%
\pgfpathlineto{\pgfqpoint{5.200055in}{1.211038in}}%
\pgfpathlineto{\pgfqpoint{5.200897in}{1.224173in}}%
\pgfpathlineto{\pgfqpoint{5.201319in}{1.217606in}}%
\pgfpathlineto{\pgfqpoint{5.201740in}{1.204471in}}%
\pgfpathlineto{\pgfqpoint{5.202161in}{1.230741in}}%
\pgfpathlineto{\pgfqpoint{5.203004in}{1.217606in}}%
\pgfpathlineto{\pgfqpoint{5.203425in}{1.224173in}}%
\pgfpathlineto{\pgfqpoint{5.203846in}{1.217606in}}%
\pgfpathlineto{\pgfqpoint{5.204689in}{1.211038in}}%
\pgfpathlineto{\pgfqpoint{5.205953in}{1.230741in}}%
\pgfpathlineto{\pgfqpoint{5.206374in}{1.224173in}}%
\pgfpathlineto{\pgfqpoint{5.206795in}{1.224173in}}%
\pgfpathlineto{\pgfqpoint{5.207216in}{1.270146in}}%
\pgfpathlineto{\pgfqpoint{5.207638in}{1.230741in}}%
\pgfpathlineto{\pgfqpoint{5.208480in}{1.211038in}}%
\pgfpathlineto{\pgfqpoint{5.209323in}{1.204471in}}%
\pgfpathlineto{\pgfqpoint{5.209744in}{1.224173in}}%
\pgfpathlineto{\pgfqpoint{5.211008in}{1.217606in}}%
\pgfpathlineto{\pgfqpoint{5.212272in}{1.224173in}}%
\pgfpathlineto{\pgfqpoint{5.212693in}{1.224173in}}%
\pgfpathlineto{\pgfqpoint{5.213536in}{1.217606in}}%
\pgfpathlineto{\pgfqpoint{5.214799in}{1.230741in}}%
\pgfpathlineto{\pgfqpoint{5.215221in}{1.230741in}}%
\pgfpathlineto{\pgfqpoint{5.215642in}{1.217606in}}%
\pgfpathlineto{\pgfqpoint{5.216485in}{1.224173in}}%
\pgfpathlineto{\pgfqpoint{5.216906in}{1.217606in}}%
\pgfpathlineto{\pgfqpoint{5.218170in}{1.237308in}}%
\pgfpathlineto{\pgfqpoint{5.219433in}{1.211038in}}%
\pgfpathlineto{\pgfqpoint{5.219855in}{1.211038in}}%
\pgfpathlineto{\pgfqpoint{5.220276in}{1.230741in}}%
\pgfpathlineto{\pgfqpoint{5.220697in}{1.184768in}}%
\pgfpathlineto{\pgfqpoint{5.221119in}{1.224173in}}%
\pgfpathlineto{\pgfqpoint{5.221540in}{1.237308in}}%
\pgfpathlineto{\pgfqpoint{5.221961in}{1.224173in}}%
\pgfpathlineto{\pgfqpoint{5.222382in}{1.217606in}}%
\pgfpathlineto{\pgfqpoint{5.222804in}{1.237308in}}%
\pgfpathlineto{\pgfqpoint{5.223225in}{1.224173in}}%
\pgfpathlineto{\pgfqpoint{5.223646in}{1.224173in}}%
\pgfpathlineto{\pgfqpoint{5.224068in}{1.230741in}}%
\pgfpathlineto{\pgfqpoint{5.224489in}{1.224173in}}%
\pgfpathlineto{\pgfqpoint{5.225753in}{1.224173in}}%
\pgfpathlineto{\pgfqpoint{5.227016in}{1.230741in}}%
\pgfpathlineto{\pgfqpoint{5.227438in}{1.217606in}}%
\pgfpathlineto{\pgfqpoint{5.228280in}{1.224173in}}%
\pgfpathlineto{\pgfqpoint{5.228702in}{1.217606in}}%
\pgfpathlineto{\pgfqpoint{5.229123in}{1.237308in}}%
\pgfpathlineto{\pgfqpoint{5.229544in}{1.204471in}}%
\pgfpathlineto{\pgfqpoint{5.229965in}{1.217606in}}%
\pgfpathlineto{\pgfqpoint{5.230387in}{1.230741in}}%
\pgfpathlineto{\pgfqpoint{5.231229in}{1.224173in}}%
\pgfpathlineto{\pgfqpoint{5.231651in}{1.211038in}}%
\pgfpathlineto{\pgfqpoint{5.232072in}{1.230741in}}%
\pgfpathlineto{\pgfqpoint{5.232493in}{1.230741in}}%
\pgfpathlineto{\pgfqpoint{5.232914in}{1.243876in}}%
\pgfpathlineto{\pgfqpoint{5.234178in}{1.217606in}}%
\pgfpathlineto{\pgfqpoint{5.234599in}{1.217606in}}%
\pgfpathlineto{\pgfqpoint{5.235863in}{1.224173in}}%
\pgfpathlineto{\pgfqpoint{5.236285in}{1.224173in}}%
\pgfpathlineto{\pgfqpoint{5.236706in}{1.237308in}}%
\pgfpathlineto{\pgfqpoint{5.237548in}{1.230741in}}%
\pgfpathlineto{\pgfqpoint{5.237970in}{1.230741in}}%
\pgfpathlineto{\pgfqpoint{5.239655in}{1.211038in}}%
\pgfpathlineto{\pgfqpoint{5.240919in}{1.230741in}}%
\pgfpathlineto{\pgfqpoint{5.241761in}{1.224173in}}%
\pgfpathlineto{\pgfqpoint{5.242183in}{1.237308in}}%
\pgfpathlineto{\pgfqpoint{5.242604in}{1.211038in}}%
\pgfpathlineto{\pgfqpoint{5.243025in}{1.250443in}}%
\pgfpathlineto{\pgfqpoint{5.244710in}{1.217606in}}%
\pgfpathlineto{\pgfqpoint{5.245974in}{1.224173in}}%
\pgfpathlineto{\pgfqpoint{5.246395in}{1.211038in}}%
\pgfpathlineto{\pgfqpoint{5.246817in}{1.250443in}}%
\pgfpathlineto{\pgfqpoint{5.247238in}{1.230741in}}%
\pgfpathlineto{\pgfqpoint{5.247659in}{1.224173in}}%
\pgfpathlineto{\pgfqpoint{5.248080in}{1.237308in}}%
\pgfpathlineto{\pgfqpoint{5.248502in}{1.224173in}}%
\pgfpathlineto{\pgfqpoint{5.249344in}{1.224173in}}%
\pgfpathlineto{\pgfqpoint{5.250187in}{1.204471in}}%
\pgfpathlineto{\pgfqpoint{5.250608in}{1.230741in}}%
\pgfpathlineto{\pgfqpoint{5.251451in}{1.217606in}}%
\pgfpathlineto{\pgfqpoint{5.253136in}{1.237308in}}%
\pgfpathlineto{\pgfqpoint{5.253978in}{1.217606in}}%
\pgfpathlineto{\pgfqpoint{5.254400in}{1.224173in}}%
\pgfpathlineto{\pgfqpoint{5.255242in}{1.224173in}}%
\pgfpathlineto{\pgfqpoint{5.255663in}{1.217606in}}%
\pgfpathlineto{\pgfqpoint{5.256085in}{1.224173in}}%
\pgfpathlineto{\pgfqpoint{5.256506in}{1.224173in}}%
\pgfpathlineto{\pgfqpoint{5.257770in}{1.217606in}}%
\pgfpathlineto{\pgfqpoint{5.258191in}{1.237308in}}%
\pgfpathlineto{\pgfqpoint{5.258612in}{1.224173in}}%
\pgfpathlineto{\pgfqpoint{5.259455in}{1.217606in}}%
\pgfpathlineto{\pgfqpoint{5.259876in}{1.224173in}}%
\pgfpathlineto{\pgfqpoint{5.260297in}{1.211038in}}%
\pgfpathlineto{\pgfqpoint{5.260719in}{1.230741in}}%
\pgfpathlineto{\pgfqpoint{5.262404in}{1.217606in}}%
\pgfpathlineto{\pgfqpoint{5.263668in}{1.237308in}}%
\pgfpathlineto{\pgfqpoint{5.264510in}{1.217606in}}%
\pgfpathlineto{\pgfqpoint{5.264932in}{1.224173in}}%
\pgfpathlineto{\pgfqpoint{5.265353in}{1.217606in}}%
\pgfpathlineto{\pgfqpoint{5.265774in}{1.237308in}}%
\pgfpathlineto{\pgfqpoint{5.266195in}{1.224173in}}%
\pgfpathlineto{\pgfqpoint{5.266617in}{1.224173in}}%
\pgfpathlineto{\pgfqpoint{5.267880in}{1.217606in}}%
\pgfpathlineto{\pgfqpoint{5.268723in}{1.224173in}}%
\pgfpathlineto{\pgfqpoint{5.269566in}{1.230741in}}%
\pgfpathlineto{\pgfqpoint{5.269987in}{1.211038in}}%
\pgfpathlineto{\pgfqpoint{5.270829in}{1.230741in}}%
\pgfpathlineto{\pgfqpoint{5.271251in}{1.211038in}}%
\pgfpathlineto{\pgfqpoint{5.271672in}{1.217606in}}%
\pgfpathlineto{\pgfqpoint{5.273357in}{1.230741in}}%
\pgfpathlineto{\pgfqpoint{5.274200in}{1.217606in}}%
\pgfpathlineto{\pgfqpoint{5.274621in}{1.230741in}}%
\pgfpathlineto{\pgfqpoint{5.275463in}{1.224173in}}%
\pgfpathlineto{\pgfqpoint{5.275885in}{1.230741in}}%
\pgfpathlineto{\pgfqpoint{5.277149in}{1.211038in}}%
\pgfpathlineto{\pgfqpoint{5.278412in}{1.230741in}}%
\pgfpathlineto{\pgfqpoint{5.279255in}{1.217606in}}%
\pgfpathlineto{\pgfqpoint{5.280519in}{1.230741in}}%
\pgfpathlineto{\pgfqpoint{5.281361in}{1.224173in}}%
\pgfpathlineto{\pgfqpoint{5.281783in}{1.230741in}}%
\pgfpathlineto{\pgfqpoint{5.282625in}{1.237308in}}%
\pgfpathlineto{\pgfqpoint{5.283046in}{1.217606in}}%
\pgfpathlineto{\pgfqpoint{5.283889in}{1.230741in}}%
\pgfpathlineto{\pgfqpoint{5.284310in}{1.211038in}}%
\pgfpathlineto{\pgfqpoint{5.285153in}{1.217606in}}%
\pgfpathlineto{\pgfqpoint{5.285995in}{1.237308in}}%
\pgfpathlineto{\pgfqpoint{5.286417in}{1.224173in}}%
\pgfpathlineto{\pgfqpoint{5.286838in}{1.224173in}}%
\pgfpathlineto{\pgfqpoint{5.287259in}{1.230741in}}%
\pgfpathlineto{\pgfqpoint{5.288102in}{1.217606in}}%
\pgfpathlineto{\pgfqpoint{5.288523in}{1.224173in}}%
\pgfpathlineto{\pgfqpoint{5.288944in}{1.217606in}}%
\pgfpathlineto{\pgfqpoint{5.290208in}{1.230741in}}%
\pgfpathlineto{\pgfqpoint{5.291893in}{1.217606in}}%
\pgfpathlineto{\pgfqpoint{5.292315in}{1.237308in}}%
\pgfpathlineto{\pgfqpoint{5.293157in}{1.230741in}}%
\pgfpathlineto{\pgfqpoint{5.293578in}{1.237308in}}%
\pgfpathlineto{\pgfqpoint{5.294421in}{1.217606in}}%
\pgfpathlineto{\pgfqpoint{5.294842in}{1.224173in}}%
\pgfpathlineto{\pgfqpoint{5.295264in}{1.211038in}}%
\pgfpathlineto{\pgfqpoint{5.295685in}{1.224173in}}%
\pgfpathlineto{\pgfqpoint{5.296106in}{1.230741in}}%
\pgfpathlineto{\pgfqpoint{5.296527in}{1.224173in}}%
\pgfpathlineto{\pgfqpoint{5.296949in}{1.211038in}}%
\pgfpathlineto{\pgfqpoint{5.297370in}{1.217606in}}%
\pgfpathlineto{\pgfqpoint{5.298212in}{1.230741in}}%
\pgfpathlineto{\pgfqpoint{5.298634in}{1.224173in}}%
\pgfpathlineto{\pgfqpoint{5.299055in}{1.224173in}}%
\pgfpathlineto{\pgfqpoint{5.300319in}{1.211038in}}%
\pgfpathlineto{\pgfqpoint{5.301161in}{1.237308in}}%
\pgfpathlineto{\pgfqpoint{5.301583in}{1.224173in}}%
\pgfpathlineto{\pgfqpoint{5.302425in}{1.224173in}}%
\pgfpathlineto{\pgfqpoint{5.303268in}{1.217606in}}%
\pgfpathlineto{\pgfqpoint{5.304532in}{1.224173in}}%
\pgfpathlineto{\pgfqpoint{5.305795in}{1.211038in}}%
\pgfpathlineto{\pgfqpoint{5.307481in}{1.230741in}}%
\pgfpathlineto{\pgfqpoint{5.308323in}{1.224173in}}%
\pgfpathlineto{\pgfqpoint{5.308744in}{1.230741in}}%
\pgfpathlineto{\pgfqpoint{5.309166in}{1.224173in}}%
\pgfpathlineto{\pgfqpoint{5.309587in}{1.211038in}}%
\pgfpathlineto{\pgfqpoint{5.310008in}{1.230741in}}%
\pgfpathlineto{\pgfqpoint{5.311693in}{1.217606in}}%
\pgfpathlineto{\pgfqpoint{5.312536in}{1.237308in}}%
\pgfpathlineto{\pgfqpoint{5.314221in}{1.204471in}}%
\pgfpathlineto{\pgfqpoint{5.314642in}{1.211038in}}%
\pgfpathlineto{\pgfqpoint{5.315906in}{1.257011in}}%
\pgfpathlineto{\pgfqpoint{5.317170in}{1.217606in}}%
\pgfpathlineto{\pgfqpoint{5.318855in}{1.237308in}}%
\pgfpathlineto{\pgfqpoint{5.320540in}{1.217606in}}%
\pgfpathlineto{\pgfqpoint{5.320962in}{1.217606in}}%
\pgfpathlineto{\pgfqpoint{5.321383in}{1.237308in}}%
\pgfpathlineto{\pgfqpoint{5.321804in}{1.230741in}}%
\pgfpathlineto{\pgfqpoint{5.322647in}{1.224173in}}%
\pgfpathlineto{\pgfqpoint{5.323068in}{1.230741in}}%
\pgfpathlineto{\pgfqpoint{5.323489in}{1.217606in}}%
\pgfpathlineto{\pgfqpoint{5.323910in}{1.230741in}}%
\pgfpathlineto{\pgfqpoint{5.324332in}{1.230741in}}%
\pgfpathlineto{\pgfqpoint{5.326017in}{1.211038in}}%
\pgfpathlineto{\pgfqpoint{5.327281in}{1.237308in}}%
\pgfpathlineto{\pgfqpoint{5.328123in}{1.237308in}}%
\pgfpathlineto{\pgfqpoint{5.329387in}{1.191335in}}%
\pgfpathlineto{\pgfqpoint{5.330651in}{1.230741in}}%
\pgfpathlineto{\pgfqpoint{5.332336in}{1.217606in}}%
\pgfpathlineto{\pgfqpoint{5.333179in}{1.230741in}}%
\pgfpathlineto{\pgfqpoint{5.334864in}{1.211038in}}%
\pgfpathlineto{\pgfqpoint{5.336970in}{1.230741in}}%
\pgfpathlineto{\pgfqpoint{5.337813in}{1.217606in}}%
\pgfpathlineto{\pgfqpoint{5.338234in}{1.224173in}}%
\pgfpathlineto{\pgfqpoint{5.339076in}{1.230741in}}%
\pgfpathlineto{\pgfqpoint{5.340340in}{1.204471in}}%
\pgfpathlineto{\pgfqpoint{5.341183in}{1.237308in}}%
\pgfpathlineto{\pgfqpoint{5.342025in}{1.224173in}}%
\pgfpathlineto{\pgfqpoint{5.342447in}{1.217606in}}%
\pgfpathlineto{\pgfqpoint{5.344132in}{1.237308in}}%
\pgfpathlineto{\pgfqpoint{5.344974in}{1.217606in}}%
\pgfpathlineto{\pgfqpoint{5.345396in}{1.230741in}}%
\pgfpathlineto{\pgfqpoint{5.345817in}{1.224173in}}%
\pgfpathlineto{\pgfqpoint{5.346238in}{1.237308in}}%
\pgfpathlineto{\pgfqpoint{5.346659in}{1.217606in}}%
\pgfpathlineto{\pgfqpoint{5.348345in}{1.230741in}}%
\pgfpathlineto{\pgfqpoint{5.349608in}{1.230741in}}%
\pgfpathlineto{\pgfqpoint{5.350451in}{1.211038in}}%
\pgfpathlineto{\pgfqpoint{5.351715in}{1.230741in}}%
\pgfpathlineto{\pgfqpoint{5.352557in}{1.204471in}}%
\pgfpathlineto{\pgfqpoint{5.353821in}{1.243876in}}%
\pgfpathlineto{\pgfqpoint{5.354664in}{1.204471in}}%
\pgfpathlineto{\pgfqpoint{5.355506in}{1.224173in}}%
\pgfpathlineto{\pgfqpoint{5.356770in}{1.230741in}}%
\pgfpathlineto{\pgfqpoint{5.357191in}{1.230741in}}%
\pgfpathlineto{\pgfqpoint{5.358455in}{1.217606in}}%
\pgfpathlineto{\pgfqpoint{5.358877in}{1.217606in}}%
\pgfpathlineto{\pgfqpoint{5.359298in}{1.230741in}}%
\pgfpathlineto{\pgfqpoint{5.360140in}{1.224173in}}%
\pgfpathlineto{\pgfqpoint{5.360983in}{1.230741in}}%
\pgfpathlineto{\pgfqpoint{5.362247in}{1.211038in}}%
\pgfpathlineto{\pgfqpoint{5.362668in}{1.217606in}}%
\pgfpathlineto{\pgfqpoint{5.363511in}{1.243876in}}%
\pgfpathlineto{\pgfqpoint{5.363932in}{1.237308in}}%
\pgfpathlineto{\pgfqpoint{5.365196in}{1.217606in}}%
\pgfpathlineto{\pgfqpoint{5.366038in}{1.243876in}}%
\pgfpathlineto{\pgfqpoint{5.366881in}{1.230741in}}%
\pgfpathlineto{\pgfqpoint{5.367302in}{1.217606in}}%
\pgfpathlineto{\pgfqpoint{5.367723in}{1.230741in}}%
\pgfpathlineto{\pgfqpoint{5.368145in}{1.243876in}}%
\pgfpathlineto{\pgfqpoint{5.368566in}{1.230741in}}%
\pgfpathlineto{\pgfqpoint{5.369830in}{1.224173in}}%
\pgfpathlineto{\pgfqpoint{5.370251in}{1.224173in}}%
\pgfpathlineto{\pgfqpoint{5.371094in}{1.217606in}}%
\pgfpathlineto{\pgfqpoint{5.371936in}{1.237308in}}%
\pgfpathlineto{\pgfqpoint{5.372357in}{1.230741in}}%
\pgfpathlineto{\pgfqpoint{5.373200in}{1.217606in}}%
\pgfpathlineto{\pgfqpoint{5.373621in}{1.230741in}}%
\pgfpathlineto{\pgfqpoint{5.374043in}{1.224173in}}%
\pgfpathlineto{\pgfqpoint{5.374885in}{1.217606in}}%
\pgfpathlineto{\pgfqpoint{5.375728in}{1.230741in}}%
\pgfpathlineto{\pgfqpoint{5.376149in}{1.224173in}}%
\pgfpathlineto{\pgfqpoint{5.376570in}{1.217606in}}%
\pgfpathlineto{\pgfqpoint{5.376992in}{1.224173in}}%
\pgfpathlineto{\pgfqpoint{5.377413in}{1.224173in}}%
\pgfpathlineto{\pgfqpoint{5.377834in}{1.217606in}}%
\pgfpathlineto{\pgfqpoint{5.378255in}{1.224173in}}%
\pgfpathlineto{\pgfqpoint{5.379098in}{1.230741in}}%
\pgfpathlineto{\pgfqpoint{5.380362in}{1.217606in}}%
\pgfpathlineto{\pgfqpoint{5.380783in}{1.230741in}}%
\pgfpathlineto{\pgfqpoint{5.381204in}{1.204471in}}%
\pgfpathlineto{\pgfqpoint{5.381626in}{1.217606in}}%
\pgfpathlineto{\pgfqpoint{5.382889in}{1.224173in}}%
\pgfpathlineto{\pgfqpoint{5.384153in}{1.224173in}}%
\pgfpathlineto{\pgfqpoint{5.384575in}{1.230741in}}%
\pgfpathlineto{\pgfqpoint{5.384996in}{1.224173in}}%
\pgfpathlineto{\pgfqpoint{5.385838in}{1.224173in}}%
\pgfpathlineto{\pgfqpoint{5.386260in}{1.217606in}}%
\pgfpathlineto{\pgfqpoint{5.386681in}{1.224173in}}%
\pgfpathlineto{\pgfqpoint{5.387945in}{1.224173in}}%
\pgfpathlineto{\pgfqpoint{5.388366in}{1.230741in}}%
\pgfpathlineto{\pgfqpoint{5.388787in}{1.224173in}}%
\pgfpathlineto{\pgfqpoint{5.389209in}{1.224173in}}%
\pgfpathlineto{\pgfqpoint{5.389630in}{1.230741in}}%
\pgfpathlineto{\pgfqpoint{5.390472in}{1.217606in}}%
\pgfpathlineto{\pgfqpoint{5.390894in}{1.224173in}}%
\pgfpathlineto{\pgfqpoint{5.391736in}{1.224173in}}%
\pgfpathlineto{\pgfqpoint{5.392158in}{1.217606in}}%
\pgfpathlineto{\pgfqpoint{5.393421in}{1.237308in}}%
\pgfpathlineto{\pgfqpoint{5.395528in}{1.211038in}}%
\pgfpathlineto{\pgfqpoint{5.397213in}{1.230741in}}%
\pgfpathlineto{\pgfqpoint{5.397634in}{1.211038in}}%
\pgfpathlineto{\pgfqpoint{5.398055in}{1.224173in}}%
\pgfpathlineto{\pgfqpoint{5.398477in}{1.224173in}}%
\pgfpathlineto{\pgfqpoint{5.399741in}{1.230741in}}%
\pgfpathlineto{\pgfqpoint{5.400583in}{1.217606in}}%
\pgfpathlineto{\pgfqpoint{5.401004in}{1.224173in}}%
\pgfpathlineto{\pgfqpoint{5.401426in}{1.224173in}}%
\pgfpathlineto{\pgfqpoint{5.402268in}{1.217606in}}%
\pgfpathlineto{\pgfqpoint{5.403532in}{1.230741in}}%
\pgfpathlineto{\pgfqpoint{5.405217in}{1.217606in}}%
\pgfpathlineto{\pgfqpoint{5.406060in}{1.217606in}}%
\pgfpathlineto{\pgfqpoint{5.406481in}{1.230741in}}%
\pgfpathlineto{\pgfqpoint{5.407324in}{1.224173in}}%
\pgfpathlineto{\pgfqpoint{5.407745in}{1.217606in}}%
\pgfpathlineto{\pgfqpoint{5.409430in}{1.237308in}}%
\pgfpathlineto{\pgfqpoint{5.409851in}{1.237308in}}%
\pgfpathlineto{\pgfqpoint{5.410694in}{1.211038in}}%
\pgfpathlineto{\pgfqpoint{5.411115in}{1.237308in}}%
\pgfpathlineto{\pgfqpoint{5.411958in}{1.224173in}}%
\pgfpathlineto{\pgfqpoint{5.412379in}{1.217606in}}%
\pgfpathlineto{\pgfqpoint{5.412800in}{1.224173in}}%
\pgfpathlineto{\pgfqpoint{5.413221in}{1.224173in}}%
\pgfpathlineto{\pgfqpoint{5.414064in}{1.237308in}}%
\pgfpathlineto{\pgfqpoint{5.414485in}{1.224173in}}%
\pgfpathlineto{\pgfqpoint{5.414907in}{1.230741in}}%
\pgfpathlineto{\pgfqpoint{5.415328in}{1.237308in}}%
\pgfpathlineto{\pgfqpoint{5.416592in}{1.217606in}}%
\pgfpathlineto{\pgfqpoint{5.417434in}{1.230741in}}%
\pgfpathlineto{\pgfqpoint{5.417855in}{1.211038in}}%
\pgfpathlineto{\pgfqpoint{5.418277in}{1.230741in}}%
\pgfpathlineto{\pgfqpoint{5.418698in}{1.224173in}}%
\pgfpathlineto{\pgfqpoint{5.419119in}{1.237308in}}%
\pgfpathlineto{\pgfqpoint{5.419541in}{1.224173in}}%
\pgfpathlineto{\pgfqpoint{5.419962in}{1.224173in}}%
\pgfpathlineto{\pgfqpoint{5.420804in}{1.211038in}}%
\pgfpathlineto{\pgfqpoint{5.421226in}{1.237308in}}%
\pgfpathlineto{\pgfqpoint{5.421647in}{1.217606in}}%
\pgfpathlineto{\pgfqpoint{5.422911in}{1.211038in}}%
\pgfpathlineto{\pgfqpoint{5.424596in}{1.224173in}}%
\pgfpathlineto{\pgfqpoint{5.425017in}{1.224173in}}%
\pgfpathlineto{\pgfqpoint{5.425860in}{1.211038in}}%
\pgfpathlineto{\pgfqpoint{5.426281in}{1.217606in}}%
\pgfpathlineto{\pgfqpoint{5.426702in}{1.217606in}}%
\pgfpathlineto{\pgfqpoint{5.427545in}{1.211038in}}%
\pgfpathlineto{\pgfqpoint{5.427966in}{1.217606in}}%
\pgfpathlineto{\pgfqpoint{5.428387in}{1.211038in}}%
\pgfpathlineto{\pgfqpoint{5.428809in}{1.211038in}}%
\pgfpathlineto{\pgfqpoint{5.430073in}{1.224173in}}%
\pgfpathlineto{\pgfqpoint{5.430915in}{1.211038in}}%
\pgfpathlineto{\pgfqpoint{5.432179in}{1.230741in}}%
\pgfpathlineto{\pgfqpoint{5.433443in}{1.217606in}}%
\pgfpathlineto{\pgfqpoint{5.434285in}{1.230741in}}%
\pgfpathlineto{\pgfqpoint{5.434707in}{1.211038in}}%
\pgfpathlineto{\pgfqpoint{5.435128in}{1.224173in}}%
\pgfpathlineto{\pgfqpoint{5.435549in}{1.224173in}}%
\pgfpathlineto{\pgfqpoint{5.435970in}{1.217606in}}%
\pgfpathlineto{\pgfqpoint{5.436392in}{1.224173in}}%
\pgfpathlineto{\pgfqpoint{5.438498in}{1.224173in}}%
\pgfpathlineto{\pgfqpoint{5.438919in}{1.237308in}}%
\pgfpathlineto{\pgfqpoint{5.439341in}{1.230741in}}%
\pgfpathlineto{\pgfqpoint{5.440605in}{1.224173in}}%
\pgfpathlineto{\pgfqpoint{5.441447in}{1.224173in}}%
\pgfpathlineto{\pgfqpoint{5.442711in}{1.217606in}}%
\pgfpathlineto{\pgfqpoint{5.444396in}{1.237308in}}%
\pgfpathlineto{\pgfqpoint{5.446081in}{1.211038in}}%
\pgfpathlineto{\pgfqpoint{5.447766in}{1.224173in}}%
\pgfpathlineto{\pgfqpoint{5.448609in}{1.224173in}}%
\pgfpathlineto{\pgfqpoint{5.449030in}{1.217606in}}%
\pgfpathlineto{\pgfqpoint{5.449451in}{1.230741in}}%
\pgfpathlineto{\pgfqpoint{5.450294in}{1.224173in}}%
\pgfpathlineto{\pgfqpoint{5.451558in}{1.224173in}}%
\pgfpathlineto{\pgfqpoint{5.452400in}{1.217606in}}%
\pgfpathlineto{\pgfqpoint{5.453664in}{1.230741in}}%
\pgfpathlineto{\pgfqpoint{5.454507in}{1.224173in}}%
\pgfpathlineto{\pgfqpoint{5.455349in}{1.230741in}}%
\pgfpathlineto{\pgfqpoint{5.456192in}{1.224173in}}%
\pgfpathlineto{\pgfqpoint{5.456613in}{1.230741in}}%
\pgfpathlineto{\pgfqpoint{5.457034in}{1.224173in}}%
\pgfpathlineto{\pgfqpoint{5.457456in}{1.224173in}}%
\pgfpathlineto{\pgfqpoint{5.457877in}{1.217606in}}%
\pgfpathlineto{\pgfqpoint{5.458298in}{1.224173in}}%
\pgfpathlineto{\pgfqpoint{5.459141in}{1.224173in}}%
\pgfpathlineto{\pgfqpoint{5.459562in}{1.237308in}}%
\pgfpathlineto{\pgfqpoint{5.460405in}{1.230741in}}%
\pgfpathlineto{\pgfqpoint{5.461668in}{1.224173in}}%
\pgfpathlineto{\pgfqpoint{5.462090in}{1.224173in}}%
\pgfpathlineto{\pgfqpoint{5.462511in}{1.230741in}}%
\pgfpathlineto{\pgfqpoint{5.462932in}{1.224173in}}%
\pgfpathlineto{\pgfqpoint{5.463354in}{1.217606in}}%
\pgfpathlineto{\pgfqpoint{5.464617in}{1.230741in}}%
\pgfpathlineto{\pgfqpoint{5.465039in}{1.224173in}}%
\pgfpathlineto{\pgfqpoint{5.465460in}{1.230741in}}%
\pgfpathlineto{\pgfqpoint{5.465881in}{1.230741in}}%
\pgfpathlineto{\pgfqpoint{5.467145in}{1.217606in}}%
\pgfpathlineto{\pgfqpoint{5.467988in}{1.224173in}}%
\pgfpathlineto{\pgfqpoint{5.468409in}{1.211038in}}%
\pgfpathlineto{\pgfqpoint{5.468830in}{1.224173in}}%
\pgfpathlineto{\pgfqpoint{5.469673in}{1.237308in}}%
\pgfpathlineto{\pgfqpoint{5.470515in}{1.243876in}}%
\pgfpathlineto{\pgfqpoint{5.471358in}{1.217606in}}%
\pgfpathlineto{\pgfqpoint{5.472622in}{1.224173in}}%
\pgfpathlineto{\pgfqpoint{5.473043in}{1.224173in}}%
\pgfpathlineto{\pgfqpoint{5.473464in}{1.230741in}}%
\pgfpathlineto{\pgfqpoint{5.473885in}{1.224173in}}%
\pgfpathlineto{\pgfqpoint{5.474728in}{1.217606in}}%
\pgfpathlineto{\pgfqpoint{5.475992in}{1.230741in}}%
\pgfpathlineto{\pgfqpoint{5.477256in}{1.217606in}}%
\pgfpathlineto{\pgfqpoint{5.477677in}{1.224173in}}%
\pgfpathlineto{\pgfqpoint{5.478520in}{1.211038in}}%
\pgfpathlineto{\pgfqpoint{5.479783in}{1.230741in}}%
\pgfpathlineto{\pgfqpoint{5.480205in}{1.211038in}}%
\pgfpathlineto{\pgfqpoint{5.480626in}{1.224173in}}%
\pgfpathlineto{\pgfqpoint{5.481047in}{1.224173in}}%
\pgfpathlineto{\pgfqpoint{5.481890in}{1.230741in}}%
\pgfpathlineto{\pgfqpoint{5.483996in}{1.211038in}}%
\pgfpathlineto{\pgfqpoint{5.485260in}{1.230741in}}%
\pgfpathlineto{\pgfqpoint{5.486524in}{1.224173in}}%
\pgfpathlineto{\pgfqpoint{5.487788in}{1.230741in}}%
\pgfpathlineto{\pgfqpoint{5.488209in}{1.230741in}}%
\pgfpathlineto{\pgfqpoint{5.488630in}{1.217606in}}%
\pgfpathlineto{\pgfqpoint{5.489051in}{1.224173in}}%
\pgfpathlineto{\pgfqpoint{5.489473in}{1.230741in}}%
\pgfpathlineto{\pgfqpoint{5.489894in}{1.224173in}}%
\pgfpathlineto{\pgfqpoint{5.490315in}{1.224173in}}%
\pgfpathlineto{\pgfqpoint{5.490737in}{1.230741in}}%
\pgfpathlineto{\pgfqpoint{5.491158in}{1.224173in}}%
\pgfpathlineto{\pgfqpoint{5.491579in}{1.224173in}}%
\pgfpathlineto{\pgfqpoint{5.492422in}{1.237308in}}%
\pgfpathlineto{\pgfqpoint{5.493264in}{1.217606in}}%
\pgfpathlineto{\pgfqpoint{5.493686in}{1.224173in}}%
\pgfpathlineto{\pgfqpoint{5.494949in}{1.237308in}}%
\pgfpathlineto{\pgfqpoint{5.496634in}{1.217606in}}%
\pgfpathlineto{\pgfqpoint{5.497056in}{1.230741in}}%
\pgfpathlineto{\pgfqpoint{5.497477in}{1.224173in}}%
\pgfpathlineto{\pgfqpoint{5.497898in}{1.217606in}}%
\pgfpathlineto{\pgfqpoint{5.499162in}{1.230741in}}%
\pgfpathlineto{\pgfqpoint{5.500426in}{1.224173in}}%
\pgfpathlineto{\pgfqpoint{5.500847in}{1.230741in}}%
\pgfpathlineto{\pgfqpoint{5.501269in}{1.224173in}}%
\pgfpathlineto{\pgfqpoint{5.501690in}{1.211038in}}%
\pgfpathlineto{\pgfqpoint{5.502111in}{1.230741in}}%
\pgfpathlineto{\pgfqpoint{5.502532in}{1.217606in}}%
\pgfpathlineto{\pgfqpoint{5.502954in}{1.217606in}}%
\pgfpathlineto{\pgfqpoint{5.504217in}{1.224173in}}%
\pgfpathlineto{\pgfqpoint{5.504639in}{1.224173in}}%
\pgfpathlineto{\pgfqpoint{5.505060in}{1.237308in}}%
\pgfpathlineto{\pgfqpoint{5.505481in}{1.230741in}}%
\pgfpathlineto{\pgfqpoint{5.506745in}{1.217606in}}%
\pgfpathlineto{\pgfqpoint{5.507166in}{1.230741in}}%
\pgfpathlineto{\pgfqpoint{5.507588in}{1.211038in}}%
\pgfpathlineto{\pgfqpoint{5.508009in}{1.224173in}}%
\pgfpathlineto{\pgfqpoint{5.508430in}{1.197903in}}%
\pgfpathlineto{\pgfqpoint{5.508852in}{1.204471in}}%
\pgfpathlineto{\pgfqpoint{5.510115in}{1.230741in}}%
\pgfpathlineto{\pgfqpoint{5.510537in}{1.230741in}}%
\pgfpathlineto{\pgfqpoint{5.511801in}{1.211038in}}%
\pgfpathlineto{\pgfqpoint{5.512222in}{1.230741in}}%
\pgfpathlineto{\pgfqpoint{5.513064in}{1.224173in}}%
\pgfpathlineto{\pgfqpoint{5.513486in}{1.230741in}}%
\pgfpathlineto{\pgfqpoint{5.514328in}{1.217606in}}%
\pgfpathlineto{\pgfqpoint{5.514749in}{1.224173in}}%
\pgfpathlineto{\pgfqpoint{5.516013in}{1.237308in}}%
\pgfpathlineto{\pgfqpoint{5.516435in}{1.217606in}}%
\pgfpathlineto{\pgfqpoint{5.517277in}{1.224173in}}%
\pgfpathlineto{\pgfqpoint{5.517698in}{1.211038in}}%
\pgfpathlineto{\pgfqpoint{5.518120in}{1.230741in}}%
\pgfpathlineto{\pgfqpoint{5.518541in}{1.217606in}}%
\pgfpathlineto{\pgfqpoint{5.518962in}{1.211038in}}%
\pgfpathlineto{\pgfqpoint{5.520226in}{1.230741in}}%
\pgfpathlineto{\pgfqpoint{5.521069in}{1.237308in}}%
\pgfpathlineto{\pgfqpoint{5.521490in}{1.217606in}}%
\pgfpathlineto{\pgfqpoint{5.522332in}{1.237308in}}%
\pgfpathlineto{\pgfqpoint{5.522754in}{1.217606in}}%
\pgfpathlineto{\pgfqpoint{5.523175in}{1.224173in}}%
\pgfpathlineto{\pgfqpoint{5.523596in}{1.230741in}}%
\pgfpathlineto{\pgfqpoint{5.524018in}{1.217606in}}%
\pgfpathlineto{\pgfqpoint{5.524439in}{1.230741in}}%
\pgfpathlineto{\pgfqpoint{5.524860in}{1.230741in}}%
\pgfpathlineto{\pgfqpoint{5.525281in}{1.237308in}}%
\pgfpathlineto{\pgfqpoint{5.526967in}{1.211038in}}%
\pgfpathlineto{\pgfqpoint{5.527809in}{1.237308in}}%
\pgfpathlineto{\pgfqpoint{5.528230in}{1.217606in}}%
\pgfpathlineto{\pgfqpoint{5.528652in}{1.224173in}}%
\pgfpathlineto{\pgfqpoint{5.529073in}{1.217606in}}%
\pgfpathlineto{\pgfqpoint{5.529494in}{1.217606in}}%
\pgfpathlineto{\pgfqpoint{5.530758in}{1.230741in}}%
\pgfpathlineto{\pgfqpoint{5.531179in}{1.230741in}}%
\pgfpathlineto{\pgfqpoint{5.532022in}{1.217606in}}%
\pgfpathlineto{\pgfqpoint{5.532443in}{1.243876in}}%
\pgfpathlineto{\pgfqpoint{5.532864in}{1.217606in}}%
\pgfpathlineto{\pgfqpoint{5.533286in}{1.217606in}}%
\pgfpathlineto{\pgfqpoint{5.534971in}{1.230741in}}%
\pgfpathlineto{\pgfqpoint{5.537077in}{1.211038in}}%
\pgfpathlineto{\pgfqpoint{5.538341in}{1.230741in}}%
\pgfpathlineto{\pgfqpoint{5.540026in}{1.217606in}}%
\pgfpathlineto{\pgfqpoint{5.540869in}{1.230741in}}%
\pgfpathlineto{\pgfqpoint{5.541290in}{1.211038in}}%
\pgfpathlineto{\pgfqpoint{5.542133in}{1.217606in}}%
\pgfpathlineto{\pgfqpoint{5.542975in}{1.237308in}}%
\pgfpathlineto{\pgfqpoint{5.543396in}{1.230741in}}%
\pgfpathlineto{\pgfqpoint{5.544239in}{1.217606in}}%
\pgfpathlineto{\pgfqpoint{5.545503in}{1.237308in}}%
\pgfpathlineto{\pgfqpoint{5.546767in}{1.224173in}}%
\pgfpathlineto{\pgfqpoint{5.547188in}{1.224173in}}%
\pgfpathlineto{\pgfqpoint{5.547609in}{1.243876in}}%
\pgfpathlineto{\pgfqpoint{5.548030in}{1.217606in}}%
\pgfpathlineto{\pgfqpoint{5.548452in}{1.217606in}}%
\pgfpathlineto{\pgfqpoint{5.549716in}{1.250443in}}%
\pgfpathlineto{\pgfqpoint{5.550137in}{1.237308in}}%
\pgfpathlineto{\pgfqpoint{5.550558in}{1.237308in}}%
\pgfpathlineto{\pgfqpoint{5.552243in}{1.211038in}}%
\pgfpathlineto{\pgfqpoint{5.553928in}{1.237308in}}%
\pgfpathlineto{\pgfqpoint{5.554350in}{1.211038in}}%
\pgfpathlineto{\pgfqpoint{5.555192in}{1.224173in}}%
\pgfpathlineto{\pgfqpoint{5.555613in}{1.230741in}}%
\pgfpathlineto{\pgfqpoint{5.556456in}{1.211038in}}%
\pgfpathlineto{\pgfqpoint{5.556877in}{1.217606in}}%
\pgfpathlineto{\pgfqpoint{5.557299in}{1.217606in}}%
\pgfpathlineto{\pgfqpoint{5.558562in}{1.230741in}}%
\pgfpathlineto{\pgfqpoint{5.559826in}{1.211038in}}%
\pgfpathlineto{\pgfqpoint{5.560669in}{1.230741in}}%
\pgfpathlineto{\pgfqpoint{5.561090in}{1.204471in}}%
\pgfpathlineto{\pgfqpoint{5.561511in}{1.224173in}}%
\pgfpathlineto{\pgfqpoint{5.561933in}{1.230741in}}%
\pgfpathlineto{\pgfqpoint{5.562354in}{1.224173in}}%
\pgfpathlineto{\pgfqpoint{5.563196in}{1.204471in}}%
\pgfpathlineto{\pgfqpoint{5.563618in}{1.211038in}}%
\pgfpathlineto{\pgfqpoint{5.564460in}{1.230741in}}%
\pgfpathlineto{\pgfqpoint{5.565303in}{1.211038in}}%
\pgfpathlineto{\pgfqpoint{5.565724in}{1.230741in}}%
\pgfpathlineto{\pgfqpoint{5.566567in}{1.224173in}}%
\pgfpathlineto{\pgfqpoint{5.566988in}{1.230741in}}%
\pgfpathlineto{\pgfqpoint{5.567409in}{1.217606in}}%
\pgfpathlineto{\pgfqpoint{5.567830in}{1.224173in}}%
\pgfpathlineto{\pgfqpoint{5.568252in}{1.230741in}}%
\pgfpathlineto{\pgfqpoint{5.568673in}{1.224173in}}%
\pgfpathlineto{\pgfqpoint{5.569516in}{1.224173in}}%
\pgfpathlineto{\pgfqpoint{5.570779in}{1.230741in}}%
\pgfpathlineto{\pgfqpoint{5.571201in}{1.230741in}}%
\pgfpathlineto{\pgfqpoint{5.571622in}{1.211038in}}%
\pgfpathlineto{\pgfqpoint{5.572465in}{1.217606in}}%
\pgfpathlineto{\pgfqpoint{5.572886in}{1.217606in}}%
\pgfpathlineto{\pgfqpoint{5.573728in}{1.230741in}}%
\pgfpathlineto{\pgfqpoint{5.574150in}{1.211038in}}%
\pgfpathlineto{\pgfqpoint{5.574571in}{1.217606in}}%
\pgfpathlineto{\pgfqpoint{5.574992in}{1.237308in}}%
\pgfpathlineto{\pgfqpoint{5.575414in}{1.217606in}}%
\pgfpathlineto{\pgfqpoint{5.575835in}{1.211038in}}%
\pgfpathlineto{\pgfqpoint{5.577099in}{1.230741in}}%
\pgfpathlineto{\pgfqpoint{5.577941in}{1.217606in}}%
\pgfpathlineto{\pgfqpoint{5.578362in}{1.237308in}}%
\pgfpathlineto{\pgfqpoint{5.578784in}{1.230741in}}%
\pgfpathlineto{\pgfqpoint{5.579205in}{1.224173in}}%
\pgfpathlineto{\pgfqpoint{5.579626in}{1.237308in}}%
\pgfpathlineto{\pgfqpoint{5.580048in}{1.224173in}}%
\pgfpathlineto{\pgfqpoint{5.580890in}{1.224173in}}%
\pgfpathlineto{\pgfqpoint{5.582154in}{1.230741in}}%
\pgfpathlineto{\pgfqpoint{5.583418in}{1.217606in}}%
\pgfpathlineto{\pgfqpoint{5.585103in}{1.230741in}}%
\pgfpathlineto{\pgfqpoint{5.585524in}{1.224173in}}%
\pgfpathlineto{\pgfqpoint{5.585945in}{1.237308in}}%
\pgfpathlineto{\pgfqpoint{5.586367in}{1.224173in}}%
\pgfpathlineto{\pgfqpoint{5.587631in}{1.224173in}}%
\pgfpathlineto{\pgfqpoint{5.588052in}{1.230741in}}%
\pgfpathlineto{\pgfqpoint{5.588473in}{1.217606in}}%
\pgfpathlineto{\pgfqpoint{5.588894in}{1.237308in}}%
\pgfpathlineto{\pgfqpoint{5.589316in}{1.243876in}}%
\pgfpathlineto{\pgfqpoint{5.590580in}{1.224173in}}%
\pgfpathlineto{\pgfqpoint{5.591422in}{1.224173in}}%
\pgfpathlineto{\pgfqpoint{5.591843in}{1.211038in}}%
\pgfpathlineto{\pgfqpoint{5.592265in}{1.230741in}}%
\pgfpathlineto{\pgfqpoint{5.592686in}{1.224173in}}%
\pgfpathlineto{\pgfqpoint{5.593107in}{1.230741in}}%
\pgfpathlineto{\pgfqpoint{5.593950in}{1.230741in}}%
\pgfpathlineto{\pgfqpoint{5.595635in}{1.217606in}}%
\pgfpathlineto{\pgfqpoint{5.596056in}{1.237308in}}%
\pgfpathlineto{\pgfqpoint{5.596477in}{1.224173in}}%
\pgfpathlineto{\pgfqpoint{5.596899in}{1.224173in}}%
\pgfpathlineto{\pgfqpoint{5.597741in}{1.217606in}}%
\pgfpathlineto{\pgfqpoint{5.598163in}{1.224173in}}%
\pgfpathlineto{\pgfqpoint{5.598584in}{1.079687in}}%
\pgfpathlineto{\pgfqpoint{5.599005in}{1.224173in}}%
\pgfpathlineto{\pgfqpoint{5.600269in}{1.217606in}}%
\pgfpathlineto{\pgfqpoint{5.601111in}{1.230741in}}%
\pgfpathlineto{\pgfqpoint{5.601533in}{1.224173in}}%
\pgfpathlineto{\pgfqpoint{5.602375in}{1.217606in}}%
\pgfpathlineto{\pgfqpoint{5.602797in}{1.165065in}}%
\pgfpathlineto{\pgfqpoint{5.603218in}{1.230741in}}%
\pgfpathlineto{\pgfqpoint{5.604060in}{1.217606in}}%
\pgfpathlineto{\pgfqpoint{5.604482in}{1.224173in}}%
\pgfpathlineto{\pgfqpoint{5.604903in}{1.217606in}}%
\pgfpathlineto{\pgfqpoint{5.605324in}{1.224173in}}%
\pgfpathlineto{\pgfqpoint{5.605746in}{1.224173in}}%
\pgfpathlineto{\pgfqpoint{5.606167in}{1.230741in}}%
\pgfpathlineto{\pgfqpoint{5.607431in}{1.211038in}}%
\pgfpathlineto{\pgfqpoint{5.609116in}{1.230741in}}%
\pgfpathlineto{\pgfqpoint{5.609537in}{1.204471in}}%
\pgfpathlineto{\pgfqpoint{5.609958in}{1.230741in}}%
\pgfpathlineto{\pgfqpoint{5.610380in}{1.230741in}}%
\pgfpathlineto{\pgfqpoint{5.610801in}{1.224173in}}%
\pgfpathlineto{\pgfqpoint{5.611222in}{1.237308in}}%
\pgfpathlineto{\pgfqpoint{5.611643in}{1.224173in}}%
\pgfpathlineto{\pgfqpoint{5.612065in}{1.375227in}}%
\pgfpathlineto{\pgfqpoint{5.612486in}{1.204471in}}%
\pgfpathlineto{\pgfqpoint{5.615014in}{1.230741in}}%
\pgfpathlineto{\pgfqpoint{5.615014in}{1.230741in}}%
\pgfusepath{stroke}%
\end{pgfscope}%
\begin{pgfscope}%
\pgfsetrectcap%
\pgfsetmiterjoin%
\pgfsetlinewidth{0.803000pt}%
\definecolor{currentstroke}{rgb}{0.000000,0.000000,0.000000}%
\pgfsetstrokecolor{currentstroke}%
\pgfsetdash{}{0pt}%
\pgfpathmoveto{\pgfqpoint{0.885050in}{0.587778in}}%
\pgfpathlineto{\pgfqpoint{0.885050in}{1.873704in}}%
\pgfusepath{stroke}%
\end{pgfscope}%
\begin{pgfscope}%
\pgfsetrectcap%
\pgfsetmiterjoin%
\pgfsetlinewidth{0.803000pt}%
\definecolor{currentstroke}{rgb}{0.000000,0.000000,0.000000}%
\pgfsetstrokecolor{currentstroke}%
\pgfsetdash{}{0pt}%
\pgfpathmoveto{\pgfqpoint{5.840250in}{0.587778in}}%
\pgfpathlineto{\pgfqpoint{5.840250in}{1.873704in}}%
\pgfusepath{stroke}%
\end{pgfscope}%
\begin{pgfscope}%
\pgfsetrectcap%
\pgfsetmiterjoin%
\pgfsetlinewidth{0.803000pt}%
\definecolor{currentstroke}{rgb}{0.000000,0.000000,0.000000}%
\pgfsetstrokecolor{currentstroke}%
\pgfsetdash{}{0pt}%
\pgfpathmoveto{\pgfqpoint{0.885050in}{0.587778in}}%
\pgfpathlineto{\pgfqpoint{5.840250in}{0.587778in}}%
\pgfusepath{stroke}%
\end{pgfscope}%
\begin{pgfscope}%
\pgfsetrectcap%
\pgfsetmiterjoin%
\pgfsetlinewidth{0.803000pt}%
\definecolor{currentstroke}{rgb}{0.000000,0.000000,0.000000}%
\pgfsetstrokecolor{currentstroke}%
\pgfsetdash{}{0pt}%
\pgfpathmoveto{\pgfqpoint{0.885050in}{1.873704in}}%
\pgfpathlineto{\pgfqpoint{5.840250in}{1.873704in}}%
\pgfusepath{stroke}%
\end{pgfscope}%
\begin{pgfscope}%
\definecolor{textcolor}{rgb}{0.000000,0.000000,0.000000}%
\pgfsetstrokecolor{textcolor}%
\pgfsetfillcolor{textcolor}%
\pgftext[x=3.362650in,y=1.957037in,,base]{\color{textcolor}{\sffamily\fontsize{12.000000}{14.400000}\selectfont\catcode`\^=\active\def^{\ifmmode\sp\else\^{}\fi}\catcode`\%=\active\def%{\%}opóźnienie}}%
\end{pgfscope}%
\begin{pgfscope}%
\pgfsetbuttcap%
\pgfsetmiterjoin%
\definecolor{currentfill}{rgb}{1.000000,1.000000,1.000000}%
\pgfsetfillcolor{currentfill}%
\pgfsetfillopacity{0.800000}%
\pgfsetlinewidth{1.003750pt}%
\definecolor{currentstroke}{rgb}{0.800000,0.800000,0.800000}%
\pgfsetstrokecolor{currentstroke}%
\pgfsetstrokeopacity{0.800000}%
\pgfsetdash{}{0pt}%
\pgfpathmoveto{\pgfqpoint{0.097222in}{0.069444in}}%
\pgfpathlineto{\pgfqpoint{0.834703in}{0.069444in}}%
\pgfpathquadraticcurveto{\pgfqpoint{0.862481in}{0.069444in}}{\pgfqpoint{0.862481in}{0.097222in}}%
\pgfpathlineto{\pgfqpoint{0.862481in}{0.491048in}}%
\pgfpathquadraticcurveto{\pgfqpoint{0.862481in}{0.518826in}}{\pgfqpoint{0.834703in}{0.518826in}}%
\pgfpathlineto{\pgfqpoint{0.097222in}{0.518826in}}%
\pgfpathquadraticcurveto{\pgfqpoint{0.069444in}{0.518826in}}{\pgfqpoint{0.069444in}{0.491048in}}%
\pgfpathlineto{\pgfqpoint{0.069444in}{0.097222in}}%
\pgfpathquadraticcurveto{\pgfqpoint{0.069444in}{0.069444in}}{\pgfqpoint{0.097222in}{0.069444in}}%
\pgfpathlineto{\pgfqpoint{0.097222in}{0.069444in}}%
\pgfpathclose%
\pgfusepath{stroke,fill}%
\end{pgfscope}%
\begin{pgfscope}%
\pgfsetbuttcap%
\pgfsetmiterjoin%
\definecolor{currentfill}{rgb}{0.145098,0.145098,1.000000}%
\pgfsetfillcolor{currentfill}%
\pgfsetlinewidth{1.003750pt}%
\definecolor{currentstroke}{rgb}{0.145098,0.145098,1.000000}%
\pgfsetstrokecolor{currentstroke}%
\pgfsetdash{}{0pt}%
\pgfpathmoveto{\pgfqpoint{0.125000in}{0.357747in}}%
\pgfpathlineto{\pgfqpoint{0.402778in}{0.357747in}}%
\pgfpathlineto{\pgfqpoint{0.402778in}{0.454969in}}%
\pgfpathlineto{\pgfqpoint{0.125000in}{0.454969in}}%
\pgfpathlineto{\pgfqpoint{0.125000in}{0.357747in}}%
\pgfpathclose%
\pgfusepath{stroke,fill}%
\end{pgfscope}%
\begin{pgfscope}%
\definecolor{textcolor}{rgb}{0.000000,0.000000,0.000000}%
\pgfsetstrokecolor{textcolor}%
\pgfsetfillcolor{textcolor}%
\pgftext[x=0.513889in,y=0.357747in,left,base]{\color{textcolor}{\sffamily\fontsize{10.000000}{12.000000}\selectfont\catcode`\^=\active\def^{\ifmmode\sp\else\^{}\fi}\catcode`\%=\active\def%{\%}CPU}}%
\end{pgfscope}%
\begin{pgfscope}%
\pgfsetbuttcap%
\pgfsetmiterjoin%
\definecolor{currentfill}{rgb}{1.000000,0.145098,0.145098}%
\pgfsetfillcolor{currentfill}%
\pgfsetlinewidth{1.003750pt}%
\definecolor{currentstroke}{rgb}{1.000000,0.145098,0.145098}%
\pgfsetstrokecolor{currentstroke}%
\pgfsetdash{}{0pt}%
\pgfpathmoveto{\pgfqpoint{0.125000in}{0.153890in}}%
\pgfpathlineto{\pgfqpoint{0.402778in}{0.153890in}}%
\pgfpathlineto{\pgfqpoint{0.402778in}{0.251112in}}%
\pgfpathlineto{\pgfqpoint{0.125000in}{0.251112in}}%
\pgfpathlineto{\pgfqpoint{0.125000in}{0.153890in}}%
\pgfpathclose%
\pgfusepath{stroke,fill}%
\end{pgfscope}%
\begin{pgfscope}%
\definecolor{textcolor}{rgb}{0.000000,0.000000,0.000000}%
\pgfsetstrokecolor{textcolor}%
\pgfsetfillcolor{textcolor}%
\pgftext[x=0.513889in,y=0.153890in,left,base]{\color{textcolor}{\sffamily\fontsize{10.000000}{12.000000}\selectfont\catcode`\^=\active\def^{\ifmmode\sp\else\^{}\fi}\catcode`\%=\active\def%{\%}GPU}}%
\end{pgfscope}%
\end{pgfpicture}%
\makeatother%
\endgroup%
}
    \caption{Przebieg trwania obliczeń dla bufora o rozmiarze 1024}
    \label{fig:Przebieg trwania obliczeń dla bufora o rozmiarze 1024}
\end{figure}

Przebieg trwania obliczeń dla bufora o rozmiarze 1024 przedstawia kolejno: czas trwania obliczeń, obciążenie systemu oraz opóźnienie. Można zauważyć odmienną charakterystykę reakcji na ilość przetwarzanych danych. W przypadku CPU, czas obliczeń oraz obciążenie utrzymują się na podobnym poziomie, gdzie w przypadku GPU następuje duża niestabilność omawianych wartości. Prawdopodobnie jest to związane z charakterystyką pracy GPU. Zwiększenie się ilości aktywnych syntezatorów, zwiększa prawdopodobieństwo konieczności zaprzestania pracy części procesorów w związku z wykonywaniem się gałęzi instrukcji warunkowej. Patrząc na statystykę opóźnienia można również dostrzec różnicę między implementacjami. W obu przypadkach wartości te sięgają maksymalnie dziesiątych części milisekundy, co wskazuje na dużą stabilność działania systemu.