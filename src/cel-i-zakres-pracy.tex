\chapter*{Cel i zakres pracy}
\addcontentsline{toc}{chapter}{Cel i zakres pracy}

Celem niniejszej pracy jest zaprojektowanie oraz implementacja systemu, pozwalającego na przetestowanie możliwości wykorzystania karty graficznej względem CPU, do przetwarzania sygnałów dźwiękowych. System ma umożliwiać syntezowanie oraz przetwarzanie dźwięku w różnych powszechnie stosowanych formatach (biorąc pod uwagę częstotliwość próbkowania oraz wielkość próbki) zarówno w czasie rzeczywistym, jak i w trybie offline. Tryb offline pozwala na syntezę/przetworzenie sygnału w możliwie najkrótszym dla urządzenia czasie. Na ogół pozwala to na wykonanie skomplikowanych obliczeń, nawet w sytuacji kiedy nie jest możliwe przetworzenie danych w czasie rzeczywistym. W tym przypadku zostanie to wykorzystane w celu porównania czasów wykonania, co nie było by możliwe dla czasu rzeczywistego. System musi udostępniać możliwość gromadzenia danych statystycznych dotyczących wydajności swojej pracy, oraz umożliwiać przeprowadzanie jednolitych testów wydajnościowych. W swojej uniwersalności powinien pozwalać na wprowadzanie modyfikacji w nieskomplikowany sposób, co umożliwi ekstensywne testowanie algorytmów oraz łączenie ich w bardziej złożonych scenariuszach użycia, jak również zmianę wykorzystanej platformy bez konieczności modyfikacji fundamentów jego działania. Skutkuje to eliminacją możliwie największej liczby czynników zewnętrznych, mogących zaburzyć pomiary wydajności. System powinien przetwarzać dźwięk w formacie PCM, który jest najbardziej powszechnie stosowanym formatem w przemyśle muzycznym oraz być w stanie odczytywać podstawowe sygnały zawarte w MIDI. Biorąc to wszystko pod uwagę, powstały system będzie stanowić podstawę podczas porównania możliwości CPU i GPU dla przedstawionego problemu. Praca ta nie skupia się na optymalizacji powstałych rozwiązań, a na próbie weryfikacji omawianej koncepcji.