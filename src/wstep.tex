\chapter*{Wstęp}
\addcontentsline{toc}{chapter}{Wstęp}

Przetwarzanie sygnałów dźwiękowych jest dużym obszarem informatyki. Wraz z jego rozwojem, powstawaniem nowych algorytmów, wzrostem standardów dotyczących jakości dźwięku, a także zwiększaniem się ilości przetwarzanych danych, pojawia się coraz większe zapotrzebowanie na moc obliczeniową. W wielu przypadkach, przy zastosowaniu obecnych technik przetwarzania dźwięku, obliczenia w czasie rzeczywistym, stają się nieosiągalne, co mocno utrudnia pracę wielu osobom, a w niektórych przypadkach sprawia, że staje się ona nie możliwa. Wynika z tego, iż ciągły rozwój, a co za tym idzie, ciągły wzrost wymagań oprogramowania, wymusza na użytkownikach okresową wymianę podzespołów na coraz to wydajniejsze. Powstaje pytanie, czy nie istnieje żadne inne rozwiązanie, które pozwoliło by przynajmniej częściowo zniwelować ten problem. W obecny standard komputerów osobistych włączona jest obecność karty graficznej - odmiany procesora, przeznaczonego przede wszystkim do przetwarzania i generowania obrazu. Wraz z rozwojem tej technologii zaczęto zauważać, że karty graficzne posiadają ogromny potencjał obliczeniowy, który można by było wykorzystać w innych dziedzinach informatyki. Wraz z pojawieniem się technologii takich jak CUDA czy OpenCL, rozpowszechniono wykorzystanie karty graficznej do obliczeń ogólnego przeznaczenia. Dziedzina przetwarzania sygnałów dźwiękowych, ze względu na swoją naturę, jest jedną z dziedzin informatyki, która mogłaby skorzystać na wykorzystaniu tych technologii w celu przyspieszenia obliczeń i tym samym umożliwienia szerszemu gronu osób na korzystanie z bardziej zaawansowanych technik przetwarzania dźwięku. Temat ten był do tej pory pomijany, jedyne informacje jakie można było znaleźć, klasyfikują się do źródeł anegdotycznych. Nie licząc sporadycznych wpisów internautów na temat pojedynczych prób, ciężko jest znaleźć oprogramowanie wykorzystujące GPU do przetwarzania dźwięku w sposób spełniający standardy przemysłu muzycznego.
